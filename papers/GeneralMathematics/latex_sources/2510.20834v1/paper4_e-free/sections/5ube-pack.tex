\newpage

%%%%%%%%%%%%%%%%%%%%%%%%%%%%%%%%%%%%%%%%%%%%%%%%%%%%%%%%%%%%%%%%%%%%%%%%
% 5. Tube packing  (revised version)
%%%%%%%%%%%%%%%%%%%%%%%%%%%%%%%%%%%%%%%%%%%%%%%%%%%%%%%%%%%%%%%%%%%%%%%%
\section{Tube packing}\label{sec:tube-pack}

\medskip
\noindent\textbf{Convention.}
In this section, by a “tube” of a cap~$\Theta$ we mean, by default, the \emph{truncated} tube
$\widetilde{\mathcal T}_{\Theta}$; the axial layer $\{|x|\le\rho/4\}$ inside $Q_\lambda$ is ignored.
This technical simplification does not affect the global exponents and is consistent with the regime
separation in §\,\ref{sec:kakeya}. See also the discussion of boundary layers in App.~\ref{app:C4}.%
\footnote{On the layer $|t|\ll\lambda^{-3/2}$ the non-overlap issue is negligible: its contribution is already controlled by measure; see the rigorous treatment in App.~\ref{app:C4}.}

For convenience we repeat the scale parameters (see §\,\ref{subsec:scales}):
\(
\rho=\lambda^{-1/2},\ r=\lambda^{-2/3},\ D=\lambda^{1/12},\ \alpha=c_0 r D^{1/2}.
\)

%-----------------------------------------------------------------------
\subsection{Tube family}\label{subsec:tubes}
%-----------------------------------------------------------------------
Throughout this section we keep the parameters of §\,\ref{subsec:scales}:
\(
\rho=\lambda^{-1/2},\ r=\lambda^{-2/3},\ D=\lambda^{1/12},\ \alpha=c_0 r D^{1/2}.
\)
Let $\Theta$ be a cap of radius $r$ with center $\xi_\Theta$ (see §\,\ref{subsec:caps}).
Define the \emph{truncated tube}
\[
   \widetilde{\mathcal T}_{\Theta}
      := \Bigl\{(t,x)\in Q_{\lambda} \;:\;
              |x-2t\,\xi_{\Theta}|\le \rho,\;
              |x|>\rho/4 \Bigr\},
\]
where the cylinder $Q_{\lambda}$ is given in~\eqref{eq:def-cylinder}.
Since the working $t$–interval has length $\asymp \lambda^{-3/2}$, the volume of one tube is
\[
   \bigl|\widetilde{\mathcal T}_{\Theta}\bigr|
   \simeq
   \rho^{3}\,\lambda^{-3/2}
   \;=\; \lambda^{-3}.
\]
For the multiplicity set
\[
   M(t,x):=\sum_{\Theta}\mathbf 1_{\widetilde{\mathcal T}_\Theta}(t,x).
\]

%-----------------------------------------------------------------------
\subsection{Pairwise overlap estimate}\label{subsec:pair-overlap}
%-----------------------------------------------------------------------
The key step is a \emph{pairwise} estimate of the measure of intersection of two truncated tubes.

\begin{lemma}[Pairwise overlap]\label{lem:pair}
Let $\Theta,\Theta'$ be caps of radius $r$ with angular separation 
$\delta=\angle(\xi_\Theta,\xi_{\Theta'})\in[r,1]$. Then
\begin{equation}\label{eq:pair-overlap-min}
   \bigl|\widetilde{\mathcal T}_{\Theta}\cap \widetilde{\mathcal T}_{\Theta'}\bigr|
   \ \lesssim\ 
   \rho^{3}\cdot \min\!\left\{\frac{\rho}{\lambda\,\delta},\ \lambda^{-3/2}\right\}.
\end{equation}
In particular (on our angular range $\delta\in[r,1]$),
\begin{equation}\label{eq:pair-overlap}
   \bigl|\widetilde{\mathcal T}_{\Theta}\cap \widetilde{\mathcal T}_{\Theta'}\bigr|
   \ \lesssim\ \frac{\rho^{4}}{\lambda\,\delta}.
\end{equation}
\end{lemma}

\begin{proof}[Idea of the proof]
If $(t,x)\in \widetilde{\mathcal T}_{\Theta}\cap \widetilde{\mathcal T}_{\Theta'}$, then 
$|x-2t\xi_{\Theta}|\le\rho$ and $|x-2t\xi_{\Theta'}|\le\rho$. Subtracting, we get
$|2t(\xi_\Theta-\xi_{\Theta'})|\lesssim\rho$. Since $|\xi_\Theta-\xi_{\Theta'}|\sim \lambda\delta$
(for small angles), the admissible time extent is bounded by
\[
   |t|\ \lesssim\ \min\!\left\{\frac{\rho}{\lambda\,\delta},\ \lambda^{-3/2}\right\},
\]
where the second bound is the length of the working $t$–interval. For each admissible $t$, the $x$–slice
is the intersection of two three–dimensional balls of radius $\rho$ whose centers are within distance 
$\lesssim\rho$; its measure is $\lesssim\rho^{3}$. Integrating in $t$ over the indicated interval gives
\eqref{eq:pair-overlap-min}. Inequality \eqref{eq:pair-overlap} is its weakening, sufficient for the subsequent sum.
The condition $|x|>\rho/4$ only decreases the intersection.
\end{proof}




%-----------------------------------------------------------------------
\subsection{$L^{2}$ estimate for the sum of indicators}\label{subsec:tube-L2}
%-----------------------------------------------------------------------
Introduce the notation
\[
   S \;:=\; 
   \sum_{\Theta,\Theta'}\,
   \bigl|\widetilde{\mathcal T}_{\Theta}\cap \widetilde{\mathcal T}_{\Theta'}\bigr|
   \;=\;
   \Bigl\|\sum_{\Theta}\mathbf 1_{\widetilde{\mathcal T}_{\Theta}}\Bigr\|_{L^{2}(Q_{\lambda})}^{2}.
\]
The summation is over a \emph{fixed local family} of caps, compatible with the
“six-color” selection from §~\ref{subsec:broad-def} (one representative cap per $\alpha$–cap);
this prevents overcounting and keeps constants independent of the full cardinality of
$\{\Theta\}$. \emph{The overlap multiplicity of local cells is bounded by a universal constant,
so summation over $\Theta$ does not introduce an extra factor depending on $\#\{\Theta\}$.}

Split the pairs by the size of $\delta=\angle(\xi_\Theta,\xi_{\Theta'})$ into dyadic annuli:
$\delta\sim 2^{j}\alpha$, $j=0,1,\dots,J$, where $J\lesssim \log(1/\alpha)$.
For fixed $\Theta$ the number of $\Theta'$ with $\delta\sim 2^{j}\alpha$ is controlled by the area
of a narrow annulus around $\xi_\Theta$ (see Remark~\ref{rem:A11}):
\[
   \#\Bigl\{\Theta':\,\angle(\xi_\Theta,\xi_{\Theta'})\sim 2^{j}\alpha\Bigr\}
   \ \lesssim\ 
   \frac{(2^{j}\alpha)\cdot\alpha}{r^{2}}
   \ =\ 2^{j}\,\Bigl(\frac{\alpha}{r}\Bigr)^{2}
   \ \sim\ 2^{j}D.
\]
Using \eqref{eq:pair-overlap}, the contribution of one such annulus to the sum over $\Theta'$ equals
\[
   \underbrace{2^{j}D}_{\text{number of neighbors}}\cdot
   \underbrace{\frac{\rho^{4}}{\lambda\,(2^{j}\alpha)}}_{\text{pair overlap}}
   \ =\ 
   D\,\frac{\rho^{4}}{\lambda\,\alpha}
   \quad(\text{independent of }j).
\]
Summing over $j=0,\dots,J$ and using the log budget (§\,\ref{subsec:log-budget}, the factor $J\lesssim\log(1/\alpha)$ is absorbed), we obtain
\[
   \sum_{\Theta'}\bigl|\widetilde{\mathcal T}_{\Theta}\cap \widetilde{\mathcal T}_{\Theta'}\bigr|
   \ \lesssim\ 
   D\,\frac{\rho^{4}}{\lambda\,\alpha}.
\]
An analogous estimate holds after summation over $\Theta$ in the local family, whence
\[
   S \;\lesssim\; 
   D\,\frac{\rho^{4}}{\lambda\,\alpha}
   \;=\;
   D^{1/2}\,\frac{\rho^{4}}{\lambda\,r}
   \;=\;
   D^{1/2}\,\lambda^{-7/3},
\]
since $\alpha=c_{0}rD^{1/2}$, $\rho=\lambda^{-1/2}$, and $r=\lambda^{-2/3}$. Thus
\begin{equation}\label{eq:L2-tubes}
   \Bigl\|\sum_{\Theta}\mathbf 1_{\widetilde{\mathcal T}_{\Theta}}
   \Bigr\|_{L^{2}(Q_{\lambda})}
   \;\lesssim\;
   \lambda^{-7/6}\,D^{1/4}.
\end{equation}
\emph{In particular}, from the stronger form \eqref{eq:pair-overlap-min} (for $\delta\le 1$ the minimum is
$\lambda^{-3/2}$) one obtains the alternative cumulative estimate
\(
   S\ \lesssim\ D\,\lambda^{-3}
\)
and hence
\(
   \bigl\|\sum_{\Theta}\mathbf 1_{\widetilde{\mathcal T}_{\Theta}}\bigr\|_{L^{2}(Q_{\lambda})}
   \ \lesssim\ \lambda^{-3/2}D^{1/2}.
\)
We will refer to the baseline \eqref{eq:L2-tubes}; the strengthened version can be used if desired and does not
affect the global balance.

%-----------------------------------------------------------------------
\subsection{Contribution to the global balance}\label{subsec:tube-balance}
%-----------------------------------------------------------------------
From \eqref{eq:L2-tubes} it follows that the “tube packing” block contributes
\[
   -\tfrac{7}{6}\ \text{in }\lambda,
   \qquad
   +\tfrac{1}{4}\ \text{in }D.
\]
For comparison with classical estimates one is allowed to substitute more conservative
exponents $\bigl(-\tfrac12,\,+\tfrac14\bigr)$; both variants keep the cumulative
balance negative (see §\,\ref{sec:balance}). Recall also that this block is not used together
with \emph{Robust\,Kakeya} (§\,\ref{sec:kakeya}); the regimes are mutually exclusive.
