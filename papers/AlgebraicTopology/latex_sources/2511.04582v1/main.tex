\documentclass[a4paper,reqno]{amsart}
%%%%%%%%%%%%%%%%%%%%%%%%%%%%%%%%%
%%%     SPECIFIC COMMANDS     %%%
%%%%%%%%%%%%%%%%%%%%%%%%%%%%%%%%%

\newcommand{\projrel}{\mathbin{\dot\sim}}

\newcommand{\inprod}{\mathbin{\lrcorner}}

\newcommand{\starop}{\mathbin{\star}}

\usepackage{manfnt}
\newcommand*\cube{\reflectbox{\mbox{\mancube}}}

\newcommand{\sF}{\mathsf{F}}
\newcommand{\sE}{\mathsf{E}}

% WEBS AND FOAMS
\newcommand{\web}{\mathbf{Web}}

\newcommand{\prefoam}{{\widetilde{\mathbf{Foam}}}}
\newcommand{\presfoam}{{\widetilde{\mathbf{SFoam}}}}
\newcommand{\pregfoam}{{\widetilde{\mathbf{GFoam}}}}

\newcommand{\foam}{{\mathbf{Foam}}}
\newcommand{\sfoam}{{\mathbf{SFoam}}}
\newcommand{\gfoam}{{\mathbf{GFoam}}}

\DeclareMathOperator{\qdeg}{qdeg}

\newcommand{\undcomp}{\mathfrak{sl}}

\newcommand{\markedweb}[1][d]{\web^{\greenmarking}_{#1}}
\newcommand{\markedfoam}[1][d]{\foam^{\greenmarking}_{#1}}
\newcommand{\markedsfoam}[1][d]{\sfoam^{\greenmarking}_{#1}}
\newcommand{\markedgfoam}[1][d]{\gfoam^{\greenmarking}_{#1}}

\newcommand{\greenmarking}[1][1.5]{{\mathrel{\tikz \node[green_mark={#1 pt}]{};}}}
\newcommand{\roundmarking}[1][1.5]{{\mathrel{\tikz \node[round_mark={#1 pt}]{};}}}
\newcommand{\textdot}[1][1]{{\mathrel{\tikz \fdot[0][0][#1];}}}


% GENERIC CATEGORIES
\newcommand{\gMod}{\fg\md\mathrm{Mod}}

\DeclareMathOperator{\uHom}{\underline{\Hom}}
\DeclareMathOperator{\uEnd}{\underline{\End}}
\DeclareMathOperator{\uMod}{\underline{\Mod}}
\DeclareMathOperator{\ugMod}{\fg\md\underline{\Mod}}

\DeclareMathOperator{\Ch}{Ch}


% STRUCTURE
\newcommand{\ring}{\Bbbk}
\newcommand{\grp}{G}
\newcommand{\bil}{\mu}

\newcommand{\ringfoam}{\Bbbk^\mathfrak{f}}
\newcommand{\grpfoam}{\mathbb{Z}^2}
\newcommand{\bilfoam}{\mu^\mathfrak{f}}

% LIE 
\newcommand{\grsl}{\mathfrak{csl}}
\newcommand{\grgl}{\mathfrak{cgl}}

\newcommand{\slt}{{\mathfrak{sl}_{2}}}
\newcommand{\glt}{{\mathfrak{gl}_{2}}}
\newcommand{\sloo}{{\mathfrak{sl}_{1|1}}}
\newcommand{\gloo}{{\mathfrak{gl}_{1|1}}}

\newcommand{\varf}{\lambda_\mathsf{f}}
\newcommand{\vare}{\lambda_\mathsf{e}}
\newcommand{\varh}{\lambda_\mathsf{h}}
\newcommand{\delh}{\delta_\mathsf{h}}

\newcommand{\lief}{f}
\newcommand{\liee}{e}
\newcommand{\lieh}{h}

\newcommand{\sff}{\mathsf{f}}
\newcommand{\sfe}{\mathsf{e}}
\newcommand{\sfh}{\mathsf{h}}

\newcommand{\Der}{\mathrm{Der}}

% HOMOLOGIES
\newcommand{\Kom}{\mathrm{Kom}}
\newcommand{\slKom}{\Kom_{\slt}}
\newcommand{\glKom}{\Kom_{\glt}}

\newcommand{\Kh}{\mathrm{Kh}}
\newcommand{\OKh}{\mathrm{OKh}}
\newcommand{\CKh}{\mathrm{CKh}}

\newcommand{\slKh}{\Kh_{\slt}}
\newcommand{\slOKh}{\OKh_{\slt}}
\newcommand{\slCKh}{\CKh_{\slt}}
\newcommand{\glKh}{\Kh_{\glt}}
\newcommand{\glOKh}{\OKh_{\glt}}
\newcommand{\glCKh}{\CKh_{\glt}}

\newcommand{\redslKh}{\widetilde{\Kh}_{\slt}}
\newcommand{\redslOKh}{\widetilde{\OKh}_{\slt}}
\newcommand{\redslCKh}{\widetilde{\CKh}_{\slt}}

\newcommand{\redOKh}{\widetilde{\OKh}}
\newcommand{\redKom}{\widetilde{\Kom}}


%%%%%%%%%%%%%%%%%%%%%%%%%%%%%%%%%
%%%     EVERYDAY COMMANDS     %%%
%%%%%%%%%%%%%%%%%%%%%%%%%%%%%%%%%
% everyday maths
\newcommand{\abs}[1]{\left\lvert#1\right\lvert}
\newcommand{\brak}[1]{\langle #1\rangle}
\newcommand{\Id}{\mathrm{Id}}
\newcommand{\id}{\mathrm{id}}

\DeclareMathOperator{\tr}{tr}
\DeclareMathOperator{\rank}{rank}
\DeclareMathOperator{\im}{im}

\DeclareMathOperator{\End}{End}
\DeclareMathOperator{\Hom}{Hom}
\DeclareMathOperator{\HOM}{HOM}
\DeclareMathOperator{\END}{END}
\DeclareMathOperator{\twoHom}{2Hom}
\DeclareMathOperator{\twoEnd}{2End}
\DeclareMathOperator{\colim}{colim}
\DeclareMathOperator{\ob}{ob}
\DeclareMathOperator{\Fun}{Fun}
\DeclareMathOperator{\Cone}{Cone}

\DeclareMathOperator{\Mod}{Mod}
\DeclareMathOperator{\op}{op}
\DeclareMathOperator{\MOD}{MOD}

% require amsmath. still need to write the dash after: \nbd-  
\newcommand{\nbd}{\nobreakdash}

% LIE ALGEBRAS
\newcommand{\fg}{\mathfrak{g}}
\newcommand{\fh}{\mathfrak{h}}
\newcommand{\fsl}{\mathfrak{sl}}
\newcommand{\fgl}{\mathfrak{gl}}

% UTILITIES
\newcommand{\an}{\text{ and }}
\newcommand{\md}{\text{-}}
\newcommand{\ov}[1]{{\overline{#1}}}
\newcommand{\ul}[1]{{\underline{#1}}}

% diamond qed
\newcommand\xqed[1]{%
  \leavevmode\unskip\penalty9999 \hbox{}\nobreak\hfill
  \quad\hbox{#1}}
\newcommand\diamondqed{\xqed{$\diamond$}}

% technical commands
\newcommand{\ra}{\rightarrow}
\newcommand{\xra}[1]{\xrightarrow{#1}}

\newcommand{\me}[1]{
    \relax\ifmmode\mathbf{\textcolor{blue}{PV: #1}}\else\textbf{\textcolor{blue}{ME: #1}}\fi%
}
\newcommand{\ls}[1]{
    \relax\ifmmode\mathbf{\textcolor{orange}{LS: #1}}\else\textbf{\textcolor{orange}{LS: #1}}\fi%
}
\newcommand{\todo}[1]{
    \relax\ifmmode\mathbf{\textcolor{red}{TODO: #1}}\else\textbf{\textcolor{red}{TODO: #1}}\fi%
}


%%%%%%%%%%%%%%%%%%%%%%%%%%%%%%
%%%     FONT SHORTCUTS     %%%
%%%%%%%%%%%%%%%%%%%%%%%%%%%%%%
\newcommand{\cA}{\mathcal{A}}
\newcommand{\cB}{\mathcal{B}}
\newcommand{\cC}{\mathcal{C}}
\newcommand{\cD}{\mathcal{D}}
\newcommand{\cE}{\mathcal{E}}
\newcommand{\cF}{\mathcal{F}}
\newcommand{\cG}{\mathcal{G}}
\newcommand{\cH}{\mathcal{H}}
\newcommand{\cI}{\mathcal{I}}
\newcommand{\cJ}{\mathcal{J}}
\newcommand{\cK}{\mathcal{K}}
\newcommand{\cL}{\mathcal{L}}
\newcommand{\cM}{\mathcal{M}}
\newcommand{\cN}{\mathcal{N}}
\newcommand{\cO}{\mathcal{O}}
\newcommand{\cP}{\mathcal{P}}
\newcommand{\cQ}{\mathcal{Q}}
\newcommand{\cR}{\mathcal{R}}
\newcommand{\cS}{\mathcal{S}}
\newcommand{\cT}{\mathcal{T}}
\newcommand{\cU}{\mathcal{U}}
\newcommand{\cV}{\mathcal{V}}
\newcommand{\cW}{\mathcal{W}}

\newcommand{\bA}{\mathbb{A}}
\newcommand{\bB}{\mathbb{B}}
\newcommand{\bC}{\mathbb{C}}
\newcommand{\bD}{\mathbb{D}}
\newcommand{\bE}{\mathbb{E}}
\newcommand{\bF}{\mathbb{F}}
\newcommand{\bG}{\mathbb{G}}
\newcommand{\bH}{\mathbb{H}}
\newcommand{\bI}{\mathbb{I}}
\newcommand{\bJ}{\mathbb{J}}
\newcommand{\bK}{\mathbb{K}}
\newcommand{\bL}{\mathbb{L}}
\newcommand{\bM}{\mathbb{M}}
\newcommand{\bN}{\mathbb{N}}
\newcommand{\bO}{\mathbb{O}}
\newcommand{\bP}{\mathbb{P}}
\newcommand{\bQ}{\mathbb{Q}}
\newcommand{\bR}{\mathbb{R}}
\newcommand{\bS}{\mathbb{S}}
\newcommand{\bT}{\mathbb{T}}
\newcommand{\bU}{\mathbb{U}}
% \newcommand{\bV}{\mathbb{V}}
\newcommand{\bW}{\mathbb{W}}
\newcommand{\bX}{\mathbb{X}}
\newcommand{\bY}{\mathbb{Y}}
\newcommand{\bZ}{\mathbb{Z}}

\newcommand{\rA}{\mathrm{A}}
\newcommand{\rB}{\mathrm{B}}
\newcommand{\rC}{\mathrm{C}}
\newcommand{\rD}{\mathrm{D}}
\newcommand{\rE}{\mathrm{E}}
\newcommand{\rF}{\mathrm{F}}
\newcommand{\rG}{\mathrm{G}}
\newcommand{\rH}{\mathrm{H}}
\newcommand{\rI}{\mathrm{I}}
\newcommand{\rJ}{\mathrm{J}}
\newcommand{\rK}{\mathrm{K}}
\newcommand{\rL}{\mathrm{L}}
\newcommand{\rM}{\mathrm{M}}
\newcommand{\rN}{\mathrm{N}}
\newcommand{\rO}{\mathrm{O}}
\newcommand{\rP}{\mathrm{P}}
\newcommand{\rQ}{\mathrm{Q}}
\newcommand{\rR}{\mathrm{R}}
\newcommand{\rS}{\mathrm{S}}
\newcommand{\rT}{\mathrm{T}}
\newcommand{\rU}{\mathrm{U}}
\newcommand{\rV}{\mathrm{V}}
\newcommand{\rW}{\mathrm{W}}
\newcommand{\rX}{\mathrm{X}}
\newcommand{\rY}{\mathrm{Y}}
\newcommand{\rZ}{\mathrm{Z}}

\newcommand{\fA}{\mathfrak{A}}
\newcommand{\fB}{\mathfrak{B}}
\newcommand{\fC}{\mathfrak{C}}
\newcommand{\fD}{\mathfrak{D}}
\newcommand{\fE}{\mathfrak{E}}
\newcommand{\fF}{\mathfrak{F}}
\newcommand{\fG}{\mathfrak{G}}
\newcommand{\fH}{\mathfrak{H}}
\newcommand{\fI}{\mathfrak{I}}
\newcommand{\fJ}{\mathfrak{J}}
\newcommand{\fK}{\mathfrak{K}}
\newcommand{\fL}{\mathfrak{L}}
\newcommand{\fM}{\mathfrak{M}}
\newcommand{\fN}{\mathfrak{N}}
\newcommand{\fO}{\mathfrak{O}}
\newcommand{\fQ}{\mathfrak{Q}}
\newcommand{\fR}{\mathfrak{R}}
\newcommand{\fS}{\mathfrak{S}}
\newcommand{\fT}{\mathfrak{T}}
\newcommand{\fU}{\mathfrak{U}}
\newcommand{\fV}{\mathfrak{V}}
\newcommand{\fW}{\mathfrak{W}}




\begin{document}

\title{Embedding Calculus, Goodwillie calculus and link invariants}
\author{Hyeonhee Jin}
\address{Max Planck Institute for Mathematics, Vivatsgasse 7, 53111 Bonn, Germany}
\email{hh17wlog@gmail.com}


\setcounter{tocdepth}{1}

\begin{abstract}
We study Goodwillie-Weiss' embedding calculus through its relationship with Goodwillie’s functor calculus.
Specifically, building on a result of Tillmann and Weiss, we construct a functorial complement for \(T_{n}\)-embeddings that takes values in Heuts’ categorical $n$-excisive approximation of pointed spaces.
We also establish an analogue of Stallings' theorem for lower central series in the context of \(T_{n}\)-embeddings of \(P \times I\) into \(D^{d}\) for any compact manifold \(P\).
As an application, we show that the embedding tower of string links detects Milnor invariants.
\end{abstract}


\maketitle



\section{Introduction}
The study of embedding spaces $\emb(M,N)$ between manifolds $M$ and $N$ is an important topic in geometric topology.  
One powerful homotopical tool for analyzing such spaces is the \emph{embedding calculus} of Goodwillie and Weiss.  
A closely related but distinct framework is Goodwillie’s functor calculus on the identity functor of pointed spaces.
In this paper, we investigate how these two frameworks interact and show that this relationship yields new insight into the embedding tower; in particular, the embedding tower detects Milnor invariants of string links.

Embedding calculus approximates the space of embeddings \(\emb(M,N)\) by a sequence of \enquote{polynomial} approximations.
\[\begin{tikzcd}[ampersand replacement=\&,cramped,column sep=tiny]
	{\emb(M,N)} \& {T_{\infty}\emb(M,N)} \& \cdots \& {T_2\emb(M,N)} \& {T_1\emb(M,N)}
	\arrow[from=1-1, to=1-2]
	\arrow[from=1-2, to=1-3]
	\arrow[from=1-3, to=1-4]
	\arrow[from=1-4, to=1-5]
\end{tikzcd}\]
Each \(n\)-th stage records how collections of configurations of up to \(n\) disjoint discs in \(M\) can be embedded into \(N\), together with the compatibility data of these embeddings.
The first stage records immersions, and the fibers of the map \(T_{n}\emb(M,N) \to T_{n-1} \emb(M,N)\) admit an explicit description.
Thus, the tower provides an inductive procedure for interpolating between immersions and embeddings.

A recurring idea in geometric topology is to understand embeddings via their complements. 
In the context of embedding calculus, one might hope to extract information about the complement of an embedding from the complements of the embedded discs.
Tillmann and Weiss \cite{TW16} showed that this approach is effective when the codimension is at least three:
for an embedding \(e \colon M \hookrightarrow N\), 
\[
    N \sminus e(M) \to \holim_{U \in \Disc(M)} N \sminus e(U)
\]
is a weak equivalence when \(\dim N - \dim M \geq 3\).

Moreover, they note that after passing to the \(n\)-th Goodwillie approximation of spaces, the above map becomes an equivalence in all codimensions:
\[
    P_{n}(\id_{\pS})(N \sminus e(M)) \xrightarrow{\simeq} \holim_{U \in \Discn(M)} P_{n}(\id_{\pS})(N \sminus e(U))
\]
Since the Goodwillie tower of pointed spaces converges for simply connected spaces, this recovers the previous equivalence when \(N\) is simply connected, \(\dim N - \dim M \geq 3\), and \(n= \infty\).


The result of Tillmann and Weiss suggests the existence of a well-defined notion of complement for the space of \(T_{n}\)-embeddings \(T_{n}\emb(M,N)\), taking values in the categorical \(n\)-th Goodwillie approximation of pointed spaces \(\PnS\) introduced in \cite{H18a}.
We construct such a notion by lifting Tillmann and Weiss' result to the level of \(\infty\)-categories.
This yields the \emph{\(T_{n}\)-complement} functor: 

\begin{bigthm}\label{theorem: embedding calc}
   Let \(M\) be a compact manifold with boundary and let \(N\) be a manifold with boundary. Fix an embedding \(i \colon \partial M \hookrightarrow \partial N\) and a base point \(x\) in \(\partial N \sminus i(\partial M)\).
    There is a commutative diagram of \(\infty\)-categories
    \[ \begin{tikzcd}
            {\bdemb(M,N)} \arrow[r,"C"] \arrow[d] & \pS^{\partial/} \arrow[d,"\nSigma"] \\
            {T_n\bdemb(M,N)} \arrow[r,"C_n"]        & \PnS^{\nSigma \partial/}                   
        \end{tikzcd}
    \]
    where \(C\) sends an embedding \(e \in \bdemb(M,N)\) to the pointed map \((\partial{N}\sminus i(\partial{M}),x) \to (N\sminus e(M),x)\). 
    The commutativity of the diagram shows that \(C_{n}\) extends the complement functor from embeddings to \(T_{n}\)-embeddings. 
\end{bigthm}

Here \(\pS^{\partial/}\) is the category of pointed spaces under \(\partial \coloneq \partial N \sminus i(\partial M)\) and the right vertical map is the induced map on undercategories determined by \(\nSigma \colon \pS \to \PnS\), the \(n\)-excisive approximation to \(\pS\). 

\begin{remark}
    The restriction to manifolds with boundary is to ensure that a canonical basepoint can be chosen in the complement.
    We expect an analogous statement for manifolds without boundary and unpointed spaces.
\end{remark}

When \(n=1\), we can describe \(C_{1}\) as follows. The following description was suggested by Oscar Randal-Williams.
For an embedding \(f \colon M \hookrightarrow N\) there is a homotopy cofiber sequence in \(\pS\)
\[
    N\sminus f(M)_{+} \to N_{+} \to M^{\nu_f}
\]
where \(M^{\nu _{f}}\) denotes the Thom space of the normal bundle \(\nu _{f}\). 
Passing to spectra and taking Spanier-Whitehead duals gives a cofiber sequence in \(\Spct\)
\[
    D(\Sigma^{\infty}_{+} N\sminus f(M)) \leftarrow N^{-TN} \leftarrow M^{-\nu _{f} -TM}
\]
For a general bundle monomorphism \(F \in  T_{1}\emb(M,N)\simeq \mathrm{Mono}(TM,TN)\), 
we replace \(\nu _{f}\) by \(V_{F} \coloneq F^{*}TN/TM\) and define \(C_{1}(F)\) to be the spectrum \(D(\cofib(M^{-V_{F}-TM}\to N^{-TN}))\).
This provides a continuous \enquote{complement} functor for general immersions.

\subsection*{Stallings' theorem for \(T_{n}\)-complements in \(D^{d}\)}

For a finite pointed space \(X\), the \(n\)-excisive approximation \(P_{n}(\id_{\pS})(X)\) has vanishing \((n+1)\)-fold iterated Whitehead products \cite{BD10,SC15}. 
Hence each \(n\)-th stage of the Goodwillie tower for spaces can be regarded as a homotopical analogue of the \(n\)-nilpotent quotients of groups. 

Stallings' theorem for lower central series \cite{S65} states that for groups \(A,B\) if a group homomorphism \(\phi \colon A \to B\) induces an isomorphism on the first group homology and surjection on the second group homology, then it induces isomorphisms on all nilpotent quotients: 
\[
    \phi_{n} \colon A/A_{n+1} \xrightarrow{\cong} B/B_{n+1} \quad \textrm{for all } n \in \N 
\]
Here \(\{A_{n}\} \) and \(\{B_{n}\}\) denote the lower central series of \(A\) and \(B\), respectively. 

Analogously, for pointed spaces \(X,Y\), a homology equivalence \(f \colon X \to Y\) induces equivalences on all \(n\)-nilpotent approximations:
\[
    P_{n}(f) \colon P_{n}(X) \xrightarrow{\simeq} P_{n}(Y) \quad \textrm{for all } n \in \N
\]

As an example, consider an embedding \(e \colon P \times I \hookrightarrow D^{d-1} \times I \cong D^{d}\) that restricts to an embedding \(i \colon P \hookrightarrow D^{d-1}\) on \(P \times \set{0}\). 
The boundary inclusion induces a map on complements
\[
    D^{d-1}\sminus i(P)  \rightarrow D^{d}\sminus e(P\times I) 
\]
which is a homology equivalence by Alexander duality. 
By the discussion above, it follows that the induced map on \(n\)-nilpotent approximations is an equivalence:
\[
    P_{n}(D^{d-1}\sminus i(P)) \xrightarrow{\simeq} P_{n}(D^{d} \sminus e(P \times I)) 
\]

We extend this statement to the complements of \(T_{n}\)-embeddings of \(P \times I\) to \(D^{d}\).


\begin{bigthm}\label{thm:D^n eq}
    Let \(P\) be a compact manifold. Let \(i\) be an embedding of \(P\) into the interior of \(D^{n-1}\).
    For all \(n \in \N\), the \(T_{n}\)-complement functor \(C_{n}\) lands in the subcategory of equivalences
    \[
        T_{n}\bdemb(P \times I,D^{d}) \xrightarrow{C_{n}} ({\PnS}^{\simeq})^{\nSigma D^{d-1} \setminus i(P)/}
    \]
    i.e. For \(\eta \in T_{n}\bdemb(P \times I, D^{d})\), the boundary inclusion induced map \(\nSigma(D^{d-1}\sminus i(P)) \to C_{n}(\eta)\) is an equivalence in \(\PnS\). 
\end{bigthm}



\subsection*{Artin Representation for string links}

We now apply Theorem~\ref{thm:D^n eq} to the case of string links and show that the Artin representation for string links factors through the embedding tower.


The classical Artin representation is a group homomorphism
\[
     PB(k) \to \Aut_{\Gp}(F(k)),
\]
from the pure braid group on \(k\)-strands to the automorphism group of the free group on \(k\) generators. 
It can be described as follows. 
For each pure braid \(\beta \in \pi_{1}\mathrm{Emb}(\underline{k},D^{2})\), where \(\ul{k} \coloneq \{1,\cdots,k\}\), there are inclusions 
\[
    (D^{2}\sminus \ul{k})\times \{0\} \xhookrightarrow[\simeq]{i_0(\beta)} D^{3} \sminus \beta \xhookleftarrow[\simeq]{i_{1}(\beta)} (D^{2}\sminus\ul{k}) \times \{1\}.
\]
Both \(i_{0}(\beta)\) and \(i_{1}(\beta)\) are homotopy equivalences. 
Composing \(i_{0}(\beta)\) with a homotopy inverse of \(i_{1}(\beta)\) yields a homotopy automorphism of \(D^{2} \sminus \ul{k}\), and hence an automorphism of the free group \(F(k)= \pi_{1}(D^{2} \sminus \ul{k})\). 


Similarly, for a string link \(L \in \pi_{0}\bdemb(kD^{1},D^{3})\), where \(kD^{1} \coloneq \sqcup^{k}D^{1}\), we have inclusions of the top and bottom complements \(i_0(L),i_{1}(L) \colon D^{2}\sminus \ul{k} \hookrightarrow D^{3} \sminus L\).
Here \(i_0(L)\) and \(i_{1}(L)\) are homology equivalences by Alexander duality, but not homotopy equivalences in general. Nevertheless, by Stallings' theorem, both maps induce isomorphism on the \(n\)-nilpotent quotients of the fundamental groups. 
The composition of these isomorphisms defines, for each \(n\), the \textit{Artin representation for string links}
\[
    {A_{n}} \colon {\pi_{0}\bdemb(kD^{1},D^{3})} \rightarrow {\Aut_{\Gp}(F(k)/F(k)_{n+1})} \quad n\in \N
\]
These representations encode rich information:
Milnor's \(\bar{\mu}\)-invariants \cite{M54} can be extracted from them via the Magnus expansion, and Habegger and Lin \cite{HL90} showed that they completely classify string links up to link homotopy.

The discussion in the previous subsection shows that there exists a space-level analogue of the Artin representation 
\[
    \bdemb(kD^{1}, D^{3}) \to \Aut_{\PnS}(\nSigma \kcirc )
\] which factors through the space of \(T_{n}\)-embeddings.

Using the result of Biedermann and Dwyer \cite{BD10} which identifies \(\pi_{1}P_{n}(\id_{\pS})(\kcirc)\) as \(F(k)/F(k)_{n+1}\), we identify the induced map on \(\pi_{0}\) as the Artin representation for string links.

\begin{bigcor}\label{corollary: Artin}
    There exists a map of spaces
\[
       {\bdemb(kD^{1},D^3)} \rightarrow {T_n\bdemb(kD^{1},D^3)} \xrightarrow{\mathscr{A}_{n}} \Aut_{\PnS}(\nSigma\kcirc)
\]
such that the induced map on path components 
\[
    \pi_{0} {\bdemb(kD^{1},D^3)}  \to \pi_{0}\Aut_{\PnS}(\nSigma\kcirc) \cong \Aut_{\mathrm{Gp}}(F(k)/F(k)_{n+1})
\] is the Artin representation for string links. In particular, the \(n\)-th stage of the embedding tower for string links detects Milnor invariants of length \(\leq n+1\). 
\end{bigcor}

\subsection*{Related Works}
Koytcheff \cite{K16} showed, using configuration space integrals, that Milnor's triple linking number \(\mu_{123}\) factors through the \((2,2,2)\)-stage of the multi-variable embedding tower for string links.
In \cite{CKKS17}, it was shown that Koschorke's \(\kappa\)-invariant \cite{K97} can be recovered from the \((1,\cdots,1)\)-stage of the multi-variable tower for link maps of string links \(T_{(1,\cdots,1)}\mathrm{Link}_{\partial}(kD^{1},D^{3})\).
Another closely related work by Munson \cite{M11} connects Koschorke's \(\kappa\) invariant to the map between the layers of embedding and functor calculus towers. 
An explicit map comparing the Lie bracket structure in homogeneous layers of embedding tower of \(\bdemb(D^{1},D^{3})\) and those of \(P_{n}(\id_{\pS})\) is given in \cite{K25}.
Malin \cite{M24b} showed that for a framed manifold \(M\), the Lie structures on the stable embedding functor \(\Sigma_{+}^{\infty}E(-,M)\) arising from embedding calculus and from functor calculus coincides. 

\subsection*{Acknowledgements} 
I would like to thank Manuel Krannich and Jaco Ruit for helpful conversations. 
Special thanks go to Luciana Basualdo Bonatto and Peter Teichner for discussions on Milnor invariants and the embedding tower of string links and to Samuel Mu\~noz-Ech\'aniz for suggesting the use of Lemma \ref{lemma : P_n of complement is n sheaf}. 
I am grateful to Kaif Hilman for suggestions on the formulation of Theorem \ref{Prop: commutativity with emb} and discussions on naturality statements. 
I sincerely thank Oscar Randal-Williams for generously sharing the key ideas for the proof of Theorem \ref{prop: Bordism equivalence}, discussing functorial approaches to taking complements and giving feedback on the first draft.


\section{Preliminaries}

In this section, we fix notation and review results in the literature needed for the remainder of the paper.

\subsection{Conventions}\label{sec:conventions}
Unless specified otherwise, we follow the notations and conventions of \cite{L09}. 
\begin{itemize}
    \item For a natural number \(n\), let \(\ul{n}\) denote \(\set{1,\cdots,n}\).
    \item \(\iS\) denotes the \(\infty\)-category of spaces.
    \item The letters \(\icat{A},\icat{B},\icat{C},\cdots\) stand for \(\infty\)-categories.
    \item The letters \(\topcat{A},\topcat{B},\topcat{C},\cdots\) stand for \(\Top\) or \(\mathrm{Kan}\)-enriched categories.
    \item For \(\iC\) and \(\infty\)-category and \(c \in \iC\) an object, let \(\iC_{/c}\) be the \textit{over category} of \(\iC\) and \(\iC^{c/}\) be the \textit{under category} of \(\iC\). 
    \item \(\iC_{/c}^{\op}\) and \({\iC^{c/}}^{\op}\) will denote \((\iC_{/c})^{\op}\) and \((\iC^{c/})^{\op}\) respectively.
    \item Given a functor \(f \colon \iC \to \iD\) and an \(\infty\)-category \(\iE\), we let \(f^{\ast} \colon \Fun(\iD,\iE) \rightarrow \Fun(\iC,\iE)\) denote the precomposition with \(f\), \(f_{!} \colon \Fun(\iC,\iE) \to \Fun(\iD,\iE) \) the left Kan extension along \(f\) and \(f_{*} \colon \Fun(\iD,\iE) \to \Fun(\iC,\iE)\) the right Kan extension along \(f\).
    \item For adjunctions, 
    \[\begin{tikzcd}[ampersand replacement=\&,cramped]
	    \iC \& \iD
	    \arrow["F", shift left, from=1-1, to=1-2]
	    \arrow["G", shift left, from=1-2, to=1-1]
    \end{tikzcd}\]
    left adjoints will be written at the top and the right adjoints will be written at the bottom.
\end{itemize}


\subsection{Goodwillie Calculus}\label{sec:goodwillie calc} 

\subsubsection{Goodwillie calculus on functors}
For a finite set \(S\), let \(\mathscr{P}(S)\) denote the poset category of subsets of \(S\) ordered by inclusion. Let \(\mathscr{P}_{  i}(S)\)
and \(\mathscr{P}_{>i}(S)\) denote the subposets spanned by subsets of cardinality at most and greater than \(i\),respectively.

\begin{definition} Let \(\iC\) be a \(\infty\)-category. An \textit{n-cube} in \(\iC\) is a functor from \(\mathscr{P}(\ul{n})\) to \(\iC\). 
    An n-cube is called \textit{cartesian} if it is a limit diagram and \textit{strongly cocartesian} if every two-dimensional face is a pushout. 
\end{definition}


\begin{definition}[n-excisive functors]
	A functor \(F \colon \iC \to \iD\) is \textit{\(n\)-excisive} if it sends strongly cocartesian \((n+1)\)-cubes to cartesian ones. 
	Let \(\mathrm{Exc}^{n}(\iC,\iD)\) denote the full subcategory of \(\Fun(\iC,\iD)\) spanned by the \(n\)-excisive functors. 
\end{definition}


When \(\iC\) and \(\iD\) admit suitable limits and colimits, every functor \(F \colon \iC \to \iD\) admits a tower of universal excisive approximations.

\begin{proposition}[Goodwillie towers]\cite{G03}\cite[Chapter 6]{L17}
	Let \(\iC\) be an \(\infty\)-category which admits finite colimits and has a final object, and let \(\iD\) be a compactly generated \(\infty\)-category. 
    
    \begin{enumerate}
        \item For each \(n\), the inclusion
        \[       
            \mathrm{Exc}^{n}(\iC,\iD) \hookrightarrow \Fun(\iC,\iD)
        \] admits a left exact left adjoint \(P_{n}\).

        \item If \(n \leq m\), every \(n\)-excisive functor is \(m\)-excisive. Hence we have inclusions
        \[
           \Fun(\iC,\iD) \supset \cdots \supset \mathrm{Exc}^{2}(\iC,\iD) \supset \mathrm{Exc}^{1}(\iC,\iD) 
        \] and for each \(F \in \Fun(\iC,\iD)\) we obtain a natural tower of functors
        \[
            F \to \cdots \to P_{2}F \to P_{1}F
        \]
    \end{enumerate}
\end{proposition}

The unit map \(F \to P_{n}F\) is called the \textit{\(n\)-excisive approximation} of \(F\).

\begin{example}[Goodwillie tower of the identity on pointed spaces]\label{eg: Goodwillie tower of spaces}
    The Goodwillie tower of the identity functor \(\id_{\pS}\) has the following properties:
    \begin{enumerate}[label={(\alph*)}]
        \item \(P_{1}\id_{\pS} \simeq \Omega^{\infty}\Sigma^{\infty}\). Hence the Goodwillie tower of \(\id_{\pS}\) interpolates between the stable homotopy type and the unstable homotopy type.     
        \item The \(n\)-th \emph{layer} of the tower,
        \(
            D_{n}(\id_{\pS}) \eqcolon \fib (P_{n}(\id_{\pS})\to P_{n-1}(\id_{\pS}))
        \) is equivalent to the infinite loop space 
        \[
            D_{n}(\id_{\pS}) \simeq \Omega^{\infty}((\partial_{n}\id \wedge (\nSigma X)^{\smash n})_{h\Sigma_{n}})
        \]
        where \(\partial_{n}\id\) is a spectrum with \(\Sigma_{n}\) action, equivalent to the \(n\)-th term of the desuspended spectral Lie operad~\cite{C05}. 
        \item For a finite pointed space \(X\), its \(n\)-th approximation \(P_{n}(X)\) has vanishing iterated \((n+1)\)-fold Whitehead products~\cite{SC15}. 
        Moreover, \(\pi_{1}P_{n}(\kcirc)\) is isomorphic to the \(n\)-th nilpotent quotient of the free group with \(k\) generators \(F(k)/F(k)_{n+1}\)~\cite{BD10}.
    \end{enumerate}
\end{example}



\subsubsection{Goodwillie calculus on categories}
In this paper, we work with the categorical Goodwillie calculus defined by Heuts in \cite{H18a}.
This assigns to each pointed compactly generated \(\infty\)-category \(\iC\), a tower of pointed compactly generated \(\infty\)-categories \(\{\mathscr{P}_{n}\iC\}_{n \in \N}\) together with adjunctions

\[\begin{tikzcd}[ampersand replacement=\&]
	\iC \& {\mathscr{P}_n\iC}
	\arrow["\nSigma", shift left, from=1-1, to=1-2]
	\arrow["{\Omega_n^{\infty}}", shift left, from=1-2, to=1-1]
\end{tikzcd}\]

which is called the \emph{\(n\)-excisive approximation to \(\iC\)}.


We recall the following properties of categorical \(n\)-excisive approximation:

    \begin{enumerate}[label={(\alph*)}]
        \item The identity functor \(\id_{\mathscr{P}_{n}\iC}\) is \(n\)-excisive.
        \item The unit \(\eta \colon \id_{\iC} \to \Omega_{n}^{\infty}\nSigma\) is equivalent to the \(n\)-excisive approximation of the identity functor \(\id_{\iC} \to P_{n}(\id_{\iC})\).
        \item \(\Sigma_{1}^{\infty} \colon \iC \to \mathscr{P}_{1}\iC\) is the stabilisation of \(\iC\). In particular, \(\mathscr{P}_{1}\pS \simeq \Spct\).
        \item For each \(m \leq n\), there is a canonical equivalence \(\mathscr{P}_{m}(\mathscr{P}_{n}\iC) \simeq \mathscr{P}_{m}\iC\). 
        The corresponding adjunctions 
            \[\begin{tikzcd}[ampersand replacement=\&]
                {\mathscr{P}_{n}\iC} \& {\mathscr{P}_{m}\iC}
                \arrow["{\Sigma_{n,m}^{\infty}}", shift left, from=1-1, to=1-2]
                \arrow["{\Omega_{n,m}^{\infty}}", shift left, from=1-2, to=1-1]
            \end{tikzcd}\]
        assemble to a \textit{Goodwillie tower} of \(\iC\):
        \[\begin{tikzcd}[ampersand replacement=\&,cramped]
        \iC \\
        \cdots \& {\mathscr{P}_{3}\iC} \& {\mathscr{P}_{2}\iC} \& {\mathscr{P}_{1}\iC}
        \arrow[from=1-1, to=2-1]
        \arrow["{\Sigma_{3}^{\infty}}"{description}, from=1-1, to=2-2]
        \arrow["{\Sigma_{2}^{\infty}}"{description}, curve={height=-6pt}, from=1-1, to=2-3]
        \arrow["{\Sigma_{1}^{\infty}}"{description}, curve={height=-12pt}, from=1-1, to=2-4]
        \arrow[from=2-1, to=2-2]
        \arrow["{\Sigma_{3,2}^{\infty}}"', from=2-2, to=2-3]
        \arrow["{\Sigma_{2,1}^{\infty}}"', from=2-3, to=2-4]
        \end{tikzcd}\]
    \end{enumerate}





\subsection{Embedding Calculus}\label{sec:embedding calculus}

\begin{notation}
Let \(\topcat{Mfld}_{d}\) denote the topologically enriched category of smooth \(d\)-dimensional manifolds and codimension 0 embeddings.  
Denote the \(\infty\)-category obtained by taking its coherent nerve by \(\Mfld_{d}\). 
Let \(\Disc_{d}\) be the full subcategory spanned by manifolds diffeomorphic to finite disjoint unions of \(\R^{d}\), and \(\Disc_{\leq n,d}\) denote the full subcategory spanned by manifolds diffeomorphic to \(S \times \R^{d}\) with \(|S| \leq n\). 
We write \(\iota_{n} \colon \Disc_{\leq n,d} \hookrightarrow \Mfld_{d}\) for the inclusion functor.
\end{notation}





Let \(\iC\) be a \(\infty\)-category with small limits. 

There is a reflective adjunction of \(\infty\)-categories
\[\begin{tikzcd}
	{\Fun(\Mfld_{d}^{\op},\iC)} & {\Fun(\Disc_{\leq n,d}^{\op},\iC)}
	\arrow[""{name=0, anchor=center, inner sep=0}, "{\iota_{n}^*}", shift left, from=1-1, to=1-2]
	\arrow[""{name=1, anchor=center, inner sep=0}, "{{\iota_{n}}_{*}}", shift left, hook', from=1-2, to=1-1]
\end{tikzcd}\]

\begin{definition}[Manifold \(n\)-excisive functors]
A functor \( F \in \Fun(\Mfld_{d}^{\op},\iC) \) is called \emph{manifold \(n\)-excisive} if the unit of the adjunction
\[
	F \longrightarrow \lst{\iota_{n}}\ust{\iota_{n}}F \eqcolon T_{n}F
\]
is an equivalence.  
\end{definition}


The following characterisation of manifold \(n\)-excisive functors for \(\iC = \iS\) is due to \cite{W99a} and \cite[Theorem~7.2]{dW12};
For general \(\iC\), see \cite[Theorem~5.3]{A24}.

\begin{definition}[Good functors]
    A functor \(F \in \Fun(\Mfld_{d}^{\op},\iC)\) is called \textit{good} if for all \(M \in \Mfld_{d}\) and all increasing sequence of open subsets \(U_{0} \subset U_{1} \subset \cdots \subset M \) with \(\cup_{i}U_{i} = M\), the natural map
    \[
        F(M) \to \lim_{i}F(U_{i})
    \]
    is an equivalence.  
\end{definition}

\begin{notation}
    Let \(M \in \Mfld_{d}\), and let \(A_{0},\cdots,A_{n} \subset M\) be pairwise disjoint closed subsets. 
    For each \(S \subset \ul{n}\) denote \(\cup_{i \in S} A_{i}\) by \(A_{S}\). 
    Let \(X_{(M,\ul{A})}\) denote the contravariant \((n+1)\)-cube in \(\Mfld_{d}\)
    \[
       X_{(M,\ul{A})} \colon \mathscr{P}(\ul{n+1})^{\op} \to \Mfld_{d}, \quad X_{(M,\ul{A})} (S) = M \sminus A_{S}
    \]
\end{notation}

\begin{thm}\cite{W99a,dW12,A24}\label{thm: original}
   Let \(\iC\) be an \(\infty\)-category with small limits and let \(F \in \Fun(\Mfld_{d}^{\op},\iC)\). The following are equivalent.
    \begin{enumerate}
        \item \(F\) is manifold \(n\)-excisive.
        \item \(F\) is good and for all \(M \in \Mfld_{d}\) and pairwise disjoint closed subsets \(A_{0},\cdots,A_{n} \subset M\), \(F\) sends \(X_{(M,\ul{A})}\) to a cartesian \((n+1)\)-cube. i.e. the canonical map
    \[
      F(M) \to \lim_{\emptyset \neq S \subset \ul{n+1}}F(M \sminus A_{S})
    \]
    is an equivalence.
    \end{enumerate}
\end{thm}


In this paper we use a variant of the above result for manifolds with boundary.
In addition, to work with a category whose objects are \(T_{n}\)-embeddings rather than whose morphisms are, we consider presheaves on an overcategory.



\begin{notation}\label{Notation: Mfld partial /N}
For each natural number \(d\) fix a \((d-1)\)-dimensional manifold \(P_{d}\), and For each \(d \leq d'\), fix an embedding \(i_{d,d'} \colon P_{d} \hookrightarrow P_{d'}\), taking \(i_{d,d} = \id\).
Let \(\topcat{Mfld}_{\partial,A}\) denote the topologically enriched category whose objects are manifolds \(M\) of any dimension with boundary equipped with a diffeomorphism \(\partial M \cong P_{\mathrm{dim}(M)}\), and whose morphisms are neat embeddings that restrict to the prescribed boundary maps.
Let \(\topcat{Mfld}_{\partial,d}\) to be the full subcatgory of \(\topcat{Mfld}_{\partial,A}\) spanned by \(d\)-manifolds,
and let \(\topcat{Disc_{\leq n,\partial,d}}\) be the full subcategory of \(\topcat{Mfld}_{\partial,d}\) spanned by objects that are diffeomorphic to \(P_{d} \times [0,1) \sqcup S \times \R^{d}\) where \(|S| \leq n\).
Denote the \(\infty\)-categories obtained by taking coherent nerve by \(\Mfld_{\partial,A}\),\(\Mfld_{\partial,d}\) and \(\Disc_{ \leq n, \partial,d}\). 
Fix \(N \in \Mfld_{\partial}\) of \(\dim(N)>d\) and let \(\uMfldN\) and \(\uDiscnN\) denote \(\Mfld_{\partial,d} \times_{\Mfld_{\partial,A}} {\Mfld_{\partial,A}}_{/N}\) and \(\Disc_{\partial,\leq n,d} \times_{\Mfld_{\partial,A}} {\Mfld_{\partial,A}}_{/N}\) respectively. 
From now on, we fix \(d\) and omit it from the notation.
\end{notation}

\begin{remark}For \(M \in \uMfld\), denote by \( E_{M} \) the representable presheaf \(\bdemb(-,M)\) by \(E_{M}\).
Since the presheaf category of an over category is equivalent to the over category of presheaf category,
we have \(\Psh(\uMfldN) \simeq \Psh(\uMfld)_{/E_{N}}\) and \(\Psh(\uDiscnN) \simeq \Psh(\uDiscn)_{/\ust{\iota_{n}}E_{N}}\).
The fiber of the right fibration 
\[
    \Psh({\uDiscn}_{/N}) \simeq \Psh({\uDiscn})_{/{\iota_n}^{\ast}E_{N}} \to \Psh({\uDiscn})
\]
at \({\iota_n}^{\ast}E_{M}\) is \(\Map_{\Psh({\uDiscn})}({\iota_n}^{\ast}E_{M},{\iota_n}^{\ast}E_{N}) \simeq T_{n}\bdemb(M,N)\).
\end{remark}


We define the variant of the above notions for \(\Fun(\uMfldN^{\op},\iC)\).





\begin{definition}
A functor \( F \in \Fun(\uMfldN^{\op},\iC)\) is called \emph{manifold \(n\)-excisive} if it lies in the image of the right Kan extension
\(
    \Fun(\uDiscnN^{\op},\iC) \xrightarrow{\ust{\iota_{n}}} \Fun(\uMfldN^{\op},\iC).
\)

\end{definition}


Each embedding \(e \colon M \hookrightarrow N \in \uMfldN\) determines a functor \(L_{e} \colon \cO(M) \longrightarrow \uMfldN\) from \(\cO(M)\), the poset of open subsets of \(M\), such that the diagram
\begin{equation}\label{eq:Le}
   \begin{tikzcd}[ampersand replacement=\&,cramped]
        \& \uMfldN \\
        {\cO(M)} \& \uMfld
        \arrow[from=1-2, to=2-2]
        \arrow["L_{e}",from=2-1, to=1-2]
        \arrow[from=2-1, to=2-2]
    \end{tikzcd}
\end{equation}
commutes.


\begin{definition}
    A functor \(F \in \Fun(\uMfldN^{\op},\iC)\) is called \textit{good} 
    if for all \(e \colon M \hookrightarrow N \in \uMfldN\) and increasing sequence of open subsets \(U_{0} \subset U_{1} \subset \cdots \subset M \) with \(\cup_{i}U_{i} = M\), the natural map
    \[
        F(L_{e}(M)) \to \lim_{i}F(L_{e}(U_{i}))
    \]
    is an equivalence.  
\end{definition}



\begin{notation}\label{notation: manifold cube}
    Let \(M\) be a \(d\)-manifold which admits a finite handle decomposition.
    Let \(A_{0},\cdots,A_{n} \subset M\) be pairwise disjoint closed subsets in the interior of \(M\) such that  \(M \sminus A_{S}\) again admits a finite handle decomposition. 
    and let \(L_{e}\) be the functor of \eqref{eq:Le}.
    We let \(X^{e}_{(M,\ul{A})}\) denote the contravariant \((n+1)\)-cube in \(\uMfldN\) defined as 
    \(X^{e}_{(M,A)} (S) = L_{e}(M \sminus A_{S})\) for each \(S \subset \ul{n+1}\).
\end{notation}

Theroem \ref{thm: original} generalises to this setting.

\begin{lemma}\label{lemma: characterising n-sheaves}
   Let \(\iC\) be an \(\infty\)-category with small limits and let \(F \in \Fun(\uMfldN^{\op},\iC)\). The following are equivalent:
    \begin{enumerate}[label={(\alph*)}]
        \item \(F\) is manifold \(n\)-excisive. 
        \item \(F\) is good, and for all \(e \colon M\to N \in \uMfldN\) and pairwise disjoint closed subsets \(A_{0},\cdots,A_{n}\) in the interior of \(M\) that satisfy the assumption of \ref{notation: manifold cube}, \(F\) sends \(X^{e}_{(M,A)}\) to a cartesian \((n+1)\)-cube. i.e. the canonical map
    \[
      F(L_{e}(M)) \to \lim_{\emptyset \neq S \subset \ul{n+1}}F(L_{e}(M \sminus A_{S}))
    \]
    is an equivalence.
    \end{enumerate}
\end{lemma}


\begin{proof}
First we note that although condition (b) is usually stated for arbitrary disjoint closed subsets \(A_{0},\cdots,A_{n}\) of \(M\), inspection of the proofs shows that it suffices to verify the condition for cubes satisfying~\ref{notation: manifold cube}. 
The argument of \cite[Section 3,Section 5]{A24} generalises easily to this setting, using the contractibility of the space of collars is contractible and the fact that \(\uMfldN \to \Mfld_{\partial}\) is a right fibration so the discussion of \cite[Remark 5.7.]{A24} applies. 
\end{proof}

\begin{remark}\label{rem: discrete to infinity}
There is an equivalence of \(\infty\)-categories
\[
    \Fun^{\le n}(\uMfldN^{\op},\iC)
    \;\simeq\;
    \Fun_{\mathrm{istp}}^{\le n}(\cO_{d}(N)^{\op},\iC),
\]
\nolinebreak
where \(\Fun^{\le n}(\uMfldN^{\op},\iC)\) denotes the essential image of \(\iota_{n\ast}\) 
and \(\cO_{d}(N)\) denotes the poset of \(d\)-dimensional submanifolds of \(N\) ordered by inclusion.
\(\Fun_{\mathrm{istp}}^{\le n}(\cO_{d}(N),\iC)\) is the full subcategory of isotopy-invariant functors on \(\cO_{d}(N)\) which are right Kan extended from the subposet \(\cO_{d,\leq n}(N)\) of submanifolds diffeomorphic to \(\R^{d} \times S\) with \(|S| \le n\).
The analogous statements of Lemma~\ref{lemma: characterising n-sheaves} for \(\cO_{d}(N)\) also holds. 
(This is shown in \cite[Lemma~3.13]{A24} in the case \(\dim(M)=\dim(N)\), but the same argument applies verbatim in the present setting.)
\end{remark}



\section{\(T_{n}\)-complements}\label{sec: Tn complement}

In this section we construct a functor
\(
	\varphi_{n} \colon \uMfldN^{\op} \longrightarrow \uPnS
\)
which sends an embedding of a compact manifold into a \(N\) to the \(n\)-th Goodwillie approximation of its complement.  
We then extend this construction to \emph{\(T_{n}\)-embeddings}, i.e.\ objects of \(\Psh(\uDiscnN)\).

Since for any \(\infty\)-category \(\iC\), the Yoneda embedding \(\iC \hookrightarrow \Psh(\iC)\) exhibits \(\Psh(\iC)\) as the free cocompletion of~\(\iC\), a natural approach to defining an extension out of \(\Psh(\uDiscnN)^{\op}\) is to specify it as the unique limit-preserving functor restricting to a prescribed functor on~\(\uDiscnN\).  
Accordingly, we define the \emph{\(T_{n}\)-complement} functor
\(
	\overline{\varphi}_{n} \colon \Psh(\uDiscnN)^{\op} \longrightarrow \uPnS
\)
to be the unique limit-preserving functor which sends an embedding \(e_{U}\colon U \hookrightarrow N\) of at most \(n\) discs into \(N\) to the \(n\)-th Goodwillie approximation of its complement \(\nSigma N \sminus e_{U}(U)\).

By \cite{TW16}, the functor \(\varphi_{n}\) is manifold \(n\)-excisive, and hence determined by its restriction to \(\DiscnN\). 
We therefore obtain a commutative diagram of \(\infty\)-categories:
\begin{equation}\label{eq:recovers genuine emb}
    \begin{tikzcd}[ampersand replacement=\&,cramped]
        {\uMfldN^{\op}} \arrow[r, "\varphi_{n}"] \arrow[d] 
        \& \uPnS \\
        {\Psh(\uDiscnN)^{\op}} \arrow[ur, "\overline{\varphi}_{n}"']
    \end{tikzcd}
\end{equation}
where the vertical map takes an embedding \(M \hookrightarrow N\) to the induced map \(\ust{\iota_{n}}E_{M} \to \ust{\iota}E_{N}\) on presheaves on \(\Discn\).

In particular, this shows that every \(T_{n}\)-embedding admits a well-defined complement in \(\PnS\).  
This stands in contrast to the situation when one considers the full homotopy type of the complement, rather than its \(n\)-nilpotent approximation: the natural map in \(\iS\)
\[
	N \setminus e(M) \longrightarrow \lim_{U \in \Disc(M)} N \setminus e(U)
\]
is an equivalence only when \(\dim N - \dim M \ge 3\).

The layers of the embedding tower and the Goodwillie tower admit accessible descriptions.
This motivates constructing a map between the towers that allows us to compare the layers.
We show that the collection \(\{\overline{\varphi}_{n}\}_{n \ge 1}\) assembles into a lax map of towers---that is, for each \(n\) there exists a natural transformation fitting into a diagram
\[\begin{tikzcd}
	{\Psh(\uDiscnN)^{\op}} & \uPnS \\
	{\Psh({\Disc_{\partial, \leq n-1}}_{/N})^{\op}} & {{\mathscr{P}_{n-1}\mathscr{S}_{\ast}}^{\partial_c /}}
	\arrow["{\overline{\varphi}_n}", from=1-1, to=1-2]
	\arrow[from=1-1, to=2-1]
	\arrow[Rightarrow, from=1-2, to=2-1]
	\arrow[from=1-2, to=2-2]
	\arrow["{\overline{\varphi}_{n-1}}"', from=2-1, to=2-2]
\end{tikzcd}\]

For each \(\eta \in \Psh(\DiscnN)\), the natural transformation evaluated at \(\eta\) is a map of the form
\[
    \nmnSigma\ovph_{n}(\eta) \simeq  \nmnSigma\lim_{e_{U} \in {\DiscnN}_{/\eta}} \nSigma N\sminus e_{U}(U) \to \lim_{e_{V} \in {{{\mathscr{D}\mathrm{isc}}_{\leq n-1}}_{/N}}_{/\eta}} \Sigma^{\infty}_{n-1} N \sminus e_{V}(V) \simeq \ovph_{n-1}\iota_{n,n-1}^{\ast}(\eta).
\]

For an actual embedding \( e \colon M \hookrightarrow N \) of a compact manifold, both the source and target of this map are equivalent to \( \Sigma_{n-1}^{\infty}(N \setminus e(M)) \) by \eqref{eq:recovers genuine emb}, and the natural transformation evaluated at the image of \(e\) is an equivalence.
In particular, this induces a map on layers for the path components lying in the image of the evaluation map \(\Psh(\mathscr{M}\mathrm{fld}_{\partial /N}^{\mathrm{cpt}}) \to \Psh(\DiscnN)\).



\bigskip

We will keep track of a chosen inclusion of a sub–CW–complex in the complement of the boundary throughout the construction, using the following notation.

\begin{notation}[\(\partial_{c}\)]
    Let \(\uMfldN\) be the category defined in \ref{Notation: Mfld partial /N}. 
    Fix a nonempty pointed sub-CW-complex \(\partial_c \subset \partial N \sminus i(P_{d})\).
    We write \(\uPnS\) for the undercategory of \( \mathscr{P}_{n}\mathscr{S}_{\ast} \) under \( \nSigma\partial_{c} \).
\end{notation}



 


\subsection{\(n\)-th approximation of the complement}
In this subsection we define the functor
\[
    \varphi_{n} \colon \uMfldN^{\op} \rightarrow \uPnS
\]
which assigns to an embedding \(e \colon M \hookrightarrow N\) of a manifold admitting a finite handle decomposition, the \(n\)-th Goodwillie approximation \(\nSigma \partial_{c} \rightarrow \nSigma N \sminus e(M)\),
and show that \(\varphi_n\) is manifold \(n\)-excisive. 

\medskip
We do so by constructing a functor \(\varphi_{n}'\) from the poset \(\cO_{d}(N)\) of \(d\)-manifolds in \(N\)  to \(\uPnS\).
We then verify that \(\varphi_{n}'\) is isotopy invariant, good, and manifold \(n\)-excisive.  
By Remark~\ref{rem: discrete to infinity}, this promotes to a manifold \(n\)-excisive functor \(\varphi_{n} \colon \uMfldN \to \uPnS\).


\begin{notation}
Let \(\cO_{d}^{\mathrm{cpt}}(N)\) denote the subposet of \(\cO_{d}(N)\) spanned by \(d\)-dimensional submanifolds admitting a finite handle decomposition. 
Note that the subposet \(\cO_{d, n}(N)\) consisting of submanifolds diffeomorphic to disjoint unions of at most \(n\) copies of \(\R^{d}\) is contained in \(\cO_{d}^{\mathrm{cpt}}(N)\).
\end{notation}


Define a functor from \(\cO_{d}^{\mathrm{cpt}}(N)\) to \(\Top_{*}^{\partial_{c}/}\) by assinging to each \(M \in \cO_{d}^{\mathrm{cpt}}(N)\)
\[
   i_{M} \colon \partial_{c} \hookrightarrow \partial N \sminus i(\partial M) \hookrightarrow N \sminus M.
\]
Applying the coherent nerve yields \(\cO_{d}^{\mathrm{cpt}}(N) \longrightarrow \uS\).
We now define \(\varphi_{n}'\) to be the right Kan extension of the composite \(\cO_{d}^{\mathrm{cpt}}(N) \rightarrow \uS \xrightarrow{\nSigma} \uPnS \) along the inclusion \(\cO_{d}^{\mathrm{cpt}}(N) \hookrightarrow \cO_{d}(N)\).

The argument of \cite[Lemma~1.3.1]{TW16} shows that \(\varphi_{n}'\) is good and carries isotopy equivalences to equivalences. 
It is also observed in \cite[Remark~1.3.2]{TW16} that \(\varphi_{n}'\) is manifold \(n\)-excisive; 
We include the proof here for reader's convenience.


\begin{lemma}\cite[Remark~1.3.2]{TW16}\label{lemma : P_n of complement is n sheaf}
    \(\varphi_{n}'\) is manifold \(n\)-excisive.
\end{lemma}

\begin{proof}
    By lemma~\ref{lemma: characterising n-sheaves} and the remark following it, it is enough to check that \(\varphi_{n}'\) sends each \(X_{(M,\ul{A})}\) satisfying the conditions of \ref{notation: manifold cube} to a cartesian cube.
    By assumtion, \(M \sminus A_{S} \in \cO_{d}^{\mathrm{cpt}}(N)\) and
    \[
      \varphi_{n}'(M) \to \lim_{\emptyset \neq S \subset \ul{n+1}} \varphi_{n}'(M \sminus A_{S})
    \]
    is equivalent to
    \[
       (\nSigma\partial_{c} \to \nSigma N \sminus M) \to \lim_{\emptyset \neq S \subset \ul{n+1}}(\nSigma\partial_{c} \to \nSigma N \sminus (M \sminus A_{S}))
    \]
    Since the forgetful functor \(\uPnS \to \PnS\) reflects limits, it's enough to show that
    \[
       \nSigma N \sminus M \to \lim_{\emptyset \neq S \subset \ul{n+1}}\nSigma N \sminus (M \sminus A_{S})
    \]
    is an equivalence. 
    This cube is the image under \(\nSigma\) of a \((n+1)\)-cube in \(\pS\) whose two-dimensional faces are of the form
    \[
        \begin{tikzcd}
            N \backslash (M\backslash A_{S\cap T}) \arrow[d] \arrow[r] & N \backslash (M\backslash A_{T}) \arrow[d] \\
            N \backslash(M \backslash A_{S}) \arrow[r]                 & N \backslash(M\backslash A_{S\cup T})     
        \end{tikzcd}
    \] for subsets \(S,T \subset \ul{n+1}\). Since \(N \backslash(M\backslash A_{S\cup T}) = (N \backslash(M\backslash A_{S})) \cup (N \backslash(M\backslash A_{T}))\) and
    \(N \backslash(M\backslash A_{S\cap T}) = (N \backslash(M\backslash A_{S})) \cap (N \backslash(M\backslash A_{T}))\), each such square is cocartesian in \(\pS\).
    Now \(\nSigma\) is a left adjoint and every strongly cocartesian \((n+1)\)-cube is cartesian in \(\PnS\), and the claim follows.
\end{proof}

Let \(\varphi_{n} \colon \Mfld_{\partial/N}^{\op} \to \uPnS\) denote the manifold \(n\)-excisive functor corresponding to \(\varphi_{n}'\) under the equivalence of Remark~\ref{rem: discrete to infinity}.  
Note that if \(e_{M} \colon M \hookrightarrow N\) is an object of \(\Mfld_{\partial/N}^{\mathrm{cpt}}\), then
\[
    \varphi_{n}(e_{M}) \simeq 
    \bigl(\,\nSigma\partial_{c} \longrightarrow \nSigma (N \setminus e_{M}(M))\,\bigr)
    \in \uPnS.
\]

\begin{remark}
A more categorically natural construction should be possible at the level of the bordism double category \(\Bord\) of \cite{KK22}, using the complement presheaves developed in \cite[Section~4]{KK25}.  
\end{remark}




\subsection{\(T_{n}\)-complement}


In this subsection we define the \(T_{n}\)-complement functor
\[
	\overline{\varphi}_{n} \colon \Psh(\uDiscnN)^{\op} \longrightarrow \uPnS
\]
and show that these functors assemble into a lax morphism of towers.  
Intuitively, \(\overline{\varphi}_{n}\) encodes the \(n\)-th Goodwillie approximation of complements not only for genuine embeddings but also for their \(T_{n}\)-analogues represented by a map of presheaves.


To make this precise, recall that for an \(\infty\)-category \(\icat{C}\), the Yoneda embedding
\[
	y_{\icat{C}} \colon \icat{C} \hookrightarrow \Psh(\icat{C})
\]
exhibits \(\Psh(\icat{C})\) as the free cocompletion of~\(\icat{C}\).  
Thus, if \(\icat{E}\) admits small limits, precomposition with \(y_{\icat{C}}^{\op}\) induces an equivalence
\[
    \Fun^{R}(\Psh(\icat{C})^{\op},\icat{E}) \xlongrightarrow[\simeq]{\ust{y_{\iC}^{\op}}} \Fun(\icat{C}^{\op},\icat{E}),
\]
where \(\Fun^{R}(-,-)\) denotes the full subcategory of \(\Fun(-,-)\) spanned by functors that preserve small limits. 
This universal property allows us to extend the Goodwillie \(n\)-th approximation of complements \(\varphi_{n}\) to arbitrary presheaves on \(\uDiscnN\);

\begin{definition}[\(T_{n}\)-complement]
    Let \(\overline{\varphi}_{n}\) be the essentially unique object of \(\Fun^{R}(\Psh(\uDiscnN)^{\op},\uPnS)\) such that 
    \(\overline{\varphi}_{n}y_{\Disc} \simeq \varphi_{n}\iota_{n} \in \Fun(\uDiscnN^{\op},\uPnS)\).
	We refer to \(\overline{\varphi}_{n}\) as the \emph{\(T_{n}\)-complement functor}.
\end{definition}


The following shows that the \(T_{n}\)-complement functor recovers the \(n\)-nilpotent approximation of the complement of genuine embeddings. 

\begin{thm}\label{Prop: commutativity with emb}
	There is a commutative diagram of \(\infty\)-categories:
	\[
	\begin{tikzcd}
		{\uMfldN^{\op}} \arrow[r, "\varphi_{n}"] \arrow[d, "y_{\Mfld}^{\op}"'] 
			& \uPnS \\
		{\Psh(\uMfldN)^{\op}} 
			\arrow[d, "(\iota_{n}^{\ast})^{\op}"'] 
			& \\
		{\Psh(\uDiscnN)^{\op}} 
			\arrow[uur, "\overline{\varphi}_{n}"']
	\end{tikzcd}
	\]
\end{thm}

\begin{proof}
    Consider the sequence of natural transformations
    \begin{equation*}
        \begin{split}        
            \varphi_{n} &\rightarrow \lst{\iota_{n}}\iota_{n}^{*}\varphi_{n} \\
                & \simeq \lst{\iota_{n}} y_{\Disc_{  n}}^{*}\ovph_{n} \\
                & \simeq y_{\Mfld}^{*}(\iota_{n}^{*})^{*}\ovph_{n}
        \end{split}
    \end{equation*}
    The first map is an equivalence by Lemma \ref{lemma : P_n of complement is n sheaf}. 
    The second equivalence is the definition of \(\ovph_{n}\). 
    The final equivalence arises from the compatibility between the Yoneda embedding and full subcategory inclusions.  
    For a full subcategory inclusion \(\iota \colon \iC_{0} \subset \iC\) and an \(\infty\)-category \(\icat{E}\) admitting small limits, the following diagram commutes~\cite[Lemma~5.2.6.7]{L09}:
    \[\begin{tikzcd}
	{\Fun^{R}(\Psh(\icat{C})^{\op},\icat{E})} & {\Fun^{R}(\Psh(\icat{C}_0)^{\op},\icat{E})} \\
	{\Fun(\icat{C}^{\op},\icat{E}) } & {\Fun(\icat{C}_0^{\op},\icat{E}) }
	\arrow["{y_{\iC}^*}"', from=1-1, to=2-1]
	\arrow["{(\iota^*)^*}"', from=1-2, to=1-1]
	\arrow["{y_{\iC_0}^*}", from=1-2, to=2-2]
	\arrow["{\iota_{*}}", from=2-2, to=2-1]
\end{tikzcd}\]
\end{proof}


The layers of the embedding tower and the Goodwillie tower admit accessible descriptions.
This motivates constructing a map between the towers that allows us to compare the layers.
We show that the collection \(\{\overline{\varphi}_{n}\}_{n \ge 1}\) assembles into a lax map of towers, which restricts to a map of towers on the image of genuine embeddings of compact manifolds.  
In particular, this induces a map on layers for the path components lying in the image of the evaluation map \(\Psh(\uMfldN) \to \Psh(\DiscnN)\).

Recall that the adjunction \(\nmnSigma  \colon\PnS \rightleftarrows {\mathscr{P}_{n-1}\pS} \colon \Omega^{\infty}_{n,n-1}\) induces a sliced adjunction,
which we continue to denote by \(\nmnSigma  \colon\uPnS \rightleftarrows {\mathscr{P}_{n-1}\pS}^{\partial_{c}/}\colon \Omega^{\infty}_{n,n-1}\). 

\begin{proposition}\label{prop: lax map of towers}
    There is a natural transformation of functors fitting into the following diagram:
\[\begin{tikzcd}
	{\Psh(\uDiscnN)^{\op}} & \uPnS \\
	{\Psh({\Disc_{\partial,\leq n-1}}_{/N})^{\op}} & {{\mathscr{P}_{n-1}\mathscr{S}_{\ast}}^{\partial_c /}}
	\arrow["{\overline{\varphi}_n}", from=1-1, to=1-2]
	\arrow["{(\inmn^*)^{\op}}"', from=1-1, to=2-1]
	\arrow[Rightarrow, from=1-2, to=2-1]
	\arrow["\nmnSigma", from=1-2, to=2-2]
	\arrow["{\overline{\varphi}_{n-1}}"', from=2-1, to=2-2]
\end{tikzcd}\]
    This natural transformation becomes an equivalence after precomposition with \((\iota_{n}^{*})^{\op}y_{\Mfld}^{\op}(\icpt)^{\op}\).
\end{proposition}



    

\begin{proof}
For this proof, we work in the opposite categories and adopt the following notations:
\begin{itemize}
        \item \(\iD_{0} \coloneqq \Mfld_{\mathrm{cpt}}\) and \(\iD \coloneqq \Mfld\). Denote the inclusion \(\iD_{0} \hookrightarrow \iD\) by \(k\);
		\item \(\iE_{n} \coloneqq (\uPnS)^{\op}\);
		\item \(G \coloneqq (\nmnSigma)^{\op}\);
		\item \(\iC_{n} \coloneqq \uDiscnN\) and \(\iC_{n-1} \coloneqq {\Disc_{\partial,\le n-1}}_{/N}\);
		\item \(\phi_{n} \coloneqq \varphi_{n}^{\op}\) and \(\overline{\phi}_{n} \coloneqq \overline{\varphi}_{n}^{\op}\);
		\item \(\ops_{n}\) denotes the right adjoint of \(\overline{\phi}_{n}\).
	\end{itemize}
Note that \(\overline{\phi}_{n} \colon \Psh(\iC_{n})\to\iE_{n}\) is the unique colimit preserving functor that restricts to \(\iota_{n}^{\ast}\phi_{n} \colon \iC_{n} \to \iE_{n}\).
The right adjoint \(\ops_{n}\) exists and is given by \(i_{n}^{\ast}\phi_{n}^{\ast}y_{\iE_{n}}\)~\cite[5.2.6.5]{L09}. 

\subsubsection*{Construction of the natural transformation.}
We construct the desired natural transformation as the Beck-Chevalley mate of a natrual tranformation \(A \colon \iota_{n,n-1}^{\ast}\ops_{n} \to \ops_{n-1}G\).
\[\begin{tikzcd}
	{\Psh(\iC_n)} & {\iE_n} \\
	{\Psh(\iC_{n-1})} & {\iE_{n-1}}
	\arrow["{\inmn^*}"', from=1-1, to=2-1]
	\arrow["A"{description}, Rightarrow, from=1-1, to=2-2]
	\arrow["{\ops_{n}}"', from=1-2, to=1-1]
	\arrow["G", from=1-2, to=2-2]
	\arrow["{\ops_{n-1}}", from=2-2, to=2-1]
\end{tikzcd}\]
More precisely, our desired natural transformation is the composite
\[
    \oph_{n-1}\inmn^{*} \to \oph_{n-1}\inmn^{*}\ops_{n}\oph_{n} \to \oph_{n-1}\ops_{n-1}G\oph_{n} \to G\oph_{n}
\]
where the first arrow is induced by the unit of adjunction, the second arrow by \(A\), and the last arrow by the counit. 

\(A\) is defined as the composite
\begin{align*}
    \inmn^{*}\ops_{n} &\simeq \inmn^{*}i_{n}^{*}\phi_{n}^{*}y_{\iE_{n}}\\
                    &\rightarrow \inmn^{*}i_{n}^{*}\phi_{n}^{*}{G}^{*}y_{\iE_{n-1}}G \\
                    &\simeq i_{n-1}^{*}\phi_{n-1}^{*}y_{\iE_{n-1}}G \\
                    &\simeq \ops_{n-1}G.
\end{align*}

The second arrow is induced by the natural tranformation \(y_{\iE_{n}} \to {G}^{*}y_{\iE_{n-1}}G\) corresponding to the canonical map \(\Map_{\iE_{n}}(-,-) \to \Map_{\iE_{n-1}}(G(-),G(-))\).
The third line follows from \(\iota_{n}\inmn \simeq \iota_{n-1}\) and \(G \phi_{n}\iota_{n-1} \simeq \phi_{n-1}\iota_{n-1}\).


\subsubsection*{The natural transformation on representables.}
To prove that the defined natural transformation becomes an equivalence when precomposed with \(i_{n}^{*}y_{\iD}k\), first note that the composite  
\[\oph_{n-1}\inmn^{*}i_{n}^{*}y_{\iD}k \to \oph_{n-1}\inmn^{*}\ops_{n}\oph_{n}i_{n}^{*}y_{\iD} k \to \oph_{n-1}\ops_{n-1}G\oph_{n}i_{n}^{*}y_{\iD}k \to G\oph_{n}i_{n}^{*}y_{\iD}k \]
is equivalant to 
\[
    \oph_{n-1}\iota_{n-1}^{*}y_{\iD}k \xrightarrow{u} \oph_{n-1}\iota_{n-1}^{*}\phi_{n-1}^{*}y_{\iE_{n-1}}\phi_{n-1}k \xrightarrow{v} \phi_{n-1}k
\]
where the first map is induced by natural tranformation \(u \colon y_{\iD} \to \phi_{n-1}^{*}y_{\iE_{n-1}}\phi_{n-1}\), and the second map is induced from the counit map \(v \colon \oph_{n-1}\ops_{n-1} \to \id\).

We omit the subscript \((n-1)\) from now on and denote \(\iC = \Disc_{n-1}\), the full subcategory inclusions by
\(\iC \xhookrightarrow{\iota_{0}} \iD_{0} \xhookrightarrow{k} \iD\) and \(\iC \xhookrightarrow{\iota} \iD\).
Note that \(\iota \simeq k \iota_{0}\).

We have the following commutative diagram:

\[\begin{tikzcd}[ampersand replacement=\&,column sep=tiny]
	{\Map_{\Fun(\iD_{0},\iE)}({\iota_{0}}_{!}\oph y_{\iC},X)} \& {\Map_{\Fun(\iC,\Psh(\iC))}(y_{\iC},\ops \ust{\iota_{0}} X)} \\
	{\Map_{\Fun(\iD_{0},\iE)}(\oph\iota^* y_{\iD}k,X)} \& {\Map_{\Fun(\iD_{0},\Psh(\iC))}(\iota^* y_{\iD}k,\ops X)} \\
	{\Map_{\Fun(\iD_{0},\iE)}(\oph\iota^* \phi^* y_{\iE} \phi k,X)} \& {\Map_{\Fun(\iD_{0},\Psh(\iC))}(\iota^*\phi^*y_{\iE}\phi\icpt,\ops X)} \\
	{\Map_{\Fun(\iD_{0},\iE)}(\phi k,X)} \& {\Map_{\Fun(\iD_{0},\iE)}(\phi k,X)}
	\arrow["\simeq"', from=1-2, to=1-1]
	\arrow["\simeq", from=2-1, to=1-1]
	\arrow["\alpha"', from=2-2, to=1-2]
	\arrow["\simeq"', from=2-2, to=2-1]
	\arrow["{u^*}", from=3-1, to=2-1]
	\arrow["\beta"', from=3-2, to=2-2]
	\arrow["\simeq"', from=3-2, to=3-1]
	\arrow["{v^*}", from=4-1, to=3-1]
	\arrow["\gamma"', from=4-2, to=3-2]
	\arrow[equals, from=4-2, to=4-1]
\end{tikzcd}\]

All the horisontal arrows are given by adjunctions. 
The upper left vertical arrow is an equivalence given by ~\cite[Lemma~5.2.6.7]{L09}.
Here \(\gamma\) is induced by \(\Map_{\iE}(-,-) \to \Map_{\Psh(\iC)}(\ops(-),\ops(-))\)
and \(\beta\) denotes the map induced by \(u\). \(\alpha\) is the restriction to the subcategory \(\iota_{0} \colon \iC \hookrightarrow \iD_{0}\).


It therefore suffics to show that the composite of the right vertical arrows is an equivalence. 
This follows from the commutativity of the following diagram:
\[\begin{tikzcd}[ampersand replacement=\&]
	{\Map_{\Fun(\iC,\Psh(\iC))}(y_{\iC},\ops{\iota_{0}}^*X)} \& {\Map_{\Fun(\iC,\iE)}(\phi\iota,X \iota_{0})} \\
	{\Map_{\Fun(\iD_{0},\Psh(\iC))}(\iota^*y_{\iD} k,\ops X)} \\
	{\Map_{\Fun(\iD_{0},\Psh(\iC))}(\iota^*\phi^*y_{\iE}\phi k,\ops X)} \\
	{\Map_{\Fun(\iD_{0},\iE)}(\phi k,X)} \& {\Map_{\Fun(\iD_{0},\iE)}(\phi k,X)}
	\arrow["\simeq"',from=1-1, to=1-2]
	\arrow["\alpha", from=2-1, to=1-1]
	\arrow["\beta", from=3-1, to=2-1]
	\arrow["\gamma", from=4-1, to=3-1]
	\arrow["\delta"', from=4-2, to=1-2]
	\arrow["\simeq"{description}, shift left=3, draw=none, from=4-2, to=1-2]
	\arrow[equals, from=4-2, to=4-1]
\end{tikzcd}\]
Here \(\delta\) is the map given by restricton to \(\iC\). The diagram commutes since \(\gamma\) is the map
\[
    \Map_{\Psh(\iC)}(\ops(-),\ops(-)) \simeq \Map_{\Psh(\iC)}(\Map_{\iE}(\phi\iota(\bullet),(-)),\Map_{\iE}(\phi\iota(\bullet),(-)))
\]
induced by postcomposition and the upper horisontal arrow is the evaluation at the identity morphism which is an equivalence by Yoneda lemma.
Finally, \(\delta\) is an equivalence since \(\phi k \simeq \lsh{\iota_{0}}\ust{\iota_{0}}\phi k\).


\end{proof}


\section{\(T_{n}\)-Stallings' theorem for some \(T_{n}\)-complements in \(D^{d}\)}

In \ref{sec: Stallings}, we prove an analogue of Stallings' theorem for \(T_{n}\)-complements of \(T_{n}\)-embeddings of \(P \times I\) into \(D^{d}\).
More precisely, given a \(T_{n}\)-embedding, we show that the boundary inclusion induced map from \(\nSigma D^{d-1}\sminus P\) to the \(T_{n}\)-complement is an equivalence in \(\PnS\). 

Two observations make the argument work for embeddings of \(P \times I\) into \(D^{d-1} \times I \cong D^{d}\).

First, in this case, there exists a \(T_{n}\)-embedding such that boundary inclusion induced map is an equivalence in \(\PnS\):
This is given by the image of \(i\times I\) in \(T_{n}\bdemb(P \times I,D^{d})\) since
\(
    \nSigma D^{d-1} \sminus i(P) \rightarrow \nSigma D^{d} \sminus (i\times I)(P \times I) 
\)
is an equivalence, and by Proposition~\ref{Prop: commutativity with emb}, the right-hand side identifies with the \(T_{n}\)-complement of the corresponding \(T_{n}\)-embedding.
In the language of \cite{KK24}, this follows because \(E_{i \times I}\) is the identity 2-morphism from \(E_{P\times I}\) to \(E_{D^{d-1}\times I}\) in the Morita category of algebras and bimodules of \(\Psh(\Discn)\). 
For our purposes, it is enough that this bimodule map becomes an equivalence upon passing to \(\PnS\). 


Second, for any embedding \(e\colon M \hookrightarrow N\),
when the ambient manifold \(N\) is contractible, the \(n\)-th approximation of the complement \(\nSigma N \sminus e(M)\) can be recovered from the data of the tubular neighborhood of \(e(M)\).
Recall that \(\PnS\) is defined as the colimit
\[
    \colim(\pS \to \mathscr{T}_{n}\pS \to \mathscr{T}_{n}(\mathscr{T}_{n}\pS) \to \cdots )
\]
where each functor \(\mathscr{T}_{n}\) sends an object \(X\) to the punctured \((n+1)\)-cube determined by \(\Sigma X\) and inclusions of wedge summands.
If \(N\) is contractible, \(\Sigma (N\sminus M) \) is equivalent to the Thom space of the normal bundle of \(e\). 
Thus \(\nSigma N \sminus M\) is equivalent to an object determined by the Thom space together with the inclusion maps of wedges.
Using this, the problem reduces to the case where one only needs to exhibit a \(T_{n}\)-embedding for which the boundary inclusion map is an equivalence.

In the case of string links, we identify the homotopy class of the induced equivalence as the Artin representation for string links in \ref{sec: string links}. 
In particualr, although the \(T_{n}\)-complment is equivalent to something that doesn't depend on the embedding, the induced equivalence can carry nontrivial and interesting information.


\subsection{Stallings' theorem for \(T_{n}\)-complemnts}\label{sec: Stallings}

To state the theorem cleanly, we introduce the following defintion. 
    

Let \(\iC\) be an \(\infty\)-category and let \(X,Y \in \iC\). We denote by \(\iC^{(X,Y)/}\) the pullback \(\iC^{X/} \times_{\iC} \iC^{Y/}\).
If \(\iC\) admits coproducts, then the left fibration \(\iC^{(X,Y)/} \to \iC\) straightens to the functor
\[
    \Map_{\iC}(X,-) \times \Map_{\iC}(Y,-) \simeq \Map_{\iC}(X \amalg Y,-),
\] hence there is an equivalence \(\iC^{(X,Y)/} \simeq \iC^{X \amalg Y /}\).
We refer to the full subcategory
\[
    (\iC^{\simeq})^{(X,Y)/} \subset \iC^{(X,Y)/} \simeq \iC^{X \amalg Y/}
\]
as the \emph{category of equivalence cospans} from \(X\) to \(Y\).


Note that there is a natural map from \((\iC^{\simeq})^{(X,X)}\) to the automorphism space of \(X\).
Indeed, the composite natural transformation
\[
    \Map_{\iC^{\simeq}}(X,-) \times \Map_{\iC^{\simeq}}(X,-) \simeq \Map_{\iC^{\simeq}}(X,-) \times \Map_{\iC^{\simeq}}(-,X) \to \Map_{\iC^{\simeq}}(X,X)
\]
where the first equivalence is given by \(\iC^{\simeq} \simeq (\iC^{\simeq})^{\op}\) and the second map is given by composition of morphisms in \(\iC^{\simeq}\), unstraightens to
\[
    (\iC^{\simeq})^{(X,X)/} \to \Aut_{\iC}(X) \times \iC^{\simeq} \to \Aut_{\iC}(X).
\]

On objects, the composite sends \(X \xrightarrow{f} \bullet \xleftarrow{g} X\) to \(g^{-1}f\).


Now we state our theorem.


\begin{notation}
For an embedding \(e \in \bdemb(-,D^{d})\), write \(D_{(-)} \coloneq D^{d} \sminus e(-)\).
Fix a compact manifold \(P\), an embedding into the interior \(i \colon P \hookrightarrow \mathrm{int}(D^{d-1})\), and a point \(x\) in \(\partial D^{d-1}\).
All embeddings restrict to \(i\) on \(P \times \{0\}\) and \(P \times \{1\}\).
Let \(\partial_{c} \subset \partial D^{d} \sminus \partial (P \times I)\) be the pointed sub CW-complex \((D_{P} \times \{0\}) \cup (x \times I) \cup  (D_{P} \times \set{1}) \) pointed at \(x \times \set{0}\).
Thus \(\partial_{c} \simeq D_{P} \vee D_{P}\).
\end{notation}

\begin{thm}\label{prop: Bordism equivalence}
    For \((P,i,x)\) defined as above,
    \[
        T_{n}\bdemb(P \times I,D^{d}) \simeq T_{n}\bdemb(P \times I,D^{d})^{\op} \to \Psh({\uDiscn}_{/D^{d}})^{\op} \xrightarrow{\ovph_n} {\PnS}^{\nSigma D_{P} \vee \nSigma D_{P}/}
    \]
    lands in the full subcategory of equivalence cospans \(({\PnS^{\simeq}})^{(\nSigma D_{P}, \nSigma D_{P})}\) 
\end{thm}

\begin{corollary}\label{cor:Tn to Aut}
    For \((P,i,x)\) defined as above, there is a map of spaces
    \[
        T_{n}\bdemb(P \times I,D^{d}) \xrightarrow{\ovph_{n}} ({\PnS^{\simeq}})^{(\nSigma D_{P}, \nSigma D_{P})} \to \Aut_{\PnS}(\nSigma D_{P}) 
    \]
\end{corollary}

For the proof of Theorem \ref{prop: Bordism equivalence} we will use the following construction.
\begin{definition}(\(n\)-pointed cone)
    Let \(\iC\) be an \(\infty\)-category which admits finite colimits and has a final object, and let \(X \in \iC\). 
    The \(n\)-\textit{pointed cone on \(X\)} is a strongly cocartesian \(n\)-cube \(C_{n}(X)\) in \(\iC\) defined as the left Kan extension of its restriction to \(\mathscr{P}_{\leq 1}(\ul{n})\), where for \(S \in {\mathscr{P}_{\leq 1}(\ul{n})}\)
    \[
        C_{n}(X)(S) = 
        \begin{cases}
            X & \text{if } S = \emptyset, \\[4pt]
            \ast & \text{if } |S| = 1.
        \end{cases}
    \]
\end{definition}

\begin{example}
    When \(\iC = \pS\) and \(X \in \pS\), for each \(S \subset \ul{n}\) we have
    \[
        C_{n}(X)(S) =
        \begin{cases}
            X &\text{if } S = \emptyset, \\[4pt]
            \bigvee_{|S|-1} \Sigma X &\text{otherwise}.
        \end{cases}
    \]
\end{example}

\begin{example}\label{example: n+1 pointed cone undercat.}
    Let \(t \in \pS\) and consider \(\iC = \pS^{t/}\). 
    For \(i_{X} \colon t \to X \in \pS^{t/}\) and \(S \subset \ul{n}\),
    \[
        C_{n+1}(i_{X})(S) = \begin{cases} 
            i_{X} &\mbox{if } S = \emptyset \\
        t \to \wedg_{|S|-1} \Sigma t \to \wedg_{|S|-1} \Sigma X & \mbox{otherwise }  
        \end{cases}
    \] 
    Here we use the fact that pushouts in \(\pS^{t/}\) are computed by taking the corresponding pushouts in \(\pS\).


    Applying the composite
    \(
        \pS^{t/} \xrightarrow{\nSigma} \PnS^{\nSigma t /} \longrightarrow \PnS
    \)
    yields a strongly cocartesian \((n+1)\)-cube.
    Since \(\id_{\PnS}\) is \(n\)-excisive and the forgetful functor \(\uPnS \to \PnS\) creates limits, the cube \(\nSigma C_{n+1}(i_{X})\) is cartesian in \(\uPnS\).
    In particular,
    \begin{align*}
        (\nSigma t \to \nSigma X) &\simeq \lim_{ \emptyset \neq S \subset \ul{n+1}}(\nSigma t \to \nSigma \wedg_{|S|-1} \Sigma X) \\
                & \simeq \nSigma t \xrightarrow{\simeq}  \lim_{ \emptyset \neq S \subset \ul{n+1}}\nSigma \wedg_{|S|-1} \Sigma t \to \lim_{ \emptyset \neq S \subset \ul{n+1}}\nSigma \wedg_{|S|-1} \Sigma X
    \end{align*}

\end{example}


\begin{proof}[Proof of Theorem \ref{prop: Bordism equivalence}]
It is enough to check the claim on objects.
    
We first express the value of the functor on a general element as a limit.
Let \(\mathscr{U} \subset \cO_{n}(P \times I)\) be a contractible Weiss \(n\)-cover of \(P \times I\) (for instance, \(\mathscr{U} = \cO_{n}(P \times I)\)). 
Let \( \eta \colon { \iota_n }^{\ast} E_{P \times I} \rightarrow {\iota_n}^{\ast} E_{D^{d}}\) be an element of \(T_{n}\bdemb(P \times I, D^{d})\). 
We have \( {\iota_n}^{\ast}E_{P \times I} \simeq \colim_{U \in \mathscr{U}} E_{U}\) in \(\Psh(\uDiscn)\) (\cite[Lemma 5.9]{KK24}).
Since colimits in over categories are computed by the forgetful map, \(\eta \simeq \colim_{U \in \mathscr{U}} \eta_{U}\) where \(\eta_{U}\) is the composite \(E_{U} \to {\iota_n}^{\ast}E_{P\times I} \xrightarrow{\eta} {\iota_n}^{\ast}E_{D^{d}}\). 
By the Yoneda lemma, \(\eta_{U} \simeq y(U \xrightarrow{e_{U}} D^{d})\) and \(\eta \simeq \lim_{U \in \mathscr{U}} y(e_{U})\) in \(\Psh({\uDiscn}_{/D^{d}})^{\op}\). 
Applying \(\ovph_{n}\),
\begin{align*}
    \ovph_{n}(\eta)    &\simeq \lim_{U \in \mathscr{U}} \ovph_{n}(y(U \xrightarrow{e_{U}} D^n)) \\
                &\simeq \lim_{U \in \mathscr{U}} \varphi_{n}\iota_{n}(U \xrightarrow{e_{U}} D^n) \\
                &\simeq \lim_{U \in \mathscr{U}} (\nSigma \partial_{c} \to \nSigma D_{e_{U}(U)}) \\
                &\simeq (\nSigma \partial_{c} \to \lim_{U \in \mathscr{U}} \nSigma D_{e_{U}(U)}) \\
                &\simeq (\nSigma D_{P} \to \lim_{U \in \mathscr{U}} \nSigma D_{e_{U}(U)} \leftarrow \nSigma D_{P}) \\
\end{align*}

The first equivalence holds since \(\ovph_{n}\) preserves limits. The second and third equivalences are definitions of \(\ovph_{n}\) and \(\varphi_{n}\). The fourth equivalence holds since limit of an under category is determined by the forgetful map. The last equivalence holds since \(\nSigma\partial_{c} \simeq \nSigma D_{P} \vee \nSigma D_{P}\).

Thus it remains to show that the canonical map
 \[
    j(\eta) \colon (\nSigma D_{P} \to \lim_{U \in \mathscr{U}} \nSigma D_{e_{U}(U)}) \simeq \lim_{U \in \mathscr{U}} (\nSigma D_{P} \xrightarrow{\nSigma i_{e_{U}}} \nSigma D_{e_{U}(U)})
\]
is an equivalence.

Discussion in Example \ref{example: n+1 pointed cone undercat.} shows that we further have equivalences

\begin{equation*}
    \begin{split}
    j(\eta) & \simeq \lim_{U \in \mathscr{U}} \lim_{ \emptyset \neq S \subset \ul{n+1}} (\nSigma D_{P} \to  \nSigma \wedg_{|S|-1}  \Sigma D_{e_{U}(U)})\\
            & \simeq \lim_{U \in \mathscr{U}}  (\nSigma D_{P} \xrightarrow{\simeq} \lim_{ \emptyset \neq S \subset \ul{n+1}}\nSigma \wedg_{|S|-1} \Sigma D_{P} \xrightarrow{f} \lim_{ \emptyset \neq S \subset \ul{n+1}} \nSigma \wedg_{|S|-1} \Sigma D_{e_{U}(U)})\\
    \end{split}
\end{equation*}
Where \(f\) is \(\nSigma \wedg_{|S|-1} \Sigma i_{e_{U}}\).
The map \(\Sigma D_{P} \xrightarrow{\Sigma i_{e_{U}}} \Sigma D_{e_{U}(U)}\) is equivalent to the map induced on the homotopy cofibers of the inclusions \(D_{P} \to D^{d-1}\) and \(D_{e_{U}(U)} \to D^{d-1} \times I\). Hence the map is equivalent to 
\begin{align*}
    & \mathrm{Th}(\nu_{i(P)}) \to \mathrm{Th}(\nu_{e_{U}(U)}) \\
    \simeq & \mathrm{Th}(\nu_{i(P)}) \to  \mathrm{Th}(\nu_{i(P)}) \vee (\wedg_{|U|} S^{d})
\end{align*}
the inclusion of a wedge summand of the Thom space of normal bundle of \(i(P)\). 
Hence, \(j(\eta)\) is equivalent to an object of \(\PnS^{\nSigma D_{P}/}\) which does not depend on the choice of \(\{e_{U}\}_{U \in \mathscr{U}}\).

In particular, we have the following commutative diagram in \(\PnS\)
\[\begin{tikzcd}[ampersand replacement=\&,cramped]
	\& {\nSigma D_{P}} \\
	{\lim_{U \in \mathscr{U}} \nSigma D_{e_{U}(U)}} \& {\lim_{ \emptyset \neq S \subset \ul{n+1}} \nSigma \wedg_{|S|-1} \Th(\nu _{i(P)}) \vee (\vee_{|U|}S^{d}) } \& {\lim_{U \in \mathscr{U}} \nSigma D_{(i\times I)(U)}} \& {\nSigma D_{P}\times I}
	\arrow["{j(\eta)}"{description}, from=1-2, to=2-1]
	\arrow[from=1-2, to=2-2]
	\arrow["{j(\iota_{n}^{\ast}y(i \times I))}"{description}, from=1-2, to=2-3]
	\arrow["{\nSigma(D_{P} \times \set{0} \to D_P \times I)}", curve={height=-6pt}, from=1-2, to=2-4]
	\arrow["\simeq"', from=2-1, to=2-2]
	\arrow["\simeq", from=2-3, to=2-2]
	\arrow["\simeq", from=2-4, to=2-3]
\end{tikzcd}\]
where the right most horisontal arrow is an equivalence by \ref{Prop: commutativity with emb} and the rightmost vertical map is an equivalence.

\end{proof}





\subsection{The case of string links}\label{sec: string links}
Now we prove Corollary~\ref{corollary: Artin}.

\begin{notation}
    Let \(P = \ul{k}\), and fix an embedding \(i \colon \ul{k} \to \mathrm{int}(D^{2})\). We write \(\ul{k} \times I\) as \(kI\).
For a group \(G\), let \(G_{n}\coloneq [G,G_{n-1}]\) with \(G_{0}=G\) denote its lower central series.
In this section, for \(X \in \pS\) we write \(P_{n}(X)\) for the value \(P_{n}(\id_{\pS})(X)\). 
Recall that \(\pi_{1}P_{n}(\kcirc) \cong F(k)/F(k)_{n+1}\).
\end{notation}


\begin{lemma} The group homomorphism 
    \[
        \pi_{0}\Aut_{\PnS}(\nSigma\kcirc) \xrightarrow{\pi_{1}\nOmega} \Aut_{\Gp}(\pi_{1}P_{n}(\kcirc))
    \] 
    is an isomorphism.  
\end{lemma}
\begin{proof}
    For injectivity, consider the inclusion
    \[
        \Aut_{\PnS}(\nSigma \kcirc) \subset \Map_{\PnS}(\nSigma \kcirc, \nSigma \kcirc) \simeq \Map_{\pS}(\kcirc,P_{n}(\kcirc)) \simeq (\Omega P_{n}(\kcirc))^{\times k}.
    \]
    This induces an injection \(\pi_{0}\Aut_{\PnS}(\nSigma \kcirc) \hookrightarrow (\pi_{1}P_{n}(\kcirc))^{\times k} \) that maps each \(f\) in \(\pi_{0}\Aut_{\PnS}(\nSigma \kcirc)\) to to \((\pi_{1}\nOmega f(x_{1}),\dots,\pi_{1}\nOmega f(x_{k}))\), where \(\{x_{1},\dots,x_{k}\}\) are the image of the wedge inclusion of the generators.

    For surjectivity, let \(\phi \in \Aut_{\Gp}(\pi_{1}P_{n}(\kcirc)) \cong \Aut_{\Gp}(F(k)/F(k)_{n+1})\). 
    Since \(F(k)\) is free,  there is a lift \(\psi\colon F(k)\to F(k)\) of \(\phi\).
    Let \(B \psi \colon \kcirc \to \kcirc\) be the induced map on classifying spaces. 
    Then \(\Sigma B \psi\) is a homology equivalence of simply connected spaces, hence a weak equivalence. 
    By definition, \(\nSigma\) factors as \(\pS \to \mathscr{T}_{n}\pS \to \PnS\), where \(\mathscr{T}_{n}\pS\) is the category of special punctured \((n+1)\)-cubes and the map \(\pS \to \mathscr{T}_{n}\pS\) maps \(X\) to the punctured \((n+1)\)-pointed cone \(C_{n}(X)\).
    Since \(\Sigma B \psi\) is an equivalence, \(B \psi\) is sent to an equivalence in \(\mathscr{T}_{n}\pS\), and hence \(\nSigma B \psi\) is an equivalence in \(\PnS\). 
\end{proof}

By Theorem \ref{thm:D^n eq} we have a map of spaces
\[
    \mathscr{A}_{n} \colon T_{n}\bdemb(kI,D^{3}) \to \Aut_{\PnS}(\nSigma \kcirc)
\]
We now identify the induced map on \(\pi_{0}\) as the Artin representaion.

\begin{corollary}
    The following diagram commutes:
    \[\begin{tikzcd}[ampersand replacement=\&,cramped]
	{\pi_0\bdemb(k  I,D^{3})} \& {\Aut_{\cat{Gp}}(F(k)/F(k)_{n+1})} \\
	{\pi_0 T_{n}\bdemb(k  I,D^{3})} \& {\pi_{0}\Aut_{\PnS}(\nSigma \kcirc)            }
	\arrow["{\textrm{Artin}}"', from=1-1, to=1-2]
	\arrow[from=1-1, to=2-1]
	\arrow["{\pi_{0}\mathscr{A}_{n}}", from=2-1, to=2-2]
	\arrow["\cong", from=2-2, to=1-2]
\end{tikzcd}\]
    Here the right vertical arrow is the composite of \(\pi_{1}\nOmega\) and conjugation by a fixed isomorphism \(\sigma\) of \(F(k)/F(k)_{n+1} \xrightarrow{\cong} \pi_1P_n(\kcirc)\).
\end{corollary}

\begin{proof}
    By Theorem \ref{Prop: commutativity with emb} we see that the composition of the left vertical map and the lower horizontal map sends each \(e \in \bdemb(kI, D^{3})\) to
   \[
        \pi_{1}P_{n}(D^{2}\sminus{\ul{k}}) \xrightarrow{\cong} \pi_{1}P_{n}(D^{3}\sminus e(kI)) \xleftarrow{\cong} \pi_{1}P_{n}(D^{2}\sminus \ul{k})
    \]
    both maps are induced by boundary inclusions.

    Since \(\pi_{1}P_{n}(X)\) is \(n\)-nilpotent for any finite pointed space \(X\), the canonical map \(\pi_{1}X \to \pi_{1}P_{n}(X)\) factors through \(\pi_{1}X/(\pi_{1}X)_{n+1}\) and the factorisation is functorial.
    In particular, we have the following commutative diagram. 
    \[
        \resizebox{\textwidth}{!}{%
        \begin{tikzcd}[ampersand replacement=\&,cramped,column sep = tiny]
            {F(k)/F(k)_{n+1}} \& {\pi_1 (D^2\sminus{\ul{k}}) /\pi_1 (D^2\sminus{\ul{k}})_{n+1}} \& {\pi_1(D^3\sminus{ e(kI) })/ \pi_1(D^3\sminus{ e(kI) })_{n+1}} \& {\pi_1 (D^2\sminus{\ul{k}}) /\pi_1 (D^2\sminus{\ul{k}})_{n+1}} \& {F(k)/F(k)_{n+1}} \\
            \& {\pi_1 P_n(D^2\sminus{\ul{k}})} \& {\pi_1 P_n(D^3\sminus{ e(kI) })} \& {\pi_1 P_n(D^2\sminus{\ul{k}})}
            \arrow[from=1-1, to=1-2]
            \arrow["\sigma"',from=1-1, to=2-2]
            \arrow[from=1-2, to=1-3]
            \arrow[from=1-2, to=2-2]
            \arrow[from=1-3, to=2-3]
            \arrow[from=1-4, to=1-3]
            \arrow[from=1-4, to=2-4]
            \arrow[from=1-5, to=1-4]
            \arrow["\sigma",from=1-5, to=2-4]
            \arrow[from=2-2, to=2-3]
            \arrow[from=2-4, to=2-3]
        \end{tikzcd}%
        }
    \]
    where all the maps are isomorphisms.
\end{proof}




%%%%% Bibliography %%%%%

\bibliographystyle{amsalpha}
\bibliography{bibliography.bib}





\end{document}
