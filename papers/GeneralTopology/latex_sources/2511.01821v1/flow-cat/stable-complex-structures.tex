\subsection{Stable complex structures}\label{subsec:stable-complex}
In this subsection we show that the flow categories of Theorem~\ref{thm:sft-flow-category} admit stable complex structures. Recall that we fixed a nondegenerate contact manifold $(Y,\lambda)$, a $\lambda$-adapted almost complex structure $J$, a real number $L > 0$. Given this, let $\scM^{Y}_{\leq L}$ the flow category of Theorem~\ref{thm:sft-flow-category}.

\begin{theorem}\label{thm:stable-complex-structure}
    The symmetric flow category $\scM^{Y,\lambda}_{\leq L}$ admits a lift to a stably complex symmetric flow category.
\end{theorem}

The proof is similar to \cite[\S 11.3]{AB21} and \cite[\S B]{AB24}. We abbreviate $\scM \coloneqq \scM^{Y,\lambda}_{\leq L}$. The morphism space is the disjoint union
$$\scM(\Gamma^-,\Gamma^+)\; =\; \djun{\Lambda}\,\scM(\Gamma^-,\Gamma^+)_\Lambda$$ 
over all functions $\Lambda\cl \Gamma^-\to \Gamma^+$. We will phrase every statement in terms of these partitions as we have done in \textsection\ref{subsec:unstructured-flow-cat}. The tangent bundle of the global Kuranishi chart satisfies 
\begin{equation}\label{eq:tangent-bundle-decomposition}
	T^+\scM(\Gamma^-,\Gamma^+)_\Lambda\,\oplus\, \wh\fg_\Lambda =\; T\scB_{\Lambda}\,\oplus\, T^v\scT_{\Lambda}
\end{equation}
while the obstruction bundle is given by
\begin{equation}\label{eq:obstruction-bundle-decomposition}
	T^-\scM(\Gamma^-,\Gamma^+)_\Lambda\, =\; E_{\Lambda}\,\oplus\, \fp\fu_\Lambda\,
\end{equation}
where $\wh\fg_\Lambda$ is the Lie algebra of the covering group $\wh G_\Lambda$, $\fp\fu_\Lambda$ is the Lie algebra of the product $\p{\gamma\in \Gamma^+}{\PU(d_{\Lambda_\gamma}+1)}$ of projective unitary groups, and $E_\Lambda$ is a finite-rank $G$-vector bundle.\par In \textsection\ref{subsec:lift-of-objects}, we define the lift of the objects $\Gamma^+$ to objects of a stably complex flow category. Subsequently, we construct the stable complex structures on the morphism spaces in \textsection\ref{subsec:stable-complex-base} and \textsection\ref{subsec:stable-complex-fiber}, summarizing the results in Proposition~\ref{prop:existence-stable-complex}.


\subsubsection{Lift of the objects}\label{subsec:lift-of-objects}
Recall that the objects of a lift of $\scM$ to a stably complex flow category $\scM^{U}$ consist of a finite sequence $\Gamma$ of Reeb orbits of action at most $L$ and a virtual vector space $V_\Gamma = (V^+_\Gamma,V^-_\Gamma)$. We will construct for each Reeb orbit $\gamma$ of $\lambda$ a virtual $S^1$-vector bundle $V_\gamma$ over the $S^1$-manifold $E\gamma$ defined in~\eqref{eq:parametrisations-reeb-orbit}.Then, we define $$V_\Gamma \coloneqq\p{\gamma\in \Gamma}{V_\gamma}\to B\Gamma$$
to be the product vector bundle. 
\begin{remark}
    If $\gamma$ is a good Reeb orbit, this virtual vector bundle is orientable. It is non-orientable otherwise.
\end{remark}

Recall that the pre-perturbation datum $\fD$  we chose for the construction of $\scM$ includes a choice of $J$-linear connection $\conn^Y$ on $T\wh Y = \xi \oplus \bC$. While the construction of global Kuranishi charts does not require any properties of $\conn^Y$ except linearity with respect to $J$, for the following constructions it will be useful to assume that $\conn^Y$ has trivial monodromy around any simple Reeb orbit of action at most $L$. 
Moreover, fix a smooth cutoff function $\chi$ on $\bR$ with 
\begin{equation}
	\chi(s) = \begin{cases}
		1\quad& s \ll 0\\
		0 \quad& s \gg 0.
	\end{cases}
\end{equation}
%\begin{construction}\label{con:vector-space-for-object}
Given a Reeb orbit $\gamma$ and a parametrization $\wt\gamma\in E\gamma$, let $c_{\wt\gamma}\cl \bR\times S^1\to \wh Y$ be the trivial cylinder over $\wt\gamma$. Then, the pullback $c_{\wt\gamma}^*T\wh Y= c_{\wt\gamma}^*\xi \oplus \bC$, equipped with the chosen almost complex structure $J$, is a complex vector bundle over $\bR\times S^1$. It carries two canonical connections: the pullback of $\conn^Y$, which is complex linear, and the connection $\conn'$ induced by the pullback of $\cL_{R_\lambda}$ on $\xi$ and the trivial connection on $\bC$. We define the connection 
\begin{equation}
	\conn^{\wt\gamma} \coloneqq \conn^Y + \chi(s)(\conn'-\conn^Y)
\end{equation}
on $\bR\times S^1$ and let $\delbar^{\wt\gamma} = (\conn^{\wt\gamma})^{0,1}$ be the associated real Cauchy--Riemann operator. Since $c_{\wt\gamma}^*\xi\oplus \bC$ is trivializable, $(c_{\wt\gamma}^*\xi\oplus \bC,\delbar^{\wt\gamma})$ extends uniquely to a Cauchy--Riemann problem $(\cV_{\wt\gamma},\delbar^{\wt\gamma})$ on the capping off of $\bR\times S^1$ at the \emph{positive} end, using the trivialization induced by the connection $\conn^Y$.\footnote{We cap off at the positive instead of the negative end in order to obtain formulas compatible with the conventions in \cite{AB24}. This is for the same reason that we define $\scM(\Gamma^-,\Gamma^+)$ to be $\Mbar^J_{\sft}(\Gamma^+,\Gamma^-)$.} As all data that depend on the parametrization $\wt\gamma$ depend smoothly on it, we obtain a smooth $S^1$-vector bundle 
$$\cV\to E\gamma$$
with an $S^1$-invariant family $\delbar^\gamma = \{\delbar^{\wt\gamma}\}_{\wt\gamma}$ of Cauchy--Riemann operators. Fix an element $\wt\gamma\in E\gamma$ and choose a finite-dimensional complex representation 
\begin{equation}\label{eq:perturbation-for-point}
	\nu_0 \cl W'\to \Omega^{0,1}_c(\bR\times S^1,c_{\wt\gamma}^*T\wh Y)
\end{equation}
that is invariant under the isotropy of $\wt\gamma$ and is sufficiently large so that $\delbar^{\wt\gamma}\,\oplus \,\nu_0$ is surjective. We require that 
\begin{equation}\label{eq:support-of-orbit-perturbation}
    \supp(\nu_0(w)) \sub [-1,1]\times S^1
\end{equation}
for each $w \in W'$. Using the $S^1$-equivariance of $\delbar^\gamma$ and the transitivity of the $S^1$-action on $E\gamma$, we obtain a finite-rank $S^1$-vector bundle $W'_\gamma\to E\gamma$ with fiber-wise trivial $S^1$-action and a linear $S^1$-equivariant map $\nu$ from $W'_\gamma$ to the bundle with fibers $\Omega^{0,1}(\bR\times S^1,c_{\wt\gamma}^*\xi)$ which surjects onto the fiber-wise cokernel of $\delbar^\gamma$. Then, we define the vector bundles 
\begin{equation}
	V_\gamma^+ = \ker(\delbar^\gamma +\nu)\qquad \qquad V^-_\gamma = W'_\gamma.
\end{equation}
%\end{construction}


\subsubsection{Stable complex structures on base spaces}\label{subsec:stable-complex-base} This is essentially a more complicated version of \cite{AB24} since our base spaces go through an additional generalized blow-up compared to those in \cite{AB24}. 

\begin{lemma}\label{lem:stable-complex-corner-blow-up}
	The generalized blow-up of an almost complex manifold, equipped with the canonical smooth structure, admits a canonical almost complex structure up to contractible choice.
\end{lemma}


\begin{proof}
We will show that the generalized blow-up $\wt M \xrightarrow{\beta}M $ of a smooth manifold $M$ with corners comes equipped with a bundle isomorphism $T\wt M \to \beta^* TM$. The choice of this isomorphism is canonical up to a contractible set. In particular, if $M$ is almost complex, we can lift its almost complex structure to the generalized blow-up uniquely up to a contractible choice. Moreover, it follows from the argument in \cite[\S B.3.1]{AB24} and Remark~\ref{rem:neighbourhood-of-exceptional-boundary} that once the bundle isomorphism is determined on the exceptional boundary locus, then it can be extended up to a contractible choice using bump functions. Hence, it suffices to construct such an isomorphism over the blow-up locus.
For simplicity, we assume that $\wt M$ is generalized blow-up of the corner $$C = \lcap{i=1}{\ell+1} B_i,$$ 
where $B_1,\dots,B_{\ell +1}$ are boundary faces of $M$. Denote the exceptional boundary face $\beta\inv(C)$ by $E$. The construction begins with choosing any two metrics $\wt g, g$ on $\wt M$ and $M$ respectively. From hereon in the proof, we use canonical to mean \textit{canonical for a fixed pair $\wt g,g$.} A metric on $M$ determines a trivialization of the normal bundles $N_i \to B_i$. This in turn yields a canonical splitting 
\begin{equation}\label{eq:splitting-over-corner}
    TM|_C \cong TC \oplus\bigoplus N_i
\end{equation}
and we write $n_i$ for the inward-pointing unit normal vector in $N_i$.
By construction of the generalized blow-up, we have a canonical isomorphism $E \,\cong\, \bP_+ (\bigoplus_i N_i)$ to the positive part of the spherical projectivisation of the normal bundle at $C$, defined in Equation~\ref{eq:positive-part}. Using the canonical trivialization of $N_i$, this shows that $E \cong C\times \Delta^{\ell} $ where $\Delta^{\ell}$ is identified with the intersection of $S^{\ell}\cap [0,\infty)^{\ell+1}$. Thus, the pair $(g,\wt g)$ yields a canonical splitting 
\begin{equation*}
    T\wt M |_E \;\cong\; N_{E /\wt M}\,\oplus \,\pr_C^*TC \,\oplus\, \pr_{\Delta}^* T\Delta^{\ell},
\end{equation*} 
with $\ker(d\beta) = \pr_{\Delta}^*T\Delta^{\ell}$. In particular, $d\beta$ restricts to an isomorphism
\begin{equation*}
     (\pr_\Delta^*T\Delta^\ell|_{C\times\star})^\perp\,\xlongrightarrow{\simeq}\, \beta^*(TC\oplus N_1)|_{C\times \star},
\end{equation*}
where $\star = (1,0,\dots 0)\in \Delta^\ell$. The tangent space $T_\star \Delta^\ell $ is canonically identified with $ \{ 0\} \times \bR^\ell$ in $\bR^{\ell+1}$ and we write $e_1,\dots,e_\ell$ for the standard basis of $\bR^\ell$.
Fix an isomorphism 
\begin{equation*}
    \Phi \cl \pr_\Delta^*T \Delta^\ell|_{C\times\star} \to \im(d\beta|_{E})^\perp
\end{equation*}
such that $\Phi(c,\star,e_i) = n_{i+1}(c)$ for $i = 1,\dots,\ell$ under the isomorphism~\eqref{eq:splitting-over-corner}. The space of such isomorphisms is a contractible space. Therefore, the bundle isomorphism 
\begin{equation}
    T\wt M |_E \;\xra{\simeq}\; N_{\wt M / E} \oplus \pr_C^*TC \oplus \pr_{\Delta}^* T\Delta^{\ell} \;\xlongrightarrow{\beta_* \oplus \beta_* \oplus \Phi}\; \beta^* TM
\end{equation}
is canonical up to contractible choice.
\end{proof}

\begin{remark}\label{rem:real-oriented-stable-complex}
    In the case of a real-oriented blow-up $\text{Bl}_D(X)$, we do not require the choice of a metric on $X$ to be able to pull back an almost complex structure. Then, the preimage $E$ of $D$ is canonically isomorphic to $\bP_{> 0}(N_{D/X})$, whence we have a free $S^1$-action on $E$. Thus, a metric on $\text{Bl}_D(X)$ gives us a decomposition $$T\text{Bl}_D(X)|_E \cong \beta^*TD\oplus \bR\oplus L,$$
    where $L$ is the canonical line of $\bP_{> 0}(N_{D/X})$. On $\beta^*TD$ we have a canonical complex structure $J_D$ and we extend it to $J$ on $T\text{Bl}_D(X)|_E$ by mapping the unit section of $L$ to the unit vector of $\bR$ (corresponding to the canonical vector field of the action). 
\end{remark}


We are now going to apply this to the base space of the topological flow category $\scM$. Given a smooth manifold $M$ with the action of a compact Lie group $G$, it will be useful to write $T\ov{M}$ for the virtual vector bundle $TM -\fg$, where $\fg$ is the Lie algebra of $M$. In the result below we will not explicitly indicate the quotient by the group action by any tangent bundle should be considered in that sense.


\begin{lemma}\label{lem:stable-complex-base}
    For any $\Gamma^-,\Gamma^+\in \cP_{\leq L}$ and any partition $\Lambda\cl \Gamma^-\to \Gamma^+$ of $\Gamma^-$, there exists 
    \begin{enumerate}[label=\normalfont\arabic*),leftmargin=20pt,ref=\normalfont\arabic*]
        \item a complex $\wh G_\Lambda$-vector bundle $I^b_\Lambda\to \scBS_\Lambda$
        \item an equivalence 
        \begin{equation}
            T\cc{\scB}^\bR_\Lambda\,\oplus\,\fp\fu_\Lambda\,\oplus\, \bR \;\simeq\;I^b_\Lambda\,\oplus\, \bR^{\Gamma^+}
        \end{equation}
        of $\wh{G}_\Lambda$-equivariant virtual vector bundles on $\scBS_\Lambda$.
        \item for any factorization $\Gamma^-\xra{\Lambda^0 } \Gamma\xra{\Lambda^1}\Gamma^+$ of $\Lambda$ a split equivariant embedding 
        \begin{equation}
            I^b_{\Lambda^0}\,\oplus I^b_{\Lambda^1}\,\to\, I^b_\Lambda
        \end{equation}
        of complex equivariant vector bundles over $\scBS_{\Lambda^0}\qtimes{\bT_\Gamma}\scBS_{\Lambda^1}$.
        \end{enumerate}
        They satisfy the following compatibility conditions.
        \begin{itemize}[leftmargin=25pt]
            \item The diagram 
        \begin{equation}\label{compatiblity-1}\begin{tikzcd}
		   I^b_{\Lambda^0}\,\oplus \bR^{\Gamma}\,\oplus I^b_{\Lambda^1}\,\oplus \bR^{\Gamma^+}\arrow[r,""] \arrow[d,""]&I^b_\Lambda \oplus \bR^{\Gamma}\,\oplus \bR^{\Gamma^+}\arrow[dd,""]\\
          T\cc{\scB}^\bR_{\Lambda^0}\,\oplus\fp\fu_{\Lambda^0}\,\oplus \bR_{\Lambda^0}\,\oplus T\cc{\scB}^\bR_{\Lambda^1}\,\oplus\fp\fu_{\Lambda^1}\,\oplus \bR_{\Lambda^1}\arrow[d,"\simeq"]\\
         \ft_{\Gamma}\,\oplus T\cc{\scBS_{\Lambda^0}\qtimes{\bT_\Gamma}\scBS_{\Lambda^1}}\,\oplus\bR_{\Lambda^0}\,\oplus\bR_{\Lambda^1}\,\oplus  \fp\fu_{\Lambda^0}\,\oplus \fp\fu_{\Lambda^1} \arrow[r,""] &\ft_{\Gamma}\,\oplus T\scBS_{\Lambda}\,\oplus \fp\fu_\Lambda\oplus\bR_\Lambda\end{tikzcd} \end{equation}
         commutes, where $\ft_\Gamma$ is identified canonically with $\bR^{\Gamma}$, $\bR_{\Lambda^1}$ is identified with $\bR_{\Lambda}$, and $\bR_{\Lambda^0}$ is identified with the normal vector of the boundary stratum of $\scBS_\Lambda$. Moreover, we identify $\fp\fu_\Lambda$ with $\fp\fu_{\Lambda^0} \oplus \fp\fu_{\Lambda^1}\oplus \frac{\fp\fu_\Lambda}{\fp\fu_{\Lambda^0}\oplus \fp\fu_{\Lambda^1}}$ and the latter summand with the normal bundle of the image of $\cc{\scB^\bR_{\Lambda^0}\qtimes{\,\bT_\Gamma}\scB^\bR_{\Lambda^1}}$ in $\ov{\scB}^\bR_\Lambda$ using Lemma~\ref{lem:boundary-up-to-stabilisation}.
        \item For any factorization $\Lambda = \Lambda^2\g \Lambda^1\g \Lambda^0$ the square
        \begin{equation}\label{compatibility-2}\begin{tikzcd}
		   I^b_{\Lambda^0}\,\oplus I^b_{\Lambda^1}\,\oplus I^b_{\Lambda^2}\arrow[r,""] \arrow[d,""]&I^b_{\Lambda_0}\,\oplus I^b_{\Lambda^{12}} \arrow[d,""]\\
         I^b_{\Lambda^{01}}\,\oplus I^b_{\Lambda^2} \arrow[r,""] & I^b_\Lambda\end{tikzcd} \end{equation}
         over $\scBS_{\Lambda^0}\qtimes{\;\bT_\Gamma}\scBS_{\Lambda^1}\qtimes{\;\;\bT_{\Gamma'}}\scBS_{\Lambda^2}$ commutes.
         \end{itemize}
\end{lemma}

\begin{proof}
     Recall that the base space of $\scM(\Gamma^-,\Gamma^+)_\Lambda$ is the manifold $\scBS_\Lambda$ equipped with an action of $\wh G_\Lambda\coloneqq\bT_{\Gamma^-}\times\bT_{\Gamma^-}\times G_\Lambda$, where $\bT_\Gamma := (S^1)^\Gamma$ and $G_\Lambda = \p{\gamma\in \Gamma^+}{U(d_{\Lambda_\gamma})}$. In particular, it carries the structure of a principal $\bT_{\Gamma^-}\times\bT_{\Gamma^-}$-bundle $\pi_\Lambda\cl \scBS_\Lambda\to \lc{\scB}^\bR_\Lambda.$ The map $\pi_\Lambda$ is given by forgetting the asymptotic markers. The space $\lc{\scB}^\bR_\Lambda$ is a stratum of the (corner) blow-up of $\lc{\scB}^P_\Lambda\times[0,1)^{\Gamma^+}$, where $\lc{\scB}^P_\Lambda$ is a real oriented blow-up of the complex manifold 
    $$\scB(\Lambda)\sub \p{\gamma\in \Gamma^+}{\Mbar_{0,\gamma\sqcup \Lambda_\gamma}(\bP^{d_{\Lambda_\gamma}},d_{\Lambda_\gamma})_{\varphi(z_\gamma) = [1:0:\dots:0]}}.$$
Given any factorization $\Lambda = \Lambda^1\g \Lambda^0$, we have an embedding 
\begin{equation}\label{eq:embeddings-base-for-stable-complex}
    \scBS_{\Lambda^0}\qtimes{\;\bT_\Gamma}\scBS_{\Lambda^1}\to \scBS_\Lambda
\end{equation}
covering
\begin{equation}\label{eq:embeddings-pardon-base-for-stable-complex}
    \scBR_{\Lambda^0}\qtimes{\;\bT_\Gamma}\scBR_{\Lambda^1}\to \scBR_\Lambda,
\end{equation}
both constructed in \textsection\ref{subsec:embeddings-base}. Using \cite{Kott}, we can construct systems $\{\wt{g}_\Lambda\}_\Lambda$ and $\{g_\Lambda\}_\Lambda$ of invariant Riemannian metrics on $\scBS_\Lambda$ and $\scBR_\Lambda$, respectively, so that the embeddings~\eqref{eq:embeddings-base-for-stable-complex} and~\eqref{eq:embeddings-pardon-base-for-stable-complex} are isometric. By Remark~\ref{rem:real-oriented-stable-complex}, the metric $g_\Lambda$ induces a complex structure on $T\lc{\scB}^P_\Lambda$. Using $\wt g_\Lambda$ and $g_\Lambda$ in Lemma~\ref{lem:stable-complex-corner-blow-up}, we get a decomposition 
\begin{equation*}
    T\scBS_\Lambda \;\cong\; T\lc{\scB}^\bR_\Lambda\oplus \ft_{\Gamma^+}\,\oplus\, \ft_{\Gamma^-}\;\cong\; T\Delta^{|\Gamma^+|-1}\,\oplus\,\beta^*T\lc{\scB}^P_\Lambda\,\oplus \,\ft_{\Gamma^+}\,\oplus \,\ft_{\Gamma^-}
\end{equation*}
Identifying $T\Delta^{k-1}$ with $\bR^k/\bR$, this shows that 
\begin{equation}\label{eq:decomposition-1}
    T\scBS_\Lambda\,\oplus\, \bR \;\cong\; T\bR^{\Gamma^+}\,\oplus\,\beta^*T\lc{\scB}^P_\Lambda\,\oplus\, \ft_{\Gamma^+}\,\oplus\, \ft_{\Gamma^-}
\end{equation}
canonically. Thus, we can set $I^{b,+}_\Lambda \coloneqq \beta^*T\lc{\scB}^P_\Lambda$. The decomposition~\eqref{eq:decomposition-1} can be further transformed to 
\begin{equation}\label{eq:decomposition-2}
    T\cc{\scB}^\bR_\Lambda\,\oplus\,\fg\fl_\Lambda\,\oplus\, \bR \;\cong\; T\bR^{\Gamma^+}\,\oplus \,\beta^*T\lc{\scB}^P_\Lambda\,\oplus\, \fp\fu_\Lambda
\end{equation}
where $\fg\fl_\Lambda \stackrel{\eqref{eq:comparing-group-quotients}}{=}\ii\fu_\Lambda\oplus\fp\fu_\Lambda$ is the Lie algebra of the complex Lie group $\cG_\Lambda$. Therefore, we can set $$I^{b,-}_\Lambda \coloneqq\fg\fl_\Lambda.$$
\noindent The maps $I^{b,-}_{\Lambda^0}\,\oplus I^{b,-}_{\Lambda^1}\to I^{b,-}_{\Lambda}$ are induced by the block matrix inclusions $\cG_{\Lambda^0}\times\cG_{\Lambda^1}\hkra\cG_{\Lambda}$, whence their compatibility in the sense of the square~\eqref{compatibility-2} is immediate. Meanwhile, the embeddings $I^{b,+}_{\Lambda^0}\,\oplus I^{b,+}_{\Lambda^1}\to I^{b,+}_{\Lambda}$ come from the embeddings~\eqref{eq:embedding-base-spaces} of base spaces. The commutativity of~\eqref{compatibility-2} follows in their case from the inductive choice of Riemannian metrics. Since the vertical maps in Diagram~\eqref{compatiblity-1} are defined using these compatible metrics, the commutativity of this diagram follows as well.
\end{proof}

\subsubsection{Stable complex structures on vertical tangent bundle}\label{subsec:stable-complex-fiber}
We will first prove the existence of stable complex structures on each morphism space $\scM(\Gamma^-,\Gamma^+)$ separately before considering their compatibilities. Our strategy is similar to the one of \cite[\S 11.3.4]{AB21}. The idea is to construct a homotopy equivalence $\wt\scT_\Lambda\to \scT_\Lambda$ with two sections $a$ and $b$ and a family of Cauchy--Riemann operators over $\wt\scT_\Lambda$ whose restrictions to $\im(a)$ and $\im(b)$ are given by 
$$\delbar^{\gamma}\,\#\, D\delbar_J\qquad \text{and}\qquad D^\bC\,\# \,(\#_{\gamma'\in \Lambda_\gamma}\delbar^{\gamma'}),$$
respectively, where $D^\bC$ is a (non-canonical) complex-linear Cauchy--Riemann operator. 

\begin{notation*}
   Given a partition $\Lambda\cl \Gamma^-\to \Gamma^+$, we write $S = S_\Lambda$ for the set of Reeb orbits $\gamma$ (counted with multiplicities) so that $\gamma\in \Gamma\sm\im(\Lambda^0)$ for some factorization $\Gamma^-\xra{\Lambda^0}\Gamma\xra{\Lambda^1}\to \Gamma^+$ of $\Lambda$. Note that we allow $\Lambda^1 = \ide_{\Gamma^+}$.
\end{notation*}

\begin{lemma}\label{lem:auxiliary-thickening}
	There exists for each function $\Lambda\cl \Gamma^-\to \Gamma^+$, there exists a $G_\Lambda$-equivariant fiber bundle 
	$$q\cl \scT_{\Lambda}^c\to \scT_{\Lambda}$$ with fibers given by $[0,1]$. It admits two equivariant sections $a_\infty$ and $a_0$.
\end{lemma}

\begin{proof}
	Recall that we chose a pre-perturbation datum $\fD = (\wt\lambda,\conn,p)$ for the construction of $\scM$. We observe first that we can construct the global Kuranishi chart for buildings $\Nbar^{\,J}(\Gamma^-,\Gamma^+)_\Lambda$ in the trivial symplectic cobordism using the $\fD$. In this case, the base space agrees with the base space $\scBS_\Lambda$ of the global Kuranishi chart for $\Mbar^{\,J}_{\sft}(\Gamma^-,\Gamma^+)_\Lambda$, but the thickening is defined using the family $\cZ^c_\lambda$ in Definition~\ref{de:family-of-cobordism-buildings} instead of Definition~\ref{de:family-of-tree}. There exists a canonical rel--$C^1$ submersion 
    \begin{equation}\label{eq:forgetful-1}
        \scT^c_\Lambda\;\to\; \scT_\Lambda,
    \end{equation}
    which is the identity on the level of base spaces. Its fibers are canonically identified with $[0,1]$. The map $\scT^c_\Lambda\to\scT_\Lambda$ has two canonical sections
     \begin{equation*}
         a_0 \cl \Nbar^{\,J}(\Gamma^-,\Gamma^-)\ov{\times}_{B\Gamma^-}\scT_\Lambda \hkra \scT^c_\Lambda\qquad \quad a_\infty \cl \scT_\Lambda\ov{\times}_{B\Gamma^+}  \Nbar^{\,J}(\Gamma^+,\Gamma^+)\times \hkra \scT^c_\Lambda,
     \end{equation*}
     given by adding trivial cylinders (in the cobordism) at the positive puncture or at the negative one. If the compactified domain is given by the sphere in the middle of Figure~\ref{fig:auxiliary-spheres}, then the section $a_\infty$ is given by the configuration on the left, while $a_0$ is given by the right configuration on the right hand side:
     
	\begin{figure}[H]
		\centering
		\includegraphics[scale=1.4]{./on-top.png}\hspace{0.8cm}
		\includegraphics[scale=1.2]{./smooth.png}\hspace{.8cm}
		\includegraphics[scale=0.9]{./at-bottom.png}
		\label{fig:auxiliary-spheres}
		\caption{$\qquad\qquad\qquad$}
	\end{figure}
	\noindent Here the black spheres are those components that are mapped to the symplectic cobordism (being trivial cylinders in the left and right configuration), while the gray components are mapped to the symplectisation. The point labeled by $\infty$ is the positive puncture, while all other marked points correspond to negative punctures.
\end{proof}

Fix a Riemannian metric $h$ on $\wh Y$ and let $B_r(\gamma)$ be the ball of radius $r > 0$ around a Reeb orbit $\gamma$. Fix some number $\delta > 0$ that is less than the injectivity radius of $h$. Let $d$ be the associated distance and $\Phi$ be the parallel transport of $h$ and write $\Phi_{x\to y}$ for parallel transport along the unique geodesic connecting $x$ to $y$, whenever $d(x,y)$ is sufficiently small.

\begin{proposition}\label{prop:interpolating-to-complex}
	There exists a principal $G'_\Lambda$-bundle $\wt\scT_\Lambda\to \scT^c_\Lambda$, which admits a $G'_\Lambda\times \wh G_\Lambda$-vector bundle $\cW\to \wt\scT_{\Lambda}$, a vector bundle $W^v_\Lambda$, and a complex virtual vector bundle $I^v_{\Lambda}$ so that
	\begin{equation}
		a_\infty^*\cW \,\cong\, T^v\scT_{\Lambda}\,\oplus\,\bR^{\Gamma^+}\,\oplus\, V^+_{\Gamma^+}\,\oplus\, V^-_{\Gamma^-}\,\oplus I^{v,-}_\Lambda\,\oplus\, W^v_\Lambda
	\end{equation}
        and
    \begin{equation}
        a_0^*\cW \,\cong\,E_\Lambda\,\oplus\, I^{v,+}_{\Lambda}\,\oplus\, V_{\Gamma^-}^+\,\oplus\, V_{\Gamma^+}^-\,\oplus\, W^v_\Lambda,
    \end{equation}
	where $a_\infty$ and $a_0$ are the sections of Lemma~\ref{lem:auxiliary-thickening}.
\end{proposition}

\begin{proof}
	Let $p' \gg p$ be a large prime number, and let $\wt\fD = (\wt\lambda,\conn,p',p)$ be the pre-perturbation datum for the moduli spaces of buildings in the trivial symplectic cobordism. Then, we can construct the global Kuranishi chart for $\Nbar^{\,J}(\Gamma^-,\Gamma^+)$ using the base space $\scB^c_\Lambda$ defined in \textsection\ref{subsec:cobordism-base}. Let 
	$$\wt\cB_\Lambda \sub \p{\gamma\in \Gamma^+}{\Mbar_{0,\gamma\sqcup \Lambda_\gamma}(\bP^{d'_{\Lambda_\gamma}}\times \bP^{d_{\Lambda_\gamma}},(d'_{\Lambda_\gamma},d_{\Lambda_\gamma}))_{z_\gamma\mapsto (e_0,e_0)}}$$
	be the preimage of $\scB^c_\Lambda\times\scB_\Lambda$. The argument of \cite[Lemma~7.3]{HS22} shows that $\wt\cB_\Lambda$ is unobstructed, thus a complex manifold with a complex $\cG'_{\Lambda}$-action. We can thus define 
	\begin{equation}
		\wt\scB_\Lambda = \scB^c_\Lambda\times_{\cB_\Lambda}\wt\cB_\Lambda.
	\end{equation}
	The only difference between $\scB^c_\Lambda$ and $\wt\scB_\Lambda$ is that framings in the former are constant on the domains of trivial cylinders, while those in $\scB^c_\Lambda$ are not (they have degree $(p'-p)\cA_{\wt\lambda}(\gamma)$). Thus, the universal family $\wt\cC_\Lambda\to \wt\scB_\Lambda$ allows us to access these domains. In order to lift this to the thickenings, we fix a map $\zeta'_\cU$ as in~\eqref{eq:slice-map} that determines the unitary framings in $\scB^c_\Lambda$ and set 
	\begin{equation}
		\wt\scT_\Lambda \coloneqq \set{(\wt\varphi,\varphi,u,w)\in \wt\scB_\Lambda\times_{\scB^c_\Lambda}\scT^c_\Lambda\mid T_{\wt\varphi} = T_u\, \zeta'_\cU(\wt\varphi,u) = 0}.
	\end{equation}
	The conditions we impose are open in the fiber product, so this is a rel--$C^1$ manifold with corners. Moreover, the canonical forgetful map $\wt\scT_\Lambda\to \scT^c_\Lambda$ is a principal $G'_\Lambda$-bundle, where $G'_\Lambda$ is a product of unitary groups. Let $\cC^{sc}\sub \wt\cC_\Lambda\inn$ be the locus of points that are mapped to the symplectic cobordism. Then, the height functions of the respective map give us a rel--$C^1$ function
	\begin{equation}
	\eva_\bR\cl \wt \cC^{sc}\to \bR,
	\end{equation}
	which extends to a continuous function $\ov\eva_\bR\wt\cC_\Lambda\to [-\infty,\infty]$. 
	Using $\eva_\bR$, we can extend $\chi$ to a rel--$C^1$- map  $\wt\chi\cl \wt\cC_\Lambda\to [0,1]$. Note that $\wt\chi$ is invariant under the covering group action.\par
	Let $\cF\to \wt\scT_\Lambda$ and $\cE^{0,1}\to \wt\scT_\Lambda$ be the s bundles with
	$$\cF_{(\wt\varphi,u,w)} = W^{\ell,2,\delta}\lbr{\wt\cC\inn_{\wt\varphi},u^*T\wt Y} \qquad \qquad \cE^{0,1}_{(\varphi,u,w)} = W^{\ell,2,\delta}\lbr{\wt\cC\inn_{\wt\varphi},\Lambda^{0,1*}_{\wt\cC\inn_{\wt\varphi}}\otimes u^*T\wt Y},$$
	where $\ell \ge 6$ and $\delta > 0$ is a sufficiently small exponential weight.
	Let $B_u\in \Omega^{0,1}(\dot{C},\End(u^*T\wt Y))$ be the difference $D^{\conn^Y}_u - (\conn^Y\cdot)^{0,1}$. We define the rel--$C^1$ family $\wt D_{(\wt\varphi,u,w)}$ of Cauchy--Riemann operators by 
	\begin{equation}\label{eq:interpolating-cr-operator}
		\wt D_{(\wt\varphi,\varphi,u,w)}(\xi) = D^{\conn^Y}_u(\xi) -\wt\chi(\wt\varphi,\varphi,u,w,\cdot)B_u(\xi).
	\end{equation}
	Writing the domain of $a_\infty(\varphi,u,w)$ as $\djun{\gamma\in \Gamma^+}C^\infty_\gamma\vee C_\gamma$, where $C^\infty_\gamma$ is the `cylinder on top' (see Figure \ref{fig:auxiliary-spheres}), this operator is the restriction of
	\begin{equation}
		\wt D_{a(\varphi,u,w)} =\begin{cases}
			D^{\conn^Y}_u(\xi) \quad& \text{on }C^\infty(C_\gamma,u^*T\wt Y) \\
			\delbar^{\wt\gamma}\quad& \text{on }C^\infty(C_\gamma^\infty,c_{\wt\gamma}^*T\wt Y),
		\end{cases}
	\end{equation}  
	where $c_{\wt\gamma}$ is the trivial cylinder a parametrisation $\wt\gamma$ of $\gamma$, induced by $u$ and the asymptotic marker at $z_\gamma$. On the other hand, writing the domain of $a_0(\varphi,u,w)$ as $\djun{\gamma\in \Gamma^+}C_\gamma\vee \bigvee\limits_{\gamma'\in \Lambda_\gamma}C^0_{\gamma'}$, where $C^0_{\gamma'}$ are the `additional cylinders at the bottom', $\wt D_{b(\varphi,u,w)}$ is the restriction of 
	\begin{equation}
		\wt D_{b(\varphi,u,w)} =\begin{cases}
			(\conn^Y\xi)^{0,1} \quad& \text{on }C^\infty(C_\gamma,u^*T\wt Y) \\
			\delbar^{\wt\gamma'}\quad& \text{on }(C_{\gamma'}^0,c_{\wt\gamma'}^*T\wt Y),
		\end{cases}
	\end{equation}
	where $\wt\gamma'$ is the parametrization of $\gamma'$ determined by $u$ and the asymptotic marker at $z_{\gamma'}$.\par 
	We use finite-dimensional complex vector spaces of perturbations to achieve surjectivity of $\wt D$. However, first we extend the maps $\nu_\gamma\cl V^-_\gamma\to \Omega^{0,1}_c(\bR\times S^1,u_\gamma^*T\wt Y)^\bR$ of \textsection\ref{subsec:lift-of-objects} to maps $\nu_\gamma\cl p_\gamma^*V^-_\gamma\to \cE^{0,1}$ for $\gamma\in \Gamma^+\sqcup \Gamma^-$, where $p_\gamma$ is the evaluation map~\eqref{eq:evaluation-to-reeb-orbit}. We now explain how we can construct extensions of the perturbations~\eqref{eq:perturbation-for-point}.

    \begin{construction}\label{con:extending-perturbations}  We have the evaluation map  
    \begin{equation}
    \eva_Y\cl \wt\cC\inn_\Lambda \to Y : (\wt\varphi,u,w,z) \mapsto \pr_Y(u(z)),
    \end{equation} Choose now for $\gamma\in \Gamma^\pm$ a neighborhood $U_\gamma^\pm$ of the canonical section $\sigma_\gamma\sub \wt\cC_\Lambda$ given by the respective marked point so that
	\begin{itemize}
		\item $\eva_Y({U^\pm_\gamma})\sub B_\delta(\gamma)$,
		\item the closures are pairwise disjoint, and
		\item the closures do not meet the critical points of $\wt\cC_\Lambda\to \wt\scT_\Lambda$.
	\end{itemize}  
	Fix for each $\gamma$ a rel--$C^1$ invariant cut-off function $j_\gamma\cl \wt\cC_\Lambda\to [0,1]$ which is identically $1$ near $\sigma_\gamma$ and supported in $U_\gamma$. Recall that we have a canonical retraction $r \cl B_\delta(\gamma)\to \im(\gamma)$. Let $\Phi_x$ be the parallel transport along the normal geodesic from $r(x)$ to $x$ (using the Riemannian metric $h$ on $Y$ chosen above). Then, the vector bundle map
    \begin{equation}
    \wt\nu_\gamma \cl \wt\scT_\Lambda\times_{E\gamma} V_\gamma^- \,\to\, \cE^{0,1}
    \end{equation}
    at $y = (\wt\varphi,\varphi,u,w)$ is given by
    \begin{equation}\label{eq:extension-orbit-lift}
    	\wt\nu_\gamma(y,v) \coloneqq 
    \begin{cases}
        j_\gamma(y,\cdot)\wt\chi(\eva_\bR(y,\cdot))\,\Phi_{u_Y(\cdot)}[\nu_\gamma(p_\gamma(y),v)(r_\gamma(u_Y(\cdot)))]\qquad & \text{on }\wt\cC_y\cap U^\pm_\gamma,\\
        0 \quad & \text{otherwise}.
    \end{cases}
    \end{equation}
	where the superscript of $U^\pm_\gamma$ is determined by whether $\gamma$ labels a positive or negative puncture. This is well-defined even if $\gamma$ is multiply-covered by the definition of the map $\nu_\gamma$ in \textsection\ref{subsec:lift-of-objects}.
\end{construction}

   \noindent By~\cite{Kott}, we can arrange for the choices in Construction~\ref{con:extending-perturbations} to be compatible across boundary strata. We now do a similar extension for $\nu_\gamma$ whenever $\gamma$ is a Reeb orbit at which a building in $\Mbar_{\sft}^{\,J}(\Gamma^+,\Gamma^-)$ breaks. Since any element of $\Gamma^+$ has action $\leq L$ and each component has a unique positive puncture, we have $\cA_\lambda(\gamma)\leq L$. By the construction in Theorem~\ref{thm:flo_bim}, each boundary stratum of $\wt\scT_\Lambda$ admits a map
    \begin{equation}\label{eq:bottom-section}
        \wt\scT_{\Lambda^{01}}\to\scT^c_{\Lambda^{01}}\to G_\Lambda\times_{G_{\Lambda^0}\times G_{\Lambda^1}}\im\lbr{\scT^c_{\Lambda^0}\ov{\times}_{B\Gamma}\scT_{\Lambda^1}\hkra \wt\scT_\Lambda},
    \end{equation}
    respectively,
    \begin{equation}\label{eq:infinity-section}
        \wt\scT_{\Lambda^{10}}\to \scT^c_{\Lambda^{10}}\to G_\Lambda\times_{G_{\Lambda^0}\times G_{\Lambda^1}}\im\lbr{\scT_{\Lambda^0}\ov{\times}_{B\Gamma}\scT^c_{\Lambda^1}\hkra \wt\scT_\Lambda}
    \end{equation}
	for some $\Gamma$ and partitions $\Lambda^1 \cl \Gamma\to \Gamma^+$ and $\Lambda^0 \cl \Gamma^-\to \Gamma$ with $\Lambda^1\g\Lambda^0 = \Lambda$. The first map in each case is induced by the forgetful map $\wt\scT_\Lambda\to \scT^c_\Lambda$, while the second map is a vector bundle map (cf. Lemma~\ref{lem:boundary-up-to-stabilisation}). The evaluation map $p_\gamma\cl \scT_{\Lambda^1}\to E\gamma$ for $\gamma\in \Gamma$ is invariant under the $G_{\Lambda^1}$-action, so it induces a rel--$C^1$ map 
	\begin{equation}\label{eq:extension-orbit-evaluation}
		\wt p_\gamma\cl \wt\scT_{\Lambda^{01}}\to E\gamma
	\end{equation}
    that is equivariant with respect to the action of $G_\Lambda$.
    \begin{construction}\label{con:extending-breaking-perturabtions} Let $U_{ij}\sub \wt\scT_{\Lambda^{ij}}$ be an open subset of the boundary stratum so that for any curve in $U_{ij}$ the component that is mapped to the symplectization is either directly above the Reeb orbits $\gamma\in \Gamma$ or below it. In other words, $U_{ij}$ is the complement of a union of some of the boundary strata. We can use the same definition as in Construction~\ref{con:extending-perturbations} once we have constructed for each such $\gamma$ an equivariant extension of the map $\wt{p}_\gamma|_{U^\gamma_{ij}}$. We discuss the case of the embedding~\eqref{eq:infinity-section}, and will just write $U$. Let $\sigma_\gamma\cl \wt\scT_{\Lambda^{ij}}\to \wt\cC_\Lambda$ be the section given by the nodal point labeled by $\gamma\in \Gamma^+$. For each such $\gamma$, let $U''_\gamma$ be a neighborhood of $\sigma_\gamma(U)$ so that 
    \begin{itemize}
        \item $\eva_Y(U''_\gamma\sm \im(\sigma_\gamma))\sub B_\delta(\gamma)$,
        \item $\eva_\bR|_{U''_\gamma\cap \wt\cC\inn_\Lambda}$ is a fiber-wise submersion
        \item $\cc{U''_\gamma}$ does not intersect $\cc{U_{\gamma'}}$ for any $\gamma'\in \Gamma^+\sqcup \Gamma'$
        \item $\cc{U''_\gamma}\cap \cc{U''_{\gamma'}} = \emst$ if $\gamma\neq \gamma'$ are Reeb orbits at which buildings in $\wt\scT_\Lambda$ break,\footnote{Note that $\gamma$ might appear twice as a Reeb orbit at which a building breaks. These two occurrences are distinct, although we omit this from the notation.}
        \item $\cc{U''_\gamma}$ only meets the critical points of $\wt\cC_\Lambda\to \wt\scT_\Lambda$ in $\im(\sigma_\gamma)$.
    \end{itemize}
    In particular, we may assume that the intersection $\wt\cC_y\cap U''_\gamma$ with a fiber is contained in the unique irreducible component $C^\bullet_y$ of $\wt\cC_y$ that is mapped to the symplectic cobordism. Set $U'_\gamma := U''_\gamma\sm\im(\sigma_\gamma)$. The asymptotic marker at the positive puncture $\Lambda^1(\gamma)$ and the matching isomorphism on nodes induce for each $y\in U$ a path $L^\gamma_y\sub C_y^\bullet$ from the unique positive `puncture' (which is a point in $\wt\cC_\Lambda$) of $C_y^\bullet$ to $\sigma_\gamma(y)$. Since $\pi|_{U_\gamma}\cl U'_\gamma\to \pi(U'_\gamma)$ is a rel--$C^1$ submersion, we can use Ehresmann's Lemma and the connection induced by the chosen Riemannian metric and the Fubini-Study metric on complex projective space to obtain for each $y \in \pi(U'_\gamma)$ an interval $L_y \sub C^\bullet_y$, possibly replacing the whole thickening by an invariant neighborhood of the zero locus $\fs\inv(0)$. 
    Writing $d$ for the Gromov--Hausdorff metric on $\wt\scT_\Lambda$, we define 
    \begin{equation}
        R_\gamma\cl \pi(U'_\gamma) \to [0,\infty) : y \mapsto d(y,\wt\scT_\Lambda)
    \end{equation}
    be the distance from the boundary stratum, where $\pi \cl \wt\cC_\Lambda\to \wt\scT_\Lambda$ is the universal family. Then, we can extend $\sigma_\gamma$ to a rel--$C^1$ map $\wt\sigma_\gamma\cl \pi(U'_\gamma) \to U''_\gamma$ by letting $\wt\sigma_\gamma(y)=:z$ be the unique point in the fiber so that 
    $$\eva_\bR(y,z) = R_\gamma(y) \qquad\qquad z\in L_y.$$
    Such a point exists and is unique due to the asymptotic behavior of the curves near nodes. Finally, we can define $\wt{p}_{\gamma,\gamma'}\cl \pi(U'_\gamma)\to E\wt\gamma$ via the composition 
    \begin{equation*}
      \pi(U'_\gamma)\,\xra{\wt\sigma_\gamma}\, U''_{\gamma}\,\xra{(y,z)\,\mapsto r(\eva_Y(u,z))} \,\im(\gamma) \,\xra{\simeq} \, E\gamma,
    \end{equation*}
    where the first map associates to $y$ the point $r(\eva_Y(\sigma_{\gamma'}(y))$ and the second map is the inverse of $\wt\gamma\mapsto \wt\gamma(1)$. Now we may define the extension by the term~\eqref{eq:extension-orbit-lift} with $p_\gamma$ replaced by $\wt p_\gamma$.
    \end{construction}
    
    \noindent We define 
	\begin{equation}
		W^v_\Lambda \coloneqq \bigoplus\limits_{\Gamma^- \prec \Gamma\prec \Gamma^+}\bigoplus\limits_{\gamma\in \Gamma} V_\gamma^-
	\end{equation}
	equipped with the induced map $\wt\nu_\Lambda\cl W^v_\Lambda \to \cE^{0,1}$.
	Since $q \cl \wt\scT_\Lambda\to \scT_\Lambda$ is proper, the preimage $q\inv(\fs\inv(0))$ is compact. Thus, we may find a finite-rank complex $G$-vector bundle $I^{v,-}_{\Lambda}\to \scT_\Lambda$ over the `original' thickening and an equivariant map $\kappa_\Lambda\cl I^{v,-}_\Lambda\to \cE^{0,1}$ so that the operator
	\begin{equation}\label{eq:perturbed-homotopy-of-cr}
		\wt D_{(\wt\varphi,u,w)}+\nu_{\Gamma^+}+\nu_{\Gamma^-}+\wt\nu_\Lambda+\kappa_\Lambda \cl \cF_{(\wt\varphi,u,w)}\,\oplus\, V_{\Gamma^+}^-\oplus\, V_{\Gamma^-}^-\oplus\, W^v_\Lambda\oplus {I^{v,-}_\Lambda}_{_{(\wt\varphi,u,w)}}\to\, \cE^{0,1}_{(\wt\varphi,u,w)}
	\end{equation}
	is surjective for any ${(\wt\varphi,\varphi,u)}\in q\inv(\fs\inv(0))$. Shrinking $\scT_\Lambda$, we may assume~\eqref{eq:perturbed-homotopy-of-cr} is surjective on all of $\wt\scT_\Lambda$. We define 
	\begin{equation}
		\wt\cW \coloneqq \ker\lbr{\wt D + \mu_\Lambda+\nu_{\Gamma^+}+\nu_{\Gamma^-}+\wt\nu_\Lambda+\kappa_\Lambda} 
	\end{equation}
	where $\mu_\Lambda\cl E_\Lambda\to \cE^{0,1}$ is the perturbation space chosen in the construction of the global Kuranishi chart. Since $\wt\cW$ is invariant under the free $G'_\Lambda$ action, the quotient $\cW \coloneqq \wt\cW/G'_\Lambda$ admits a vector bundle map $\cW\to \scT^c_\Lambda$,
    Since $\wt\nu_\Lambda$ vanishes near $a_\infty$ and $a_0$, we have a canonical isomorphism
	\begin{equation*}
		a_\infty^*\cW \;\cong\; q^*T^v\scT_\Lambda\,\oplus\, \bR^{\Gamma^+} \,\oplus\, V_{\Gamma^+}^+\,\oplus\, V_{\Gamma^-}^-\,\oplus\, I^{v,-}_\Lambda\,\oplus\, W^v_\Lambda,
	\end{equation*}
	where $\ker(D_y + \mu_\Lambda(y)) = T^v_y\scT_\Lambda\oplus \bR^{\Gamma^+}$ because we quotient by $\bR$-translations in the target. Meanwhile, 
	\begin{equation*}
		a_0^*\cW \;\cong\; E_\Lambda\,\oplus\, I^{v,+}_\Lambda\,\oplus\, V_{\Gamma^-}^+\,\oplus\, V_{\Gamma^+}^-\,\oplus\, W^v_\Lambda,
	\end{equation*}
	whose first part $I^{v,+}_\Lambda \coloneqq \ker((\conn^Y\cdot)^{0,1} + \mu_\Lambda+\kappa_\Lambda)$ is the kernel of a complex-linear surjective Cauchy--Riemann operator and thus carries a canonical complex structure.
\end{proof}


\subsubsection{Compatibilities of stable complex structures}  We can now make Theorem~\ref{thm:stable-complex-structure} precise, phrasing the statement in terms of the partitions $\Lambda$ to keep the notation consistent. 

\begin{proposition}\label{prop:existence-stable-complex}
	Possibly after shrinking the thickenings of the morphism spaces of $\scM$, there exists for any pair of objects $\Gamma^-,\Gamma^+$ and any partition $\Lambda\cl \Gamma^-\to\Gamma^+$ an equivalence 
    \begin{equation}
     T\scT_{\Lambda} \,\oplus\, V_{\Gamma^+}\,\oplus\, W_{\Lambda}\,\oplus \bR\;\simeq\; I_{\Lambda}\,\oplus\, V_{\Gamma^-}\,\oplus\, W_{\Lambda}
    \end{equation}
    of virtual vector bundles that is compatible with the symmetric actions. For any factorization $\Gamma^- \xra{\Lambda^0} \Gamma\xra{\Lambda^1}\Gamma^+$ of $\Lambda$ we have an equivariant split embeddings
    \begin{equation}\label{eq:embeddings-index-bundles}
         I_{\Lambda^0}\,\oplus\, I_{\Lambda^1}\;\to \;I_{\Lambda}
    \end{equation}
     \begin{equation}\label{eq:embeddings-stabilising-bundles}
         W_{\Lambda^0}\,\oplus\, W_{\Lambda^1}\;\to \;W_{\Lambda}
    \end{equation}
    over $\scT_{\Lambda^0}\ov{\times}_{B\Gamma}\scT_{\Lambda^1}$ so that the square of a
    \begin{equation*}\begin{tikzcd}
		T\scT_{\Lambda^0}\,\oplus\, V_{\Gamma^-}\,\oplus\, W_{\Lambda^0}\,\oplus\, \bR_{\Lambda^0}\,\oplus T\scT_{\Lambda^1}\,\oplus\, V_{\Gamma}\,\oplus\, W_{\Lambda^1}\,\oplus\, \bR_{\Lambda^1} \arrow[r,""] \arrow[d,""]& T\scT_{\Lambda}\,\oplus  V_{\Gamma^+}\,\oplus\, W_{\Lambda}\,\oplus\, \bR_\Lambda \arrow[d,""]\\ 
        I_{\Lambda^0}\,\oplus\, V_{\Gamma^-}\,\oplus\, W_{\Lambda^0}\,\oplus  I_{\Lambda^1}\,\oplus\, V_{\Gamma}\,\oplus\, W_{\Lambda^1}\arrow[r,""] &  I_{\Lambda}\,\oplus\, V_{\Gamma^-}\,\oplus\, W_{\Lambda} \end{tikzcd} \end{equation*}
    commutes, where $\bR_{\Lambda^1}$ is mapped to $\bR_\Lambda$ and $\bR_{\Lambda^1}$ to the normal bundle of the boundary stratum. Moreover, the restrictions are compatible across the boundary strata of codimension $2$.
\end{proposition}

\begin{proof}[Proof of Proposition~\ref{prop:existence-stable-complex}]
    Recall that $$T\cT_\Lambda = (T\ov{\scB}^\bR_\Lambda\,\oplus\, T^v\scT_\Lambda,E_\Lambda\,\oplus\,\fp\fu_\Lambda),$$
    where $\fp\fu_\Lambda$ is the Lie algebra of $\p{\gamma\in \Gamma^+}{\PU_{d_{\Lambda_\gamma}+1}(\bC)}$.
    We define
    \begin{equation}
        I_\Lambda \coloneqq I^b_\Lambda\,\oplus\,I^v_\Lambda\quad\qquad U_\Lambda \coloneqq (0,\bR) \qquad\quad W_\Lambda \coloneqq \bR^{\Gamma^+}\oplus W^v_\Lambda.
    \end{equation}
    By Lemma~\ref{lem:obtain-isomorphism-from-homotopy} and Lemma~\ref{lem:stable-complex-base}, it suffices to show that we can choose the vector bundles of Proposition~\ref{prop:interpolating-to-complex} compatibly over boundary strata. 
    Due to the decompositions~\eqref{eq:tangent-bundle-decomposition} and~\eqref{eq:obstruction-bundle-decomposition}, we can discuss the compatibility of the stable complex structures on the tangent bundle of the base spaces and the vertical tangent bundles separately. The compatibility for the base spaces was shown in \S\ref{subsec:stable-complex-base}.
    The existence of the split embeddings~\eqref{eq:embeddings-index-bundles} follows from Lemma~\ref{lem:stable-complex-base} and by constructing the bundles $I^{v,-}_{\Lambda}$ in the proof of Proposition~\ref{prop:interpolating-to-complex} inductively as in Step 2 of the proof of Lemma~\ref{lem:embedding-inductive-step}, respectively, of Proposition~\ref{prop:embedding-gkc}. Meanwhile, the existence of the split embeddings $I^{v,+}_{\Lambda^0}\,\oplus\, I^{v,+}_{\Lambda^1}\to I^{v,+}_\Lambda$ follows immediately from the inductive construction of the perturbation spaces. This completes the proof.
\end{proof}

\begin{lemma}\label{lem:obtain-isomorphism-from-homotopy}
    Suppose $\scX$ is a symmetric flow category and $\scY$ a symmetric flow bimodule from $\scX$ to itself, admitting for each $x,y\in \scX$ a fibration $q_{xy}\cl \scY(x,y)\to \scX(x,y)$ with fibers given by the interval $[0,1]$, so that the system $\{q_{xy}\}$ is compatible with the structural maps of the flow bimodule, including the symmetric actions. Suppose also that $q_{x,y}$ admits two boundary sections $a_+,a_-$ so that the squares
    \begin{equation*}\begin{tikzcd}
		\scX(x,y)\,\ov{\times}_{y}\,\scX(y,z) \arrow[d,"a_-\times \ide"] \arrow[r,""]&\scX(x,z)\arrow[d,"a_-"]\\ 
       \scY(x,y)\,\ov{\times}_{y}\,\scX(y,z)  \arrow[r,""]&\scY(x,z) \end{tikzcd} \qquad \qquad \begin{tikzcd}
		\scX(x,y)\,\ov{\times}_{y}\,\scX(y,z) \arrow[d,"\ide\times a_+"] \arrow[r,""]&\scX(x,z)\arrow[d,"a_+"]\\ 
       \scX(x,y)\,\ov{\times}_{y}\,\scY(y,z)  \arrow[r,""]&\scY(x,z) \end{tikzcd} \end{equation*}
    commute. Then, for any vector bundle $\scW\to \scY$ that is compatibly a flow bimodule itself, there exists a contractible space of systems $\{\Phi_{x,y}\}_{x,y}$ of compatible isomorphisms 
    \begin{equation}\label{eq:compatible-isomorphisms}
        \Phi_{xy}\cl a_+^*\scW(x,y)\to a_-^*\scW(x,y)
    \end{equation}
\end{lemma}

\begin{proof}
    Using \cite{Kott}, we inductively choose a system of invariant Riemannian metrics so that the composition maps are isometric. Then, we use the Riemannian metrics to inductively construct countable locally finite open covers $\cU^{xy} = \{B_{r_i}(x_i)\}_i$ so that $\cU^{xz}\cap( \scX(x,y)\ov{\times}-y\scX(y,z))= \cU^{xy}\,\ov{\times}_{y}\,\cU^{yz}$. Inductively, choose trivializations $$\wt U_i := q\inv(U_i)\xra{\psi_i} U_i\times [0,1],$$ so that $\psi_i(a_-(b)) = (b,0)$ and $\psi_i(a_+(b)) = (b,1)$, and $$W|_{\wt U_i}\xra{\wt\psi_i} U_i\times[0,1]\times W_{a_+(x_i)},$$
    covering $\psi_i$ and which are compatible across boundary strata. Inductively, define orderings of the open covers $\cU^{xy}$ so that the inclusions $\cU^{xy}\,\ov{\times}_{y}\,\cU^{yz}\hkra \cU^{xz}$ are order-preserving. Then, the construction in the proof of \cite[Theorem~(14.3.1)]{tD08} carries over to yield the desired compatible isomorphisms.
\end{proof}


