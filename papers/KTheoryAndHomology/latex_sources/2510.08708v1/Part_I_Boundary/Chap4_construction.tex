\subsection{Relative motives and motivic cohomology} 
\label{SS: relative motive} 

In this subsection, we first recall some basic facts about motivic categories. We then use motivic categories to give the definition of motivic cohomology and collect the needed facts about it. Finally, we give the Gysin map of motivic cohomology corresponding to the map of the Shimura varieties that we are interested in. 

\subsubsection{Motivic categories}
Let $\Lambda$ be a commutative ring with a unit and let $\mathrm{DM}_{B,c}(S, \Lambda)$ be the \textit{triangulated category of constructible Beilinson motives over a base excellent scheme $S$ with $\Lambda$-coefficients} defined in \cite[Def.15.1.1]{Cisinski_Deglise}, which is a symmetric monoidal category and we denote by $1_{S}$ the unit of the category, which is the Tate motive \cite[Introduction B, page xxi]{Cisinski_Deglise}. We write $\mathrm{DM}_{B,c}(S)$ for $\mathrm{DM}_{B,c}(S, \QQ)$. 

\begin{fact}
We have the following properties: 
 \begin{itemize}
     \item \textnormal{\cite[Theorem 15.2.4]{Cisinski_Deglise}} The triangulated category $\mathrm{DM}_{B,c}(S)$ has the formalism of Grothendieck's six-functors 
     \begin{equation*}
        (f^{*}, f_{*}, f_{!}, f^{!},\underline{\Hom}, \otimes)
    \end{equation*}
     and dualizing functor $\mathbb{D}$. 
     \item \textnormal{\cite[Remark 11.1.14]{Cisinski_Deglise}} Over a field $F$, we have the following equivalence of triangulated categories: 
           \begin{equation}
           \label{eq: isomorphism Beilinson motive to geometric motive}
               \mathrm{DM}_{B,c}(\Spec (F)) \cong \mathrm{DM}_{gm}(F), 
           \end{equation}
           where the right hand side is the triangulated category of geometric motives \textnormal{\cite[Definition 2.1.1, Page 5]{V00}}. We use the following isomorphism: for any smooth $f \colon X \rightarrow \Spec (F)$, the motive $f_{*}1_{X} \in \mathrm{DM}_{B,c}(\Spec (F))$ is mapped to the dual of the motive $M_{gm}(X) \in \mathrm{DM}_{gm}(F)$ defined in \textnormal{\cite[Definition 2.1.1]{V00}}.
\end{itemize}
 \end{fact}

 \begin{proposition}[{\cite[Proposition 1.7]{PL15}}]
       For a regular immersion $f \colon Y \rightarrow X$ with $\dim X - \dim Y = d$ and any $M \in \mathrm{DM}_{B,c}(S)$, we have the following absolute purity isomorphism: 
       \begin{equation*}
           f^{*}(M) \cong f^{!}M(d)[2d]. 
       \end{equation*}
 \end{proposition}

 \begin{remark}
     It follows from \cite[\href{https://stacks.math.columbia.edu/tag/0E9J}{Tag 0E9J}]{stacks-project} that if $Y$ and $X$ are regular schemes, then $f$ is a regular immersion.
 \end{remark}

 \par Let $\Lambda$ be a field. Corti and Hanamura \cite[Definition 2.9]{Corti-Hanamura00}  defined the category of relative Chow motives $\mathrm{CHM}(S, \Lambda)$ with $\Lambda$-coefficients over a base scheme $S$ generalizing the category of Chow motives over a field. We write $\mathrm{CHM}(S)$ for $\mathrm{CHM}(S, \QQ)$. The relationship between $\mathrm{CHM}(S)$ and $\mathrm{DM}_{B,c}(S)$ proved by Jin is as follows: 
 
 \begin{proposition}[{\cite[Theorem 3.17]{Jin16}}]
 \label{Prop: Jin}
 Let $\mathrm{Chow}(S)$ be the heart of the Chow weight structure on $\mathrm{DM}_{B,c}(S)$ constructed in \textnormal{\cite[Th\'{e}or\`{e}me 3.3.]{Hebert10}}. Then there is a natural functor 
 \[
        F \colon \mathrm{CHM}(S) \rightarrow \mathrm{Chow}(S),
 \]
 which is an equivalence of the two categories $\mathrm{CHM}(S)$ and $\mathrm{Chow}(S)$. 
 \end{proposition}

 \begin{convention}
     This proposition tells us that for any object $V \in \mathrm{CHM}(S)$, $F(V)$ is an object of the category $\mathrm{DM}_{B,c}(S)$. In order to simplify notation, we write $V$ for $F(V)$. 
 \end{convention}

 Finally, we recall the definition of motives of Abelian type which will be used in the proof of vanishing on the boundary for motivic cohomology.
 \begin{definition}[{\cite[Definition 1.1]{Wild_15}}]
     Let $F$ be a perfect field and $\bar{F}$ be its algebraic closure. 
     \begin{enumerate}
         \item  The category of \textit{Chow motives of Abelian type} over $\bar{F}$ is defined as a strict full additive tensor sub-category $\mathrm{CHM}^{Ab}(\bar{F})$ of $\mathrm{CHM}(\bar{F})$ generated by the following: 
         \begin{enumerate}
             \item Shifts of Tate motives $\ZZ(m)[2m]$, for $m \in \ZZ$, 
             \item Chow motives of Abelian varieties over $\bar{F}$. 
         \end{enumerate}
         \item Define the category of \textit{Chow motives of Abelian type over $\bar{F}$} as the (strict) full (additive tensor) sub-category $\mathrm{CHM}^{Ab}(F)$ of $\mathrm{CHM}(F)$ whose base change to $\bar{F}$ lies in $\mathrm{CHM}^{Ab}(\bar{F})$. 
         \item The category $\mathrm{DM}_{gm}^{Ab}(F)$ is defined as the (strict) full triangulated subcategory of $\mathrm{DM}_{gm}(F)$ generated by $\mathrm{CHM}^{Ab}(F)$. 
     \end{enumerate}
 \end{definition}

 \begin{notation}
    We denote by $\mathrm{DM}_{B, c}^{Ab}(\Spec (\QQ))$ the sub-category of $\mathrm{DM}_{B, c}(\Spec (\QQ))$ that is isomorphic to $\mathrm{DM}_{gm}^{Ab}(F)$ via the isomorphim in equation (\ref{eq: isomorphism Beilinson motive to geometric motive}).
 \end{notation}

\subsubsection{The localization exact triangle} We recall the localization exact triangle of Beilinson motives that will be used in the proof of vanishing on the boundary for motivic cohomology. 

\par Let 
\begin{equation*}
    \xymatrix{
        U \ar[r]^{j} & X & \ar[l]_{i} Y 
    }
\end{equation*}
be a diagram in $\mathrm{Sch}(\QQ)$ where $j$ is an open immersion and $i$ is the complementary reduced closed immersion. We assume that $U$ has the same dimension as $X$ and that $Y$ has codimension $c$ in $X$. 

\begin{proposition}[{\cite[Proposition 2.3.3]{Cisinski_Deglise}}]
\label{Prop: motivic localization}
    Let $N$ be an object of $\mathrm{DM}_{B,c}(S)$. We have the following exact triangle
    \begin{equation*}
        \xymatrix{
            j_{!}j^{*}N \ar[r] & N \ar[r] & i_{*}i^{*}N \ar[r] & j_{!}j^{*}N[1]
        }
    \end{equation*}
    in $\mathrm{DM}_{B,c}(S)$. 
\end{proposition}

\subsubsection{Motivic cohomology}
 
 \begin{definition}[Motivic Cohomology] \label{Def: motivic cohomology}
 If $W$ is an object of $\mathrm{DM}_{B,c}(S)$ and ${1}_{S}$ is the unit object of $\mathrm{DM}_{B,c}(S)$, then the \textit{motivic cohomology} is defined as the $\Lambda$-module
 \[
        \mathrm{H}_{\mathrm{M}}^{*}(S,W) = \Hom_{\mathrm{DM}_{B,c}(S)}(1_{S},W[*]).
 \]
 \end{definition}

 \begin{remark}
    \begin{itemize}
        \item It follows from \cite[Corrollary 14.2.14]{Cisinski_Deglise} that this definition is compatible with the K-theoretical one: 
         for any regular scheme $X$, we have a canonical isomorphism
         \[
                \mathrm{H}_{\mathrm{M}}^{q}(X, \QQ(p)) \cong \mathrm{K}_{2p-q}(X)_{\QQ}^{(p)},
         \]
         where the right-hand side is the $p$-th graded piece of the $\gamma$-filtration of an algebraic $\K$-group.
         \item Recall that in \cite[Page 21]{Wild_09}, for any $M \in \mathrm{DM}_{gm}(\QQ)$, the motivic cohomology $\mathrm{H}_{M}^{i}(M, \QQ(j))$ is defined as 
         \begin{equation}
         \label{eq: def of motivic cohomology of motives}
             \mathrm{H}_{M}^{i}(M, \QQ(j)): = \Hom_{\mathrm{DM}_{gm}(\QQ)}(M, \QQ(i)[j]). 
         \end{equation}
         We can see that this definition is compatible with our definition through the isomorphism in equation (\ref{eq: isomorphism Beilinson motive to geometric motive}). 
    \end{itemize}
 \end{remark}
 

\begin{notation}
Let $\mathcal{G}$ temporarily denote any of the three groups $\{\GL_2, H, G \}$, so we have the associated Shimura variety $\Sh_{\calG}$. 
\end{notation}

\begin{lemma}[{\cite[Theorem 8.6]{Ancona15}}]
\label{lemma: Ancona}
Let $F$ be a number field. There is an additive functor 
\[
    \mu_{M} \colon \mathrm{Rep}_{F}(\calG) \rightarrow \mathrm{CHM}(\Sh_{\calG}, F)
\]
from the category of representations of $\calG$ over $F$ to the category of relative Chow motives over $\mathrm{CHM}(\Sh_{\calG}, F)$ with the following properties: 
\begin{enumerate}
\item The functor $\mu_{M}$ preserves tensor product and duals;
\item For the similitude character $\mu \colon \calG \rightarrow \mathbf{G}_{m}$, $\mu_{M}(\mu)$ is the Lefschetz motive 
$\QQ(1)$\footnote{The Lefschetz motive $\QQ(1)$ has weight $-2$, and its image under the Hodge realization functor is $\RR(1): = 2\pi i$. };
\item If $V$ denotes the defining representation of $\mathcal{G}$, then $\mu_{M}(V) = \mathrm{h}^{1}(A)(1)$, where $A$ is the universal $\mathrm{PEL}$ abelian variety over $\Sh_{\mathcal{G}}$;
   
\end{enumerate}
\end{lemma}

\begin{remark} We give four remarks about Lemma \ref{lemma: Ancona}. 
\begin{enumerate}
    \item The construction by Ancona in Lemma \ref{lemma: Ancona} works much more generally for arbitrary PEL Shimura varieties, but in this paper, we shall only need it for the above three groups.
    \item Our normalization is compatible with Ancona's original normalization, but dual to the normalization of \cite[Theorem 8.3.1]{LSZ22} where they send the multiplier representation to $\QQ(-1)$ and the defining representation to $\mathrm{h}^{1}(A)$.
    \item The Ancona functor $\mu_{M}$ is faithful. \cite[REMARK 5.10]{Torzewski19}. 
    \item For any $V \in \mathrm{Rep}_{F}(\calG)$ and any integer $d$, we have $\mu_{M}(V(d)) = \mu_{M}(V)(d)$. Hence, we can use the twist $(d)$ before or after applying ancona functor. 
\end{enumerate}
\end{remark}

\par By applying Torzewski's general theorem \cite[Theorem 9.7]{Torzewski19} in the setting of $G$ and $H$, we get the following result: 
\begin{proposition}[``Branching" for motivic sheaves]
For the map of Shimura varieties $\iota: M \rightarrow S$, we have the following commutative diagram of functors 
\[
    \begin{tikzcd}
        \mathrm{Rep}_{F}(G) \ar[d, "\iota^{*}"] \ar[r, "\mu_{M}"] & \mathrm{CHM}(S, F) \ar[d, "\iota^{*}"] \\
        \mathrm{Rep}_{F}(H) \ar[r, "\mu_{M}"] & \mathrm{CHM}(M, F), 
    \end{tikzcd}
\]
where the left-hand $\iota^{*}$ denotes the restriction of representations, and the right-hand $\iota^{*}$ denotes the pullback of relative Chow motives.

\end{proposition}

\begin{convention}
\begin{itemize}
    \item For any $\mathcal{G}$, we write $\mathrm{Rep}(\mathcal{G})$ for $\mathrm{Rep}_{\QQ}(\mathcal{G})$. For any $V \in \mathrm{Rep}(\mathcal{G})$, we also write $V$ for $\mu_{M}(V)$. 
    \item Recall that we define the elements of $\mathrm{Rep}(G)$ and $\mathrm{Rep}(H)$ as the Weil restriction of elements of $\mathrm{Rep}_{E}(G)$ and $\mathrm{Rep}_{E}(H)$. 
    \item Recall that we always let $M = \Sh_{H}$ and $S = \Sh_{G}$. 
\end{itemize}
\end{convention}

\par From the above discussion, for a representation of $\calG$ denoted as $V$, we get a relative Chow motive $V \in \mathrm{CHM}(\Sh_{\calG})$ over Shimura variety $\Sh_{\calG}$, which can also be viewed as an object of $\mathrm{DM}_{B,c}(\Sh_{\calG})$. So for $V$, we can define the motivic cohomology $\mathrm{H}_{\mathrm{M}}^{*}(\Sh_{\calG}, V(*))$ \footnote{In this paper, in order to apply Beilinson regulator, we define all motivic cohomology groups to be $\QQ$-vector spaces, which is the same as Weil restriction to $\QQ$ of the motivic cohomology groups defined in \cite{LSZ22}.}.
 
 \begin{proposition}[Gysin morphism for motivic cohomology]
 \label{Prop: Gysin for motivic cohomology}
For the map of Shimura varieties $\iota \colon M \rightarrow S$ defined earlier and for the representations $W \in \mathrm{Rep}(H)$ and $V \in \mathrm{Rep}(G)$ satisfying $W \hookrightarrow \iota^{*}V$ in $\mathrm{Rep}(H)$, we have the Gysin morphism 
\[
    \iota_{*} \colon \mathrm{H}^{1}_{\mathrm{M}}(M, W(1)) \rightarrow \mathrm{H}^{3}_{\mathrm{M}}(S, V(2)), 
\]
which is induced by 
\begin{equation}
\label{eq: Gysin morphism in DM}
    (\iota_{*}W(1)[1] \rightarrow V(2)[3]) \in \mathrm{DM}_{B,c}(S). 
\end{equation}
\end{proposition}
\begin{proof}
 From the above discussions, we have a map $W \hookrightarrow \iota^{*}V$ in $\mathrm{DM}_{B,c}(M)$, where $\iota^{*}$ is the pullback functor on the category of Beilinson motives. By the absolute purity isomorphism \cite[Proposition 1.7]{PL15}, the right hand side is $\iota^{*}V \cong \iota^{!}V(1)[2]$. By adjunction, we get the morphism $\iota_{!}W \rightarrow V(1)[2]$ in $\mathrm{DM}_{B,c}(S)$. Since $\iota$ is proper, we have $\iota_{*} = \iota_{!}$. So we get the morphism 
 \begin{equation*}
    \iota_{*}W(1)[1] \rightarrow V(2)[3] \in \mathrm{DM}_{B,c}(S). 
 \end{equation*}
 Then by applying the functor $N \longmapsto \Hom_{\mathrm{DM}_{B,c}(S)}(1_{S}, N)$ to the morphism, we get 
\[
\Hom_{\mathrm{DM}_{B,c}(S)}(1_{S}, \iota_{*}W(1)[1]) \rightarrow \Hom_{\mathrm{DM}_{B,c}(S)}(1_{S}, V(2)[3]). 
\]
The right-hand side is $\mathrm{H}^{3}_{\mathrm{M}}(S, V(2))$, and by adjunction and $\iota^{*} 1_{S} = 1_{M}$, the left-hand side is 
\begin{equation*}
    \Hom_{\mathrm{DM}_{B,c}(M)}(1_{M}, W(1)[1]), 
\end{equation*} which is $\mathrm{H}^{1}_{\mathrm{M}}(M, W(1))$. 
\end{proof}

 \subsection{Mixed Hodge modules and absolute Hodge cohomology}
 \label{SS:MHM and abs Hodge coh}
 
 In this subsection, we first recall some basic facts about mixed Hodge modules, which can be viewed as a relative version of mixed Hodge structures, ``archimedean" analogues of mixed $l{\textendash}\mathrm{adic}$ perverse sheaves, or the Hodge realization of the mixed motivic sheaves which have not been defined. We then use mixed Hodge modules to define absolute Hodge cohomology, which can be seen as the Hodge realization of the motivic cohomology defined in Definition \ref{Def: motivic cohomology}, and collect the needed facts about it. Finally, we give the Gysin map of absolute Hodge cohomology corresponding to the map of the Shimura varieties that we are interested in.  For this part, we mainly follow \cite[\S2.2]{LemmaI15}. 

 \subsubsection{Mixed Hodge modules}
Let $A \subset \RR$ be a subfield and $\mathrm{Sch}(\QQ)$ be the category of quasi-projective $\QQ$-schemes. For $X \in \mathrm{Sch}(\QQ)$, we have the abelian category $\mathrm{MHM}_{A}(X/\RR)$ of real algebraic mixed $A$-Hodge modules \cite[Definition A.2.4]{HW98}.  Let $\mathrm{D}_{c}^{b}(X^{\mathrm{an}}_{\CC}, A)$ be the bounded derived category of sheaves for the analytic topology of $A$-vector spaces with constructible cohomology objects and let $\mathrm{Perv}_{A}(X^{\mathrm{an}}_{\CC}) \subset \mathrm{D}_{c}^{b}(X^{\mathrm{an}}_{\CC}, A)$ be the abelian category of perverse sheaves on $X^{\mathrm{an}}_{\CC}$.   

 \begin{fact}
    We have the following properties: 
    \begin{enumerate}
        \item \textnormal{\cite{Beilinson87}} The natural functor $\mathrm{D}^{b}(\mathrm{Perv}_{A}(X^{\mathrm{an}}_{\CC})) \rightarrow \mathrm{D}_{c}^{b}(X^{\mathrm{an}}_{\CC}, A)$ is an equivalence of categories. 
        \item \textnormal{\cite[Theorem 0.1]{MHM90}} There is a faithful and exact functor 
        \begin{equation*}
            rat \colon \mathrm{MHM}_{A}(X/\RR) \rightarrow \mathrm{Perv}_{A}(X^{\mathrm{an}}_{\CC}). 
        \end{equation*}
        We also use the same notation to denote its derived functor 
         \begin{equation*}
             rat \colon \mathrm{D}^{b}(\mathrm{MHM}_{A}(X/\RR)) \rightarrow \mathrm{D}_{c}^{b}(X^{\mathrm{an}}_{\CC}, A). 
         \end{equation*}
        Also, for $M \in \mathrm{D}^{b}(\mathrm{MHM}_{A}(X/\RR))$, we have $rat(\mathcal{H}^{i}M) =  \prescript{p}{}{\mathcal{H}^{i}}rat(M)$, where $\mathcal{H}^{i}M$ is the standard cohomology functor and $\prescript{p}{}{\mathcal{H}^{i}}$ is the perverse cohomology functor.
        \item \textnormal{\cite{S2}, \cite{MHM90}, {\cite[Theorem A.2.5]{HW98}}} We have six-functors formalism with 
        \begin{equation*}
            (f^{*}, f_{*}, f_{!},f^{!}, \underline{\Hom}, \otimes)
        \end{equation*}
        and the dualizing functor $\mathbb{D}$ on the derived category $\mathrm{D}^{b}(\mathrm{MHM}_{A}(X/\RR))$, which is a symmetric monoidal category. Furthermore, these functors commute with the functor $rat$.
    \end{enumerate}
 \end{fact}

 We now collect some facts about the relationship between variations of Hodge structures and mixed Hodge modules. 
 
\par Assume that $X$ is smooth and pure of dimension $d$. Then for any local system $V$ of $A$-vector spaces on $X_{\CC}^{\mathrm{an}}$, by the definition of perverse sheaves the complex $V[d]$ concentrated in degree $-d$ is an object of $\mathrm{Perv}_{A}(X^{\mathrm{an}}_{\CC})$. 
\par Let $\mathrm{MHM}_{A}(X/\RR)^{s}$ be the full subcategory of $\mathrm{MHM}_{A}(X/\RR)$ whose objects are those $M \in \mathrm{MHM}_{A}(X/\RR)$ such that $rat(M) = V[d]$ for some local system $V$. Denote by $\mathrm{Var}_{A}(X/\RR)$ the category of real admissible polarizable variations of mixed $A$-Hodge structures over $X$\cite[Definition A.2.4 b]{HW98}.

\begin{fact}[{\cite[Definition A.2.4.b]{HW98}}] 
    There is an equivalence of categories
    \[
        \mathrm{Var}_{A}(X/\RR) \cong \mathrm{MHM}_{A}(X/\RR)^{s}.  
    \]
    From this, we have the following equivalence of abelian categories:
    \[
        \mathrm{MHM}_{A}(\Spec(\QQ)/\RR) \cong \mathrm{MHS}_{A}^{+},
    \]
    where the right hand side denotes the abelian category of mixed polarizable $A$-Hodge structures \textnormal{\cite[Lemma  A.2.2]{HW98}}. 
\end{fact}

\begin{convention}
    \begin{enumerate}
        \item 
            We use the convention that a variation of mixed Hodge structure is viewed as a mixed Hodge module but not a shift of mixed Hodge modules, which is the same as the one used in \cite{Bow04}. When $X$ is smooth and pure of dimension $d$, the embedding 
            \begin{equation*}
                \iota \colon \mathrm{Var}_{A}(X/\RR) \rightarrow \mathrm{D}^{\mathrm{b}}(\mathrm{MHM}_{A}(X/\RR))
            \end{equation*}
            is characterized by the fact: for any object $V \in \mathrm{Var}_{A}(X/\RR)$, the complex $rat(\iota(V))[-d]$ is a single non-trivial constituent in degree zero,  which means that it is a local system. When the context is clear, we always write $V$ for $\iota(V)$. 
        \item In order to simplify terminology, when $A = \RR$ we will write ``variation of Hodge structure" instead of ``real admissible polarizable variation of mixed $\RR$-Hodge structure" and write ``mixed Hodge modules" for ``real algebraic mixed $\RR$-Hodge modules". 
    \end{enumerate}
\end{convention}

\begin{proposition}[{\cite[Proposition 2.3.12, Proposition 2.3.14, Lemme 2.3.7]{Bl06}}]
\label{Prop: compare between VHS and MHM (tensor and pullback)}
    Under the embedding
    \begin{equation*}
        \iota \colon \mathrm{Var}_{\RR}(X/\RR) \rightarrow \mathrm{D}^{\mathrm{b}}(\mathrm{MHM}_{\RR}(X)/\RR), 
    \end{equation*}
    we have the following three consequences. 
    \begin{enumerate}
        \item For any $V, W \in \mathrm{Var}_{\RR}(X/\RR)$, we have 
             \begin{equation*}
                 \iota(V \otimes W)[d] = \iota(V) \otimes \iota(W), 
             \end{equation*}
             where the left tensor product is the tensor product in $\mathrm{Var}_{\RR}(X/\RR)$ and the right tensor product is the tensor product in $\mathrm{D}^{b}(\mathrm{MHM}_{\RR}(X)/\RR)$. 
        \item Let $f \colon X \rightarrow Y$ be a morphism between smooth algebraic varieties with $\dim X - \dim Y = d$. For any $V \in \mathrm{Var}_{\RR}(X/\RR)$, we have the identity: 
        \begin{equation*}
            \iota((f^{s})^{*}V) = f^{*}\iota(V)[d], 
        \end{equation*}
        where $(f^{s})^{*}$ is pullback in $\mathrm{Var}_{\RR}(X/\RR)$ and $f^{*}$ is the pullback in $\mathrm{D}^{b}(\mathrm{MHM}_{\RR}(X)/\RR)$. 
        In particular, if $s\colon X \rightarrow \Spec(\QQ)$ is a smooth scheme which is pure of dimension $d$, $A_{X}(n)$ (resp., $A(n)$) is the Tate variation on $X$ (resp., $\Spec (\QQ)$) viewed as an object of $\mathrm{MHM}_{A}(X/\RR)$ (resp., $\mathrm{MHS}_{A}^{+}$), we have the equality $A_{X}(n) = s^{*}A(n)[d]$ in $\mathrm{MHM}_{A}(X/\RR)$. 
        \item For any $V \in \mathrm{Var}_{\RR}(X/\RR)$, the following identity holds: 
        \begin{equation*}
            \mathbb{D}(\iota(V)) = \iota(V^{\vee})(d), 
        \end{equation*}
        where $(\cdot)^{\vee}$ denotes the dualizing functor in $\mathrm{Var}_{\RR}(X/\RR)$ and $\mathbb{D}$ stands for the dualizing functor in $\mathrm{D}^{b}(\mathrm{MHM}_{\RR}(X)/\RR)$. 
    \end{enumerate}
\end{proposition}

\begin{remark}
    From the proposition, we can see that the unit $1_{X}$ in the symmetric monoidal category $\mathrm{D}^{b}(\mathrm{MHM}_{\RR}(X)/\RR)$ is ${A(n)_{X}}[-d]$. 
\end{remark}

\begin{convention}
    When the scheme $X$ is clear from the context, we write $A(n)$ for $A(n)_{X}$.  
\end{convention}

\begin{theorem}[{\cite[Theorem 2]{Saito88}}]
 Let $X \in \mathrm{Sch}(\QQ)$ be smooth and pure of dimension $d$. A variation of Hodge structure of weight $w$ is a mixed Hodge module of weight $w + d$ via the above identification. 
\end{theorem}

Finally, we state the absolute purity theorem for mixed Hodge modules, which will be used in the construction of the Gysin map for absolute Hodge cohomology.

\begin{proposition}[{\cite[Proposition 2.3.16]{Bl06}}]
\label{Prop: absolute purity for MHM}
    Let $f\colon X \rightarrow Y$ be a morphism of pure relative dimension $d$ between smooth algebraic varieties. Then for any $V \in \mathrm{Var}_{\RR}(Y/\RR)$, we have the identity: 
    \begin{equation*}
        f^{!}V = f^{*}V(d)[2d]. 
    \end{equation*}
\end{proposition}

\subsubsection{The localization exact triangle} We recall the localization exact triangle of mixed Hodge modules that will be used in the proof of vanishing on the boundary for absolute Hodge cohomology. 

\par Let 
\begin{equation*}
    \xymatrix{
        U \ar[r]^{j} & X & \ar[l]_{i} Y 
    }
\end{equation*}
be a diagram in $\mathrm{Sch}(\QQ)$ where $j$ is an open immersion and $i$ is the complementary reduced closed immersion. We assume that $U$ has the same dimension as $X$ and that $Y$ has codimension $c$ in $X$. 

\begin{proposition}[{\cite[(4.4.1)]{MHM90}}]
\label{Prop: MHM localization}
    Let $N$ be an object of $\mathrm{D}^{b}(\mathrm{MHM}_{\RR}(X/\RR))$, we have the following exact triangle
    \begin{equation*}
        \xymatrix{
            j_{!}j^{*}N \ar[r] & N \ar[r] & i_{*}i^{*}N \ar[r] & j_{!}j^{*}N[1]
        }
    \end{equation*}
    in $\mathrm{D}^{b}(\mathrm{MHM}_{\RR}(X/\RR))$. 
\end{proposition}

\begin{definition}[Singular (Betti) cohomology]
    \begin{enumerate}
        \item 
        For $X \in \mathrm{Sch}(\QQ)$ with the structure map $s \colon X \rightarrow \Spec(\QQ)$, by \cite[Corollary A.1.7.c]{HW98}, $\mathcal{H}^{i}s_{*}s^{*}A(0)$ is the $i$-th singular cohomology of the underlying topological space of $X_{\CC}^{\mathrm{an}}$ with coefficients in $A$. It also has the mixed Hodge structure constructed by Deligne with the involution induced by the complex conjugation on $X(\CC)$. From this, we get that if $X$ is smooth of pure dimension $d$, and if $A(0)$ denotes the trivial Tate variation of Hodge structures on $X$, then the \textit{$i$-th singular cohomology} is $\mathcal{H}^{i-d}s_{*}A(0)$. 
        \item 
        For any $X$ and any $M \in \mathrm{MHM}_{A}(X/\RR)$, we define the \textit{$i$-th singular cohomology} of $X$ with $M$ as coefficients, $\mathrm{H}^{i}(X, M)$, as the group $\mathcal{H}^{i-d}s_{*}M$; similarly, we define the compactly supported cohomology $\mathrm{H}_{c}^{i}(X,M)$ as $\mathcal{H}^{i-d}s_{!}M$. 
    \end{enumerate}
\end{definition}

\begin{corollary}
\label{Corollary: Localization long exact seq}
    Let $M$ be an object of $\mathrm{D}^{b}(\mathrm{MHM}_{\RR}(U/\RR))$ and assume X is proper. We have the following long exact sequence of mixed Hodge structures
    \begin{equation} \label{eq: les of singular cohomology}
        \cdots \rightarrow \mathrm{H}_{c}^{i}(U,M) \rightarrow \mathrm{H}^{i}(U,M) \rightarrow \mathrm{H}^{i-c}(Y,i^{*}j_{*}M) \rightarrow \mathrm{H}_{c}^{i+1}(U,M)  \rightarrow \cdots.  
    \end{equation}
\end{corollary}
\begin{proof}
    Let $N = j_{*}M$. Then apply the above triangle to get the exact triangle:
    \[
        \xymatrix{
            j_{!}M \ar[r] & j_{*}M \ar[r] & i_{*}i^{*}j_{*}M \ar[r] & j_{!}M[1]
        }. 
    \]
    Let $s\colon X \rightarrow \Spec(\QQ)$ be the structure morphism of $X$ and apply the functor $s_{*}$ to the exact triangle and then take cohomology $\mathcal{H}^{i-d}$, where $d = \dim X$. 
\end{proof}

\subsubsection{Absolute Hodge cohomology}
We now give the definition of absolute Hodge cohomology and construct its Gysin morphism in the case that we are interested in. 

\begin{definition}[Absolute Hodge cohomology]
Let $X$ be the scheme as above. The \textit{$i$-th real absolute Hodge cohomology} $\mathrm{H}^{i}_{H}(X, M)$\footnote{The standard notation for real absolute Hodge cohomology is $\mathrm{H}^{i}_{H}(X/\RR, M)$, which is good to distinguish with complex absolute Hodge cohomology. However, in our paper, we only work with real absolute Hodge cohomology, so we use $\mathrm{H}^{i}_{H}(X, M)$ to denote real absolute Hodge cohomology.} of $X$ with coefficients in $M$ is 
\begin{align*}
       \mathrm{H}^{i}_{H}(X, M) := & \Hom_{\mathrm{D}^{b}(\mathrm{MHM}_{A}(X/\RR))}(A(0)_{X}, M[i]) \\
       = & \Hom_{\mathrm{D}^{b}(\mathrm{MHM}_{A}(X/\RR))}(s^{*}A(0)[d], M[i]) \\
       \cong & \Hom_{\mathrm{D}^{b}(\mathrm{MHS}^{+}_{A})}(A(0), s_{*}M[i-d]),     
\end{align*}
where the second isomorphism comes from adjunction. 
\end{definition}

\begin{remark}
\begin{enumerate} 
    \item This definition is the ``archimedean" analogue of the definition of motivic cohomology given in Definition \ref{Def: motivic cohomology}. 
    \item If the abelian category $\mathrm{MHS}_{\RR}^{+}$ has cohomological dimension $1$ \cite[Corollary 1.10]{Bei83}, then by the Leray spectral sequence, we have the short exact sequence 
    \begin{equation}
    \label{Eq: ses for abs Hodge cohomology}
    0 \rightarrow \Ext^{1}_{\mathrm{MHS}_{\RR}^{+}}(\RR(0), \mathrm{H}^{i-1}(X,M)) \rightarrow \mathrm{H}_{H}^{i}(X,M) \rightarrow \Hom_{\mathrm{MHS}_{\RR}^{+}}(\RR(0), \mathrm{H}^{i}(X,M)) \rightarrow 0
    \end{equation}
    for all $i$ and all $M \in \mathrm{D}^{b}(\mathrm{MHM}_{\RR}(X/\RR))$. 
\end{enumerate}
\end{remark}

Now we come to the setting of Shimura varieties. 
\begin{notation}
\begin{itemize}
  \item 
    Recall that there is a $\RR$-linear tensor functor associating an algebraic representation of the group underlying a given Shimura variety to the corresponding variation of Hodge structure on the  Shimura variety, and we denote the functor as $\mu_{H}$ (see \cite[2]{Bow04}). 
  \item Similar to the last section, we let $\mathcal{G}$ temporarily denote any of the three groups $\{\GL_2,   H, G \}$, and we have the associated Shimura variety $\Sh_{\calG}$. So if $V \in \mathrm{Rep}_{\QQ}(\mathcal{G})$\footnote{For $H$ and $G$, the meaning of $V \in \mathrm{Rep}_{\QQ}(G)$ (reps. $V \in \mathrm{Rep}_{\QQ}(H)$) is that there exists some $V^{\prime} \in \mathrm{Rep}_{E}(G)$ (resp., $V^{\prime} \in \mathrm{Rep}_{E}(H)$) such that  $V = \Res_{E/\QQ} V^{\prime}$. }, we have the variation of Hodge stucture $\mu_{H}(V) \in \mathrm{Var}_{\RR}(\Sh_{\mathcal{G}})$. 
  \item Compose with the embedding $\mathrm{Var}_{\RR}(\Sh_{\mathcal{G}}/\RR) \rightarrow \mathrm{D}^{b}(\mathrm{MHM}_{\RR}(\Sh_{\mathcal{G}})/\RR)$, we can view $\mu_{H}(V)$ as an object of $\mathrm{D}^{b}(\mathrm{MHM}_{\RR}(\Sh_{\mathcal{G}}/\RR))$ which has a single non-trivial constituent in degree $0$. We will write $V$ instead of $\mu_{H}(V)$ when there is no confusion. In other words, we can view $V \in \mathrm{Rep}_{\QQ}(\mathcal{G})$ as an object of $\mathrm{D}^{b}(\mathrm{MHM}_{\RR}(\Sh_{\mathcal{G}}/\RR))$, and we can then define the absolute Hodge cohomology $\mathrm{H}_{H}^{i}(\Sh_{\mathcal{G}}, V)$. 
  \item Recall that we always let $M = \Sh_{H}$ and $S = \Sh_{G}$. 
\end{itemize}
\end{notation}

\begin{remark}
    Unlike the motivic setting, the ``branching'' for variation of Hodge structrues is direct from the definition: 
    for the map of Shimura varieties $\iota \colon M \rightarrow S$, we have the following commutative diagram of functors 
    \[
        \begin{tikzcd}
            \mathrm{Rep}(G) \ar[d, "\iota^{*}"] \ar[r, "\mu_{H}"] & \mathrm{Var}_{\RR}(S) \ar[d, "\iota^{*}"] \\
            \mathrm{Rep}(H) \ar[r, "\mu_{H}"] & \mathrm{Var}_{\RR}(M), 
        \end{tikzcd}
    \]
where the left-hand $\iota^{*}$ denotes the restriction of representations, and the right-hand $\iota^{*}$ denotes the pullback of variation of Hodge structures.
\end{remark}

Just as with motivic cohomolgy, we have the following Gysin morphism. 

\begin{proposition}[Gysin morphism for absolute Hodge cohomology]
\label{Prop: Gysin for abs Hdg cohomology}
    For the map of Shimura varieties $\iota \colon M \rightarrow S$ defined as before and for the representations $W \in \mathrm{Rep}_{\QQ}(H)$ and $V \in \mathrm{Rep}_{\QQ}(G)$ satisfying $W \hookrightarrow \iota^{*}V$ in $\mathrm{Rep}_{\QQ}(H)$, we have the Gysin morphism 
    \[
    \iota_{*} \colon \mathrm{H}^{1}_{H}(M, W(1)) \rightarrow \mathrm{H}^{3}_{H}(S, V(2)), 
    \]
    which is induced by 
    \begin{equation}
    \label{eq: Gysin morphism in derived category of MHM}
        (\iota_{*}W(1) \rightarrow V(2)[1]) \in \mathrm{D}^{b}(\mathrm{MHM}_{\RR}(S/\RR)). 
    \end{equation}
\end{proposition}
\begin{proof}
From the above discussions, we have a map $W \hookrightarrow (\iota^{s})^{*}V$ in $\mathrm{D}^{b}(\mathrm{MHM}_{\RR}(M)/\RR)$, where $\iota^{*}$ is the pullback functor for $\mathrm{Var}_{\RR}(\Sh_{\mathcal{G}})$. It follows from Proposition \ref{Prop: compare between VHS and MHM (tensor and pullback)} that $(\iota^{s})^{*}V = \iota^{*}V[-1]$,  where $\iota^{*}$ is the pullback functor for $\mathrm{D}^{b}(\mathrm{MHM}_{\RR}(S/\RR))$. By the absolute purity isomorphism (see Proposition \ref{Prop: absolute purity for MHM}), the right-hand side is $\iota^{*}V \cong \iota^{!}V(1)[2]$. By adjunction, we get the morphism $\iota_{!}W \rightarrow V(1)[1]$ in $\mathrm{D}^{\mathrm{b}}(\mathrm{MHM}_{\RR}(S/\RR))$. Since $\iota$ is proper, we have $\iota_{*} = \iota_{!}$, and we get the morphism 
\begin{equation*}
    (\iota_{*}W(1) \rightarrow V(2)[1]) \in \mathrm{D}^{b}(\mathrm{MHM}_{\RR}(S/\RR)). 
\end{equation*}
Finally, applying the functor $N \mapsto \Hom_{\mathrm{D}^{b}(\mathrm{MHM}(S/\RR))}(\RR(0)_{S}, N[2])$ to the map, we get the Gysin map $\mathrm{H}^{1}_{H}(M, W(1)) \rightarrow \mathrm{H}^{3}_{H}(S, V(2))$. 
\end{proof}

 \subsection{Beilinson regulator and functoriality}
 \label{SS: Beilinson regulator}
 In this subsection, we give the definition of Beilinson regulator using the Hodge realization functor for smooth quasi-projective varieties\footnote{This level of generality is enough for our applications.} over $\QQ$ and state its functorial properties.

 \begin{definition}[Hodge realization]
     \label{def: Hodge realization}
      For a smooth quasi-projective variety $S$ over $\QQ$, let $S_{\CC}$ and $S_{\RR}$ be the base change of $S$ to $\CC$ and $\RR$. The \textit{Hodge realization functor} is defined as the composition of the following functors: 
      \[
            \begin{tikzcd}
                R_{H} \colon \mathrm{DM}_{B,c}(S) \ar[r, "f_{1}^{*}"] &  \mathrm{DM}_{B,c}(S_{\RR}) \ar[r, "f_{2}^{*}"] & \mathrm{DM}_{B,c}(S_{\CC}) \ar[r, "R_{H}^{Bou}"] & \mathrm{D}(\mathrm{MHM}(S_{\CC})), 
            \end{tikzcd}
      \]
      where $f_{1} \colon S_{\RR} \rightarrow S$ and $f_{2} \colon S_{\CC} \rightarrow S_{\RR}$ are the canonical map and the functor $R_{H}^{Bou}$ is the Hodge realization functor from the triangulated category of constructible Beilinson motives $\mathrm{DM}_{B,c}(S_{\CC})$ over $S_{\CC}$ to the (unbounded) derived category $\mathrm{D}(\mathrm{MHM}(S_{\CC}))$ of algebraic mixed Hodge modules over $S_{\CC}$ defined in \cite[Definition 169, Theorem 43]{Bouli22}.
      \par It follows from the \'{e}tale descent property of constructible Beilinson motives (see \cite[Theorem 14.3.4]{Cisinski_Deglise}), the definition of the cateogory $\mathrm{MHM}_{\RR}(S/\RR)$ (see \cite[Definition A.2.4]{HW98}), and the functoriality of $R_{H}$ (see \cite[Theorem 43 (ii0)]{Bouli22}) that the image of $R_{H}$ lies in $\mathrm{D}^{b}(\mathrm{MHM}_{\RR}(S/\RR))$. Hence, we get the covariant Hodge realization functor 
      \begin{equation*}
          R_{H} \colon \mathrm{DM}_{B,c}(S) \rightarrow \mathrm{D}(\mathrm{MHM}_{\RR}(S/\RR)). 
      \end{equation*}
 \end{definition}

 \begin{remark}
     \begin{itemize}
         \item As the functor $R_{H}^{Bou}$ (see \cite[Page 1]{Bouli22}) is an extension of the functor 
               \[
                    \begin{tikzcd}[row sep = 0]
                        \mathrm{Variety} / S_{\CC} \ar[r] & \mathrm{D}(\mathrm{MHM}(S_{\CC})) \\
                            (f \colon X \mapsto S) \ar[r, mapsto] & f_{*} \RR_{X}, 
                    \end{tikzcd}
               \]
               where $\mathrm{Variety} / S_{\CC}$ is the category of $\CC$-varieties over $S_{\CC}$, we see that when we restrict $R_{H}^{Bou}$ to the sub-category $\mathrm{CHM}(S_{\CC})$, the image lies in $\mathrm{D}^{b}(\mathrm{MHM}(S_{\CC}))$. 
         \item We normalize the realization functor such that for any $V$ in the category of relative Chow motives $\mathrm{CHM}(S)$ over $S$ with $\dim S = d$, we have 
         \begin{equation*}
             R_{H}(V) = V[-d] \in \mathrm{D}^{\mathrm{b}}(\mathrm{MHM}_{\RR}(S/\RR)). 
         \end{equation*}
               
     \end{itemize}
 \end{remark}

 \begin{definition}[Beilinson regulator]
 \label{def: Beilinson regulator}
     For a smooth quasi-projective variety $S$ over $\QQ$, $\{i, j\} \in \ZZ$ and $W \in \mathrm{DM}_{B,c}(S)$, the \textit{Beilinson regulator} $r_H$ is defined as follows: 
     \begin{align*}
         & r_{H} \colon \mathrm{H}_{\mathrm{M}}^{i}(S,W(j)) = \Hom_{\mathrm{DM}_{B,c}(S)}(1_{S},W(j)[i]) \\ 
         & \rightarrow \mathrm{H}^{i}_{H}(S, W(j)) = \Hom_{\mathrm{D}^{b}(\mathrm{MHM}_{\RR}(S/\RR))}(R_{H}(1_{S}), R_H(W)(j)[i]),
     \end{align*}
     where $R_{H}$ is the Hodge realization functor defined above. 
 \end{definition}

 \begin{remark}
     \begin{itemize}
         \item When $\dim S = d$, we have 
         \begin{equation*}
             R_{H}(1_{S}) = \RR(0)_{S}[-d], 
         \end{equation*}
         which is the unit of $\mathrm{D}^{b}(\mathrm{MHM}_{\RR}(S/\RR))$. 
     \end{itemize}
 \end{remark}

The following proposition tells us that our definition is compatible with Beilinson's original definition (see \cite[\S 2]{Beilinson_conjecture_original}). 
 \begin{proposition}
 \label{Prop: compatible with Beilinson regulator}
 When $W = \QQ$, if we choose a canonical isomorphism 
 \begin{equation*}
     \mathrm{H}_{\mathrm{M}}^{i}(S, \QQ(j)) \cong \mathrm{K}_{2j-i}(S)_{\QQ}^{(j)},
 \end{equation*}
 then the regulator map 
 \begin{equation*}
     r_{H} \colon \mathrm{H}_{\mathrm{M}}^{i}(S,\QQ(j)) \rightarrow  \mathrm{H}^{i}_{H}(S, \RR(j))
 \end{equation*}
 is the same as the original definition of Beilinson in \textnormal{\cite[\S 2]{Beilinson_conjecture_original}}. 
     
 \end{proposition}

 \begin{proof}
     It follows from the functoriality of $R_{H}$ that the Beilinson regulator can be defined as 
     \begin{align*}
         & r_{H} \colon \mathrm{H}_{\mathrm{M}}^{i}(S,\QQ(j)) = \Hom_{\mathrm{DM}_{B, c}(\Spec (\QQ))}(1_{\Spec (\QQ)}, a_{*} 1_{S} (j)[i]) \\ 
         & \rightarrow \mathrm{H}^{i}_{H}(S, \RR(j)) = \Hom_{\mathrm{D}^{\mathrm{b}}(\mathrm{MHS}^{+}_{\RR})}(R_{H}(1_{\Spec (\QQ)}), R_H(a_{*} 1_{S})(j)[i]),
     \end{align*}
     where $a \colon S \rightarrow \Spec (\QQ)$ is the structure morphism of $S$. It can be seen from the definition of Huber's contravariant Hodge realization functor $R_{H}^{Hu} \colon \mathrm{DM}_{gm}(\QQ) \rightarrow \mathrm{D}^{\mathrm{b}}(\mathrm{MHS}^{+}_{\RR
     })$ (see \cite[Theorem 2.3.3]{Huber00} and \cite[Theorem B.2.2]{Huber04}) that in this case $r_{H}$ is the same as 
     \begin{align*}
         & r_{H}^{Hu} \colon \mathrm{H}_{\mathrm{M}}^{i}(S,\QQ(j)) \cong \Hom_{\mathrm{DM}_{gm}(\QQ)}((a_{*} 1_{S})^{*}, 1_{\Spec (\QQ)} (j)[i]) \\ 
         & \rightarrow \mathrm{H}^{i}_{H}(S, \RR(j)) = \Hom_{\mathrm{D}^{\mathrm{b}}(\mathrm{MHS}^{+}_{\RR})}(R_{H}^{Hu}(1_{\Spec (\QQ)}), R_H^{Hu}(a_{*} 1_{S})(j)[i]),
     \end{align*}
     where the notation $(\cdot)^{*}$ means taking dual. 
     Finally, we have the commutative diagram 
     \[
        \begin{tikzcd}
            & \mathrm{K}_{2j - i}(S) \ar[d, "p"] \ar[rd,"r_{H}^{B}"] &  \\
          \mathrm{H}_{\mathrm{M}}^{i}(S,\QQ(j))  \ar[ru, hook] \ar[r, "id"] & \mathrm{H}_{\mathrm{M}}^{i}(S,\QQ(j)) \ar[r, "r_{H}^{Hu}"] &  \mathrm{H}^{i}_{H}(S, \RR(j)), 
        \end{tikzcd}
     \]
     where $r_{H}^{B}$ is Beilinson's original definition of his regulator in \textnormal{\cite[\S 2]{Beilinson_conjecture_original}}, and the map $p$ is defined in \cite[Corollary 4.2.2]{Huber00}. The commutativity of the right triangle follows from \cite[Definition 18.2.6, Proposition 18.2.8]{Huberbook} and \cite[Corollary 4.2.3]{Huber04} and the commutativity of the left triangle comes from the canonical choice of $\mathrm{H}_{\mathrm{M}}^{i}(S, \QQ(j)) \cong \mathrm{K}_{2j-i}(S)_{\QQ}^{(j)}$. Thus, the proposition is proved.
 \end{proof}

 By the definition of the Beilinson regulator and functoriality of the Hodge realization functor, we have the following functorial properties of the Beilinson regulator. 

 \begin{proposition}
 \label{Prop: functoriality of Beilinson regulator}
     \begin{enumerate}
         \item For the morphism of Shimura varieties $p \colon M \rightarrow \Sh_{\GL_2}$ induced by the projection $H \rightarrow \GL_2$, we have the commutative diagram. 
               \[
                     \begin{tikzcd}
                        \mathrm{H}^{1}_{M}(\Sh_{\GL_2}, \Sym^{n} V_{2}(1)) \ar[r, "p^{*}"] \ar[d, "r_H"] & \mathrm{H}^{1}_{M}(M, W(1)) \ar[d, "r_H"] \\
                         \mathrm{H}^{1}_{H}(\Sh_{\GL_2}, \Sym^{n} V_{2}(1)) \ar[r, "p^{*}"]  & \mathrm{H}^{1}_{H}(M, W(1))  
                     \end{tikzcd}, 
               \]
               where $p^{*}$ is the pullback of the corresponding cohomomlogy theory. 
         \item  For the closed immersion of Shimura varieties $\iota \colon M  \rightarrow S$ and $V, W$ as in Proposition \ref{Prop: Gysin for motivic cohomology} and Proposition \ref{Prop: Gysin for abs Hdg cohomology}, we have the following commutative diagram:  \[
                        \xymatrix{
                            \mathrm{H}^{i}_{M}(M, W(1)) \ar[r]^{\iota_{*}} \ar[d]_{r_H} & \mathrm{H}^{3}_{M}(S, V(2)) \ar[d]_{r_H} \\
                            \mathrm{H}^{1}_{H}(M, W(1)) \ar[r]^{\iota_{*}}  & \mathrm{H}^{3}_{H}(S, V(2))  \\
                        }.  
                    \]
     \end{enumerate}
 \end{proposition}
   
\subsection{Motivic classes and their Hodge realizations} 
\label{SS: motivic class}
 In this subsection, we give the constructions of the motivic classes that we are interested in and compute their Hodge realizations. 

 \subsubsection{Eisenstein symbols}
 \label{SSS: Eisenstein symbol}
 We first recall the construction of Eisenstein symbols. 

 Recall that the \textit{Eisenstein symbol} \cite[\S3]{Beilinson_Modular_Curve} is a $\QQ$-linear map 
 \begin{equation*}
     Eis^{n}_{M} \colon \mathcal{B}_{n} \longrightarrow \mathrm{H}_{M}^{n + 1}(E^n, \QQ(n + 1))
 \end{equation*}
where $\mathrm{H}_{M}^{n + 1}(E^n, \QQ(n + 1))$ is the direct limit over compact open subgroups $K \subset \GL_{2}(\AAA_f)$ of the motivic cohomology $\mathrm{H}_{M}^{n + 1}(E^n_K, \QQ(n + 1))$ of the $n$-th fiber product of the universal elliptic curve $E_K / M_K$ over the modular curve of level $K$. 

\begin{definition}
\label{Def: source Eisenstein symbol}
The notation $\mathcal{B}_{n}$ stands for the space of locally constant $\QQ$-valued functions $\phi_f$ on $\GL_2(\AAA_f)$ statisfying the following conditions: 
\begin{enumerate}
    \item for all $a, d \in \QQ$ such that $ad > 0$ and for all $b \in \AAA_f$, we have 
          \begin{equation}
          \label{eqn:Eis_Symbol_(1)}
              \phi_f( \begin{pmatrix}
                        a & b \\
                        0  & d \\
                      \end{pmatrix} g) = a^{-1} d^{n+1} \phi_f(g), 
          \end{equation}
    \item \begin{equation*}
             \phi_f(\begin{pmatrix}
                        1 &  0 \\
                        0   & -1 \\
                    \end{pmatrix}g) = \phi_f(g), 
          \end{equation*}
     \item for all $k \in \hat{\ZZ}^{\times}$, we have 
           \begin{equation*}
               \phi_{f}(\begin{pmatrix}
                            1 & 0\\
                            0  & k\\ 
                        \end{pmatrix}g) = \phi_f(g). 
           \end{equation*}
\end{enumerate}
\end{definition}

\begin{remark}
    \begin{itemize}
    \item 
        The source of the Eisenstein symbol can identified with a space of $\QQ$-valued functions $\mathcal{F}^{n}$ on $\GL_2(\AAA_f)$ in \cite[2.1.2, P7]{Beilinson_Modular_Curve} by the following map
        \begin{equation*}
            \Psi_f \in \mathcal{F}^{n} \mapsto (\phi_f \colon g \mapsto \phi_f(g) = \Psi_f(^{t}g)) \in \mathcal{B}^{n}. 
        \end{equation*}
    \item At present, many of the results relating special value of $L$-functions to regulators rely on Eisenstein symbols (\cite{Beilinson_Modular_Curve}, \cite{Deninger89}, \cite{Deninger90}, \cite{Kings98}, \cite{LemmaI15}, \cite{LemmaII17}, \cite{Kato04}, ...).
    \end{itemize}
\end{remark}

\begin{notation}
    By the definition of the motivic sheaf $\Sym V_2(1)$, the motivic cohomology $\mathrm{H}_{M}^{1}(M, \Sym^{n} V_{2}(1))$ is a direct summand of $\mathrm{H}_{M}^{n + 1}(E^n, \QQ(n + 1))$. Moreover, the Eisenstein symbol map factors through the inclusion $\mathrm{H}_{M}^{1}(M, \Sym^{n} V_{2}(1)) \subset \mathrm{H}_{M}^{n + 1}(E^n, \QQ(n + 1))$ and we will also use the notation $Eis_{M}^{n}$ to denote the map
    \begin{equation*}
        Eis_{M}^{n} \colon \mathcal{B}_{n} \longrightarrow \mathrm{H}_{M}^{1}(M, \Sym^{n} V_{2}(1)). 
    \end{equation*}
\end{notation}

Let $\nu$ be a finite-order Hecke character, so its archimedean component $\nu_{\infty}$ is a unitary character. 

\begin{definition}
\label{Def: In(v)}
Let $I_{n}(\nu)$ be the space of locally constant $\bar{\QQ}$-valued functions $f$ such that for all $a, d \in \QQ_{+}^{\times}$, $\alpha, \delta \in \hat{\ZZ}^{\times}$, $b \in \AAA_f$ and $g \in \GL_2(\AAA_f)$, 
          the function $f$ satisfies that 
          \begin{equation}
          \label{eq: In(v) action}
              f(\begin{pmatrix}
                  a\alpha & b \\ 
                          & d\delta \\
                \end{pmatrix}g) = a^{-1}d^{n + 1}\nu(\alpha)f(g), 
          \end{equation}
Thus, the space $I_n(\nu)$ is endowed with the action of $\GL_2(\AAA_f)$ by right translation. 
\end{definition}

\begin{lemma}
\label{Lemma: decomposition of source of Eisenstein symbol}
    There is a $\GL_2(\AAA_f)$-equivariant decomposition 
    \begin{equation*}
        \mathcal{B}_{n, \bar{\QQ}} = \bigoplus_{\nu, \mathrm{sgn}(\nu) = (-1)^{n}} I_n(\nu),
    \end{equation*}
    where the direct sum is indexed by all finite-order Hecke characters of sign $(-1)^{n}$. 
\end{lemma}

\begin{proof}
    The proof follows from the proof of \cite[Lemma 4.3]{LemmaII17}. Let $T_2$ be the diagonal maximal torus of $\GL_2$. We are interested in the action of $T_2(\AAA_f)$ on $\mathcal{B}_{n, \bar{\QQ}}$ by left translation. It follows from the decomposition $\AAA_f^{\times} = \QQ^{\times}_{+} \hat{\ZZ}^{\times}$ and equation (\ref{eqn:Eis_Symbol_(1)}) that we only need to consider the action of $T_2(\hat{\ZZ}^{\times})$. Since each function in $\mathcal{B}_{n, \bar{\QQ}}$ is locally constant, it is fixed by an compact open subgroup of $\GL_2(\hat{\ZZ}^{\times})$. Hence, $\mathcal{B}_{n, \bar{\QQ}}$ can be decomposed into a direct sum of finite-dimensional $\bar{\QQ}$-vector spaces that are stable under the action of $T_{2}(\hat{\ZZ}^{\times})$. For each of the finite-dimensional vector spaces $V$, $V$ is a direct sum of finite order Hecke characters
    \begin{equation*}
        \chi \colon \begin{pmatrix}
                \alpha & 0 \\
                0 & \delta  
              \end{pmatrix} \mapsto \nu(\alpha) \chi(\delta)
    \end{equation*}
    because the action is continuous.
    By equation (\ref{eq: In(v) action}), we can see that $\chi$ is trivial. Finally, if we plug $\begin{pmatrix}
                                                                                                -1 & 0 \\
                                                                                                0 & -1 
                                                                                           \end{pmatrix}$                                                      into  equation (\ref{eqn:Eis_Symbol_(1)}), then we get that $\nu$ has to have sign $(-1)^{n}$. 
\end{proof}

\subsubsection{Motivic classes and their image under the Beilinson regulator}
\label{SSS: motivic classes and Hodge}
Using the Eisenstein symbol, we will give the construction of  motivic classes, which has been done in \cite[Definition 9.2.3]{LSZ22} and compute 
their Hodge realizations. 

\begin{construction}
\label{Construction: motivic classes}
    We compose the Eisenstein symbol with the pullback 
    \begin{equation*}
    p^{*} \colon \mathrm{H}^{1}_{M}(\Sh_{\GL_2},\mathrm{Sym}^{n} V_2(1)) \rightarrow \mathrm{H}^{1}_{M}(M, W(1))
    \end{equation*}
    and the Gysin map 
    \begin{equation*}
        \iota_{*} \colon \mathrm{H}^{1}_{M}(M, W(1)) \rightarrow \mathrm{H}^{3}_{M}(S, V(2))
    \end{equation*}
    to get the map 
    \begin{equation*}
        \mathcal{E}is^{n}_{M} \colon \mathcal{B}_{n} \rightarrow \mathrm{H}^{3}_{M}(S, V(2)). 
    \end{equation*}
    Hence, for any $\phi_{f} \in \mathcal{B}_{n}$, we a have motivic class $\mathcal{E}is_{M}^{n}(\phi_f) \in \mathrm{H}^{3}_{M}(S, V(2))$. In summary, we have the following diagram\footnote{Our construction is the Weil restriction to $\QQ$ of the construction over $E$ in \cite[Definition 9.2.3.]{LSZ22} and it is more convenient for Beilinson's conjectures.}: 
    \[
        \begin{tikzcd}[row sep=0pt]
            \mathcal{B}_n \ar[r, "Eis^n_M"] & \mathrm{H}^{1}_{M}(\Sh_{\GL_2}, \mathrm{Sym}^{n}V_2(1)) \ar[r, "p^{*}"] & \mathrm{H}^{1}_{M}(M, W(1)) \ar[r, "\iota_{*}"] & \mathrm{H}^{3}_{M}(S,V(2)) \\
            \phi_f \ar[r, mapsto] & Eis^n_M(\phi_f) \ar[rr, mapsto] & & \mathcal{E}is^{n}_{M}(\phi_f) 
        \end{tikzcd}. 
    \]    
\end{construction}

By Proposition \ref{Prop: functoriality of Beilinson regulator}, we have the following commutative diagram: 
\[
    \begin{tikzcd}
        \mathcal{B}_n \ar[r, "Eis^n_M"] \ar[d, "r_{H}"] & \mathrm{H}^{1}_{M}(\Sh_{\GL_2}, \mathrm{Sym}^{n}V_2(1)) \ar[r, "p^{*}_{M}"] \ar[d, "r_{H}"] & \mathrm{H}^{1}_{M}(M, W(1)) \ar[r, "\iota_{M, *}"] \ar[d,"r_{H}"] & \mathrm{H}^{3}_{M}(S,V(2)) \ar[d, "r_{H}"] \\
        \mathcal{B}_{n, \RR} \ar[r, "Eis^n_H"]  & \mathrm{H}^{1}_{H}(\Sh_{\GL_2}, \mathrm{Sym}^{n}V_2(1)) \ar[r, "p^{*}_{H}"] & \mathrm{H}^{1}_{H}(M, W(1)) \ar[r, "\iota_{H, *}"]  &  \mathrm{H}^{3}_{H}(S,V(2))
    \end{tikzcd}, 
\]
where we use the subscript $M$ and $H$ to distinguish maps of motivic cohomology and absolute Hodge cohomology. 
Hence, the image of the motivic classes $\mathcal{E}is_{M}^{n}(\phi_f)$ under the Beilinson regulator $r_{H}$ is 
\begin{align*}
    \mathcal{E}is_{H}^{n}(\phi_f) &= r_{H} (\iota_{M, *} \circ p_{M}^{*} \circ Eis_{M}^{n}(\phi_{f})) \\
                                  &= \iota_{H, *} \circ p_{H}^{*} \circ Eis_{H}^{n}(r_{H} (\phi_{f})). 
\end{align*}    

\begin{convention}
    In the rest of the paper, when the context is clear, we will ignore the subscript $M$ and $H$, and only write $\iota_{*}$, $p^{*}$ and $Eis^{n}$. 
\end{convention}



