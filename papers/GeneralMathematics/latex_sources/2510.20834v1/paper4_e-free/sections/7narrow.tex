\newpage

%=====================================================================
% 7. Narrow cascade (revised version)
%=====================================================================
\section{Narrow cascade}\label{sec:narrow}

In the “narrow’’ regime the spectrum of the solution is concentrated in a thin cluster of caps
whose angular density is substantially higher than average. The method goes back to Bennett–Carbery–Tao and Guth; here it is adapted to the density threshold \(\sim D\) and a double \(7/8\) rescaling. The \emph{narrow} regime is used \emph{mutually exclusively} with the \emph{Robust\,Kakeya} block (§\ref{sec:kakeya}).

%--------------------------------------------------------------------
\subsection{Problem setup}\label{subsec:narrow-setup}
%--------------------------------------------------------------------

\paragraph{Trigger for the narrow regime.}
We say we are in the \emph{narrow regime} if in a given sextuple of frequencies
at least five lie in a single $O(\alpha)$–cluster. Let $G$
be the corresponding family of caps and set
\[
  F_{\mathrm{narrow}} := \sum_{\Theta\in G} F_\Theta.
\]
This block is used \emph{mutually exclusively} with Robust–Kakeya:
if $G$ satisfies the angular density threshold
\eqref{eq:kakeya-density} (see §\ref{subsec:kakeya-density}),
then §\,\ref{sec:kakeya} is activated; the \emph{narrow cascade} is applied only
in case this threshold is \emph{violated} while there is a cluster of $\ge5/6$.

Set
\[
  r=\lambda^{-2/3}, \qquad
  D=\lambda^{1/12}, \qquad
  \alpha = c_{0}\,r\,D^{1/2}=c_{0}\,\lambda^{-5/8},
  \quad 0<c_{0}\ll1 .
\]
Select a family \(G\subset\{\Theta\}\) of caps of radius \(r\) that \emph{does not}
satisfy the high-density condition, i.e.
\[
 \max_{\Theta\in G}\#\{\Theta'\in G:\ \angle(\xi_\Theta,\xi_{\Theta'})\le\alpha\}
 \ \le\ c_*D ,
\]
where \(c_{*}\in(0,1)\) is a universal constant. For such a family, and
in the presence of a cluster consisting of $\ge5/6$ frequencies, the narrow cascade is initiated.
Define
\[
  \operatorname{supp}_{\xi}\widehat{F_{\mathrm{narrow}}}
    \subset \bigcup_{\Theta\in G}\Theta .
\]

%--------------------------------------------------------------------
\subsection{Double $7/8$ rescaling}\label{subsec:narrow-iter}
%--------------------------------------------------------------------
Perform two iterations
\[
  \lambda_{0}=\lambda,\qquad
  \lambda_{1}=\lambda^{7/8},\qquad
  \lambda_{2}=\lambda^{49/64}.
\]

\paragraph{Angular expansion.}
After the $j$-th step,
\( r_{j}=\lambda_{j}^{-2/3}=\lambda^{-\frac23(7/8)^{j}} \).
Comparing the areas \(r_j^{2}\) and \(\alpha^{2}\sim\lambda^{-5/4}\), we get
\[
  \log_{\lambda}\!\frac{r_{1}^{2}}{\alpha^{2}}=\frac{1}{12},
  \qquad
  \log_{\lambda}\!\frac{r_{2}^{2}}{\alpha^{2}}=\frac{11}{48}>0 .
\]
Thus already the first step removes the flow from the “narrow’’ regime; the second provides a reserve \(\lambda^{\,11/48}\) to absorb log losses.

\smallskip
\noindent\textit{Caveat.}
In the comparisons of this subsection, $\alpha$ is considered fixed at the \emph{initial} scale~$\lambda$;
departure from the narrow regime is understood in terms of clustering with the threshold~$\alpha$ from \S\ref{subsec:narrow-setup}.

\paragraph{Local time gain.}
At each step, when \(|\partial_t\Phi|\gtrsim \lambda_{j}^{1/2}\) (see frequency localization in \S\ref{sec:kernel}, \S\ref{subsec:gradients}),
one integration by parts in time yields a factor \(\lambda_{j}^{-1/2}\).
Hence
\begin{equation}\label{eq:narrow-product}
  \prod_{j=0}^{1}\lambda_{j}^{-1/2}
  \;=\;\lambda^{-1/2}\,\lambda^{-7/16}
  \;=\;\lambda^{-15/16}.
\end{equation}
This is the \emph{local} gain for each cell obtained by the double \(7/8\) rescaling and pigeonholing in \(|\partial_t\Phi|\).


%--------------------------------------------------------------------
\subsection{Globalization of the narrow gain}\label{subsec:narrow-glue}
%--------------------------------------------------------------------
Assume that on each narrow cell obtained after the double $7/8$ rescaling and local
frequency localization by $\mu_6$, the local time gain
\eqref{eq:narrow-product} holds:
\[
   \Lambda_{\mathrm{loc}} := \lambda^{-15/16}.
\]
We need to transfer this to a global $L^6$ bound on $Q_\lambda$.

\paragraph{Framework for globalization (refined form).}
We perform six integrations by parts in time at the kernel level (operator $L_t$,
see §\ref{sec:kernel}), then pass to $TT^{*}$ and carry out another six IBPs
in the variables $s,s'$ and two transverse directions $y,y'$ of each copy of the kernel
(for a total of $12$ IBPs across both stages). The local multiplier $\Lambda_{\mathrm{loc}}$ 
is included in the amplitude at the level of a single narrow cell.

\begin{lemma}[Distribution of local weight in $TT^{*}$]\label{lem:12-ibp-weight}
Let $K^\#$ be the modified kernel after $6$ temporal IBPs, and
suppose each narrow window carries a local multiplier $\Lambda_{\mathrm{loc}}$.
Then after passing to $TT^{*}$ and performing six more IBPs in $(s,s',y,y')$ the quadratic form satisfies
\[
   \|F_{\mathrm{narrow}}\|_{L^6(Q_\lambda)}
   \ \lesssim\
   \Lambda_{\mathrm{loc}}^{\,1/12}\,
   \Biggl(\sum_{\Theta\in G}\|F_\Theta\|_{L^6(Q_\lambda)}^2\Biggr)^{1/2},
\]
that is, the global factor is
\[
   \Lambda_{\mathrm{glob}} \;=\; \Lambda_{\mathrm{loc}}^{1/12}
   \;=\; \lambda^{-\frac{1}{12}\cdot\frac{15}{16}} \;=\; \lambda^{-5/64}.
\]
\end{lemma}

\begin{proof}[Proof (factorized $TT^{*}$)]
After $6$ IBPs in $t$ the kernel can be written as a sum over narrow cells $C$:
\[
  K^\#(t,x;s,y)\ =\ \sum_{C}\ \Lambda_{\mathrm{loc}}\,
  \Big(\prod_{j=1}^{6} W^{(t)}_{j,C}(t,x)\Big)\,K^{(0)}_{C}(t,x;s,y),
\]
where $0\le W^{(t)}_{j,C}\le 1$ are normalized weights from $L_t$,
and $K^{(0)}_{C}$ is the phase part with unit amplitude (cf. factorization in §\ref{sec:kernel}).

At the $TT^{*}$ stage consider the kernel $(TT^{*})$ for each $C$ and perform another six IBPs
(in $s,s'$ and in two transverse directions $y,y'$ of each copy). This yields additional
normalized weights $W^{(\mathrm{TT}^\ast)}_{1,C},\dots,W^{(\mathrm{TT}^\ast)}_{6,C}$ with $0\le W^{(\mathrm{TT}^\ast)}_{\ell,C}\le 1$,
and a factorized majorant:
\[
  |K_{TT^{*},C}(t,x;s,y)|\ \lesssim\
  \Lambda_{\mathrm{loc}}\,
  \prod_{j=1}^{6} W^{(t)}_{j,C}(t,x)\cdot
  \prod_{\ell=1}^{6} W^{(\mathrm{TT}^\ast)}_{\ell,C}(t,x;s,y).
\]
Thus the $TT^{*}$ kernel for fixed $C$ contains the product of $12$ \emph{normalized}
factors, multiplied by $\Lambda_{\mathrm{loc}}$.

Apply Hölder’s inequality with equal weights $1/12$ (or AM–GM after measure normalization).
Since each of the $12$ factors defines an operator with $L^2\to L^2$ norm
$\lesssim 1$ (by Schur/IBP normalization) and the covering by cells has bounded multiplicity,
we obtain for the quadratic form $TT^{*}$ the estimate
\[
  \langle T f, Tf\rangle\ \lesssim\ \Lambda_{\mathrm{loc}}^{\,1/12}\,\|f\|_{L^2}^2,
\]
that is, $\|T\|_{2\to2}\lesssim \Lambda_{\mathrm{loc}}^{\,1/12}$.
Passing this to the standard $L^6$ scheme (via trilinear insertion/decoupling in the narrow
class $G$) and using bounded multiplicity of covering, we obtain the lemma.
Logarithmic factors are absorbed by §\ref{subsec:log-budget}.
\end{proof}

\paragraph{Conclusion.}
With Lemma~\ref{lem:12-ibp-weight} we have
\[
  \|F_{\mathrm{narrow}}\|_{L^6(Q_\lambda)}
  \ \lesssim\
  \lambda^{-5/64}\,
  \Biggl(\sum_{\Theta\in G} \|F_\Theta\|_{L^6(Q_\lambda)}^2\Biggr)^{1/2}.
\]
Thus the contribution of the “narrow cascade’’ block to the global balance is
\[
   \boxed{-\tfrac{5}{64}\ \text{in }\lambda,\qquad 0\ \text{in }D}.
\]

%--------------------------------------------------------------------
\subsection{Contribution of the narrow block}\label{subsec:narrow-balance}
%--------------------------------------------------------------------
\[
  -\dfrac{5}{64}\ \text{ in }\lambda, 
  \qquad 0\ \text{ in }D .
\]
\begin{remark}
Two steps of \(7/8\) rescaling are sufficient, since \(r_{2}^{2}/\alpha^{2}\gtrsim\lambda^{11/48}\); an additional third step would reduce the net gain in \(\lambda\) without providing new geometric benefit. The \emph{narrow} block is not activated where the \emph{Robust\,Kakeya} block (§\ref{sec:kakeya}) is used; the regimes are mutually exclusive. Note also that the “spatial’’ IBPs in the narrow block are performed at the $TT^\ast$ stage in the variables $(s,s',y,y')$; the operator $L_{x'}$ at the kernel level in the narrow regime is not applied (cf. explanation in App.~\ref{app:C3}).
\end{remark}
