
\pgfdeclareradialshading{fadeoutdisk}{\pgfpointorigin}{
  color(0)=(pgftransparent!100);
  color(22)=(pgftransparent!100);
  color(24)=(pgftransparent!0);
  color(30)=(pgftransparent!0)%
}
\pgfdeclarefading{fade out disk}{\pgfuseshading{fadeoutdisk}}

\newcommand{\switch}[3]{
\begin{tikzpicture}[scale=1,very thick]
% \filldraw [lightgray] (0,0) circle (1.41);
\draw [#1] (-1.41, 0) to (0, 0);
\draw [#2] (0, 0) to [in=240,out=0] (1, 1);
\draw [#3] (0, 0) to [in=120,out=0] (1, -1);

\fill[white,path fading=fade out disk] (0,0) circle (1.45);
\end{tikzpicture}
}

\newcommand{\splitsource}[4]{
\begin{tikzpicture}[scale=1.5,thick,baseline={([yshift=-.8ex]current bounding box.center)}]

\draw (0, 0) to node {\outline{$e$}} (0.5, 0);
\draw [#1] (0.5, 0) to [out=0,in=240] node [pos=0.6] {\outline{$a$}} (1, 0.5);
\draw [#2] (-0.5, 0.5) to [out=-60,in=180] node [pos=0.4] {\outline{$b$}} (0, 0);
\draw [#3] (-0.5, -0.5) to [out=60,in=180] node [pos=0.4]{\outline{$c$}} (0, 0);
\draw [#4] (0.5, 0) to [out=0,in=120] node [pos=0.6] {\outline{$d$}} (1, -0.5);

\end{tikzpicture}
}

\newcommand{\splittarget}[8]{
\begin{tikzpicture}[scale=1.5,thick,baseline={([yshift=-.8ex]current bounding box.center)}]

\draw [#1] (-0.5, 0.5) to [out=-60,in=180] (0, 0.2);
\draw [#2] (0, 0.2) to (0.5, 0.2);
\draw [#3] (0.5, 0.2) to [out=0,in=240] (1, 0.5);
\draw [#4] (-0.5, -0.5) to [out=60,in=180] (0, -0.2);
\draw [#5] (0, -0.2) to (0.5, -0.2);
\draw [#6] (0.5, -0.2) to [out=0,in=120] (1, -0.5);

\draw [#7] (0, -0.2) to [out=0,in=180] (0.5, 0.2);
\draw [#8] (0, 0.2) to [out=0,in=180] (0.5, -0.2);

\ifthenelse{\equal{#7}{red}}{\ifthenelse{\equal{#8}{blue}}{\node [dot] at (0.25,0) {};}}{}

\ifthenelse{\equal{#7}{blue}}{\ifthenelse{\equal{#8}{red}}{\node [dot] at (0.25,0) {};}}

% \ifthenelse{\and{\equal{#7}{red}}{\equal{#8}{blue}}}{\node [dot] at (0.25,0) {};}{}

% \ifthenelse{\and{\equal{#7}{blue}}{\equal{#8}{red}}}{\node [dot] at (0.25,0) {};}{}

\end{tikzpicture}
}

\newcommand{\tikzdisk}[2]{
\begin{tikzpicture}[scale=0.85,very thick]
\draw [transparent] (0,0) circle (1.5);  % Bounding box.
\tikzdiskcore{#1}{#2}
\end{tikzpicture}
}

\newcommand{\tikzdisksplit}[6]{
\begin{tikzpicture}[scale=1,very thick]
\tikzdiskcore{#1}{#2}

\pgfmathsetmacro{\sa}{(360 * (#3-1) / \n) - 180 + (90 * #5)}
\pgfmathsetmacro{\ea}{(360 * (#4-1) / \n) - 180 + (90 * #6)}
\draw [red, looseness=1] (#3) to [out=\sa,in=\ea] (#4);

\ifthenelse{#5=0}{\node [dot] at (#3) {};}{}
\ifthenelse{#6=0}{\node [dot] at (#4) {};}{}
\end{tikzpicture}
}

\newcommand{\tikzdiskcore}[2]{

\setsepchar{,}
\readlist\angles{#1}
\readlist\colours{#2}

\pgfmathsetmacro{\n}{\listlen\angles[]}
\pgfmathsetmacro{\nn}{\n+1}
\pgfmathsetmacro{\nnn}{\n-1}

\tikzset{B/.style={blue}}
\tikzset{R/.style={red}}
\tikzset{G/.style={gray, dotted}}

% Make coordinates.
\foreach \k [
count=\i from 1,
evaluate=\i as \ii using {ifthenelse(\i==\nn,1,\i)},
evaluate=\ii as \a using {\angles[\ii]},
evaluate=\k as \d using {360*\k / \n}
] in {0,1,...,\n} {
    \pgfmathsetmacro{\rr}{ifthenelse(\a==180,1,ifthenelse(\a==90,1.25,1.5)}
    \coordinate (\i) at (\d:\rr);
}


% Make the backing disk.
\path [draw=none, fill=none] (1)
\foreach \k 
[
count=\i from 1,
count=\j from 2,
evaluate=\j as \jj using {ifthenelse(\j==\nn,1,\j)},
evaluate=\i as \a using {\angles[\i]}, 
evaluate=\jj as \aa using {\angles[\jj]},
evaluate=\k as \d using {360*\k / \n}
] in {0,1,...,\nnn} {
    .. controls +({ifthenelse(\a==180,\d+90,ifthenelse(\a==90,\d+90+45,\d+90+90)}:0.25) and +({ifthenelse(\aa==180,\d-60,ifthenelse(\aa==90,\d-60-45,\d-60-90)}:0.25) .. (\j)
};

% Add arcs.
\foreach \k 
[
count=\i from 1, 
count=\j from 2,
evaluate=\j as \jj using {ifthenelse(\j==\nn,1,\j)},
evaluate=\i as \a using {\angles[\i]},
evaluate=\jj as \aa using {\angles[\jj]},
evaluate=\k as \d using {360*\k / \n}
] in {0,1,...,\nnn} {
    \pgfmathsetmacro{\xx}{ifthenelse(\a==180,\d+90,ifthenelse(\a==90,\d+90+45,\d+90+90)}
    \pgfmathsetmacro{\yy}{ifthenelse(\aa==180,\d-60,ifthenelse(\aa==90,\d-60-45,\d-60-90)}
    % This is so ugly, but this is the only way I could get this to work.
    % Turns out you can't nest \ifthenelse inside any smaller part of a tikz \draw command :(
    \ifthenelse{\equal{\colours[\i]}{R}}{\draw [R] (\i) to [out=\xx,in=\yy] (\j);}{}
    \ifthenelse{\equal{\colours[\i]}{G}}{\draw [G] (\i) to [out=\xx,in=\yy] (\j);}{}
    \ifthenelse{\equal{\colours[\i]}{B}}{\draw [B] (\i) to [out=\xx,in=\yy] (\j);}{}
}

% Add dots (where needed).
\foreach \i
[
evaluate=\i as \a using {\angles[\i]},
] in {1,...,\n} {
    \ifthenelse{\a=90}{\node [dot] at (\i) {};}{}
}
}

\newcommand{\standardannulus}[4]
{
% +ve, -ve, hor, vert
\begin{tikzpicture}[scale=1.5,very thick]

\draw [thin] (-1, 1.1) to [out=90, in=90, looseness=0.5] (1, 1.1);
\draw [thin, fill=none] (-1, 1.1) 
 to [out=270, in=90] (-1, -0.7) to [out=270, in=270, looseness=0.5] (1, -0.7)
 to [out=90, in=270] (1, 1.1) to [out=270, in=270, looseness=0.5] (-1, 1.1);
\draw [thin] (-1, -0.7) to [out=270, in=270, looseness=0.5] (1, -0.7);

\draw [thin] (-1, 1.1) to (-1, -0.7);
\draw [thin] (1, 1.1) to (1, -0.7);
\draw [thin, dotted] (-1, -0.7) to [out=90, in=90, looseness=0.5] (1, -0.7);

\draw [dotted, looseness=0.5] (-1,0.2) to [out=90,in=90] (1,0.2);
\draw (0, 0.8) to (0, 0.55);
\draw (0, -1) to (0, -0.75);

\draw (-1,0.2) to [out=270,in=165] (-0.75, 0);
\draw (1,0.2) to [out=270,in=15] (0.75, 0);

\draw [#1] (-0.75, 0) to [out=-15, in=270] (0,0.55); % positive slope, left
\draw [#1] (0.75, 0) to [out=195, in=90] (0,-0.75); % positive slope, right
\draw [#2] (-0.75, 0) to [out=-15, in=90] (0,-0.75); % negative slope, left
\draw [#2] (0.75, 0) to [out=195, in=270] (0,0.55); % negative slope, right

\draw [#3, looseness=0.75] (-0.75, 0) to [out=-15, in=195] (0.75, 0); % horizontal
\draw [#4] (0, -0.75) to (0, 0.55); % vertical

\ifthenelse{\equal{#3}{red}}{\ifthenelse{\equal{#4}{blue}}{\node [dot] at (0,-0.0835) {};}}{}

\ifthenelse{\equal{#3}{blue}}{\ifthenelse{\equal{#4}{red}}{\node [dot] at (0,-0.0835) {};}}

\end{tikzpicture}
}