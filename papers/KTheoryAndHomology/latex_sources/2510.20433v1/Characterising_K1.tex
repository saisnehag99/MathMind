\documentclass[a4paper, 11pt,reqno]{amsart}
\usepackage{amsfonts, amsthm, amssymb, amsmath, stackengine, scalerel}
\usepackage{subfig}
\usepackage{mathrsfs,array,tikz-cd}
\usepackage{eucal,fullpage,times,color,enumerate,accents, comment}
\usepackage[all]{xy}
\usepackage{url}
\usepackage{hyperref}
\usepackage{enumitem}
\usepackage[new]{old-arrows}
\usepackage{extpfeil}
\usepackage{verbatim}  %for comment environment
%\usepackage[notcite,notref]{showkeys}
\usetikzlibrary{graphs,decorations.pathmorphing,decorations.markings}
\usepackage{stackengine}
\usepackage{mleftright}
\usepackage{float}
\usepackage{bbm} %% for blackboard bold numbers
\usepackage{multicol} %% for multipe columns, e.g. when itemising
% \usepackage{svg} %%to include svg files for graphics

\input xy
\xyoption{all}

\setlength{\textwidth}{6.5in}
\setlength{\oddsidemargin}{-0.1in}
\setlength{\evensidemargin}{-0.1in}

\def\foo{\,\ThisStyle{\ensurestackMath{%
  \bigsqcup\stackengine{-0pt}{\!}{\SavedStyle\!^{\mathord{\uparrow}}}{O}{l}{F}{T}{S}}}\,}
\DeclareMathOperator*{\foobarX}{\foo}
\newcommand\dirsup{\!\foobarX}

  \newcommand{\pullbackcorner}[1][dl]{\save*!/#1-1pc/#1:(-1,1)@^{|-}\restore}
\newextarrow{\xbigtoto}{{20}{20}{20}{20}}
   {\bigRelbar\bigRelbar{\bigtwoarrowsleft\rightarrow\rightarrow}}    
\newcommand{\calN}{{\mathcal{N}}}
\newcommand{\calX}{{\mathcal{X}}}
\newcommand{\calY}{{\mathcal{Y}}}
\newcommand{\calO}{{\mathcal{O}}}
\newcommand{\calL}{{\mathcal{L}}}
\newcommand{\calU}{{\mathcal{U}}}
\newcommand{\calV}{{\mathcal{V}}}
\newcommand{\calJ}{\mathcal{J}}
\newcommand{\calA}{{\mathcal{A}}}
\newcommand{\calB}{{\mathcal{B}}}
\newcommand{\calM}{{\mathcal{M}}}
\newcommand{\calD}{{\mathcal{D}}}
\newcommand{\calQ}{\mathcal{Q}}
\newcommand{\E}{\mathcal{E}}
\newcommand{\calF}{\mathcal{F}}
\newcommand{\calG}{\mathcal{G}}
\newcommand{\calI}{\mathcal{I}}
\newcommand{\calH}{\mathcal{H}}
\newcommand{\calC}{\mathcal{C}}
\newcommand{\calS}{\mathcal{S}}
\newcommand{\calP}{\mathcal{P}}
\newcommand{\calT}{\mathcal{T}}
\newcommand{\calW}{\mathcal{W}}
\newcommand{\calHom}{\mathcal{H}\mathrm{om}}
\newcommand{\calExt}{\mathcal{E}\mathrm{xt}}


\newcommand{\G}{\mathbf{G}}


\renewcommand{\P}{\mathbf{P}}
\newcommand{\Spec}{\mathrm{Spec}}
\newcommand{\Hens}{\mathrm{Hens}}
\newcommand{\Gal}{\mathrm{Gal}}
\newcommand{\Gpd}{\mathrm{Gpd}}
\newcommand{\Hom}{\mathrm{Hom}}
\newcommand{\Isom}{\mathrm{Isom}}
\newcommand{\Fun}{\mathrm{Fun}}
\newcommand{\Lan}{\mathrm{Lan}}
\newcommand{\Ran}{\mathrm{Ran}}
\newcommand{\Map}{\mathrm{Map}}
\newcommand{\End}{\mathrm{End}}
\newcommand{\Aut}{\mathrm{Aut}}
\newcommand{\Sing}{\mathrm{Sing}}
\newcommand{\Shv}{\mathrm{Shv}}
\newcommand{\Ind}{\mathrm{Ind}}
\newcommand{\Pro}{\mathrm{Pro}}
\newcommand{\GL}{\mathrm{GL}}
\newcommand{\PGL}{\mathrm{PGL}}
\newcommand{\Ext}{\mathrm{Ext}}
\newcommand{\Tor}{\mathrm{Tor}}
\newcommand{\Coh}{\mathrm{Coh}}
\newcommand{\QC}{\mathrm{QCoh}}
\newcommand{\Sch}{\mathrm{Sch}}
\newcommand{\Cartier}{\mathrm{Cartier}}
\newcommand{\Sm}{\mathrm{Sm}}
\newcommand{\Vect}{\mathrm{Vect}}
\newcommand{\Sym}{\mathrm{Sym}}
\newcommand{\Frob}{\mathrm{Frob}}
\newcommand{\Mod}{\mathrm{Mod}}
\newcommand{\Spc}{\mathrm{Spc}}
\newcommand{\Alg}{\mathrm{Alg}}
\newcommand{\Mon}{\mathrm{Mon}}
\newcommand{\Set}{\mathrm{Set}}
\newcommand{\Ab}{\mathrm{Ab}}
\newcommand{\Ann}{\mathrm{Ann}}
\newcommand{\ob}{\mathrm{ob}}
\newcommand{\ord}{\mathrm{ord}}
\newcommand{\rk}{\mathrm{rk}}
\newcommand{\ch}{\mathrm{char}}
\newcommand{\D}{\mathcal{D}}
\renewcommand{\L}{\mathrm{L}}
\newcommand{\W}{\mathrm{W}}
\newcommand{\Ch}{\mathrm{Ch}}
\newcommand{\Pic}{\mathrm{Pic}}
\newcommand{\fl}{\mathrm{fl}}
\newcommand{\zar}{\mathrm{Zar}}
\newcommand{\an}{\mathrm{an}}
\newcommand{\prop}{\mathrm{prop}}
\newcommand{\im}{\mathrm{im}}
\newcommand{\id}{\mathrm{id}}
\newcommand{\ev}{\mathrm{ev}}
\newcommand{\coker}{\mathrm{coker}}
\newcommand{\Fil}{\mathrm{Fil}}
\newcommand{\pr}{\mathrm{pr}}
\newcommand{\Tr}{\mathrm{Tr}}
\newcommand{\Bl}{\mathrm{Bl}}
\newcommand{\Frac}{\mathrm{Frac}}
\newcommand{\gr}{\mathrm{gr}}
\newcommand{\dR}{\mathrm{dR}}
\newcommand{\ddR}{\mathrm{ddR}}
\newcommand{\red}{\mathrm{red}}
\newcommand{\sh}{\mathrm{sh}}
\newcommand{\fpt}{\mathrm{fpt}}
\newcommand{\perf}{\mathrm{perf}}
\renewcommand{\inf}{\mathrm{inf}}
\newcommand{\crys}{\mathrm{crys}}
\newcommand{\st}{\mathrm{st}}
\newcommand{\cl}{\mathrm{cl}}
\newcommand{\can}{\mathrm{can}}
\newcommand{\Alt}{\mathrm{Alt}}
\newcommand{\comp}{\mathrm{comp}}
\newcommand{\pre}{\mathrm{pre}}
\newcommand{\Var}{\mathcal{V}\mathrm{ar}}
\newcommand{\Tmod}{\mathbb{T}\mathrm{-mod}}
\newcommand{\conj}{\mathrm{conj}}
\newcommand{\opp}{\mathrm{op}}    % opposite category
\newcommand{\Free}{\mathrm{Free}}
\newcommand{\Forget}{\mathrm{Forget}}
\newcommand{\Log}{\mathrm{Log}}
\newcommand{\nil}{\mathrm{NIL}}
\newcommand{\grp}{\mathrm{grp}}
\renewcommand{\ker}{\mathrm{ker}}
\newcommand{\cok}{\mathrm{cok}}
\newcommand{\naive}{\mathrm{naive}}
\newcommand{\e}{\mathrm{e}}
\newcommand{\Cone}{\mathrm{Cone}}
\newcommand{\Cech}{\mathrm{Cech}}
\newcommand{\Tot}{\mathop{\mathrm{Tot}}}
\newcommand{\Comp}{\mathrm{Comp}}
\newcommand{\Der}{\mathrm{Der}}
\newcommand{\Sect}{\mathrm{Sect}}
\newcommand{\val}{\mathrm{val}}
\newcommand{\glob}{\mathrm{glob}}
\newcommand{\Ho}{\mathrm{Ho}}
\newcommand{\cart}{\mathrm{cart}}
\newcommand{\cont}{\mathrm{cont}}
\newcommand{\Loc}{\mathrm{Loc}}
\newcommand{\Cov}{\mathrm{Cov}}
\newcommand{\Rep}{\mathrm{Rep}}
\newcommand{\Berk}{\mathrm{Berk}}
\newcommand{\Cons}{\mathrm{Cons}}
\newcommand{\Stab}{\mathrm{Stab}}
\newcommand{\AdEx}{\mathrm{AdjExt}}
\newcommand{\Gl}{\mathrm{Gl}}
\newcommand{\Frm}{\textbf{Frm}}
\newcommand{\argu}{\text{(---)}}
\newcommand{\fram}{\mathfrak{m}}
\newcommand{\fran}{\mathfrak{n}}
\newcommand{\frap}{\mathfrak{p}}
\newcommand{\Frap}{\mathfrak{P}}
\newcommand{\fraa}{\mathfrak{a}}
\newcommand{\ISpec}{\mathrm{ISpec}}
\newcommand{\qp}{Q_{+}}
\newcommand{\rf}{\mathsf{rad}_\F}
\newcommand{\Alin}{\mathcal{A}_{\mathrm{Lin}}}
\newcommand{\A}{\mathcal{A}}
\newcommand{\M}{\mathcal{M}}
\newcommand{\AffBerk}{\mathbb{A}_{\mathrm{Berk}}}
\newcommand{\BerkD}{\mathbb{D}}

\newcommand{\Z}{\mathbb{Z}}
\newcommand{\N}{\mathbb{N}}
\newcommand{\Q}{\mathbb{Q}}
\newcommand{\R}{\mathbb{R}}
\newcommand{\com}{\mathsf{c}}
\newcommand{\bF}{\mathbb{F}}
\newcommand{\F}{\mathcal{F}}
\newcommand{\C}{\mathbb{C}}
\newcommand{\vv}{\mathsf{v}}
\newcommand{\thT}{\mathbb{T}}
\newcommand{\ACVF}{\mathrm{ACVF}}

\newcommand{\Sp}{\mathrm{Sp}}
\newcommand{\lav}{\overleftarrow{av}}
\newcommand{\baseS}{\mathcal{S}}
\newcommand{\Eff}{\mathrm{Eff}}
\newcommand{\modd}{\mathrm{mod}}
\newcommand{\Znq}{\mathbb{Z}_{\neq 0}}
\newcommand{\FL}{\mathfrak{P}_{\overleftarrow{\Lambda}}}
\newcommand{\gav}{|\cdot|}    % generic absolute value |.|
\newcommand{\lrangle}{\langle\rangle}
\newcommand{\lrangles}[1]{\langle#1\rangle}
\newcommand{\lranglet}[2]{\langle#1,#2\rangle}
\newcommand{\cod}{\mathrm{cod}}
\newcommand{\arr}{\mathrm{arr}}


\newcommand{\SC}{\calS\calC}
\newcommand{\Wtop}{W_{\mathrm{top}}}
\newcommand{\Wbot}{W_{\mathrm{bot}}}
\newcommand{\MSF}{{\overline{M}| F}}
\newcommand{\piMSF}{\pi_0(\MSF)}
\newcommand{\OSF}{{{O|\calS F}}}
\newcommand{\OF}{{{O|F}}}
\newcommand{\gm}[1]{{{|#1|}}}
\newcommand{\cofib}{\mathrm{cofib}}
\newcommand{\fib}{\mathrm{fib}}
\newcommand{\tx}{3\times 3}


\def\forkindep{\mathrel{\raise0.2ex\hbox{\ooalign{\hidewidth$\vert$\hidewidth\cr\raise-0.9ex\hbox{$\smile$}}}}}


\newcommand{\cmot}{\chi^{\mathrm{mot}}}
\newcommand{\sk}{\mathbbm{1}_k}

%%%% Medium Sized Bullet
\makeatletter
\newcommand*\dotp{\mathpalette\dotp@{.5}}
\newcommand*\dotp@[2]{\mathbin{\vcenter{\hbox{\scalebox{#2}{$\m@th#1\bullet$}}}}}
\makeatother


\newcommand{\Sub}{\mathrm{Sub}}
\DeclareSymbolFont{matha}{OML}{txmi}{m}{it}% txfonts
\DeclareMathSymbol{\varv}{\mathord}{matha}{118}

\newcommand{\s}{\tilde{s}}
\newcommand{\tk}{\,\text{,}\,}

%\newcommand{\comment}[1]{}

\newcommand{\cosimp}[3]{\xymatrix@1{#1 \ar@<.4ex>[r] \ar@<-.4ex>[r] & {\ }#2 \ar@<0.8ex>[r] \ar[r] \ar@<-.8ex>[r] & {\ } #3 \ar@<1.2ex>[r] \ar@<.4ex>[r] \ar@<-.4ex>[r] \ar@<-1.2ex>[r] & \cdots }}
%%%% Pullback corner
\newsavebox{\pullback}
\sbox\pullback{%
	\begin{tikzpicture}%
	\draw (0,0) -- (1ex,0ex);%
	\draw (1ex,0ex) -- (1ex,1ex);%
	\end{tikzpicture}}


\definecolor{quotemark}{gray}{0.7}
\makeatletter
\newlength\origparskip

\newcommand{\fquote}{%
	\@ifnextchar[{\fquote@i}{\fquote@i[]}%]
}

\def\fquote@i[#1]{%
	\@ifnextchar[{\fquote@ii{#1}}{\fquote@ii{#1}[]}%]
}%

\def\fquote@ii#1[#2]{%
	\def\pqm@tempa{#1}%
	\def\pqm@tempb{#2}%
	\noindent
	\list
	{}
	{\setlength{\leftmargin}{0.3\textwidth}%
		\setlength{\rightmargin}{0.1\textwidth}%
		\setlength{\origparskip}{\parskip}}%
	\item[]%
	\begin{picture}(0,0)%
	\put(-15,-8){\makebox(0,0){\scalebox{4}{%
				\textcolor{quotemark}{\textquotedblright}}}}%
	\end{picture}%
	\begingroup
	\itshape
	\ignorespaces}%

\def\endfquote{%
	\endgroup
	\par
	\raggedleft
	\ifx\pqm@tempa\empty
	\else
	{\bfseries --- \pqm@tempa\par}%
	\setlength{\parskip}{\origparskip}%
	\ifx\pqm@tempb\empty
	\else
	(\pqm@tempb)%
	\fi
	\fi
	\par
	\endlist}
\makeatother

%%%% end quote/epigraphs 

%%%% Double underline
\def\doubleunderline#1{\underline{\underline{#1}}}

%%% Double overline
\def\doubleoverline#1{\overline{\overline{#1}}}

%%%% Quotients
\newcommand{\bigslant}[2]{{\raisebox{.2em}{$#1$}\left/\raisebox{-.2em}{$#2$}\right.}}

%%%% Comments
\newcommand{\MING}[1]{{\color{blue} {\tiny \bf (M:)} {\bf #1}}}


%%%% Adjusting size of norms
\newcommand\norm[1]{\left\lVert#1\right\rVert}


\DeclareMathOperator*{\colim}{colim}

\newcommand{\equalizer}[2]{\xymatrix@1{#1 \ar@<.4ex>[r] \ar@<-0.4ex>[r] & {\ } #2}}

\newcommand{\adjunction}[4]{\xymatrix@1{#1{\ } \ar@<-0.3ex>[r]_{ {\scriptstyle #2}} & {\ } #3 \ar@<-0.3ex>[l]_{ {\scriptstyle #4}}}}
% Here's how the above command works: \adjunction{a}{b}{c}{d} has categories a and c, b:a ---> c (right adjoint, going from left to right), and a <---- c:d (left adjoint, going from right to left)











%%% Double Exact Sequence

\makeatletter
\newcommand*{\relrelbarsep}{.386ex}
\newcommand*{\relrelbar}{%
	\mathrel{%
		\mathpalette\@relrelbar\relrelbarsep
	}%
}
\newcommand*{\@relrelbar}[2]{%
	\raise#2\hbox to 0pt{$\m@th#1\relbar$\hss}%
	\lower#2\hbox{$\m@th#1\relbar$}%
}
\providecommand*{\rightrightarrowsfill@}{%
	\arrowfill@\relrelbar\relrelbar\rightrightarrows
}
\providecommand*{\leftleftarrowsfill@}{%
	\arrowfill@\leftleftarrows\relrelbar\relrelbar
}
\providecommand*{\xrightrightarrows}[2][]{%
	\ext@arrow 0359\rightrightarrowsfill@{#1}{#2}%
}
\providecommand*{\xleftleftarrows}[2][]{%
	\ext@arrow 3095\leftleftarrowsfill@{#1}{#2}%
}
\makeatother

%%%% Distinguished Squares

\newcommand{\dsquare}[4]{
	\begin{tikzcd}[ampersand replacement=\&]
	#1\ar[r, >->]\ar[d, {Circle[open]}->] \ar[dr, phantom, "\square"]\& #2 \ar[d, {Circle[open]}->]\\
	#3 \ar[r, >->] \& #4
	\end{tikzcd}}

%%%% Filled Distinguished Squares

\newcommand{\dsquaref}[8]{
	\begin{tikzcd}[ampersand replacement=\&]
	#1\ar[r, >->,"#5"]\ar[d, swap,{Circle[open]}->,"#6"] \ar[dr, phantom, "\square"]\& #2 \ar[d, {Circle[open]}->,"#8"]\\
	#3 \ar[r, >->,"#7"] \& #4
	\end{tikzcd}}

%%%% Pseudo-Commutative Square

\newcommand{\pcsquaref}[8]{
	\begin{tikzcd}[ampersand replacement=\&]
	#1\ar[r, >->,"#5"]\ar[d, swap,{Circle[open]}->,"#6"] \ar[dr, phantom, "\circlearrowleft"]\& #2 \ar[d, {Circle[open]}->,"#8"]\\
	#3 \ar[r, >->,"#7"] \& #4
	\end{tikzcd}}


%%% Commutative Square

\newcommand{\squaref}[8]{
	\begin{tikzcd}[ampersand replacement=\&]
	#1\ar[r,"#5"]\ar[d, swap,"#6"] \& #2 \ar[d, "#8"]\\
	#3 \ar[r, "#7"] \& #4
	\end{tikzcd}}

\newcommand{\rtail}{\rightarrowtail}
\newcommand{\ltail}{\leftarrowtail}
\newcommand{\xrtail}[1]{\overset{#1}{\rtail}}
\newcommand{\xltail}[1]{\overset{#1}{\leftarrowtail}}
\newcommand{\otail}{\,\,\stackengine{0pt}{\hspace{.81ex}$\rightarrow$}{$\circ$}{O}{l}{F}{F}{L}} %%% circular tail
\newcommand{\xotail}[1]{\overset{\,\,\,\,\,#1}{\otail}}

\newcommand{\xdtail}[1]{\overset{#1}{\twoheadrightarrow}}
\newcommand{\ones}[2]{\begin{pmatrix}
		#1 \\ #2
\end{pmatrix}}

\newcommand{\Cplus}{{\calC^{\oplus}}}
\newcommand{\Ghat}{{\widehat{G}\calC}}

\newcommand{\MC}{\mathcal{M}_{\calC}}
\newcommand{\EC}{\mathcal{E}_{\calC}}
\newcommand{\Ar}{\mathrm{Ar}}
\newcommand{\Ars}{\Ar_{\square}}
\newcommand{\Art}{\Ar_{\triangle}}

\def\rmk#1{\textcolor{red}{ #1 }}
\def\Rmk#1{\textcolor{blue}{ #1 }}

\setcounter{tocdepth}{1}
\begin{document}
\bibliographystyle{alpha}
\newtheorem{theorem}{Theorem}[section]
\newtheorem*{theorem*}{Theorem}
\newtheorem*{condition*}{Condition}
\newtheorem*{definition*}{Definition}
\newtheorem*{corollary*}{Corollary}
\newtheorem*{conjecture}{Conjecture}
\newtheorem{proposition}[theorem]{Proposition}
\newtheorem{lemma}[theorem]{Lemma}
\newtheorem{corollary}[theorem]{Corollary}
\newtheorem{claim}[theorem]{Claim}
\newtheorem{conclusion}[theorem]{Conclusion}
\newtheorem{hypothesis}[theorem]{Hypothesis}
\newtheorem{summarytheorem}[theorem]{Summary Theorem}
\newtheorem{summaryexample}[theorem]{Summary Example}

\theoremstyle{definition}
\newtheorem{definition}[theorem]{Definition}
\newtheorem{question}{Question}
\newtheorem{remark}[theorem]{Remark}
\newtheorem{observation}[theorem]{Observation}
\newtheorem{discussion}[theorem]{Discussion}
\newtheorem{guess}[theorem]{Guess}
\newtheorem{example}[theorem]{Example}
\newtheorem{condition}[theorem]{Condition}
\newtheorem{warning}[theorem]{Warning}
\newtheorem{notation}[theorem]{Notation}
\newtheorem{construction}[theorem]{Construction}
\newtheorem{problem}[theorem]{Problem}
\newtheorem{fact}[theorem]{Fact}
\newtheorem{thesis}[theorem]{Thesis}
\newtheorem{convention}[theorem]{Convention}
\newtheorem{summary}[theorem]{Summary}
\newtheorem{reminder}[theorem]{Reminder}



\newtheorem{naivedefinition}[theorem]{Naive Definition}
\newtheorem{maintheorem}{Theorem}
\renewcommand*{\themaintheorem}{\Alph{maintheorem}}
\newtheorem*{theorem:GilletGrayson}{Theorem A}
\newtheorem*{theorem:Generators}{Theorem B}
\newtheorem*{theorem:R1}{Theorem C}
\newtheorem*{theorem:R2}{Theorem D}


\title{$K_1(\Var)$ is presented by stratified birational equivalences}
% \title{$K_1$ of varieties is presented by stratified birational equivalences}
% \title{$K_1(\Var)$ is generated by quasi-automorphisms}

\author{Ming Ng}
% \address{Graduate School of Informatics, Nagoya University}
% \email{ng.ming.k0@a.mail.nagoya-u.ac.jp}
\begin{abstract} This paper provides a complete presentation of $K_1(\Var)$, the $K_1$ group of varieties, resolving and simplifying a problem left open in \cite{ZakhK1}. Our approach adapts Gillet-Grayson's $G$-Construction to define an un-delooped $K$-theory spectrum of varieties. There are two levels on which one can read the present paper. On a technical level, we streamline and extend previous results on the $K$-theory of exact categories to a broader class of categories, including $\Var$. On a more conceptual level, our investigations bring into focus an interesting generalisation of automorphisms (``double exact squares'') which generate $K_1$. For varieties, this isolates a subclass of birational equivalences which we call {\em stratified}, but the construction extends to a wide range of non-additive contexts (e.g. $o$-minimal structures, definable sets etc.). This raises a challenging question: what kind of information do these generalised automorphisms calibrate? 
\end{abstract}

 \thanks{Research partially supported by EPSRC Grant EP/V028812/1 and a FY2024 JSPS Postdoctoral Fellowship (Short-Term).}
\maketitle

Our understanding of $K$-theory is changing. Recent efforts to extend tools from classical algebraic $K$-theory to non-additive settings have led us to make decisions on what its essential features are. One perspective, influenced by Waldhausen's $S_\bullet$-construction \cite{Wald}, is that $K$-theory is a framework for analysing the finite assembly and disassembly of objects; non-additive applications of this insight can be found in Campbell's $\widetilde{S}_\bullet$-construction \cite{Campbell} as well as Zakharevich's Assemblers \cite{ZakhAss}. A related perspective emphasises the view that $K$-theory breaks an object into two types of pieces. This underpins Campbell-Zakharevich's framework of {\em CGW categories} \cite{CGW}, which formalises key similarities between exact categories and the category of varieties $\Var_k$. 



In a different line of work: the study of $K_1$ of arbitrary exact categories properly began with Gillet-Grayson’s $G$-construction \cite{GG}, which provides an elementary description of its generators. This description was later refined by Sherman \cite{ShermanAbelian, ShermanExact} and Nenashev \cite{NenGen}, before culminating in Nenashev's characterisation of the complete set of relations for $K_1$ \cite{Nen0, Nen1}. 



The present paper unites the two lines of investigation by extending the $K_1$ results to a subclass of CGW categories known as {\em pCGW categories}. These include not only exact categories and varieties, but also finite sets, $o$-minimal structures, and definable sets. Our analysis leads to two alternative, complete presentations of $K_1$ applicable to a broad range of non-additive contexts. For varieties, our results show $K_1(\Var_k)$ is generated by subclass of birational equivalences we call {\em stratified} (for details, see discussion above Corollary~\ref{cor:BirIso}). We conclude with a broad range of interesting test problems exploring the implications.

\subsection*{Background Overview} Let us develop the previous remark that $K$-theory is an abstract framework for breaking an object into two different types of pieces. Consider the following two definitions.
\begin{itemize}
	\item Let $R$ be a ring. We define $K_0(R)$ as

	$$K_0(R):=\bigslant{\left\{\stackanchor{free abelian group}{fin. gen. proj. $R$-modules}\right\}}{\stackanchor{$[M]=[M']$, if $M\cong M'$}{$[M]=[M']+[M'']$, if $M'\to M\to M''$}}$$
	where $M'\to M\to M''$ is a short exact sequence. 
	\item Let $\Var_k$ be the category of $k$-varieties, i.e. reduced separated schemes of finite type over field $k$. We define $K_0(\Var_k)$ as 
	
	
	$$K_0(\Var_k):=\bigslant{\left\{\stackanchor{free abelian group}{$k$-varieties}\right\}}{\stackanchor{$[X]=[X']$, if $X\cong X'$}{$[X]=[U]+[X\setminus U]$, if $U\hookrightarrow X$ is a closed immersion}}$$
	
	%% Scheme of Finite Type over K is a scheme together with morphism X->K where X is a locally ringed space with a cover of spectra of rings, and K is Spec(K). We have a finite morphism of sheaves, reduced to a single ring homomorphism. Equivalently, it is a scheme with a finite cover of spectra of fin. gen. K-algebras
	%% Separated: A separated morphism f:X->Y is one whose diagonal morphism is a closed immersion. Recall: a diagonal morphism is the map X -> X x_S X, where X x_S X is the kernel pair and the map is induced by the identity map on X. "Hausdorff"
	%%  reduced "no purely infinitesimal elements.": An algebraic scheme is reduced if all stalks of the structure sheaf are local rings without nonzero nilpotent elements.
\end{itemize}
The analogy is clear. A short exact sequence $M'\to M\to M''$ decomposes the $R$-module $M$ into two distinct pieces, $M'$ and $M''$, with $M'\to M$ an admissible mono and $M\to M''$ as an admissible epi. Similarly, $K_0(\Var_k)$ decomposes a variety $X$ into $U$ and $X\setminus U$, with $U\hookrightarrow X$ a closed immersion and $X\setminus U\hookrightarrow X$ an open immersion. %% When viewed in $K_0(R)$, this translates to the equation $[M]=[M']+[M'']$, reflecting that $M$ is, in essence, constructed from these two components. 
In both cases, an object is broken into two pieces and viewed as an abstract sum of the two components. But what are the essential features of this decomposition? 

Quillen \cite{QuilenExact} introduced the notion of an exact category as a pair $(\calC, \calS)$, where $\calC$ is an additive category, and $\calS$ is a family of sequences $M' \to M \to M''$ satisfying conditions analogous to short exact sequences in abelian categories. This includes the natural condition that admissible monos $M' \to M$ are kernels of admissible epis $M \to M''$, and admissible epis are cokernels of admissible monos. These conditions enabled Quillen to construct a $K$-theory spectrum $K\calC$, recovering $K_0(R)$ on $\pi_0$ when $\calC$ is the category of finitely-generated projective $R$-modules.
\begin{comment}
Notes: See Weibel, II.7,  in particular Definition 7.1 and Example 7.1.1; See Weibel IV, Defintiion 6.3.2.
\end{comment}

Campbell-Zakharevich \cite{CGW} re-examined Quillen's framework, and made a crucial observation: instead of requiring the two classes of morphisms to compose, it suffices to encode their interaction formally. This added flexibility allows us to extend Quillen's $K$-theory to non-additive settings like varieties, where sequences like $U \hookrightarrow X \hookleftarrow X \setminus U$ clearly do not compose. This is made precise by \cite[Thm 4.3]{CGW}, which adapts Quillen's argument to construct a $K$-theory spectrum $K\Var_k$ recovering $K_0(\Var_k)$ on $\pi_0$. 

Importantly, the Campbell-Zakharevich construction allows us to define the $K$-groups $K_n(\Var_k):=\pi_n (K\Var_k)$ for all $n$. It is natural to ask:

\begin{question}\label{qn:key} What kind of information do the higher $K$-groups of varieties encode?
\end{question}

This is a challenging question. In classical algebraic $K$-theory, the coarseness of $K_0$ as an invariant may be measured by the fact that $K_0(F)=\Z$ for all fields $F$, whereas $K_1(F)\cong F^\times$. %\footnote{Recall: this is because vector spaces of the same finite dimension are isomorphic. For $K_1$, see Weibel III, Example 1.1.2.} 
Is there an analogous story in the setting of varieties? More explicitly, how might we measure the loss of information in $K_0(\Var_k)$? To what extent can this information be recovered in the higher $K$-groups? The following summary theorem gives a snapshot of the current landscape.

\begin{summarytheorem}\label{sumthm:KVar} Let $k$ be an algebraically closed field of characteristic 0, and equip $K_0(\Var_k)$ with a ring structure by defining $[X]\cdot [Y]:= [(X\times_k Y)_{\red}]$. Two $k$-varieties $X,Y$ are said to be \emph{piecewise isomorphic} if $X$ and $Y$ admit finite partitions 
	$$X_1,\dots, X_n \qquad\text{and}\qquad  Y_1,\dots, Y_n$$
into locally closed subvarieties such that $X_i\cong Y_i$ for all $n$. The following is known:
\begin{enumerate}[label=(\roman*)]
	\item Define $SK_0(\Var_k)$ as the freely generated {\em semiring} on $[X]$ subject to $[X]=[Z]+[X\setminus Z]$. \underline{Then} two $k$-varieties $X,Y$ are piecewise isomorphic iff $[X]=[Y]$ in $SK_0(\Var_k)$. 
	\item Let $X,Y$ $k$-varieties such that $\dim X\leq 1$. \underline{Then} $[X]=[Y]$ in $K_0(\Var_k)$ iff they are piecewise isomorphic.
	\item There exists $k$-varieties $X$ and $Y$ such that $[X]=[Y]$ in $K_0(\Var_k)$ and yet fail to be piecewise isomorphic.
	\item Let $X$ be a $k$-variety of any non-negative dimension containing only finitely many rational curves. \underline{Then} for any $k$-variety $Y$, $[X]=[Y]$ in $K_0(\Var_k)$ iff they are piecewise isomorphic.
\end{enumerate}
\end{summarytheorem}
\begin{proof} (i) appears to be folklore, and is recorded in \cite{Beke} as well as \cite[Corollary 1.4.9, Chapter 2]{CNSMotivicIntegration}. (ii) is \cite[Proposition 6]{LiuSeb}. For (iii), various constructions are now known but the first example goes back to Borisov \cite{Borisov}. (iv) is \cite[Theorem 5]{LiuSeb}.
\end{proof} %% Another example, after Borisov: arXiv:1606.04210

Summary Theorem~\ref{sumthm:KVar} sharpens our understanding of what is at stake. Given our high-level characterisation of $K$-theory as an abstract framework for analysing the finite assembly and decompositions of objects, the following question was natural:

\begin{question}[{{\cite[Question 1.2]{LL}}}] Is it true that two $k$-varieties are piecewise isomorphic iff they agree in $K_0(\Var_k)$?
\end{question}

In the setting of characteristic 0, item (iii) of the Summary Theorem answers no, signalling a loss of information on the level of $K_0$. Item (i) tells us the information is lost precisely because $K_0(\Var_k)$ involves group completion -- akin to an Eilenberg Swindle. Item (ii) tells us that piecewise isomorphism and equivalence in $K_0(\Var_k)$ coincide so long as the varieties are of sufficiently low dimension. Put otherwise, the algebraic barriers to geometric information only occur at the higher dimensions. Item (iv) is subtler, and raises interesting questions about how taking piecewise isomorphisms of complex varieties relates to the ampleness of their canonical line bundles (cf. the algebraic hyperbolicity conjecture for surfaces). %\footnote{Why? Recall that Mori proved that a Fano variety over $\mathbb{C}$ carries many rational curves due to the negativity of the canonical bundle $K_X$, which in a sense describes the ``curvature'' of the cotangent sheaf $\Omega_X$; a translation of Manin's conjecture quantifies how the ``amount'' of negativity of $K_X$ predicts the ``amount'' of rational curves.}. 

In light of this discussion, let us return to Question~\ref{qn:key}. Some promising initial progress has been made. Using the formalism of Assemblers, Zakharevich constructs an alternative $K$-theory spectrum of varieties, before leveraging its connection with Waldhausen categories to obtain a partial characterisation of $K_1(\Var_k)$ \cite[Theorem B]{ZakhK1}. Inspired by Borisov's work \cite{Borisov}, this was later developed in \cite{ZakhLefschetz} to illuminate a subtle geometric insight: the failure to extend birational automorphisms of varieties to piecewise isomorphisms is tightly connected to the Lefschetz motive $[\mathbb{A}^1]$ being a zero divisor in $K_0(\Var_k)$. In a different vein: \cite{CWZ} identifies non-trivial elements in $K_n(\Var_k)$ by lifting various motivic measures $K_0(\Var_k)\to K_0(\calC)$ to the level of spectra $K\Var_k\to K\calC$. 

\subsection*{Discussion of Main Results} Until recently, a full characterisation of any higher $K$-group of varieties was not known. In her original paper, Zakharevich \cite[Theorem B]{ZakhK1} identifies the generators of $K_1(\Var_k)$ and some key relations, but does not prove their completeness. Independently from us, an intriguing recent collaboration between algebraic topologists and experts in homological stability has uncovered a homological proof \cite[Prop. 4.1]{SrokaScissors} that Zakharevich's presentation is in fact complete. 

We take a different approach. Whereas \cite{SrokaScissors} utilises homological methods to analyse $K_1$, the present paper instead relies on techniques from simplicial homotopy theory. Further, whereas \cite{ZakhK1} relies on the connection between $\Var_k$ and Waldhausen categories, we instead focus on the (tighter) connection between $\Var_k$ and exact categories. This sets up the following theorem.


\begin{maintheorem}[Theorem~\ref{thm:Gconstruction}]\label{thm:GG} Let $\calC$ be a pCGW category, and $\calS\calC$ the simplicial set obtained by applying the $S_\bullet$-construction. Then, there exists a simplicial set $G\calC$ such that there is a homotopy equivalence
	$$|G\calC|\simeq \Omega |\calS\calC|.$$
	In particular, $\pi_n |G\calC|=K_n\calC$ for all $n$.
\end{maintheorem}

In broad strokes: Theorem~\ref{thm:GG}  extends Gillet-Grayson's $G$-construction on exact categories \cite{GG} to a wider class of categories including $\Var_k$. The beauty of the $G$-construction is that it translates a topological problem (i.e. characterising $\pi_1$ of a loop space) into a simplicial one, which is more combinatorial and easier to work with. To show that this gives us sufficient leverage to characterise $K_1$ will, of course, take the rest of the paper. Let us also remark that while one can prove Theorem~\ref{thm:GG} by adapting the original proof \cite{GG} to our setting, we provide a more streamlined argument (Theorem~\ref{thm:thmC'}) inspired by Grayson's framework of dominant functors \cite{GraysonThmC}.





The previous remarks underscore a more fundamental difference. Both \cite{ZakhK1} and \cite{SrokaScissors} are concerned with the $K$-theory of Assemblers, whereas our paper builds on \cite{CGW} to develop the $K$-theory of so-called {\em pCGW categories}.  Precise definitions will be given in due course; for now, it suffices to think of Assemblers and pCGW categories as two distinct yet equivalent ways of defining the $K$-theory spectrum of varieties.\footnote{For the curious reader: the weak equivalence of these spectra as spaces is \cite[Theorems 7.8 and 9.1]{CGW}.} This difference becomes apparent when comparing our respective presentations of $K_1(\Var_k)$. In our language: 

 \begin{maintheorem}[Proposition~\ref{prop:baseline}]\label{thm:K1} Let $\calC$ be a pCGW category. Then $K_1(\calC)$ is generated by {\em double exact squares}, i.e. by pairs of distinguished squares in $\calC$ with identical nodes
\begin{equation}\label{eq:Bl}
l:=\left(\, \dsquaref{O}{C}{A}{B}{ }{ }{f_1}{g_1} \quad,\quad  \dsquaref{O}{C}{A}{B}{ }{ }{f'_1}{g'_1} \,\right),
\end{equation}
modulo the following relations
\begin{itemize}
	\item[(B1)] $\left\langle\left(\dsquaref{O}{A}{O}{A}{}{}{}{1}\,,\,\dsquaref{O}{A}{O}{A}{}{}{}{1}\right)\right\rangle=0$;
	\item[(B2)]  $\left\langle\left(\dsquaref{O}{O}{A}{A}{}{}{1}{}\,,\,\dsquaref{O}{O}{A}{A}{}{}{1}{}\right)\right\rangle=0;$
	\item[(B3)] 
	Suppose $f_2\colon A\xrtail{f_0} B\xrtail{f_1} C$ and $f'_2\colon A\xrtail{f'_0} B\xrtail{f'_1} C$. Under technical conditions, the composition splits:
	% Under certain technical conditions, the following splitting relation holds	
		$$\!\!\!\!\!\!\!\!\!\!\!\!\!\!\!\!\!\!\!\!\!\!\small{\left\langle\left(\dsquaref{O}{\frac{B}{A}}{A}{B}{}{}{f_0}{g_0} \,,\, \dsquaref{O}{\frac{B}{A}}{A}{B}{}{}{f'_0}{g'_0} \right)\right\rangle + \left\langle\left(\dsquaref{O}{\frac{C}{B}}{B}{C}{}{}{f_1}{g_1} \,,\, \dsquaref{O}{\frac{C}{B}}{B}{C}{}{}{f'_1}{g'_1} \right)\right\rangle} = 
	\small{	\left\langle \left(\dsquaref{O}{\frac{C}{A}}{A}{C}{}{}{f_2}{g_2} \,,\, \dsquaref{O}{\frac{C}{A}}{A}{C}{}{}{f'_2}{g'_2} \right)\right\rangle}.$$
\end{itemize}
 \end{maintheorem}
\[\]


Surprisingly, this presentation appears to be new, even in the context of exact categories. How does it compare with $K_1$ of an Assembler? The following informal discussion may be illuminating.

\subsubsection*{On Generators} Double exact squares describe the breaking of an object into two distinct pieces -- for instance, Equation~\eqref{eq:Bl} shows $B$ being broken into $A$ and $C$. Interestingly, these squares generalise the usual notion of an automorphism -- see Example~\ref{ex:Aut}. By contrast, \cite{ZakhK1} shows that $K_1$ of an Assembler is generated by {\em piecewise automorphisms}, which break an object into $n$ many pieces simultaneously. This difference reflects a trade-off between simplicity vs. flexibility. Our Theorem~\ref{thm:K1} presents a simpler set of generators for $K_1(\Var_k)$ than \cite{ZakhK1}, which can be advantageous when e.g. constructing derived motivic measures, as done in \cite{CWZ}.\footnote{Technically, \cite{CWZ} views $\Var_k$ as a so-called {\em subtractive category} before applying the {\em $\widetilde{S}_\bullet$-construction} as defined in \cite{Campbell}, but this is equivalent to viewing $\Var_k$ as a CGW category and applying the $S_\bullet$-construction; see \cite[Example 7.4]{CGW}.} On the other hand, the generality of piecewise automorphisms makes the Assemblers formalism better suited for investigating e.g. scissors congruence of convex polytopes, as done in \cite{SrokaScissors}, where simultaneous decomposition is more natural.\footnote{{\em Details.} Define two polytopes $P$ and $Q$ to be scissors congruent if: (i) $P=\bigcup_{i=1}^m P_i$ and $Q=\bigcup_{i=1}^m Q_i$ such that $P_i\cong Q_i$, and (ii) $P_i\cap P_j=Q_i\cap Q_j=\emptyset$ for $i\neq j$. The key hypothesis here is convexity -- arbitrary pairwise unions $P_j \cup P_k$ may fail to form a convex polytope, so decomposition and reassembly is best done simultaneously. 
} 

\subsubsection*{On Relations} There is an interesting discrepancy between the $K_1$ relations of \cite{ZakhK1} and Theorem~\ref{thm:K1}. In Zakharevich's presentation (which we restate as Theorem~\ref{thm:ZakhB}), the composition of piecewise automorphisms always splits in $K_1$. More precisely: $$\left\langle A\xrightrightarrows[f_2]{f_1} B\right\rangle + \left\langle B\xrightrightarrows[g_2]{g_1} C \right\rangle = \left\langle A\xrightrightarrows[g_2f_2]{g_1f_1} C\right\rangle \qquad \text{in} \, K_1,$$
 where $f_i,g_i$ are piecewise automorphisms. Figure~\ref{fig:K1Split} gives an informal illustration.
	\begin{figure}[h!]
	\centering
	\includegraphics[scale=0.9]{1simp-split.pdf}
	\caption{LHS: the piecewise automorphisms induced by closed immersions $A\xrtail{f} B$ and $B\xrtail{g} C$. RHS: the piecewise automorphism induced by their composition $A\xrtail{ gf} C$.} \label{fig:K1Split}
\end{figure}  	

By contrast, in our Theorem~\ref{thm:K1}, composition only splits under a technical condition. Proposition~\ref{prop:ass} clarifies this by showing
$$\lrangles{f} + \lrangles{g}=\lrangles{g\circ f} + \lrangles{l_2}  \qquad \text{in} \, K_1,$$
with $\lrangles{l_2}$ measuring the obstruction to splitting. Importantly, this obstruction term $\lrangles{l_2}$ is non-zero in general -- see 
Example~\ref{ex:KeyExample}. Our investigations lead to Theorem~\ref{thm:Nenashev}, the final main result of our paper. This gives an alternative presentation of $K_1$, extending previous work of Nenashev \cite{Nen1}. 

\begin{maintheorem}[Corollary~\ref{cor:Nen}]\label{thm:Nenashev} Let $\calC$ be a pCGW category. Then $K_1(\calC)$ is generated by double exact squares subject to the following relations:
\begin{enumerate}[label=(N\arabic*)]
	\item $\lrangles{l}=0$ if $l$ is a pair of identical squares, say
	\begin{equation*}
	l=\left(\dsquaref{O}{C}{A}{B}{}{}{f}{g} \quad ,\quad \dsquaref{O}{C}{A}{B}{}{}{f}{g}\right).
	\end{equation*}
%	Any identical pair of exact squares will be called a \emph{diagonal} double exact square.
	\item Given a so-called optimal $\tx$ diagram
	\begin{equation*} 
	\left(\begin{tikzcd}
	X_{00}\ar[r, >->, "f_{0}"] 
	\ar[d,>->,swap,"h_{0}"] & X_{01} \ar[d, >->,"h_{1}"] & X_{02}  
	\ar[d,>->, "h_{2}"] \ar[l, {Circle[open]}->,swap,"g_{0}"]\\
	X_{10}  \ar[r,>->,"f_{1}"]  &	X_{11}  \ar[dr,phantom,"\circlearrowleft"] & X_{12} \ar[l, {Circle[open]}->,swap,"g_1"] \\
	X_{20} \ar[u, {Circle[open]}->,"j_{0}"] \ar[r,>->,"f_2"]& X_{21} \ar[u, {Circle[open]}->,"j_1"]& X_{22}  \ar[u, {Circle[open]}->,swap,"j_2"] \ar[l, {Circle[open]}->,swap,"g_{2}"]
	\end{tikzcd} \qquad\text{,}\qquad  \begin{tikzcd}
	X_{00}\ar[r, >->, "f'_{0}"] 
	\ar[d,>->,swap,"h'_{0}"]  & X_{01} \ar[d, >->,"h'_{1}"] & X_{02}  
	\ar[d,>->, "h'_{2}"] \ar[l, {Circle[open]}->,swap,"g'_{0}"]\\
	X_{10}  \ar[r,>->,"f'_{1}"]  &	X_{11}  \ar[dr,phantom,"\circlearrowleft"] & X_{12} \ar[l, {Circle[open]}->,swap,"g'_1"] \\
	X_{20} \ar[u, {Circle[open]}->,"j'_{0}"] \ar[r,>->,"f'_2"]& X_{21} \ar[u, {Circle[open]}->,"j'_1"]& X_{22}  \ar[u, {Circle[open]}->,swap,"j'_2"] \ar[l, {Circle[open]}->,swap,"g'_{2}"]
	\end{tikzcd}\right)
	\end{equation*}
	defined by the following 6 double exact squares
	\begin{equation*}
l_{i}:= \left(	\dsquaref{O}{X_{i2}}{X_{i0}}{X_{i1}}{ }{ }{ f_{i}}{g_{i} } \, \,,\,\,	\dsquaref{O}{X_{i2}}{X_{i0}}{X_{i1}}{ }{ }{ f'_{i}}{g'_{i} }\right) \qquad	l^{i}:= \left(\dsquaref{O}{X_{2i}}{X_{0i}}{X_{1i}}{ }{}{h_{i} }{ j_{i}} \, \,,\,\, \dsquaref{O}{X_{2i}}{X_{0i}}{X_{1i}}{ }{}{h'_{i} }{ j'_{i}} \right)
	\end{equation*}
for all $i\in \{0,1,2\}$, the following 6-term relation holds
	\begin{equation*}
	\lrangles{l_0} + \lrangles{l_2} - \lrangles{l_1} = \lrangles{l^0} + \lrangles{l^2} - \lrangles{l^1} \quad .
	%\lrangles{l_{0}}+ \lrangles{l_{2}} - \lrangles{l_{1}} = \lrangles{l^{0} + \lrangles{l^{2}} - \lrangles{l^{1}}
	\end{equation*}
\end{enumerate}
\end{maintheorem}

% Assemblers weak equivalences effectively hardcodes the splitting of sequences at the categorical level; This draws parallels with the classical Additivity Theorem [on the level of spectra], 

This subtle discrepancy brings into focus key differences between the two $K$-theory frameworks, particularly in their choice of weak equivalences. Zakharevich models Assemblers via a special class of Waldhausen categories whose cofibration sequences all split up to weak equivalence (see \cite[Theorem 1.9]{ZakhK1}). This may seem a strong condition, but Zakharevich relies on a non-standard choice of weak equivalences that effectively hardcodes the splitting at the categorical level. To illustrate, consider the sequence $$\ast\hookrightarrow \mathbb{P}^1\hookleftarrow \mathbb{A}^1.$$
In the pCGW setting,  where weak equivalences are simply isomorphisms, this sequence clearly fails to split since $\mathbb{P}^1\not\cong\mathbb{A}^1\coprod \{\ast\}$. By contrast, in Zakharevich's framework, it splits up to weak equivalence since 
$$\mathbb{P}^1\dashleftarrow \{\mathbb{A}^1\}, \{\ast\} \xrtail{=} \{\mathbb{A}^1\}, \{\ast\}$$
defines a weak equivalence in the sense of \cite{ZakhK1}, Definition 1.7. We discuss the implications for the $K$-theory presentation more fully in Section~\ref{sec:NAKtheory}. 

Finally, we remark that our results extend to settings not modelled by the Assemblers framework. One obvious example is exact categories. More interestingly, there is suggestive evidence that they also apply to matroids (Example~\ref{ex:Matroid}), although some details remain to be worked out (see Section~\ref{sec:ProbMatroids}).

\subsubsection*{Implications for Characterising $K_n$}

Many of the results of the present paper are inspired by Nenashev's work \cite{NenGen} characterising $K_1(\calC)$ for an exact category $\calC$. Grayson \cite{GraysonBinary} later extended this to characterise $K_n(\calC)$ for all $n$, and we expect our approach to generalise similarly to the higher $K$-groups of varieties (or, more generally, the higher $K$-groups of pCGW categories). It is currently unclear how one might analogously extend the methods from \cite{ZakhK1} or \cite{SrokaScissors}.

%%% Are there substantive connections between homological stability and the descriptions of these higher $K$-groups? We do not assert this, but this certainly seems a natural question to consider.

\subsection*{Acknowledgements} This paper benefitted greatly from discussions with C. Eppolito, M. Kamsma, A. Nanavaty, A. Nenashev, M. Sarazola, S. Vasey, C. Weibel, C. Winges, T. Wittich and I. Zakharevich -- it is a pleasure to thank them all for their thoughtful insights as well as their generosity in sharing ideas. Special thanks are due to C. Malkiewich, B. Noohi and J. Pajwani. The author also thanks M. Malliaris for encouragement, and for reminding him that a cup is not always a doughnut.

\tableofcontents

\section{Preliminaries}
% Simplicial arguments in algebraic topology often involve manipulating formal data extracted from our categories of interest. An interesting moral emerging from the emerging field of combinatorial $K$-theory is that if we can be precise about what the formal analogies are with algebraic $K$-theory, then we can perform analogous constructions and prove analogous theorems in our setting.
\subsection{CGW Categories} The key definition in \cite{CGW} is the {\em CGW category}. Informally, this is a category equipped with two subclasses of maps, $\M$ and $\E$ (analogous to admissible monos and epis in exact categories), along with a collection of square diagrams (``distinguished squares'') that encode how the $\M$ and $\E$-morphisms interact. 

This framework is presented using the language of double categories. Recall that a \emph{double category} $\calC$ is an internal category in $\mathrm{Cat}$. For our purposes, we will require the following refinement.

\begin{definition}\label{def:goodDC} A \emph{good double category} is a triple of categories $(\calC,\M,\E)$ presented by the data:
	\begin{itemize}
		\item \emph{Objects.} All three categories have the same objects: $\ob(\E)=\ob(\M)=\ob(\calC)$.
		\item \emph{Morphisms.} 
		\begin{itemize}
			\item[] \textbf{$\M$-morphisms:} $\M$ is a subcategory of $\calC$. Its morphisms are denoted $\rtail$.
			\item[] \textbf{$\E$-morphisms:} Either $\E$ or $\E^{\opp}$ is a subcategory of $\calC$. Its morphisms are denoted $\otail$. %\footnote{And so $A\otail C$ either corresponds to $A\to C$ in $\calC$ (if $\E\subseteq \calC$) or $C\to A$ (if $E^\opp\subseteq \calC$).} 
		\end{itemize}
		\item \emph{Distinguished Squares.} A collection of square diagrams, denoted
	\[	\begin{tikzcd} 
A \ar[r, >->,"f'"]\ar[d, swap, {Circle[open]}->,"g'"] \ar[dr, phantom, "\square"]& B \ar[d, {Circle[open]}->,"g"]\\
C \ar[r, >->,"f"] & D
\end{tikzcd}\]
	where $f,f'\in \M$ and $g,g'\in \E$. These squares are closed under horizontal and vertical composition. 
\smallskip
In addition, we require:
\begin{itemize}
	\item[\(\diamond\)] {\bf Commutativity.} Each distinguished square is a commutative diagram in $\calC$ (after adjusting for whether $\E$ or $\E^\opp$ is taken as a subcategory). 
	\item[\(\diamond\)] {\bf Closure under isomorphisms.} Any commutative square in $\calC$ with the indicated shape above is distinguished whenever either {\em both} $\M$-morphisms or {\em both} $\E$-morphisms are isomorphisms. 
\end{itemize}
	\end{itemize}
\end{definition}

\begin{convention}%[Ambient vs. Double Category] 
	We denote a good double category by $\calC=(\M,\E)$, though we often abbreviate to $\calC$ when no confusion arises.  When $\calC$ is viewed as an ordinary 1-category, we call it the \emph{ambient category}, emphasising that it contains $\M$ and either $\E$ or $\E^{\opp}$ as subcategories.  
	In particular:
	\[
	A \xotail{g} B \;=\;
	\begin{cases}
	A \xrightarrow{g} B & \text{if } \E\subseteq\calC,\\
	B \xrightarrow{g} A & \text{if } \E^{\opp}\subseteq\calC.
	\end{cases}
	\]
\end{convention}

We now introduce a couple of helper definitions, before defining a CGW category. 

\begin{definition}\label{def:help} Let $C=(\M,\E)$ be a good double category, and $\D$ be any (ordinary) category.
\begin{enumerate}
\item Define $\Ars\E$ 
	\begin{itemize}
		\item Objects: Morphisms $A\otail B$ in $\E$.
		\item Morphisms:
		$\Hom_{\Ars\E}(A\xotail{ g} B, A'\xotail{g'} B')=
		\left\{\begin{tabular}{c}
		\hbox{distinguished} \\  \hbox{squares}
		\end{tabular}
		\hbox{$\dsquaref{A}{A'}{B}{B'}{}{g}{}{g'}$}\right\}.$	 
	\end{itemize}
$\Ars\M$ is defined analogously.
\item Define $\Art\D$ 
\begin{itemize}
	\item Objects: Morphisms $A\rightarrow B$ in $\D$.
	\item Morphisms: 
	$\Hom_{\Art\D}(A\xrightarrow{\,\,\,f} B, A'\xrightarrow{\,\,\,f'} B')=\left\{      \begin{tabular}{c}
	commutative \\ squares
	\end{tabular}\, \begin{tikzcd}
	A \ar[r,"\cong"] \ar[d,"f",swap] \ar[dr,phantom,"\circlearrowleft"] & A' \ar[d,"f'"]\\
	B \ar[r] & B'
	\end{tikzcd} \right\}.$	 
\end{itemize}
\end{enumerate}
	
\end{definition}



\begin{definition}[CGW Category]\label{def:CGWcat} A \emph{CGW category} $(\calC,\varphi,c,k)$ consists of the following data
	\begin{itemize}
		\item A good double category $\calC=(\M,\E)$;
		\item An isomorphism of categories $\varphi\colon \mathrm{iso}\M\to \mathrm{iso}\E$ which is identity on objects;
		%Two subcategories $\mathrm{iso}\M\subseteq \M$ and $\mathrm{iso}\E\subseteq \E$ whose morphisms are the isomorphisms of $\calC$;


	\item An equivalence of categories 
		$$k\colon \Ars\E\longrightarrow\Art\M \qquad \qquad \qquad\text{and}\qquad\qquad\quad  c\colon \Ars\M\to \Art\E$$
		\[ \begin{tikzcd} A \ar[r,>->,"f"] \ar[dr,phantom,"\square"] \ar[d,{Circle[open]->},swap,"g"] & A' \ar[d,{Circle[open]->},"g'"] \\
		B \ar[r,>->,"f'",blue]  & B'
		\end{tikzcd} \longmapsto 
		\begin{tikzcd} \ker(g) \ar[d,>->,swap,"k(g)"] \ar[dr,phantom,yshift=0.4em,xshift=-0.2em,"\circlearrowleft"] \ar[r,>->,"\cong"]& \ker(g') \ar[d,>->,"k(g')"]\\
		B \ar[r,>->,blue,"f'"]& B'
		\end{tikzcd} \qquad \qquad\quad \begin{tikzcd} A \ar[r,>->,"f"] \ar[dr,phantom,"\square"] \ar[d,{Circle[open]->},swap,"g"] & A' \ar[d,{Circle[open]->},violet,"g'"] \\
		B \ar[r,>->,"f'"]  & B'
		\end{tikzcd} \longmapsto 
		\begin{tikzcd} A' \ar[d,{Circle[open]->},violet,"g'",swap] \ar[dr,phantom,yshift=0.1em,xshift=0.2em,"\circlearrowleft"] &\coker(f) \ar[d,{Circle[open]->},"\cong"]\ar[l,{Circle[open]->},swap,"c(f)"]\\
		B' & \coker(f') \ar[l,{Circle[open]->},swap,"c(f')"]
		\end{tikzcd}  \]
		\end{itemize}
		satisfying the axioms:
		\begin{itemize}
			\item[(Z)] \emph{Basepoint object.} $\calC$ contains an object $O$ initial in both $\E$ and $\M$.
			\item[(I)] \emph{Isomorphisms.} Let $\psi\colon A\to B$ be an isomorphism in ambient category $\calC$. Then: 
			\begin{itemize}
				\item $\psi$ belongs to $\mathrm{iso}\M$, which we denote suggestively as $\psi\colon A\rtail B$. 
				\item If $\E$ is a subcategory of $\calC$, then $\varphi(\psi)\colon A\otail B$ corresponds to $\psi\colon A\to B$ in $\calC$.
				\item If $\E^\opp$ is a subcategory of $\calC$, then $\varphi(\psi)\colon A\otail B$ corresponds to $\psi^{-1}\colon B\to A$ in $\calC$.
			\end{itemize}
			 \item[(M)] \emph{Monicity.} Every $\M$ and $\E$-morphism is monic. Furthermore:
			 \begin{itemize}
			 	\item  Every $\M$-morphism is monic in $\calC$. 
			 	\item If $\E\subseteq \calC$, then every $\E$-morphism is monic in $\calC$; otherwise if $\E^\opp\subseteq \calC$ then every $\E^\opp$-morphism is epi in $\calC$.
			 \end{itemize}
			 \item[(K)] \emph{Formal kernels and cokernels.} For any $f\colon A\rtail B$ in $\M$, there exists a \emph{formal cokernel}, denoted $c(f)\colon\coker (f)\otail B,$ 
		with a distinguished square as below left.
			 \begin{equation*}
			 \dsquaref{O}{\coker(f)}{A}{B}{ }{ }{f}{c(f)} \qquad \dsquaref{O}{A}{\ker(g)}{B}{ }{ }{k(g)}{g}
			 \end{equation*}
			 Dually, for any $g\colon A\otail B$ in $\E$, there exists a \emph{formal kernel}, denoted $k(g)\colon \ker(g)\rtail B$, with a distinguished square as above right. 
			 
Formal cokernels are unique up to (codomain-preserving) isomorphism: if $f'\colon C\otail B$ is another $\E$-morphism with distinguished square
			 \begin{equation*}
			 \dsquaref{O}{C}{A}{B}{ }{ }{f}{f'}, 
			 \end{equation*}
			 then there exists an isomorphism $\gamma\colon \coker(f)\rtail C$ such that the rightmost square in
			 \begin{equation*}
			 \begin{tikzcd}
		O \ar[dr,phantom,"\square"] \ar[d,{Circle[open]}-> ] \ar[r,>->]& \coker(f)\ar[dr,phantom,"\square"] \ar[d,{Circle[open]}->,"c(f)"] \ar[r,>->,"\gamma"]& C \ar[d,{Circle[open]}->,"f'"]\\
		A \ar[r,>->,swap,"f"]& B \ar[r,>->,swap,"1"] & B
			 \end{tikzcd}
			 \end{equation*}
is distinguished. We sometimes call $c(f)$ the {\em canonical quotient} of $f$. Formal kernels are unique in the analogous sense.
% commutes when regarded as a diagram in the ambient category $\calC$. Notice the rightmost square is also distinguished since distinguished squares are closed under isomorphisms. [Version before requiring distinguished squares to commute: ]
 
			 \end{itemize}
\end{definition}

\begin{remark}\label{rem:INV} The $\E$-morphisms of a distinguished square have isomorphic formal kernels and its $\M$-morphisms have isomorphic formal cokernels -- this follows from $c$ and $k$ mapping distinguished squares to the morphisms of $\Ar_{\triangle}$. Thus, by Axiom (K), $c$ and $k$ are inverse on objects (up to isomorphism).\footnote{\label{fn:INV} 
	Strictly speaking, $c$ and $k$ are only inverses up to codomain-preserving isomorphism, not inverses on the nose. % -- for instance, $\frac{B}{A}$ may have more automorphisms than $A$ (and so the isomorphism class of formal cokernels may be bigger than the class of formal kernels). 
In practice, however, eliding this distinction is usually harmless since distinguished squares commute and are closed under isomorphisms.} %%% Why stable under isomorphism? They are closed under isomorphisms + compose.
\end{remark}

\begin{comment} An alternative proof that $\M$-morphisms of distinguished squares have isomorphic quotients. Consider a distinguished square
\begin{equation*}
\dsquaref{A}{B}{C}{D}{f}{}{g}{} \quad,
\end{equation*}
we have $\frac{B}{A}\cong \frac{D}{C}$. Since distinguished squares compose vertically, the following square
\begin{equation*}
\dsquaref{O}{\frac{B}{A}}{C}{D}{}{}{g}{}
\end{equation*}
is distinguished. Since formal cokernels are unique (up to isomorphism), conclude that $\frac{B}{A}\cong \frac{D}{C}$.
\end{comment}

There is a natural notion of structure-preserving functors and subcategories in the CGW context. A {\em CGW functor} of CGW categories is a double functor 
$$F\colon (\M,\E)\to (\M',\E')$$
that commutes with the functors $c$ and $k$, as follows:
\[\begin{tikzcd}
\Ars\M \ar[r,"c"] \ar[d,swap,"\Ars F"]& \Art \E \ar[d,"\Art F"] \\
\Ars\M' \ar[r,"c'"]& \Art \E' 
\end{tikzcd}\qquad \begin{tikzcd}
\Ars\E \ar[r,"k"] \ar[d,swap,"\Ars F"]& \Art \M \ar[d,"\Art F"] \\
\Ars\E' \ar[r,"k'"]& \Art \M' 
\end{tikzcd} \quad .\] 
For a CGW category $(\calC,\varphi,c,k)$, a {\em CGW subcategory} is a sub-double category $\calD\subseteq \calC$ such that the obvious restrictions $(\calD,\phi|_{\calD}, c|_{\calD}, k|_{\calD})$ forms a CGW category. %That is, the structure maps on $\calC$  restrict to define a CGW category on $\calD$. 


\begin{convention} When context permits, we suppress $(\varphi,c,k)$ and simply write $\calC=(\M,\E)$ for a CGW category, or just $\calC$. 
\end{convention}

We now discuss Axioms (K) and (I) in more detail below, followed by some illustrative examples.

\subsubsection*{Quotients in CGW Categories} CGW categories are agnostic about whether formal cokernels arise from taking quotients in the {\em additive} setting (e.g. $R$-modules) or taking complements in the {\em non-additive setting} (e.g. varieties). Either way, the formal properties remain consistent. To reinforce this perspective, we adopt the following suggestive convention.


\begin{convention}[``Quotient''] We typically denote the formal cokernel of $f\colon A\rtail B$ as $\frac{B}{A}$, whenever the map $f$ is clear from context, which we refer to as a  \emph{quotient}.
	  %This is, of course, an abuse of language, but this is justified by our framework which makes precise how e.g. open complements of closed immersion of varieties behave formally like quotients of abelian groups.
\end{convention}

The following property about quotients, made possible by Helper Definition~\ref{def:help}, will play a key role in our analysis.

\begin{lemma}[Quotients respect Filtrations]\label{lem:quotFilt} Let  $$P_{0}\xrtail{f_1} P_{1}\xrtail{g_1} P_{2}$$
be a sequence of $\M$-morphisms in a CGW category. Choose formal quotients 
$$(f_2,g_2,h_2)$$
corresponding to the $\M$-morphisms $(f_1,g_1\circ f_1,g_1)$. Then, there exists a diagram of distinguished squares
\begin{equation}\label{eq:quot-filt}
\begin{tikzcd}
P_0 \ar[r, >->,"f_1"] \ar[dr,phantom,"\square"] & P_1 \ar[dr,phantom,"\square"] \ar[r, >->,"g_1"] & P_2 \\
O \ar[r,>->] \ar[u,{Circle[open]}->]&	P_{1/0} \ar[r, >->,"j_1"] \ar[u, {Circle[open]}->,"f_2"] \ar[dr,phantom,"\square"]  & P_{2/0}\ar[u, {Circle[open]}->,"g_2"] \\
& O \ar[r,>->] \ar[u,{Circle[open]}->]& P_{2/1} \ar[u, {Circle[open]}->,"j_2"] 
\end{tikzcd}	\qquad ,
\end{equation}
such that $h_2=g_2\circ j_2$. 
\end{lemma}
\begin{proof} This adapts \cite[Lemma 2.10]{CGW}. Notice $g_1$ induces a morphism
	$$(P_0\rtail P_1) \longrightarrow (P_0\rtail P_2) \in\Ar_{\triangle}\M\,.$$
Applying the functor $k^{-1}\colon \Ar_{\triangle}\M\to \Ar_{\square}\E$ constructs the top right square of Diagram~\eqref{eq:quot-filt}. In general, the quotients $k^{-1}(P_0\xrtail{f_1} P_1)$ and $k^{-1}(P_0\xrtail{g_1\circ f_1}P_2)$ need not coincide with our chosen $(f_2,g_2)$. Nonetheless, by Axiom (K), all choices of quotients are canonically related by codomain-preserving isomorphisms. Hence, since distinguished squares commute and are closed under isomorphism, we may identify these choices (cf. Footnote~\ref{fn:INV}). Extending this argument, we may also choose $j_2\colon P_{2/1}\otail P_{2/0}$ such that $h_2=g_2\circ j_2$. 
\end{proof}

\begin{comment}
Previous sketch of proof: First, apply Axiom (K) to obtain distinguished squares
\[\dsquaref{O}{P_{1/0}}{P_0}{P_1}{}{ }{f_1}{f_2} \qquad \dsquaref{O}{P_{2/0}}{P_0}{P_2}{ }{}{g_1f_1}{g_2} \qquad  \dsquaref{O}{P_{2/1}}{P_1}{P_2}{}{ }{g_1}{} .\] 
Notice that $P_0\rtail P_1\rtail P_2$ yields a morphism 
$$(P_0\xrtail{f_1} P_1) \xrtail{g_1} (P_0\xrtail{g_1f_1} P_2)$$
in $\Ar_{\Delta}\M$. Applying $k^{-1}$ and Axiom (K), this yields the distinguished square 
\[\dsquaref{P_{1/0}}{P_{2/0}}{P_{1}}{P_{2}}{h_1}{f_2 }{g_1}{g_2}.\] %%% Why the bottom arrow is $g_1$? We didn't say how the morphisms got mapped under $k$ and $c$, but it is the obvious way. This is also implicit in Lemma 2.9 of \cite{CGW}
This in turn can be interpreted as a morphism between $h_1$ and $g_1$ in $\Ar_\square \M$. Applying $c$ yields an isomorphism between $c(P_{1}\xrtail{g_1} P_2)=P_{2/1}$ and $c(P_{1/0}\xrtail{h_1}P_{2/0})$, and so this gives the bottom square.
\end{comment}



\subsubsection*{Isomorphisms in CGW Categories} In their original definition, CGW categories were not required to be good double categories. Our reason for invoking goodness is to make the interaction between distinguished squares and isomorphisms more precise. This not only simplifies the original Axiom (I) in \cite[Definition 2.5]{CGW}, but also provides the technical control needed for our simplicial arguments to work smoothly (e.g. Lemmas~\ref{lem:SherLoopAdd} and~\ref{lem:SherLoopSplit}).

\begin{comment}
	In addition, goodness allows us to streamline and generalise the original Axiom (I) in \cite[Definition 2.5]{CGW} as follows. 
	
\begin{observation} If $A' \xrtail{f} A$ and $B'\xrtail{f'} B$ are both isomorphisms, and $A\xotail{g} B$ is a morphism in $\E$, then  
	\begin{equation*}
	\begin{tikzcd}
	A' \ar[dr, phantom, "\square"]\ar[d, {Circle[open]}->,swap,"\varphi(f'^{-1})\circ g\circ \varphi(f)"]\ar[r,>->,"f"] & A\ar[d, {Circle[open]}->,"g"] \\
	B' \ar[r,>->,"f'"]& B
	\end{tikzcd}
	\end{equation*}
	is distinguished. Dually, if $A'\xotail{g'} B'$ and $A\xotail{g} B$ are isomorphisms, and $A\rtail B$ is a morphism in $\M$, then
	\begin{equation*}
	\begin{tikzcd}
	A' \ar[drrr, phantom, "\square"]\ar[d, {Circle[open]}->,swap,"g'"]\ar[rrr,>->,"\varphi^{-1}(g^{-1})\circ f\circ \varphi^{-1}(g')"] &&& A\ar[d, {Circle[open]}->,"g"] \\
	B' \ar[rrr,>->,"f"]&&& B
	\end{tikzcd}
	\end{equation*}
	is also distinguished.	
\end{observation}

The proof is straightforward, and we leave it to the reader. 
\end{comment}
Ultimately, this is no real loss in generality since our definition still covers all the major examples in the original paper \cite{CGW}. The underlying reason for this is the following observation.

\begin{observation}\label{obs:goodISOs} Let $\calC=(\M,\E)$ be any CGW category. Suppose 
\begin{equation}\label{eq:PBPODistSq}
\begin{tikzcd} 
A \ar[r, >->,"f'"]\ar[d, swap, {Circle[open]}->,"g'"] & B \ar[d, {Circle[open]}->,"g"]\\
C \ar[r, >->,"f"] & D
\end{tikzcd}
\end{equation}
defines a commutative diagram in the ambient category $\calC$. If $f$ and $f'$ are isomorphisms, then Diagram~\eqref{eq:PBPODistSq} defines a pullback and a pushout in $\calC$. The same holds if $g$ and $g'$ are isomorphisms. 
\end{observation}
\begin{comment}
\begin{proof} Suppose
	\[	\begin{tikzcd} 
	A \ar[r, >->,"f'"]\ar[d, swap, {Circle[open]}->,"g'"] & B \ar[d, {Circle[open]}->,"g"]\\
	C \ar[r, >->,"f"] & D
	\end{tikzcd}\]
	defines a commutative diagram in the ambient category $\calC$. We first want to show that if either $f'$ and $f$ are isomorphisms, or $g$ and $g'$ are isomorphisms, then this diagram defines a pullback square in the ambient category. There are two main cases to check.%\footnote{\MING{Streamline/comment out.}}
	\begin{itemize}
		\item \textbf{Case 1:} {\em $\E^{\opp}$ is a subcategory of $\calC$.} In which case, consider the following diagram in $\calC$ 
		\[\begin{tikzcd}
	R  \ar[ddr,bend right,"j_0"] \ar[dr,dashed, "\exists"]\ar[rrd,bend left,"j_1"]\\
	&	C \ar[r, "f"] \ar[d, swap,"g'"]& D\ar[d, "g"]\\
		&A  \ar[r,"f'"] & B \\
		\end{tikzcd}\] 
		where the solid arrows define a commutative diagram. Notice the reversal of the vertical arrows. Suppose $f$ and $f'$ are isomorphisms.
		Then, we claim that the map $\exists:=f^{-1}\circ j_1$ is the unique map making the whole diagram commute. Commutativity follows from checking  $j_0=g'\circ \exists$ and $j_1 = f\circ \exists$. The second identity is obvious. The first identity follows from computing
		\begin{align*}
		g\circ f = f'\circ g' & \implies (f')^{-1}\circ g\circ f\circ \exists= g'\circ \exists\\
		&\implies (f')^{-1}\circ g \circ j_1 = g'\circ \exists \\
		& \implies (f')^{-1}\circ f'\circ j_0 = g'\circ \exists\\
		&\implies j_0=g' \circ \exists.
		\end{align*}
		Uniqueness follows from noting that any other map $l\colon R\to C$ making the diagram commute must satisfy $j_1=f\circ l$ and thus $f^{-1}\circ j_1=l.$ 

		The case when $g$ and $g'$ are isomorphisms follows by symmetry. 
		\item \textbf{Case 2:} {\em $\E$ is a subcategory of $\calC$.} Analogous to Case 1. 
	\end{itemize}
In summary: we have shown pullback squares are closed under isomorphisms. The argument for pushout squares is entirely analogous, and in fact was already worked out in \cite[Lemma 4.3]{5autManifolds}.
\end{proof}
\end{comment}
We omit the routine proof; see e.g. \cite[Lemma 4.3]{5autManifolds}. Observation~\ref{obs:goodISOs} essentially restates the obvious: pullback and pushout squares are closed under isomorphisms. As we shall see, this will be useful for verifying goodness.


\subsubsection*{Examples} We review some standard examples of CGW categories, plus two new ones; some further details can be found in \cite[\S 4]{CGW} and \cite[\S 3]{SarazolaShapiro}.


\begin{example}[Exact Categories]\label{ex:Exact} For an exact category $\calC$, define a CGW category $\calC=(\M,\E)$ by setting
$$\M=\{\text{admissible monomorphisms}\}\qquad \E=\{\text{admissible epimorphisms}\}^\opp.$$	
The basepoint object is the zero object in $\calC$. Notice by taking the opposite category, the zero object becomes initial in $\E$, as required by Axiom (Z).
 %\footnote{Notice that we take the opposite category for $\E$, and so the zero object is initial in $\E$ as required by Axiom (Z).} 
 The distinguished squares are the biCartesian squares (= both pushouts and pullbacks in the ambient category $\calC$). By Observation~\ref{obs:goodISOs}, $\calC=(\M,\E)$ is a good double category. The equivalences $k$ and $c$ map admissible epis to kernels and admissible monos to cokernels, respectively. 
 % For details on the other CGW axioms, see \cite[Example 3.1]{CGW}. 

\end{example}
\begin{comment} Why does $c$ and $k$ map distinguished squares to the required morphisms in $\Ar_{\triangle}$? 

For $c$, this is \cite[Prop. 2.12]{Buhler}. As for $k$, this is essentially because biCartesian squares compose. To see why: 

\[\begin{tikzcd} 
 \ker p \ar[r,->>] \ar[dr,phantom,"\square"] \ar[d,>->,swap,"f"] & O \ar[d,>->]\\
 B \ar[r,->>,"p"] \ar[d,>->,swap,"q"] \ar[dr,phantom,"\square"]  & C \ar[d,>->,"r"]\\
 B' \ar[r,->>,swap,"p'"] & C'
\end{tikzcd}\]

To show $k$ is well-defined, it suffices to show that $\ker p$ is the kernel of $p'$, but this follows from the fact that biCartesian squares compose. 
\end{comment}



\begin{example}[Finite Sets]\label{ex:FinSet} Given $\mathrm{FinSet}$, define a CGW category $\mathrm{FinSet}=(\M,\E)$ by setting 
	$$\M=\E=\{\text{injections}\}.$$
The basepoint object $O$ is the empty set, the distinguished squares are the pushout squares. By Observation~\ref{obs:goodISOs}, this defines a good double category. The equivalences $c$ and $k$ are given by taking any inclusion $A\hookrightarrow B$ to the inclusion $B\setminus A\hookrightarrow B$. 

\begin{comment}
Why are functors $c$ and $k$ well-defined? Consider

\[\dsquare{A}{B}{C}{D}.\] 
$B=(B\setminus A) \cup A$, and $C=(C\setminus A)\cup A$. Since $D=(B\setminus A)\cup A\cup (C\setminus A)$, this gives 
$$D\setminus C = B\setminus A.$$
The rest follows by applying the fact that distinguished squares compose.
\end{comment}
\end{example}

\begin{example}[Varieties]\label{ex:Var} Given $\Var_k$, define a CGW category $\Var_k=(\M,\E)$ by setting
	$$\M=\{\text{closed immersions}\} \qquad \E=\{\text{open immersions}\}.$$
The basepoint object $O$ is the empty variety. The distinguished squares are the pullback squares
	\[\dsquaref{A}{B}{C}{D}{}{}{f}{g}\]
in which $\im f\cup \im g= D$; notice this implies goodness of $\calC$.\footnote{In particular, we get $\im f\cup \im g= D$ for free if either f or g are isomorphisms.} Axioms (I) and (M) follow from standard properties of closed and open immersions. For Axiom (K), let $c$ and $k$ take a morphism to the inclusion of its complement. In particular, their corresponding distinguished squares are unique due to the uniqueness of reduced scheme structure\footnote{The key hypothesis here is that varieties are {\em reduced}. For general schemes, enforcing Axiom (K) is obstructed by the fact that a closed subset may admit different structure sheaves -- see e.g. \cite[Example 3.2.6]{Hartshorne}. %% See also Exercise 3.11, which gives the universal property of reduced scheme structure on closed sets, they factor through any other closed subscheme inclusion with the same underlying subspace.
} on locally closed subsets \cite[Tag 01J3]{stacks-project}. Finally, to see why $c$ and $k$ define the required functors in Definition~\ref{def:CGWcat}, use the fact that distinguished squares compose. 

\begin{comment} Some details.

Axiom (I) holds by definition. Axiom (M) is verified by noting that open and closed immersions satisfy base-change in the category of varieties. For technical details, see https://stacks.math.columbia.edu/tag/042K

 Why Axiom (K)? Fairly obvious that $c$ and $k$ as defined yield a distinguished square. To see why complements are unique in the sense described, there is a slight technicality in that closed immersions are only required to be surjective on the structure sheaf, not necessarily isomorphic. There also exists closed subschemes with the same underlying topological space but not the same structure sheaves. But this is OK if we restrict to reduced subschemes.
 
Some useful links:   https://math.stackexchange.com/questions/374944/questions-on-reduced-induced-closed-subscheme?rq=1
https://math.stackexchange.com/questions/4090464/why-is-the-reduced-scheme-structure-on-a-locally-closed-subset-unique?rq=1
 https://stacks.math.columbia.edu/tag/01J3
 
 An example of a non-reduced closed subschemes same underlying topological space, but (apparently) differents structure sheaves, see:
 https://math.stackexchange.com/questions/4490412/does-every-closed-subset-in-a-scheme-correspond-to-a-unique-closed-subscheme
 
 Finally, let's check functoriality. Given a distinguished 
\[\dsquare{A}{B}{C}{D},\] 
composing pullback squares yields
 \[\begin{tikzcd} 
 O\ar[r, >->] \ar[dr,phantom,"\square"] \ar[d,{Circle[open]->}] &  \frac{B}{A} \ar[d,{Circle[open]->},"g"] \\
 A \ar[r,>->] \ar[d,{Circle[open]->}] \ar[dr,phantom,"\square"]  & B \ar[d,{Circle[open]->},"h"] \\
 C\ar[r,>->,"f"] & D
 \end{tikzcd}\quad.\]
 One would therefore like to say that this defines a morphism
 $$\frac{B}{A}\otail B \to \frac{B}{A}\otail D \in \Ar_{\triangle}\E$$
 except we do  not know if the $\frac{B}{A}\otail B\otail D$ corresponds to $\coker(f)$ itself. Nonetheless, it is still isomorphic to it, which gives the LHS diagram
 $$\begin{tikzcd}
 \coker(f) \ar[r,>->,"\gamma"] \ar[d,{Circle[open]->},"\zeta",swap] \ar[dr,phantom,"\circlearrowleft"] & \frac{B}{A} \ar[d,{Circle[open]->},"h\circ g"] \\
 D \ar[r,>->,"1"] & D
 \end{tikzcd} \qquad \begin{tikzcd}
 B \ar[d,{Circle[open]->},"h"] & \frac{B}{A} \ar[d,{Circle[open]->},"\gamma^{-1}"] \ar[l,{Circle[open]->},"g"]\\
 D & \coker(f) \ar[l,{Circle[open]->},"\zeta"]
 \end{tikzcd}$$
 In other words $(h\circ g)\circ \gamma = \zeta \iff h\circ g = \zeta\circ\gamma^{-1}$, which gives the RHS diagram.
 
An alternative argument, if we only care about the objects: to show that $\frac{B}{A}$ is isomorphic to the cokernel of $f$, it remains to check that $D=C\cup \frac{B}{A}$. But this follows from the fact that $B=\frac{B}{A}\cup A$ and $A\subseteq \calC$. And so $D\setminus C \subseteq B\setminus A$. 
 To show why $\frac{C}{A}\cong \frac{D}{B}$ for the kernels, the same reasoning applies. Consider
 \[\begin{tikzcd} 
O  \ar[r,>->] \ar[d,{Circle[open]->}] & A \ar[r,>->] \ar[d,{Circle[open]->}] \ar[dr,phantom,"\square"]  & B \ar[d,{Circle[open]->}] \\
\frac{C}{A} \ar[r,>->] & C\ar[r,>->,"f"] & D
\end{tikzcd}\]
In other words, given the distinguished square, the morphsim $f$ yields a morphism in $\Ar_{\triangle}\M$ whereby
$$(\frac{C}{A}\rtail C) \xrtail{f} (\frac{C}{A}\rtail D).$$
\end{comment}
\end{example}

For those interested in model theory, the following example will be suggestive.

% Notice: we only need formulas for definable sets, expressible in the language; not necessarily a theory.
\begin{example}[Definable Sets] Fix $\Sigma$ to be a first-order language and $M$ a $\Sigma$-structure. The term {\em definable} will always mean with parameters from $M$. Following \cite{KrajicekScanlon}, denote $\mathrm{Def}(M)$ as the category with objects the definable sets of $M$ and its powers $M^n$, and morphisms the definable functions.\footnote{Alternatively: one can work syntactically and define definable sets as functors from the category of $\thT$-models to $\Set$, where $\thT$ is some fixed first-order theory. Zakharevich defined its $K$-theory via her Assemblers framework in \cite[Example 3.4]{ZakhAss}.} We upgrade this to a CGW category by setting
	$$\M=\E=\{\text{definable injections}\}.$$ 		
Axioms (I) and (M) are thus satisfied by definition. For Axiom (Z), set the basepoint object $O$ as $\emptyset$. Define the distinguished squares as the pullback squares
\[\dsquaref{A}{B}{C}{D}{}{}{f}{g}\]
where $\im f \cup \im g=D$. %% Definable sets are closed under finite Boolean operations.
Finally, any definable injection $f\colon C\rtail D$ can be mapped to its formal cokernel 
\[\dsquaref{O}{D\setminus f(C) }{C}{D}{}{}{f}{c(f)}\]
where $c(f)$ is the obvious inclusion of the complement. The same holds for the formal kernels.
\begin{comment} Details.

The general intuition is that we may, informally, just treat the definable sets as if they were sets when it comes to verifying the structural properties. 

To see why $D\setminus f(C)$ is definable, recall $f$ being definable means $\Gamma_f:=\{ (x,y)\in C\times D\mid f(x)=y\}$. In particular, this means $f(C)$ is definable [since definable sets are closed under projection], and thus so is $D\setminus f(C)$. We can let $g$ be the obvious inclusion. Uniqueness of the corresponding distinguished squares corresponds to the uniqueness of Boolean complements. 

Let us double-check that it is unique up to {\em definable} isomorphism.

 Notice that any other definable subsets such that 
$$\dsquaref{O}{X}{C}{D}{}{}{f}{g}$$
gives the image $g(X)=D\setminus f(C)$. So we may define a function 
$$\gamma \colon D\setminus f(C)\to X$$
 that sends $y\in D\setminus f(C)$ to $x\in X$ such that $g(x)=y$. To show that this defines a definbale isomorphism, check:
 \begin{itemize}
 \item Definable -- $\gamma$ as written is a definable relation on $ D\setminus f(C)\times X$.
 \item Single-valued, i.e. $\exists! x\in X$ such that $\gamma(y)=x$. Suppose $\gamma(y)=x$ and $x'$. Then $g(x)=y=g(x')$, and so $x=x'$ by injectivity of $g$.
 \item Well-defined, i.e. $x=y\implies \gamma(x)=\gamma(y)$. We know $g\circ \gamma(x)= x$ and $g\circ \gamma (y) = y$. And so if $x=y$, then $g\circ \gamma (x)=g\circ \gamma (y)\implies \gamma (x)=\gamma(y)$, again by injectivity.
 \item Total -- immediate from our comment about the image $g(X)=D\setminus f(C)$. 
 \item Injective: Suppose $\gamma(x)=\gamma(y)$. Then $g\circ \gamma (x)=x$ and $g\circ \gamma (y)=y$, and so $x=y$.
 \item Surjective: again, this follows from $g(X)=D\setminus f(C)$.
 \end{itemize}

And so $\gamma$ as defined is a definable isomorphism between the $D\setminus f(C)\to X$. In fact, by construction, we have that $g\circ \gamma=c(f)$, where $c(f)$ is the original inclusion.

So uniqueness is OK.


As for why this extends to the functors $c$ and $k$ in the definition, again -- just as in the case of varieties -- apply the fact that the distinguished squares compose. 
\end{comment}
\end{example}


\begin{remark}\label{rem:logic} For the curious non-logician: the category of definable sets $\mathrm{Def}(M)$ can be viewed as an abstraction of $\Var_k$. Consider $\mathbb{C}$ as an algebraically closed field with characteristic 0. In which case, the objects of $\mathrm{Def}(\mathbb{C})$ are the constructible subsets of complex affine $n$-spaces \cite[Corollary 3.2.8]{DMarker}. In fact, there exists an isomorphism of Grothendieck rings $K_0(\Var_\C)\cong K_0(\mathrm{Def}(\C))$ \cite{Beke} -- this indicates how logic may pull the cut-and-paste combinatorics away from the underlying algebraic geometry.
%% It was claimed by Pillay here  https://www.newton.ac.uk/files/seminar/20050329113012301-148953.pdf that definable subsets over the prime field $\mathbb{Q}$ of \ACF_0 yield the affine varieties, and allowing parameter from $\mathbb{C}$ yields the full category of algebraic varieties. This doesn't seem to match with what Marker has written; also, the full category of algebraic varieties requires gluing, and it's not clear to me how this happens -- although there is some basis of comparison between the category of varieties and C-definable sets because of Beke's result. 
\end{remark}



Finally, let us mention another generalisation of exact categories: {\em proto-exact categories}, introduced by Dyckerhoff-Kapranov \cite{HigherSegal}. A particularly challenging example comes from \cite{ProtoExactMatroids}, which showed that the category of pointed matroids form a proto-exact category, yielding a $K$-theory spectrum.

 Informally, a  {\em matroid} abstracts the notion of linear independence. It consists of a finite set $E$ and a collection of subsets (``{\em flats}'') that are closed under dependency -- akin to subspaces in a vector space.\footnote{For those interested in homological stability: compare this, perhaps, with the perspective from Tits buildings.} Matroids bridge combinatorics and geometry, and have surprisingly deep links to algebraic geometry and Hodge Theory \cite{Katz,BakerHodgeCombinatorics}. It is therefore very interesting that they also define a CGW category.

\begin{comment} We should be able to generalise this to matroids, not necessarily pointed. Unlike proto-exact categories, CGW categories do not require a zero object.
\end{comment}

\begin{comment}
Intuition: "B covers A", this is a kind of minimality condition. It says that B is the smallest thing that's strictly bigger than A. The final condition of flats tell us that there is a coherence between all the B covering A: that the "strictly bigger than A" bit forms a partition of E\setminus A. Thinking about vectorial matroids is helpful for illuminating the intuition behind the definitions.
\end{comment}

\begin{example}[Matroids]\label{ex:Matroid} Let $M=(E,\calF,\bullet_M)$ be a {\em pointed matroid}, where $E$ is a finite set, and $\calF\subseteq 2^{E}$ the set of flats of matroid $M$ and $\bullet_M$ the distinguished base-point. A {\em strong map} of pointed matroids $f\colon M\to N$ is a function $f\colon E_M\to E_N$ such that $f(\bullet_M)=\bullet_N$ and $f^{-1}A\in\calF(M)$ for all $A\in\calF(N)$. By \cite[Lemma 2.12]{ProtoExactMatroids}, pointed matroids and strong maps form a category $\mathrm{Mat}_\bullet$. 

Next, denote $\widetilde{E}:=E\setminus \{\bullet_M\}$. Given any $S\subseteq \widetilde{E}$, denote $M|S$ to be the {\em restriction of $M$ to $S$} and $M/S$ to be the {\em contraction of $M$ to $S$} (for details, see e.g. \cite{OxleyMatroids} or \cite[\S 2]{ProtoExactMatroids}.)
%	\begin{itemize}
	%	\item[] $M|S$ to be the {\em restriction of $M$ to $S$}, with groundset $S$ and flats
	%	$$\mathcal{F}(M|S):=\{\left((A\cap S, \bullet_M \right)\mid A\in \calF(M)\};$$
%		\item[] $M/S$ to be the {\em contraction of $M$ to $S$} with groundset $E\setminus S$ and flats 
	%	$$\calF(M/S):=\{\left(A\setminus S, \bullet_M\right) \mid S\subseteq A\in\calF(M)\}.$$
%	\end{itemize}  
We upgrade $\mathrm{Mat}_\bullet$ to a CGW category $\mathrm{Mat}_\bullet=(\M,\E)$ by setting
	$$\M=\{\text{strong maps that can be factored $N\xrightarrow{\sim} M|S\hookrightarrow M$, for some $S\subseteq \widetilde{E}_M$}\}$$
		$$\E=\{\text{strong maps that can be factored $M\twoheadrightarrow M/S\xrightarrow{\sim} N$, for some $S\subseteq \widetilde{E}_M$}\}^{\opp}.$$
Notice $\E$ is the {\em opposite} category of contractions, analogous to Example~\ref{ex:Exact}. In fact, one can apply \cite[Lemma 5.3]{ProtoExactMatroids} to check that $\M$ and $\E$ are closed under isomorphisms and composition, satisfying Axiom (I). Finally, let the distinguished squares be the biCartesian squares in $\mathrm{Mat}_\bullet$. %; as before, notice these are closed under isomorphisms. %%% For reference: see Proposition 4.8 of the original draft of the paper.

Axioms (M) and (Z) follow from \cite[Lemma 5.4]{ProtoExactMatroids}: a strong map $f$ is monic in $\mathrm{Mat}_\bullet$ iff $f$ is injective on the underlying set, and $f$ is epi iff $f$ is surjective. In particular, this implies all morphisms in $\M$ and $\E$ are monic, and so the matroid $O:=(\{\ast\},\ast)$ is initial in both. In addition, translating \cite[Props. 5.7 and 5.8]{ProtoExactMatroids} to our setting: any $P\xrtail{i'} Q\xotail{j'} N$ or 
$P\xotail{j} M\xrtail{i} N$ can be completed into a distinguished square
\[\dsquaref{P}{Q}{M}{N}{i'}{j}{i}{j'}\,.\]
Setting $P=O$, this gives the formal kernels and cokernels of Axiom (K), unique up to isomorphism by the biCartesian property. Finally, since biCartesian squares compose, this assignment extends to define the required functors $c$ and $k$.
%As before, this assignment extends to the required functors $c$ and $k$ since biCartesian squares compose.
\begin{comment}
As for why this defines the functors $c$ and $k$ as stated in our definition, it is worth revisiting the proof for exact categories; everything essentially follows from the universal property of biCartesian squares, including the fact that biCartesian squares compose.

For $k$, one can review Buhler's proof in \cite[Prop 2.12]{Buhler} -- notice this makes use of the fact that we have zero objects, and that kernels/cokernels are defined via these bCartesian squares featuring the 0 object (Property 4 & 5 of proto-exact categories). 

Alternatively, for a slicer argument: given
$$\dsquare{A}{B}{A'}{B'},$$
we can always compose vertically to get
$$\begin{tikzcd}
A \ar[r,>->] \ar[d,{Circle[open]->}] \ar[dr,phantom,"\square"]& B \ar[d,{Circle[open]->}]\\
A' \ar[r,>->]\ar[d,{Circle[open]->}] \ar[dr,phantom,"\square"]& B' \ar[d,{Circle[open]->}]\\
O \ar[r,>->] & \frac{B'}{A'}
\end{tikzcd}.$$
\end{comment}

\end{example}



\subsection{The $K$-Theory of pCGW categories}\label{sec:pCGW} The main result of \cite[\S 4]{CGW} is that Quillen's $Q$-Construction \cite{QuilenExact} can be applied to any CGW category to define a $K$-theory spectrum. However, this paper will focus on a particularly well-behaved class of CGW categories, which we call {\em pCGW categories.} 

Informally, pCGW categories are CGW categories $\calC=(\M,\E)$ where $\M$ is closed under a formal kind of pushout. This, of course, generalises the familiar fact that admissible monos are  closed under pushouts in exact categories but it is instructive to understand why a generalisation is needed. Consider Example~\ref{ex:FinSet} where $\M$ is the category of finite sets and injections. In which case,
$$A\leftarrow \emptyset \to A \qquad \text{where $A\neq\emptyset$}$$
does not have a pushout in $\M$ since the map $A\coprod A\to A$ is not monic. Nonetheless, this issue can be circumvented by weakening the universal pushout property, as follows. % The following key definition makes this precise.
%% Reference for admissible monos being closed under pushouts. \cite[Exercise II.7.8]{WeibelKBook},


\begin{definition}[Restricted Pushout]\label{def:restPO} Let $\M$ be a category whose morphsims are all monic. Suppose $D$ is a span
	$$C\leftarrow A \rightarrow B.$$
Define $\M_D$ to be the category whose objects are pullback squares of the form
	\[\squaref{A}{B}{C}{X}{}{}{}{} \qquad, \]
and whose morphisms are natural transformations where all components are the identity {\em except} at $X$. A {\em class of optimal squares} for $D$ is a subcategory $\widetilde{\M}_D\subseteq \M_D$ satisfying:
\begin{enumerate}[label=(\roman*)]
	\item $\widetilde{\M}_D$ has all squares where $A\xrightarrow{\cong} C$ above is an isomorphism.
	\item $\widetilde{\M}_D$ has an initial object. We denote this $B\star_A C$, and call it the {\em restricted pushout} of $D$. 
\end{enumerate}
%We say that $\M$ {\em contains all restricted pushouts} if, given any span $D$, there exists a canonical choice of optimal squares for $D$.
We say $\M$ \emph{contains all restricted pushouts} if for every span $D$ there exists a chosen optimal class $\widetilde{\M}_D$ with initial object $B\star_A C$.

\begin{comment} 

1) It seems that the formal contingency for this technology to work is some kind of initiality, which allows us these restricted pushouts to behave in a functorial way. Notice then that initiality in this span category (if it exists) guarantees that this restricted pushout is unique up to isomorphism.

A thought: what if we have weakly initial objects instead? So we have a moprhism from the restricted pushout, but not necessarily unique. This would be useful for incorporating the formalism of independent squares, which may be the right way to model matroids. 

2) What exactly do we mean by natural transformations where all components are identity except at $X$? It means to map one diagram to another, the morphisms have to assemble into squares of the form, e.g. 
$$\begin{tikzcd}
C \ar[r,>->] \ar[d,"1"]& X_0 \ar[d,>->,"1"]\\
C \ar[r,>->]& X_1
\end{tikzcd}$$
\end{comment}
\end{definition}

We now introduce the key definition of a pCGW category, before turning to examples. 

\begin{definition}[pCGW Category]\label{def:pCGW} Let $\calC=(\M,\E)$ be a CGW category. We call $\calC$ a \emph{pCGW category} if $\M$ contains all restricted pushouts, subject to the following axioms:
	\begin{itemize}
		\item[(A)] \emph{Formal Direct Sums.} Denote the restricted pushout of $C\leftarrowtail O\rtail B$ as $B\oplus C:=B\star_O C$, which we also call {\em formal direct sums}. Then, there exists a canonical pair of distinguished squares  %%% This formalises the statement: "a choice of direct sums"
		\[\dsquaref{O}{B}{C}{B\oplus C}{}{}{p_C}{q_B}\quad \text{and}\quad \dsquaref{O}{C}{B}{B \oplus C}{}{}{p_B}{q_C},\]
		which we call \emph{direct sum squares}. 
		\begin{comment} 
		REMARK: Below is wrong, there are non-split exact sequences of the form 0->A->A+B->B->0.
		
		Formal direct sums are \emph{universally disjoint} in the following sense: given any pair of distinguished squares
		\[\dsquaref{O}{B}{C}{B\oplus C}{}{}{v}{v'} \qquad \dsquaref{O}{B}{C}{B\oplus C}{}{}{w}{w'}\]
		isomorphic to a direct sum square, there exists an isomorphism $\Psi\colon B\oplus C\rtail B\oplus C$ in $\mathrm{iso}\M$ such that 
		$$\Psi\circ v=w \,\,\text{in}\,\,\M \quad \text{and} \quad \varphi(\Psi)\circ v'=w' \,\,\text{in}\,\,\E.$$
		\end{comment}
		\item[(PQ)]\emph{Preserves Quotients.}  Given any $B\star_A C$, there exists a canonical isomorphism $\frac{B\star_A C}{C}\cong \frac{B}{A}$. This assembles into a commutative diagram in the ambient category $\calC$
		\[\begin{tikzcd}
		A \ar[r,>->,"f"] \ar[d,>->,"g"]& B \ar[d,>->,"g'"] \ar[dr,phantom,"\circlearrowleft"] & \frac{B}{A}\ar[l,swap,"c(f)",{Circle[open]}->] \ar[d,>->,"\cong"]\ \\
		C \ar[r,>->,"f'"]& B\star_AC & \frac{B\star_A C}{C} \ar[l,swap,"c(f')",{Circle[open]}->]
		\end{tikzcd}\]
Note the RHS square need not be distinguished; it is only required to commute in $\calC$.

		%\footnote{One advantage of working in a {\em good} CGW category (Definition~\ref{def:goodDC}) is that we avoid the machinery of {\em pseudo-commutative squares} used in prior work (\cite[\S 5]{CGW} and \cite[Definition 3.1]{SarazolaShapiro}).}
		\item[(DS)] \emph{Compatibility with Distinguished Squares.} Given a diagram of distinguished squares \[\begin{tikzcd}
		C \ar[d, {Circle[open]}->] \ar[dr, phantom, "\square"]& A \ar[dr, phantom, "\square"]\ar[d, {Circle[open]}->]  \ar[l, >->] \ar[r, >->] & B \ar[d, {Circle[open]}->] \\
		C' & A' \ar[l, >->] \ar[r, >->] & B' 
		\end{tikzcd}\]
		there is an induced map $B\star_A C\otail B'\star_{A'}C'$ such that the two induced squares 
		\[\dsquare{B}{B\star_AC}{B'}{B'\star_{A'}C'}\qquad \dsquare{C}{B\star_AC}{C'}{B'\star_{A'}C'}\]
		are distinguished.
	\end{itemize}
\end{definition}

\begin{remark} For transparency, Definition~\ref{def:pCGW} requires only that $\M$ contains all restricted pushouts; no such requirement is placed on $\E$. 
\end{remark}

\begin{convention}\label{conv:restrPO-quotient} We adopt the following conventions for quotients and formal direct sums.
\begin{itemize}
	\item Since we generally work only up to isomorphism (cf. Remark~\ref{rem:INV}), we typically fix a representative of $\frac{B\star_A C}{C}$ and tacitly identify it with some choice of quotient of $f$ via the canonical isomorphism of Axiom~(PQ).
	
	%		Initially, we wanted to identify a canonical choice of quotient $\frac{B\star_A C}{C}=\frac{B}{A}$, but this may not be well-defined, for reasons similar to those mentioned in Remark~\ref{rem:INV}.
	\item Axiom~(A) specifies canonical direct sum squares. In light of Axiom~(PQ), we take
	$$
	\begin{tikzcd}
	O \ar[r,>->] \ar[d,>->] & B \ar[d,>->,"p_B"] \ar[dr,phantom,"\circlearrowleft"] 
	& \ar[l,{Circle[open]}->,swap,"="] B \ar[d,>->,"="] \\
	C \ar[r,>->,"p_C"] & B\oplus C & \ar[l,{Circle[open]}->,swap,"q_B"] B
	\end{tikzcd}\qquad.
	$$
	In particular, this implies $p_B=q_B$ if $\E\subseteq\calC$, and $q_B\circ p_B=1_B$ if $\E^\opp\subseteq\calC$. Hence, for suggestiveness, we sometimes write, e.g. $p_B = 1\oplus C$.
	
\end{itemize}	
\end{convention}

\begin{remark} Definition~\ref{def:restPO} of restricted pushouts is a synthesis of \cite[Def. 5.3]{CGW} and \cite[Def. 3.2]{SarazolaShapiro}. The reason behind requiring Condition (i) for optimality is to ensure that restricted pushouts behave functorially in the following sense: 
\end{remark}

\begin{fact}\label{facts:restrictedpushouts} Given a pCGW category, consider the diagram 
	$$C\leftarrowtail A\rtail B \rtail B'$$
	Then $B'\star_B \left(B\star_A C\right)\cong B'\star_A C$. More explicitly, the composite of restricted pushouts in Diagram~\eqref{eq:rpusCOMPOSE} is the restricted pushout of the outer span.		
	\begin{equation}\label{eq:rpusCOMPOSE}
	\begin{tikzcd}
	A \ar[r,>->] \ar[d,>->] & B \ar[d,>->] \ar[r,>->] & B' \ar[d,>->] \\
	C \ar[r,>->] & B\star_A C \ar[r,>->]& B'\star_B \left(B\star_A C\right)
	\end{tikzcd} 
	\end{equation}	
\end{fact}
\begin{proof} Translate the proof of \cite[Corollary A.2]{SarazolaShapiro}.
\end{proof}

\begin{comment} Let's examine the proof of Lemma A.1 in \cite{SarazolaShapiro}, Corollary A.2 follows immediately.

Everything's the same except we don't use the notion of pseudo-commutative squares, but their argument easily adapts to our case still. Let us follow our convention of identifying $B\star_A C/C=B/A$. Let us rewrite their diagram:
\[\begin{tikzcd}
A\ar[r,>->] \ar[d,>->]&B \ar[d,>->,blue] \ar[dr,phantom,"\circlearrowleft"] & B/A \ar[dd,bend left, xshift=1.5em,"h_1",blue] \ar[d,>->,"="] \ar[l,{Circle[open]->},blue]\\ C \ar[dr,phantom,"g"] \ar[d,>->,"="] \ar[r,>->] & B\star_A C \ar[dr,phantom,"\square"]\ar[d,>->,blue] &B/A \ar[l,{Circle[open]->}] \ar[d,>->,"h_0"]\\
C \ar[r,>->] & D & D/C\ar[l,{Circle[open]->},blue,"j"]
\end{tikzcd}\]

$h_1$ is given as an isomorphism, and the blue arrows commute. The LHS column is obtained by taking restricted pushout, with the bottom square being ``good'' or ``optimal'' by hypothesis. For the RHS column: the top square is by Axiom (PQ), the bottom squre is by applying functor $c$ to the $\M$-square in the bottom LHS corner. Since distinguished squares are commutative, the RHS column also commutes. 

We want to show that $h_0$ induced by the functor $c$ is indeed  an isomorphism as well, in fact equal to $h_1$. This crucially uses the fact that RHS column commutes regardless of whether we use $h_0$ or $h_1$. If $\E\subseteq \calC$, then we get $j\circ h_0 = j\circ h_1$ implies $h_0=h_1$ since $j$ is monic in $\calC$. Alternatively, if $\E^\opp\subseteq \calC$, then this gives $h_0\circ j= h_1\circ j$ implies $h_0=h_1$ since $j$ is now epi. And we are done. 
\end{comment}


We now revisit the examples from before.\footnote{To avoid later confusion: when we specify the class of optimal squares to be ``{\em all pullback squares} such that \dots ''  we are of course restricting to $\M$ (and not all pullback squares in the ambient CGW category).}

\begin{example}[Exact Categories] Let $\star$ be the usual pushout, and let the optimal squares be the class of pullback squares such that the induced map $B\cup_A C\to X$ is an admissible mono. Then $\star$ is well-defined since the pushout of any span of admissible monos is also a pullback. Axiom (PQ) follows from such pushouts preserving cokernels, Axiom (DS) from the pasting law for pushouts. \begin{comment} Remark: Why pushout of admissible monos a pullback? Why preserve cokernels? See Prop. 2.12 of Buhler's Exact Categories. Also, why is the induced morphism given by the pushout property an admissible mono? This is required by fiat. Sarazola-Shapiro, Example 3.1 has more details on why this results in good behaviour, and why pullbacks play a role. 
	
Now why Axiom (DS)?  We want to show the following squares are biCartesian:

\[\begin{tikzcd} B' \ar[r,>->] \ar[d,->>] & B' \star_{A'} C' \ar[d,->>] \\ B \ar[r,>->] & B\star_AC  \end{tikzcd} \quad\text{and}\quad \begin{tikzcd} C' \ar[r,>->] \ar[d,->>] & B' \star_{A'} C' \ar[d,->>] \\ C \ar[r,>->] & B\star_AC  \end{tikzcd} \]

We know we have biCartesian squares
\[\begin{tikzcd} A \ar[r, tail] \ar[d, tail] & B \ar[d, tail] \\ C \ar[r, tail] & B \star_A C \end{tikzcd}\begin{tikzcd} A' \ar[r, tail] \ar[d, tail] & B' \ar[d, tail] \\ C' \ar[r, tail] & B' \star_{A'} C' \end{tikzcd}\]


We have the following commutative diagram
\[\begin{tikzcd}
& C' \ar[dr,->>]\\
A' \ar[ur,>->]\ar[r,->>] \ar[d,>->]& A \ar[d,>->] \ar[r,>->]& C\ar[d,>->] \\
B' \ar[r,->>] & B \ar[r,>->] & B\star_AC\\
\end{tikzcd}\]
And so by the pushout property, we get an induced morphism
\[\begin{tikzcd} 
A' \ar[r, tail] \ar[d, tail] & B' \ar[d, tail] \ar[ddr,bend left]\\ C' \ar[r, tail] \ar[drr,bend right]& B' \star_{A'} C' \ar[dr,dashed] \\
&& B\star_A C \end{tikzcd}\]
It remains to show that the induced squares are biCartesian. 
By functoriality of pushouts we have that the outer rectangle is a pushout.
\[\begin{tikzcd}
A' \ar[r,->>] \ar[d,>->]& A \ar[d,>->] \ar[r,>->]& B \ar[d,>->] \\
C' \ar[r,->>] & C \ar[r,>->]& B\star_A C
\end{tikzcd} \]
Now consider the diagram 
\[\begin{tikzcd}
A' \ar[r,->] \ar[d,>->]& B' \ar[d,>->] \ar[r,->>]& B \ar[d,>->] \\
C' \ar[r,->] & B'\star_{A' }C' \ar[r,->,dashed]& B\star_A C
\end{tikzcd} \]
By the pushout property [inducing the dashed arrow], we know that the outer rectangle of this diagram is the same as the one above. That is, it is a pushout square. Since the LHS is also a pushout square, so is the RHS square. And by Buhler Prop 2.12, this gives a biCartesian square. 

It remains to show that the induced dashed map is an admissible epi. Consider the kernel in the short exact sequence $X\xrtail{f} B' \twoheadrightarrow B$. This assembles into the pushout diagram
\[\begin{tikzcd}
X \ar[d,>->"f"] \ar[r,->>]& O \ar[d,>->]\\
B' \ar[r,->>] \ar[d,>->]& B \ar[d,>->]\\
B'\star_{A'} C' \ar[r]& B\star_{A}C
\end{tikzcd}\]
Since pushouts compose, the outer rectangle is a cofibration square, and so $B'\star_{A'}C'\to B\star_A C$ is an admissible epi with kernel $X\rtail B'\rtail B'\star_{A'}C'.$
	\end{comment} 
	For Axiom (A), pick the obvious biCartesian squares in the exact category:
	\[\begin{tikzcd}
	B \ar[r,>->,"p_B"] \ar[d] & B\oplus C \ar[d,->>,"q_C"] \\
	O \ar[r]&  C
	\end{tikzcd}\qquad\begin{tikzcd}
	C \ar[r,>->,"p_C"] \ar[d] & B\oplus C \ar[d,->>,"q_B"] \\
	O \ar[r]&  B
	\end{tikzcd}.\]
	\begin{comment} This follows from Lemma 2.7, in Buehler \cite{Buhler}.
	\end{comment} 
\end{example}

\begin{example}[Varieties]\label{ex:pCGWVar} Let $\star$ denote the pushout in the category of schemes (not just varieties), and let the optimal squares be all pullback squares. Leveraging the results in \cite{Schwede} and the Stacks Project \cite[Tag 0ECH]{stacks-project}, one can verify:
\begin{itemize}
	\item[$\diamond$] Pushouts along closed immersions exist in the category of schemes, and yield squares of closed immersions \cite[Corollary 3.9]{Schwede}. 
	\item[$\diamond$] Such pushouts are also pullbacks.
\begin{comment} See https://math.stackexchange.com/questions/484314/is-a-pushout-w-of-schemes-along-a-closed-subscheme-also-a-pullback

Sketch of argument. The affine case: 

\emph{Affine case:} Given a cospan of commutative rings \( A \twoheadrightarrow B \leftarrow C \) with \( A \twoheadrightarrow B \) surjective, let \( P := A \times_B C \). Then \( \Spec(P) \cong \Spec(A) \cup_{\Spec(B)} \Spec(C) \) \cite[Thm.~3.4 and 3.5]{Schwede} [notice that $C$ in Schwede's proof of Thereom  3.4 is precisely $A\star_B C.$]

Write \( I := \ker(A \twoheadrightarrow B) \). Then \( P \to C \) is surjective with kernel \( \{(a,0) \mid a \in I\} \), which projects to \( I \subseteq A \). [This uses the fact that the category of commutative rings is a regular category. In plainer terms: since $A\twoheadrightarrow B$ is surjective, then given any $c\in C$, there exists $a\in A$ such that $f(a)=g(c)$ in $B$. And so $(a,c)\in P$, and thus the projection map $P\to C$ is surjective.]

Deduce $$A \otimes_P C \cong A/I = B $$ and so \( \Spec(B) \cong \Spec(A) \times_{\Spec(P)} \Spec(C) \). \emph{General case:} Given a span of closed immersions \( X \ltail Y \rtail Z \), the pushout exists and is locally affine. Apply above argument.
\end{comment}	

\begin{comment} Detail 1: Why $A \otimes_P C \cong A/I$? 

We construct explicit maps between the two rings and show that they are inverse. 

For $$\Phi\colon A/I\to A\otimes_P C,$$ we map  $a+I\mapsto a\otimes 1$. To show that this is well-defined, we show that multiplication by any $a\in I$ trivialises the tensor.

Recall that the maps from $P$ are projections, $\pi_A\colon P\to A, \pi_C\colon P\to C$. An element in $A\otimes_P C$ is finite sum of $a\otimes c$ where $(a,c)\in A\times C$ subject to relations $$\pi_A(p)\cdot a\otimes c = a\otimes \pi_C(p)\cdot c$ for any $p\in P.$ This is the $P$-balanced relation. 

What if $a\in I$? In which case, by above $(a,0)\in P$. Hence, $(a\cdot 1)\otimes 1 = 1\otimes (1\cdot 0)= 0$. 
Now we can show well-definedness. For $a+I=a'+I$ iff $a-a'\in I$. But then 
$$(a-a')\otimes 1=0$$ which by bilinearity yields 
$$a\otimes 1= a'\otimes 1.$$

So the assignment is well-defined. The fact that this is a homomorphism follows from bilinearity and multipication of tensors. 


As for the converse direction $$\Psi\colon A\otimes_P\ C\to A/I.$$
We know there exists such a ring homomorphism by the pushout property, but let's define it explicitly. Given $f\colon A\to B$ and $g\colon C\to B$, we can define $\theta\colon A\times C\to B$ as $f(a)g(c)$. This is bilinear and $P$-balanced (in the sense that it respects the tensor relations), and so this induces a map $\Psi\colon A\otimes_P C \to B$ which sends $(a\otimes c)\mapsto f(a)g(c).$ Since $B=A/I$, we can take $f\colon A\to A/I$ be the usual quotient map sending $a\mapsto a+I$, and $g\colon C\to A/I$ sending $c\mapsto a_c$ where $(a_c,c)\in P$. This is well-defined since any $(a_c,c),(a'_c,c)\in P$ must all agree in $B$. 

It remains to check inverses:
\begin{itemize}
\item why is $\Phi$ and $\Psi$ inverse? $\Psi\circ \Phi(a+I)=\Psi(a\otimes 1)=a+I$. 
\item Now  suppose $\Phi\circ \Psi(a\otimes c)= \Phi(a\cdot a_c + I) = (a\cdot a_c)\otimes 1 = a\otimes c$, where the last equality is by the fact that $\pi_C(a_c,c)=c$.
\end{itemize}

This shows the isomorphism.
\end{comment}

\begin{comment} Detail 2: How to extend from the affine case to the general? 

Let \(f\!:X\rtail Y\) and \(g\!:X\rtail Z\) be closed immersions and let \(P:=Y\cup_X Z\) denote the pushout scheme (constructed by gluing affine pushouts; see Schwede). We show the canonical map
\[
\iota\colon X \longrightarrow Y\times_{P} Z
\]
is an isomorphism.


By construction of $P$, Schwede \cite[Corollary 3.9]{Schwede}, there exists an affine open cover \(\{U_\alpha\}_{\alpha}\) of \(P\) such that for each \(\alpha\) the restriction of the diagram to \(U_\alpha\) arises from an affine pushout. Concretely, for every \(\alpha\) there are rings \(A_\alpha,C_\alpha,B_\alpha\) with
\[
U_\alpha\cap Y \cong \Spec A_\alpha,\quad U_\alpha\cap Z \cong \Spec C_\alpha,\quad U_\alpha\cap X \cong \Spec B_\alpha,
\]
and \(U_\alpha\cong \Spec(P_\alpha)\) with \(P_\alpha \cong A_\alpha\times_{B_\alpha} C_\alpha\). (These identifications are part of the local description of the glued pushout.)

By the affine computation (see above), for each \(\alpha\) the canonical map
\[
\iota|_{U_\alpha}\colon X\times_P U_\alpha \longrightarrow (Y\times_P Z)\times_P U_\alpha
\]
identifies \(\Spec B_\alpha\) with \(\Spec A_\alpha\times_{\Spec P_\alpha}\Spec C_\alpha\); hence \(\iota|_{U_\alpha}\) is an isomorphism. The \(U_\alpha\) cover \(P\), and thus cover the pullback $Y\times_P Z$. (For details, see stacks Tag 01JO -- notice we don't have refine the pre-images via an affine open cover because our maps are closed immersions, and thus affine). Hence, \(\iota\) is an isomorphism on an open cover of the \emph{target} \(Y\times_P Z\). 

Scheme morphisms are morphisms of locally ringed spaces, but they reduce to checking on an affine open cover, see Ravi Vakil, Exercise 6.3.C. https://math.stanford.edu/~vakil/216blog/FOAGdec3014public.pdf 

And so an isomorphism on the affine open cover for both schemes implies \(\iota\) is an isomorphism and thus the pushout square is cartesian.
\end{comment}
	
	\item[$\diamond$] If \( A \rtail B \), \( A \rtail C \) are closed immersions of varieties, the pushout scheme \( B \cup_A C \) is also a variety.
\begin{comment} We need to check the pushout scheme is reduced, separated, quasi-compact and of locally finite type.

For separatedness, see Tag 0ECH, Lemma 37.67.4. For locally finite type, see Lemma 37.67.5 -- in particular, recall that $\Spec(k)$ is locally noetherian since one affine is itself, and any field $k$ is Noetherian (= satisfies ascending chain condition on ideals) since any field only has two ideals.

What about being reduced? Recall: a scheme is a locally ringed space, where each stalk $\mathcal{O}_{X,x}$ is a local ring (has unique maximal ideal). A reduced ring is a ring with no non-zero nilpotent elements. This criterion is obviously local. Some helpful lemmas: 
\begin{itemize}
\item Lemma 28.3.2: A scheme is reduced iff for every affine open $U\subseteq X$, the ring $\mathcal{O}_X(U)$ is reduced.

https://stacks.math.columbia.edu/tag/01IZ
\item An affine scheme is $\Spec(R)$ is reduced iff $R$ is a reduced ring 

https://stacks.math.columbia.edu/tag/01OJ
\item Say a property is affine-local if we can check it on any affine cover. Reducedness is affine local, see comments after Lemma 5.3.2 here http://math.stanford.edu/~vakil/216blog/FOAGnov1817public.pdf

Essentially, this is because reducedness is a stalkwise-local property. A scheme $X$ is reduced iff all its stalks are reduced. If the scheme admits an affine cover such that all of these guys are reduced, this means all the stalks of the scheme are reduced. And any other affine cover will just be a repartitioning of the existing reduced stalks.
\end{itemize}


A scheme $X$ is reduced iff there exists some affine cover $\{U_i\}$ of $X$ such that $\mathcal{O}_X(U_i)$ is reduced for all $i$. By Schwede's construction, we can take the affines to be of the form $V_i\cup_{Z_i}U_i$, for affines $U_i\subseteq B,W_{ij}\subseteq
A,V_i\subseteq C$ -- which are all reduced by hypothesis. So
$V_i\cup_{Z_i}U_i$ corresponds to a pullback of reduced rings, which makes it reduced.




Finally, let's check that the pushout is quasi-compact. Every variety are of finite type over $\Spec(k)$, the schemes $A,B,C$ are quasi-compact. [Recall: a scheme is quasi-compact if its underlying topological space is quasi-compact; there is a relative version wrt morphisms that is also helpful  https://stacks.math.columbia.edu/tag/01K2]

Let $\{F_i\}_{i\in I}$ be a cover of $B\star_A C$. Consider the canonical maps $B\rightarrow B\star_A C\leftarrow C$, and pull back each $F_i$ along these maps to obtain open covers $\{F^B_{i}\}$ of $B$ and $\{F^C_i\}$ of $C$. Since $B$ and $C$ are quasi-compact, there exists finite subcollections $\{F_{i_1}^B,\dots, F^B_{i_m}\}$ and $\{F_{j_1}^C,\dots, F^C_{j_n}\}$ that cover $B$ and $C$ respecitvely.

Now consider finite set of index pairs $(i_k,j_l)$ for $1\leq k \leq m, 1\leq l\leq n$. For each such pair, define $W_{kl}:=F^B_{i_k}\cap F^C_{jl}\cap A$, viewed as an open subset of $A$ via the pullbacks along closed immersion. These $W_{kl}$ form a finite open cover of $A$. Each pushout $F^B_{i_k}\cup_{W_{kl}}F^C_{j_l}$ defines an open subset of $B\star_A C$, and each such open is some $F_i$ from the original cover. The total collection of these pushouts is finite since we only have finitely many such indices; it is also clear that it is a cover by examining the underlying topological space. Hence, the original open cover $\{F_i\}$ admits a finite subcover, showing quasi-compactness. 
\end{comment}	
\end{itemize}
A key observation: many properties of the pushout — reducedness, separatedness, locally of finite type, and the pullback property — can be checked affine-locally. To show the pushout is of {\em finite} type (not just {\em locally finite} type), it suffices to check it is also quasi-compact -- this follows from gluing two quasi-compact schemes along a quasi-compact subscheme. Finally, since optimal squares are pullbacks (by hypothesis), this ensures the induced pushout morphism $B\cup_A C\to X$ is a closed immersion.  In sum: $\star$ is well-defined.


\begin{comment} Previous footnote.

{\em Proof Sketch.} Straightforward, but we elaborate. To show $B\cup_A C\to X$ is a closed immersion, we work affine-locally on $X$. Take $U=\Spec (N) \subseteq X$, and write the pre-images 
$$B\cap U=\Spec (S),\qquad C\cap U=\Spec (T),\quad \text{and}\quad A\cap U=\Spec (R).$$
Since $B\rtail X, C\rtail X$ are closed immersions, there exists ideals $I,J\subseteq N$ such that $S\cong N/I$ and $T\cong N/J$. Since $A=B\times_X C$ by optimality, we have $R\cong (N/I)\otimes_N (N/J) \cong N/(I+J)$, and so $S\times _R T \cong (N/I)\times_{N/(I+J)} (N/J).$ However, we know that $ \Spec(S\times_R T)=\Spec(S)\cup_{\Spec (R)} \Spec(T) \subseteq B\cup_A C$ by \cite[Theorem 3.4]{Schwede}. The claim then follows from verifying that the canonical ring map $\varphi\colon N\to (N/I)\times_{N/(I+J)} (N/J)$ is surjective with $\ker(\varphi)=I\cap J$.

\end{comment}

\begin{comment} More details.
 Why is the induced pushout morphism a closed immersion? 
Consider 
\[\begin{tikzcd}
A \ar[r,>->] \ar[d,>->]& B \ar[d,>->] \ar[ddr,>->, bend left]\\C \ar[r,>->] \ar[drr,bend right,>-> ] & P \ar[dr,dashed,"\phi"] \\
&& X
\end{tikzcd}\]

To check $P\to X$ is a closed immersion, it suffices to check on an affine open cover. [See Stacks 01QN, Lemma 29.2.1 (2)]
So take affine open $$U=\Spec N\subseteq X,$$ with pre-images
$$B\cap U=\Spec S, \qquad C\cap U =\Spec T \qquad A\cap U =\Spec R.$$
Since $B\rtail X, C\rtail X$ are closed immersions, there exist ideals $I,J\subseteq N$ with 
$$S\cong N/ I \qquad T\cong N/J.$$
See Stacks Tag 01QN. Since $A=B\cap C$, this means 
$$R\cong S\otimes_N T\cong N/(I+J)$$
where $S\otimes_N T$ corresponds to the pushout in the category of commutative rings. The second isomorphism, that is
$$(N/I)\otimes_N (N/J)\cong N/(I+J),$$
essentially follows from Yoneda's Lemma and universal properties of the tensor product. For details, see: https://math.stackexchange.com/questions/720507/tensor-product-of-quotient-rings-a-proof-using-yoneda-lemma

(The above link proves this in a more general setting; in our case, recall that $N\otimes_N N.$)


(Additional note:) Spectrum turns tensor product of rings into fiber product is because it is a right adjoint to the global sections functor, and thus preserves limits. But we have to adjust for the fact that $\Spec$ is contravariant, and so turns colimits into limits.

https://math.stackexchange.com/questions/3302849/why-does-the-tensor-product-of-rings-correspond-to-the-product-of-their-spectra       )

Now by work of Schwede, Theorem 3.4 - 3.5, we know that the affine pushout on this affine patch corresponds to the pullback of rings
$$S\times_R T = (N/I)\times_{N/(I+J)} (N/J).$$

To show that there exists a surjection $\phi$
\[\begin{tikzcd}
N \ar[ddr,bend right,->>]  \ar[dr,dashed,"\psi"]\ar[drr,bend left,->>]\\
& S\times_R T \ar[r,->>,"h"] \ar[d,->>,swap,"j"]& S \ar[d,->>,"k"] \\ & T \ar[r,->>,"l"] & R 
\end{tikzcd},\]
we prove there exists an isomorphism
$$(N/I)\times_{N/(I+J)}(N/J)\cong N/(I\cap J).$$

To see why, define
$$\phi\colon N\to (N/I)\times_{N/(I+J)} (N/J)$$
as sending $n$ to its images in the two factors. Notice its kernel is precisely those $n$ which maps to 0 in both $N/I$ and $N/J$, and so $\ker\phi=I\cap J.$ To see $\phi$ is surjective, consider $(\overline{n}_1,\overline{n}_2)$ of the pullback. By definition of the pullback, $\overline{n}_1$ and $\overline{n}_2$ have the same image in $N/(I+J)$ and so choosing representatives $n_1,n_2\in N$,  we have 
$$n_1-n_2=i+j, \qquad \text{for some}\,\, i\in I, j\in J.$$
Then define
$$n:=n_1-i=n_2+j$$
is an element of $N$ whose image in $N/I$ is $\overline{n}_1$ and whose image in $N/J$ is $\overline{n}_2$. And so, $\phi(n)=(\overline{n}_1,\overline{n}_2)$, showing surjectivity. In other words, on an affine patch $U=\Spec(N)$, the pushout $P$ restricts to 
$$P\cap U \cong \Spec(N/(I\cap J)),$$
and the canonical map $\Spec(N/(I\cap J))\to \Spec(N)$ is the closed immersion cut out by the ideal $I\cap J$. Since $X$ is covered by such affines, conclude $P\to X$ is a closed immersion.
\end{comment}

\begin{comment} Separatedness fails taking pushouts of open immersions. J. Pajwani: consider $\mathbb{A}^1\ltail \mathbb{A}^1\setminus \{0\}\rtail \mathbb{A}^1$, the pushout will have two origin points that cannot be separated from one another.
\end{comment}

Axioms (PQ) and (DS) follow from the pushout property and examining the pushout construction in \cite[\S 2-3]{Schwede}.\footnote{It is easy to see the pushout property yields a map $v\colon B\star_A C\to B'\star_{A'} C'$; verifying that $v$ also defines an open immersion, however, is somewhat involved. For a more formal perspective on why Axiom (DS) holds, see \cite[Proposition A.4]{SarazolaShapiro}.} For Axiom (A), let the direct sum squares be the coproduct squares with standard coproduct injection maps.

%\footnote{It is easy to see the pushout property yields a map $u\colon B\star_A C\to B'\star_{A'} C'$. Since $B'\star_{A'}C'$ is quasi-compact, we can show $u$ is an open immersion by checking affine-locally. Axiom (DS) then essentially follows from examining how localisation of commutative rings interact with certain pullbacks. For a different perspective on why Axiom (DS) holds, see \cite[Proposition A.4]{SarazolaShapiro}.} 

\begin{comment}
Axiom (PQ): By construction, $B\star_A C=C\cup (B\setminus A)$ on the underlying topological space. Construct the obvious distinguished (pullback) square. (Notice we don't need to appeal to uniqueness of reduced scheme structure here since the complement is an open immersion).

Axiom (DS): One way to see this is to adapt Sarazola-Shapiro's argument in Proposition A.4 -- 
notice our notion of optimal squares matches their notion of good squares. One will then need to check the additional axioms required by the ECGW formalism. A small note: there appears to be a gap in their proof of Proposition A.4, however. It is not clear to me why their square
$$\begin{tikzcd}
A \ar[r,>->] \ar[d,>->]& B \ar[d,>->] \\
C \ar[r,>->]& \coker f
\end{tikzcd}$$
is a good square; this is needed for them to apply their Lemma A.1 to get the final result.


Below, is another argument showing that Axiom (DS) holds; the only thing that needs checking is that we obtain an open immersion; we've left out some details because it requires some involved commutative algebra to nail down.

\begin{enumerate}
\item\textbf{Existence.} Consider the diagram
\[\begin{tikzcd}
A \ar[r,>->] \ar[d,>->]& B \ar[d,>->] \ar[r,{Circle[open]->}] & B' \ar[d,>->]\\
C \ar[r,>->] \ar[d,{Circle[open]}->]& B\star_A C  \ar[r,dashed,"u"] & B'\star_{A'} C' \\
C' \ar[urr,>->]
\end{tikzcd} \]
Why does the diagram commute? Obviously the restricted pushout square commutes.
Notice that distinguished squares are pullback squares in $\Var_k$, and so they too commute. This means, looking at the setup for Axiom (DS), we know that 
$$A\rtail C \otail C' \rtail  B'\star_{A'} C'$$
is the same as 
$$A\otail A' \rtail C' \rtail  B'\star_{A'} C'$$
which by the restricted pushout square is the same as
$$A\otail A' \rtail B' \rtail  B'\star_{A'} C'$$
and also 
$$A\rtail B \otail B' \rtail  B'\star_{A'} C'.$$
Since the restricted pushout is a pushout in the category of schemes, this gives us an induced map making the diagram commute -- we know that it is unique.
\item \textbf{Description of Image} For simplicity, denote the pushouts and induced morphism as 
\[
P:=B\star_A C,\qquad P':=B'\star_{A'} C' \qquad \text{and}\qquad u\colon P\to P'.
\]
If we look at the underlying map of topological spaces, $u$ is defined on $B$ and $C$, in fact $\frac{B}{A}$ and $C$. So to characterise $\mathrm{Im}(u)$, it suffices to characterise $u(C)\cup u(\frac{B}{A})$. Apply Axioms (PQ) and (DS) to get the diagrams
Now consider the diagrams
\[\begin{tikzcd}
&& C \ar[d]\ar[dr]\\
\frac{B}{A} \ar[d,"\cong"] \ar[r]& B \ar[r] \ar[d]& B\star_A C \ar[d,"u"] & C' \ar[dl]\\
\frac{B'}{A'} \ar[r] \ar[drr,"\cong",swap]& B'\ar[r] & B'\star_{A'} C' \\
&& B'\star_{A'}C'/C' \ar[u]
\end{tikzcd}\]
The LHS square is from the given distinguished squares in Axiom (DS) and applying functor $c$; the bottom triangle comes is Axiom (PQ).

This diagram shows: $u(\frac{B}{A})$ is disjoint from $C'$, and thus disjoint from $u(C)$ and $\frac{C'}{C}$ in the pushout $P'$. In set-theoretic terms, sin 
we have
\[
P' \;=\; \frac{P'}{C'} \;\uplus\; C' \;=\; u(\frac{B}{A})\;\uplus\; u(C)\;\uplus\;\frac{C'}{C}.
\]
based off of the isomorphisms from Axioms (PQ) & (DS). Hence $\operatorname{Im}(u)=P'\setminus\frac{C'}{C}$ and $\operatorname{Im}(u)$ is open in $P'$. Hence, the squares satisfy the union-of-images requirement for distinguished squares in $\Var_k$; notice this is a set-theoretic requirement.

\item \textbf{Open Immersion.} LATER.

\item \textbf{The squares define a pullback}. Notice we have the sequence
$$C\otail C'\rtail B'\star_{A'}C'.$$
By \cite[Lemma 2.9]{CGW}, this can always be completed into a unique distinguished square. 
$$\dsquare{C}{ X }{ C'}{B'\star_{A'}C'}.$$
This is constructed by first examining the sequence of $\M$-morphisms
$$\frac{C'}{C}\rtail C'\rtail B'\star_{A'}C',$$
and then applying the $c$ functor to obtain the unique (up to isomorphism) cokernel of this morphism. But we showed above that $B\star_A C\otail B'\star_{A'}C'$ is an $\E$-morphism, whose complement is $\frac{C'}{C}$. So we may identify $X\otail B'\star_{A'}C'$ with $B\star_A C\otail B'\star_{A'}C'.$  The other distinguished square follows by symmetry.
\end{enumerate}


\end{comment}
\end{example}




\begin{example}[Definable Sets] Let $M$ be a $\Sigma$-structure that: (i) eliminates imaginaries; and (ii) contains at least two distinct elements. Then the (ambient) category $\mathrm{Def}(M)$ admits coproducts\footnote{This result is well-known for the weak syntactic category \cite{Harnik}, but the same argument works semantically. Fix definable sets $D_0\subseteq M^k$ and $D_1\subseteq M^l$. Let $P=M^k\times M^l\times M$, and define an equivalence relation $E\subseteq P^2$ by
		$$ (\overline{x}_0,\overline{y}_0,z_0)\sim (\overline{x}_1,\overline{y}_1,z_1)\iff (z_0=z_1\land \overline{x}_0=\overline{x}_1)\vee (z_0\neq z_1\land \overline{y}_0=\overline{y}_1).$$
		By elimination of imaginaries, $Q:=P/E$ is definable. In fact, it can be definably identified with $M^k\coprod M^l$, and so there exists definable injections $M^k,M^l\rtail Q$ whose disjoint images cover $Q$. Restricting to $D_0$ and $D_1$ yields the coproduct $D_0\coprod D_1\subseteq Q$.} and quotients of equivalence relations. Given a span of definable injections $B\xltail{f} A\xrtail{g} C$, define 
	$$B\star_AC:=\bigslant{ B\coprod C}{\sim}$$
	where $\sim$ is the obvious equivalence relation generated by $f(a)\sim g(a), a\in A$. A routine check shows that $\sim$ is expressible as a {\em definable} equivalence relation, with transitivity following from injectivity of $f$ and $g$. In other words, $B\star_A C$ defines a pushout in the ambient category $\mathrm{Def}(M)$. %\footnote{For clarity: we mean a pushout in the standard category of definable sets with (not necessarily injective) definable functions.}
	\begin{comment} Details on why the obvious equivalence relation is definable. 		
	
	Suppose $x,y\in B\coprod C$. One can then define $x\sim y$
	\begin{enumerate}
	\item For $x,y\in B$, 
	$$x\sim y \leftrightarrow x=y.$$
	The same holds if $x,y\in C.$
	\item Suppose $x\in B$ and $y\in C.$ Then $x\sim y$ and $y\sim x$ just in case
	$$\exists a\in A . (x=f(a) \land y=g(a)).$$
	\end{enumerate}
	It is clear Conditions (1) and (2) correspond to definable subsets of $B\coprod C$, and so yields a definable relation $\sim$. It remains to check that this definition of $\sim$ gives an equivalence relation. Reflexivity and symmetry is clear. For transitivity, the only non-trivial case to check is $x\sim y\sim z$ where the pairs $(x,y)$ and $(y,z)$ are in different sorts (i.e. one lies in $B$, the other in $C$). Suppose for instance $x,z\in B$ and $y\in C$. Then there exists $a,a'\in A$ such that $x=f(a)$ and $y=g(a)$, while $y=g(a')$ and $z=f(a')$.  Since $g$ is injective, deduce that $a=a'$, and so $x=f(a)=f(a')=z$. The case when $x,z\in C$ and $y\in B$ is analogous.
	
	%%% Notice y cannot be in both Sorts B and C, since they are disjoint!
	
	We have defined a definable relation, and shown that it is an equivalence relation. Since $M$ eliminates imaginaries, the quotient of this definable equivalence relation exists. 
	
	It is also worth checking why this is a pushout. The argument is exactly the same as in $\Set$, the only technicality is to check that the induced pushout map is definable. But this is obvious since definable sets are closed under Boolean operations. We detail this below in the next comment section.

	\end{comment}
Now let the optimal squares be all pullback squares; another straightforward check shows $\star$ is well-defined. 
	\begin{comment} We need to show: (1) the pushout is a pullback; (2) the induced pushout morphism amongst optimal squares with the same span is injective.
	(1) follows from the definition 
\[\begin{tikzcd}
N \ar[ddr,bend right,>->,swap,"k"]  \ar[dr,dashed,"\psi"]\ar[drr,bend left,>->,"l"]\\
& A \ar[r,>->,"f"] \ar[d,>->,swap,"g"]& B \ar[d,>->,""] \\ & C \ar[r,>->,""] & B\star_A C
\end{tikzcd}\]
We omit labelling the coproduct injection maps because they are simply inclusions, by construction.

If we have $n\in N$ such that $l(n)=k(n)$ in $B\star_A C$, by definition this means there exists some $a\in A$ such that $f(a)=l(n)$ and $k(n)=g(a)$. Construct the definable set (in fact, a definable relation):
$$F_\psi:=\{ (n,a)\in N\times A \mid f(a)=l(n)\quad \& \quad g(a)=k(n)\}.$$
We claim this is the graph of a function. We know this is total, by our argument above. Further, this $a$ is unique by injectivity of $f$ and $g$ -- this gives single-valuedness. It is also clearly well-defined. Hence, $\psi$ is a definable function. It is also clear that it is unique since any other function has to satisfy the exact same criterion.


(2) Suppose we have a pushout square, plus a pullback square with $D$ at the lower right corner. The pushout property yields a definable map $\varphi$, which we want to show is injective.

\[\begin{tikzcd}
& A \ar[r,>->,"f"] \ar[d,>->,swap,"g"]& B \ar[d,>->,""] \ar[ddr,>->, bend left, "l"]\\ & C \ar[drr,bend right,>->,swap,"k"] \ar[r,>->,""]  & B\star_A C \ar[dr,dashed,"\varphi"]\\
&&& D
\end{tikzcd}\]
It helps to first make definability of $\varphi$ explicit -- this works for general $l$ and $k$. By construction, each element $x\in B\star_A C$ is definably either an element $x\in B$ or $x\in C$, modulo identifications along $A$. Define the set
$$F_\varphi:= \{ (x,d)\in B\star_A C \times D \mid (x\in B \land d=l(x))\bigvee (x\in C\land d=k(x))\}$$
This is the graph of a definable function. It is total (every element is in $B$ or $C$). It is single-valued (we only need to check what happens on the overlaps. The only time where $b\in B$ is equivalent to $c\in C$ in the pushout is if there exists $a\in A$ such that $f(a)=b$ and $g(a)=c$. But in which case, $l f(a) = kg(a)$). This also shows the function $\varphi$ is well-defined.

So why is it an injection? Suppose $x,y\in B\star_A C$ such that $\varphi(x)=\varphi(y)$. We aim to show $x=y$ in $B\star_A C$.

We now check the various cases.
\begin{itemize}
\item $x,y\in B$. Then both $\varphi(x)=l(x),\varphi(y)=l(y)$. Since $l$ is injective, $x=y$.
\item $x,y\in C$. Same logic: $\varphi(x)=k(x), \varphi(y)=k(y)$, and so $k$ injective implies $x=y$.
\item $x\in B$, $y\in C$. Then $\varphi(x)=l(x)$, $\varphi(y)=k(y)$. If $l(x)=k(y)$, then we have elements $x\in B$ and $y\in C$ that map to the same element in $D$. Since the outer square is a pullback, there exists some $a\in A$ such that $f(a)=x$ and $g(a)=y$. And so $x\sim y$ in $B\star_A C$ by construction. 
\item $x\in C$, $y\in B$. Analogous to above.
\end{itemize}

Hence, in all cases $\varphi(x)=\varphi(y)$ implies $x=y$ in $B\star_A C$, so $\varphi$ is injective. 

	\end{comment}
	
Axioms (PQ) and (DS) follow for similar reasons as in $\Var_k$. For Axiom (A), the (definable) coproduct square is distinguished since $\M=\E=\{$definable injections$\}$.
	
\begin{comment} Axiom (PQ) follows by construction. If $B\star_A C\setminus C$, then the only thing left is $B\setminus f(A)$, which we write as $B\setminus A$. It is obvious there is a definable injection $B\setminus A \to B\star_A C$, again by construction (where identifications occur only along $A$.)

For Axiom (DS), the same reasoning as in $\Var_k$ shows that we have the following induced squares
\[\begin{tikzcd}
		B \ar[r] \ar[d,blue,"h_0"]& B\star_A C \ar[d,"\psi"]\\
		B' \ar[r,blue]& B'\star_{A'} C'
		\end{tikzcd} \qquad \begin{tikzcd}
		C \ar[r] \ar[d,blue,"h_1"]& B\star_A C \ar[d,"\psi"]\\
		C' \ar[r,blue]& B'\star_{A'} C'
		\end{tikzcd}.\]
It remains to show why this is distinguished.
\begin{itemize}
\item Why is $B\star_A C\to B'\star_{A'}C'$ an injection?  We will need both LHS and RHS squares.

Label the morphisms in the distinguished squares from Axiom (DS) as follows
\[\begin{tikzcd}
C \ar[d, {Circle[open]}->,"h_1",swap] \ar[dr, phantom, "\square"]& A \ar[dr, phantom, "\square"]\ar[d, {Circle[open]}->,"h_2"]  \ar[l, >->,swap,"g"] \ar[r, >->,"f"] & B \ar[d, {Circle[open]}->,"h_0"] \\
C' & A' \ar[l, >->,"g'",swap] \ar[r, >->,"f'"] & B' 
\end{tikzcd}\]


 Suppose $x,y\in B\star_A C$ such that $\psi(x)=\psi(y)$. Let's do a case-splitting. If $x,y\in B$, then the blue arrows guarantee that $x=y$ since they are injections. Same if $x,y\in C$. 
 
 
 Now suppose $x\in B,y\in C$. Then this means $h_0(x)=\psi(x)=\psi(y)=h_1(y)$. Notice $h_0(x)\in B', h_1(y)\in C'$. Hence, by the pushout construction, $h_0(x)=h_1(y)$ in $B'\star_{A'}C'$, here exists $a'\in A$ such that 
 $$f'(a')=h_0(x)\,\text{in}\, B'\qquad \text{and}\qquad g'(a')=h_1(y) \,\text{in}\, C'.$$
Now examine the distinguished squares given Axiom (DS). Since distinguished squares are also pullbacks, the above identities mean there exists $a_0,a_1\in A$ such that 
$$h_2(a_0)=a'=h_2(a_1)\qquad f(a_0)=x\,\text{in}\, B,\quad g(a_1)=y\,\text{in}\, C$$
Since $h_2$ is injective, this implies $a_0=a_1$. Since $f(a_0)=x$ and $g(a_0)=y$, this implies $x= y$ in  $B\star_A C$ by definition. 
The other case $x\in C,y\in B$ follows by symmetry. Hence, $\psi$ is an injection.


\item Why is the induced square distinguished? As in the case of $\Var_k$, the complement of $B\star_A C\otail B'\star_{A'}C$  is $\frac{C'}{C}$.  By \cite[Lemma 2.9]{CGW}, we obtain a unique distinguished square. 
$$\dsquare{C}{ X }{ C'}{B'\star_{A'}C'},$$
where $X\otail B'\star_{A'}C'$ is the complement of $\frac{C'}{C}\rtail C'\rtail B'\star_{A'}C'$, which is $B\star_A C\otail B'\star_{A'}C$. Anaogously, we also obtain a distinguished square
$$\dsquare{B}{ X' }{ B'}{B'\star_{A'}C'},$$
where the same argument works; notice $\frac{C'}{C}\cong \frac{A'}{A}\cong \frac{B'}{B}$  by the setup of Axiom (DS).

\end{itemize}
		\end{comment}
		
\end{example}

\begin{example}[Matroids]\label{ex:MatroidspCGW} There is a technical barrier. Suppose $M_0$ and $M_1$ are matroids with groundsets $E_0$ and $E_1$, and assume $M_0|T=M_1|T=N$ where $E_0\cap E_1=T$. An {\em amalgam} of $M_0$ and $M_1$ is a matroid $M$ on $E_0\cup E_1$ such that $M|E_0=M_0$ and $M|E_1=M_1$. Unfortunately, as noted in \cite[\S 12.4]{OxleyMatroids}, amalgams do not always exist.  In our setting, this means an arbitrary span $M_0\ltail N\rtail M_1$ may not be completable into a commutative square
	\[\begin{tikzcd}
	N \ar[r,>->] \ar[d,>->] & M_0 \ar[d,>->]\\
	M_1 \ar[r,>->] & M
	\end{tikzcd}\]
so long as we require that $M$ has groundset $E_0\cup E_1$. %That is, one cannot generally glue matroids $M_0$ and $M_1$ across their common restriction.
One solution is to relax this requirement, and allow $M$ to have any groundset. The question then becomes: given a span $M_0\ltail N\rtail M_1$, is it always possible to embed $M_0$ and $M_1$ into a larger matroid $M$? Can we regard this larger matroid as being initial in the sense of Definition~\ref{def:restPO}? Further discussion of this problem is deferred to Section~\ref{sec:ProbMatroids}. 

%\MING{Alternatively: one might wish to restrict to a class of matroids that glue well ... see modular matroids. These are "sticky". However, the problem: modular matroids may not be closed under contractions.}

\end{example}


We now set up the $K$-theory of pCGW categories via Waldhausen's $S_\bullet$-construction.

\begin{construction}[$S_\bullet$-Construction]\label{cons:SDOT}  Let $\calC$ be a pCGW category. Define $S \calC$ to be the simplicial set with $n$-simplices $S_n\calC$ given by flag diagrams
\[\begin{tikzcd} 
	M_{00}\ar[r, >->] & M_{01} \ar[r, >->] & M_{02} \ar[r, >->] & \dots\ar[r, >->]  & M_{0n} \\
	&	M_{11} \ar[r, >->] \ar[u, {Circle[open]}->]& M_{12} \ar[r, >->]\ar[u, {Circle[open]}->]  & \dots \ar[r,>->] & M_{1n}  \ar[u, {Circle[open]}->]\\
	&& M_{22} \ar[r,>->]  \ar[u, {Circle[open]}->]  & \dots   \ar[r,>->] & M_{2n} \ar[u, {Circle[open]}->] \\
	&&& \vdots\ar[u, {Circle[open]}->]  & \vdots \ar[u, {Circle[open]}->] \\
	&&&& M_{nn} \ar[u, {Circle[open]}->] 
	\end{tikzcd}\]	
subject to the conditions
\begin{enumerate}[label=(\roman*)]
	\item $M_{ii}=O$ for all $i$
	\item Every subdiagram 
	\[\begin{tikzcd}
	M_{ji} \ar[r,>->] \ar[dr,phantom,"\square"]& M_{jl}
\\
M_{ki} \ar[r,>->] \ar[u, {Circle[open]}->] & M_{kl}\ar[u, {Circle[open]}->] 	\end{tikzcd}\]
for $j<k$ and $i<l$ is distinguished. 
%together with choices 
\end{enumerate}	% \cite[\S 1.3]{Wald},
We will often represent an $n$-simplex as a sequence of $\M$-morphisms
$$O=M_0\rtail M_1\rtail M_2\rtail \dots \rtail  M_n$$
together with choices of quotients 
$$M_{j/i}:=\frac{M_j}{M_i} \qquad {i<j}.$$	
%This is justified by quotients respecting filtrations, Lemma~\ref{lem:quotFilt}. 
Degeneracy maps are obtained by duplicating an $M_i$, face maps by forgetting an $M_i$, with the addendum that forgetting $M_0$ means factoring out by $M_1$.
\end{construction}

This sets up the following translation of Theorems 4.3 and 7.8 in \cite{CGW}.

\begin{theorem}[Presentation Theorem]\label{thm:PresK0} Let $\calC$ be a pCGW category and define its $K$-theory spectrum
	$$K\calC:=\Omega|\calS\calC|,$$
with associated $K$-groups $K_n(\calC):=\pi_n K\calC$. Then $K_0(\calC)$ is the free abelian group generated by objects of $\calC$ modulo the relation that for any distinguished square
$$\dsquare{A}{B}{D}{C},$$
we have $[D]+[B]=[A]+[C].$	
\end{theorem}
\begin{comment}
\begin{proof} 	There are various ways to see this; here is one such proof. Start by applying the $Q$-construction to $\calC$ to obtain the spectra $K^Q(\calC)$.  By \cite[Thm 4.3]{CGW}, $\pi_0(K^Q(\calC))$ is precisely the free abelian group on objects of $\calC$ modulo the distinguished square relation above. Apply the standard edgewise subdivision argument to show that $K\calC$ and $K^Q(\calC)$ are weakly equivalent as spaces \cite[Thm. 7.8]{CGW}.
\end{proof}
\end{comment}


\begin{remark} In fact, Theorem~\ref{thm:PresK0} holds for any CGW category equipped with a formal direct sum satisfying Axiom (A), Definition~\ref{def:pCGW}. In the case of matroids, define 
$M_1\oplus M_2$ to be the matroid on ground set $E_1\coprod E_2$, with flats of the form $F_1\coprod F_2$ where $F_1\in \calF(M_1)$ and $F_2\in \calF(M_2)$. 
\end{remark}



\subsection{Simplicial Loops \& Fibers} Re-examining Presentation Theorem~\ref{thm:PresK0}, notice the loop space features in the definition of $K\calC$. Following \cite[\S 2]{GG}, we present a simplified model of this construction.

\begin{convention} A simplicial set is a contravariant functor $X\colon \Delta^\opp\to \Set$. We sometimes write $X[n]$ as shorthand for $X([n])$. If $A,B\in\Delta$, we write $AB$ to mean the disjoint union of $A$ followed by $B$, where elements of $A$ are below those of $B$. A 0-simplex is sometimes called a {\em vertex}, a 1-simplex an {\em edge}.
\end{convention}

To motivate, recall that the loop space $\Omega Z$ of a pointed topological space $Z$ is the space of based loops $\Map(S^1,Z)$. Now fix a simplicial set $X$ with basepoint $O\in X[0]$. A simplicial loop may look like two 1-simplices glued together at the ends
$$O\xrightrightarrows[g]{f}x,$$
while a homotopy of such loops may look like
\[\begin{tikzcd}
O \ar[r,bend left, "f_0"] \ar[r,bend right,swap, "g_0"] \ar[rr,bend left=80, "f_1"] \ar[rr,bend right=80, swap, "g_1"]& x \ar[r,"h"] & y
\end{tikzcd}\]
where we glue 2-simplices in the triangles $f_0hf_1^{-1}$ and $g_0hg_1^{-1}$ along a shared 1-simplex $h$. One can then extend this picture in a natural way to the higher homotopies as follows:

\begin{construction}[Simplicial Loops]\label{cons:LOOP} Given any simplicial set $X$, define 
	$$\Omega X(A):=\displaystyle\lim_{\leftarrow} \left( \begin{tikzcd}
	\{O\} \ar[r,hook] & X([0]) & X([0]A) \ar[d] \ar[l]\\
	& X([0]A) \ar[u] \ar[r] & X(A)
	\end{tikzcd}\right).$$
\end{construction}

Notice that Construction~\ref{cons:LOOP} works for any simplicial set $X$. This suggests the obvious definition:

\begin{definition}[$G$-Construction] Let $\calC$ be a pCGW category. The {\em $G$-Construction} on $\calC$ is defined by applying the Simplicial Loop Construction to $\calS\calC$,
	$$G\calC := \Omega S\calC.$$
\end{definition}

Although the $G$-construction is well-defined for ordinary CGW categories, we will rely on the restricted pushouts of $\M$-morphisms to show that $G\calC\simeq K\calC$ (and thus $K_n(\calC)\cong \pi_n|G\calC|$). Our approach mirrors what was done by Gillet-Grayson \cite{GG}, who established $G\calC\simeq K\calC$ for exact categories. A key notion in their argument is the so-called {\em right fiber}. %a simplicial analogue of a homotopy fiber.
 
\begin{definition}[Right Fiber]\label{def:fiber} Suppose $F\colon X\to Y$ is a map of simplicial sets, $A\in\Delta$ and $\rho\in Y(A)$. We define $\rho|F$ (``the right fiber over $\rho$'' ) by 
	
	$$(\rho|F)(B):=\displaystyle\lim_{\leftarrow}\left( \begin{tikzcd}
	&& X(B) \ar[d]\\
	& Y(AB) \ar[r] \ar[d] & Y(B)\\
	\{\rho\} \ar[r,hook] &  Y(A)
	\end{tikzcd}\right).$$
One may regard $\rho|F$ as the simplicial analogue of the homotopy fiber of $|F|$ over $\rho$. We write $\rho | Y$ for $\rho|1_Y$.  %%% Why? Fix a point, we can take the fiber over it. Except that we want this up to homotopy; recall that homotopy is tracked by higher dimensional simplices.
\end{definition}


We can now restate our problem. To show that $G\calC\simeq K\calC$, we need to show that geometric realisation and the loop space constructions commute up to homotopy equivalence, i.e. 
$$|\Omega\calS\calC|\simeq \Omega|\calS\calC|.$$
The following key observation tells us when this happens.

	\begin{observation}[Key Observation]\label{obs:homotopycartesian} For any simplicial set $X$, consider the commutative square 
		\begin{equation}\label{eq:GhomCar}
		\begin{tikzcd}
		\Omega X \ar[r,"t"] \ar[d,"b"] & O|X \ar[d,"q"]\\
		O|X \ar[r,"q"] & X
		\end{tikzcd}.
		\end{equation}
where $b$ and $t$ are the projection maps %forget one of the components X([0]A)
and $q$ the obvious face map. \underline{Then}:
\begin{enumerate}[label=(\roman*)]
	\item $O|X$ is contractible.
	\item $O|q\simeq \Omega X$
	\item  $|\Omega X|\simeq \Omega |X|$ iff this square is homotopy Cartesian.	
\end{enumerate}
\end{observation} 
\begin{proof} (i) follows from \cite[Lemma 1.4]{GG}. %% Informally, all of its $n$-simplices are connected to the base-point $O$.
	 (ii) is clear from unpacking definitions. 
	\begin{comment}
	  Explicitly, notice
	$$(O|q)(B):=\displaystyle\lim_{\leftarrow}\left( \begin{tikzcd}
	&& O|X(B) \ar[d,"q"]\\
	& X([0]B) \ar[r] \ar[d] & X(B)\\
	\{O\} \ar[r,hook] &  X([0])
	\end{tikzcd}\right)$$
	where
	$$(O|X)(B):=\displaystyle\lim_{\leftarrow}\left( \begin{tikzcd}
	& X([0]B) \ar[r,"q"] \ar[d] & X(B)\\
	\{O\} \ar[r,hook] &  X([0])
	\end{tikzcd}\right).$$
	In other words, the $B$-simplices of $(O|q)$ correspond to a pair of $B+1$ simplices in $X$ that agree at the base-point $O$, and on the $B$th face. 
	\end{comment}
	For (iii), first take the homotopy pullback $P$ of 
	$$|O|X|\to |X|\leftarrow |O|X|$$
	in the homotopy category of spaces. Taking the geometric realisation of Diagram~\eqref{eq:GhomCar}, this yields a map $|\Omega X|\to P$, which is a homotopy equivalence iff Diagram~\eqref{eq:GhomCar} defines a homotopy pullback. Finally, since $O|X$ is contractible, the homotopy pullback $P$ is equivalent to the homotopy pullback of 
	$$ \ast \to |X| \leftarrow \ast, $$
	which is the loop space $\Omega|X|$.
\end{proof}

Key Observation~\ref{obs:homotopycartesian} suggests the following proof strategy. By item (iii), in order to show $G\calC\simeq K\calC$ it suffices to verify that the square
	\begin{equation}%\label{eq:GhomCar}
\begin{tikzcd}
\Omega \calS\calC \ar[r,"t"] \ar[d,"b"] & O|\calS\calC \ar[d,"q"]\\
O|\calS\calC \ar[r,"q"] & \calS\calC
\end{tikzcd}.
\end{equation}
is homotopy Cartesian. By item (ii), this is equivalent to showing that 
	\begin{equation}%\label{eq:GhomCar}
\begin{tikzcd}
O|q  \ar[r,"t"] \ar[d,"b"] & O|\calS\calC \ar[d,"q"]\\
O|\calS\calC \ar[r,"q"] & \calS\calC
\end{tikzcd}.
\end{equation}
is homotopy Cartesian. %\footnote{To streamline notation, we have chosen to leave the labels of the maps in the new diagram unchanged; we hope this will not cause too much confusion.} 

To demonstrate this, we will rely on a simplicial translation of Quillen's Theorem B:

\begin{theorem}[\cite{GG}, Theorem B']\label{thm:B'} Suppose $F\colon X\to Y$ is a map of simplicial sets. Suppose for any $A\in \Delta,$ any $\rho\in Y(A)$, and any $f\colon A'\to A$, the induced map
	$$ \rho|F \to f^\ast \rho|F$$
is a homotopy equivalence. Then the square
$$\squaref{\rho|F}{X}{\rho|Y}{Y}{ }{ }{ }{}$$
is homotopy Cartesian.	
\end{theorem}

\begin{convention} Following \cite{GG,GGErratum}, we hereafter refer to Theorem~\ref{thm:B'} as Theorem B'.
\end{convention}


\section{A Technical Result on Right Fibers}\label{sec:RightFib}

\begin{convention}\label{conv:pcgw-C} Hereafter, unless stated otherwise: $\calC$ denotes a fixed pCGW category, and all definitions and constructions are made with respect to its pCGW structure.
\end{convention}


The goal of this section is to prove Theorem~\ref{thm:thmC'}, which essentially says: given a nice simplicial map $F\colon Y\to \calS\calC$ where $\calC$ is a pCGW category, the right fiber $O|F$ admits a nice description. %In addition, the direct sum induces an $H$-space structure on $O|F$. 
The results here are technical, and are inspired by Grayson's framework of {\em dominant functors} \cite{GraysonThmC}. 

There are two main applications of Theorem~\ref{thm:thmC'}. First, the key result that $G\calC\simeq K\calC$ is obtained as a straightforward corollary (Theorem~\ref{thm:Gconstruction}). Second, it sets up the proof of Theorem~\ref{thm:Sherman}, which gives an initial characterisation of the generators of $K_1(\calC)$; the details will be deferred to Section~\ref{sec:generators}.


\subsection{$H$-Space Structure} We introduce the new notion of a {\em moral subset}, which is similar to a simplicial subset of $\SC$ except we allow for a shift in dimensions.

% A reference on simplicial sets: https://kerodon.net/tag/0004
\begin{definition}[Moral Subset] Fix $k\in \N$. A {\em $k$-moral subset} of $\SC$ is a simplicial set $Y$ together with a map
	$$F\colon Y\longrightarrow \calS\calC$$	
such that:
\begin{enumerate}[label=(\roman*)]
	\item If $k=0$, then $F$ is the inclusion of a simplicial subset $Y\subseteq \SC$.
% Explicitly:	$Y[n]\subseteq \SC[n]$ $\forall n$, and $F$ is the inclusion map.
	\item If $k\geq 1$, fix a base-point $O_Y:=(O\rtail \dots \rtail O)\in \SC[k-1]$. Then, all $n$-simplices in $Y$ have the form
	$$(O_Y\rtail J_0\rtail J_1\rtail \dots \rtail J_n)\in \SC([k-1][n])$$
The map $F$ is defined by forgetting $O_Y$ and factoring the rest by $J_0$.
\begin{comment} Remark: We could in principle choose base-points $\rho$ other than $O_Y$, but we would require $Y$ to be closed under restricted pushouts $A\leftarrow \rho\to B$ in order to define the addition map below. We will not need this generality, so it's cleaner to just have $O_Y.$
\end{comment}
	\item $Y$ inherits its face and degeneracy maps from $\SC$ in the obvious way. In addition, we require $Y$ to be closed under direct sums of $n$-simplices.\footnote{By which we mean we can take direct sums of distinguished squares in the obvious manner, see Lemma~\ref{lem:DirectSum}.}
\end{enumerate}
\end{definition}

%The following example gives an example of a non-trivial moral subset, and justifies the definition.

\begin{example} The quotient map $q\colon O|\SC\to \SC$ from Observation~\ref{obs:homotopycartesian} is a 1-moral subset. 
\end{example}


\begin{construction}[Addition Map]\label{cons:Hspace} Fix $A\in \Delta$, where $A=[a]$. Given a $k$-moral subset $F\colon Y \to \calS\calC$ and $\overline{M}\in \SC (A)$, represent a $q$-simplex $W$ in $\overline{M}|  F$ as
	\begin{equation}\label{eq:WMSF}
	W=\begin{pmatrix}
	\doubleunderline{\Wtop}\\ \Wbot
	\end{pmatrix}=\begin{pmatrix} &  && &(O_Y&\rtail) &  \doubleunderline{J_0\rtail J_1 \rtail \dots \rtail J_q}\\ O\rtail M_1&\rtail &\dots &\rtail &M_a &\rtail &  L_0\rtail L_1\rtail \dots \rtail L_q\end{pmatrix}
	\end{equation}	
where $\Wtop$ is a simplex of $Y$, $\Wbot$ a simplex of $\SC$ and the double line represents the identity
 \[\begin{tikzcd}
O=F\left(J_0\right) \ar[d,equal] \ar[r,>->]& \dots \ar[r,>->]& F\left(J_q\right) \ar[d,equal]  \\
O=\frac{L_0}{L_0} \ar[r,>->]& \dots \ar[r,>->] & \frac{L_q}{L_0}
\end{tikzcd}.\]
Parentheses are placed around $(O_Y\rtail)$ in Equation~\eqref{eq:WMSF} to indicate that when $Y$ is a $0$-moral subset, we delete $O_Y\rtail$ from $\Wtop$ and set $J_0=O$. 

Define addition $(W,W')\mapsto W+W'$ by setting
$$W+W':=\begin{pmatrix} &(O_Y  &\rtail) & J_0\oplus J'_0\rtail \dots \rtail J_q\oplus J'_q\\ O \rtail \dots \rtail& M_a &\rtail & \doubleoverline{L_0\star_{M_a}L'_0\rtail \dots \rtail L_q\star_{M_a}L'_q} \end{pmatrix},$$
with the following quotients:
\begin{itemize}
	\item For $J_{i}\oplus J'_i$:
	$$\frac{J_i\oplus J'_i}{J_j\oplus J'_j}:=\frac{J_i}{J_j}\oplus \frac{J'_i}{J'_j}.$$
	\item For $L_i\star_{M_a}L'_i:$
	$$\frac{L_i\star_{M_a}L'_i}{M_j}:=\begin{cases}
	\frac{L_i}{M_a}\oplus \frac{L'_i}{M_a} \qquad \text{if } j=a\\
	\frac{L_i\star_{M_a}L'_i}{M_j} \qquad  \,\,\text{ by Axiom (K), if } 1\leq j<a \,\,
	\end{cases},$$
and recursively,
	$$\frac{L_i\star_{M_a} L'_i}{L_j\star_{M_a}L'_j}:=F\left(\frac{J_i\oplus J'_i}{J_j\oplus J'_j}\right).$$
\begin{comment}
In the case $k=0$, $F$ is just the usual inclusion functor so the input remains unchanged.
In the case $k\geq 1$, then $F(J_i/J_j)$ becomes the quotient of $\frac{J_i}{J_0}\rtail \frac{J_j}{J_0}$.
\end{comment}	
	
\end{itemize}
\end{construction}



%\begin{remark} In the case where $F=\id$ and $\overline{M}=O$, it will be convenient to represent the $n$-simplices of $O|\calS\calC$ as filtrations of the form $O\rtail J_0 \to \dots \rtail J_n$. 
	 %[where $J_0$ need not be $O$].
%\end{remark}



%Recall: an {\em $H$-space} is a triple $(X,e,\dotp)$ whereby $X$ is a space, $e\in X$ is a point, and $\dotp\colon X\times X\to X$ is a continuous map such that $e\dotp e=e$ and the maps $x\mapsto x\dotp e$ and $x\mapsto e\dotp x$ are homotopic to the identity map. 


\begin{claim}\label{claim:Hspace} The addition map above turns $\overline{M}|  F$ into a homotopy associative and commutative $H$-space, making $\piMSF$ a monoid.
\end{claim}
\begin{proof}
This follows from a series of basic checks:
	\begin{enumerate}[label=(\alph*)]
		\item {\em Well-definedness.} One can check Construction~\ref{cons:Hspace} indeed defines a simplicial map $$+\colon \overline{M}|F\times \overline{M}|F\to \overline{M}|F.$$
		Choices of quotients are valid in $\SC$ by Lemma~\ref{lem:Hspace1}. The top row of $W+W'$ defines a simplex in $Y$ since moral subsets are closed under direct sums.  Finally, apply Fact~\ref{facts:restrictedpushouts} and Lemma~\ref{lem:DirectSum} to verify that $(W,W')\mapsto W+W'$ commutes with face and degeneracy maps.
		\begin{comment}
		Let us set the stage: how do we get $n$-simplices from $+$ map? Let's look at the bottom row of $W+W$:
		$$O\rtail M_1\rtail  \dots \rtail M_a\rtail L_0\star_{M_a} L'_0\rtail  L_1\star_{M_a} L'_1\rtail \dots \rtail  L_q\star_{M_a} L'_q$$
		This is well-defined by taking restricted pushouts. We know what the quotients of all these terms are with respect to $M_1$. By the fact that quotients respect filtrations (Lemma~\ref{lem:quotFilt}), this defines a sequence of distinguished squares.  For the $O\rtail \dots \rtail M_a$, everything stays as is; for the $ L_i\star_{M_a} L'_i$ onwards, the choice of quotients is already given, and justified by Lemma~\ref{lem:Hspace1}
		Quotienting by each term along this filtration, we get a $q+a$-simplex on the bottom row. The same can be done for the top row. 
		
		The fact they still glue together to define an $n$-simplex in the right fibre follows from the fact that the $q$-faces of the top and bottom rows of $W$ and $W'$ already agree. So in both the top and bottom row, the same morphisms are being used to build the $q$-face of $W+W'$.
		
		To show $+$ is a simplicial map, we need to show it commutes with faces and degeneracies: $d(W+W)=d(W)+d(W)$ and $s(W+W)=s(W)+s(W)$. This essentially follows from the fact that functoriality of restricted pushouts allows us to do things component-wise. 
		Two observations:
		\begin{itemize}
		\item Notice we leave the $O\rtail M_0\rtail \dots M_a$  as an untouched constant -- $W+W'$ is a $q$-simplex and so the only relevant face and degeneracy maps are the ones induced by $[q]$. 
		\item The $n$-simplex is constructed from the $\M$-filtration on the top line; quotients are obtained by successively applying the helper functor $c$, and closing them up to obtain distinguished squares (Lemma~\ref{lem:quotFilt}). So to show that addition commutes with face and degeneracy maps, it suffices to show that addition commutes on the level of the $\M$-morphisms. The rest follows from uniqueness of quotients.
		\end{itemize}
		Notice also that 
		
		Details below.
		
		\begin{itemize} Because of how we constructed the $n$-simplex
			\item \textbf{Face Maps.} 
			A face either deletes an intermediate column $L_i$ [or $J_i$], or deletes the first row of the simplex. For the intermediate column, this is done by composing the distinguished squares, which of course corresponds to composing the $\M$-morphisms. 
			
		For any $i<j<k$, consider the diagram, taking repeated restricted pushouts
		\[\begin{tikzcd}
		M_a \ar[r,>->] \ar[d,>->]& L_i \ar[d,>->,] \ar[r,>->,"f"] & L_j \ar[d,>->] \ar[r,>->,"g"]& L_k \ar[d,>->]\\
		L'_i \ar[r,>->] \ar[d,>->,swap,"f'"]& L_i\star_{M_a} L'_i \ar[r,>->] \ar[d,>->]& L_j\star_{M_a} L'_i\ar[d,>->] \ar[r,>->] & L_k\star_{M_a} L'_i\ar[d,>->]\\
		L'_j \ar[r,>->] \ar[d,>->,swap,"g'"	]& L_i\star_{M_a} L'_j  \ar[d,>->]\ar[r,>->] & L_j\star_{M_a} L'_j  \ar[d,>->]\ar[r,>->] & L_k\star_{M_a} L'_j \ar[d,>->] \\
		L'_k \ar[r,>->] & L_i\star_{M_a} L'_k \ar[r,>->] & L_j\star_{M_a} L'_k \ar[r,>->] & L_k\star_{M_a} L'_k
		\end{tikzcd}\]

	That this diagram commutes follows from functoriality of restricted pushouts. In other words, we get that 
		$$(g\circ f)\star (g'\circ f')= (g\star g')\circ (f\star f'),$$
		as $\M$-morphisms between
		$$L_i\star_{M_a} L'_i \rtail L_k\star_{M_a} L'_k .$$
		\medskip 



The same also holds for formal direct sums, where we simply let $M_a=O$, and so we can apply the same argument to the top row. Hence, conclude that 
$$d(W+W)=d(W)+d(W)$$
for the non-0th face maps of intermediate columns. 
		
Now let's check the 0th face map. If $Y$ is a $k$-moral subset with $k\geq 1$, then the face maps commute for the same reason as before. In the case where $Y$ is a $0$-moral subset, then we will need to quotient the top line by $J_0$ in order to make sure the bottom of the $\M$-filtration is still $O$, so we need to show adding
$$O\rtail \frac{J_1}{J_0}\rtail \frac{J_2}{J_0}\rtail \frac{J_3}{J_0}\rtail \frac{J_4}{J_0}\rtail \dots ,\quad\text{and}\quad O\rtail \frac{J'_1}{J'_0}\rtail \frac{J'_2}{J'_0}\rtail \frac{J'_3}{J'_0}\rtail \frac{J'_4}{J'_0}\rtail \dots ,\quad  $$
which gives
$$O\rtail \frac{J_1}{J_0}\oplus  \frac{J'_1}{J'_0}\rtail  \frac{J_2}{J_0}\oplus  \frac{J'_2}{J'_0}\rtail \frac{J'_3}{J'_0} \oplus \frac{J'_3}{J'_0}\rtail \dots ,\quad  $$
is the same as
$$O\rtail \frac{J_1\oplus J'_1}{J_0\oplus J'_0}\rtail  \frac{J_2\oplus J_2'}{J_0\oplus J'_0}\rtail \dots $$
which holds by construction.
		 \item \textbf{Degeneracies.}  A degeneracy inserts an identity column $1$. By functoriality of restricted pushouts, direct sums of identities are identities, and restricted pushouts of identities are also identities. So the $\M$-filtration is the same either way. 
		 
		 This is also clear by functoriality of restricted pushouts. In more detail, given $J_i\xrtail{1} J_i$ and $J'_i\xrtail{1}J'_i$ , they add together to get
		 $$J_i\oplus J'_i\xrtail{1\oplus 1} J_i\oplus J'_i,$$
		 which is also the identity $1_{J_i\oplus J'_1}$. The same is also true when we taked $L_i\xrtail{1} L_i$ and  $L'_i\xrtail{1} L'_i$ combine to yield $L_i\star_{M_a} L'_i\xrtail{1} L_i\star_{M_a} L'_i$ -- this follows from the functoriality of restricted pushouts. This also means that their corresponding quotients are the same; since distinguished squares are closed under isomorphisms, we can always insert a degenerate column, and the degenerate column is the same both ways. 
		\end{itemize}
		\end{comment}
		\item {\em Identity.} The 0-simplex 
		\begin{equation}\label{eq:addID}
		\begin{pmatrix}  &(O_Y \rtail) & \doubleunderline{\,O\, }\\ O \rtail M_1\rtail \dots \rtail &M_a  \xrtail{1} & M_a \end{pmatrix}, 
		\end{equation}
	represents the additive identity. %, where $M_a\xrtail{1} M_a$ denotes the identity map.\newline 
Why? Use the fact that restricted pushouts are initial to deduce $J\oplus O\cong J$ for any $J\in\calC$, and $L_i\star_{M_a} M_a\cong L_i$. When adding the higher $n$-simplices, extend the 0-simplex by adding degeneracies.
		\item {\em Associativity and Commutativity.} Since restricted pushouts (and direct sums) are initial, they are associative and commutative up to natural isomorphism. These induce simplicial homotopies that make $|\MSF|$ into a homotopy associative and commutative $H$-space. 
\begin{comment} How might this simplicial homotopy look like? The proof is similar to that of our proof of the Permutation Lemma~\ref{lem:permute}.

Let $\tilde{+}\colon Y\times Y\to Y$ send $(W,W')\mapsto W'+W$.  
We can construct a simplicial homotopy $(Y\times Y)\times [1]([n])\to Y([n])$ between $+$ and $\tilde{+}$ by taking each simplex and $\alpha$, and changing things term by term via the natural isomorphism. Similarly, one can do that for the two different maps $(Y\times Y\times Y)\to Y$ that correspond to associativity of addition.
\end{comment}
	\end{enumerate}
\end{proof}

\subsection{The Main Result} We now leverage the $H$-space structure of $O|F$ to describe the right fiber, assuming $F$ satisfies a technical condition we call {\em cofinality.} %This can be read as requiring $F$ to be, in some sense, surjective.

\begin{definition}[Cofinality]\label{def:cofinal} Suppose $F\colon Y\to \calS\calC$ is a $k$-moral subset. We say $F$ has {\em cofinal image} if:
	 % define the {\em image of $F$} as
%	$$\im F:=\{ M\in \calS\calC[1] \mid M\cong F(K) \,\text{for some }\, K\in Y[1]\}.$$
\begin{enumerate}[label=(\roman*)]
	\item {\bf Case 1: $k=0$.} For any $O\rtail M\in \SC,$ there exists $T\in\ob\calC$ such that $$O\rtail M\oplus T \quad \in Y[1].$$
	\item {\bf Case 2: $k\geq 1$.} For any $C\in\ob\calC$, 
	$$O_Y\rtail C\,\,\in Y[0]\qquad \text{and}\qquad O_Y\rtail O\rtail C\in Y[1].$$
\end{enumerate}	
% We call $\im F$ \emph{cofinal} in $\calS\calC$ if for any $M\in \calS\calC[1]$, there exists $T\in\calS\calC[1]$ such that $M\oplus T\in \im F$.
\end{definition}

\begin{example} It is clear the map $q\colon O|\SC\to \SC$ has cofinal image by construction.
\end{example}

%\begin{remark} The original definition of cofinality \cite{GraysonThmC} was done for arbitrary exact functors. Our definition can be viewed as a simplicial translation of this, when $k=0$. 
%\end{remark}

%\begin{remark} Cofinality can be understood as a generalisation of (essential) surjectivity. Suppose we have a pCGW subcategory $\calD\subseteq \calC$ where all objects of $\calC$ are isomorphic to some object in $\calD$ (we leave open the possibility that $\calC$ and $\calD$ may have different distinguished squares). Then the corresponding simplicial inclusion $F\colon \calS\calD\hookrightarrow \SC$ is a 0-moral subset with cofinal image, where we let $T=O$.
%\end{remark}


This sets up the main theorem of this section.

\begin{theorem}\label{thm:thmC'}  Let $\calC$ be a pCGW category, and $F\colon Y\to \calS\calC$ a moral subset with cofinal image. \underline{Then} %the square 
	\begin{equation}\label{eq:ThmC'}
	\begin{tikzcd}
	O|F\ar[d] \ar[r] & Y\ar[d]\\
	O|\calS\calC \ar[r]& \calS\calC 
	\end{tikzcd}	
	\end{equation}
	is a homotopy Cartesian square. 
\end{theorem}

\begin{proof} Say the map $F$ is \emph{dominant} if $+$ induces a group structure on $\piMSF$ given \emph{any} $\overline{M}\in\calS\calC(A)$ for \emph{any} $A\in\Delta$. By Claim~\ref{claim:Hspace}, we already know $\piMSF$ is a monoid; dominance thus asserts the existence of inverses. Theorem~\ref{thm:thmC'} then follows from establishing two main implications.
	\begin{itemize}
		\item Step 1: If $F$ has cofinal image, then $F$ is dominant.
		\item Step 2: If $F$ is dominant, then Diagram~\eqref{eq:ThmC'} is a homotopy 
		Cartesian square.
	\end{itemize}
	
\subsubsection*{Step 1: Cofinality implies dominance} Fix some $\overline{M}\in\calS\calC(A)$. The monoid
 $\piMSF$ can be presented as
\begin{itemize}
	\item Generators: Vertices of $\overline{M}| F$, represented as
	$$W=\begin{pmatrix}  &&&&(O_Y&\rtail) &  \doubleunderline{J_0}\\ O\rtail M_1&\rtail &\dots &\rtail &M_a &\rtail &  N\end{pmatrix}$$
	
	%	$$W=\begin{pmatrix} &  &  \doubleunderline{\Wtop}\\ O\rtail M_1\rtail \dots \rtail &M_a \rtail & N \end{pmatrix},\qquad \Wtop\in Y([0])$$	
	\item Relations: 1-simplices of $\overline{M}|  F$.
\end{itemize} 	
As before, we omit the choice of quotients. In fact, any two vertices with the same presentation will have isomorphic quotients, and are thus connected by a 1-simplex.\footnote{The choice of 1-simplex is obvious, but justification takes some work. Informal proof sketch: construct the obvious simplex for the bottom row using the fact that quotients respect filtrations (Lemma~\ref{lem:quotFilt}) and are unique up to isomorphism (Axiom (K)). To show it is indeed a simplex, we must show the rightmost column of isomorphism squares are all distinguished; by goodness, it suffices to show they commute in the ambient category. We get the top square for free by Axiom (K). To see the square below it also commutes, use the fact that all $\E$-morphisms are monic  (if $\E\subseteq \calC$) or epi (if $\E^{\opp}\subseteq \calC$); this uses Axiom (M). Keep going.}
\begin{comment}
\subsection{Constructing a 1-Simplex}\label{sec:1simp} The choice of 1-simplex is obvious, but some work is needed to check that the candidate indeed defines a 1-simplex. Suppose we have two vertices $W$ and $W'$ with the same presentation. We start with the bottom row: denote the  $\frac{N}{M_i}$ for the quotients of $W$ and $\frac{N'}{M_i}$ for $W'$. Now apply the fact that quotients respect filtrations (Lemma~\ref{lem:quotFilt}) to construct the obvious $a+2$-simplex in $\SC$:
\begin{equation}\label{eq:a+2Simp}
\begin{tikzcd}%[ampersand replacement=\&]
O=M_0 \ar[r, >->] & M_1 \ar[r, >->] & M_2 \ar[r, >->] & \dots\ar[r, >->]  & M_a \ar[r,>->]& N \ar[r,>->,"1"] \ar[dr,phantom,"\square"]& N \\
&	M_{1/0} \ar[r, >->] \ar[u, {Circle[open]}->]& M_{2/0} \ar[r, >->]\ar[u, {Circle[open]}->]  & \dots \ar[r,>->] & M_{a/0}  \ar[u, {Circle[open]}->] \ar[r,>->,"f_0"] & \frac{N}{M_{0}} \ar[u, {Circle[open]}->,"g_0"]  \ar[r,>->,"\psi_0"] \ar[dr,phantom,"\square"]& \frac{N'}{M_{0}}   \ar[u, {Circle[open]}->,swap,"g_0'"] \\
&& M_{2/1} \ar[r,>->]  \ar[u, {Circle[open]}->]  & \dots   \ar[r,>->] & M_{a/1} \ar[u, {Circle[open]}->] \ar[r,>->,"f_1"]  & \frac{N}{M_{1}} \ar[r,>->,"\psi_1"]  \ar[u, {Circle[open]}->,"g_1"] & \frac{N'}{M_{1}} \ar[u, {Circle[open]}->,swap,"g'_1"] \\
&&& \vdots\ar[u, {Circle[open]}->]  & \vdots \ar[u, {Circle[open]}->] & \vdots  \ar[u, {Circle[open]}->] & \vdots   \ar[u, {Circle[open]}->] \\
&&&&  &\frac{N}{M_{a}}  \ar[u, {Circle[open]}->] \ar[r,>->,"\psi_n"] \ar[dr,phantom,"\square"]& \frac{N'}{M_{a}} \ar[u, {Circle[open]}->] \\
&&&&& O \ar[r,>->] \ar[u, {Circle[open]}->] &  \ar[u, {Circle[open]}->] O
\end{tikzcd}
\end{equation}
where the $\psi_i$ are the relevant isomorphisms from Axiom (K).  


To verify that Diagram~\eqref{eq:a+2Simp} is an $a+2$-simplex, it suffices to check the indicated squares in the rightmost column are distinguished. Let us examine this in detail.
\begin{itemize}
	\item \textbf{Top and Bottom Squares.} The top square is distinguished by Axiom (K). The bottom square is also obviously distinguished. %The bottom square is distinguished since distinguished squares are closed under isomorphisms.
	\item \textbf{Intermediate Squares.} Consider the square
	\begin{equation}\label{eq:1stSQ}
	\begin{tikzcd} \frac{N}{M_{1}} \ar[r,>->,"\psi_1"] \ar[d, {Circle[open]}->, "g_1"] & \frac{N'}{M_{1}} \ar[d, {Circle[open]}->, "g'_1"] \\ 
	\frac{N}{M_{0}} \ar[r, >->,"\psi_{0}"] & \frac{N'}{M_{0}}.\end{tikzcd}
	\end{equation}
	\begin{itemize}
		\item {\em Case 1: $\E\subseteq \calC$.}  By Axiom (K) and Lemma~\ref{lem:quotFilt} (``Quotients respect filtrations''), we have that   $g_0'  g_1'  \psi_1 = g_0 g_1$ and $g'_0  \psi_0 = g_0$ in $\calC$. This yields
		$$g'_0g'_1\psi_1 = g'_0\psi_0g_1,$$
		which in turn implies $$g'_1\psi_1=\psi_0g_1$$ 
		since all morphsims in $\E$ are monic in ambient category $\calC$. Conclude that the given square is distinguished since distinguished squares are closed under isomorphisms. 
		\item {\em Case 2:  $\E^\opp\subseteq \calC$.} Analogously, we get $g'_0=\psi_0g_0$ and $\psi_1g_1g_0=g'_1g'_0$ in $\calC$, and so 
		$$ \psi_1 g_1 g_0 = g'_1\psi_0 g_0$$
		in $\calC$. Since all morphisms in $\E^\opp$ are epi in $\calC$,  conclude 
		$$ \psi_1 g_1 = g'_1\psi_0.$$
	\end{itemize}
	By Cases 1 and 2, we've shown Diagram~\eqref{eq:1stSQ} is a distinguished square for any pCGW category. An inductive argument applies the same reasoning to the remaining squares, verifying that each is distinguished. 
\end{itemize}
In sum: we've shown that Diagram~\eqref{eq:a+2Simp} is an $a+2$-simplex of $\calS\calC$. It is clear  forgetting the final or second last column corresponds to the bottom row of $W$ and $W'$ in $\MSF$ respectively. 



The top row can be defined (and verified) analogously. It only remains to show that the a+2-simplex on the bottom row and the k+1-simplex on the top row glue. But that's clear since $N\rtail N$ and $J_0\rtail J_0$ all have quotient $O$, so we glue the $O\rtail O$ to $O\rtail O$. So the two simplices can be glued together to form the desired 1-simplex $W\to W'$.
\end{comment}
Hence, simplify the generators to
\begin{equation}\label{eq:W}
W=\begin{pmatrix}
(O_Y&\rtail) & \doubleunderline{J_0} \\ M_a&\rtail& N
\end{pmatrix}.
\end{equation}

To construct its inverse in $\piMSF$, it will be helpful to consider the case of $0$-moral subsets separately.
\subsubsection*{\textbf{Case I: $k= 0$}} Since $F$ has cofinal image, there exists some $T\in\ob\calC$ such that 
$$O\rtail \frac{N}{M_a}\oplus T\quad \in Y[1].$$
Setting $N':=M_a\oplus T$, it follows from Fact~\ref{facts:restrictedpushouts} that $$N\star_{M_a} N'\cong N\oplus T.$$ 
% Apply  Fact~\ref{facts:restrictedpushouts} to $T\leftarrowtail O \rtail M_a \rtail N$.
Thus, define a new vertex $W'$ whereby
$$W+ \underbrace{\begin{pmatrix}
&& \doubleunderline{O} \\ M_a&\rtail& N'
\end{pmatrix}}_{W'}=\begin{pmatrix}
&& \doubleunderline{\quad \,O\, \quad } \\ M_a&\rtail& N\oplus T
\end{pmatrix}.$$
We show $W'$ is the inverse of $W$ by verifying that $W+W'$ lies in the connected component of the additive identity. This is because
$$\begin{pmatrix}
&& \doubleunderline{\quad O \rtail \frac{N}{M_a}\oplus T\quad\,\, } \\ M_a&\rtail& M_a \rtail N\oplus T
\end{pmatrix}$$
defines a 1-simplex in $\piMSF$, since $\frac{N}{M_a}\oplus T\cong \frac{N\oplus T}{M_a}$ by Lemma~\ref{lem:ADTsquares}. Since the $0^{\mathrm{th}}$ face map acts by forgetting the second $M_a$ on the bottom row and factoring the top by $\frac{N}{M_a}\oplus T$, conclude that the 1-simplex connects the additive identity (Equation~\eqref{eq:addID}) to $W+W'$.


\subsubsection*{\textbf{Case II: $k\geq 1$}} The argument is similar. Given $W$ as above, use cofinality to define a vertex $W'$ whereby
$$W+ \underbrace{\begin{pmatrix}
	O_Y&\rtail & \doubleunderline{\quad \,\frac{N}{M_a}\quad \,} \\ M_a&\rtail& M_a\oplus J_0
	\end{pmatrix}}_{W'}=\begin{pmatrix}
O_Y&\rtail & \doubleunderline{ \frac{N}{M_a} \oplus J_0} \\ M_a&\rtail& N\oplus J_0
\end{pmatrix}.$$
The sum $W+W'$ is then connected to the additive identity via the 1-simplex
$$\begin{pmatrix}
O_Y&\rtail& \doubleunderline{ O \rtail \frac{N}{M_a}\oplus J_0} \\ M_a&\rtail& M_a \rtail N\oplus J_0
\end{pmatrix}.$$
	

In sum: given any $W\in \piMSF$, we can use cofinality to construct its inverse with respect to $+$. Since  $\overline{M}\in\calS\calC(A)$ and $A\in\Delta$ were chosen arbitrarily, this shows that $F$ is dominant.
	
	\subsubsection*{Step 2: $O|F$ as a homotopy pullback} Fix $\overline{M}\in\calS\calC(A)$ for some $A\in\Delta$, and fix $f\colon A'\to A$. By Theorem B' (Theorem~\ref{thm:B'}), it suffices to show that the base-change map $f^\ast\colon \MSF\to f^\ast\MSF $ is a homotopy equivalence.  By Step 1, we can use the fact that $\piMSF$ is a group since $F$ is dominant.
	
	\subsubsection*{Step 2a: A reduction} %% 2-of-3 property for homotopy equivalences
	Let $g\colon [0]\to A'$ be any morphism in $\Delta$. To show that $f^\ast$ is a homotopy equivalence, it suffices to show that $(fg)^\ast=g^\ast f^\ast$ and $g^\ast$ are. In fact, since both $fg$ and $g$ have $[0]$ as source, it suffices to show that 
	$$f_i\colon [0]\to A,\qquad f_i(0)=i\,\,\text{for}\, i\in A,$$
	induces a homotopy equivalence for any $A\in\Delta$. Notice $f^\ast_i$ defines a map
	$$f^\ast_i\colon \MSF \to O| F$$
	since $O$ is the only vertex of $\calS \calC$. %% and so $f^\ast(M)=O$.
	
	\subsubsection*{Step 2b: The base case} Define a map 
	\begin{align}
	h_0\colon O|  F &\longrightarrow \MSF \nonumber\\
	{\begin{pmatrix} &(O_Y\rtail) & \doubleunderline{J_0\rtail \dots \rtail J_q}\\ &O \rtail & L_0\rtail \dots \rtail L_q \end{pmatrix}} & \longmapsto {\begin{pmatrix} & (O_Y\rtail) & J_0\rtail \dots \rtail J_q\\ O\rtail \dots \rtail  &M_a \rtail &\doubleoverline{M_a\oplus  L_0\rtail \dots \rtail M_a\oplus L_q}\end{pmatrix}}\nonumber 
	\end{align}
	with quotients defined as
	\begin{itemize}
		\item $\frac{M_a\oplus L_j}{M_a\oplus L_k}:=\frac{L_j}{L_k}\left(= F\left(\frac{J_j}{J_k}\right)\right)$, 
		\item $\frac{M_a\oplus L_j}{M_a}:=L_j$, \qquad $\frac{M_a \oplus L_j}{M_i}:=\frac{M_a}{M_i}\oplus L_j$ 
	\end{itemize}
	
	
	To show that $f^\ast_i$ is a homotopy equivalence (for arbitrary $i$), it suffices to establish the following claim.
	
	\begin{claim}\label{claim:thmC'} The maps $f^\ast_i\circ h_0$ and $h_0$ are homotopy equivalences. 
	\end{claim}
	\begin{proof}[Proof of Claim] Two main checks.
		\begin{enumerate}[label=(\roman*)]
			\item \emph{On $f^\ast_i\circ h_0$.} The map $f^\ast_i\circ h_0\colon O|F\to O|F$ sends %\footnote{$f^\ast_i$ corresponds to picking the $i$th coordinate in $\MSF$ and forgetting the rest; so this gives $M_i\rtail M_a\oplus L_0 \dots$. But we need the filtration to have $O$ at the bottom, so we quotient by $M_i$.}
			
			$$ \begin{pmatrix} & (O_Y\rtail)& \doubleunderline{J_0\rtail \dots \rtail J_q}\\ &O \rtail & L_0\rtail \dots \rtail L_q \end{pmatrix}  \longmapsto
			\begin{pmatrix}  (O_Y\rtail) & J_0\rtail \dots \rtail J_q\\ O \rtail &\doubleoverline{\frac{M_a}{M_i}\oplus  L_0\rtail \dots \rtail \frac{M_a}{M_i}\oplus  L_q}\end{pmatrix} $$ %% Notice that the quotients still match up with $J_q/J_0$ since we set ($\frac{M_a}{M_i}\oplus L_j/\frac{M_a}{M_i}\oplus L_0)\cong L_{j}/L_0$
			for $0\leq i\leq a$. Since $O|F$ is an $H$-space, we can reformulate $f_i^\ast\circ h_0$ more suggestively as 
			$$\left(f_i^\ast\circ h_0\right) (W) = W + \begin{pmatrix}
			(O_Y\rtail) & O \\
			O\rtail & \doubleoverline{\frac{M_a}{M_i}}
			\end{pmatrix}^{(q)},$$
where the notation ``$(q)$'' indicates a (potentially degenerate) $q$-simplex formed by repeating the vertex $q$ times. Since $\pi_0(O|F)$ is a group on the vertices of $O|F$, there exists a vertex $V$ such that 
			$$\begin{pmatrix}
		(O_Y\rtail )	& O \\
			O\rtail & \doubleoverline{\frac{M_a}{M_i}}
			\end{pmatrix} +   V \sim \begin{pmatrix}
		(O_Y\rtail )	& O \\
			O\rtail & \doubleoverline{O}
			\end{pmatrix}.$$
Since $+$ is a simplicial map, it commutes with face and degeneracy maps. In particular, degeneracies preserve identities and inverses: if $x + x^{-1} \sim e$ for a vertex $x$, then $$s_0^q(x) + s_0^q(x^{-1})\sim s_0^{q}(x+x^{-1}) \sim s_0^q(e).$$
Thus, define $h_1\colon O|F\to O | F$ by setting
			$$ h_1(W) = W +  V^{(q)}$$
			for any simplex $W$. Since $+$ is homotopy associative and commutative, conclude that
			$$f_i^\ast \circ h_0\circ h_1\sim 1,\qquad h_1\circ f_i^\ast \circ h_0\sim 1.$$
			\item \emph{On $h_0$.} Notice: $f^\ast_a\circ h_0$ is isomorphic to the identity map on $O|F$. It therefore suffices to show $h_0\circ f^\ast_a$ is homotopic to the identity map $1$ on $\MSF$. But this follows from the natural isomorphism 
			$$h_0\circ f^\ast_a \cong 1,$$
			induced by the isomorphism
			$$M_a\oplus \frac{L_j}{M_a}\cong \frac{M_a\oplus L_j}{M_a}= L_j,\qquad \text{for all}\, j,$$
		a consequence of Lemma~\ref{lem:ADTsquares} (iii) and the choice of quotients by $h_0$.\footnote{Notice the specific choice of quotients by $h_0$ is crucial; otherwise, the isomorphism may fail to hold since e.g. not all short exact sequences split.}
		%%% Remark: In Grayson and the original Gillet-Grayson paper, the natural isomorphism involves looking at 1+h_0\circ f^*_a\cong 1+1, and then using the group structure to subtract the 1's. But this seems unnecessary; the isomorphism can be constructed directly, and really follows from the choice of quotients made before.
		\end{enumerate}
		This completes proof of Claim~\ref{claim:thmC'}.
	\end{proof}
	
	\subsubsection*{Step 3: Finish.} Fix a $k$-moral subset $F\colon Y\to\calS\calC$ that has cofinal image. Step 1 showed that $F$ is dominant, i.e. $\piMSF$ is a group for any $\overline{M}\in\SC(A)$. Step 2 combines this with Theorem B' to show that Diagram~\eqref{eq:ThmC'} is homotopy Cartesian. This proves the theorem.
\end{proof}

The following corollary justifies viewing the right fiber (Definition~\ref{def:fiber}) as the simplicial analogue of a homotopy fiber, and will be useful in Section~\ref{sec:generators}.

\begin{corollary}\label{cor:HtpyFib} Suppose $F\colon Y\to \calS\calC$ is $k$-moral subset with cofinal image. Then $|O|F|$ is homotopy equivalent to the homotopy fiber of $|F|$.
\end{corollary}
\begin{proof} By Observation~\ref{obs:homotopycartesian},  $O|\calS\calC$ is contractible. Since the homotopy fiber of $|F|$ is the homotopy pullback of the cospan  $\ast \to |\calS\calC| \xleftarrow{|F|} Y $, the statement follows.
\end{proof}


In addition, we now obtain a key result of this paper regarding the $G$-construction.

\begin{theorem}\label{thm:Gconstruction} Given any pCGW category $\calC$, there is a homotopy equivalence $$|G\calC|\xrightarrow{\sim}\Omega |S\calC|.$$
Further, direct sum induces an $H$-space structure on $G\calC$.
\end{theorem}
\begin{proof} Let us review Key Observation~\ref{obs:homotopycartesian}. By item (iii), $|G\calC|=|\Omega\calS\calC|\simeq \Omega|\calS\calC|$ iff
		\begin{equation*}%\label{eq:GhomCar}
	\begin{tikzcd}
	\Omega \calS\calC \ar[r,"t"] \ar[d,"b"] & O|\calS\calC \ar[d,"q"]\\
	O|\calS\calC \ar[r,"q"] & \calS\calC
	\end{tikzcd}
	\end{equation*}
is homotopy Cartesian. By item (ii), we have $O|q\simeq \Omega \calS\calC$. Since $q$ is a $1$-moral subset with cofinal image, the rest follows from Theorem~\ref{thm:thmC'}; the $H$-space structure on $G\calC$ comes from Construction~\ref{cons:Hspace}.%  and (once again) the fact that $G\calC\simeq O|q.$ 
\end{proof}

\begin{remark} The equivalence in Theorem~\ref{thm:Gconstruction} established between $G\calC$ and $K\calC$ is one of topological spaces, not of infinite loop spaces or spectra.
\end{remark}

\begin{discussion}[Comparison with other proofs] Our approach combines ideas from two existing proofs of Theorem~\ref{thm:Gconstruction} when $\calC$ is an exact category. While all three proofs rely on Theorem B', there are some key differences worth mentioning.
%% There is another approach, by Waldhausen et. al."An Un-delooped version of $K$-Theory." See Remark 2.7 of the paper. 
	
	In broad strokes, Theorem B' says: if the induced map on fibers $\rho|F\to f^\ast \rho|F$ is a homotopy equivalence for {\em any} $f\colon A'\to A$, then $\rho|F$ is a homotopy pullback. In their original  paper \cite{GG}, Gillet-Grayson simplifies this condition by restricting to the maps $f_0,f_1\colon [0]\to [1]$; compare this with Step 2 of our proof of Theorem~\ref{thm:thmC'}.  Their argument is technical, but reflects the informal intuition: to understand how objects break into finitely many pieces, it suffices to understand how to break a single object into two.
	
	A different proof appears in \cite[Thm 8.2]{GraysonThmC}, where Grayson uses Theorem B' to study {\em dominant} exact functors. By contrast, notice we define dominance for {\em simplicial maps}. This adjustment accounts for the fact that $O|\SC$ does not {\em a priori} correspond to a pCGW category. Further, our definitions of moral subsets and cofinality were designed for our arguments to go through smoothly; no real attempts at generality were made. Whereas Grayson gives a full characterisation of dominant functors \cite[Thm 2.1]{GraysonThmC}, his proof does not translate well to our setting; for our purposes, it suffices to identify a sufficient criterion for dominance, i.e. cofinality.
	%%% Why does it not translate well? Grayson proceeds by arguing that the generators of \piMSF can be simplified by simply forgetting the top row. This is fine if we are looking at simplicial maps induced by exact functors or even CGW functors, since the top row will always be identically $O$. This is no longer true if we are looking at k-moral subsets; trying to disentangle this becomes messy and distracts from the main thrust of this paper.
\end{discussion}

\section{Generators of $K_1(\calC)$}\label{sec:generators}

Having established Theorem~\ref{thm:Gconstruction}, we now begin to deliver on our promise that the $G$-construction leads to an explicit description of $K_1(\calC)$. Section~\ref{sec:Gcons} unpacks the definition of the $G$-construction. Sections~\ref{sec:techprelim} and ~\ref{sec:genK1} work to obtain an increasingly sharp description of the generators of $K_1$; our analysis extends various results from \cite{ShermanAbelian,ShermanExact,NenGen}.

\subsection{Review of $G$-Construction}\label{sec:Gcons} 

\begin{construction}[$G$-Construction]\label{cons:Gcons} An $n$-simplex of $G\calC$ is a pair of flag diagrams of the form
	
	\[\small{\begin{tikzcd}%[ampersand replacement=\&]
		P_0\ar[r, >->] & P_1 \ar[r, >->] & P_2 \ar[r, >->] & \dots\ar[r, >->]  & P_n \\
		&	P_{1/0} \ar[r, >->] \ar[u, {Circle[open]}->]& P_{2/0} \ar[r, >->]\ar[u, {Circle[open]}->]  & \dots \ar[r,>->] & P_{n/0}  \ar[u, {Circle[open]}->]\\
		&& P_{2/1} \ar[r,>->]  \ar[u, {Circle[open]}->]  & \dots   \ar[r,>->] & P_{n/1} \ar[u, {Circle[open]}->] \\
		&&& \vdots\ar[u, {Circle[open]}->]  & \vdots \ar[u, {Circle[open]}->] \\
		&&&& P_{n/n-1} \ar[u, {Circle[open]}->] 
		\end{tikzcd}}\quad \small{\begin{tikzcd}%[ampersand replacement=\&]
		P'_0\ar[r, >->] & P'_1 \ar[r, >->] & P'_2 \ar[r, >->] & \dots\ar[r, >->]  & P'_n \\
		&	P_{1/0} \ar[r, >->] \ar[u, {Circle[open]}->]& P_{2/0} \ar[r, >->]\ar[u, {Circle[open]}->]  & \dots \ar[r,>->] & P_{n/0}  \ar[u, {Circle[open]}->]\\
		&& P_{2/1} \ar[r,>->]  \ar[u, {Circle[open]}->]  & \dots   \ar[r,>->] & P_{n/1} \ar[u, {Circle[open]}->] \\
		&&& \vdots\ar[u, {Circle[open]}->]  & \vdots \ar[u, {Circle[open]}->] \\
		&&&& P_{n/n-1} \ar[u, {Circle[open]}->] 
		\end{tikzcd}}\]
	subject to the conditions: 
	\begin{enumerate}[label=(\roman*)]
	\item All drawn squares are distinguished. Further, the quotient index square
		\[ \begin{tikzcd}
		P_{j/i} \ar[r,>->] \ar[dr, phantom, "\square"] & P_{k/i} \\
		P_{j/l} \ar[u, {Circle[open]}->]\ar[r,>->]& P_{k/l} \ar[u, {Circle[open]}->]
		\end{tikzcd}\qquad \text{where $i<l<j<k$}\]
is the same in both flag diagrams. %\footnote{\MING{Say something about how the coinciding of quotient diagrams reflects the gluing of faces.}}
		\item Every quotient index triangle defines a distinguished square 
		\[\begin{tikzcd}
		P_{j/i} \ar[r,>->] \ar[dr, phantom, "\square"]& P_{k/i} \\
		O \ar[u, {Circle[open]}->]\ar[r,>->]& P_{k/j} \ar[u, {Circle[open]}->]
		\end{tikzcd} \qquad  \text{for any $i<j<k$},\]
		and is the same in both flag diagrams.
		\item Any $P_j\rtail P_k$ and $P'_j\rtail P'_k$ in the filtration can be completed into distinguished squares
		\begin{equation*}\label{eq:triangleDS}
		\begin{tikzcd}
		P_{j} \ar[dr, phantom, "\square"]\ar[r,>->] & P_k \\
		O \ar[u, {Circle[open]}->]\ar[r,>->]& P_{k/j} \ar[u, {Circle[open]}->]
		\end{tikzcd},\quad \begin{tikzcd}
		P'_{j} \ar[r,>->]\ar[dr, phantom, "\square"] & P'_k \\
		O \ar[u, {Circle[open]}->]\ar[r,>->]& P_{k/j} \ar[u, {Circle[open]}->]
		\end{tikzcd}\qquad  \text{for any $j<k$}.
		\end{equation*}
	\end{enumerate}
\end{construction}


\begin{convention}\label{con:vertex} Technically, an $n$-simplex of $G\calC$ is a pair of $(n+1)$-simplices in $\calS\calC$
	$$O\rtail P_0\rtail \dots \rtail P_n\qquad \qquad O\rtail P'_0\rtail \dots \rtail P'_n$$
	such that the $0^\text{th}$ faces (= forgetting $O$ and quotienting by $P_0$) agree. Here we omit the base-point $O$ for simplicity. Therefore, a {\em vertex} of $G\calC$ is a pair $(M,N)\in\calC\times\calC$,
	  and an {\em edge} $(M,N)\to (M',N')$ is a pair of distinguished squares with identical quotient
\[\left(\dsquare{O}{C}{M}{M'}, \dsquare{O}{C}{N}{N'}\right).\]
We sometimes also represent a vertex as $\begin{pmatrix}
M\\N
\end{pmatrix}$.
%\footnote{Not to be confused with the notation from Construction~\ref{cons:Hspace}, where $n$-simplices of a right fiber are denoted $W=\begin{pmatrix}
%	\doubleunderline{\Wtop}\\ \Wbot
%	\end{pmatrix}$, with double underlines.} 
\end{convention}

%\begin{remark} Since quotients are required to agree in $G\calC$, not every vertex $(M,N)\in G\calC$ is connected to the base point $(O,O)$ by an edge, since e.g. it may be the case that $M\not\cong N$. Contrast this with $\SC$.
%\end{remark}



\subsection{Sherman Loops \& Splitting}\label{sec:techprelim} Here we extend Sherman's analysis of $K_1$ for exact categories \cite{ShermanAbelian,ShermanExact} to pCGW categories. A guiding principle in his approach is that restricting to split exact sequences helps clarify the general case. This section adapts Sherman's insight to the pCGW setting, culminating in a first characterisation of the generators of $K_1(\calC)$, Theorem~\ref{thm:Sherman}. 

% Recall: an exact category $\A$ is an additive category possessing a ``formal class of short exact sequences''\footnote{More precisely, a class of triples of objects $(M'\to M\to M'')$ in $\A$ satisfying specific formal properties also satisfied by the class of short exact sequences.}.

\begin{convention}\label{conv:K1} Denote $G\calC^o$ to be the component of $G\calC$ containing the vertex $(O,O)$; we call this the {\em base-point component}. Hereafter, we take as definition
	$$K_1(\calC):=\pi_1|G\calC^o|.$$
Notice: by Theorem~\ref{thm:GG}, this is isomorphic to the standard definitions of $K_1$ (e.g. via the $S_\bullet$-construction), justifying our choice of definition.
\end{convention}

\begin{comment} Reminder: given a connected space $X$, its loop space $\Omega X$ need not be connected.

\end{comment}

\begin{construction}[Sherman Loop]\label{cons:eltG}  A \emph{Sherman triple} $(\alpha,\beta,\theta)$  consists of the following data:
\begin{itemize}
	\item Two $\M$-morphisms $A\xrtail{\alpha} B,$ $A'\xrtail{\beta} B'$;
	\item An isomorphism $\theta\colon A\oplus C\oplus B'\rtail A'\oplus C' \oplus B$, where $C$ and $C'$ are specific choices of quotients
	\begin{equation}\label{eq:doublesq}
	\dsquaref{O}{C}{A}{B}{}{}{\alpha}{\delta}\qquad \dsquaref{O}{C'}{A'}{B'}{}{}{\beta}{\gamma}.
	\end{equation}	
\end{itemize}
Its associated \emph{Sherman loop} is the homotopy class $G(\alpha,\beta,\theta)$ in $K_1(\calC)$ represented by the loop 
\begin{equation}\label{eq:shermanloop}
\begin{pmatrix}
O \\ O
\end{pmatrix} \to  \ones{A}{A} \to  \ones{A\oplus C \oplus B'}{B\oplus B'} \to  \ones{A'\oplus C'\oplus B}{B'\oplus B} \leftarrow \ones{A'}{A'} \leftarrow \ones{O}{O},
\end{equation}
where the arrows denote the obvious 1-simplices.\footnote{\label{fn:1simSherLoop}\emph{Details.} The middle arrow in Equation~\eqref{eq:shermanloop} applies $\theta$ on the top row and the canonical permutation isomorphism $B\oplus B'\to B'\oplus B$ on the bottom; the rest of the 1-simplices are defined by applying Lemma~\ref{lem:ADTsquares}. %%% One will also need to check that the quotients are the same, but this is easy.
}  
\end{construction}


\begin{remark} Fix a pair of $\M$-morphisms $\alpha,\beta$. One easily checks that any two Sherman triples $(\alpha,\beta,\theta)$ and $(\alpha,\beta,\theta')$ define the same Sherman loop in $K_1(\calC)$ up to homotopy -- see e.g. \cite[\S 1]{ShermanAbelian}.
\end{remark}

\begin{comment}
\begin{claim} The homotopy class of $G(\alpha,\beta,\Theta)$ in $K_1(\calC)$ depends only on the choice of $\alpha,\beta$ and $\theta$; here, $\theta$ just an isomorphism  as opposed to a family of isomorphisms, as in Sheman.
\end{claim}
\begin{proof} Keeping the presentation from Construction~\ref{cons:eltG}, consider another pair of 
distinguished squares
\begin{equation}
\dsquaref{O}{C_1}{A}{B}{}{}{\alpha}{\delta_1}\qquad \dsquaref{O}{C'_1}{A'}{B'}{}{}{\beta}{\gamma_1}
\end{equation}	
with isomorphism $\theta_1\colon A\oplus C_1\oplus B' \to A'\oplus C'_1\oplus B$. By Axiom (K), there exist unique isomorphisms $\gamma \colon C\otail C_1$, $\delta\colon C'\otail C'_1$ such that
\begin{equation}
\begin{tikzcd}
C \ar[dr,swap,{Circle[open]}->,"\delta"] \ar[rr,{Circle[open]}->,"\gamma"] && C_1 \ar[dl,{Circle[open]}->,"\delta_1"]\\
& B
\end{tikzcd}\qquad \begin{tikzcd}
C' \ar[dr,swap,{Circle[open]}->,"\gamma"] \ar[rr,{Circle[open]}->,"\sigma"] && C'_1 \ar[dl,{Circle[open]}->,"\gamma_1"]\\
& B'
\end{tikzcd}.
\end{equation}
Obviously, this yields a commutative diagram of isomorphisms
\begin{equation}
\begin{tikzcd}
A\oplus C \oplus B \ar[r,"\theta"] \ar[d,swap,"1\oplus\gamma\oplus 1"] & A'\oplus C'\oplus B' \ar[d,"1\oplus \sigma\oplus 1"]\\
A\oplus C_1\oplus B \ar[r,"\theta_1"]& A'\oplus C'_1\oplus B' 
\end{tikzcd}
\end{equation}
which we use to construct the diagram below.
\begin{equation}\label{eq:G2simp}
\begin{tikzcd} 
& \ones{A\oplus C_1\oplus B}{B\oplus B'} \ar[r] & \ones{A'\oplus C'_1\oplus B}{B'\oplus B} \\ 
\\
\ones{A}{A} \ar[uur] \ar[r] & \ones{A\oplus C \oplus B}{B\oplus B'} \ar[r] \ar[uu] \ar[uur] & \ones{A'\oplus C'\oplus B}{B'\oplus B} \ar[uu] & \ar[l] \ar[uul] \ones{A'}{A'}
\end{tikzcd}
\end{equation}
As a warm-up, notice the isomorphisms allow us to construct the following 2-simplex in $G\calC$
\[\left(\begin{tikzcd}
A\ar[r, >->] & A\oplus C\oplus B' \ar[r, >->] & A\oplus C_1\oplus B'  \\
&	C\oplus B' \ar[r, >->] \ar[u, {Circle[open]}->]& C_1\oplus B'\ar[u, {Circle[open]}->]  \\
&& O \ar[u, {Circle[open]}->] 
\end{tikzcd} ,\quad \begin{tikzcd}
A \ar[r, >->] & B\oplus B' \ar[r, >->] & B\oplus B'  \\
& C\oplus B\ar[r, >->] \ar[u, {Circle[open]}->]& C_1\oplus B'\ar[u, {Circle[open]}->]  \\
&& O \ar[u, {Circle[open]}->] 
\end{tikzcd}\right)\]
and so the leftmost triangle in Diagram~\eqref{eq:G2simp} bounds a 2-simplex. A similar argument shows the other triangles in the diagram also bound a 2-simplex. The claim thus follows.
\end{proof}
\end{comment}


\begin{definition}[Split] Call a distinguished square of the form 
	\begin{equation*}\label{eq:splitDS}
	\dsquaref{O}{C}{A}{B}{}{}{f}{g}
	\end{equation*} 
	an \emph{exact square}. 
	\begin{enumerate}[label=(\roman*)]
		\item An exact square is called {\em split} if there exists an isomorphism 
		$$\Psi\colon B\rtail A\oplus C$$ 
		such that the following squares commute
		\[\begin{tikzcd}
		A \ar[r,>->,"f"] \ar[d,>->,swap,"1_A"]& B \ar[d,>->,"\Psi"] \\
		A \ar[r,>->,"p_A"] & A\oplus C
		\end{tikzcd}\qquad \begin{tikzcd}
		B \ar[d, {Circle[open]}->,swap,"\varphi(\Psi)"] & C \ar[l, {Circle[open]}->,swap,"g"] \ar[d, {Circle[open]}->,"1_C"]\\
		A\oplus C & C \ar[l, {Circle[open]}->,swap,"q_C"]	 	 
		\end{tikzcd}\]
		in $\M$ and $\E$ respectively, where $p_A$ and $q_C$ are the morphisms from the obvious direct sum square.
		\item  Call an $\M$-morphism $A\rtail B$ is called \emph{split} if its corresponding exact square [obtained from Axiom (K)] is split.
	\end{enumerate}
 %\footnote{This is equivalent to the usual definition of a split monic in an exact category, i.e. a monic $f\colon A\rtail B$ with a section $g\colon B\to A$ such that $gf=\id_A$. This follows from every monic having a formal cokernel/quotient, and the splitting lemma.} 
	
\end{definition}

\begin{remark} A sanity check: suppose $\calC$ is an exact category, and so $\varphi(\Psi)$ corresponds to $$\Psi^{-1}\colon A\oplus C\to B$$
by Axiom (I). A split exact square means there exists an isomorphism $\Psi$ such that $\Psi\circ f = p_A$ and $g\circ \Psi^{-1}=q_C$ in the ambient category. Since $g\circ \Psi^{-1}=q_C\iff g=q_C\circ \Psi$, this recovers the usual definition of a split exact sequence. 
\end{remark}
	
\begin{comment} Suppose we restrict $\calC$ to $\Cplus$, whose exact squares are precisely the split exact squares. For Sherman's Theorem, we will want to apply the previous theorem on right fibers of moral subsets.

So why is $\calS\Cplus$ a moral subset?

$\calS \Cplus$ is clearly a simplicial subset of $\calS\calC$, so it remains to check that it is closed under direct sums. 

Given any two split exact squares
$$\dsquare{O}{\frac{B}{A}}{A}{B}\qquad \dsquare{O}{\frac{B'}{A'}}{A'}{B'},$$
with isomorphisms $A\oplus \frac{B}{A}\cong B$  and $A'\oplus \frac{B'}{A'}\cong B'.$ Re-expressed in the CGW context, this means we have distinguished squares
$$\dsquare{O}{A\oplus \frac{B}{A}}{O}{B}\qquad \dsquare{O}{A'\oplus \frac{B'}{A'}}{O}{B'},$$
which we know we can add to get
$$\dsquaref{O}{B\oplus B'}{O}{(A\oplus \frac{B}{A})\oplus A'\oplus \frac{B'}{A'}}{}{}{}{\Psi}.$$
In other words: adding isomorphisms yields another isomorphism. For simplicity, denote $$Z\oplus Z':=(A\oplus \frac{B}{A})\oplus A'\oplus \frac{B'}{A'}$$

It remains to check the following diagrams commute:
$$\begin{tikzcd}
A \oplus A' \ar[r,>->] \ar[d,>->]& B\oplus B' \ar[d,>->,"\zeta"] \\
A \oplus A' \ar[r,>->] & Z \oplus Z'
\end{tikzcd} \qquad \begin{tikzcd}
B\oplus B' \ar[d,{Circle[open]->},"\varphi(\zeta)=\Psi"]  & \frac{B}{A}\oplus \frac{B'}{A'} \ar[d,{Circle[open]->}] \ar[l,{Circle[open]->},"f\oplus f'",swap] 
\\
Z\oplus Z'  & \frac{B}{A}\oplus \frac{B'}{A'}\ar[l,{Circle[open]->},"g\oplus g'"] 
\end{tikzcd}$$
By hypothesis, we know that 
$$A\rtail Z \qquad \text{and}\qquad A'\rtail Z'$$
is equal to
$$A\rtail B\rtail  Z \qquad \text{and}\qquad A'\rtail B'\rtail Z'$$
with the induced isomorphisms. Hence, the $\M$-square essentially follows from taking restricted pushouts
$$Z'\ltail B'\ltail A'\ltail O \rtail A\rtail B\rtail Z$$
and performing a diagram-chase. As for the $\E$-square, it is clear the $\M$-square is a restricted pushout. Hence, apply Axiom (PQ). By convention, we assume the $\E$-morphism $\frac{B}{A}\oplus\frac{B'}{A'}\otail \frac{B}{A}\oplus\frac{B'}{A'}$ is the identity. 
\begin{itemize}
\item  If $\E\subseteq\calC$, then $\Psi=\zeta$, and the whole thing follows from commutativity of the square supplied by Axiom (PQ). 
\item If $\E^\opp\subseteq \calC$, then $\Psi=\zeta^{-1}$. Axiom (PQ) thus yields
$$g\oplus g'\circ \zeta =f\oplus f' \iff  g\oplus g' = f\oplus f'\circ \zeta^{-1}$$
in $\calC$, which is the $\E$-square once we reverse the arrows.
\end{itemize}
\end{comment}	
	
	
\begin{theorem}\label{thm:Sherman} $K_1(\calC)$ is generated by Sherman loops $G(\alpha,\beta,\theta)$. %where $\calC$ is a pCGW category. 
\end{theorem}
\begin{proof} The proof relies on two helper constructions: (i) $\Cplus$, a pCGW subcategory of $\calC$; and (ii) $\Ghat$, a simplicial subset of $G\calC$. These encode the splitting data of $\calC$ and $G\calC$ respectively, and so admit particularly nice presentations on the level of $\pi_1$. We will relate them to $K_1(\calC)$ to establish the theorem. 
	
Throughout, all $\pi_1$'s are taken at the obvious base-point components (e.g. the one containing $(O,O)$ in $\Ghat$), which we suppress from the notation.

\subsubsection*{Step 0: Setup} This step introduces the two helper constructions, and records some preliminary observations. 

\begin{construction} Define $\calC^{\oplus}$ to be the CGW subcategory of $\calC$ whose exact squares are precisely the split exact squares of $\calC$.
\end{construction}

Since $\calS\Cplus$ is closed under taking direct sums, the simplicial map
$$\calS F\colon \calS\Cplus\hookrightarrow \calS\calC,$$
induced by the inclusion CGW functor $F\colon \Cplus \hookrightarrow \calC$, is a $0$-moral subset. In fact, since $\calS\Cplus[1]=\SC[1]$, $\calS F$ has cofinal image. One can therefore apply Theorem~\ref{thm:thmC'} to construct a homotopy Cartesian diagram
 \begin{equation}\label{eq:SherHCD}
 \begin{tikzcd}
 O|\calS F \ar[d,"r"] \ar[r,"p"] & \calS \Cplus \ar[d,"\calS F"]\\
 O|\calS\calC \ar[r,"q"]& \calS\calC 
 \end{tikzcd}\quad .
 \end{equation}

\begin{construction} Following the convention of Construction~\ref{cons:Hspace}, represent an $n$-simplex in $G\calC$ as
%	$$\begin{pmatrix} & O\rtail & \doubleunderline{K_0\rtail \dots \rtail K_n}\\ &O \rtail &  L_0\rtail \dots \rtail L_n \end{pmatrix}$$
	$$\begin{pmatrix} O \rtail& \doubleunderline{L_0\rtail \dots \rtail L_n}\\ O \rtail& M_0\rtail \dots \rtail M_n \end{pmatrix}$$
	Define $\Ghat$ to be the simplicial subset of $G\calC$ where the $\M$-morphisms in the top row are all split. 
\end{construction}

Some observations. Notice this mirrors $O|\calS F$, except the top row of an $n$-simplex in $\Ghat$ is an $(n+1)$-simplex in $\SC$ whereas the top row in $O|\calS F$ is an $n$-simplex (whose $\M$-morphisms are also all split). In addition, the simplices used to define $G(\alpha,\beta,\theta)$ also belong to $\Ghat$, and so $G(\alpha,\beta,\theta)\in \pi_1|\Ghat|$.

\subsubsection*{Step 1: Relating $\Ghat$ and $\calS\Cplus$} By Corollary~\ref{cor:HtpyFib}, $|O|p|$ and $|O|q|$ define homotopy fibers. We thus work towards extracting a homotopy fiber sequence from Diagram~\eqref{eq:SherHCD}. 
% It is straightforward to check that $p$ and $q$ are 1-moral subsets, and so we may apply Corollary~\ref{cor:HtpyFib}.

\begin{claim}\label{claim:Sherhtpyfib} $|\Ghat|$ is homotopy equivalent to $|O|p|$ and $|G\calC|$.
\end{claim}
\begin{proof} Take the geometric realisation of Diagram~\eqref{eq:SherHCD} to obtain a homotopy Cartesian square of spaces. As is well-known, the map $r$ induces a homotopy equivalence on the homotopy fibers $|O|p|\to |O|q|$. Next, Observation~\ref{obs:homotopycartesian} (ii) notes that $G\calC \simeq O|q$, essentially by unpacking definitions. One can similarly verify that $\Ghat\simeq O|p$. Assembling all the equivalences, conclude that $|G\calC|\simeq |O|q|\simeq |O|p|\simeq  |\Ghat|$.
\end{proof}

\begin{comment}

 Some calculations.
	\begin{itemize}
\item $O|q\simeq \Omega \calS\calC$, by Observation~\ref{obs:homotopycartesian}.
\item 	An $n$-simplex of $O|\calS F$ is of the form
\begin{equation*}%\label{eq:OSF}
W=\begin{pmatrix} & & \doubleunderline{O=K_0\rtail \dots \rtail K_n}\\ &O \rtail & \quad  L_0\quad\rtail \dots \rtail L_n \end{pmatrix};
\end{equation*}	
the map $p$ acts by projecting the top row, the map $r$ projects the bottom row.  An $n$-simplex of $O|p$ is therefore a triple
$$V=\begin{pmatrix}
&  \doubleunderline{O=K_0\rtail \dots \rtail K_n}\\
O \rtail &  \doubleunderline{\quad L_0\quad\rtail \dots \rtail L_n}\\
O \rtail &  \quad M_0\quad\rtail \dots \rtail M_n
\end{pmatrix}$$
where the double lines indicate equality of the corresponding quotients. Notice the top and bottom rows are required to be filtrations in $\calS\Cplus$, while the middle row is a filtration in $\calS \calC$. The gluing of the middle and bottom row encodes all the info on the top row. Once we forget the top row, we essentially get $\Ghat$, (after swapping the rows) to get 
$$V^\ast=\begin{pmatrix}
O \rtail &  \doubleunderline{\quad M_0\quad\rtail \dots \rtail M_n}\\
O \rtail &  \quad L_0\quad\rtail \dots \rtail L_n
\end{pmatrix}.$$
The original map $O|p\to O|\calS F$ works by projecting the top two rows of $V$. Presented as $V^*$, this amounts to projecting the bottom row.
\end{itemize}
\end{comment}
In particular, consider the homotopy fiber sequence associated to $|O|p|\to |\OSF| \xrightarrow{p} |\calS \Cplus|$. Unwinding the proof of Claim~\ref{claim:Sherhtpyfib} yields the following key observation.

\begin{observation}\label{obs:htpyseq} The homotopy equivalence $O|p\simeq \Ghat$ identifies a projection map $v\colon \Ghat\to \OSF$, sending an $n$-simplex to its bottom row. In particular, the exact sequence 
	\begin{equation}%\label{eq:SherHseq1}
	\begin{tikzcd}
 \pi_1\Omega \gm{\calS\Cplus} \ar[r] &	\pi_1\gm{O|p} \ar[r] & \pi_1\gm{O|\calS F}\ar[r,"p_\ast"] & \pi_1\gm{\calS\Cplus}
	\end{tikzcd}
	\end{equation}
may be reformulated as
	\begin{equation}\label{eq:SherHseq}
	\begin{tikzcd}
\pi_1\Omega\gm{\calS\Cplus} \ar[r] &	\pi_1\gm{\Ghat}\ar[r,"v_\ast"] & \pi_1\gm{O|\calS F}\ar[r,"p_\ast"] & \pi_1\gm{\calS\Cplus}
	\end{tikzcd},
	\end{equation}
where $v_\ast$ is induced by $v$. For further details, see \cite[Cor. 1.8]{GG}. 
\end{observation}

\subsubsection*{Step 2: Generators of $\pi_1|O|\calS F|$} The argument is standard -- no surprises. The 1-simplices of the form
\begin{equation}\label{eq:treeOSF}
\ones{ & O\rtail N}{ O\rtail & \doubleoverline{O\rtail N}}
\end{equation}
yield a maximal tree for the 1-skeleton of $|\OSF|$, connecting the base-point of $\OSF$ to any of its vertices. 
 \begin{comment}
\footnote{Why? A typical vertex of $\OSF$ is of the form 
	$$\ones{ & O}{ O\rtail & \doubleoverline{N}}.$$
Since there is a canonical 1-simplex 
	$$\ones{ & O\rtail O}{ O\rtail & \doubleoverline{O\rtail N}}$$
	that connects $\ones{O}{O}$ to the vertex, and since the vertex was chosen arbirarily, the claim follows.}
	\end{comment}
Thus by \cite[Lemma IV.3.4]{WeibelKBook}, the total set of 1-simplices of $\OSF$ generate $\pi_1\gm{\OSF}$. We therefore represent the generators of $\pi_1|\OSF|$ as 
\begin{equation}
\ones{ & O\rtail A}{ O\rtail & \doubleoverline{O\rtail A}} \ones{ & O\rtail C}{ O\rtail & \doubleoverline{A\rtail B}} \ones{ & O\rtail B}{ O\rtail & \doubleoverline{O\rtail B}}^{-1}.
\end{equation}
For simplicity, generators are sometimes represented just by the middle term above, since the other two edges can be recovered from the maximal tree.

\begin{comment}The above codes the loop
\begin{equation}\label{eq:loopOSF}
\begin{tikzcd}[ampersand replacement=\&]
\ones{ & O}{ O\rtail & \doubleoverline{A}} \ar[rr,"{{\ones{ & O\rtail C}{ O\rtail \!\!\!\!\!\!& \doubleoverline{A\rtail B}}}}"]\&\& \ones{ & O}{ O\rtail & \doubleoverline{B}}\\
\& \ones{ & O}{ O\rtail & \doubleoverline{O}} \ar[ur,swap,"{{\ones{ & O\rtail B}{ O\rtail \!\!\!\!\!\!& \doubleoverline{O\rtail B}}}}"] \ar[ul,"{{\ones{ & O\rtail A}{ O\rtail \!\!\!\!\!\!& \doubleoverline{O\rtail A}}}}"]
\end{tikzcd}.
\end{equation}
\end{comment}


\subsubsection*{Step 3: Constructing a Sherman Loop} Review Homotopy Exact Sequence~\eqref{eq:SherHseq}. Given $v_*(x)\in\pi_1|\OSF|$ for any $x\in \pi|\Ghat|$, we can use its presentation from Step 2 to construct a Sherman Loop $G(\alpha,\beta,\theta)$.

\begin{itemize}
	\item {\em The two $\M$-morphisms.} Since $\OSF$ is an $H$-space,  $v_*(x)$ can be expressed as a difference of two 1-simplices, let us say
	\begin{equation}\label{eq:Sher1sim}
	\ones{ & O\rtail C}{ O\rtail & \doubleoverline{A\rtail B}} \quad\text{and}\quad \ones{ & O\rtail C'}{ O\rtail & \doubleoverline{A'\rtail B'}}.
	\end{equation}
	This yields the $\M$-morphisms $\alpha\colon A\rtail B$ and $\beta\colon A'\rtail B'$.
	\item The isomorphism $\theta$. Recall the map $p\colon \OSF\to\calS\Cplus$ acts by projection on the top row. Thus $p_\ast v_\ast(x)\in \pi_1|\calS\Cplus|$ corresponds to the difference between
	\begin{equation}\label{eq:SherSCloops}
	(O\rtail A)(O\rtail C)(O\rtail B')^{-1}\qquad \text{and} \qquad (O\rtail A')(O\rtail C')(O\rtail B')^{-1}. 
	\end{equation}
	By Presentation Theorem~\ref{thm:PresK0}, we know that $\pi_1|\SC^{\oplus}|=K_0(\Cplus)$, so let us rewrite the above as
	\begin{equation}\label{eq:SherDiff}
	[A]+[C]-[B] \qquad \text{and} \qquad [A']+[C']-[B']. %%% Notice these don't automatically give 0 because we don't know if they split.
	\end{equation}
	We don't know if, e.g. $[A]+[C]=[B]$ in $K_0(\Cplus)$ since $A\rtail B$ may not be split. Nonetheless, by exactness of Homotopy Sequence~\eqref{eq:SherHseq}, we can deduce
	 \begin{equation}
	 [A]+[C]-[B] - \left([A']+[C']-[B']\right) = 0,
	 \end{equation}
	 and so
	 \begin{equation}
	 [A]+[C]+[B'] = [A']+[C']+[B].
	 \end{equation}
Since $[M]=[N]$ in $K_0(\Cplus)$ iff $M$ and $N$ are stably isomorphic\footnote{The same argument for exact categories extends to the pCGW setting.},
 %\footnote{\emph{Details.} By \cite[Thm 4.3]{CGW}, $K_0(\Cplus)=F/R$ is the free abelian group $F$ generated by objects in $\Cplus$ modulo the relation $R$ that $[P]+[Q]=[Z]$ iff $P\oplus Q \cong Z$. Suppose $[M]=[N]$ in $K_0(\Cplus)$. On the level of the free group $F$, this implies $\overline{M}-\overline{N}=\sum^n_i \left( \overline{P_i} + \overline{Q_i} - \overline{P_i\oplus Q_i}\right)$, for some finite set of $P_i,Q_i\in\calC$. Rearranging terms gives $[M]+\sum^n_i[P_i\oplus Q_i]=[N] + \sum^n_i [P_i] + \sum^n_i [Q_i]$ in $F$. Since we are in a free group, these translate to an isomorphism $M\oplus (\bigoplus^{n}_{i=1} P_i\oplus Q_i)\cong N\oplus (\bigoplus^{n}_{i=1}P_i\oplus \bigoplus^{n}_{i=1}Q_i)$. §The rest follows from noting that $\bigoplus^{n}_{i=1}P_i\oplus Q_i\cong \bigoplus^{n}_{i=1}P_i\oplus \bigoplus^{n}_{i=1} Q_i$.} 
there exists some $Z\in\calC$ such that 
	 \begin{equation}
	 A\oplus C \oplus B' \oplus Z \cong A'\oplus C' \oplus B \oplus Z. %%% $[A\oplus C\oplus B']=[A'\oplus C'\oplus B]$ iff $[A]+[C]+[B']=[A']+[C']+[B]$ since $+$ is defined by direct sum. And so the equation follows by stable isomorphism.
	 \end{equation}	 
	Now notice that
	 \begin{equation}\label{eq:Z}
	 \ones{ & O\rtail Z}{ O\rtail & \doubleoverline{O\rtail Z}} \ones{ & O\rtail O}{ O\rtail & \doubleoverline{Z\rtail Z}} \ones{ & O\rtail Z}{ O\rtail & \doubleoverline{O\rtail Z}}^{-1}
	 \end{equation}
	 is null-homotopic, Hence, we can always modify the representation of $v_*(x)$ by adding Loop~\eqref{eq:Z} to Equation~\eqref{eq:Sher1sim} without changing the homotopy class. As such, without loss of generality, assume $Z=O$, giving the isomorphism 
	 \begin{equation}\label{eq:Shertheta}
	 \theta\colon A\oplus C\oplus B'\xrightarrow{\cong} A'\oplus C'\oplus B.
	 \end{equation}
\end{itemize}

%%% In other words: we may assume that we've already incorporated Z into the representation of v(x). In which case, by Equation above, we get a direct isomorphism and not a stable isomorphism (after incorporating Z).



\subsubsection*{Step 4: An Equivalence} The following claim tells us that any $x\in \pi_1|\Ghat|$ looks like a Sherman loop when viewed in $\pi_1|\OSF|$.

\begin{claim}\label{claim:SherLookLoop} Given any $x\in \pi_1|\Ghat|$, there exists a Sherman loop $G(\alpha,\beta,\theta)$ such that $v_*(x)$ and $v_*(G(\alpha,\beta,\theta))$ are homotopic.
\end{claim}
\begin{proof} Given $x\in \pi_1|\Ghat|$, construct a Sherman Loop $G(\alpha,\beta,\theta)$ as in Step 3. In particular, $v_*(G(\alpha,\beta,\theta))$ is well-defined since  $G(\alpha,\beta,\theta)\in \pi_1|\Ghat|$.
	
Consider the following diagram in $\pi_1|\OSF|$
\begin{equation}\label{eq:SherLoopHtpy}
\small{\begin{tikzcd}
\ones{ & O}{ O\rtail & \doubleoverline{A}} \ar[d,red,""{name=X1}] \ar[r,blue]& \ones{ & O}{ O\rtail & \doubleoverline{B\oplus B'}} \ar[r,blue]& \ones{ & O}{ O\rtail & \doubleoverline{B'\oplus B}}&  \ones{ & O}{ O\rtail & \doubleoverline{A'}} \ar[l,blue] \ar[dl,red,""{name=X7},{xshift=10pt},{yshift=-5pt}] \\
\ones{ & O}{ O\rtail & \doubleoverline{B}} \ar[ur,""{name=X2}]  \ar[from=X2, to=X1,phantom, "\small{(1)}"] & \ones{&O}{O&\doubleoverline{O}} \ar[ur,""{name=X5}]\ar[l,red,""{name=X3}]  \ar[from=X3, to=X2,phantom, "\small{(2)}",{xshift=20pt},{yshift=5pt}] \ar[r,red]   \ar[u,""{name=X4}]  \ar[from=X4, to=X5,phantom, "\small{(3)}"] & \ones{ & O}{ O\rtail & \doubleoverline{B'}} \ar[u,""{name=X6}]\ar[from=X5, to=X6,phantom, "\small{(4)}",{yshift=-5pt}] \ar[u,""{name=X4}]  \ar[from=X6, to=X7,phantom, "\small{(5)}",{yshift=2.5pt}]
\end{tikzcd}}
\end{equation}
The edges of the diagram are obvious, and record various paths between vertices
\begin{equation}\label{eq:Shervertices}
\ones{ & O}{ O\rtail & \doubleoverline{A}} \dashrightarrow \ones{ & O}{ O\rtail & \doubleoverline{A'}},
\end{equation}
e.g. by concatenating along the blue edges, by concatenating along the red edges, etc.  



A couple of key observations. First, notice that all triangles in Diagram~\eqref{eq:SherLoopHtpy} define boundaries of 2-simplices, listed below. 
\begin{align*}
\small{(1) \, \ones{ & O \rtail C \rtail  C\oplus B' }{ O\rtail & \doubleoverline{A\rtail B\rtail B\oplus B'}}, \,\, (2) \, \ones{ & O \rtail B \rtail  B\oplus B' }{ O\rtail & \doubleoverline{O\rtail B\rtail B\oplus B'}},\,\, (3) \, \ones{ & O \rtail B\oplus B' \rtail  B'\oplus B }{ O\rtail & \doubleoverline{O\rtail B\oplus B'\rtail B'\oplus B}} }
\end{align*}
\begin{align*}
\small{(4) \, \ones{ & O \rtail B' \rtail B' \oplus B }{ O\rtail & \doubleoverline{O\rtail B'\rtail B'\oplus B}}, \,\, (5) \, \ones{ & O \rtail C' \rtail  C'\oplus B }{ O\rtail & \doubleoverline{A'\rtail B'\rtail B'\oplus B}}.}
\end{align*}
Hence, the blue and red paths in Diagram~\eqref{eq:SherLoopHtpy} between the two vertices ~\eqref{eq:Shervertices} are homotopic. 

Second, recall from Observation~\ref{obs:htpyseq} that $v\colon \Ghat\to \OSF$ acts by projecting the bottom row. Thus, $v_*(G(\alpha,\beta,\theta))$ corresponds to the loop  %% See commented out notes for Step 1. $v$ acts by projecting the bottom row of these vertices; so the connecting 1-simplices are the obvious ones to make the edge work. In principle, we could have worked with $G\calC$, but $\Ghat$ makes it clearer which rows we're projecting.
\begin{equation}
\footnotesize{	\ones{ & O\rtail A}{ O\rtail & \doubleoverline{O\rtail A}} \ones{ & O\rtail C\oplus B'}{ O\rtail & \doubleoverline{A\rtail B\oplus B'}} \ones{ & O\rtail O}{ O\rtail & \doubleoverline{B\oplus B'\rtail B'\oplus B}} \ones{ & O\rtail C\oplus B'}{ O\rtail & \doubleoverline{A'\rtail B'\oplus B}}^{-1} \ones{ & O\rtail A'}{ O\rtail & \doubleoverline{O\rtail A'}}^{-1}} \nonumber 
\end{equation}
whereas $v_*(x)$, the difference between two generators, corresponds to the loop 
\begin{equation}
\footnotesize{	\ones{ & O\rtail A}{ O\rtail & \doubleoverline{O\rtail A}} \ones{ & O\rtail C}{ O\rtail & \doubleoverline{A\rtail B}} 
\ones{ & O\rtail B}{ O\rtail & \doubleoverline{O\rtail B}}^{-1} \ones{ & O\rtail B'}{ O\rtail & \doubleoverline{O\rtail B'}} \ones{ & O\rtail C'}{ O\rtail & \doubleoverline{A'\rtail B'}}^{-1} \ones{ & O\rtail A'}{ O\rtail & \doubleoverline{O\rtail A'}}^{-1}}. \nonumber 
\end{equation}
In particular, the loop $v_*(G(\alpha,\beta,\theta))$ corresponds to composing along the blue edges in Diagram~\eqref{eq:SherLoopHtpy} whereas $v_*(x)$ corresponds to composing along the red edges. Hence, they are homotopy equivalent by our previous observation. Conclude that $v_*(G(\alpha,\beta,\theta))$ and $v_*(x)$ have the same homotopy class.
\end{proof}


\subsubsection*{Step 5: Finish} Let  $x$ be an element of $K_1(\calC)$. By Theorem~\ref{thm:Gconstruction} and Claim~\ref{claim:Sherhtpyfib}, we know
$$\Omega |\calS\calC|\simeq |G\calC|\simeq |\Ghat|,$$ 
and so regard $x\in \pi_1|\Ghat|$.\footnote{If $|G\calC|\simeq |\Ghat|$, why not simply regard $x\in \pi_1|G\calC|$ and use $\pi_1|G\calC|$ in Homotopy Sequence~\eqref{eq:SherHseq}? Answer: the map $v\colon \Ghat\to \OSF$ admits a useful explicit description (= projection of the bottom row), which streamlines the proof of Claim~\ref{claim:SherLookLoop}.} In particular, $x$ is an element in Homotopy Sequence~\eqref{eq:SherHseq}. By Claim~\ref{claim:SherLookLoop}, there exists a Sherman loop $G(\alpha,\beta,\theta)$ such that $v_*(G(\alpha,\beta,\theta))=v_*(x)$ in $\pi_1|\OSF|$. In other words, the difference $x-G(\alpha,\beta,\theta)$ vanishes in $\pi_1|\OSF|$, and thus lies in the image of 
$$ K_1(\Cplus)=\pi_1\Omega|\calS\Cplus|\longrightarrow \pi_1|\Ghat|=K_1(\calC).$$ To finish, we quote a couple of technical facts about Sherman Loops whose proof we defer to Appendix~\ref{app:Sherman Loops}. By Lemmas~\ref{lem:SherLoopSplit} and \ref{lem:SherSplitSher}, $K_1(\Cplus)$ is generated by Sherman Loops, and thus so is its image in $K_1(\calC)$. By Lemma~\ref{lem:SherLoopAdd}, the sum of two Sherman Loops is still a Sherman Loop. As such, since
$$x=x-G(\alpha,\beta,\theta)+G(\alpha,\beta,\theta),$$
conclude that $x$ is homotopic to a Sherman Loop in $K_1(\calC)$.
\end{proof}

%%% There is a natural map from K_1(\Cplus)\to K_1(\C), where $K_1(\C)$ has potentially more relations than that of $K_1(\Cplus)$. So if we just map the elements to its representatives in this group, Sherman Loops will still be Sherman loops, and all $y\in$ image will be f(x)=y for some x in $K_1(\Cplus)$, and the rest follows from the fact that x is generated by some combination of sherman loops and this combination is respected by the group homomorphism.  

\begin{discussion} Our proof streamlines Sherman's approach in \cite{ShermanExact}. For instance, Sherman's original argument for Claim~\ref{claim:Sherhtpyfib} proceeds by defining a pair of exact functors 
$$\Delta\colon \calC\to \calC\times \calC$$
$$\Delta'\colon \Cplus\to\Cplus\times \calC,$$
where $\Delta$ is the diagonal, and $\Delta'$ is the diagonal composed with the obvious inclusion map. Sherman then asserts as obvious that the cofiber of $\calS\Delta$ is homotopy equivalent to $|\calS\calC|$, presumably by analogy with short exact sequences of abelian groups. However, there are subtleties. It is not generally true that $\cofib(\Delta)\simeq X$ for a diagonal map of some space $X$ --  e.g. consider $\Delta\colon S^1\rightarrow S^1\times S^1$. %, which embeds a circle $S^1$ into a torus.
A potential remedy would be to first prove $\cofib(\calS\Delta)$ and $\calS\calC$ are equivalent as $\mathbb{E}_\infty$-spaces before recovering the desired result, but this is rather involved. Alternatively, one can proceed directly by unpacking definitions, as in our proof of Claim~\ref{claim:Sherhtpyfib}.

\begin{comment} \begin{claim} The homotopy cofiber of the diagonal map $\Delta\colon S^1\to S^1\times S^1$, defined by $x\mapsto (x,x)$ is not homotopy equivalent to $S^1$.
\end{claim}
\begin{proof} $\Delta$ embeds a circle $S^1$ into a diagonal line on the torus. Observe that $\cofib(\Delta)=M_\Delta/(S^1\times \{0\})$, where $M_\Delta:=(S^1\times I)\cup_{\Delta}(S^1\times S^1)$ is the mapping cylinder of $\Delta$. Informally, we can think of $\cofib(\Delta)$ as the torus $S^1\times S^1$ with a cone attached along the diagonal. 

Notice $\cofib(\Delta)=A\cup_\Delta B$ where $A:=S^1\times S^1$, $B:=$ Cone on $S^1$, and $A\cap B=S^1$. We compute that:
\begin{itemize}
\item $H_\ast(A)=H_\ast(T^2) = \Z$ in degrees 0 and 2, and $\Z^2$ in degree 1.
\item $H_\ast(B)=H_\ast(\text{cone on}\, S^1)=\Z$ in degree 0, 0 otherwise due to contractibility.
\item $H_\ast(A\cap B)=H_\ast (S^1)=\Z$ in degree 0, and $\Z$ in degree 1. 
\end{itemize}

So applying Mayer-Vietoris, in degree 2: 
$$0\to H_2(A)\oplus H_2(B)=\Z\oplus 0 \to H_2(\cofib(\Delta))\to H_1(A\cap B)=\Z\xrightarrow{\varphi_1} H_1(A)\oplus H_1(B)=\Z^2\oplus 0.$$

The map $H_1(S^1)\to H_1(T^2)$ is induced by the diagonal, and thus sends the generator to the diagonal class $(1,1)$. So the image of $\varphi_1$ is the subgroup generated by $(1,1)$, which is a rank-1 subgroup of $\Z^2$. Thus $\ker(\varphi_1)=0$, and $\coker(\varphi_1)=\Z$. 

This assembles into the long exact sequence
$$0\to \Z\xrightarrow{\gamma} H_2(\cofib(\Delta))\xrightarrow{\delta} \Z\xrightarrow{\varphi_1} \Z^2.$$
$\Z\to H_2(\cofib(\Delta))$ is injective, by exactness, and so the image is isomorphic to $\Z$. Further, by exactness, $\im(\delta)=\ker(\varphi_1)=0$.
And so this assembles into the short exact sequence
$$0\to \Z\xrightarrow{\gamma} H_2(\cofib(\Delta))\to \ker(\varphi_1)=0\to 0.$$
Hence, we get $\ker(\varphi_1)\cong H_2/\im(\gamma)$, which implies $H_2(\cofib(\Delta))=\Z$, not zero. So the cofiber has non-trivial second homology, unlike $S^1$, and thus cannot be homotopy equivalent to $S^1$.
\end{proof}
\end{comment}

\end{discussion}





\subsection{Double Exact Squares}\label{sec:genK1}
\begin{comment} Remark: coskeleton and truncation are adjoint to each other; it suffices to reduce to the study of 1-simplices.
\end{comment}
Given an exact category, Nenashev \cite{Nen0,NenGen} shows that its $K_1$ is generated by so-called {\em double short exact sequences} -- sharpening Sherman's original result. We extend his analysis to the pCGW setting. 
%our context. As usual, $\calC$ here will denote a pCGW category.
\begin{definition}[Double Exact Squares]\label{def:DES} A \emph{double exact square} is a pair of exact squares with identical nodes, but possibly different morphisms:
		\[l:=\left(\, \dsquaref{O}{C}{A}{B}{ }{ }{f_1}{g_1} \quad,\quad  \dsquaref{O}{C}{A}{B}{ }{ }{f_2}{g_2} \,\right).\]	
Since this defines an edge $(A,A)\to (B,B)$, any double exact square defines a loop 
\begin{equation}
\begin{tikzcd}
(A,A) \ar[rr,"l"] && (B,B)\\
& (O,O) \ar[ul,"e(A)"] \ar[ur,swap,"e(B)"]
\end{tikzcd}
\end{equation}
where $e(A)$ and $e(B)$ are the obvious edges from the base-point, e.g.	\[e(A):=\left(\, \dsquaref{O}{A}{O}{A}{ }{ }{}{1_A} \quad,\quad  \dsquaref{O}{A}{O}{A}{ }{ }{}{1_A} \,\right).\]	
We call this the \emph{canonical loop of $l$}, and denote it as $\mu(l)$. We denote $\lrangles{l}$ to be its homotopy class in $K_1(\calC)$.
\end{definition}

As the following example illustrates, double exact squares generalise automorphisms in $\calC$. 

\begin{example}[Automorphisms]\label{ex:Aut} If $(A,\alpha)\in \Aut(\calC)$ is an automorphism, we write
	$$l(\alpha)=\left(\, \dsquaref{O}{O}{A}{A}{ }{ }{1_A}{} \quad,\quad  \dsquaref{O}{O}{A}{A}{ }{ }{\alpha}{}  \,\right).$$
%In particular, given any object $A\in\calC$, denote the standard edge from $(O,O)$ to $(A,A)$ in $G\calC$ as $l(1_A)$. 
\end{example}

To prove that $K_1(\calC)$ is generated by double exact squares, it suffices to show that any Sherman Loop is homotopic to the canonical loop of a double exact square; the rest follows from Theorem~\ref{thm:Sherman}. We first 
set some conventions (which the reader may safely skim on first reading), before stating our main theorem.

\begin{convention}[Coordinate-wise Definition of 1-simplices]\label{conv:coord} Consider two exact squares
	\[h_0:= \left(\dsquaref{O}{M''}{M'}{M}{}{}{m_0}{m_1}\right) \qquad h_1:= \left(\dsquaref{O}{N''}{N'}{N}{}{}{n_0}{n_1}\right).\]	
	\begin{itemize}
		\item If $M''=N''$, we usually denote the corresponding 1-simplex as $(h_0,h_1)\colon (M',N')\to (M,N)$. 
		\item If $M''=O=N''$, we denote the corresponding 1-simplex as $(m_0,n_0)\colon (M',N')\to (M,N)$.
		\item If we wish to take their direct sum (see Lemma~\ref{lem:DirectSum}), this will be denoted
		\[h_0\oplus h_1:= \left(\dsquaref{O}{M''\oplus N''}{M'\oplus N'}{M\oplus N}{}{}{m_0\oplus n_0}{m_1\oplus n_1}\right).\]	
		\item We can also ``add'' an object $C$ to $h_0$ in the obvious way (see Lemma~\ref{lem:ADTsquares}) as follows:
		\[h_0\oplus C:=  \left(\dsquaref{O}{M''\oplus C}{M'}{M\oplus C}{}{}{m_0\oplus C}{m_1\oplus 1}\right).\]	
	\end{itemize}
\end{convention}


\begin{theorem}\label{thm:NenashevDSES} Given any $x\in K_1(\calC)$, there exists a double exact square $l$ such that $x=\mu(l)$.
\end{theorem}
\begin{proof} By Theorem~\ref{thm:Sherman}, we may assume $x$ is a Sherman Loop $G(\alpha,\beta,\theta)$ arising from a pair of exact squares
\begin{equation}
\dsquaref{O}{C}{A}{B}{ }{ }{\alpha}{\delta} \quad,\quad  \dsquaref{O}{C'}{A'}{B'}{ }{ }{\alpha'}{\delta'}
\end{equation}
and an isomorphism $\theta\colon A \oplus C \oplus B' \xrtail{\cong} A'\oplus C'\oplus B$. To turn this into a double exact square, first construct the following distinguished squares
\begin{equation}\label{eq:s0s1}
s_0:=\left(\dsquaref{O}{C\oplus C'}{A\oplus A'}{A\oplus C\oplus B'}{ }{ }{f_0}{g_0}\right) \quad,\quad  s_1:=\left( \dsquaref{O}{C\oplus C'}{A\oplus A'}{A'\oplus C' \oplus B}{ }{ }{f_1}{g_1}\right)
\end{equation}
\begin{equation}\label{eq:coordinateMAPS}
f_0=\begin{pmatrix}
1 & 0\\
0 & 0\\
0 & \alpha'
\end{pmatrix},\,\, f_1=\begin{pmatrix}
0& 1\\
0 & 0\\
\alpha & 0
\end{pmatrix},\,\, g_0=\begin{pmatrix}
0 & 0\\
1 & 0\\
0 & \delta'
\end{pmatrix},\,\, g_1=\begin{pmatrix}
0 & 0\\
0 & 1\\
\delta & 0
\end{pmatrix}.
\end{equation}
The matrix notation indicates, e.g. how the morphism $f_1\colon A\oplus A'\rtail A'\oplus C'\oplus B$ is constructed from $\alpha\colon A\rtail B$ and $1\colon A'\rtail A'$. To see why $s_0$ and $s_1$ are distinguished, apply Lemmas~\ref{lem:ADTsquares} and ~\ref{lem:DirectSum}.\footnote{Note: for $s_1$, one must also permute the direct summands via the obvious permutation isomorphisms.} % Notice for $s_1$, we "twist" the morphisms by composing the square with permutation isomorphisms.

We then apply the isomorphism $\theta$ to define the double exact square
\begin{equation}
l(x):=\left(\dsquaref{O}{C\oplus C'}{A\oplus A'}{A'\oplus C'\oplus B}{ }{}{ \theta\circ f_0}{\varphi(\theta)\circ g_0} \quad,\quad  \dsquaref{O}{C\oplus C'}{A\oplus A'}{A'\oplus C' \oplus B}{ }{ }{f_1}{g_1}\right).
\end{equation}
 %% We will also have to apply permutation isomorphisms, but this is OK since these define distinguished squares, and so we can modify things by vertical composition.

It thus remains to show that $\mu(l(x))$ is homotopic to $G(\alpha,\beta,\theta)$ in $K_1(\calC)$. In fact, since $K_1(\calC)$ is abelian (see Observation~\ref{obs:K1DirectSum}), it suffices to show that they are freely homotopic. This is accomplished by the following series of lemmas. %% free-homotopic, they don't have to have the same base-point.

\begin{convention} To ease notation, denote $P:=A\oplus C\oplus B'$ and $Q:=A'\oplus C'\oplus B$. 
\end{convention}

\begin{lemma}\label{lem:Nen1} $\mu(l(x))$ is freely homotopic to the loop
	\begin{equation}\label{eq:NenashevL1}
\begin{tikzcd} (P,Q) \ar[rr,"(\theta\text{,}1)"] && (Q,Q)\\
& (A\oplus A',A\oplus A') \ar[ul,"(s_0\text{,}s_1)"] \ar[ur,swap,"(s_1\text{,}s_1)"]
\end{tikzcd}
	\end{equation}
\end{lemma}
\begin{proof}[Proof of Lemma] Consider the diagram
\begin{equation}
\begin{tikzcd} & (P,Q) \ar[dddr,violet,"(\theta\text{,}1)", ""{name=PD}]\\
 \\
 \\
(A\oplus A',A\oplus A') \ar[uuur,violet,"(s_0\text{,}s_1)",""{name=PU}] \ar[rr,teal,bend left=10,"l(x)"] \ar[rr,violet,bend right =10, "(s_1\text{,}s_1)"] \ar[from=PU, to=PD,phantom,"(1)"] && (Q,Q)\\
\\
& (O,O) \ar[uul,xshift=-1em,teal,"e(A\oplus A')", ""{name=UL}] \ar[uur,swap,teal,"e(Q)", ""{name=UR}] \ar[from=UL, to=UR, phantom, "(2)"]
\end{tikzcd}	
\end{equation}
The statement follows from observing that Triangles (1) and (2) form 2-simplices. Triangle (2) is obvious. Triangle (1) is given by
	
\begin{tikzcd}
A\oplus A'\ar[r, >->,"f_0"] & P \ar[dr,phantom,"\square"] \ar[r, >->,"\theta"] & Q\\
&	C\oplus C' \ar[r, >->,"1"] \ar[u, {Circle[open]}->,"g_0"]& C\oplus C' \ar[u, {Circle[open]}->,swap,"\varphi(\theta)\circ g_0"] \\
&& O \ar[u, {Circle[open]}->] 
\end{tikzcd}\quad \begin{tikzcd}
A\oplus A'\ar[r, >->,"f_1"] & Q \ar[r, >->,"1"]  \ar[dr,phantom,"\square"]  & Q\\
&	C\oplus C' \ar[r, >->,"1"]\ar[u, {Circle[open]}->,"g_1"]& C\oplus C' \ar[u, {Circle[open]}->,"g_1	"] \\
&& O \ar[u, {Circle[open]}->] 
\end{tikzcd}
%(To see why the indicated square on the LHS is distinguished, recall that if $\E^{\opp}$, $\varphi$ )	
	
\end{proof}


\begin{lemma}\label{lem:Nen2} The loop $G(\alpha,\beta,\theta)$ is freely homotopic to the loop
	\begin{equation}\label{eq:NenashevL2}
\begin{tikzcd} (P,B\oplus B') \ar[rr,"(\theta\text{,}1)"] && (Q,B\oplus B')\\
& (A\oplus A',A\oplus A') \ar[ul,"(s_0\text{,}s)"] \ar[ur,swap,"(s_1\text{,}s)"]
\end{tikzcd}
	\end{equation}
where $s$ is the distinguished square
\begin{equation}\label{eq:s}
s:=\left(\dsquaref{O}{C\oplus C'}{A\oplus A'}{B\oplus B'}{}{}{\alpha\oplus \alpha'}{\delta\oplus \delta'}\right)\qquad .
\end{equation}
\end{lemma}
\begin{proof}[Proof of Lemma] Consider the diagram
	\begin{equation}\label{eq:NenSherLoop}
	\begin{tikzcd}
(P,B\oplus B') \ar[rr,red,"(\theta\text{,}1)"]&& (Q,B\oplus B')\\
\\
(A,A) \ar[r] \ar[uu,teal,"(a_0\text{,} b_0)"]& (A\oplus A',A\oplus A') \ar[uur,red,swap,"(s_1\text{,}s)",""{name=R}] \ar[uul,red,"(s_0\text{,}s)",""{name=L}] \ar[from=L, to=R, phantom, "(\star)"]& (A',A') \ar[l] \ar[uu,swap,teal,"(a_1\text{,}b_1)"]\\
\\
& (O,O) \ar[uul,teal,"e(A)"] \ar[uur,teal,swap,"e(A')"] \ar[uu,"e(A\oplus A')"]
	\end{tikzcd}
	\end{equation}
where the 1-simplices $(a_0,b_0)$ and $(a_1,b_1)$ are defined by the obvious data.
\begin{comment}
$$(a_0,b_0):=\left( \begin{tikzcd}  O \ar[rr,>->] \ar[d,{Circle[open]->}] \ar[drr,phantom,"\square"]&& C\oplus B'\ar[d,{Circle[open]->}]\\
A \ar[rr,>->,swap,"1\oplus C\oplus B'"]&&  A\oplus C\oplus B' \end{tikzcd}  \quad,\quad  \begin{tikzcd}  O \ar[rr,>->] \ar[d,{Circle[open]->}] \ar[drr,phantom,"\square"]&& C\oplus B'\ar[d,{Circle[open]->}]\\
A \ar[rr,>->,swap,"\alpha\oplus B'"]&&  B\oplus B' \end{tikzcd} \right)$$
$$(a_1,b_1):=\left( \begin{tikzcd}  O \ar[rr,>->] \ar[d,{Circle[open]->}] \ar[drr,phantom,"\square"]&& C'\oplus B\ar[d,{Circle[open]->}]\\
A' \ar[rr,>->,swap,"1\oplus C'\oplus B"]&&  A'\oplus C'\oplus B \end{tikzcd}  \quad,\quad  \begin{tikzcd}  O \ar[rr,>->] \ar[d,{Circle[open]->}] \ar[drr,phantom,"\square"]&& C'\oplus B\ar[d,{Circle[open]->}]\\
A' \ar[rr,>->,swap,"B\oplus \alpha'"]&&  B\oplus B' \end{tikzcd} \right)$$

Notice that we've encoded the twist $\tau\colon B\oplus B'\to B'\oplus B$ of the Sherman Loop into the quotient of $(a_1,b_1)$, which has quotient $C'\oplus B$, which is what the quotient of the original $A'\rtail B'\oplus B$ would look like.
\end{comment}
Notice: the outer loop of Diagram~\eqref{eq:NenSherLoop} is equivalent\footnote{A small difference: the top edge here is $(\theta,1)\colon (P,B\oplus B')\to (Q,B\oplus B')$, as opposed to $(\theta,\tau)\colon (P,B\oplus B')\to (Q,B'\oplus B)$, but the twist is already encoded in $(a_1,b_1)\colon (A',A')\to (Q,B\oplus B')$.}  to the Sherman Loop $G(\alpha,\beta,\theta)$ while the triangle $(\star)$ is Loop~\eqref{eq:NenashevL2}. 

It remains to check that all other triangles in Diagram~\eqref{eq:NenSherLoop} bound 2-simplices. The bottom two triangles are immediate. The top left triangle bounds
$$\begin{tikzcd}
A\ar[r, >->] & A\oplus A' \ar[dr,phantom,"\square"] \ar[r, >->,"f_0"] & P\\
&	A' \ar[r, >->,"C\oplus \alpha' "] \ar[u, {Circle[open]}->]& C\oplus B' \ar[u, {Circle[open]}->,swap,"A\oplus 1"] \\
&O \ar[r,>->] \ar[u,{Circle[open]->}] \ar[ur,phantom,"\square"]& C\oplus C' \ar[u, {Circle[open]}->,swap,"1\oplus \delta'"] 
\end{tikzcd}\quad \begin{tikzcd}
A\ar[r, >->] & A\oplus A' \ar[dr,phantom,"\square"] \ar[r, >->,"\alpha\oplus \alpha'"] & B\oplus B'\\
&	A' \ar[r, >->,"C\oplus \alpha'"] \ar[u, {Circle[open]}->, ]& C\oplus B' \ar[u, {Circle[open]}->,swap," \delta\oplus 1 "] \\
&O \ar[r,>->] \ar[u,{Circle[open]->}] \ar[ur,phantom,"\square"]& C\oplus C' \ar[u, {Circle[open]}->,swap,"1\oplus \delta'"] 
\end{tikzcd}$$
That this defines a 2-simplex follows from Lemmas~\ref{lem:ADTsquares} and \ref{lem:DirectSum}. The top right triangle is analogous. %Conclude $G(\alpha,\beta,\theta)$ and Loop~\eqref{eq:NenashevL2} are indeed freely homotopic.
\begin{comment}  Details for the top left triangle. The LHS is the direct sum of the following two flags

$$\begin{tikzcd}
A\ar[r, >->] & A \ar[dr,phantom,"\square"] \ar[r, >->] & A\oplus C\\
&	O\ar[r, >->] \ar[u, {Circle[open]}->]& C  \ar[u, {Circle[open]}->] \\
&O \ar[r,>->] \ar[u,{Circle[open]->}] \ar[ur,phantom,"\square"]& C  \ar[u, {Circle[open]}->] 
\end{tikzcd}\quad \begin{tikzcd}
O \ar[r, >->] & A' \ar[dr,phantom,"\square"] \ar[r, >->,"\alpha'"] & B'\\
&	A' \ar[r, >->,"\alpha'"] \ar[u, {Circle[open]}->, ]& B' \ar[u, {Circle[open]}->,swap," 1 "] \\
&O \ar[r,>->] \ar[u,{Circle[open]->}] \ar[ur,phantom,"\square"]&  C' \ar[u, {Circle[open]}->,swap,"\delta'"] 
\end{tikzcd}$$


For the RHS triangle, it is the direct sum of


$$\begin{tikzcd}
A\ar[r, >->] & A \ar[dr,phantom,"\square"] \ar[r, >->,"\alpha"] &B \\
&	O\ar[r, >->] \ar[u, {Circle[open]}->]& C  \ar[u, {Circle[open]}->,swap,"\delta"] \\
&O \ar[r,>->] \ar[u,{Circle[open]->}] \ar[ur,phantom,"\square"]& C  \ar[u, {Circle[open]}->] 
\end{tikzcd}\quad \begin{tikzcd}
O \ar[r, >->] & A' \ar[dr,phantom,"\square"] \ar[r, >->,"\alpha'"] & B'\\
&	A' \ar[r, >->,"\alpha'"] \ar[u, {Circle[open]}->, ]& B' \ar[u, {Circle[open]}->,swap," 1 "] \\
&O \ar[r,>->] \ar[u,{Circle[open]->}] \ar[ur,phantom,"\square"]&  C' \ar[u, {Circle[open]}->,swap,"\delta'"] 
\end{tikzcd}$$


The same argument works for the top right triangle; notice for the RHS triangle, this would come from adding 
$$\begin{tikzcd}
O \ar[r, >->] & A\ar[dr,phantom,"\square"] \ar[r, >->,"\alpha"] & B\\
&	A\ar[r, >->,"\alpha"] \ar[u, {Circle[open]}->, ]& B\ar[u, {Circle[open]}->,swap," 1 "] \\
&O \ar[r,>->] \ar[u,{Circle[open]->}] \ar[ur,phantom,"\square"]&  C \ar[u, {Circle[open]}->,swap,"\delta"]\end{tikzcd}  \qquad \begin{tikzcd}
A'\ar[r, >->] & A' \ar[dr,phantom,"\square"] \ar[r, >->,"\alpha'"] &B' \\
&	O\ar[r, >->] \ar[u, {Circle[open]}->]& C'  \ar[u, {Circle[open]}->,swap,"\delta'"] \\
&O \ar[r,>->] \ar[u,{Circle[open]->}] \ar[ur,phantom,"\square"]& C'  \ar[u, {Circle[open]}->] 
\end{tikzcd}\quad 
$$
This would yield 

$$ \begin{tikzcd}
A'\ar[r, >->] & A\oplus A' \ar[dr,phantom,"\square"] \ar[r, >->,"\alpha\oplus \alpha'"] & B\oplus B'\\
&	A \ar[r, >->,"\alpha\oplus C'"] \ar[u, {Circle[open]}->, ]& B\oplus C' \ar[u, {Circle[open]}->,swap," 1\oplus \delta' "] \\
&O \ar[r,>->] \ar[u,{Circle[open]->}] \ar[ur,phantom,"\square"]& C\oplus C' \ar[u, {Circle[open]}->,swap,"\delta\oplus 1"] 
\end{tikzcd}$$

The $\E$-morphism is $(1\oplus \delta')\circ (\delta\oplus 1)$ instead of $(\delta\oplus 1)\circ (1\oplus \delta')$, but we can take them to be the same since $\alpha\oplus\alpha'$ has isomorphic quotients. 


OTOH, the LHS square is obtained by adding
$$\begin{tikzcd}
A'\ar[r, >->] &  A' \ar[dr,phantom,"\square"] \ar[r, >->,"1\oplus C' "] & A'\oplus C' \\
&	O\ar[r, >->] \ar[u, {Circle[open]}->]& C'  \ar[u, {Circle[open]}-> ] \\
&O \ar[r,>->] \ar[u,{Circle[open]->}] \ar[ur,phantom,"\square"]& C'  \ar[u, {Circle[open]}->] 
\end{tikzcd}\quad  \begin{tikzcd}
O \ar[r, >->] & A\ar[dr,phantom,"\square"] \ar[r, >->,"\alpha"] & B\\
&	A\ar[r, >->,"\alpha"] \ar[u, {Circle[open]}->, ]& B\ar[u, {Circle[open]}->,swap," 1 "] \\
&O \ar[r,>->] \ar[u,{Circle[open]->}] \ar[ur,phantom,"\square"]&  C \ar[u, {Circle[open]}->,swap,"\delta"]\end{tikzcd}  \qquad 
$$
and so adding them yields

$$\begin{tikzcd}
A' \ar[r, >->] & A\oplus A'\ar[dr,phantom,"\square"] \ar[r, >->] & B\oplus A'\oplus C' \\
&	A\ar[r, >->,"\alpha\oplus C'"] \ar[u, {Circle[open]}->, ]& B\oplus C' \ar[u, {Circle[open]}->] \\
&O \ar[r,>->] \ar[u,{Circle[open]->}] \ar[ur,phantom,"\square"]&  C\oplus C' \ar[u, {Circle[open]}->,swap,"\delta\oplus 1"]\end{tikzcd}  \qquad 
$$
And so we've defined a 2-simplex. The only technicality is that $B\oplus A'\oplus C'$ is not the same as $A'\oplus C'\oplus B$. However, this is easily fixed: we know there exists a permutation isomorphism $\zeta\colon B\oplus A'\oplus C'\to A'\oplus C'\oplus B$, so compose the upstairs with 
$$\dsquaref{B\oplus C'}{B\oplus C'}{B\oplus A'\oplus C'}{A'\oplus C'\oplus B}{1}{g}{\zeta}{\varphi(\zeta)\circ g}$$
So the quotient square remains undisturbed, but we've just modified the the upper RHS square to match what we want.
\end{comment}
\end{proof}

\begin{lemma}\label{lem:Nen3} Denote $V:=\left(B\oplus B'\right)\star_{(A\oplus A')} Q$ and $W:=C\oplus C'\oplus C\oplus C'$. \underline{Then}, there exists distinguished squares of the form
	\begin{equation}
	t:=\left(\dsquaref{C\oplus C'}{W}{Q}{V}{}{g_1}{h_t}{j}\right) \qquad t':=\left(\dsquaref{O}{C\oplus C'}{Q}{V}{}{}{h_t}{k_t}\right)
	\end{equation}
		\begin{equation}\label{eq:u-u'}
	u:=\left(\dsquaref{C'\oplus C'}{W}{B\oplus B'}{V}{}{\delta\oplus\delta'}{h_u}{j}\right) \qquad u':=\left(\dsquaref{O}{C\oplus C'}{B\oplus B'}{V}{}{}{h_u}{k_u}\right).
	\end{equation}
\end{lemma}
\begin{proof} Applying Axiom (DS), Definition~\ref{def:pCGW} to the diagram 
	\[\begin{tikzcd}
	C\oplus C' \ar[d, {Circle[open]}->,"\delta\oplus \delta'"] \ar[dr, phantom, "\square"]& O \ar[dr, phantom, "\square"]\ar[d, {Circle[open]}->]  \ar[l, >->] \ar[r, >->] & C\oplus C' \ar[d, {Circle[open]}->,"g_1"] \\
	B\oplus B' & A\oplus A' \ar[l, >->,swap,"\alpha\oplus \alpha'"] \ar[r, >->,"f_1"] & Q 
	\end{tikzcd},\]
conclude that there exists a distinguished square of the form $t$. %By Axiom (PQ), we know that
%$$\frac{C\oplus C'\oplus C\oplus C'}{C\oplus C'}\cong C\oplus C'.$$  
Since $t$ is distinguished, deduce that $\frac{V}{Q}\cong C\oplus C'$ (Remark~\ref{rem:INV}), and thus there exists a distinguished square of the form $t'$. The same argument shows there exists distinguished squares of the form $u$ and $u'$.
\end{proof}

\begin{lemma}\label{lem:Nen4} Loops~\eqref{eq:NenashevL1} and \eqref{eq:NenashevL2} are homotopic.
\end{lemma}
\begin{proof}[Proof of Lemma] Let $V:=\left(B\oplus B'\right)\star_{(A\oplus A')} Q$ and $t$ as in Lemma~\ref{lem:Nen3}, and $s_0,s_1$ as in Equation~\eqref{eq:s0s1}. Define $\s$ as the horizontal composition of $s_1$ and $t$
	\[\s:=\left(\begin{tikzcd}
O \ar[dr, phantom, "\square"]\ar[d, {Circle[open]}->]   \ar[r, >->] & C\oplus C' \ar[d, {Circle[open]}->,"g_1"] \ar[r,>->] \ar[dr, phantom, "\square"] & W \ar[d, {Circle[open]}->,"j"] \\
 A\oplus A' \ar[r, >->,"f_1"] & Q  \ar[r,>->,"h_t"] & V
\end{tikzcd}\right).\]	
Now let $s,u$ as in Equations~\eqref{eq:s} and~\eqref{eq:u-u'}. By construction of $V$, the $\M$-morphism $h_t\circ f_1$ equals 
$$h_u\circ (\alpha\oplus\alpha')\colon A\oplus A' \rtail B\oplus B'\rtail V.$$
Hence, $\tilde{s}$ also equals the horizontal composition of $s$ and $u$.  This will allow us to construct a loop $L$ that Loops~\eqref{eq:NenashevL1} and \eqref{eq:NenashevL2} are both homotopic to. 

Consider the diagram
\begin{equation}\label{eq:NenBigL1}
\!\!\!\!\!\!\!\!\!\!\!\!\!\!\!\!\!\!\!\!\!\begin{tikzcd}
(P\oplus C\oplus C',V) \ar[rr,"(\theta\oplus 1\text{,}1)" {xshift=-30pt},blue]&& (Q\oplus C\oplus C',V)\\
\\
(P,Q) \ar[rr,swap,"(\theta\text{,}1)",""{name=X},violet] \ar[uu,"(1\oplus C\oplus C'\text{,} t')",""{name=Z}] \ar[uurr,"(\theta\oplus C\oplus C'\text{,}\,t')" {yshift=10pt,xshift=50pt},""{name=Y}]  \ar[from=X, to=Y,phantom, "(4)"] \ar[from=Z, to=Y,phantom, "(3)"] && (Q,Q) \ar[uu,swap,"(1\oplus C\oplus  C'\text{,}t')"]\\
\\
& (A\oplus A',A\oplus A') \ar[uul,swap,"(s_0\text{,}s_1)",""{name=X2},violet] \ar[uur,"(s_1\text{,}s_1)",""{name=Y2},violet] 
\ar[uuuul,bend left=120, xshift=-0.5em,""{name=X1},"(s_0\oplus C\oplus C'\text{,}\,\,\s)",blue]  \ar[uuuur,bend right=120, xshift=0.5em,""{name=Y1},swap,"(s_1\oplus C\oplus C'\text{,}\s)",blue] 
\ar[from=X1, to=X2,phantom, "(1)"] 
\ar[from=Y1, to=Y2,phantom, "(2)"]
\end{tikzcd}\qquad
\end{equation}
The purple edges are Loop~\eqref{eq:NenashevL1}, the blue edges form an outer loop, which we denote $L$. To show that the two loops are homotopic, it suffices to check that Triangles (1) - (4) are boundaries of 2-simplices -- this is worked out explicitly in Claim~\ref{claim:NenL1}. Analogously, one can construct the diagram
\begin{equation}\label{eq:NenBigL2}
\!\!\!\!\!\!\!\!\!\!\!\!\!\begin{tikzcd}
(P\oplus C\oplus  C',V) \ar[rr,"(\theta\oplus 1\text{,}1)" {xshift=-30pt},blue]&& (Q\oplus C\oplus C',V)\\
\\
(P,B\oplus B') \ar[rr,swap,"(\theta\text{,}1)",""{name=X},red] \ar[uu,"(1\oplus C\oplus C'\text{,} u')",""{name=Z}] \ar[uurr,"(\theta\oplus C\oplus  C'\text{,}\,u')" {yshift=10pt,xshift=50pt},""{name=Y}]  \ar[from=X, to=Y,phantom, "(4')"] \ar[from=Z, to=Y,phantom, "(3')"] && (Q,B\oplus B') \ar[uu,swap,"(1\oplus C\oplus C'\text{,} u')"]\\
\\
& (A\oplus A',A\oplus A') \ar[uul,swap,"(s_0\text{,}s)",""{name=X2},red] \ar[uur,"(s_1\text{,}s)",""{name=Y2},red] 
\ar[uuuul,bend left=120, xshift=-0.7em,""{name=X1},"(s_0\oplus C\oplus C'\text{,}\,\,\s)",blue]  \ar[uuuur,bend right=120, xshift=0.7em,""{name=Y1},swap,"(s_1\oplus C\oplus C'\text{,}\s)",blue] 
\ar[from=X1, to=X2,phantom, "(1')"] 
\ar[from=Y1, to=Y2,phantom, "(2')"]
\end{tikzcd}
\end{equation}
where the red edges are Loop~\eqref{eq:NenashevL2} and the blue edges are loop $L$. A similar check shows Triangles (1') - (4') are also boundaries of 2-simplices (details in Claim~\ref{claim:NenL2}). Conclude that Loop~\eqref{eq:NenashevL1} and  Loop~\eqref{eq:NenashevL2} are both homotopic to loop $L$, and thus homotopic to each other as well. 
\end{proof}

\subsubsection*{Proof of Theorem~\ref{thm:NenashevDSES}} Given any Sherman Loop $G(\alpha,\beta,\theta)\in K_1(\calC)$, we construct another loop $\mu(l(x))$ where $l(x)$ is a double exact square. Reviewing our work:
\begin{itemize}
	\item Lemma~\ref{lem:Nen1} shows $\mu(l(x))$ is freely homotopic to Loop~\eqref{eq:NenashevL1}.
	\item Lemma~\ref{lem:Nen2} shows $G(\alpha,\beta,\theta)$ is freely homotopic to Loop~\eqref{eq:NenashevL2}. \item Lemma~\ref{lem:Nen4} shows  Loops~\eqref{eq:NenashevL1} and ~\eqref{eq:NenashevL2} are homotopic.
\end{itemize}
Since $K_1(\calC)$ is an abelian group, deduce that $G(\alpha,\beta,\theta)$ is homotopic to $\mu (l(x))$. Since $K_1(\calC)$ is generated by Sherman Loops (Theorem~\ref{thm:Sherman}), conclude that $K_1(\calC)$ is generated by double exact squares.
\end{proof}

\begin{discussion}\label{dis:NenPushout} Although we were guided by Nenashev's proof in \cite{NenGen}, a direct translation to our setting does not work. The key issue is the isomorphism 
	$$B  \oplus \frac{B}{A} \cong  B\star_A B,$$
which holds in exact categories and is central to \cite[Lemma 2.6]{NenGen}, but (as Inna Zakharevich pointed out to the author) fails in $\Var_k$ -- see Figure~\ref{fig:Inna}. To get around this, we construct the homotopies explicitly, verifying by hand that the chosen diagrams define valid 2-simplices. The methodological upshot: while pushouts in exact categories are arguably better behaved, restricted pushouts still retain enough formal properties to make the analysis go through (cf. Lemmas~\ref{lem:Nen3} and \ref{lem:DirectSum}). 
\end{discussion}
\vspace{-1.5em}
\begin{figure}[H]
	\centering
	\includegraphics[scale=0.5]{ZakhEx.pdf}
	\caption{$\mathbb{A}^1\star_{\{\ast\}}\mathbb{A}^1$ is not isomorphic to $\mathbb{A}^1\coprod \left(\mathbb{A}^1\setminus \{\ast\} \right)$.
	}\label{fig:Inna}
\end{figure}

\section{Relations of $K_1(\calC)$}\label{sec:Nenashev}

Having characterised the generators of $K_1(\calC)$ for pCGW categories, we now work to determine its relations. We first give a baseline characterisation in Proposition~\ref{prop:baseline}. We then sharpen our understanding by comparing this to other descriptions of $K_1$ by Nenashev \cite{Nen0,Nen1} (for exact categories) and Zakharevich \cite{ZakhK1} (for Assemblers). A guiding observation is Warning~\ref{warning:compose}, which highlights a technical subtlety regarding the composition of 1-simplices in $K_1$. This brings into focus an apparent discrepancy between our account and Zakharevich's, which we investigate.

\subsection{A Baseline Argument} Recall from Convention~\ref{conv:K1} that $K_1(\calC)=\pi_1|G\calC^o|$. Hence, applying the standard presentation of the fundamental group of a connected simplicial space, we get the following:

\begin{proposition}\label{prop:baseline} $K_1(\calC)$ is generated by symbols $\lrangles{f}$, for isomorphism classes of double exact squares $f$ in $G\calC$, modulo the relations:
\begin{itemize}
	\item[(B1)] Given any $A\in\calC$, the standard edge $e(A)\colon (O,O)\to (A,A)$ of $G\calC$ vanishes. That is,
	\[\left\langle\left(\dsquaref{O}{A}{O}{A}{}{}{}{1}\,,\,\dsquaref{O}{A}{O}{A}{}{}{}{1}\right)\right\rangle=0.\]
	\item[(B2)]  Given any $A\in\calC$, the degenerate 1-simplex $\id_A\colon (A,A)\to (A,A)$ of $G\calC$ vanishes. That is, 
	\[\left\langle\left(\dsquaref{O}{O}{A}{A}{}{}{1}{}\,,\,\dsquaref{O}{O}{A}{A}{}{}{1}{}\right)\right\rangle=0.\]
	\item[(B3)] Given double exact squares of the form
	\begin{equation}\label{eq:baseline1SIMPa}
	\begin{small}
	l_A:=\left(\dsquaref{O}{X}{A}{B}{}{}{f_0}{g_0} \,,\, \dsquaref{O}{X}{A}{B}{}{}{f'_0}{g'_0} \right) \quad 	l_B:=\left(\dsquaref{O}{Y}{B}{C}{}{}{f_1}{g_1} \,,\, \dsquaref{O}{Y}{B}{C}{}{}{f'_1}{g'_1} \right)
	\end{small}
	\end{equation}
\begin{equation}\label{eq:baseline1SIMPb}
\begin{small}
l_C:=\left(\dsquaref{O}{Z}{A}{C}{}{}{f_2}{g_2} \,,\, \dsquaref{O}{Z}{A}{C}{}{}{f'_2}{g'_2} \right) 
\end{small}
\end{equation}
that assemble into a 2-simplex in $G\calC$
	\begin{equation}\label{eq:baseline2SIMP}
\begin{tikzcd}
A\ar[r, >->," f_0"] & B  \ar[dr,phantom,"\square"] \ar[r, >->,"f_1"] & C\\
& X \ar[dr,phantom,"\square"] \ar[dr,phantom,"\square"]\ar[r, >->,swap,"h_1"] \ar[u, {Circle[open]}->,"g_0"]& Z \ar[u, {Circle[open]}->, swap, "g_2"] \\
& O \ar[r,>->] \ar[u, {Circle[open]}->]& Y\ar[u, {Circle[open]}->,swap,"h_2"] 
\end{tikzcd}\quad \begin{tikzcd}
A\ar[r, >->," f'_0"] & B  \ar[dr,phantom,"\square"] \ar[r, >->,"f'_1"] & C\\
& X \ar[dr,phantom,"\square"] \ar[dr,phantom,"\square"]\ar[r, >->,swap,"h_1"] \ar[u, {Circle[open]}->,"g'_0"]& Z \ar[u, {Circle[open]}->, swap, "g'_2"] \\
& O \ar[r,>->] \ar[u, {Circle[open]}->]& Y\ar[u, {Circle[open]}->,swap,"h_2"] 
\end{tikzcd}
\end{equation}
we have that 
$$ \lrangles{l_A} + \lrangles{l_B} = \lrangles{l_C}.$$
\end{itemize}	
\end{proposition}	
\begin{proof} Let $\iota \colon X\hookrightarrow G\calC^o$ be the simplicial subset generated by vertices $(A,A)$ and double exact squares. By Theorem~\ref{thm:NenashevDSES}, the inclusion map induces an isomorphism $\iota_*\colon \pi_1|X|\to \pi_1|G\calC^o|$, so it suffices to describe $\pi_1|X|$.\footnote{{\em Details.} Surjectivity follows directly from Theorem~\ref{thm:NenashevDSES}. For injectivity, any 2-simplex relation in $G\calC^o$ can be refined, up to homotopy, to lie in $X$: each of its edges represents a canonical loop in $\pi_1|G\calC^o|$ (up to choice of maximal spanning tree -- see Remark~\ref{rem:piGC}), which by Theorem~\ref{thm:NenashevDSES} is homotopic the canonical loop of some double exact square.}
\begin{comment} We claim this induces an isomorphism $\iota_*\colon \pi_1|X|\to \pi_1(G\calC^o)$. Surjectivity is exactly Theorem~\ref{thm:NenashevDSES}: any loop in $K_1(\calC)$ is homotopic to the canonical loop of a double exact square. Further, any 2-simplex in $G\calC^o$ comprises 1-simplices as its edges. After declaring the appropriate maximal spanning tree in $G\calC^o$ (see Remark~\ref{rem:piGC}), these can be viewed as loops in $K_1$ as well, and so are homotopic to canonical loops of double exact squares. Put otherwise, any homotopy in $G\calC^o$ between double exact squares can be refined, up to homotopy, to lie in $X$. This gives injectivity. 
\end{comment}
Define $\Gamma$ as the set of 1-simplices $\{(O,O)\to (A,A)\}$; this is clearly a maximal tree spanning the 1-skeleton of $X$. We therefore obtain the standard presentation:
	$$\pi_1|X|:={{\bigslant{\pi_0(X[1])}{\stackanchor{$\langle t\rangle $ = 0 if $t\in\Gamma$, and $\lrangles{\id_A}=0$ for any degenerate 1-simplex}{$d_1(x)=d_2(x) + d_0(x),\quad \forall x\in \pi_0(X[2])$}}}}\qquad .$$
\begin{comment}
Sketch of proof behind standard presentation. Suppose $X$ is a connected simplicial space with maximal tree $\Gamma$. The fundamental group of its geometric realisation is determined by its 2-skeleton. Thus, as a first reduction, take the 2-truncation of $X$. Next, collapse the maximal spanning tree $\Gamma$ in $X$ to a single vertex $O$. Notice that since $\Gamma$ is contractible, this implies $|X|\simeq |X/\Gamma|$, and so $\pi_1(|X|)\cong \pi_1(|X/\Gamma|)$. Geometrically, collapsing $\Gamma$ has the effect of turning all 1-simplices $t\notin\Gamma$ into loops based at vertex $O$. 

It is well-established (e.g. \cite[Prop. IV.8.4]{WeibelKBook}) that if $X_\bullet$ is any simplicial space with $X[0]=\{\ast\}$, then $\pi_1|X_\bullet|$ is the free group on the 1-simplices modulo the relations $d_1(x)=d_2(x)d_0(x)$ for every $x\in \pi_0(X[2])$.\footnote{Alternatively, one may wish to prove this directly by applying Van Kampen's Theorem to the skeletal filtration of $X$.} Notice this gives us the right generators of $\pi_1|X/\Gamma|$ and one of its relations. To get the remaining relations, notice that obviously $\lrangles{t}=0$ if $t\in\Gamma$ since we contracted $X$ by $\Gamma$. Finally, the geometric realisation of any simplicial set associates an $n$-cell to any \emph{non-degenerate} $n$-simplex. Put otherwise, the degenerate 1-simplices of $X$ do not contribute to $\pi_1|X/\Gamma|$ and so $\lrangles{id_A}=0$. And we are done. 
\end{comment}  
$\pi_0 (X[1])$ corresponds to the isomorphism classes of double exact squares. Further, any $x\in \pi_0 (X[2])$ can be represented as Diagram~\ref{eq:baseline2SIMP}, where $d_0(x)=l_B$, $d_2(x)=l_A$ and $d_1=l_C$. Putting everything together yields the stated presentation of $K_1$ in the proposition.
\end{proof}

\begin{comment} I was going to use the notation e.g. $\pi_0|X[2]|$, but this doesn't look very nice. The variation in notation shouldn't be too much of a problem for the reader.
\end{comment}

\begin{remark}\label{rem:piGC} Was appealing to Theorem~\ref{thm:NenashevDSES} necessary? In principle, one could apply the above argument to $GC^o$ directly, but there are complications. Notice the obvious choice set of 1-simplices $$\{(O,O)\to (A,A')\}$$ 
only connects to vertices where $A\cong A'$, and so fails to form a maximal tree in general. Of course, this can always be upgraded to some maximal tree $T$ by Zorn's Lemma, but we would no longer have an explicit description of $T$ -- and thus no explicit description of the relations for $\pi_1|G\calC^o|$ either.%\footnote{Nonetheless, Proposition~\ref{prop:baseline} clarifies the various constructions introduced in Section~\ref{sec:generators}. The Sherman Loops generating $K_1$ can be seen as avatars of the 1-simplices of $G\calC$, whereas the canonical loops of double exact squares, featured in Theorem~\ref{thm:NenashevDSES}, implicitly specify a maximal spanning tree.} 
\end{remark}


\begin{warning}[Composition of 1-simplices]\label{warning:compose} Proposition~\ref{prop:baseline} does {\em not} assert that any three double exact squares $l_A,l_B,l_C$ give the identity
	$$ \lrangles{l_A} + \lrangles{l_B} = \lrangles{l_C},\qquad \text{whenever\,\, $f_0\circ f_1 = f_2$ \,\,and\,\, $f'_0\circ f'_1 = f'_2$.}$$
Why? While composition in $\calS\calC$ yields a pair of flag diagrams, notice these define a 2-simplex in $G\calC$ just in case the quotient index triangles in Equation~\eqref{eq:baseline2SIMP} 
\[\left(\dsquaref{O}{Y}{X}{Z}{ }{ }{h_1}{h_2 }\quad ,\quad\dsquaref{O}{Y}{X}{Z}{ }{ }{h_1}{h_2 }\right)\]
are two copies of the same square.
\end{warning}

Keeping Warning~\ref{warning:compose} in mind will help us appreciate the work done in the subsequent sections. On one level, we extend Nenashev's work on exact categories \cite{Nen0,Nen1} to our setting, providing yet another characterisation of $K_1(\calC)$. On another level, Nenashev's presentation clarifies {\em how} composition of 1-simplices splits in $K_1$, illuminating the difference between our approach and Zakharevich's. 


\subsection{Admissible Triples} 
 A {\em triangle contour} $\mathcal{T}$ in $G\calC$ 
\begin{equation}\label{eq:tricontour}
\begin{tikzcd}
&(P_1,P'_1) \ar[dr,"e_1"]\\
(P_0,P'_0) \ar[ur,"e_0"] \ar[rr,"e_2"] && (P_2,P'_2)
\end{tikzcd} 
\end{equation}
is given by three pairs of distinguished squares of the form 
\begin{equation*}
e_0:=\small{\left(\dsquaref{O}{P_{1/0}}{P_0}{P_1}{}{}{\alpha_{0,1}}{\alpha_{1/0,1}} \quad,\quad \dsquaref{O}{P_{1/0}}{P'_0}{P'_1}{}{}{\alpha'_{0,1}}{\alpha'_{1/0,1}}\right)}\quad e_1:=\small{\left(\dsquaref{O}{P_{2/1}}{P_1}{P_2}{}{}{\alpha_{1,2}}{\alpha_{2/1,2}} \quad,\quad \dsquaref{O}{P_{2/1}}{P'_1}{P'_2}{}{}{\alpha'_{1,2}}{\alpha'_{2/1,2}}\right)} 
\end{equation*}

\begin{equation*}
e_2:=\small{\left(\dsquaref{O}{P_{2/0}}{P_0}{P_2}{}{}{\alpha_{0,2}}{\alpha_{2/0,2}} \quad,\quad \dsquaref{O}{P_{2/0}}{P'_0}{P'_2}{}{}{\alpha'_{0,2}}{\alpha'_{2/0,2}}\right)}.
\end{equation*}


\begin{definition}[Admissible Triple]\label{def:AdmTrip} We call a triple $\calT=(e_0,e_1,e_2)$ of the above form \emph{admissible} if $$\alpha_{1,2}\circ \alpha_{0,1}=\alpha_{0,2}\qquad \text{and}\qquad \alpha'_{1,2}\circ \alpha'_{0,1}=\alpha'_{0,2}\,\,.$$ 
In which case, one can apply Lemma~\ref{lem:quotFilt} and complete $\calT$ to the following pair of diagrams:
	\begin{equation}\label{eq:admtripDEF}
\begin{tikzcd}
P_0\ar[r, >->," \alpha_{0,1} "] & P_1  \ar[dr,phantom,"\square"] \ar[r, >->,"\alpha_{1,2}"] & P_2\\
&	P_{1/0} \ar[dr,phantom,"\square"] \ar[dr,phantom,"\square"]\ar[r, >->,swap,"\alpha_{1/0,2/0}"] \ar[u, {Circle[open]}->,"\alpha_{1/0,1}"]& P_{2/0} \ar[u, {Circle[open]}->, swap, "\alpha_{2/0,2}"] \\
& O \ar[r,>->] \ar[u, {Circle[open]}->]& P_{2/1} \ar[u, {Circle[open]}->,swap,"\alpha_{2/1,2/0}"] 
\end{tikzcd}\quad \begin{tikzcd}
P'_0\ar[r, >->," \alpha'_{0,1} "]  & P'_1 \ar[dr,phantom,"\square"] \ar[r, >->,"\alpha'_{1,2}"] & P'_2\\
 &	P_{1/0} \ar[dr,phantom,"\square"] \ar[r, >->,swap,"\alpha'_{1/0,2/0}"] \ar[u, {Circle[open]}->,"\alpha'_{1/0,1}"]& P_{2/0} \ar[u, {Circle[open]}->, swap, "\alpha'_{2/0,2}"] \\
& O \ar[r,>->] \ar[u, {Circle[open]}->]& P_{2/1} \ar[u, {Circle[open]}->,swap,"\alpha'_{2/1,2/0}"] 
\end{tikzcd}
\end{equation}
 In particular, define 
\begin{equation}\label{eq:admiAssocDS}
l(\calT):=\left(\dsquaref{O}{P_{2/1}}{P_{1/0}}{P_{2/0}}{}{}{\alpha_{1/0,2/0}}{\alpha_{2/1,2/0}} \quad \, \quad \dsquaref{O}{P_{2/1}}{P_{1/0}}{P_{2/0}}{}{}{\alpha'_{1/0,2/0}}{\alpha'_{2/1,2/0}}\right)
\end{equation}
to be the \emph{double exact square associated to admissible triple $\calT$}. Following Definition~\ref{def:DES}, denote $\mu (l(\calT))$ to be the canonical loop of $l(\calT)$.
\end{definition}


Any admissible triple $\calT=(e_0,e_1,e_{2})$ defines a loop $e_0e_1e^{-1}_2$, which we also denote using $\calT$. Notice: if $$\alpha_{1/0,2/0}=\alpha'_{1/0,2/0}\qquad 
\text{and}\qquad \alpha_{2/1,2/0}=\alpha'_{2/1,2/0}$$ 
in Diagram~\eqref{eq:admtripDEF}, then the loop $\calT$ bounds a 2-simplex in $G\calC$. However, even when the condition does not hold, we can still say something meaningful about the (free) homotopy class of $\calT$:

\begin{lemma}\label{lem:admtriple} Let $\calT$ be an admissible triple, with $l(\calT)$ its associated double exact square. If the 1-simplices of $\calT$ are all double exact squares, then the loop $\calT$ is freely homotopic to $\mu(l(\calT))$.
\end{lemma}

This lemma extends \cite[Corollary 4.3]{Nen0}. The proof is more involved than Nenashev's original argument (for similar reasons as in Discussion~\ref{dis:NenPushout}), and will be deferred to Section~\ref{sec:admtriples}.

\subsection{Nenashev Relations} To define Nenashev's relations on $K_1(\calC)$, we shall need the following generalisation of double exact squares. A \emph{$3 \times 3$ diagram} in a pCGW category $\calC$ is a pair of diagrams
\begin{equation}\label{eq:txDiagram}
\left(\begin{tikzcd}
X_{00}\ar[r, >->, "f_{0}"] 
\ar[d,>->,swap,"h_{0}"] & X_{01} \ar[d, >->,"h_{1}"] & X_{02}  
\ar[d,>->, "h_{2}"] \ar[l, {Circle[open]}->,swap,"g_{0}"]\\
X_{10}  \ar[r,>->,"f_{1}"]  &	X_{11}  \ar[dr,phantom,"\circlearrowleft"] & X_{12} \ar[l, {Circle[open]}->,swap,"g_1"] \\
X_{20} \ar[u, {Circle[open]}->,"j_{0}"] \ar[r,>->,"f_2"]& X_{21} \ar[u, {Circle[open]}->,"j_1"]& X_{22}  \ar[u, {Circle[open]}->,swap,"j_2"] \ar[l, {Circle[open]}->,swap,"g_{2}"]
\end{tikzcd} \qquad\text{,}\qquad  \begin{tikzcd}
X_{00}\ar[r, >->, "f'_{0}"] 
\ar[d,>->,swap,"h'_{0}"] & X_{01} \ar[d, >->,"h'_{1}"] & X_{02}  
\ar[d,>->, "h'_{2}"] \ar[l, {Circle[open]}->,swap,"g'_{0}"]\\
X_{10}  \ar[r,>->,"f'_{1}"]  &	X_{11}  \ar[dr,phantom,"\circlearrowleft"] & X_{12} \ar[l, {Circle[open]}->,swap,"g'_1"] \\
X_{20} \ar[u, {Circle[open]}->,"j'_{0}"] \ar[r,>->,"f'_2"]& X_{21} \ar[u, {Circle[open]}->,"j'_1"]& X_{22}  \ar[u, {Circle[open]}->,swap,"j'_2"] \ar[l, {Circle[open]}->,swap,"g'_{2}"]
\end{tikzcd}\right)
\end{equation}
on the same objects subject to the following conditions:
\begin{itemize} 
	\item The horizontal and vertical rows of each diagram define exact squares. Explicitly, a $\tx$ diagram is defined by 6 double exact squares
	\begin{equation*}
l_{i}:= \left(	\dsquaref{O}{X_{i2}}{X_{i0}}{X_{i1}}{ }{ }{ f_{i}}{g_{i} } \, \,,\,\,	\dsquaref{O}{X_{i2}}{X_{i0}}{X_{i1}}{ }{ }{ f'_{i}}{g'_{i} }\right) \qquad	l^{i}:= \left(\dsquaref{O}{X_{2i}}{X_{0i}}{X_{1i}}{ }{}{h_{i} }{ j_{i}} \, \,,\,\, \dsquaref{O}{X_{2i}}{X_{0i}}{X_{1i}}{ }{}{h'_{i} }{ j'_{i}} \right),
\end{equation*}
for all $i\in \{0,1,2\}$.
	\item The bottom right $\E$-square commutes. No conditions are imposed on the other squares. % -- in particular, the mixed squares need not be distinguished. 
\begin{comment}
Note to self: might be able to get rid of the $\E$-square condition; Diagrams (P1) and (P2) show that the $\E$-square commutes up to codomain-preserving isomorphism. Is this enough? If we want to use Proposition~\ref{prop:baseline}: given two exact squares with isomorphic quotients, the obvious flag diagram may not be a 2-simplex.
\end{comment}
\end{itemize}


These $\tx$ diagrams were originally defined as commutative diagrams in exact categories. In this setting, one can form the pushout $Z:=X_{01}\star_{X_{00}} X_{10}$ and check that the induced map $v\colon Z\rtail X_{11}$ is an admissible mono -- this plays a key role in the proof of \cite[Prop. 5.1]{Nen0}. However, this argument breaks down in general pCGW categories: restricted pushouts may fail to yield an $\M$-morphism $v\colon Z\rtail X_{11}$ unless the original $\M$-square in Diagram~\eqref{eq:txDiagram} is {\em optimal} (in the sense of Definition~\ref{def:restPO}). We circumvent this problem by hardcoding the desired pushout behaviour into the following definition. 



\begin{definition}[Optimal $\tx$ Diagrams]\label{def:OPTtx} Call a $\tx$ diagram \emph{optimal} if there exists an object $Z$ along with three pairs of $\M$-morphisms 
	$$v,v'\colon Z\rtail X_{11},\quad u,u'\colon X_{01}\rtail Z \quad \text{and}\quad w,w'\colon X_{10}\rtail Z.$$
These are required to satisfy the identities
\begin{align}\label{eq:OPTID}
&u\circ f_0=w\circ h_0  &  v\circ u = h_1  &&  v\circ w = f_1 \\
& u'\circ f'_0=w'\circ h'_0 & v'\circ u' = h'_1 && v'\circ w '= f'_1\,\,, \nonumber 
\end{align}
and assemble into the following diagrams 
\begin{enumerate}[label=(P\arabic*)]
	\item \[
\begin{tikzcd}
X_{01}\ar[r, >->,"u"] \ar[dr,phantom,"\square"] & Z \ar[dr,phantom,"\square"] \ar[r, >->,"v"] & X_{11} \\
O \ar[r,>->] \ar[u,{Circle[open]}->]&	X_{20} \ar[r, >->,"f_2"] \ar[u, {Circle[open]}->] \ar[dr,phantom,"\square"]  & X_{21}\ar[u, {Circle[open]}->,"j_1"] \\
& O \ar[r,>->] \ar[u,{Circle[open]}->]& X_{22} \ar[u, {Circle[open]}->,"g_2"] 
\end{tikzcd}	 \quad 	\begin{tikzcd}
	X_{01}\ar[r, >->,"u'"] \ar[dr,phantom,"\square"] \ar[dr,phantom,"\square"] & Z \ar[dr,phantom,"\square"] \ar[r, >->,"v'"] & X_{11} \\
	O \ar[r,>->] \ar[u,{Circle[open]}->]&	X_{20} \ar[dr,phantom,"\square"] \ar[r, >->,"f'_2"] \ar[u, {Circle[open]}->]& X_{21}\ar[u, {Circle[open]}->,"j'_1"] \\
	& O \ar[r,>->] \ar[u,{Circle[open]}->]& X_{22} \ar[u, {Circle[open]}->,"g_2'"] 
	\end{tikzcd}	
	\]
		\item \[
	\begin{tikzcd}
	X_{10}\ar[r, >->,"w"]\ar[dr,phantom,"\square"]  & Z \ar[dr,phantom,"\square"] \ar[r, >->,"v"] & X_{11} \\
	O \ar[r,>->] \ar[u,{Circle[open]}->]&	X_{02} \ar[dr,phantom,"\square"]  \ar[r, >->,"h_2"] \ar[u, {Circle[open]}->]& X_{12}\ar[u, {Circle[open]}->,"g_1"] \\
	&O \ar[r,>->] \ar[u,{Circle[open]}->]& X_{22} \ar[u, {Circle[open]}->,"j_2"] 
	\end{tikzcd}	 \quad 	\begin{tikzcd}
	X_{10}\ar[r, >->,"w'"] \ar[dr,phantom,"\square"] & Z \ar[dr,phantom,"\square"] \ar[r, >->,"v'"] & X_{11} \\
O \ar[r,>->] \ar[u,{Circle[open]}->]	&	X_{02} \ar[dr,phantom,"\square"]  \ar[r, >->,"h'_2"] \ar[u, {Circle[open]}->]& X_{12}\ar[u, {Circle[open]}->,"g'_1"] \\
	&O \ar[r,>->] \ar[u,{Circle[open]}->]& X_{22} \ar[u, {Circle[open]}->,"j'_2"] 
	\end{tikzcd}	
	\]
	\item \[
	\begin{tikzcd}
	X_{00}\ar[r, >->,"f_0"] & X_{01} \ar[dr,phantom,"\square"] \ar[r, >->,"u"] & Z \\
	&	X_{02} \ar[r, >->] \ar[u, {Circle[open]}->,"g_0"]& X_{02}\oplus X_{20}\ar[u, {Circle[open]}->] \\
	&& X_{20} \ar[u, {Circle[open]}->] 
	\end{tikzcd}	 \quad 		\begin{tikzcd}
	X_{00}\ar[r, >->,"f'_0"] & X_{01} \ar[dr,phantom,"\square"] \ar[r, >->,"u'"] & Z \\
	&	X_{02} \ar[r, >->] \ar[u, {Circle[open]}->,"g'_0"]& X_{02}\oplus X_{20}\ar[u, {Circle[open]}->] \\
	&& X_{20} \ar[u, {Circle[open]}->] 
	\end{tikzcd}	
	\]
	
		\item \[
	\begin{tikzcd}
	X_{00}\ar[r, >->,"h_0"] & X_{10} \ar[dr,phantom,"\square"] \ar[r, >->,"w"] & Z \\
	&	X_{20} \ar[r, >->] \ar[u, {Circle[open]}->,"j_0"]& X_{02}\oplus X_{20}\ar[u, {Circle[open]}->] \\
	&& X_{02} \ar[u, {Circle[open]}->,] 
	\end{tikzcd}	 \quad 		\begin{tikzcd}
	X_{00}\ar[r, >->,"h'_0"] & X_{10} \ar[dr,phantom,"\square"] \ar[r, >->,"w'"] & Z \\
	&	X_{20} \ar[r, >->] \ar[u, {Circle[open]}->,"j'_0"]& X_{02}\oplus X_{20}\ar[u, {Circle[open]}->] \\
	&& X_{02} \ar[u, {Circle[open]}->] 
	\end{tikzcd}	
	\]
\end{enumerate}	
\end{definition}

\begin{remark} A sanity check: notice the Optimality Identities~\eqref{eq:OPTID} imply that the $\M$-squares in Diagram~\eqref{eq:txDiagram} commute, e.g. $h_1\circ f_0=f_1\circ h_0$. 
% Further, Diagrams (P1) and (P2) show that $j_1\circ g_2$ and $g_1\circ j_2$ are both quotients of $v$ -- and thus they are equivalent up to codomain-preserving isomorphism.
\end{remark}

\begin{definition}\label{def:NenDC} Define $\calD(\calC)$ to be the abelian group with generators $\lrangles{l}$ for all double exact squares $l$ in $\calC$ subject to the following relations. 
	\begin{enumerate}[label=(N\arabic*)]
	\item $\lrangles{l}=0$ if 
\begin{equation}\label{eq:NenR1}
l=\left(\dsquaref{O}{C}{A}{B}{}{}{f}{g} \quad ,\quad \dsquaref{O}{C}{A}{B}{}{}{f}{g}\right)
\end{equation}
Any identical pair of exact squares will be called a \emph{diagonal} square.
	\item Given an optimal $\tx$ diagram
	\begin{equation}\label{eq:3x3}
\left(\begin{tikzcd}
X_{00}\ar[r, >->, "f_{0}"] 
\ar[d,>->,swap,"h_{0}"]  & X_{01} \ar[d, >->,"h_{1}"] & X_{02}  
\ar[d,>->, "h_{2}"] \ar[l, {Circle[open]}->,swap,"g_{0}"]\\
X_{10}  \ar[r,>->,"f_{1}"]  &	X_{11}  \ar[dr,phantom,"\circlearrowleft"] & X_{12} \ar[l, {Circle[open]}->,swap,"g_1"] \\
X_{20} \ar[u, {Circle[open]}->,"j_{0}"] \ar[r,>->,"f_2"]& X_{21} \ar[u, {Circle[open]}->,"j_1"]& X_{22}  \ar[u, {Circle[open]}->,swap,"j_2"] \ar[l, {Circle[open]}->,swap,"g_{2}"]
\end{tikzcd} \qquad\text{,}\qquad  \begin{tikzcd}
X_{00}\ar[r, >->, "f'_{0}"] 
\ar[d,>->,swap,"h'_{0}"] & X_{01} \ar[d, >->,"h'_{1}"] & X_{02}  
\ar[d,>->, "h'_{2}"] \ar[l, {Circle[open]}->,swap,"g'_{0}"]\\
X_{10}  \ar[r,>->,"f'_{1}"]  &	X_{11} \ar[dr,phantom,"\circlearrowleft"]  & X_{12} \ar[l, {Circle[open]}->,swap,"g'_1"] \\
X_{20} \ar[u, {Circle[open]}->,"j'_{0}"] \ar[r,>->,"f'_2"]& X_{21} \ar[u, {Circle[open]}->,"j'_1"]& X_{22}  \ar[u, {Circle[open]}->,swap,"j'_2"] \ar[l, {Circle[open]}->,swap,"g'_{2}"]
\end{tikzcd}\right)
\end{equation}
defined by the following 6 double exact squares
	\begin{equation*}
l_{i}:= \left(	\dsquaref{O}{X_{i2}}{X_{i0}}{X_{i1}}{ }{ }{ f_{i}}{g_{i} } \, \,,\,\,	\dsquaref{O}{X_{i2}}{X_{i0}}{X_{i1}}{ }{ }{ f'_{i}}{g'_{i} }\right) \qquad	l^{i}:= \left(\dsquaref{O}{X_{2i}}{X_{0i}}{X_{1i}}{ }{}{h_{i} }{ j_{i}} \, \,,\,\, \dsquaref{O}{X_{2i}}{X_{0i}}{X_{1i}}{ }{}{h'_{i} }{ j'_{i}} \right)
\end{equation*}
for all $i\in \{0,1,2\}$, the following 6-term relation holds
\begin{equation}
\lrangles{l_0} + \lrangles{l_2} - \lrangles{l_1} = \lrangles{l^0} + \lrangles{l^2} - \lrangles{l^1}. 
\end{equation}
\end{enumerate}
\end{definition}
\begin{theorem}\label{thm:NenSurj} There exists a well-defined homomorphism 
\begin{equation}\label{eq:NenSurj}
m\colon \calD(\calC)\longrightarrow K_1(\calC)
\end{equation}
that is surjective. In other words, the two relations of $\calD(\calC)$ also hold in $K_1(\calC)$.	
\end{theorem}
\begin{proof} By Theorem~\ref{thm:NenashevDSES}, we know that $K_1(\calC)$ is generated by double exact squares so both groups have the same generators. It remains to check the relations.
\begin{enumerate}[label=(N\arabic*):]
	\item Let $l$ be as in Equation~\eqref{eq:NenR1}. The corresponding loop $\mu(l)$ bounds the 2-simplex 
\[	\begin{tikzcd}
O\ar[r, >->] & A \ar[dr,phantom,"\square"] \ar[r, >->,"f"] & B \\
	&	A \ar[r, >->,"f"] \ar[u, {Circle[open]}->]& B \ar[u, {Circle[open]}->,""] \\
	&& C \ar[u, {Circle[open]}->,"g"] 
	\end{tikzcd}	 \quad 	\begin{tikzcd}
	O\ar[r, >->] & A \ar[dr,phantom,"\square"] \ar[r, >->,"f"] & B \\
	&	A \ar[r, >->,"f"] \ar[u, {Circle[open]}->]& B \ar[u, {Circle[open]}->,""] \\
	&& C \ar[u, {Circle[open]}->,"g"] 
	\end{tikzcd}
	\]
in $G\calC$, and so $\lrangles{l}=0$.
	\item 
Leveraging the fact that the $\tx$ diagram is optimal, construct the diagram
	\begin{equation}
\begin{tikzcd}
(X_{00}\tk X_{00}) \ar[rr,"l_0",blue] \ar[dd,"l^{0}",blue] \ar[dr,"\alpha_0"]&& (X_{01}\tk  X_{01}) \ar[dd,"l^1",blue] \ar[dl,swap,"\alpha_3"]\\
& (Z \tk Z) \ar[dr,"\alpha_2"]  \\
(X_{10}\tk X_{10}) \ar[rr,"l_1",blue] \ar[ur,"\alpha_1"] && (X_{11} \tk X_{11})
\end{tikzcd}
\end{equation}
where outer blue edges $\alpha:=l_0l^1(l_1)^{-1}(l^0)^{-1}$ form a loop, and the inner edges are given by 
\[\small{\alpha_0:= \left(	\dsquaref{O}{X_{02}\oplus X_{20}}{X_{00}}{Z}{ }{ }{u\circ f_0}{} \, ,\, 	\dsquaref{O}{X_{02}\oplus X_{20}}{X_{00}}{Z}{ }{ }{u'\circ f'_0}{}\right) \quad \alpha_1:= \left(	\dsquaref{O}{X_{02}}{X_{10}}{Z}{ }{ }{w}{} \, ,\, 	\dsquaref{O}{X_{02}}{X_{10}}{Z}{ }{ }{w'}{}\right)}  \]
\[\small{\alpha_2:= \left(	\dsquaref{O}{X_{22}}{Z}{X_{11}}{ }{ }{v}{} \, ,\, 	\dsquaref{O}{X_{22}}{Z'}{X_{11}}{ }{ }{v'}{}\right)  \quad  \alpha_3:= \left(	\dsquaref{O}{X_{20}}{X_{01}}{Z}{ }{ }{u}{} \, ,\, 	\dsquaref{O}{X_{20}}{X_{01}}{Z}{ }{ }{u'}{}\right)}  \] 
For orientation, we start with a basic observation.
 
 \begin{lemma}\label{lem:Nenclosedloop} 
 	Given any closed loop $l=e_0\dots e_n$ whose edges are all double exact squares, 
 	$$ \lrangles{l}= \sum^{n}_{i=0} (-1)^{\epsilon_i}\lrangles{e_i} $$
 	in $K_1(\calC)$, where the coefficient $(-1)^{\epsilon_i}$ reflects the orientation of edge $e_i$ in $l$. 
 \end{lemma}
\begin{proof}[Proof of Lemma] We spell this out for the case of $\alpha$; the general case is completely analogous. Consider the diagram
	\begin{equation}
	\begin{tikzcd}
	(X_{00}\tk X_{00}) \ar[rr,"l_0",blue] \ar[dd,"l^{0}",blue]&& (X_{01}\tk  X_{01}) \ar[dd,"l^1",blue] \\
	& (O \tk O) \ar[ur] \ar[dr] \ar[ul] \ar[dl]  \\
	(X_{10}\tk X_{10}) \ar[rr,"l_1",blue] && (X_{11} \tk X_{11})
	\end{tikzcd}
	\end{equation}
	featuring $\alpha$ as the outer loop but now with the base-point $(O,O)$ at the center. Tracing along the loop $\mu(l_0)\mu(l^1) \mu(l_1)^{-1}\mu(l^0)^{-1}$, each edge $(O,O)\to (X_{ij},X_{ij})$ is traversed exactly once in both directions, and so their contributions cancel. Hence, the loop is freely homotopic to the square boundary, and so
	\begin{equation*}
	\lrangles{\alpha} = \lrangles{l_0} + \lrangles{l^1} - \lrangles{l_1} - \lrangles{l^0}.
	\end{equation*}
\end{proof}
 
Next, appealing once more to the optimality of the $\tx$ diagram, notice:
\begin{itemize}

	\item Diagrams (P3) and (P4) are 2-simplices. By the Optimality Identities~\eqref{eq:OPTID}, they are bounded by the loops $l_0 \alpha_3 \alpha_{0}^{-1}$ and $l^0 \alpha_1 \alpha_{0}^{-1}$ respectively. Therefore, deduce that
\begin{align}\label{eq:3x3r1}
\lrangles{l_0} + \lrangles{\alpha_3} - \lrangles{\alpha_0}=0\\
\lrangles{l^0} + \lrangles{\alpha_1} - \lrangles{\alpha_0}=0. \nonumber
\end{align}
\item Diagrams (P1) and (P2) are admissible triples whose associated exact squares are $l_2$ and $l^2$ respectively. We may represent these admissible triples as $\alpha_{3}\alpha_2 (l^1)^{-1}$ and $\alpha_{1}\alpha_2 (l_1)^{-1}$ respectively; this follows from the Optimality Identities~\eqref{eq:OPTID} and the fact the $\E$-squares in the $\tx$-diagram commute. Thus, apply Lemma~\ref{lem:admtriple} to deduce
\begin{comment}
\begin{align}
\alpha_{3}\alpha_2 (l^1)^{-1}\sim l_{2}
\nonumber \\
\alpha_{1}\alpha_2 (l_1)^{-1}\sim l^2
\end{align}
and so
\end{comment}
\begin{align}\label{eq:3x3r2}
\lrangles{\alpha_3} + \lrangles{\alpha_2} - \lrangles{l^1}=\lrangles{l_2}  \\
\lrangles{\alpha_1} + \lrangles{\alpha_2} - \lrangles{l_1}= \lrangles{l^2}. \nonumber
\end{align}
\end{itemize}
Combining Equations~\eqref{eq:3x3r1} and \eqref{eq:3x3r2}, 
\begin{align*}
\lrangles{l_2} - \lrangles{l^2} & =  \lrangles{\alpha_3} + \lrangles{\alpha_2} - \lrangles{l^1} - \lrangles{\alpha_1} - \lrangles{\alpha_2} + \lrangles{l_1} \\
&= \lrangles{\alpha_3} - \lrangles{\alpha_0} - \lrangles{l^1} - \lrangles{\alpha_1} + \lrangles{\alpha_0} + \lrangles{l_1} \\
& = -\lrangles{l_0} - \lrangles{l^1} + \lrangles{l^0} + \lrangles{l_1},
\end{align*}
and so by rearranging terms, conclude
\begin{equation}
\lrangles{l_0} + \lrangles{l_2} - \lrangles{l_1} = \lrangles{l^0} + \lrangles{l^2} - \lrangles{l^1}.
\end{equation}
\end{enumerate}	
\end{proof}

Theorem~\ref{thm:NenSurj} justifies the claim that $K_1(\Var_k)$ is generated by a certain subclass of birational equivalences. Recall: a birational equivalence $f\colon X\dashrightarrow Y$ on varieties is a rational map defining an isomorphism $U\cong f(U)$, where $U\subseteq X$ and $f(U)\subseteq Y$ are non-empty opens. Call such an $f$ {\em stratified} if, in addition, $X\setminus U\cong Y\setminus f(U)$. This resembles a double exact square in $\Var_k$,\footnote{More precisely: a stratified birational equivalence yields a 1-simplex $(A,A')\to (B,B')$, with two possible choices of quotients -- $(\frac{B}{A},\frac{B}{A})$ or $(\frac{B'}{A'},\frac{B'}{A'})$. These are canonically isomorphic, so the choice does not affect the homotopy class in $K_1$. Finally, since the 1-simplex lies in the basepoint component of $G\Var_k$, it is equivalent to a double exact square in $K_1$ by Theorem~\ref{thm:NenSurj}.} except that exact squares allow the open subvariety to be empty. The following corollary shows this extra generality is not essential. 

\begin{comment} Some notes:

1)  Does every open subset define an open subscheme of a variety? Yes, by taking the restricted sheaf. 

2) The definition of stratified originally also required $X\cong Y$, but this is unnecessary. So long as $X\setminus U\cong Y\setminus f(U)$, then the 1-simplex will lie in $G\calC^o$, and can be refined to a double exact square. See the Footnote argument.

3) Why does the choice of quotient not affect the homotopy class?

Suppose $a\colon \frac{B}{A}\xrightarrow{\cong}\frac{B'}{A'}$. Then this yields the 2-simplex in $G\Var_k$

\[\begin{tikzcd}
A \ar[r,>->] & B\ar[r,>->,"1"] & B\\
& \frac{B}{A} \ar[u,"v"] \ar[r,"a"]& \frac{B'}{A'} \ar[u,swap,"v\circ a^{-1}"]
\end{tikzcd} \qquad \begin{tikzcd}
A' \ar[r,>->] & B'\ar[r,>->,"1"] & B'\\
& \frac{B}{A} \ar[u,"v'"]  \ar[r,"a"]& \frac{B'}{A'} \ar[u,swap,"v'\circ a^{-1}"]
\end{tikzcd}\]
For clarity, if the original inclusion is $z\colon \frac{B'}{A'}\to B'$, we can set $v'=z\circ a$.

It is clear that the squares are distinguished since isomorphisms interact well with distinguished squares. 
%
%Had we picked the other choice of quotients, we would have gotten
%\[\begin{tikzcd}
%A \ar[r,>->] & B\ar[r,>->] & B\\
%& \frac{B'}{A'} \ar[u,"v\circ a^{-1}"] \ar[r,"a^{-1}"]& \frac{B}{A} \ar[u,swap,"v"]
%\end{tikzcd} \qquad \begin{tikzcd}
%A' \ar[r,>->] & B'\ar[r,>->] & B'\\
%& \frac{B'}{A'} \ar[u,"z"]  \ar[r,"a^{-1}"]& \frac{B}{A} \ar[u,swap,"z\circ a"]
%\end{tikzcd}\]
%where we set $v'=z\circ a$. 

So why does this 2-simplex induce a homotopy between the two 1-simplices? By Proposition~\ref{prop:baseline}, we get $$[(A,A')\to (B,B')] + [(B,B')\to (B,B')] = [(A,A')\to (B,B')].$$
It thus suffices to show that $(B,B')\to (B,B')$ trivialises. Pick a maximal spanning tree such that the choice of quotient of $(O,O)\to (B,B')$ is $(B,B)$. Then the canonical loop bounds the 2-simplex below, and thus trivialises.
\[\begin{tikzcd}
O \ar[r,>->] & B\ar[r,>->,"1"] & B\\
& B \ar[u,"1"] \ar[r,"1"]& B \ar[u,swap,"1"]
\end{tikzcd} \qquad \begin{tikzcd}
O\ar[r,>->] & B'\ar[r,>->,"1"] & B'\\
& B \ar[u,"v"]  \ar[r,"1"]& B \ar[u,swap,"v"]
\end{tikzcd}\]


In fact, this more general presentation is what's used in Gillet-Grayson's original paper; after which, we follow Sherman and Nenashev's presentation, which identifies the quotients.
\end{comment}

\begin{corollary}\label{cor:BirIso} Let $\alpha,\beta\colon A\to A$ be any pair of automorphisms in a pCGW category $\calC$. By Axiom (I), we may either regard them as $\M$-morphisms or $\E$-morphisms. Hence, define
	$$\widetilde{l}(\alpha):=\left(\dsquaref{O}{A}{O}{A}{ }{ }{ }{\varphi(\alpha)} \quad,\quad \dsquaref{O}{A}{O}{A}{ }{ }{ }{1_A} \right).$$
Recall the definition of $l(\alpha)$ from Example~\ref{ex:Aut}. \underline{Then}:
	\begin{enumerate}[label=(\roman*)]
		\item $\lrangles{l(\alpha)}=\lrangles{\widetilde{l}(\alpha^{-1})}.$
		\item $\lrangles{l(\alpha\beta)}=\lrangles{l(\alpha)}+\lrangles{l(\beta)}$.
		\item  Define
	$$l(\alpha,\beta):=\left(\, \dsquaref{O}{O}{A}{A}{ }{ }{\alpha}{} \quad,\quad  \dsquaref{O}{O}{A}{A}{ }{ }{\beta}{}  \,\right),$$
	not to be confused with $l(\alpha\beta)$, which composes $\alpha$ with $\beta$. Then $$\lrangles{l(\alpha,\beta)}=\lrangles{l(\beta\alpha^{-1})}=\lrangles{l(\beta)}-\lrangles{l(\alpha)} \quad \text{in $K_1(\calC)$}.$$ 

  
	\end{enumerate}
\end{corollary}

\begin{remark} Suppose $\alpha,\beta\colon A\to A$ are automorphisms of a variety. By items (i) and (iii) above, $l(\alpha,\beta)=\lrangles{\widetilde{l}(\alpha\beta^{-1})}$, and so $l(\alpha,\beta)\in K_1(\Var_k)$ corresponds to an honest birational automorphism. One may also interpret $\lrangles{l(\alpha,\beta)}$ as the formal difference between automorphisms $\beta$ and $\alpha$.
\end{remark}

\begin{proof}[Proof of Corollary] Recall that $\varphi(\alpha)\colon A\otail A$ defines $\alpha\colon A\to A$ in $\calC$ if $\E\subseteq \calC$, and $\alpha^{-1}\colon A\to A$ if $\E^\opp\subseteq \calC$.
	\begin{enumerate}[label=(\roman*)] 	\item Consider the following $\tx$ diagram
		 \begin{equation*} 
	\small{	\left(\begin{tikzcd}
		O\ar[r, >->] 
		\ar[d,>-> ] & O \ar[d, >->] & O
		\ar[d,>->] \ar[l, {Circle[open]}->]\\
	A  \ar[r,>->,"1_A"]  &A  \ar[dr,phantom,"\circlearrowleft"] & O \ar[l, {Circle[open]}->] \\
		A \ar[u, {Circle[open]}->,"1_A"] \ar[r,>->,"\alpha"]& A \ar[u, {Circle[open]}->,"\varphi(\alpha^{-1})"]& O  \ar[u, {Circle[open]}->] \ar[l, {Circle[open]}->]
		\end{tikzcd} \qquad\text{,}\qquad  \begin{tikzcd}
		O\ar[r, >->] 
	\ar[d,>-> ] & O \ar[d, >->] & O
\ar[d,>->] \ar[l, {Circle[open]}->]\\
A  \ar[r,>->,"\alpha"]  &A \ar[dr,phantom,"\circlearrowleft"] & O \ar[l, {Circle[open]}->] \\
A \ar[u, {Circle[open]}->,"1_A"] \ar[r,>->,"\alpha"]& A \ar[u, {Circle[open]}->,"1_A"]& O  \ar[u, {Circle[open]}->] \ar[l, {Circle[open]}->]
\end{tikzcd}\right).}
		\end{equation*}
To show optimality, pick the following pairs of $\M$-morphisms:
		$$v:=1_A,\,v':= \alpha,\qquad u,u':= O\rtail A \qquad \text{and}\quad w,w':=1_A.$$
The claimed identity then follows by Relations (N1) and (N2).
\begin{comment} Details. Let $u,u'\colon O\to A$, $v:=1_A$, $v':=\alpha$ and $w,w':=1_A \colon A\to A.$
\begin{enumerate}[label=(P\arabic*)]
\item \[
\begin{tikzcd}
O\ar[r, >->] \ar[dr,phantom,"\square"] & A \ar[dr,phantom,"\square"] \ar[r, >->,"1_A"] & A \\
O \ar[r,>->] \ar[u,{Circle[open]}->]&	A \ar[r, >->,"\alpha"] \ar[u, {Circle[open]}->,"1_A"] \ar[dr,phantom,"\square"]  & A\ar[u, {Circle[open]}->,swap,"\varphi(\alpha^{-1})"] \\
& O \ar[r,>->] \ar[u,{Circle[open]}->]& O \ar[u, {Circle[open]}->] 
\end{tikzcd}	 \qquad 	\begin{tikzcd}
O\ar[r, >->] \ar[dr,phantom,"\square"] & A \ar[dr,phantom,"\square"] \ar[r, >->,"\alpha"] & A \\
O \ar[r,>->] \ar[u,{Circle[open]}->]&	A \ar[r, >->,"\alpha"] \ar[u, {Circle[open]}->,"1_A"] \ar[dr,phantom,"\square"]  & A\ar[u, {Circle[open]}->,swap,"1_A"] \\
& O \ar[r,>->] \ar[u,{Circle[open]}->]& O \ar[u, {Circle[open]}->] 
\end{tikzcd}
\]
\item \[
\begin{tikzcd}
A\ar[r, >->,"1_A"]\ar[dr,phantom,"\square"]  & A \ar[dr,phantom,"\square"] \ar[r, >->,"1_A"] & A \\
O \ar[r,>->] \ar[u,{Circle[open]}->]&	O \ar[dr,phantom,"\square"]  \ar[r, >->] \ar[u, {Circle[open]}->]& O\ar[u, {Circle[open]}->] \\
&O \ar[r,>->] \ar[u,{Circle[open]}->]& O \ar[u, {Circle[open]}->] 
\end{tikzcd}	 \quad 	\begin{tikzcd}
A\ar[r, >->,"1_A"]\ar[dr,phantom,"\square"]  & A \ar[dr,phantom,"\square"] \ar[r, >->,"\alpha"] & A \\
O \ar[r,>->] \ar[u,{Circle[open]}->]&	O \ar[dr,phantom,"\square"]  \ar[r, >->] \ar[u, {Circle[open]}->]& O\ar[u, {Circle[open]}->] \\
&O \ar[r,>->] \ar[u,{Circle[open]}->]& O \ar[u, {Circle[open]}->] 
\end{tikzcd}
\]
\item \[
\begin{tikzcd}
O\ar[r, >-> ] & O \ar[dr,phantom,"\square"] \ar[r, >->]  & A \\
&	O} \ar[r, >->] \ar[u, {Circle[open]}->]& O \oplus A\ar[u, {Circle[open]}->] \\
&& A \ar[u, {Circle[open]}->] 
\end{tikzcd}	 \quad 	\begin{tikzcd}
O\ar[r, >-> ] & O \ar[dr,phantom,"\square"] \ar[r, >->]  & A \\
&	O} \ar[r, >->] \ar[u, {Circle[open]}->]& O \oplus A\ar[u, {Circle[open]}->] \\
&& A \ar[u, {Circle[open]}->] 
\end{tikzcd}
\]

\item \[
\begin{tikzcd}
O \ar[r, >->] & A\ar[dr,phantom,"\square"] \ar[r, >->,"1_A"] & A \\
&	A \ar[r, >->] \ar[u, {Circle[open]}->,"1_A"]& O\oplus A\ar[u, {Circle[open]}->] \\
&& O\ar[u, {Circle[open]}->,] 
\end{tikzcd}	 \quad 	\begin{tikzcd}
O \ar[r, >->] & A\ar[dr,phantom,"\square"] \ar[r, >->,"1_A"] & A \\
&	A \ar[r, >->] \ar[u, {Circle[open]}->,"1_A"]& O\oplus A\ar[u, {Circle[open]}->] \\
&& O\ar[u, {Circle[open]}->,] 
\end{tikzcd}
\]
\end{enumerate}	
\end{comment}
\item Immediate from Relation (B3) of Proposition~\ref{prop:baseline} and the fact that $K_1$ is abelian.
\begin{comment}
This is a special case of Relation (A2) in Proposition~\ref{prop:ass}, to be proved in the next section. Notice $\lrangles{l_2}=0$ since it corresponds to the double exact square
$$\left(\dsquare{O}{O}{O}{O}\quad,\quad \dsquare{O}{O}{O}{O}\right).$$ 

Apply the following $\tx$ diagram
\begin{equation*}
\small{	\left(\begin{tikzcd}
A\ar[r, >->,"1_A"] \ar[d,>->, swap,"1_A"] \ar[dr,phantom,"\circlearrowleft"] & A \ar[d, >->,"1_A"] & O
\ar[d,>->] \ar[l, {Circle[open]}->]\\
A  \ar[r,>->,"1_A"]  &A  & O \ar[l, {Circle[open]}->] \\
O \ar[u, {Circle[open]}->] \ar[r,>->]& O \ar[u, {Circle[open]}->] & O  \ar[u, {Circle[open]}->] \ar[l, {Circle[open]}->]
\end{tikzcd} \qquad\text{,}\qquad  \begin{tikzcd}
A\ar[r, >->,"\alpha\beta"] 
\ar[d,>->,swap,"\beta"] \ar[dr,phantom,"\circlearrowleft"] & A \ar[d, >->,"1_A"] & O
\ar[d,>->] \ar[l, {Circle[open]}->]\\
A  \ar[r,>->,"\alpha"]  &A  & O \ar[l, {Circle[open]}->] \\
O \ar[u, {Circle[open]}->] \ar[r,>->]& O \ar[u, {Circle[open]}->] & O  \ar[u, {Circle[open]}->] \ar[l, {Circle[open]}->]
\end{tikzcd} \right).}
\end{equation*}
\end{comment}		
		\item The following (optimal) $\tx$ diagram
		\begin{equation*} 
		\small{	\left(\begin{tikzcd}
			A\ar[r, >->,"\alpha"] \ar[d,>->, swap,"\alpha"]  & A \ar[d, >->,"1_A"] & O
			\ar[d,>->] \ar[l, {Circle[open]}->]\\
			A  \ar[r,>->,"1_A"]  &A \ar[dr,phantom,"\circlearrowleft"] & O \ar[l, {Circle[open]}->] \\
			O \ar[u, {Circle[open]}->] \ar[r,>->]& O \ar[u, {Circle[open]}->] & O  \ar[u, {Circle[open]}->] \ar[l, {Circle[open]}->]
			\end{tikzcd} \qquad\text{,}\qquad  \begin{tikzcd}
			A\ar[r, >->,"\beta"] 
			\ar[d,>->,swap,"\alpha"] & A \ar[d, >->,"1_A"] & O
			\ar[d,>->] \ar[l, {Circle[open]}->]\\
			A  \ar[r,>->,"\beta\alpha^{-1}"]  &A \ar[dr,phantom,"\circlearrowleft"]  & O \ar[l, {Circle[open]}->] \\
			O \ar[u, {Circle[open]}->] \ar[r,>->]& O \ar[u, {Circle[open]}->] & O  \ar[u, {Circle[open]}->] \ar[l, {Circle[open]}->]
			\end{tikzcd} \right)}		
		\end{equation*}
		yields the relation $\lrangles{l(\alpha,\beta)}=\lrangles{l(\beta\alpha^{-1})}.$
		 \begin{comment} This too, is also a special case of Relation (A2), but we've written out the $\tx$ diagram for clarity. For the choices of $\M$-morphisms, pick
		 		$$v:=1_A,\,v':= \beta\alpha^{-1},\qquad u=1_A,u'=\alpha\beta^{-1} \qquad \text{and}\quad w,w':=1_A.$$
The (P1) - (P4) diagrams are easy to check, they all have $A\rtail A\rtail A$ as the top filtration, and O's everywhere else.		 		
		\end{comment}
 By item (ii), deduce
		 $$\lrangles{l(\beta\alpha^{-1})} = \lrangles{l(\beta)}+\lrangles{l(\alpha^{-1})}\quad\text{and}\quad \lrangles{l(
	\alpha^{-1})}=-\lrangles{l(\alpha)}.$$
This assembles to give $\lrangles{l(\alpha,\beta)}=\lrangles{l(\beta\alpha^{-1})}=\lrangles{l(\beta)}-\lrangles{l(\alpha)}$, as desired. 
	\end{enumerate}	
\end{proof}

\begin{discussion}[Generalising stratifiability] Any birational equivalence defines a 1-simplex in the $G$-construction, but not necessarily an element of $K_1(\Var_k)$. For this to happen, the associated 1-simplex must lie in the base-point component [since $K_1(\Var_k)=\pi_1|G\Var_k^o|$]. More concretely, a birational map $f\colon X\dashrightarrow Y$ defines an element in $K_1$ just in case 
	$$[X]=[Y]\quad \text{in}\, K_0(\Var_k).$$
It is clear stratified birational equivalences satisfy this condition, but not every birational equivalence does -- e.g. the birational equivalence $\mathbb{P}^2\dashrightarrow\mathrm{Bl}_{[0:0:1]}\mathbb{P}^2$.\footnote{To be safe, assume the base field $k$ has characteristic 0, but the example should work more generally. The argument is straightforward once we know dimension is an invariant of $K_0(\Var_k)$ [and so $[\ast]\neq [\mathbb{A}^1]$], but there does not seem to be a published reference showing this for arbitrary fields. Nonetheless, see this argument \cite{SawinField} by Will Sawin.}
\begin{comment} For the case where the base field has characteristic 0, Corollary 5 of Liu-Sebag shows dimension is an invariant.
\end{comment}
\begin{comment} Simplicially, the 1-simplex $(V,V')\to (X,Y)$ corresponding to the birational equivalence $f$ belongs to $G\calC^o$ just in case there exists a 1-simplex connecting to $(V,V')$ or $(X,Y)$. But as long as there is a 1-simplex connecting to either the source or target, then the whole 1-simplex is connected to $G\calC^o$. This is obvious, but what it means geometrically is that we could have defined a ``weakly stratified birational equivalence'' as a birational map $f\colon X\dashrightarrow Y$  whereby
$$[X\setminus U]=[Y\setminus f(U)]\quad \text{in}\, K_0(\Var_k),$$
for isomorphic opens $U\subseteq X$ and $f(U)\subseteq Y$. This is equivalent to the above since we already know $[U]=[f(U)]$, and so $[X\setminus U]=[Y\setminus f(U)]$ iff $[U]+[X\setminus U]=[Y\setminus f(U)] + [f(U)]$ iff $[X]=[Y]$. 
\end{comment}
\end{discussion}
 

\subsection{Assembler Relations}\label{sec:Ass} We now apply Theorem~\ref{thm:NenSurj} to compare our relations with Zakharevich's $K_1$ of an Assembler (Proposition~\ref{prop:ass}). As a corollary, we prove that $\calD(\calC)\cong K_1(\calC)$, and so $\calD(\calC)$ gives an alternative presentation of $K_1(\calC)$ for pCGW categories (Corollary~\ref{cor:Nen}).
 
We start with an informal overview. An {\em Assembler} is a Grothendieck site $\calA$ whose topology encodes how an object $A$ may be covered by a finite set of disjoint subobjects $\{A_i\}_{i\in I}$. To define the $K$-theory of an Assembler $\calA$, one constructs its associated {\em category of covers} $\calW(\calA)$, defined as follows:
\begin{itemize}
	\item[] \textbf{Objects:} Finite sets of objects $\{A_i\}_{i\in I}$ in $\calA$;
	\item[] \textbf{Morphisms:} Piecewise automorphisms in $\calA$. Explicitly, a morphism  $f\colon \{A_i\}_{i\in I}\to \{B_j\}_{j\in J}$ in $\calW(\calA)$ is a tuple of morphisms $f_i\colon A_i\to B_{f(i)}$ such that $\{f_i\colon A_i\to B_j\}_{i\in f^{-1}(j)}$ is a finite disjoint covering family.
\end{itemize}
The $K$-theory spectrum $K(\calA)$ is then defined as the symmetric spectrum of the $\Gamma$-space induced by $\calW(\calA)$ -- see \cite[\S 2]{ZakhAss} for details. Its relevance is that it gives an alternative construction of $K\Var_k$, which is equivalent to the CGW construction \cite[Theorems 7.8 and 9.1]{CGW}. Further, extending Muro-Tonks' model of $K_1$ of a Waldhausen Category \cite{MuroTonks}, Zakharevich gave the following $K_1$ presentation.

\begin{theorem}[{{\cite[Theorem B]{ZakhK1}}}]\label{thm:ZakhB} Let $\calA$ be an Assembler whose morphisms are closed under pullback. Then $K_1(\calA)$ is generated by a pair of morphisms 
	$$A\xrightrightarrows[g]{f} B$$
	in $\calW(\calA)$. These satisfy the relations
	\begin{enumerate}[label=(Z\arabic*)]
		\item $\lrangles{A\xrightrightarrows[f]{f} B}=0$;
		\item $\lrangles{A\xrightrightarrows[f_2]{f_1} B} + \lrangles{C\xrightrightarrows[g_2]{g_1} D}=\lrangles{A\coprod C\xrightrightarrows[f_2\coprod g_2]{f_1\coprod g_1} B\coprod D}$;
		\item $\lrangles{B\xrightrightarrows[g_2]{g_1} C} +\lrangles{A\xrightrightarrows[f_2]{f_1} B}=\lrangles{A\xrightrightarrows[g_2f_2]{g_1f_1} B}$.
	\end{enumerate}
\end{theorem}

A couple remarks are in order. First, \cite[Theorem B]{ZakhK1} leaves open the possibility that there may be more relations on $K_1$ to be identified. This incompleteness is inherited from Muro-Tonks' original model of $K_1$: although \cite[Prop 6.3]{MuroTonks} shows that their model coincides with Nenashev's model for exact categories (and is thus complete), they were unable to show the same for all Waldhausen categories. Second, Relation (Z1) clearly corresponds to the diagonal relation (N1) of $\calD(\calC)$. It remains to investigate Relations (Z2) and (Z3) in our context, which we work out below.

\begin{proposition}[Assembler Relations]\label{prop:ass} Let $f$ and $g$ be two double exact squares
		\begin{equation*}\label{eq:assf}
f:= \left(	\dsquaref{O}{X}{A}{B}{ }{ }{ f_{1}}{f_{2} } \quad ,\quad 	\dsquaref{O}{X}{A}{B}{ }{ }{ f'_{1}}{f'_{2} }\right)  \qquad\quad g:= \left(	\dsquaref{O}{Y}{C}{D}{ }{ }{ g_{1}}{g_{2} } \quad ,\quad 	\dsquaref{O}{Y}{C}{D}{ }{ }{ g'_{1}}{g'_{2} }\right). 
	\end{equation*}
Then the following relations hold in $\calD(\calC)$:	
\begin{enumerate}[label=(A\arabic*)]
	\item \emph{(Formal Direct Sums).} $\lrangles{f}+\lrangles{g}=\lrangles{f\oplus g}$.
\begin{comment}
	, where
	\begin{equation}
f\oplus g:= \left(	\dsquaref{O}{\frac{B}{A}\oplus \frac{D}{C}}{A\oplus C}{B\oplus D}{ }{ }{ f_{1}\oplus g_1 }{f_{2} \oplus g_2} \quad ,\quad 		\dsquaref{O}{\frac{B}{A}\oplus \frac{D}{C}}{A\oplus C}{B\oplus D}{ }{ }{ f'_{1}\oplus g'_1 }{f'_{2} \oplus g'_2}  \right)  	\end{equation} 
\end{comment}	
\item \emph{(Restricted Composition).} Suppose $B=C$, and so $f$ and $g$ define an admissible triple. Then
$$\lrangles{f} + \lrangles{g}=\lrangles{g\circ f} + \lrangles{l_2},$$ 
where $l_2$ is the associated double exact square 
\[l_2:=\left( \dsquaref{O}{Y}{X}{\frac{D}{A}}{ }{ }{h_1}{j_1} \quad,\quad  \dsquaref{O}{Y}{X}{\frac{D}{A}}{ }{}{h'_1}{j'_1}\right), \qquad \text{and} \qquad \lrangles{g\circ f}:= \left(	\dsquaref{O}{\frac{D}{A}}{A}{D}{ }{ }{ g_{1}f_1}{ } \quad ,\quad 	\dsquaref{O}{\frac{D}{A}}{A}{D}{ }{ }{ g'_{1}f'_1}{}\right).\]
\end{enumerate}	
\end{proposition}
\begin{proof} We derive Relations (A1) and (A2) from Relations (N1) and (N2).
\begin{itemize}
	\item[(A1):] We claim the following is an optimal $\tx$ diagram:
	\begin{equation}\label{eq:asscoprod}
	\left(\begin{tikzcd}
	A \ar[r, >->, "f_{1}"] 
	\ar[d,>->]  & B \ar[d, >->] & X
	\ar[d,>->] \ar[l, {Circle[open]}->,swap,"f_{2}"]\\
	A\oplus C \ar[r,>->,"f_{1}\oplus g_1"]  &	B\oplus D \ar[dr,phantom,"\circlearrowleft"]  & X\oplus Y\ar[l, {Circle[open]}->,swap,"f_2\oplus g_2"] \\
	C \ar[u, {Circle[open]}->] \ar[r,>->,"g_1"]& D\ar[u, {Circle[open]}-> ] & Y \ar[u, {Circle[open]}->,swap]\ar[l, {Circle[open]}->,swap,"g_{2}"]
	\end{tikzcd} \qquad\text{,}\qquad  \begin{tikzcd}
	A \ar[r, >->, "f'_{1}"] 
	\ar[d,>->]  & B \ar[d, >->] & X
	\ar[d,>->] \ar[l, {Circle[open]}->,swap,"f'_{2}"]\\
	A\oplus C \ar[r,>->,"f'_{1}\oplus g'_1"]  &	B\oplus D  \ar[dr,phantom,"\circlearrowleft"] & X\oplus Y \ar[l, {Circle[open]}->,swap,"f'_2\oplus g'_2"] \\
	C \ar[u, {Circle[open]}->] \ar[r,>->,"g'_1"]& D \ar[u, {Circle[open]}-> ]& Y \ar[u, {Circle[open]}->] \ar[l, {Circle[open]}->,swap,"g'_{2}"]
	\end{tikzcd}\right)
	\end{equation}		
	The vertical columns -- denoted $l^0,l^1,l^2$ -- define the direct sum squares from Axiom (A), Definition~\ref{def:pCGW}. The top and bottom rows correspond to $f$ and $g$. The middle rows correspond to $f\oplus g$, as defined in Lemma~\ref{lem:DirectSum}. To define the required $\M$-morphisms for optimality, take repeated restricted pushouts of 
	$$ D\xltail{g_1} C\ltail O \rtail A\xrtail{f_1} B \quad \text{and}\quad  D\xltail{g'_1} C\ltail O \rtail A\xrtail{f'_1} B;$$
the obvious choices of $\M$-morphisms are easily seen to satisfy Optimality Identities~\eqref{eq:OPTID}. Lemma~\ref{lem:DirectSum} then  supplies Diagrams (P1) – (P4) by “adding” the obvious filtrations. Finally, to see why the $\E$-squares commute, use the fact that the direct sum quotients are defined coordinate-wise, that is $f_2\oplus g_2:=(1\oplus g_2)\circ (f_2\oplus 1)$ and $f'_2\oplus g'_2:=(1\oplus g'_2)\circ (f'_2\oplus 1)$.\footnote{{\em Details.} The validity of these quotient choices is established in the proof of Lemma~\ref{lem:DirectSum}, Step~2. Using this, one verifies that the $\E$-squares are quotients of the same $\M$-morphism and hence commute up to canonical isomorphism on $Y$. By Axiom (PQ), this isomorphism lifts to $X\oplus Y$ and thus may be absorbed into the choice of direct sum quotients.} 

	\begin{comment} Details on the $\E$-square.
	By Step 2 of the Direct Sum Lemma, we define
	$$f_2\oplus g_2:=(1\oplus g_2)\circ (f_2\oplus 1)$$
	
	Reproducing the diagram from Step 2, and adjusting the notation we get the LHS below. 
	\[\begin{tikzcd}
O \ar[r,>->] \ar[dr,phantom,"\square"] \ar[d,{Circle[open]->}]& Y \ar[d,{Circle[open]->},"q^X_Y"] \ar[r,>->,"\gamma"]\ar[dddr, phantom,"\square"] & Y \ar[ddd,{Circle[open]->},"q^B_D\circ g_2"]	 \\
X \ar[dr,phantom,"\square"] \ar[d,{Circle[open]->},"f_2",swap]\ar[r,>->] & X\oplus Y   \ar[d,{Circle[open]->},"f_2\oplus 1",swap]   \\
  B\ar[d,{Circle[open]->}] \ar[r,>->] \ar[dr,phantom,"\square"]  & B\oplus Y \ar[d,{Circle[open]->},"1\oplus g_2",swap]   \\
B\oplus C \ar[r,>->,swap,"1\oplus g_1"] & B\oplus D  \ar[r,>->,"1"]& B\oplus D
	\end{tikzcd}\qquad \qquad	\begin{tikzcd}
	O \ar[r,>->] \ar[dr,phantom,"\square"] \ar[d,{Circle[open]->}]& Y \ar[d,{Circle[open]->},"g_2"]	\\
	C \ar[dr,phantom,"\square"] \ar[d,{Circle[open]->}]\ar[r,>->] & D  \ar[d,{Circle[open]->},"q^B_D"]   \\ 
	B\oplus C \ar[r,>->,swap,"1\oplus g_1"] & B\oplus D 
	\end{tikzcd}\]
The LHS diagram gives an exact square for $f_2\oplus g_2\circ q^X_Y$ and the RHS diagram\footnote{The bottom square is distinguished by same logic as in the proof of Lemma~\ref{lem:DirectSum}, Step 1.} gives one for $q_D\circ g_2$. This is where we use the coordinate-wise definition for $f_2\ oplus g_2$; it allows us to see that the $\E$-morphisms in the $\E$-square are quotients of the same $\M$-morphism, and are thus related by a codomain-preserving isomorphism $\gamma$. 

The goal is to show that this isomorphism can be absorbed into the choice of $f_2\oplus 1$.
\begin{itemize}
\item \textbf{Case 1:} Suppose $\E\subseteq\calC$. In which case, the identity becomes
$$(f_2\oplus g_2)\circ q^X_Y\circ \gamma^{-1} = q^B_D\circ g_2$$


take restricted pushouts:
\[\begin{tikzcd}
O \ar[r,>->] \ar[d,>->] & Y \ar[r,>->,"\gamma^{-1}"] \ar[d,>->,"p^X_Y"]& Y \ar[d,>->] & O \ar[l,{Circle[open]->}] \ar[d,"\cong"]\\
X \ar[r,>->]& X\oplus Y \ar[r,>->,"v_\gamma"] & X\oplus Y & O \ar[l,{Circle[open]->}]\\
& Y \ar[u,{Circle[open]->},"q^X_Y"] \ar[ur,phantom,"\circlearrowleft"]\ar[r,>->,swap,"\cong"] & Y \ar[u,{Circle[open]->},"q^X_Y",swap] 
\end{tikzcd} \]
Since $\gamma^{-1}$ is an isomorphism, its quotient is $O$, and thus by Axiom (PQ), so is $v_\gamma$. Since $\E\subseteq \calC$, we know $p^X_Y=q^X_Y$, and so $$q^X_Y\circ\gamma^{-1}=p^X_Y\circ \gamma^{-1} = v_\gamma\circ p^X_Y = v_\gamma\circ q^X_Y.$$ Hence, WLOG, we can pick $f_2\oplus g_2\circ v_\gamma$ as our choice of quotient for $f_1\oplus g_1$, where we have modified by $v_\gamma\colon X\oplus Y\to X\oplus Y$. And so that settles this case. 
\item \textbf{Case 2:} Suppose $\E^\opp\subseteq \calC$. Then the arrows reverse. In which case, the identity becomes
$$\gamma\circ q^X_Y\circ (f_2\oplus g_2)\circ q^X_Y = q^B_D\circ g_2$$
Play the same game as before to construct
\[\begin{tikzcd}
O \ar[r,>->] \ar[d,>->] & Y \ar[r,>->,"\gamma"] \ar[d,>->,"p^X_Y"]& Y \ar[d,>->] & O \ar[l,{Circle[open]->}] \ar[d,"\cong"]\\
X \ar[r,>->]& X\oplus Y \ar[d, "q^X_Y"]\ar[r,>->,"v_\gamma"] & X\oplus Y \ar[d, "q^X_Y"]& O \ar[l,{Circle[open]->}]\\
& Y \ar[ur,phantom,"\circlearrowleft"]\ar[r,>->,swap,"\cong"] & Y 
\end{tikzcd} \]

In addition, $p^X_Y\circ q^X_Y=1$, and so the bottom isomorphism is also $\gamma$, and so
$$\gamma\circ q^X_Y = q^X_Y\circ v_\gamma.$$
And so, as before, we can modify the quotient choice of $f_1\oplus g_1$ to get the $\E$-square to commute. 
\end{itemize}
	\end{comment}

	\begin{comment}	
Set $Z=Z'=B\oplus C$.
	
	Below is the diagram constructed via repeated restricted pushouts,.
	\begin{equation}
	\begin{tikzcd}
	O \ar[r,>->] \ar[d,>->] & A \ar[r,>->,"f_1"] \ar[d,>->] & B \ar[d,>->,"u"]\\
	C \ar[r,>->] \ar[d,>->,"g_1"] & A\oplus C  \ar[d,>->] \ar[r,>->,"w"] &  B\oplus C \ar[d,>->,"\varv"]\\
	D\ar[r,>->]& A\oplus D \ar[r,>->]& B\oplus D
	\end{tikzcd} \qquad 		\begin{tikzcd}
	O \ar[r,>->] \ar[d,>->] & A \ar[r,>->,"f'_1"] \ar[d,>->] & B \ar[d,>->,"u'"]\\
	C \ar[r,>->] \ar[d,>->,"g'_1"] & A\oplus C  \ar[d,>->] \ar[r,>->,"w'"] &  B\oplus C \ar[d,>->,"\varv'"]\\
	D\ar[r,>->]& A\oplus D \ar[r,>->]& B\oplus D
	\end{tikzcd}.
	\end{equation}
	
	Below are the (P1) - (P4) diagrams.
	\begin{enumerate}[label=(P\arabic*):]
	\item \qquad \begin{tikzcd}
	B \ar[r, >->,"1\oplus C"] \ar[dr,phantom,"\square"] & B\oplus C \ar[dr,phantom,"\square"] \ar[r, >->,"1\oplus g_1"] & B\oplus D \\
	O \ar[r,>->] \ar[u,{Circle[open]}->]&	C \ar[r, >->,"g_1"] \ar[u, {Circle[open]}->] \ar[dr,phantom,"\square"]  & D\ar[u, {Circle[open]}->] \\
	& O \ar[r,>->] \ar[u,{Circle[open]}->]& \frac{D}{C} \ar[u, {Circle[open]}->,"g_2"] 
	\end{tikzcd}	 \quad 	\begin{tikzcd}
	B \ar[r, >->,"1\oplus C"] \ar[dr,phantom,"\square"] & B\oplus C \ar[dr,phantom,"\square"] \ar[r, >->,"1\oplus g'_1"] & B\oplus D \\
	O \ar[r,>->] \ar[u,{Circle[open]}->]&	C \ar[r, >->,"g'_1"] \ar[u, {Circle[open]}->] \ar[dr,phantom,"\square"]  & D\ar[u, {Circle[open]}->] \\
	& O \ar[r,>->] \ar[u,{Circle[open]}->]& \frac{D}{C} \ar[u, {Circle[open]}->,"g'_2"] 
	\end{tikzcd}	
	%Add $B\rtail B\rtail B$ with $O\rtail C\rtail D$
	\[\]
	\item \quad \begin{tikzcd}
	A\oplus C\ar[r, >->,"f_1\oplus 1"]\ar[dr,phantom,"\square"]  & B\oplus C \ar[dr,phantom,"\square"] \ar[r, >->,"1\oplus g_1"] & B\oplus D \\
	O \ar[r,>->] \ar[u,{Circle[open]}->]&\frac{B}{A} \ar[dr,phantom,"\square"]  \ar[r, >->] \ar[u, {Circle[open]}->,"f_2\oplus C"]& \frac{B}{A}\oplus \frac{D}{C}\ar[u, {Circle[open]}->] \\
	&O \ar[r,>->] \ar[u,{Circle[open]}->]& \frac{D}{C}\ar[u, {Circle[open]}->] 
	\end{tikzcd}	 \quad 		\begin{tikzcd}
	A\oplus C\ar[r, >->,"f'_1\oplus 1"]\ar[dr,phantom,"\square"]  & B\oplus C \ar[dr,phantom,"\square"] \ar[r, >->,"1\oplus g'_1"] & B\oplus D \\
	O \ar[r,>->] \ar[u,{Circle[open]}->]&\frac{B}{A} \ar[dr,phantom,"\square"]  \ar[r, >->] \ar[u, {Circle[open]}->,"f'_2\oplus C"]& \frac{B}{A}\oplus \frac{D}{C}\ar[u, {Circle[open]}->] \\
	&O \ar[r,>->] \ar[u,{Circle[open]}->]& \frac{D}{C}\ar[u, {Circle[open]}->] 
	\end{tikzcd}
	\[\]	
	%[Suppose we have filtrations $A\rtail B\rtail B$ and $C\rtail C\rtail D$, each with the obvious filtration. Lemma~\ref{lem:DirectSum} (i) and (ii) tells us about the tells us how to add them. The notation of the left squares comes from Lemma~\ref{lem:ADTsquares}]
	\item \qquad \qquad \begin{tikzcd}
	A\ar[r, >->,"f_1"] & B \ar[dr,phantom,"\square"] \ar[r, >->,"1\oplus C"] & B\oplus C \\
	&	\frac{B}{A} \ar[r, >->,"1\oplus C"] \ar[u, {Circle[open]}->,"f_2"]& \frac{B}{A}\oplus C\ar[u, {Circle[open]}->,swap,"f_2\oplus C"] \\
	&& C \ar[u, {Circle[open]}->] 
	\end{tikzcd}	 \quad 	 	\begin{tikzcd}
	A\ar[r, >->,"f'_1"] & B \ar[dr,phantom,"\square"] \ar[r, >->,"1\oplus C"] & B\oplus C \\
	&	\frac{B}{A} \ar[r, >->] \ar[u, {Circle[open]}->,"f'_2"]& \frac{B}{A}\oplus C\ar[u, {Circle[open]}->,swap,"f'_2\oplus C"] \\
	&& C \ar[u, {Circle[open]}->] 
	\end{tikzcd}	
	\[\]
	%	[The whole diagram is constructed by the proof of  Lemma~\ref{lem:ADTsquares} (i).]
	\item \qquad \begin{tikzcd}
	A \ar[r, >->] & A\oplus C \ar[dr,phantom,"\square"] \ar[r, >->,"f_1\oplus C"] & B\oplus C\\
	& C \ar[r, >->] \ar[u, {Circle[open]}->]& \frac{B}{A}\oplus C \ar[u, {Circle[open]}->] \\
	&& \frac{B}{A}\ar[u, {Circle[open]}->,] 
	\end{tikzcd}	 \quad 			\begin{tikzcd}
	A \ar[r, >->] & A\oplus C \ar[dr,phantom,"\square"] \ar[r, >->,"f'_1\oplus C"] & B\oplus C\\
	& C \ar[r, >->] \ar[u, {Circle[open]}->]& \frac{B}{A}\oplus C \ar[u, {Circle[open]}->] \\
	&& \frac{B}{A}\ar[u, {Circle[open]}->,] 
	\end{tikzcd}
	%	[The whole diagram is constructed by the proof of  Lemma~\ref{lem:ADTsquares} (i).]
	\end{enumerate}	
	
	
	The $\M$-morphisms satisfy the optimality identities, essentially by their construction in Lemma~\ref{lem:DirectSum}. Diagrams (P1) and (P2) imply that the $\E$-squares commute. 
	
	\end{comment}	
	
	This sets up the final move. Since Diagram~\eqref{eq:asscoprod} is optimal, apply Relation (N2) to get 
	\[\lrangles{f}+\lrangles{g}-\lrangles{f\oplus g} = \lrangles{l^0} + \lrangles{l^2} - \lrangles{l^1}.\]
	Since $l^0, l^1$ and $l^2$ are all diagonal, deduce from (N1) that $\lrangles{l^0} = \lrangles{l^1} =\lrangles{l^2} = 0$, and so conclude 
	\[\lrangles{f}+\lrangles{g}=\lrangles{f\oplus g}. \]
	\item[(A2):] Given that $B=C$, construct the following diagram
	\begin{equation}\label{eq:asscompose}
	\left(\begin{tikzcd}
	A \ar[r, >->, "="] 
	\ar[d,>->,swap,"f_1"]  & A \ar[d, >->,"g_1f_1"] & O
	\ar[d,>->] \ar[l, {Circle[open]}->]\\ B \ar[r,>->,"g_1"]  &	D \ar[dr,phantom,"\circlearrowleft"]  & Y \ar[l, {Circle[open]}->,swap,"g_2"] \\
X \ar[u, {Circle[open]}->,"f_2"] \ar[r,>->,"h_1"]& \frac{D}{A} \ar[u, {Circle[open]}->]& Y \ar[u, {Circle[open]}->,swap,"="] \ar[l, {Circle[open]}->,swap,"j_1"]
	\end{tikzcd} \qquad\text{,}\qquad  \begin{tikzcd}
	A \ar[r, >->, "="] 
	\ar[d,>->,swap,"f'_1"]  & A \ar[d, >->,"g'_1f'_1"] & O
	\ar[d,>->] \ar[l, {Circle[open]}->]\\ B \ar[r,>->,"g'_1"]  &	D \ar[dr,phantom,"\circlearrowleft"]  & Y \ar[l, {Circle[open]}->,swap,"g'_2"] \\
X \ar[u, {Circle[open]}->,"f'_2"] \ar[r,>->,"h'_1"]& \frac{D}{A} \ar[u, {Circle[open]}->]& Y \ar[u, {Circle[open]}->,swap,"="] \ar[l, {Circle[open]}->,swap,"j'_1"]
	\end{tikzcd}  \right).
	\end{equation}		
As before, label the horizontal rows as $l_0$, $l_1$ and $l_2$ and the vertical columns as $l^0, l^1$ and $l^2$, where $l_1=g$, $l^0=f$ and $l^1=g\circ f$. It is obvious most of these define double exact squares; the only non-trivial case is $l_2$, but this is directly constructed by Lemma~\ref{lem:quotFilt}. In fact, the $\E$-morphisms $(j_1,j'_1)$ are specifically chosen by Lemma~\ref{lem:quotFilt} to ensure that the $\E$-squares commute, and so Diagram~\eqref{eq:asscompose} is a $\tx$ diagram. Finally, to check optimality, pick the following pairs of $\M$-morphisms:
	$$v:=g_1,\,v':= g'_1,\qquad u:=f_1,\,u':=f'_1 \qquad \text{and}\quad w,\, w':=1_B.$$ 
	The Optimality Identities are trivially satisfied; it is also easy to construct the required Diagrams (P1) – (P4). 
	\begin{comment} If we examine the proof of Lemma~\ref{lem:quotFilt}, one realises it makes use of the $k^{-1}=c$ argument in the original CGW paper, but strictly speaking $k$ and $c$ are only equivalent up to codomain-preserving isomorphism. So why do we get that the $\E$-squares actually commute here, and not just up to codomain-preserving isomorphism?
	
	 Answer: we may assume WLOG that this i were absorbed into the choice of $j_1$.
	 
	\end{comment}
	\begin{comment} Details on optimality.
	
	\begin{enumerate}[label=(P\arabic*):]
	\item \[\begin{tikzcd}
	A \ar[r, >->,"f_1"] \ar[dr,phantom,"\square"] & B  \ar[dr,phantom,"\square"] \ar[r, >->,"g_1"] & D \\
	O \ar[r,>->] \ar[u,{Circle[open]}->]&	\frac{B}{A} \ar[r, >->,"h_1"] \ar[u, {Circle[open]}->] \ar[dr,phantom,"\square"]  & \frac{D}{A}\ar[u, {Circle[open]}->] \\
	& O \ar[r,>->] \ar[u,{Circle[open]}->]& \frac{D}{B} \ar[u, {Circle[open]}->,"j_1"] 
	\end{tikzcd}	 \quad 	\begin{tikzcd}
	A \ar[r, >->,"f'_1"] \ar[dr,phantom,"\square"] & B  \ar[dr,phantom,"\square"] \ar[r, >->,"g'_1"] & D \\
	O \ar[r,>->] \ar[u,{Circle[open]}->]&	\frac{B}{A} \ar[r, >->,"h'_1"] \ar[u, {Circle[open]}->] \ar[dr,phantom,"\square"]  & \frac{D}{A}\ar[u, {Circle[open]}->] \\
	& O \ar[r,>->] \ar[u,{Circle[open]}->]& \frac{D}{B} \ar[u, {Circle[open]}->,"j'_1"] 
	\end{tikzcd}
	\]
	The fact that the indicated squares are distinguished follows from Lemma~\ref{lem:quotFilt}.
	\item 
	\[\begin{tikzcd}
	B\ar[r, >->]\ar[dr,phantom,"\square"]  & B\ar[dr,phantom,"\square"] \ar[r, >->,"g_1"] &  D \\
	O \ar[r,>->] \ar[u,{Circle[open]}->]& O \ar[dr,phantom,"\square"]  \ar[r, >->] \ar[u, {Circle[open]}->]& \frac{D}{B} \ar[u, {Circle[open]}->] \\
	&O \ar[r,>->] \ar[u,{Circle[open]}->]& \frac{D}{B}\ar[u, {Circle[open]}->] 
	\end{tikzcd}	 \quad 		\begin{tikzcd}
	B\ar[r, >->]\ar[dr,phantom,"\square"]  & B\ar[dr,phantom,"\square"] \ar[r, >->,"g'_1"] &  D \\
	O \ar[r,>->] \ar[u,{Circle[open]}->]& O \ar[dr,phantom,"\square"]  \ar[r, >->] \ar[u, {Circle[open]}->]& \frac{D}{B} \ar[u, {Circle[open]}->] \\
	&O \ar[r,>->] \ar[u,{Circle[open]}->]& \frac{D}{B}\ar[u, {Circle[open]}->] 
	\end{tikzcd}
	\]	
	
	\item \[
	\begin{tikzcd}
	A\ar[r, >->,"="] & A \ar[dr,phantom,"\square"] \ar[r, >->,"f_1"] & B  \\
	&	O \ar[r, >->] \ar[u, {Circle[open]}->]& \frac{B}{A}\ar[u, {Circle[open]}->] \\
	&& \frac{B}{A} \ar[u, {Circle[open]}->] 
	\end{tikzcd}	 \quad 	 \begin{tikzcd}
	A\ar[r, >->,"="] & A \ar[dr,phantom,"\square"] \ar[r, >->,"f'_1"] & B  \\
	&	O \ar[r, >->] \ar[u, {Circle[open]}->]& \frac{B}{A}\ar[u, {Circle[open]}->] \\
	&& \frac{B}{A} \ar[u, {Circle[open]}->] 
	\end{tikzcd}
	\]
	
	\item \[
	\begin{tikzcd}
	A \ar[r, >->,"f_1"] & B \ar[dr,phantom,"\square"] \ar[r, >->] & B\\
	& \frac{B}{A} \ar[r, >->] \ar[u, {Circle[open]}->]& \frac{B}{A}  \ar[u, {Circle[open]}->] \\
	&& O\ar[u, {Circle[open]}->,] 
	\end{tikzcd}	 \quad 			\begin{tikzcd}
	A \ar[r, >->,"f'_1"] & B \ar[dr,phantom,"\square"] \ar[r, >->] & B\\
	& \frac{B}{A} \ar[r, >->] \ar[u, {Circle[open]}->]& \frac{B}{A}  \ar[u, {Circle[open]}->] \\
	&& O\ar[u, {Circle[open]}->,] 
	\end{tikzcd}	
	\]  
	\end{enumerate}
	\end{comment}
	Now apply Relation (N2) to get 
	\[\lrangles{l_0}+\lrangles{l_2}-\lrangles{g} = \lrangles{f} + \lrangles{l^2} - \lrangles{g\circ f}.\]
	Since $l_0$ and $l^2$ are diagonal, apply (N1) to conclude
	\[\lrangles{g\circ f} + \lrangles{l_2}= \lrangles{f} + \lrangles{g}.\] 
\end{itemize}
\end{proof}

Relation (A2) improves upon Proposition~\ref{prop:baseline} by determining what happens to {\em any} admissible triple in $K_1$, clarifying Warning~\ref{warning:compose}. Namely, given an admissible triple $\calT=(f,g,g\circ f)$, we now know that
$$\lrangles{g\circ f} + \lrangles{l_2} = \lrangles{f}+\lrangles{g} \qquad \text{in}\, K_1.$$
Unlike Theorem~\ref{thm:ZakhB}, where composition always splits in $K_1$, here an obstruction term $\lrangles{l_2}$ appears. Of course, if $\lrangles{l_2}=0$ then Relations (A2) and (Z3) coincide, but this is not true in general, as below.  

\begin{example}\label{ex:KeyExample} Given any automorphism $\alpha\colon B\rtail B$, construct the diagram pair
	\[\begin{tikzcd}
	A \ar[r, >->] \ar[dr,phantom,"\square"] & A\oplus B  \ar[dr,phantom,"\square"] \ar[r, >->,"1_A\oplus 1_B"] & A\oplus B \\
	O \ar[r,>->] \ar[u,{Circle[open]}->]&	B \ar[r, >->,"1_B"] \ar[u, {Circle[open]}->] \ar[dr,phantom,"\square"]  & B \ar[u, {Circle[open]}->] \\
	& O \ar[r,>->] \ar[u,{Circle[open]}->]& O \ar[u, {Circle[open]}->] 
	\end{tikzcd}  \quad \begin{tikzcd}
	A \ar[r, >->] \ar[dr,phantom,"\square"] & A\oplus B  \ar[dr,phantom,"\square"] \ar[r, >->,"1_A\oplus \alpha "] & A\oplus B \\
	O \ar[r,>->] \ar[u,{Circle[open]}->]&	B \ar[r, >->,"\alpha"] \ar[u, {Circle[open]}->] \ar[dr,phantom,"\square"]  & B \ar[u, {Circle[open]}->] \\
	& O \ar[r,>->] \ar[u,{Circle[open]}->]& O \ar[u, {Circle[open]}->] 
	\end{tikzcd}	\qquad.
	\]	
Notice the obstruction term here is $\lrangles{l_2}=l(\alpha)$, the double exact square associated to $\alpha$ (Example~\ref{ex:Aut}). It is known that $l(\alpha)\neq 0$ in general, including the case of varieties -- see e.g. \cite{ZakhPtCount}.  
\end{example}

\begin{discussion}\label{dis:Comp} We suspect this discrepancy with Zakharevich's account stems from \cite[Theorem 2.1]{ZakhK1}, a key ingredient in Zakharevich's proof of Theorem~\ref{thm:ZakhB}, which models the $K$-theory of (nice) Assemblers using Waldhausen categories whose cofibration sequences all split (up to weak equivalence). We emphasise that this relies on a non-standard notion of weak equivalence \cite[Def. 1.7]{ZakhK1}. In the CGW setting, it is clearly false that all exact squares split in $\Var_k$ -- consider e.g.	\[\dsquare{O}{\mathbb{A}^1}{\{\ast\} }{ \mathbb{P}^1}.\]
\end{discussion}

\begin{remark} Proposition~\ref{prop:ass} identifies certain relations of the abelian group $\calD(\calC)$, and so naturally the proof was algebraic. However, since we are ultimately interested in $K_1(\calC)=\pi_1|G\calC^o|$, one may also appeal to the homotopical perspective to establish the relations more directly. Relation (A1) can be deduced from the $H$-space structure of $G\calC$ and applying the standard Eckmann-Hilton argument; Relation (A2) translates Lemma~\ref{lem:admtriple}, though the lemma's proof is comparatively involved (see Section~\ref{sec:admtriples}).
\end{remark}

%\begin{remark} If it turns out \cite[Theorem 2.1]{ZakhK1} does not apply to $\Var_k$ (as discussed above), then this impacts the proof of \cite[Theorem 9.1]{CGW}, which shows that the $K$-theory spectrum of varieties via Assemblers and CGW categories are equivalent.
%\end{remark}

We end with one final application. In order to show that $\calD(\calC)\cong K_1(\calC)$ for exact categories, Nenashev \cite{Nen1} constructs a homomorphism
$$b\colon K_1(\calC)\to \calD(\calC)$$
and shows that it is inverse to the map $m\colon \calD(\calC)\to K_1(\calC)$ from Equation~\eqref{eq:NenSurj}. Notice the naive map sending $\lrangles{f}\mapsto \lrangles{f}$ is not \emph{a priori} well-defined since $K_1(\calC)$ may impose relations beyond those of $\calD(\calC)$. In \cite[\S 4]{Nen1}, well-definedness of $b$ is verified through substantial combinatorial bookkeeping, before the final proof that it yields the desired isomorphism.\footnote{{\em Commentary.} With hindsight, one can interpret Nenashev’s proof as explicitly deducing an Eckmann–Hilton result from Relations (N1) and (N2) of $K_1$; compare \cite[\S 4]{Nen1} with our Observation~\ref{obs:K1DirectSum}. Notice the only place where homotopical input appears is \cite[Lemma 4.9]{Nen1}, which verifies invariance under elementary 2-simplex (triangle) moves.} In our framework, Propositions~\ref{prop:baseline} and ~\ref{prop:ass} combine to give a more direct proof.

\begin{comment} The general shape of Nenashev's argument in Section 4 involves descending from free loops to loops up to homotopy in $\pi_1$. Free loops are just combinatorial loops, no homotopical considerations. Nenashev shows cyclic invariance of free loops (pp. 8-9), additivity (Prop. 4.1), and reveresal (Cor 4.4). In Lemmas 4.5-4.9, he shows that this respects elementary homotopy equivalence by showing that $b$ of a triangle bounding a 2-simplex vanishes. Thus $b\colon K_1\to \calD(\calC)$ is well-defined.
\end{comment}


\begin{corollary}\label{cor:Nen} Given any pCGW category $\calC$, there is an isomorphism
	$$\calD(\calC)\cong K_1(\calC).$$	
\end{corollary}
\begin{proof} It suffices to check that $\calD(\calC)$ implies the relations in Proposition~\ref{prop:baseline}, which we know to be complete. Relations (B1) and (B2) are diagonal relations, and follow immediately from Relation (N1). Relation (B3) is a special case of Relation (A2) in Proposition~\ref{prop:ass}, where $\lrangles{l_2}=0$ since $l_2$ is diagonal by assumption. 
\end{proof}

\begin{remark} In fact, the proof of Corollary~\ref{cor:Nen} shows $K_1(\calC)$ can be presented as the free abelian group generated by double exact squares, modulo the Diagonal Relation (N1) (Definition~\ref{def:NenDC}) and Restricted Composition Relation (A2) (Proposition~\ref{prop:ass})	
\end{remark}


\section{Some Test Problems}

This paper began with Question~\ref{qn:key}: what information do the higher $K$-groups of varieties encode? Progress on this question requires advances on two fronts: developing the framework of non-additive $K$-theory on the one hand, and concrete applications of the newly-developed ($K$-theory) tools on the other. This paper is of the first kind, with a view towards laying the groundwork for future theorems. We conclude with some test problems and discussion.

\subsection{Non-Additive $K$-theory}\label{sec:NAKtheory}

Having characterised $K_1$, the natural next step is the following problem.

\begin{problem} Characterise $K_n(\calC)$ for $n>1$ for pCGW categories. 
\end{problem}

\begin{discussion} In a substantial generalisation of Nenashev's work, Grayson gave a complete characterisation of $K_n$ for all $n$ in the setting of exact categories \cite{GraysonBinary}. Encouraged by the results in this paper, the natural proof strategy would be to extend Grayson's argument to the pCGW setting. 
	
	However, there is an obvious barrier. Grayson characterises the $K$-groups of exact categories via binary chain complexes, and so invokes the Gillet-Waldhausen Theorem. Although an analogue of this result has been shown for extensive categories (e.g. $\mathrm{FinSet}$) \cite{SarazolaShapiroGW}, a Gillet-Waldhausen Theorem for $\Var_k$ has not been worked out yet. Nonetheless, even the case of $\mathrm{FinSet}$ is interesting. By Barratt-Priddy-Quillen, the $K$-groups of $\mathrm{FinSet}$ correspond to the stable homotopy groups of spheres, whose complete description is a longstanding open problem in homotopy theory. What might this presentation of $K_n(\mathrm{Finset})$ tell us about the stable homotopy groups of spheres? About the $J$-homomorphism and Adams $e$-invariant?
\end{discussion}

In light of Proposition~\ref{prop:ass}, perhaps a more urgent problem is the following.

\begin{problem} Resolve the discrepancy between our presentation of $K_1(\Var_k)$ and Zakharevich's.
\end{problem}

%%%% Examine how distinguished squares get mapped via the equivalence.


\begin{discussion} Here is one way to make precise Discussion~\ref{dis:Comp} regarding the choice of weak equivalences. Our proof of Theorem~\ref{thm:Sherman} makes crucial use of Lemma~\ref{lem:SherLoopSplit}, which states: if $\calC$ is a pCGW category whose exact squares all split, then $K_1(\calC)$ is generated by automorphisms. In which case, by Corollary~\ref{cor:BirIso}, composition splits in $K_1$ with no obstruction term. In light of \cite[Theorem 2.1]{ZakhK1}, can we extend Lemma~\ref{lem:SherLoopSplit} to Waldhausen categories whose cofibration sequences all split? 
\end{discussion}

\begin{discussion} Example~\ref{ex:KeyExample} shows how the obstruction term may fail to vanish in $K_1(\Var_k)$. So what happens in the Assemblers setting?

Recall from \cite[\S 5.1]{ZakhAss} the assembler $\calV_k$: its objects are $k$-varieties, its morphisms locally closed immersions, and its Grothendieck topology is generated by coverages of the form $\{Y\to X,X\setminus Y\to X\}$ for closed embeddings $Y\to X$. In Zakharevich's framework, the generators of $K_1(\calV_k)$ are finite tuples of morphisms from $\calV_k$.

\begin{comment} Side-note: locally closed immersions are generally defined as $X\to Y\to Z$, where $X\to Y$ is a closed immersion, and $Y\to Z$ is open. In general, it is false that this is equivalent to an open immersion followed by a closed immersion. However, this is fine if $Z$ is reduced (which varieties are), see  Exercise 2D https://math.stanford.edu/~vakil/0708-216/216class13.pdf 

So we may regard locally closed immersions of varieties as finite compositions of open/closed immersions, as originally defined by Zakharevich. This perspective should be helpful when translating composition of Assemblers to the CGW setting. 
\end{comment}

We now translate Example~\ref{ex:KeyExample} to Assemblers (with coproducts). Take two multimorphisms $$f, f' \colon \{\{A\}, \{B\}\} \to A \oplus B$$
given by coproduct inclusions, followed by automorphisms $1 \oplus 1$ and $1 \oplus \alpha$. These correspond to the data of Example~\ref{ex:KeyExample} in a natural way, yet their composition yields a different admissible triple: 

	\[  \begin{tikzcd}
A \ar[r, >->] \ar[dr,phantom,"\square"] & A\oplus B  \ar[dr,phantom,"\square"] \ar[r, >->,"1_A\oplus 1_B"] & A\oplus B \\
O \ar[r,>->] \ar[u,{Circle[open]}->]&	B \ar[r, >->,"1_B"] \ar[u, {Circle[open]}->] \ar[dr,phantom,"\square"]  & B \ar[u, {Circle[open]}->,swap,"A\oplus 1_B"] \\
& O \ar[r,>->] \ar[u,{Circle[open]}->]& O \ar[u, {Circle[open]}->] 
\end{tikzcd} \quad	\begin{tikzcd}
A \ar[r, >->] \ar[dr,phantom,"\square"] & A\oplus B  \ar[dr,phantom,"\square"] \ar[r, >->,"1_A\oplus \alpha"] & A\oplus B \\
O \ar[r,>->] \ar[u,{Circle[open]}->]&	B \ar[r, >->,"1_B"] \ar[u, {Circle[open]}->] \ar[dr,phantom,"\square"]  & B \ar[u, {Circle[open]}->,swap,"A\oplus\alpha"] \\
& O \ar[r,>->] \ar[u,{Circle[open]}->]& O \ar[u, {Circle[open]}->] 
\end{tikzcd}
\]	
In this case, the obstruction term is now $\lrangles{1_B,1_B}$, which vanishes. That is, the corresponding composition in our presentation of $K_1(\Var_k)$ also splits. It is presently unclear how this argument extends to arbitrary pairs of piecewise automorphisms, but it offers a template for investigating the general case\footnote{Where is the difficulty? Unlike the pCGW category $\Var_k$, which treats open and closed immersions separately, the assembler $\calV_k$ freely combines them. This wasn't a problem in the present example (which only involved coproduct inclusions and automorphisms), but will require careful handling in the general case. Arguments from \cite[\S 9]{CGW} may be helpful.}
\end{discussion}

\begin{remark} In \cite{CGW}, the authors introduce the notion of an ACGW category, which essentially formalises how $\Var_k$ resembles an abelian category.\footnote{Strictly speaking, only the category of reduced finite schemes is an ACGW category, not $\Var_k$. However, \cite[Example 6.3]{CGW} shows the two categories have equivalent $K$-theory.} As such, it is perhaps worth recalling that every abelian category $\calA$ supports a split exact structure, which in general defines a different $K$-theory from the natural exact structure on $\calA$ -- see the warning after \cite[Example II.7.1.2]{WeibelKBook}.
\end{remark}

\begin{comment} Why $\Var_k$ is not an ACGW category? Pushouts of open immersions do not respect separatedness.
\end{comment}


\begin{comment}

\begin{remark}  It is also worth carefully investigating the extent to which Zakharevich's argument in \cite{ZakhK1} extends to exact categories. Given a suitable exact category $\calC$, can we define an analogous category of covers $\calW(\calC)$? What can its $K$-theory tell us about the original $K(\calC)$? Does it differentiate between exact categories whose short exact sequences all split vs. those whose do not? %% Also: relation to direct sum K-theory?
\end{remark}
\end{comment}




Another natural question, posed to the author by Emanuele Dotto, is the following.

\begin{problem} Does the $K$-theory of pCGW categories commute with infinite products?
\end{problem}

\begin{discussion} Relevantly: Zakharevich conjectures in \cite[Remark 2.4]{ZakhPtCount} that the $K$-theory of Assemblers commutes with infinite products. However, she notes that such this result seems presently out of reach since previous results of this form were worked out for Waldhausen categories with cylinder functors (which Assemblers do not have) and exact categories.
\end{discussion}


\subsection{The Motivic Euler Characteristic} There is a well-known enrichment of the Euler Characteristic, known as the {\em motivic Euler Characteristic} or {\em compactly-supported $\mathbb{A}^1$-Euler Characteristic}, defined as a ring homomorphism
$$\chi^{\mathrm{mot}}\colon K_0(\Var_k) \to \mathrm{GW}(k)$$
valued in $GW(k)$, the Grothendieck-Witt ring of quadratic forms over field $k$. An exposition of its construction can be found in \cite{5authorZakhWick}. It is natural to ask if one can lift this to the level of $K$-theory spectra, which was proved in the affirmative by Nanavaty \cite[Theorem 1.1]{Nanavaty}. Explicitly, he constructs a map of spectra
$$K\Var_k \to \End(\sk)$$
where $\End(\sk)$ is the {\em Endomorphism Spectrum of the unit object in the motivic stable homotopy category}, recovering $\cmot$ on $\pi_0$. This sets up the problem:

\begin{problem}\label{prob:K1} Define a natural map $K_1(\Var_k)\to \pi_{1,0}(\sk)$. What geometric information does it encode?
\end{problem}

\begin{discussion} The homotopy groups of $\End(\sk)$ are defined as $\pi_{\ast,0}$. A foundational result, due to Morel \cite{Morel}, shows that $\pi_{0,0}(\sk)\cong \mathrm{GW}(k)$, so one may regard the higher homotopy groups as defining higher Grothendieck-Witt Groups. What geometric information is detected at the higher levels? Recent work by \cite{1lineMOTIVIC} tells us 
	\begin{equation}\label{eq:pi1MOREL}
	0\to K^M_2(k)/24\to \pi_{1,0}(\sk)\to k^\times/2\oplus \Z/2\to 0,
	\end{equation}
where $K^M_\ast(k)$ denotes the Milnor $K$-theory of $k$.
Combined with Theorems~\ref{thm:K1} and/or \ref{thm:Nenashev}, we now have an explicit description of both groups in Problem~\ref{prob:K1}. It remains to map the double exact squares in $\Var_k$ to $\pi_{1,0}(\sk)$ in a natural way, but it is presently unclear, e.g. how these generalised automorphisms interact with the Milnor $K$-theory term, and what this means geometrically. %The fact that Equation~\ref{eq:pi1MOREL} does not split in general presents subtleties).
\end{discussion}


Note: the canonical unit map $\End(\sk)\to KQ$ (i.e. the Hermitian $K$-theory spectrum) induces an isomorphism on the level of $\pi_0$.  We may therefore also think of  $\cmot$ as a map on $\pi_0$ of $K\Var_k\to KQ$, which may be more tractable than the full motivic-homotopy formulation. For those interested in power structures on $K_0(\Var_k)$, here is a natural question:


\begin{problem}\label{prob:ExP} Can we lift the symmetric power structure on $K_0(\Var_k)$ to all $K$-groups
	$$\Sym^{(m)}\colon K_n(\Var_k)\longrightarrow K_n(\Var_k)\qquad?$$
Some natural follow-ups:
\begin{enumerate}[label=(\roman*)]
	\item Can the compatibility of $\cmot$ with symmetric powers on $\pi_0$ be checked (or deduced) from the simplicial presentation given by the $G$-construction?
	\item If a direct lift fails, is there a natural modification of the construction that {\em does} lift -- e.g. reduced symmetric powers with the fat diagonals removed? %What does this tell us about the geometry? Alternatively, think about working over $K_0(\Def_\C)$. Or perhaps localising by $\mathbb{L}$.
\end{enumerate}	
\end{problem}

\begin{discussion} Notice Problem~\ref{prob:ExP} asks about operations on $K$-groups rather than the underlying $K$-theory spaces. This is a subtle distinction since space-level definitions can be problematic. To illustrate: given the $Q$-construction $\calQ\M$ of an exact category with exterior powers, one might try to define $\lambda$-operations on the $K$-groups via a self-map $\lambda^m\colon |\calQ\calM|\to|\calQ\calM|$. But this cannot work: such a map would induce a group homomorphism on $\pi_1$, whereas the usual $\lambda^m$ is not additive on $K_0$. 

Interestingly, these are precisely the kind of issues the $G$-construction was originally designed to resolve -- at least in algebraic $K$-theory. %\footnote{Notice: $K_0$ is now defined as $\pi_0$ of the $G$-construction, not $\pi_1$.}
Indeed, in contrast with other models of $K$-theory (e.g. the $Q$-construction, the $S_\bullet$-construction etc.), Grayson \cite{GraysonExterior,GraysonAdams} showed that the $G$-construction can be leveraged to give combinatorial descriptions of $\lambda$– and Adams operations on higher $K$-groups for suitable exact categories.
	
Returning to our context: the compatibility of $\cmot$ with power structures on $K_0$ and $GW(k)$ has so far only been established in special cases, typically via deep arithmetic (e.g. \cite{PajPalPower,PajwaniCellular}). Problem~\ref{prob:ExP} calls for a shift in perspective: what if we approach the problem simplicially instead? The jury is still out, but Theorem~\ref{thm:GG} gives a promising first clue: the $G$-construction behaves as expected on $\Var_k$. We plan to explore this direction in future work.
\end{discussion}


\begin{comment} The main insight: the $G$-construction shifts everything a degree down. Whereas in the $Q$-construction, $K_0$ was defined as $\pi_1$ of a space, and so therefore any space of maps would induce a group homomorphism on $K_0$, there are no such constraints for $\pi_0$ of the $G$-construction model.
\end{comment}

\begin{remark} Lifting the symmetric power structure on $K_0(\Var_k)$ ought to have useful implications for lifting Kapranov's motivic zeta function to a map of $K$-theory spectra -- see \cite[Question 7.6]{CWZ}. 
\end{remark}


\subsection{Combinatorics of Definable Sets} Theorem~\ref{thm:NenashevDSES} showed that $K_1(\calC)$ of a pCGW category is generated by double exact squares; Example~\ref{ex:Aut} showed that these generalise automorphisms. Automorphisms play a key role in many different areas of mathematics -- can this be extended to these double exact squares in a productive way?

For the model theorist, the automorphism group $\Aut(M)$ of a countable first-order structure $M$ encodes important information about $M$ -- e.g. $\Aut(M)$ measures the homogeneity of $M$, but there are other insights \cite{AutKayeMcPherson,EvansAut}. Some interesting questions:


\begin{problem} A countable structure $M$ is said to be {\em homogeneous} just in case every finite partial isomorphism extends to a global automorphism of $M$. Can we adapt 
	the obstruction theory developed in \cite{ZakhLefschetz} to analyse, e.g. barriers to homogeneity or amalgamation?  What about uncountable structures? 
\end{problem}

\begin{problem} To what extent is $K_1(M)$ be useful for studying non-homogeneous structures? %Structures with small $\Aut(M)?$  %Uncountable structures? %$\C$ considered as an algebraically closed field? 
\end{problem}

%\begin{remark} Thinking about $o$-minimal structures may be a good warmup problem. It would also be interesting to see if the cell decomposition theorem has a useful translation to $K$-theory.
%\end{remark}

\begin{problem} In the converse direction, by viewing $\Aut(M)$ as a topological group, model theorists were able to analyse its structure productively using a wide range of tools. Can we apply the same analysis to $K_1 (M)$? What new insights does this give? 
\end{problem}

\begin{comment}

Recalling Remark~\ref{rem:logic} on the isomorphism of Grothendieck rings  $K_0(\mathrm{Def}(\C))\cong K_0(\Var_k)$ \cite{Beke}, one may ask:

\begin{problem} Is the spectrum $K(\mathrm{Def}(\C))$ weakly equivalent to $K(\Var_k)$? What about the $K$-theory lifting of Bittner's presentation in \cite{SquaresKtheory}? Might the logical perspective be helpful in motivic measures?
\end{problem}
\end{comment}


It is also worth revisiting the original papers \cite{Krajicek,KrajicekScanlon} where the Grothendieck ring of Definable Sets was first investigated. In particular, \cite{KrajicekScanlon} introduces the so-called strong and weak Euler Characteristics on first-order structures before asking which fields admit a non-trivial strong Euler Characteristic. In light of our present work, a natural problem may be:
\begin{comment} Krajicek-Scanlon's question is Problem 7.7.
\end{comment}

\begin{problem} Lift the weak/strong Euler Characteristic on first-order structures to the level of spectra. Analyse what happens on $K_1$ -- what information does it detect? 	 In addition, are there examples of fields with strong Euler characteristics that are trivial on $K_0$ but non-trivial on $K_1$?  %Might $K_1$ highlight relevant combinatorial features of definable sets that force a strong Euler characteristic to be trivial? etc.
\end{problem}



\subsection{Matroids}\label{sec:ProbMatroids} Similar questions about generalised automorphisms may be posed regarding matroids. However, in light of Example~\ref{ex:MatroidspCGW}, a more urgent problem is the following:

\begin{problem} What is the right notion of restricted pushouts for matroids?
\end{problem}

\begin{discussion} Unlike the other pCGW categories, the restricted pushout of $M_0\ltail N\rtail M_1$ cannot be the pushout in the ambient category $\mathrm{Mat}_\bullet$ since this may not exist. To see why, consider the following example suggested to the author by Chris Eppolito: let $N=\{a,b,c,\bullet\}, M_0=\{a,b,c,d,\bullet\}$ and $M_1=\{a,b,c,e,\bullet\}$, all endowed with the uniform matroid structure of rank 2, with point $\bullet$. %This is unsurprising since we already know that amalgams of matroids fail to exist in general.	
\end{discussion}

\begin{comment} Additional notes, argument due to C. Eppolito.

\begin{enumerate}
\item Step 1, consider the Lemma below.

\begin{lemma} Forgetful functor F\colon Mat_\bullet\to \Set_\bullet$ preserves pushouts.
\end{lemma}
\begin{proof} Start by examining the cofree functor $C\colon \Set_\bullet\to Mat_\bullet$ which sends a set $S$ to the matroid whose ground set is $S$ and only flat is $S$. Examine the pushout of $X\leftarrow Z\rightarrow Y$ in $Mat_\bullet$, which we denote to be $P$. Also, denote $S^\vee$ be the pushout object of $F(X\leftarrow Z\rightarrow Y)$ in $\Set_\bullet$.

Since $C(S^\vee)$ is cofree, it is easy to see there exists a pushout morphism $p\colon P\to C(S^\vee)$. Applying the forgetful functor yields $F(p)\colon F(P)\to S$. It remains to check uniqueness of the morphism in $\Set_\bullet$.

We already know that $S^\vee$ is the pushout, so the unique pushout morphism of $F(P)$ simply precomposes the pushout morphism with $F(p)$. This in turn is unique by the pushout property of $P$, and so $F$ preserves pushouts.
\end{proof}

In particular, this means: if $P$ is a pushout in the category of matroids, then the groundset is the pushout of the groundsets of the matroid span.

\item \textbf{$\{d,e\}$ is an independent subset of $P$.} Consider the given example in the Discussion. The rank 2 matroid structure means the flats are all of the form $\{\ast\},\{\ast,s\}$ for $s$ arbitrary, and the whole ground set.
$N$ is obviously the restriction of $M_0$ and $M_1$. If the pushout of this diagram exists
$$\begin{tikzcd}
N \ar[r,>->] \ar[d,>->]& M_0 \ar[d,>->]\\
M_1 \ar[r,>->] & P
\end{tikzcd}$$
then by the above lemma $E_P=\{a,b,c,d,e,\ast\}$, as the groundset. 

Consider $W=\{a,b,c,d,e,\ast\}$ with uniform matroid structure of rank 2, with corresponding restriction maps from $M_0$ and $M_1$. The pushout map is the identity on ground sets. 

By Lemma 2.8 of this paper here (https://arxiv.org/pdf/1512.01390), given any strong map $f\colon M\to N$ of matroids, we have
$\rk_N(f(S))\leq \rk_M(S)$.

We claim this implies $S=\{d,e\}$ is an independent subset of $P$. Since $p$ is identity on elements, we get $\rk_W(p(\{d,e\}))=\rk_W(\{d,e\})=2$. The inequality 
$$r_W(p(\{d,e\}))\leq r_P(\{d,e\})$$
becomes 
$$2\leq r_P(\{d,e\}).$$
But $r_P(\{d,e\})\leq |\{d,e\}|=2$, and so $r_P(\{d,e\})=2$, and thus has rank 2 in $P$. And so $\{d,e\}$ is an independent subset of $P$.
\item \textbf{$\{a,b\}$ generates $E_P$, $P$ has rank 2.} Notice that $\{a,b\}$ generates the groundset $M_{0}$ and $M_1$ (since they all have rank 2). By definition of strong maps, one can check that that $f\colon M\to N$ implies [for $S\subseteq E_M$]
$$f(\cl_M(S))\subseteq \cl_N(f(S)).$$
See for example Proposition 3.5 of https://arxiv.org/pdf/2108.13745, but this is not difficult to work out directly. 


Now $\cl_X(\{a,b\})=E_X$ and $\cl_Y(\{a,b\})=E_Y$. 
And so since
$$E_X=\cl_X(\{a,b\})\subseteq \cl_P(\{a,b\})\qquad E_Y=\cl_Y(\{a,b\})\subseteq \cl_P(\{a,b\})$$
this gives
$$E_P = E_X \cup E_Y \subseteq \cl_P(\{a,b\}).$$

The reverse inclusion $\cl_P(\{a,b\})\subseteq E_P$ is trivial, and thus 
$$\cl_P(\{a,b\})=E_P.$$

Thus conclude that $P$ has rank 2. Further, since $\{d,e\}$ is an independent subset of $P$, this implies $\{d,e\}$ is a basis of $P$. In particular, $P$ has the structure of a uniform matorid of rank 2.
\item Finally, consider the diagram

\[\begin{tikzcd} 
N \ar[r] \ar[d] & M_{1} \ar[d]  \ar[ddr,"e\mapsto d"] \\
M_{0}\ar[r] \ar[drr,"\id",swap] & P \ar[dr,dotted] \\
&& M_{0} 
\end{tikzcd}\]

Recall that $\{d,\ast\}$ is a flat of $M_0$, and so $p^{-1}\{d,e\}=\{d,e,\ast\}$ is a flat of $P$. But this flat properly contains  \{d,e\}, and so contradicts the fact that $\{d,e\}$ generates the whole $E_P$.
\end{enumerate}


\end{comment}


\begin{discussion}[Independent Squares] A related problem was considered by the model theorists. Let $\calD$ be a category whose morphisms are all monic, and suppose $\calD$ admits a {\em stable independence notion}, i.e. a class of so-called {\em independent squares}
	$$\begin{tikzcd}
	N \ar[r,>->] \ar[d,>->]  \ar[dr,phantom,"\forkindep"]& M_1 \ar[d,>->]\\
	M_0  \ar[r,>->]& M
	\end{tikzcd}$$  
satisfying the axioms listed in \cite[Definition 5.5]{Vasey}. This includes the condition that any span in $\calD$ $$M_0\ltail N\rtail M_1$$
can be completed into an independent square.  Examples include $\mathrm{FinSet}$, the category of vector spaces over a fixed fields (with injective linear transformations), and the category of graphs (with subgraph embeddings). 

Here is the key insight from \cite[\S 5.1]{Vasey}. Under certain technical conditions, the category of independent squares associated to a given span has a weakly initial object, which is unique up to (not necessarily unique) isomorphism. Model theorists call this the {\em prime object} over the span. Some natural questions:
\begin{enumerate}
	\item  Can the definition of restricted pushouts be (meaningfully) weakened to include prime objects? 
	% Can the notion of restricted pushouts be weakened -- e.g. by allowing weak initiality or dropping the pullback condition -- to include prime objects? 
	\item Can specific classes of matroids (e.g. vectorial, or graphic matroids) admit such a structure? All classes?
	\item How do independent squares align with matroid-theoretic notions like rank and flats?
\end{enumerate}
This will be investigated in future work.
%The claim in \cite{Vasey} that multidimensional stable independence guarantees the existence of a prime object over any span was given without proof; this too will require checking.
\end{discussion}

\begin{comment}


As a warm-up exercise, the following problem may be helpful:

\begin{problem} Observe that proto-exact categories require the existence of a zero object whereas CGW categories do not. Can we adapt the arguments of \cite{ProtoExactMatroids} to show the category of (unpointed) matroids form a CGW category?
\end{problem}


One can pose similar questions to the matroid theorists. To our knowledge, the $K$-theory spectrum of matroids was first defined in \cite{ProtoExactMatroids}, where it was also showed that the higher $K$-groups are non-trivial (Theorem 6.4 of the paper). The following problems are natural.

\begin{problem} Compute $K_1(\mathrm{Mat_\bullet})$. What aspect of matroids does $K_1$ measure? In algebraic $K$-theory, one can view $K_1$ as a generalised determinant -- what might this perspective tell us about the abstract independence relations modelled by matroids?
\end{problem}

\begin{problem} Lift the Tutte Polynomial
	$$T\colon K_0(\mathrm{Mat}_\bullet)\to \mathbb{Z}[x,y]$$
to the level of $K$-theory spectra. What information is encoded on the level of $K_1$?	
\end{problem}

\end{comment}

% \newpage

\appendix
\section{Properties of Restricted Pushouts}

As explained in Section~\ref{sec:pCGW}, it is unreasonable to ask for $\M$-morphisms of CGW categories to be closed under pushouts, so we instead ask for them to be closed under {\em restricted pushouts}, a weaker notion. This section collects various technical facts about them, which hold in any pCGW category $\calC$. Finally, whenever relevant, we follow Convention~\ref{conv:coord} when labelling morphisms. 

\begin{lemma}\label{lem:Hspace1}\hfill 
	\begin{enumerate}[label=(\roman*)]
		\item Given a span $B\leftarrowtail A \rtail C$, 
		$$\frac{B}{A}\oplus \frac{C}{A}\cong \frac{B\star_A C}{A}.$$
		\item Given a span $B\leftarrowtail A \rtail C$, along with $\M$-morphisms $B\rtail B'$ and $C\rtail C'$,
		$$\frac{B'\star_A C'}{B\star_A C} \cong \frac{B'}{B}\oplus \frac{C'}{C}.$$
	\end{enumerate}
\end{lemma}

\begin{proof}\hfill 
\begin{enumerate}[label=(\roman*):]
	\item Consider the diagram
	\[\begin{tikzcd}
	\frac{C}{A}\ar[d, {Circle[open]}->] \ar[dr, phantom, "\square"]& O \ar[dr, phantom, "\square"]\ar[d, {Circle[open]}->]  \ar[l, >->] \ar[r, >->] & \frac{B}{A} \ar[d, {Circle[open]}->] \\
	C & A \ar[l, >->] \ar[r, >->] & B
	\end{tikzcd}\]
	Apply Axiom (DS) to obtain the left diagram below:
	\begin{equation}\label{eq:PPiso}
	\begin{tikzcd}
	O \ar[dr, phantom, "\square"]\ar[d, {Circle[open]}->]  \ar[r, >->] & \frac{B}{A} \ar[d, {Circle[open]}->] \ar[dr, phantom, "\square"] \ar[r,>->] &   \frac{B}{A}\oplus \frac{C}{A} \ar[d,{Circle[open]}->]\\
	A \ar[r, >->,"f"] & B \ar[r,>->,"f'"]  & B\star_A C
	\end{tikzcd}\qquad\qquad  \dsquaref{O}{\frac{B\star_AC}{A}}{A}{B\star_AC}{}{}{f'f}{} 
	\end{equation}
	Since distinguished squares compose, the outermost rectangle of the left diagram also defines a distinguished square. On the other hand, by Axiom (K), the RHS exact square exists. Since formal cokernels are unique up to isomorphism, conclude that $\frac{B}{A}\oplus \frac{C}{A}\cong \frac{B\star_A C}{A}$.
	\item Repeated applications of Fact~\ref{facts:restrictedpushouts} yields 
	\begin{equation}
	\begin{tikzcd}
	A \ar[r,>->] \ar[d,>->] & B \ar[r,>->] \ar[d,>->] & B' \ar[d,>->]\\
	C \ar[r,>->] \ar[d,>->] & B\star_A C  \ar[d,>->] \ar[r,>->] &  B'\star_A C \ar[d,>->]\\
	C'\ar[r,>->]& B\star_AC' \ar[r,>->]& B'\star_AC'
	\end{tikzcd}
	\end{equation}
	In particular, there exists an $\M$-morphism $B\star_AC\rtail B'\star_A C'$, so $\frac{B'\star_AC'}{B\star_AC}$ is well-defined and exists by Axiom (K). By Axiom (PQ), restricted pushouts preserve quotients, and so
	\begin{equation}\label{eq:iso(iii)}
	\frac{B'}{B} \cong \frac{B'\star_AC}{B\star_A C}\quad\text{and} \quad \frac{C'}{C}\cong  \frac{B\star_AC'}{B\star_A C}\,\,.
	\end{equation}
	Now consider the diagram
	\begin{equation}\label{eq:PP(iii)}
	\begin{tikzcd}
	\frac{B\star_AC'}{B\star_A C} \ar[d, {Circle[open]}->] \ar[dr, phantom, "\square"]& O \ar[dr, phantom, "\square"]\ar[d, {Circle[open]}->]  \ar[l, >->] \ar[r, >->] & \frac{B'\star_AC}{B\star_A C} \ar[d, {Circle[open]}->] \\
	B\star_A C' & B\star_A C \ar[l, >->] \ar[r, >->] & B'\star_AC
	\end{tikzcd}
	\end{equation}
	Applying the same argument from (i) to Equations~\eqref{eq:iso(iii)} and \eqref{eq:PP(iii)}, deduce the desired isomorphism.
\end{enumerate}	
\end{proof}


\begin{corollary}[Pushouts create 2-simplices]\label{cor:RPtrick} Given any span of 1-simplices in $G\calC$
\begin{equation}\label{eq:RPtrick1}
\ones{M'}{L'}\leftarrow \ones{M}{L} \to \ones{M''}{L''},
\end{equation}
taking restricted pushouts yields the following diagram
\begin{equation}\label{eq:RPtrickDiag}
\begin{tikzcd}
\ones{M}{L} \ar[rr] \ar[d] \ar[drr] && \ones{M''}{L''} \ar[d]\\
\ones{M'}{L'} \ar[rr]&& \ones{M'\star_MM''}{L'\star_L L''}
\end{tikzcd},
\end{equation}
where both triangles bound 2-simplices. In particular, this defines a homotopy between Diagram~\eqref{eq:RPtrick1} and 
\begin{equation}
\ones{M'}{L'} \to \ones{M'\star_M M''}{L'\star_L L''} \leftarrow \ones{M''}{L''}.
\end{equation}
\end{corollary}
\begin{proof} Since Diagram~\eqref{eq:RPtrick1} is a span of 1-simplices, their choice of formal quotients agree. By a harmless abuse of notation, %\footnote{The notation suggests that the 1-simplices feature the canonical choice of quotients from Axiom (K), which need not be the case. But this distinction is fairly harmless to the general thrust of the argument.}
denote these quotients as:
	$$\frac{M'}{M}=\frac{L'}{L}\quad\text{and}\quad \frac{M''}{M}=\frac{L''}{L}.$$
 Now form restricted pushouts $P:=M'\star_M M''$ and $Q:=L'\star_L L''$. Apply Lemma~\ref{lem:Hspace1} to deduce
	$$ \small{\frac{P}{M}\cong \frac{M'}{M}\oplus \frac{M''}{M} \qquad \text{and} \qquad \frac{Q}{L}\cong \frac{L'}{L}\oplus \frac{L''}{L}=\frac{M'}{M}\oplus \frac{M''}{M}.}$$
	In fact, Equation~\eqref{eq:PPiso} of the proof tells us the data assembles into diagram pairs, such as
	\[\begin{tikzcd}
	M \ar[r, >->] \ar[dr,phantom,"\square"]&  M' \ar[dr,phantom,"\square"] \ar[r, >->] & P\\
	O \ar[u,{Circle[open]->}] \ar[r,>->] & \frac{M'}{M}  \ar[u,{Circle[open]->}] \ar[r,>->]	\ar[dr,phantom,"\square"]& \frac{M'}{M}\oplus \frac{M''}{M}\ar[u, {Circle[open]}->] \\
	& O \ar[u,{Circle[open]}->] \ar[r,>->] & \frac{M''}{M} \ar[u, {Circle[open]}->] 
	\end{tikzcd}\quad  \begin{tikzcd}
	L \ar[r, >->] \ar[dr,phantom,"\square"]&  L' \ar[dr,phantom,"\square"] \ar[r, >->] & Q \\
	O \ar[u,{Circle[open]->}] \ar[r,>->] & \frac{M'}{M}  \ar[u,{Circle[open]->}] \ar[r,>->]	\ar[dr,phantom,"\square"]& \frac{M'}{M}\oplus \frac{M''}{M}\ar[u, {Circle[open]}->] \\
	& O \ar[u,{Circle[open]}->] \ar[r,>->] & \frac{M''}{M} \ar[u, {Circle[open]}->] 
	\end{tikzcd}\quad.	\]
This defines 2-simplices in $G\calC$
	$$\begin{pmatrix} & O\rtail  &  \doubleunderline{M\rtail M'\rtail P}\\ &O\rtail &L\rtail L'\rtail Q \end{pmatrix}\quad\text{and}\quad \begin{pmatrix} & O\rtail  &  \doubleunderline{M\rtail M''\rtail P}\\ &O\rtail &L\rtail L''\rtail Q \end{pmatrix},$$
which assembles into Diagram~\eqref{eq:RPtrickDiag}.
\end{proof}
\begin{comment} More explicitly: we can construct restricted pushouts, and choose its quotients to ensure that we have a 2-simplex. 
\end{comment}


Next, examining the examples of pCGW categories in Section~\ref{sec:pCGW}, notice the formal direct sums are all disjoint unions (= the coproduct injections are monic + their pullback is the initial object), and the distinguished squares are all pullbacks (possibly satisfying additional hypotheses). One may therefore wonder if formal direct sums commute with distinguished squares in some natural sense. The following lemmas make this intuition precise.

\begin{comment} Reference for why commutativity follows from coproducts being disjoint:

https://math.stackexchange.com/questions/1746354/commutations-of-pullbacks-and-coproducts
\end{comment}

\begin{convention}[Permutation Isomorphisms]\label{conv:perm-iso}  For any objects $A,B$ in $\calC$, the universal property of restricted pushouts defines a canonical {\em permutation isomorphism} $\tau\colon A\oplus B \rtail B\oplus A$. Since $\tau$ is an isomorphism, Axiom (I) defines a corresponding $\E$-morphism $\varphi(\tau)\colon A\oplus B \otail B\oplus A$. 
	% Abusing notation, we typically denote the $\E$-morphism as just $\tau$. %, The only subtlety here is the direction of the isomorphism, depending on $\E$ or $\E^\opp\subseteq \calC$ but this is not indicated by the $\varphi$ so there is no point having it there.
\end{convention}


\begin{lemma}\label{lem:ADTsquares} Given any distinguished square
	\begin{equation}\phi:=\left(\dsquaref{O}{C}{A}{B}{}{}{f}{g}\right),
	\end{equation}
the following is true:
	\begin{enumerate}[label=(\roman*)]
		\item Given any object $D$ in $\calC$, we can construct the following distinguished squares: \begin{equation}\label{eq:sumdistSQ}
		\dsquaref{O}{C\oplus D}{A}{B\oplus D}{}{}{f\oplus D}{g\oplus 1} \qquad \text{and} \qquad 
		\dsquaref{O}{C}{A\oplus D}{B\oplus D}{}{}{f\oplus 1}{g\oplus D}; 
		\end{equation} 
		\begin{equation}
		\dsquaref{C}{C\oplus D}{B}{B\oplus D}{1\oplus D}{g}{1\oplus D}{g\oplus 1}\qquad \text{and} \qquad \dsquaref{D}{C\oplus D}{A\oplus D}{B\oplus D}{C\oplus 1}{q_D}{f\oplus 1}{g\oplus 1}.
		\end{equation}
		All newly featured morphisms will be defined in the course of the proof.
		%%% Notice: If $\E\subseteq \calC$, then $q_B\circ g$ first includes $C\to B$ via $g$, before including it into the coproduct $B\oplus D$. If $\E^\opp\subseteq \calC$, then $q_B\circ g$ first projects $B\oplus D\to B$, before mapping $g\colon B\to C$.
		\item The distinguished squares in (i) can be permuted in the obvious way, e.g. 
		\begin{equation}\label{eq:sumdistSQ-perm}
		\dsquaref{O}{D\oplus C}{A}{D\oplus B}{}{}{D\oplus f}{1\oplus g} \qquad \text{and} \qquad 
		\dsquaref{O}{C}{D\oplus A}{D\oplus B}{}{}{1\oplus f}{D\oplus g}; 
		\end{equation} 
		\item Given $A\xrtail{f\oplus D}B\oplus D$ or $A\xrtail{D\oplus f} D\oplus B$ (as defined above), there exists isomorphisms
		$$ \frac{B\oplus D}{A}\cong \frac{B}{A}\oplus D \cong C\oplus D\qquad \text{and} \qquad \frac{D\oplus B}{A}\cong D\oplus \frac{B}{A} \cong D\oplus C \qquad.$$
	\end{enumerate}
\end{lemma}
\begin{proof} % To ease notation, we omit the labels of the morphisms when they are clear from context.
	\begin{enumerate}[label=(\roman*):]
		\item First, apply Fact~\ref{facts:restrictedpushouts} to
		$$D\leftarrowtail O\rtail A\rtail B,$$
		and obtain the isomorphism $B\oplus D\cong (A\oplus D)\star_A B$. Next, consider the diagram
			\[\begin{tikzcd}
		D \ar[d, {Circle[open]}->,"q_D"] \ar[dr, phantom, "\square"]& O \ar[dr, phantom, "\square"]\ar[d, {Circle[open]}->]  \ar[l, >->] \ar[r, >->] & C \ar[d, {Circle[open]}->,"g"] \\
		A\oplus D & A \ar[l, >->] \ar[r, >->,"f"] & B
		\end{tikzcd}\]
		where $q_D$ is the canonical $\E$-morphism from a direct sum square. Applying Axiom (DS), we obtain the following distinguished squares
		\[\psi:=\left(\dsquaref{C}{C\oplus D}{B}{B\oplus D}{1\oplus D}{g}{1\oplus D}{g\oplus 1}\right)\qquad \psi':=\left(\dsquaref{D}{C\oplus D}{A\oplus D}{B\oplus D}{C\oplus 1}{q_D}{f\oplus 1}{g\oplus 1}\right).\]
		Next, obtain Equation~\eqref{eq:sumdistSQ} by horizontal composition of $\psi$ with $\phi$, and vertical composition of $\psi'$ with the formal sum square 
		\[\dsquare{O}{C}{D}{C\oplus D}.\]
		\item Apply the same analysis in part (i) to the diagram
			\[\begin{tikzcd}
		C \ar[d, {Circle[open]}->,swap,"g"]  \ar[dr,phantom,"\square"]& O 
	\ar[r,>->] \ar[dr,phantom,"\square"]\ar[l,>->,swap]\ar[d, {Circle[open]}->] & D\ar[d, {Circle[open]}->,"q_D"] \ar[dr,phantom,"\square"] 	\ar[r,>->,"1"] & D \ar[d, {Circle[open]}->,"\varphi(\tau)\circ q_D"] \\
		B & A \ar[l,>->,swap,"f"] 	\ar[r,>->] & A\oplus D 	\ar[r,>->,"\tau"]& D\oplus A
		\end{tikzcd}\qquad,\]
	where $\tau\colon A\oplus D\to D\oplus A$ denotes the permutation isomorphism. 
		\begin{comment}
		RHS square is distinguished  since distinguished squares are closed under isomorphisms; the diagram commutes regardless if $\E$ or $\E^\opp\subseteq\calC$.
			 
		\[ \begin{tikzcd} O \ar[r,>->] \ar[d,>->] &  A \ar[d,>->] \ar[r,>->,"f"] & B \ar[d,>->] \\
		D \ar[r,>->] \ar[d,>->,"1"] & A\oplus D \ar[r,>->,"f\oplus 1"] \ar[d,>->,"\tau"] & B\oplus D\ar[d,>->,"\tau"] \\
		D \ar[r,>->] & D\oplus A \ar[r,>->,"1\oplus f"] & D\oplus B
		\end{tikzcd} \]	 
		\end{comment}
			\item Apply Axiom (K) to the LHS exact squares of Diagrams~\eqref{eq:sumdistSQ} and \eqref{eq:sumdistSQ-perm}.
	\end{enumerate}
\end{proof}

\begin{lemma}[Direct Sums]\label{lem:DirectSum} \hfill
	\begin{enumerate}[label=(\roman*)]
		\item Given any pair of exact squares
\begin{equation}\label{eq:dir-sum-exact-square}
\dsquaref{O}{V_0}{A}{B}{ }{ }{ f_{0}}{f_{1} } \qquad ,\qquad 	\dsquaref{O}{V_1}{C}{D}{ }{ }{ g_{0}}{g_{1} } 
\end{equation}
		we can construct an exact square
		\[\dsquaref{O}{V_0\oplus V_1}{A\oplus C}{B\oplus D}{ }{ }{ f_{0}\oplus g_0 }{f_{1} \oplus g_1}\quad .\]
In particular, this yields an isomorphism $\frac{B\oplus D}{A\oplus C} \cong \frac{B}{A}\oplus \frac{D}{C}$ by Axiom (K).
		\item Given objects $A,B$ in $\calC$, the induced $\M$-morphism $$1_A\oplus 1_B\colon A\oplus B\rtail A\oplus B$$ is the identity $1_{A\oplus B}$. In particular, direct sums commute with the degeneracy maps of $G\calC$.
		\item Continuing with Diagram~\eqref{eq:dir-sum-exact-square} as our setup, the following $\M$-square commutes
	\begin{equation}\label{eq:M-and-E}
	\begin{tikzcd}
	A\oplus C \ar[r,>->,"f_0\oplus g_0"] \ar[d,>->,swap,"\tau"] & B\oplus D \ar[d,>->,"\tau"]\\
	C\oplus A \ar[r,>->,"g_0\oplus f_0"]& D\oplus B
	\end{tikzcd}
	\end{equation}
		where the $\tau$'s denote the permutation isomorphisms.
		\item  Given a pair of flag diagrams
		\begin{equation}\label{eq:dirsum-flag}
		\begin{tikzcd}
		A \ar[r, >->,"f_0"] \ar[dr,phantom,"\square"]&  B  \ar[dr,phantom,"\square"] \ar[r, >->,"g_0"] & C\\
		O \ar[u,{Circle[open]->}] \ar[r,>->] & V_0  \ar[u,{Circle[open]->},"f_1"] \ar[r,>->,"h_0"]	\ar[dr,phantom,"\square"]& V_1 \ar[u, {Circle[open]}->,"g_1"] \\
		& O \ar[u,{Circle[open]}->] \ar[r,>->] & V_2 \ar[u, {Circle[open]}->,"h_1"] 
		\end{tikzcd}\quad  \begin{tikzcd}
		A' \ar[r, >->,"f'_0"] \ar[dr,phantom,"\square"]&  B'  \ar[dr,phantom,"\square"] \ar[r, >->,"g'_0"] & C'\\
		O \ar[u,{Circle[open]->}] \ar[r,>->] & V'_0  \ar[u,{Circle[open]->},"f'_1"] \ar[r,>->,"h'_0"]	\ar[dr,phantom,"\square"]& V'_1\ar[u, {Circle[open]}->,"g'_1"] \\
		& O \ar[u,{Circle[open]}->] \ar[r,>->] & V'_2\ar[u, {Circle[open]}->,"h'_1"] 
		\end{tikzcd} \quad,
		\end{equation}
	direct sums commute with the composition of $\M$-morphisms as follows:
			\begin{equation}\label{eq:dirsum-comp-M}
		(g_0\circ f_0)\oplus (g'_0\circ f'_0)= (g_0\oplus g'_0) \circ  (f_0\oplus  f'_0).
		\end{equation}	
In particular, direct sums commute with the face maps of $G\calC$.
		\item Continuing with Diagram~\ref{eq:dirsum-flag} as our setup, we can construct a distinguished square
		\begin{equation}\label{eq:dirsum-v}
		\dsquaref{V_0\oplus V'_0}{V_1\oplus V'_1}{B\oplus B'}{C\oplus C'}{h_0\oplus h'_0}{f_1\oplus f'_1}{g_0\oplus g'_0}{g_1\oplus g'_1}
		\end{equation}
	\end{enumerate}	
	
\end{lemma}
\begin{proof} \begin{enumerate}[label=(\roman*):]
		\item Apply Lemma~\ref{lem:ADTsquares} to construct the diagram
		\[\begin{tikzcd}
	V_1 \ar[d, {Circle[open]}->,swap,"A\oplus g_1"] \ar[dr, phantom, "\square"]& O \ar[dr, phantom, "\square"]\ar[d, {Circle[open]}->,]  \ar[l, >->] \ar[r, >->] & V_0 \ar[d, {Circle[open]}->,"f_1\oplus C"] \\
		A\oplus D & A\oplus C \ar[l, >->,"1\oplus g_0"] \ar[r, >->,swap,"f_0\oplus 1"] & B\oplus C
		\end{tikzcd}\qquad .\]
	By Fact~\ref{facts:restrictedpushouts}, $(B\oplus C)\star_{A\oplus C} (A\oplus D )\cong B\oplus D$. The rest follows from applying Axiom (DS) and the fact that distinguished squares compose horizontally. This defines Exact Square~\eqref{eq:dir-sum-exact-square}, with the $\M$-morphism:
\begin{equation}\label{eq:dirsum-M-morphism}
f_0\oplus g_0:=(f_0\oplus 1)\circ (1\oplus g_0)=(1\oplus g_0)\circ (f_0\oplus 1).
\end{equation}
		%% As before, the morphisms featured in the lemma are the ones defined by this construction.
		\item This follows from the definition of $1_A\oplus 1_B$ (Equation~\eqref{eq:dirsum-M-morphism}), and the universal property of restricted pushouts. 
		\begin{comment} Details, see diagram below.
		\[\begin{tikzcd}
		O \ar[r,>->] \ar[d,>->]  & A \ar[r,>->,"1"] \ar[d,>->] & A\ar[d,>->] \\
		B \ar[r,>->] \ar[d,>->,"1"] & A\oplus B\ar[r,>->,"1"] \ar[dr,dotted, "1\oplus 1"] \ar[d,>->,"1"] & A\oplus B\ar[d,>->] \\
		B \ar[r,>->] & A\oplus B \ar[r,>->] & A\oplus B
		\end{tikzcd}\] 
		\end{comment}
		\item As a warm-up, examine the definition of $f_0\oplus 1$ and $1\oplus f_0$ in Lemma~\ref{lem:ADTsquares}, constructed using restricted pushouts. Assemble the data into the diagram below
	\begin{equation}\label{eq:perm-iso-arg}
	\begin{tikzcd}
	O \ar[r,>->] \ar[d,>->] & A \ar[r,>->,"f_0"] \ar[d,>->] & B \ar[d,>->] \\
	C \ar[r,>->] \ar[d,>->, "1"]& A\oplus C \ar[r,>->,"f_0\oplus 1"] \ar[d,>->,"\tau"]& B\oplus C\ar[d,>->,dotted,"\tau" ]\\
	C \ar[r,>->]& C\oplus A \ar[r,>->,"1\oplus f_0"] & C\oplus B
	\end{tikzcd}\quad,
	\end{equation}
		where the permutation isomorphism $\tau\colon B\oplus C\rtail C\oplus B$ is induced by the universal property of restricted pushouts (cf. Convention~\ref{conv:perm-iso}); this yields the identity $\tau\circ (1\oplus f_0)=(f_0\oplus 1)\circ\tau$. Similarly, deduce that  $(1\oplus f_0)\circ\tau  =\tau \circ (f_0\oplus 1)$, with the obvious permutations.
		
\begin{comment} For the other identity, we extend the diagram as follows:
\begin{tikzcd}
O \ar[r,>->] \ar[d,>->] & A \ar[r,>->,"f_0"] \ar[d,>->] & B \ar[d,>->] \\
C \ar[r,>->] \ar[d,>->, "1"]& A\oplus C \ar[r,>->,"f_0\oplus 1"] \ar[d,>->,"\tau"]& B\oplus C\ar[d,>->,"\tau" ]\\
C \ar[r,>->] \ar[d,>->,"1"]& C\oplus A \ar[d,>->,"\tau^{-1}"] \ar[r,>->,"1\oplus f_0"] & C\oplus B \ar[d,>->,dotted,blue,"\tau^{-1}"]\\
C \ar[r,>->]& A\oplus C \ar[r,>->,dashed,red, "f_0\oplus 1"] & B\oplus C
\end{tikzcd}
The dotted blue arrow is also $\tau^{-1}$ by the same argument as above\footnote{This is an abuse of notation: obviously the permutations are not identical morphisms.}; the red dashed arrow is $f_0\oplus 1$ since $\tau^{-1}\circ\tau=1$ and so we have effectively just taken the pushout of $A\oplus C\ltail A\xrtail{f_0} B$, which is the definition of $f_0\oplus 1$.

[Why $\tau^{-1}\circ\tau}=1$? Write out the entire diagram for $A\oplus C\rtail C\oplus A\rtail A\oplus C$ and notice the composition of $\tau^{-1}\circ \tau$ defines a morphism or natural transformation in $\M_\calD$. Since restricted pushouts are initial, this forces $\tau^{-1}\circ\tau=1$.]
\end{comment}		

Now compare the definition of $f_0\oplus g_0$ vs. $g_0\oplus f_0$ from Equation~\eqref{eq:dirsum-M-morphism}, modified with the relevant permutations (greyed out below):
\begin{equation}\label{eq:twist-M-and-E}
\begin{tikzcd}
A\oplus C \ar[rr,>->,"f_0\oplus 1 "]  \ar[drr,>->,"f_0\oplus g_0"]\ar[d,>->,swap,"1\oplus g_0"]&& B\oplus C   \ar[d,>->,"1\oplus g_0"] \\
A\oplus D \ar[d,>->,gray,swap,"\tau"] \ar[rr,>->,swap,"f_0\oplus 1"]&& B\oplus D \ar[d,>->,gray, "\tau"]\\
\color{gray} D\oplus A \ar[rr,>->,gray, swap,"1\oplus f_0"]&& \color{gray}{D\oplus B}
\end{tikzcd} \qquad \qquad \begin{tikzcd} \color{gray}{A\oplus C} \ar[rr,>->,"1\oplus g_0",gray]  \ar[d,>->,swap,"\tau", gray] && \color{gray}{A\oplus D} \ar[d,>->,"\tau",gray] \\
C\oplus A \ar[rr,>->,"g_0\oplus 1 "]  \ar[drr,>->,"g_0\oplus f_0"]\ar[d,>->,swap,"1\oplus f_0"]&& D\oplus A   \ar[d,>->,"1\oplus f_0"] \\
C\oplus B \ar[rr,>->,swap,"g_0\oplus 1"]&& D\oplus B
\end{tikzcd}\quad,
\end{equation}
\begin{comment} WAS: A more compact diagram without the permutations, but having the permutations seemed to clarify the argument.
	\begin{equation}\label{eq:twist-M-and-E}
\begin{tikzcd}
A\oplus C \ar[rr,>->,"f_0\oplus 1 "]  \ar[drr,>->,"f_0\oplus g_0"]\ar[d,>->,swap,"1\oplus g_0"]&& B\oplus C   \ar[d,>->,"1\oplus g_0"] \\
A\oplus D  \ar[rr,>->,swap,"f_0\oplus 1"]&& B\oplus D  
\end{tikzcd} \qquad \qquad \begin{tikzcd} C\oplus A \ar[rr,>->,"g_0\oplus 1 "]  \ar[drr,>->,"g_0\oplus f_0"]\ar[d,>->,swap,"1\oplus f_0"]&& D\oplus A   \ar[d,>->,"1\oplus f_0"] \\
C\oplus B \ar[rr,>->,swap,"g_0\oplus 1"]&& D\oplus B
\end{tikzcd}\quad,
\end{equation}
\end{comment}	
Plugging in the obvious permutations, compute:
$$\tau\circ (f_0\oplus g_0)=\left(\tau\circ (f_0\oplus 1)\right)\circ (1\oplus g_0)= (1\oplus f_0) \circ \left(\tau\circ  (1\oplus g_0)\right)= (g_0\oplus f_0)\circ \tau\,.$$
%In other words, the $\M$-square in Equation~\eqref{eq:M-and-E} commutes.
\begin{comment}
A previous attempt tried to show that the corresponding $\E$-square commutes, but the current axioms only guarantee quotients up to isomorphism, which is not enough to show that the obvious $\E$-morphisms + permutation commute on-the-nose. As it turns out, the latter condition is not necessary in order to get the Permutation Lemma~\ref{lem:permute} to work.

Of course, as discussed in Remark~\ref{rem:INV}, since we are primarily interested in constructing distinguished squares, in practice we really only require things to commute up to codomain-preserving isomorphism, not necessarily on the nose.
\end{comment}
		\item  Examining part (i), notice the relevant $\M$-morphisms were all constructed via restricted pushouts. The claim then essentially follows from Fact~\ref{facts:restrictedpushouts}.
\item Proceed in stages. 
\subsubsection*{Step 0} Apply Lemma~\ref{lem:ADTsquares} to construct the LHS diagram below
\begin{equation}\label{eq:dirsum-step0}
\begin{tikzcd}
V'_0 \ar[d,{Circle[open]}->,swap,"q_{V'_{0}}"] \ar[dr,phantom,"\square"]& O \ar[l,>->]\ar[d,{Circle[open]}->] \ar[r,>->] \ar[dr,phantom,"\square"]& V_2 \ar[d,{Circle[open]}->,"h_1"] \\
V_0\oplus V'_0 \ar[d,{Circle[open]}->,swap,"f_1\oplus 1"]  \ar[dr,phantom,"\square"] & V_0 \ar[l,>->,"1\oplus V'_0",swap]\ar[d,{Circle[open]}->,"f_1"] \ar[r,>->,"h_0"] \ar[dr,phantom,"\square"] & V_1 \ar[d,{Circle[open]}->,"g_1"] \\
B\oplus V'_0 & B  \ar[l,>->,"1\oplus V'_0"] \ar[r,>->,swap,"g_0"]& C
\end{tikzcd}\quad\qquad \dsquaref{V_0\oplus V'_0}{V_1\oplus V'_0}{B\oplus V'_0}{C\oplus V'_0}{h_0\oplus 1}{f_1\oplus 1}{g_0\oplus 1}{\varv}\quad.
\end{equation}
By Axiom~(DS) we obtain the RHS distinguished square, with $\M$-morphisms labelled as in Lemma~\ref{lem:ADTsquares}. In addition, we may identify $\varv=g_1\oplus 1$ without loss of generality. 
[Why? Applying functor $k\colon\Ar_{\square}\E\to \Ar_{\triangle}\M$ (Helper Definition~\ref{def:help}) to this square yields 
$$\begin{tikzcd}
O \ar[d,{Circle[open]->}]\ar[rr,>->] \ar[drr,phantom,"\square"]&& V_1\oplus V'_0\ar[d,{Circle[open]->},"\varv"] \\
A \ar[rr,>->,swap,"(g_0\oplus 1)\circ (f_0\oplus V'_0)"] && C\oplus V'_0
\end{tikzcd}.$$
By appealing to either part (iv) of this Lemma or Fact~\ref{facts:restrictedpushouts} directly, notice that
$$(g_0\oplus 1)\circ (f_0\oplus V'_0)= (g_0\circ f_0)\oplus V'_0,$$
which corresponds to the kernel of $g_1\oplus 1$. Hence, by Axiom (K), $g_1\oplus 1$ and $\varv$ are equivalent up to codomain-preserving isomorphism, and the two morphisms can be identified (cf. Footnote~\ref{fn:INV}).] 
\begin{comment} Details.
\[\begin{tikzcd}
O \ar[r,>->] \ar[d,>->]& A \ar[r,>->,"f_0"] \ar[d,>->] \ar[dr, yshift=-0.1em,"f_0\oplus V'_0"]& B \ar[r,>->,"g_0"] \ar[d,>->] & C \ar[d,>->]\\
V'_0 \ar[r,>->]& A\oplus V'_0 \ar[r,>->,swap,"f_0\oplus 1"]& B\oplus V'_0 \ar[r,>->,swap,"g_0\oplus 1"]& C\oplus V'_0
\end{tikzcd}\]
Notice $f_0\oplus V'_0$ is adding $f_0$ with the $\M$-morphism $O\rtail V'_0$, so this justifies the appeal to part (iv).
\end{comment}


\subsubsection*{Step 1} Consider the diagram
\begin{equation}\label{eq:dirsum-step1}
\begin{tikzcd}
B\oplus V_0' \ar[d,{Circle[open]->},"1\oplus f'_1",swap] \ar[dr,phantom, "\square"] & B \ar[dr,phantom, "\square"]  \ar[d,{Circle[open]->},"q_B"] \ar[r,>->,"g_0"] \ar[l,>->,"1\oplus V'_0",swap] & C \ar[d,{Circle[open]->},"q_C"]\\
B\oplus B' & B\oplus A' \ar[r,>->,swap,"g_0\oplus 1"] \ar[l,>->, "1\oplus f'_0"] & C\oplus A'
\end{tikzcd}
\end{equation}
The LHS square is distinguished by Lemma~\ref{lem:ADTsquares}. For the RHS square, construct 
\begin{equation}
\begin{tikzcd}
& B \ar[d,{Circle[open]}->,swap,"q_B"] \ar[r,>->, dashed,blue,"u"] \ar[dr,phantom,xshift=0.5em, blue,"\square"]& C\ar[d,{Circle[open]}->,"q_C"]\\
A' \ar[r,>->,"p_A"] & B\oplus A'  \ar[r,>->,"g_0\oplus 1"]& C\oplus A'\\
O \ar[u,>->] \ar[r,>->]& B \ar[u,>->,"p_B"]  \ar[r,>->,"g_0"] &  C \ar[u,>->,swap,"p_C"] 
\end{tikzcd}\quad,
\end{equation}
with the obvious restricted pushouts in the bottom rectangle. Since $g_0\oplus 1$ induces a morphism 
	\[\left( A'\rtail B\oplus A'\right)\rtail \left(  A'\rtail C\oplus A'\right)\in \Ar_\Delta \M,\]
apply the functor $k^{-1}\colon \Ar_\triangle \M\to \Ar_{\square}\E$ to obtain the blue indicated distinguished square. Since distinguished squares commute (Definition~\ref{def:goodDC}), deduce from Convention~\ref{conv:restrPO-quotient} that $u=g_0$.
\begin{comment} Details. 
If $\E\subseteq\calC$, then $p_B=q_B$ and $p_C=q_C$. In particular, the two squares yield the identity $q_C\circ u = p_C\circ g_0 = q_C \circ g_0$, and so $u=g_0$ by Axiom (M). If $\E^\opp\subseteq \calC,$ notice the  distinguished square and restricted pushout square define commutative squares in $\calC$, which we can compose. Then, since $q_B\circ p_B=1_B$ and $q_C\circ p_C=1_C$, this gives the identity $u=g_0$. 
\end{comment}


We can therefore apply Axiom (DS) to Diagram~\eqref{eq:dirsum-step1}, which yields the distinguished square
\begin{equation}
\dsquaref{B\oplus V'_0}{C\oplus V'_0}{B\oplus B'}{C\oplus B'}{g_0\oplus 1}{1\oplus f'_1}{w_0}{w_1}\quad.
\end{equation}
In particular, Fact~\ref{facts:restrictedpushouts} shows $w_0=g_0\oplus 1$; extending this, a similar argument as in Step 0 shows $w_1=1\oplus f'_1$.

\begin{comment} Details.
\begin{tikzcd} 
O \ar[r,>->] \ar[d,>->] & A' \ar[d,>->] \ar[r,>->,"f'_0"] & B' \ar[d,>->]\\
B \ar[r,>->] \ar[d,>->,"g_0"]& B\oplus A' \ar[r,>->,"1\oplus f'_0"] \ar[d,>->,"g_0\oplus 1"]& B\oplus B' \ar[d,>->,"g_0\oplus 1"]\\
C \ar[r,>->]& C\oplus A' \ar[r,>->,"1\oplus f'_0"]& C\oplus B'
\end{tikzcd}
And so, reading the RHS column, this gives $w_0=g_0\oplus 1$. 

Next, notice the kernel of $1\oplus f'_1$ is 
$$B\oplus f'_0\colon A'\rtail B'\rtail B\oplus B'.$$ 
When post-composed with $g_0\oplus 1$, we get
$$C\oplus f'_0\colon A'\xrtail{f'_0} B'\rtail C\oplus B',$$
whose formal quotient is $1\oplus f'_1\colon C\oplus V'_0\otail C\oplus B'$.	
\end{comment}


\subsubsection*{Step 2} Assemble the distinguished squares from Steps 0 and 1 into the diagram:
\begin{equation}\label{eq:dirsum-step2a}
\begin{tikzcd}
O \ar[d,{Circle[open]->}] \ar[r,>->] \ar[dr,phantom,"\square"] & V_0 \ar[dr,phantom,"\square"] \ar[d,{Circle[open]->},"f_1",swap]\ar[r,>->] & V_0\oplus V'_0 \ar[dr,phantom,"\square"] \ar[d,{Circle[open]->},"f_1\oplus 1",swap] \ar[r,>->,,"h_0\oplus 1"] & V_1\oplus V'_0 \ar[d,{Circle[open]->}, "g_1\oplus 1"]\\
A \ar[d,{Circle[open]->}]\ar[r,>->,"f_0"] \ar[dr,phantom,blue,xshift=0.4em,"\square"] & B\ar[d,{Circle[open]->}] \ar[r,>->] \ar[dr,phantom,"\square"]  & B\oplus V'_0 \ar[d,{Circle[open]->},"1\oplus f'_1",swap]\ar[r,>->,"g_0\oplus 1"] \ar[dr,phantom,"\square"]  & C\oplus V'_0 \ar[d,{Circle[open]->},"1\oplus f'_1"]\\
A\oplus A' \ar[r,>->,swap,"f_0\oplus 1"] & B\oplus A' \ar[r,>->,swap,"1\oplus f'_0"] & B\oplus B' \ar[r,>->,swap,"g_0\oplus 1"] & C\oplus B'
\end{tikzcd}\qquad .
\end{equation}
All squares are distinguished: the blue square by the same reasoning as in Step~1, the others by Steps~0 – 1 and the setup.

 
Now recall from Equation~\eqref{eq:dirsum-M-morphism} that $f_0\oplus f'_0=(1\oplus f'_0)\circ(f_0\oplus 1)$. Since distinguished squares compose, Axiom~(K) identifies 
\[
(1\oplus f'_1)\circ(f_1\oplus 1)\;\;\sim\;\; f_1\oplus f'_1,
\]
up to codomain-preserving isomorphism. Thus, taking this to be the definition of $f_1\oplus f'_1$, we may replace the middle vertical in Diagram~\eqref{eq:dirsum-step2a} and construct the obvious diagram
\begin{equation}\label{eq:dirsum-step2b}
\begin{tikzcd}
V_0\oplus V_1' \ar[d,{Circle[open]->}] \ar[dr,phantom, "\square"] 
& V_0\oplus V_0' \ar[d,{Circle[open]->},"f_1\oplus f'_1"] \ar[dr,phantom,"\square",xshift=0.5em] 
\ar[l,>->,"1\oplus h'_0",swap] \ar[r,>->,"h_0\oplus 1"] 
& V_1\oplus V'_0 \ar[d,{Circle[open]->}]\\
B\oplus C' 
& B\oplus B' \ar[l,>->,"1\oplus g'_0"] \ar[r,>->,swap,"g_0\oplus 1"] 
& C\oplus B'
\end{tikzcd},
\end{equation}
to which Axiom (DS) applies. 
\begin{comment}
For the permuted LHS, notice Lemma~\ref{lem:ADTsquares} shows you how to permute the coordinates of the distinguished squares. So we can construct 
\begin{equation}
\begin{tikzcd}
V_0 \ar[d,{Circle[open]}->,swap,"q_{V_{0}}"] \ar[dr,phantom,"\square"]& O \ar[l,>->]\ar[d,{Circle[open]}->] \ar[r,>->] \ar[dr,phantom,"\square"]& V_2 \ar[d,{Circle[open]}->,"h'_1"] \\
V_0\oplus V'_0 \ar[d,{Circle[open]}->,swap,"1\oplus f'_1"]  \ar[dr,phantom,"\square"] & V'_0 \ar[l,>->,"V_0\oplus 1",swap]\ar[d,{Circle[open]}->,"f'_1"] \ar[r,>->,"h'_0"] \ar[dr,phantom,"\square"] & V'_1 \ar[d,{Circle[open]}->,"g'_1"] \\
V_0\oplus B' & B'  \ar[l,>->,"V_0\oplus 1"] \ar[r,>->,swap,"g'_0"]& C'
\end{tikzcd}\quad\qquad \dsquaref{V_0\oplus V'_0}{V_0\oplus V'_1}{V_0\oplus B'}{V_0\oplus C'}{1\oplus h'_0}{1\oplus f'_1}{1\oplus g'_0}{\varv'}\quad.
\end{equation}
Then construct

\begin{equation}
\begin{tikzcd}
V_0\oplus B' \ar[d,{Circle[open]->},"f_1\oplus 1",swap] \ar[dr,phantom, "\square"] & B' \ar[dr,phantom, "\square"]  \ar[d,{Circle[open]->},"q_B"] \ar[r,>->,"g'_0"] \ar[l,>->,"V_0\oplus 1",swap] & C' \ar[d,{Circle[open]->},"q_C"]\\
B\oplus B' & A\oplus B ' \ar[r,>->,swap,"1\oplus g'_0"] \ar[l,>->, "f_0\oplus 1"] & A\oplus C'
\end{tikzcd}
\end{equation}

and combine the  diagrams as in Step 2. Notice that we still get the same $\E$-morphism in the middle vertical column because $f_0\oplus f'_0=(f_0\oplus 1)\circ (1\oplus f'_0)$ as well, by definition.
\end{comment}
We thus obtain the desired distinguished square (in solid arrows):
\begin{equation}\label{eq:dirsum-step2c}
\begin{tikzcd}
\color{gray} O \ar[r,>->,gray,dashed]  \ar[d,{Circle[open]}->,dashed,gray] \ar[dr,phantom,yshift=-0.25em, xshift=0.3em,gray,"\square"] & V_0\oplus V'_0 \ar[d,{Circle[open]}->,"f_1\oplus f'_1"] \ar[r,>->,"h_0\oplus h'_0"] \ar[dr,phantom,"\square"]& V_1\oplus V'_1 \ar[d,{Circle[open]}->,"w_2"]\\
\color{gray} A\oplus A'  \ar[r,>->,gray,swap,dashed,"f_0\oplus f'_0"] & B\oplus B'  \ar[r,>->,"g_0\oplus g'_0",swap] & C\oplus C'
\end{tikzcd}\quad.
\end{equation}
It remains to determine $w_2$. Diagram~\eqref{eq:dirsum-step2c} shows $w_2$ is the quotient of 
$(g_0\oplus g'_0)\circ (f_0\oplus f'_0)$. But direct sums commute with composition of $\M$-morphisms, so
\[ (g_0\oplus g'_0)\circ(f_0\oplus f'_0) \;=\; (g_0\circ f_0)\oplus(g'_0\circ f'_0),\]
which corresponds to the formal kernel of $g_1\oplus g'_1$. In other words, $w_2$ and $g_1\oplus g'_1$ are equivalent up to codomain-preserving isomorphism, and so we may identify $w_2=g_1\oplus g'_1$.
	\end{enumerate}
\end{proof}

\begin{comment} Comment out for now.
\begin{remark} Lemma~\ref{lem:DirectSum}, item (v) plays a key role in our analysis, and can be seen as a special case of \cite[Proposition A.12]{SarazolaShapiro}. However, there appears to be a gap in their proof. In their language: given the LHS square below
\[\begin{tikzcd} A'\star_{A} A'' \ar[r,>->,"f_0"] \ar[d,{Circle[open]}->,"g_0"]& B'\star_B B'' \ar[d,{Circle[open]}->,"g_1"]\\
C'\star_C C'' \ar[r,>->,"f_1"]& D'\star_D D''
\end{tikzcd} \qquad\qquad  \begin{tikzcd}
A'\star_{A} A'' \ar[r,>->,"u"] \ar[dr,phantom,"\square"] \ar[d,{Circle[open]}->,"g_0"]& B'\star_B B'' \ar[d,{Circle[open]}->,"g_1"]\\
C'\star_C C'' \ar[r,>->,"f_1"]& D'\star_D D''
\end{tikzcd}, \]
the authors showed that there exists the RHS distinguished square exists. However, it is not clear why $u=f_0$.
\end{remark}
\end{comment}

\section{Technical Lemmas \& Some 2-Simplices}

\subsection{Technical Facts about Sherman Loops}\label{app:Sherman Loops} To finish the proof of Theorem~\ref{thm:Sherman}, we will require the following three technical lemmas. The results here extend the arguments from \cite[\S 1 - 2]{ShermanAbelian} to the setting of pCGW categories.

\begin{lemma}\label{lem:SherLoopAdd} The sum of two Sherman loops is equivalent to a Sherman loop. Explicitly, consider two pairs of $\M$-morphisms 
	$$\alpha_i\colon A_i\to X_i\qquad ,\qquad \beta_i\colon B_i\to Y_i,\qquad \text{for $i=1,2$};$$
and isomorphisms 
$$\theta_i\colon A_i\oplus \frac{X_i}{A_i}\oplus Y_i\longrightarrow B_i\oplus \frac{Y_i}{B_i}\oplus X_i ,\qquad \text{for $i=1,2$}.$$
\underline{Then} 
$$G(\alpha_1,\beta_1,\theta_1)+ G(\alpha_2,\beta_2,\theta_2)=G\left(\alpha_1\oplus \alpha_2,\beta_1\oplus \beta_2,T_2(\theta_1\oplus \theta_2)T_1^{-1}\right)$$ 
in $K_1(\calC)$, whereby $T_1$ and $T_2$ are the canonical permutation isomorphisms
$$T_1\colon A_1\oplus \frac{X_1}{A_1}\oplus Y_1\oplus A_2\oplus \frac{X_2}{A_2}\oplus Y_2\to A_1\oplus A_2\oplus \frac{X_1}{A_1}\oplus \frac{X_2}{A_2}\oplus Y_1\oplus Y_2$$
$$$$
$$T_2\colon B_1\oplus \frac{Y_1}{B_1}\oplus X_1\oplus B_2\oplus \frac{Y_2}{B_2}\oplus X_2\to B_1\oplus B_2\oplus \frac{Y_1}{B_1}\oplus \frac{Y_2}{B_2}\oplus X_1\oplus X_2.$$
$$$$
\end{lemma}
\begin{proof} There are no surprises -- the argument is the same as in \cite[Prop. 1]{ShermanAbelian}. Apply the $H$-space structure of $G\calC$ to construct the loop corresponding to $G(\alpha_1,\beta_1,\theta_1)+G(\alpha_2,\beta_2,\theta_2)$. Notice it is almost identical to $G\left(\alpha_1\oplus \alpha_2,\beta_1\oplus \beta_2,T_2(\theta_1\oplus \theta_2)T_1^{-1}\right)$ -- the only difference being that the direct summands of certain vertices are arranged in a different order. To establish homotopy equivalence, use the obvious isomorphisms to permute these summands and construct a sequence of 2-simplices connecting the two loops. [One will need to justify why the obvious diagrams do in fact define 2-simplices, but this follows from distinguished squares being closed under isomorphisms (Definition~\ref{def:goodDC}).]
\begin{comment}
Applying the $H$-space structure of $|G\calC|$, observe that $G(\alpha_1,\beta_1,\theta_1)+G(\alpha_2,\beta_2,\theta_2)$ is represented by the loop
\begin{equation}\label{eq:sherloop+}
\begin{tikzcd}
\ones{A_1\oplus \frac{X_1}{A_1}\oplus Y_1\oplus A_2\oplus \frac{X_2}{A_2}\oplus Y_2}{X_1\oplus Y_1\oplus X_2\oplus Y_2} \ar[rr,blue,"\ones{\theta_1\oplus \theta_2}{\tau_1}"]&& \ones{B_1\oplus \frac{Y_1}{B_1}\oplus X_1\oplus B_2\oplus \frac{Y_2}{B_2}\oplus X_2}{Y_1\oplus X_1\oplus Y_2\oplus X_2} \\
\ones{A_1\oplus A_2}{A_1\oplus A_2} \ar[u]&& \ones{B_1\oplus B_2}{B_1\oplus B_2} \ar[u]  \\
&\ones{O}{O} \ar[ul] \ar[ur]
\end{tikzcd}
\end{equation}
The 1-simplices are the obvious ones, with the middle 1-simplex applying $\theta_1\oplus \theta_2$ on the top row, and the permutation isomorphism $\tau_1$ on the bottom -- this is induced by the universal property of restricted pushouts, and behaves as expected. (Convention~\ref{conv:perm-iso})
 The loop corresponding to $G\left(\alpha_1\oplus \alpha_2,\beta_1\oplus \beta_2,T_2(\theta_1\oplus \theta_2)T_1^{-1}\right)$ is defined similarly, except with the summands permuted. 
\begin{equation}\label{eq:sherloop+2}
\begin{tikzcd}
\ones{A_1\oplus A_2 \oplus \frac{X_1}{A_1}\oplus \frac{X_2}{A_2}\oplus Y_1\oplus Y_2}{X_1\oplus X_2\oplus Y_1\oplus Y_2} \ar[rr,red,"\ones{T_2(\theta_1\oplus \theta_2)T_1^{-1}}{\tau_2}"]&& \ones{B_1\oplus B_2 \oplus \frac{Y_1}{B_1}\oplus \frac{Y_2}{B_2}\oplus X_1\oplus X_2}{Y_1\oplus Y_2\oplus X_1\oplus X_2} \\
\ones{A_1\oplus A_2}{A_1\oplus A_2} \ar[u]&& \ones{B_1\oplus B_2}{B_1\oplus B_2} \ar[u]  \\
&\ones{O}{O} \ar[ul] \ar[ur]
\end{tikzcd}
\end{equation}
where $\tau_2$ is the obvious permutation isomorphism on the bottom row. To show that, e.g. $A_1\oplus A_2\rtail X_1\oplus X_2\oplus Y_1\oplus Y_2$ have the quotients $\frac{X_1}{A_1}\oplus \frac{X_2}{A_2}\oplus Y_1\oplus Y_2$, use Lemma~\ref{lem:DirectSum}.


To show the two loops are equivalent, consider the diagram below.

\begin{equation}\label{eq:SherPlusFINAL}
\small{\begin{tikzcd}
&\ones{A_1\oplus A_2}{A_1\oplus A_2} \ar[ddl] \ar[ddr]\\
\\
\ones{A_1\oplus \frac{X_1}{A_1}\oplus Y_1\oplus A_2\oplus \frac{X_2}{A_2}\oplus Y_2}{X_1\oplus Y_1\oplus X_2\oplus Y_2} \ar[rr,"\ones{T_1}{\tau_3}"] \ar[ddd,blue,swap,"\ones{\theta_1\oplus \theta_2}{\tau_1}"] \ar[dddrr,swap,"\ones{T_2(\theta_1\oplus \theta_2)}{\tau_4\circ \tau_1}"]  &&  \ones{A_1\oplus A_2 \oplus \frac{X_1}{A_1}\oplus \frac{X_2}{A_2}\oplus Y_1\oplus Y_2}{X_1\oplus X_2\oplus Y_1\oplus Y_2} \ar[ddd,red,"\ones{T_2(\theta_1\oplus \theta_2)T_1^{-1}}{\tau_2}"]\\
\\
\\
\ones{B_1\oplus \frac{Y_1}{B_1}\oplus X_1\oplus B_2\oplus \frac{Y_2}{B_2}\oplus X_2}{Y_1\oplus X_1\oplus Y_2\oplus X_2} \ar[rr,swap,"\ones{T_2}{\tau_4}"]  && \ones{B_1\oplus B_2 \oplus \frac{Y_1}{B_1}\oplus \frac{Y_2}{B_2}\oplus X_1\oplus X_2}{Y_1\oplus Y_2\oplus X_1\oplus X_2}\\
\\
& \ones{B_1\oplus B_2}{B_1\oplus B_2} \ar[uur] \ar[uul]
\end{tikzcd}}
\end{equation}
where $\tau_3,\tau_4$ are the obvious permutation isomorphisms.
An easy check shows all the triangles in Diagram~\eqref{eq:SherPlusFINAL} are boundaries of 2-simplices. 


For the middle rectangle, these are all isomorphisms (and thus with trivial quotient), so it suffices to show that the $\M$-morphisms commute. The fact that $\tau_4\circ\tau_1= \tau_2\circ\tau_3$ again comes from how permutation isomorphisms are defined via the universal property of restricted pushouts; it induces a unique isomorphism permuting the summands. And so since $\tau_2\circ\tau_3$ has the same domain/codomain, it must be the same as $\tau_4\circ\tau_1$.



As for the top/bottom triangles, the top triangle bounds the 2-simplex
\[
\small{\begin{tikzcd}
A_1\oplus A_2 \ar[r, >->] & A_1\oplus \frac{X_1}{A_1}\oplus Y_1\oplus A_2\oplus \frac{X_2}{A_2}\oplus Y_2\ar[dr,phantom,"\square"] \ar[r, >->] & A_1\oplus A_2 \oplus \frac{X_1}{A_1}\oplus \frac{X_2}{A_2}\oplus Y_1\oplus Y_2\\
&	\frac{X_1}{A_1}\oplus Y_1\oplus \frac{X_2}{A_2}\oplus Y_2\ar[r, >->] \ar[u, {Circle[open]}->]& \frac{X_1}{A_1}\oplus \frac{X_2}{A_2}\oplus Y_1\oplus Y_2\ar[u, {Circle[open]}->] \\
&& O \ar[u, {Circle[open]}->] 
\end{tikzcd}}	
\]
\[
\small{\begin{tikzcd}
A_1\oplus A_2 \ar[r, >->] & X_1\oplus Y_1\oplus X_2\oplus Y_2 \ar[dr,phantom,"\square"] \ar[r, >->] & X_1\oplus X_2\oplus Y_1\oplus Y_2 \\
&	\frac{X_1}{A_1}\oplus Y_1\oplus \frac{X_2}{A_2}\oplus Y_2\ar[r, >->] \ar[u, {Circle[open]}->]& \frac{X_1}{A_1}\oplus \frac{X_2}{A_2}\oplus Y_1\oplus Y_2\ar[u, {Circle[open]}->] \\
&& O \ar[u, {Circle[open]}->] 
\end{tikzcd}}	
\]

The main thing to check is why the indicated squares are distinguished, and why they are bounded by the correct triangle. But this is straightforward since distinguished squares are closed under isomorphisms. Consider, e.g. 
\[\begin{tikzcd}
 A_1\oplus A_2 \ar[r,>->]& X_1\oplus Y_1\oplus X_2\oplus Y_2 \ar[dr,phantom,"\square"] \ar[r, >->,"1"] &  X_1\oplus Y_1\oplus X_2\oplus Y_2 \ar[r,>->,"\delta"] \ar[dr,phantom,"\square"] &  X_1\oplus X_2\oplus Y_1\oplus Y_2 \\
&	\frac{X_1}{A_1}\oplus Y_1\oplus \frac{X_2}{A_2}\oplus Y_2\ar[r, >->,"\zeta"] \ar[u, {Circle[open]}->,"g"] & \frac{X_1}{A_1}\oplus \frac{X_2}{A_2}\oplus Y_1\oplus Y_2\ar[u, {Circle[open]}->,"g\circ \varphi(\zeta)"] \ar[r,>->,"1"] & \frac{X_1}{A_1}\oplus \frac{X_2}{A_2}\oplus Y_1\oplus Y_2\ar[u, {Circle[open]}->,"\varphi(\delta)\circ g\circ \varphi(\zeta)"] &
\end{tikzcd};\]
the indicated squares are distinguished since they commute and the $\M$-morphisms are isomorphisms. Hence, the outer rectangle also is a distinguished square. 

One then checks that the top line $\M$-morphism composes to give the correct $\M$-morphism corresponding to the Sherman loop. [Why? Take repeated restricted pushouts
$$\begin{tikzcd}
&&& O \ar[r,>->] \ar[d,>->]& Y_2 \ar[d,>->] \\
&O \ar[r,>->] \ar[d,>->] & A_2 \ar[d,>->] \ar[r,>-> ]& X_2 \ar[d,>->] \ar[r,>-> ] & X_2\oplus Y_2 \ar[d,>->] \\
&A_1 \ar[d,>->] \ar[r,>-> ] & A_1\oplus A_2  \ar[dr,>->,teal,"\alpha_1\oplus \alpha_2"]\ar[d,>->,red,"\alpha_1\oplus 1",swap] \ar[r,>-> ]& A_1\oplus X_2\ar[d,>->] \ar[r,>->] & A_1\oplus X_2\oplus Y_2 \ar[d,>->] \\
O \ar[r,>->] \ar[d,>->]&X_1 \ar[d,>->] \ar[r,>-> ]& X_1\oplus A_2 \ar[d,>->,red] \ar[r,>-> ]& X_1\oplus X_2 \ar[d,>->] \ar[r,>->,violet] & X_1\oplus X_2\oplus Y_2 \ar[d,>->,violet]\\
Y_1 \ar[d,>->,"1",blue]\ar[r,>->]&X_1\oplus Y_1 \ar[dr,phantom,"\circlearrowleft"]\ar[d,>->,"1",blue]\ar[r,>->] & X_1 \oplus Y_1\oplus A_2 \ar[dr,phantom,"\circlearrowleft"] \ar[r,>->,red,"1\oplus 1\oplus \alpha_2"] \ar[d,>->,"\tau"] & X_1\oplus Y_1\oplus X_2 \ar[d,>->,"\tau"] \ar[r,>->,"1\oplus 1\oplus 1\oplus Y_2",red]  \ar[dr,phantom,"\circlearrowleft"] & X_1\oplus Y_1\oplus X_2\oplus Y_2 \ar[d,>->,"\delta",purple]\\
Y_1 \ar[r,>->]& X_1\oplus Y_1 \ar[r,>->]& X_1\oplus A_2\oplus Y_1 \ar[r,>->,"1\oplus \alpha_2\oplus 1"] & X_1\oplus X_2\oplus Y_1 \ar[r,>->,"1\oplus 1\oplus 1\oplus Y_2",swap] & X_1\oplus X_2\oplus Y_1\oplus Y_2
\end{tikzcd}$$
to get the isomorphism that gets us $A_1\oplus A_2\rtail X_1\oplus X_2\oplus Y_1\oplus Y_2$. Notes:
\begin{itemize}
\item All unmarked squares are restricted pushouts. 
\item Those indicated with a rotating arrow are commutative $\M$-squares\footnote{Although since the arrows involved are isomorphisms, they are restricted pushouts too.}, obtained by the fact that permutations commute with direct sums, Lemma~\ref{lem:DirectSum} (iii) + the universal property of restricted pushouts.
\item The direct sum of the two Sherman Loops: the $\M$-morphism is the red arrows.
\item Given two Sherman loops, we have the freedom to define a new Sherman Loop $G(\alpha_1\oplus \alpha_2, \dots )$ so that our analysis goes through. As such, define its $\M$-morphisms as follows: Compose $\alpha_1\oplus\alpha_2$ with the purple arrows + $\delta$. Notice the purple violet arrows correspond to just adding $Y_1\oplus Y_2$ to $X_1\oplus X_2$.
\end{itemize}

So the $\M$-morphisms line up since we were just permuting. We claim this corresponds to the 1-simplex of the Sherman Loop; one may ask why we are justified in identifying the $\E$-morphism of the 1-simplex of the Sherman loop with the one we just defined, but this follows from the $\M$-morphisms being the same. In other words, the quotients agree up to codomain-preserving isomorphism, so this is OK .

And thus proves the lemma.
\end{comment}
\end{proof}

\begin{lemma}\label{lem:SherLoopSplit} Let $\calC$ be a pCGW category whose exact squares all split. \underline{Then}, every element of $K_1(\calC)$ is equivalent to a loop of the form
\begin{equation}
G(A,\alpha):=\left(\begin{tikzcd}
(A,A) \ar[rr,"l(\alpha)"] && (A,A)\\
& (O,O) \ar[ul] \ar[ur]
\end{tikzcd}\right)
\end{equation}
where $l(\alpha)$ is the 1-simplex
\[l(\alpha):=\left(\, \dsquaref{O}{O}{A}{A}{ }{ }{1}{} \quad,\quad  \dsquaref{O}{O}{A}{A}{ }{ }{\alpha}{} \,\right)\]	
for some automorphism $(A,\alpha)\in\Aut(\calC)$.
\end{lemma}
\begin{proof} The proof combines an argument from \cite[\S 5]{GG} and \cite[Prop. 2]{ShermanAbelian}. Proceed in stages.
\subsubsection*{Step 1: Combinatorial Loops in $K_1(\calC)$} Suppose $z\in K_1(\calC)=\pi_1|G\calC^o|$. By the simplicial approximation theorem, $z$ can be represented by a loop formed combinatorially from 1-simplices of $G\calC$
\begin{equation}\label{eq:z1stloop}
\ones{O}{O} \to \bullet \leftarrow \bullet \to \dots \leftarrow \bullet\to \bullet \leftarrow \ones{O}{O}
\end{equation}
where we draw the 1-simplices as arrows. %We claim this loop is homotopic to one of the form
%$$\ones{O}{O} \to \bullet \to \dots \to \bullet \leftarrow \dots \leftarrow \bullet \leftarrow \ones{O}{O}.$$	
Consider one of the configurations in Diagram~\eqref{eq:z1stloop}, e.g. 
$$\ones{M'}{L'}\leftarrow \ones{M}{L} \to \ones{M''}{L''}.$$
By Corollary~\ref{cor:RPtrick}, this is homotopic to 
$$\ones{M'}{L'}\to \ones{M'\star_M M''}{L'\star_L L''} \leftarrow \ones{M''}{L''}.$$
%Abstractly, this turns a configuration 
%$$\bullet \leftarrow \bullet \to \bullet \qquad\text{into} \qquad \bullet\to \bullet\leftarrow\bullet \qquad .$$
Applying this trick multiple times, we can deform Loop~\eqref{eq:z1stloop} into one of the form
\begin{equation}\label{eq:niceloop}
\begin{tikzcd}
\ones{O}{O} \ar[r] \ar[d] & \ones{M_0}{L_0} \ar[r]& \dots\ar[r] & \ones{M_{q-1}}{L_{q-1}} \ar[d]\\
\ones{M'_0}{L'_0} \ar[r]& \dots \ar[r] & \ones{M'_{q-1}}{L'_{q-1}} \ar[r] & \ones{M}{L} 
\end{tikzcd}
\end{equation}
	\subsubsection*{Step 2: The Base Case} Start by analysing the component
\begin{equation}\label{eq:BaseCase1Simp}
\ones{O}{O}\xrightarrow{l_0} \ones{M_0}{L_0}\xrightarrow{l_1} \ones{M_1}{L_1}
\end{equation}
	 of Loop~\eqref{eq:niceloop} in $K_1(\calC)$. Suppose $l_0$ is defined by the following pair of exact squares
\begin{equation}\label{eq:Loop1stC}
l_0:=\left(\dsquaref{O}{\widehat{M_0}}{O}{M_0}{ }{ }{ }{\eta_0}\quad,\quad \dsquaref{O}{\widehat{M_0}}{O}{L_0}{ }{ }{ }{\mu_0}\right),
\end{equation}
	with isomorphisms $\eta_0$ and $\mu_0$. Recall that all isomorphisms in $\calC$ can be regarded as either an $\M$ or an $\E$-morphism, related by the functor $$\varphi\colon \mathrm{iso}\M\to \mathrm{iso}\E,$$ 
	an isomorphism of categories. We can therefore define two 1-simplices
			\[l_0':=\left(\dsquaref{O}{\widehat{M_0}}{O}{\widehat{M_0}}{ }{ }{ }{1}\quad,\quad \dsquaref{O}{\widehat{M_0}}{O}{ \widehat{M_0}}{ }{ }{ }{1}\right) \quad \text{and} \quad l_0'':=\left(
\begin{tikzcd}
O \ar[rr, >->] \ar[d,{Circle[open]->}] \ar[drr,phantom,"\square"]&&  O \ar[d,{Circle[open]->}]   \\
\widehat{M_0} \ar[rr,>->,"\varphi^{-1}(\eta_0)"] && M_0
\end{tikzcd} \,,\,       
\begin{tikzcd}
O \ar[rr, >->] \ar[d,{Circle[open]->}] \ar[drr,phantom,"\square"]&&  O \ar[d,{Circle[open]->}]   \\
\widehat{M_0} \ar[rr,>->,"\varphi^{-1}(\mu_0)"] && L_0
\end{tikzcd}
\right),\]
which assemble into the following 2-simplex
\[\begin{tikzcd}
O \ar[r, >->] \ar[dr,phantom,"\square"]&  \widehat{M_0} \ar[dr,phantom,"\square"] \ar[r, >->,"\varphi^{-1}(\eta_0) "] & M_0\\
O \ar[u,{Circle[open]->}] \ar[r,>->] & \widehat{M_0}\ar[u,{Circle[open]->},"1"] \ar[r,>->,"1"]	\ar[dr,phantom,"\square"]& \widehat{M_0}\ar[u, {Circle[open]}->,swap,"\eta_0"] \\
& O \ar[u,{Circle[open]}->] \ar[r,>->] & O  \ar[u, {Circle[open]}->] 
\end{tikzcd}\quad 
\begin{tikzcd}
	O \ar[r, >->] \ar[dr,phantom,"\square"]&  \widehat{M_0} \ar[dr,phantom,"\square"] \ar[r, >->,"\varphi^{-1}(\mu_0) "] & L_0\\
	O \ar[u,{Circle[open]->}] \ar[r,>->] & \widehat{M_0}\ar[u,{Circle[open]->},"1"] \ar[r,>->,"1"]	\ar[dr,phantom,"\square"]& \widehat{M_0}\ar[u, {Circle[open]}->,swap,"\mu_0"] \\
	& O \ar[u,{Circle[open]}->] \ar[r,>->] & O  \ar[u, {Circle[open]}->] 
	\end{tikzcd}.\]
Notice the top right squares are distinguished since distinguished squares are closed under isomorphisms. We then assemble the following diagram
\begin{equation}
\begin{tikzcd}
&& \ones{ \widehat{M_0}}{ \widehat{M_0}} \ar[d,"l''_0"] \ar[drr,blue,"l_1\circ l''_0"]\\
\ones{O}{O} \ar[urr,blue,"l'_0"] \ar[rr,red,"l_0"] && \ones{M_0}{L_0} \ar[rr,red,"l_1"]&& \ones{M_1}{L_1}
\end{tikzcd}.
\end{equation}
One easily checks the RHS triangle also bounds a 2-simplex, which implies the red path of 1-simplices is homotopic to the blue path. Hence, assume without loss of generality that both $\mu_0$ and $\eta_0$ of Equation~\eqref{eq:Loop1stC} are the identity $1\colon M_0\to M_0$.
\begin{comment} We've shown the LHS triangle. For the RHS triangle

\[\begin{tikzcd}
\widehat{M_0} \ar[r, >->,"\varphi^{-1}(\eta_0)"] \ar[dr,phantom,"\square"]&  M_0 \ar[dr,phantom,"\square"] \ar[r, >->,"\eta_1 "] & M_1\\
O \ar[u,{Circle[open]->}] \ar[r,>->] & O \ar[u,{Circle[open]->}] \ar[r,>->]	\ar[dr,phantom,"\square"]& \frac{M_1}{M_0}\ar[u, {Circle[open]}->,swap,"\eta'_1"] \\
& O \ar[u,{Circle[open]}->] \ar[r,>->] & \frac{M_1}{M_0}  \ar[u, {Circle[open]}->] 
\end{tikzcd}\quad 
\begin{tikzcd}
\widehat{M_0} \ar[r, >->,"\varphi^{-1}(\mu_0)"] \ar[dr,phantom,"\square"]&  L_0 \ar[dr,phantom,"\square"] \ar[r, >->," \mu_1"] & L_1\\
O \ar[u,{Circle[open]->}] \ar[r,>->] & O\ar[u,{Circle[open]->}] \ar[r,>->]	\ar[dr,phantom,"\square"]& \frac{M_1}{M_0} \ar[u, {Circle[open]}->,swap,"\mu'_1"] \\
& O \ar[u,{Circle[open]}->] \ar[r,>->] & \frac{M_1}{M_0} \ar[u, {Circle[open]}->] 
\end{tikzcd}.\]
\end{comment}


\subsubsection*{Step 3: The Inductive Step} Proceeding along Loop~\eqref{eq:BaseCase1Simp}, consider $l_1$, to be represented as\footnote{To improve readability, quotients here are denoted suggestively as e.g. $\frac{M_1}{M_0}$ but the proof works for any quotient representative (not necessarily the canonical one from Axiom (K)). This is the same abuse of notation made in Corollary~\ref{cor:RPtrick}.}
\[l_1:=\left(\dsquaref{O}{\frac{M_1}{M_0}}{M_0}{M_1}{ }{ }{ \eta_1}{\eta'_1}\quad,\quad \dsquaref{O}{\frac{M_1}{M_0}}{M_0}{L_1}{ }{ }{\mu_1}{\mu'_1}\right).\]
Since all exact squares split by hypothesis, this yields isomorphisms
$$\psi_M\colon   M_1\rtail M_0\oplus \frac{M_1}{M_0}\qquad \text{and} \qquad \psi_L\colon L_1\rtail M_0\oplus \frac{M_1}{M_0},$$
which we shall consider as $\M$-morphisms. This defines the following distinguished squares
\[\begin{tikzcd}
O \ar[r, >->] \ar[dr,phantom,"\square"]\ar[d,{Circle[open]->}] &  \frac{M_1}{M_0} \ar[dr,phantom,"\square"] \ar[r, >->,"1"] \ar[d,{Circle[open]->}," \eta'_1"]&  \frac{M_1}{M_0}\ar[d, {Circle[open]}->,"\varphi(\psi_M)\circ \eta'_1"]   \\
M_0 \ar[r,>->,"\eta_1"] &  M_1  \ar[r,>->,"\psi_M"]&   M_0\oplus \frac{M_1}{M_0} 
\end{tikzcd}\qquad \begin{tikzcd}
O \ar[r, >->] \ar[dr,phantom,"\square"]\ar[d,{Circle[open]->}] &  \frac{M_1}{M_0} \ar[dr,phantom,"\square"] \ar[r, >->,"1"] \ar[d,{Circle[open]->}," \mu'_1"]&   \frac{M_1}{M_0}\ar[d, {Circle[open]}->,"\varphi(\psi_L)\circ \mu'_1"]   \\
M_0 \ar[r,>->,"\mu_1"] &  L_1  \ar[r,>->,"\psi_L"]&   M_0\oplus \frac{M_1}{M_0} 
\end{tikzcd}\]
By Axiom (I), the RHS squares commute and are thus distinguished. Hence, the horizontal compositions define two exact squares. In fact, we can say more. To ease notation, denote 
$$v:=\psi_M\circ\eta_1 \qquad \text{and}\qquad v':=\varphi(\psi_M)\circ \eta'_1,$$
$$w:=\psi_L\circ\mu_1 \qquad \text{and}\qquad w':=\varphi(\psi_L)\circ \mu'_1,$$
and denote
\[\dsquaref{O}{\frac{M_1}{M_0}}{M_0}{M_0\oplus \frac{M_1}{M_0}}{}{ }{ p}{q}\]
to be the canonical direct sum square of $M_0\oplus \frac{M_1}{M_0}$.  Since both exact squares in $l_1$ are split, we know that
\begin{equation}\label{eq:IndCaseIdentities}
v=  p =w
\end{equation}
$$ v'= q = w'.$$
\begin{comment} WAS:
\begin{equation}\label{eq:IndCaseIdentities}
v=\psi_M\circ \eta_1 = p = \psi_L\circ \mu_1=w
\end{equation}
$$ v'=\varphi(\psi_M)\circ \eta'_1  = q = \varphi(\psi_L)\circ \mu'_1 =w'.$$
\end{comment}
Leverage these identities to construct the following 2-simplex
\begin{equation}\label{eq:IndCase1}
\begin{tikzcd}
O \ar[r, >->] \ar[dr,phantom,"\square"]&  M_0 \ar[dr,phantom,blue,"\square"] \ar[r, >->,"\eta_1"] & M_1\\
O \ar[u,{Circle[open]->}] \ar[r,>->] &  M_0\ar[u,{Circle[open]->}," 1"] \ar[r,>->,"v"] \ar[dr,phantom,"\square"]&   M_0\oplus \frac{M_1}{M_0} \ar[u, {Circle[open]}->,swap,"\varphi(\psi^{-1}_M)"]  \\
&O \ar[r,>->] \ar[u, {Circle[open]}->] & \frac{M_1}{M_0} \ar[u, {Circle[open]}->,swap,"q"]
\end{tikzcd} \qquad \begin{tikzcd}
O \ar[r, >->] \ar[dr,phantom,"\square"]&  M_0 \ar[dr,phantom,red,"\square"] \ar[r, >->,"\mu_1"] & L_1\\
O \ar[u,{Circle[open]->}] \ar[r,>->] &  M_0\ar[u,{Circle[open]->}," 1"] \ar[r,>->,"v"] \ar[dr,phantom,"\square"]&   M_0\oplus \frac{M_1}{M_0} \ar[u, {Circle[open]}->,swap,"\varphi(\psi_L^{-1})"]  \\
&O \ar[r,>->] \ar[u, {Circle[open]}->] & \frac{M_1}{M_0} \ar[u, {Circle[open]}->,swap,"q"]
\end{tikzcd} \quad. 
\end{equation} 
To see why the red-indicated square is distinguished, notice:
\begin{itemize}
	\item\textbf{Case 1.} Suppose $\E$ is a subcategory of $\calC$. Then 
$$	\varphi(\psi_L^{-1} )\circ v=\psi^{-1}_L\circ v=\mu_1 \qquad \text{in $\calC $}$$
	if and only if
	$$ v = \psi_L\circ \mu_1,$$
which holds by Identity~\eqref{eq:IndCaseIdentities}.
	\item \textbf{Case 2.} Suppose  $\E^{\opp}$ is a subcategory of $\calC$, and so the arrows reverse. Applying Identity~\eqref{eq:IndCaseIdentities} once more, deduce
	$$	v= \psi_L\circ \mu_1 = \varphi(\psi_L^{-1})\circ \mu_1 .$$

\end{itemize}
The claim then follows from distinguished squares being closed under isomorphisms. The case for the blue-indicated square is analogous. 
\begin{comment}
\begin{itemize}
\item\textbf{Case 1.} Suppose $\E$ is a subcategory of $\calC$. Then 
$$	\varphi(\psi_M^{-1})\circ v=\psi^{-1}_M\circ v=\eta_1$$
if and only if
$$ v = \psi_M\circ \eta_1,$$
which holds by definition of $v$.
\item \textbf{Case 2.} Suppose  $\E^{\opp}$ is a subcategory of $\calC$. Then, 
$$	v= \varphi(\psi_M^{-1})\circ \eta_1 = \psi_M \circ \eta_1$$
which again holds by definition.
\end{itemize}
\end{comment}


Having checked that Diagram~\eqref{eq:IndCase1} defines a 2-simplex, the gears line up and the inductive argument falls into place. First note that Diagram~\eqref{eq:IndCase1} defines a homotopy between 
$$\ones{O}{O}\xrightarrow{l_0} \ones{M_0}{L_0}\xrightarrow{l_1} \ones{M_1}{L_1}$$
\begin{comment}
Notice that $v'=w'$, so we do in fact get the right $\E$-morphism when we compose the distinguished squares vertically to get $l_1$.
\end{comment}
and a 1-simplex
$$l'_1\colon \ones{O}{O}\to \ones{M_1}{L_1}$$
whereby
\[l'_1:=\left(\dsquaref{O}{M_0\oplus \frac{M_1}{M_0}}{O}{M_1}{ }{ }{ }{}\quad,\quad \dsquaref{O}{M_0\oplus \frac{M_1}{M_0}}{O}{L_1}{ }{ }{}{}\right).\]
Hence, the initial segment of Loop~\eqref{eq:niceloop} is homotopic to
$$\ones{O}{O}\xrightarrow{l'_1}\ones{M_1}{L_1}\xrightarrow{l_2}\ones{M_2}{L_2},$$
deleting a term. Then, apply the Base Case argument (Step 2) to justify presenting $l'_1$ as
\[l'_1:=\left(\dsquaref{O}{M_1}{O}{M_1}{ }{ }{ }{1}\quad,\quad \dsquaref{O}{\ M_1}{O}{M_1}{ }{ }{}{1}\right),\]
which sets up our inductive step again. Keep going for the rest of Loop~\eqref{eq:niceloop} on both sides, until we finally obtain a loop of the form
\begin{comment}
\begin{equation}\label{eq:IndLoop}
\ones{O}{O} \to \ones{M_{q-1}}{M_{q-1}}\to (M,L) \leftarrow  \ones{M'_{q-1}}{M'_{q-1}} \leftarrow \ones{O}{O}
\end{equation}
At which point, we can apply the inductive argument on both sides once more to obtain the loop
\end{comment}
\begin{equation}\label{eq:IndCaseFinalLoop}
\ones{O}{O} \xrightarrow{\kappa} (M,L) \xleftarrow{\gamma}  \ones{O}{O},
\end{equation}
where 
\[\kappa:=\left(\dsquaref{O}{\widehat{M}}{O}{M}{ }{ }{ }{\kappa_0}\quad,\quad \dsquaref{O}{\widehat{M}}{O}{L}{ }{ }{}{\kappa_1}\right) \qquad\text{and}\qquad \gamma:=\left(\dsquaref{O}{\widehat{N}}{O}{M}{ }{ }{ }{\gamma_0}\quad,\quad \dsquaref{O}{\widehat{N}}{O}{L}{ }{ }{}{\gamma_1}\right)\]
Notice we can no longer apply the Base Case argument to simplify $\kappa$ or $\gamma$ since the arrows of Diagram~\eqref{eq:IndCaseFinalLoop} are in the wrong direction.
	\subsubsection*{Step 4: Finish}	A technical observation: both $\kappa$ and $\gamma$ define isomorphisms $L\xrightarrow{\cong} M$  in $\calC$ but the presentation will differ depending on whether $\E^\opp$ or $\E$ is a subcategory of $\calC$.
	\begin{itemize}
		\item[] \textbf{Case 1:} {\em $\E\subseteq \calC$.}  In which case, define $\omega:= \kappa_0\circ \kappa_1^{-1}$  and $ \lambda:=\gamma_0\circ \gamma_1^{-1}$ in $\calC$
		\item[] \textbf{Case 2:} {\em $\E^\opp \subseteq \calC$.} In which case, define $\omega:= \kappa_0^{-1}\circ \kappa_1$ and $\lambda:=\gamma^{-1}_0\circ \gamma_1$ in $\calC$.
	\end{itemize}
By Axiom (I), we may regard $\omega$ and $\lambda$ as $\M$-morphisms as well. We now construct the obvious diagram
\begin{equation}\label{eq:IndFINAL}
\begin{tikzcd}
&\ones{M}{M } \ar[r,blue,"g"]& \ones{M}{M}  \\
\ones{O}{O} \ar[ur,"f_0",blue]\ar[r,"\kappa",red] & \ones{M}{L} \ar[u,"f_1"]\ar[ur,"f_2"] & \ar[l,swap,"\gamma",red] \ones{O}{O}\ar[u, swap,"f_0",blue] 
\end{tikzcd}
\end{equation}	
whereby

\[g:= \left(\dsquaref{O}{O}{M}{M}{}{}{1}{} \quad,\quad \dsquaref{O}{O}{M}{M}{}{}{\lambda\circ\omega^{-1}}{ } \right)\qquad f_0:=\left(\dsquaref{O}{M}{O}{M}{}{}{}{1} \quad,\quad \dsquaref{O}{M}{O}{M}{}{}{}{1} \right).\]
\[f_1:=\left(\dsquaref{O}{O}{M}{M}{}{}{1}{} \quad,\quad \dsquaref{O}{O}{L}{M}{}{}{\omega}{} \right)\qquad f_2:=\left(\dsquaref{O}{O}{M}{M}{}{}{1}{} \quad,\quad \dsquaref{O}{O}{L}{M}{}{}{\lambda}{} \right) . \]
In both cases ($\E$ or $\E^\opp\subseteq \calC$), it is easy to check that the triangles of Diagram~\eqref{eq:IndFINAL} bound 2-simplices, e.g. 
\[f_1\kappa = f_0 \qquad \small{\left(\begin{tikzcd}
O \ar[r, >->," "] & M \ar[dr,phantom,"\square"] \ar[r,>->,"1"]& M \\
& \widehat{M}\ar[u,{Circle[open]->},"\kappa_0"] \ar[r,>->,swap,"\varphi^{-1}(\kappa_0)"]	 & M\ar[u, {Circle[open]}->,swap,"1"]  \\
&   & O \ar[u, {Circle[open]}->]  
\end{tikzcd}\qquad 
\begin{tikzcd}
O \ar[r, >->," "] & L \ar[dr,phantom,"\square"] \ar[r,>->,"\omega"]& M \\
& \widehat{M}\ar[u,{Circle[open]->},"\kappa_1"] \ar[r,>->,swap,"\varphi^{-1}(\kappa_0)"]	 & M \ar[u, {Circle[open]}->,swap,"1"]  \\
&   & O \ar[u, {Circle[open]}->]  
\end{tikzcd}\right)}.\]
\begin{comment} Details.
\[f_2\gamma = f_0 \qquad \small{\left(\begin{tikzcd}
O \ar[r, >->," "] & M \ar[dr,phantom,"\square"] \ar[r,>->,"1"]& M \\
& \widehat{N}\ar[u,{Circle[open]->},"\gamma_0"] \ar[r,>->,swap,"\varphi^{-1}(\gamma_0)"]	 & M \ar[u, {Circle[open]}->,swap,"1"]  \\
&   & O \ar[u, {Circle[open]}->]  
\end{tikzcd}\qquad 
\begin{tikzcd}
O \ar[r, >->," "] & L \ar[dr,phantom,"\square"] \ar[r,>->,"\lambda"]& M \\
& \widehat{N}\ar[u,{Circle[open]->},"\gamma_1"] \ar[r,>->,swap,"\varphi^{-1}(\gamma_0)"]	 & M\ar[u, {Circle[open]}->,swap,"1"]  \\
&   & O \ar[u, {Circle[open]}->]  
\end{tikzcd}\right)}.\]
\[gf_1 = f_2 \qquad \small{\left(\begin{tikzcd}
M \ar[r, >->,"1"] & M \ar[dr,phantom,"\square"] \ar[r,>->,"1"]& M \\
& O \ar[u,{Circle[open]->}] \ar[r,>->]	 & O \ar[u, {Circle[open]}->]  \\
&   & O \ar[u, {Circle[open]}->]  
\end{tikzcd}\qquad 
\begin{tikzcd}
L \ar[r, >->,"\omega "] & M \ar[dr,phantom,"\square"] \ar[r,>->,"\lambda\circ \omega^{-1}"]& M \\
& O\ar[u,{Circle[open]->}] \ar[r,>->]	 & O\ar[u, {Circle[open]}->]  \\
&   & O \ar[u, {Circle[open]}->]  
\end{tikzcd}\right)}.\]
\end{comment}
Conclude that the red loop in Diagram~\eqref{eq:IndFINAL} is homotopic to the blue loop. Notice the blue loop is precisely of the form $G(A,\alpha)$ as claimed in lemma statement, with $A:=M$ and $\alpha:=\lambda\circ \omega^{-1}$.
\end{proof}

\begin{lemma}\label{lem:SherSplitSher} The automorphism loop $G(A,\alpha)$ in Lemma~\ref{lem:SherLoopSplit} is equivalent to a Sherman Loop.
\end{lemma}
\begin{proof} Let $p_A\colon A\rtail A\oplus A$ be the canonical $\M$-morphism of a direct sum square, and $\tau_A\colon A\oplus A\rtail A\oplus A$ be the permutation isomorphism swapping components. Define the following loop 
\begin{equation}\label{eq:SherLoopSplit}
\ones{O}{O} \to \ones{A}{A} \xrightarrow{\iota_\alpha} \ones{A\oplus A}{A\oplus A} \xrightarrow{l_{\tau}}  \ones{A\oplus A}{A\oplus A} \leftarrow  \ones{O}{O}
\end{equation}
where 
$$\iota_\alpha:=\left(\dsquaref{O}{A}{A}{A\oplus A}{}{ }{p_A}{q_A}\,,\,\dsquaref{O}{A}{A}{A\oplus A}{}{ }{p_A\circ \alpha}{q_A} \right) \quad l_\tau:=\left(\dsquaref{O}{O}{A\oplus A}{A\oplus A}{}{ }{\tau_A}{ }\,,\,\dsquaref{O}{O}{A\oplus A}{A\oplus A}{}{ }{\tau_A}{ } \right)  $$ %% Why is the RHS \iota_\alpha distinguished? Take the obvious restricted pushouts.
This is a Sherman Loop $G(\alpha,0,\tau_A)$, where $0$ denotes the $\M$-morphism $O\rtail A$.
	
To show $G(A,\alpha)\sim G(\alpha,0,\tau_A)$ in $\pi_1|G\calC^o|$, we first modify $G(A,\alpha)$ in sensible ways that respects its homotopy class. An initial observation: the following loop 
\begin{equation}\label{eq:SherLoopSplit3}
\ones{O}{O} \to \ones{A}{A} \xrightarrow{\iota_\alpha} \ones{A\oplus A}{A\oplus A} \leftarrow  \ones{O}{O}.
\end{equation}
is homotopic to $G(A,\alpha)$, since the following diagram
$$\begin{tikzcd}
\ones{A}{A} \ar[rr,"\iota_\alpha"] \ar[dr,"\alpha"] && \ones{A\oplus A}{A\oplus A}\\
&\ones{A}{A} \ar[ur,"\iota_{A}"]
\end{tikzcd}$$
bounds a 2-simplex, where the 1-simplex $\iota_A$ denotes two copies of the direct sum square $A\oplus A$ (and is thus diagonal). Further, the Loop~\eqref{eq:SherLoopSplit3} is homotopic to 
\begin{equation}\label{eq:SherLoopSplit2}
\ones{O}{O} \to \ones{A}{A} \xrightarrow{\iota_\alpha} \ones{A\oplus A}{A\oplus A} \xrightarrow{1_{A\oplus A}}  \ones{A\oplus A}{A\oplus A} \leftarrow  \ones{O}{O},
\end{equation}
since all we did was insert a degenerate 1-simplex $1_{A\oplus A}$. It remains to show that $G(\alpha,0,\tau_A)$ is homotopic to Loop~\eqref{eq:SherLoopSplit2}. But this follows from noting that the triangles in the diagram below bound 2-simplices
$$\begin{tikzcd}
\ones{A\oplus A}{A\oplus A} \ar[dr,swap,"l_\tau"]  \ar[r,"1_{A\oplus A}"] & \ones{A\oplus A}{A\oplus A} & \ones{O}{O} \ar[dl] \ar[l]\\
& \ones{A\oplus A}{A\oplus A} \ar[u,"l_\tau"]
\end{tikzcd}\qquad .$$
%% Notice $l_\tau$ is diagonal, so its canonical loop bounds a 2-simplex. This does not require Nenashev's theorem, and can be argued directly.
	\vspace{-0.8em}
\end{proof}


\subsection{Explicit Descriptions of 2-Simplices}\label{sec:Nen2simp} %Theorem~\ref{thm:NenashevDSES} states any $x\in K_1(\calC)$ corresponds to a double exact square $l$. The proof proceeds by first applying Theorem~\ref{thm:Sherman} to argue that $x$ corresponds to a Sherman Loop $G(\alpha,\beta,\theta)$, before constructing a double exact square $l(x)$. The rest of the argument involves constructing a sequence of [free] homotopies connecting the two loops $G(\alpha,\beta,\theta)$ and $\mu(l(x))$. 

This section explicitly constructs the two key homotopies claimed by Lemma~\ref{lem:Nen4}, necessary to finish the proof of Theorem~\ref{thm:NenashevDSES} 


\begin{claim}\label{claim:NenL1} Loop~\eqref{eq:NenashevL1} is homotopic to loop $L$.
\end{claim}
\begin{proof} Recall: in order to establish that the two loops are homotopic, it suffices to show that the indicated triangles of the Diagram~\eqref{eq:NenBigL1} are boundaries of 2-simplices. We describe the 2-simplices explicitly below.
	\begin{itemize}
		\item Triangle (1). Consider
		\vspace{-0.1em}
		\begin{equation*}
\begin{tikzcd}
		A\oplus A'\ar[r, >->,"f_0"] & P \ar[dr,phantom,"\square"] \ar[r, >->,"1\oplus C\oplus C'"] & P\oplus C\oplus C'\\
		&	C\oplus C' \ar[r, >->] \ar[u, {Circle[open]}->,"g_0"]& C\oplus C'\oplus C\oplus C' \ar[u, {Circle[open]}->,"g_0\oplus 1"] \\
		&& C\oplus C' \ar[u, {Circle[open]}->] 
		\end{tikzcd}\quad \begin{tikzcd}
		A\oplus A'\ar[r, >->,"f_1"] & Q \ar[r, >->,"h_t"]  \ar[dr,phantom,"\square"]  & V\\
		&	C\oplus C' \ar[r, >->]\ar[u, {Circle[open]}->,"g_1"]& C\oplus C' \oplus C\oplus  C' \ar[u, {Circle[open]}->,"j"] \\
		&& C\oplus C' \ar[u, {Circle[open]}->] 
		\end{tikzcd}
		\end{equation*}
		[Why is this a 2-simplex? The indicated square on the right diagram is the distinguished square $t$ of Lemma~\ref{lem:Nen3}. The left indicated square is distinguished by Lemma~\ref{lem:ADTsquares} (i). The rest are obvious.]
	
		\item Triangle (2).
				\vspace{-0.1em}
		\begin{equation*}
\begin{tikzcd}
		A\oplus A'\ar[r, >->,"f_1"] & Q \ar[dr,phantom,"\square"] \ar[r, >->,"1\oplus C\oplus  C'"] & Q\oplus C\oplus C'\\
		&	C\oplus C' \ar[r, >->] \ar[u, {Circle[open]}->,"g_1"]& C\oplus C'\oplus C\oplus  C' \ar[u, {Circle[open]}->,"g_1\oplus 1"] \\
		&& C\oplus C' \ar[u, {Circle[open]}->] 
		\end{tikzcd}\quad \begin{tikzcd}
		A\oplus A'\ar[r, >->,"f_1"] & Q \ar[r, >->,"h_t"]  \ar[dr,phantom,"\square"]  & V\\
		&	C\oplus C' \ar[r, >->]\ar[u, {Circle[open]}->,"g_1"]& C\oplus C' \oplus C\oplus C' \ar[u, {Circle[open]}->,"j"] \\
		&& C\oplus C' \ar[u, {Circle[open]}->] 
		\end{tikzcd}
		\end{equation*}
[Why is this a 2-simplex? Analogous to Triangle (1).]
		\item Triangle (3).
		\begin{equation*}
		\begin{tikzcd}
		P\ar[r, >->,"1\oplus C\oplus C'"] & P\oplus C\oplus  C' \ar[dr,phantom,"\square"] \ar[r, >->,"\theta\oplus 1"] & Q\oplus C\oplus  C'\\
		&	 C\oplus C' \ar[r, >->,"1"] \ar[u, {Circle[open]}->,"P\oplus 1"]&  C\oplus C' \ar[u, {Circle[open]}->,"Q\oplus 1"] \\
		&& O \ar[u, {Circle[open]}->] 
		\end{tikzcd}\quad \begin{tikzcd}
		Q\ar[r, >->,"h_t "] & V \ar[r, >->,"1"]  \ar[dr,phantom,"\square"]  & V\\
		&	C\oplus C' \ar[r, >->,"1"]\ar[u, {Circle[open]}->,"k_t"]&  C\oplus C' \ar[u, {Circle[open]}->,"k_t	"] \\
		&& O \ar[u, {Circle[open]}->] 
		\end{tikzcd}
		\end{equation*}
		[To show that this is a 2-simplex, notice Lemma~\ref{lem:Nen3} already verified that
		\[t':=\left(\dsquaref{O}{C\oplus C'}{Q}{V}{}{}{h_t}{k_t}\right)\]
		is a distinguished square. The rest follows from Lemma~\ref{lem:ADTsquares} (i).]
		\begin{comment}
		In which case, let $D=C\oplus C'$, and $\theta\colon P\to Q$ be the $\M$-morphism.
Take the direct sum of filtrations $P\rtail P\rtail Q$ and $O\rtail C\oplus C'\rtail C\oplus C'$. 
		\end{comment}
		\item Triangle (4).
		
		\begin{equation*}
		\begin{tikzcd}
		P\ar[r, >->,"\theta"] & Q \ar[dr,phantom,"\square"] \ar[r, >->,"1\oplus C\oplus C'"] & Q\oplus C\oplus  C' \\
		&	O \ar[r, >->] \ar[u, {Circle[open]}->]& C\oplus C' \ar[u, {Circle[open]}->,"Q\oplus 1"] \\
		&& C\oplus C'  \ar[u, {Circle[open]}->] 
		\end{tikzcd}\quad \begin{tikzcd}
		Q\ar[r, >->,"1"] & Q \ar[r, >->,"h_t"]  \ar[dr,phantom,"\square"]  & V\\
		&	O\ar[r, >-> ]\ar[u, {Circle[open]}->]& C\oplus C' \ar[u, {Circle[open]}->,"k_t	"] \\
		&& C\oplus C' \ar[u, {Circle[open]}->] 
		\end{tikzcd}
		\end{equation*}
[Why is this a 2-simplex? Immediate from Lemma~\ref{lem:ADTsquares}.]
	\end{itemize}	
	
\end{proof}

\begin{claim}\label{claim:NenL2} Loop~\eqref{eq:NenashevL2} is homotopic to loop $L$.
\end{claim}
\begin{proof} By analogy with Claim~\ref{claim:NenL1}, all indicated triangles in Diagram~\eqref{eq:NenBigL2} can be shown to bound 2-simplices. The only subtlety lies in verifying that Triangles (1') and (2') do in fact bound the 2-simplices below, but this follows from $V$ being a restricted pushout (see remarks above Diagram~\eqref{eq:NenBigL1}).

	\begin{itemize}
		\item Triangle (1'). 
		\begin{equation*}
 \begin{tikzcd}
		A\oplus A'\ar[r, >->,"f_0"] & P \ar[dr,phantom,"\square"] \ar[r, >->,"1\oplus C\oplus  C'"] & P\oplus C\oplus  C'\\
		&	C\oplus C' \ar[r, >->] \ar[u, {Circle[open]}->,"g_0"]& C\oplus C' \oplus C\oplus C' \ar[u, {Circle[open]}->,"g_0\oplus 1"] \\
		&& C\oplus C' \ar[u, {Circle[open]}->] 
		\end{tikzcd}\quad \begin{tikzcd}
		A\oplus A'\ar[r, >->,"\alpha\oplus \alpha'"] & B\oplus B' \ar[r, >->,"h_u"]  \ar[dr,phantom,"\square"]  & V\\
		&	C\oplus C' \ar[r, >->]\ar[u, {Circle[open]}->,"\delta\oplus\delta'"]& C\oplus C' \oplus C\oplus C' \ar[u, {Circle[open]}->,"j"] \\
		&& C\oplus C' \ar[u, {Circle[open]}->] 
		\end{tikzcd}
		\end{equation*}
	%	To verify that this defines a 2-simplex of $G\calC$, notice the given square on the right diagram is the distinguished square $u$ in Lemma~\ref{lem:Nen3}. The remaining subdiagrams can be shown to be distinguished squares by the same argument for Triangle (1) in Claim~\ref{claim:NenL1}.
		\item Triangle (2').		
		\begin{equation*}
		\!\!\!\!\!\!\!\!\!\!\!\!\!\!\!\!\!\!\!	\begin{tikzcd}
		A\oplus A'\ar[r, >->,"f_1"] & Q \ar[dr,phantom,"\square"] \ar[r, >->,"1\oplus C\oplus C'"] & Q\oplus C\oplus C'\\
		&	C\oplus C' \ar[r, >->] \ar[u, {Circle[open]}->,"g_1"]& C\oplus C' \oplus C\oplus C' \ar[u, {Circle[open]}->,"g_1\oplus 1"] \\
		&& C' \ar[u, {Circle[open]}->] 
		\end{tikzcd}\quad \begin{tikzcd}
		A\oplus A'\ar[r, >->,"\alpha\oplus \alpha'"] & B\oplus B' \ar[r, >->,"h_u"]  \ar[dr,phantom,"\square"]  & V\\
		&	C\oplus C' \ar[r, >->]\ar[u, {Circle[open]}->,"\delta\oplus \delta'"]& C\oplus C' \oplus C\oplus C' \ar[u, {Circle[open]}->,"j"] \\
		&& C' \ar[u, {Circle[open]}->] 
		\end{tikzcd}
		\end{equation*}
		\item Triangle (3').
		\begin{equation*}
		\begin{tikzcd}
		P\ar[r, >->,"1\oplus C\oplus C'"] & P\oplus C\oplus C' \ar[dr,phantom,"\square"] \ar[r, >->,"\theta\oplus 1"] & Q\oplus C \oplus  C'\\
		&	 C\oplus C' \ar[r, >->,"1"] \ar[u, {Circle[open]}->,"P\oplus 1"]&  C\oplus C' \ar[u, {Circle[open]}->,"Q\oplus 1"] \\
		&& O \ar[u, {Circle[open]}->] 
		\end{tikzcd}\quad \begin{tikzcd}
		B\oplus B'\ar[r, >->,"h_u "] & V \ar[r, >->,"1"]  \ar[dr,phantom,"\square"]  & V\\
		&	C\oplus C' \ar[r, >->,"1"]\ar[u, {Circle[open]}->,"k_u"]&  C\oplus C' \ar[u, {Circle[open]}->,"k_u"] \\
		&& O \ar[u, {Circle[open]}->] 
		\end{tikzcd}
		\end{equation*}
%		To show that this is a 2-simplex, notice Lemma~\ref{lem:Nen3} already verified that
	%	\[u':=\left(\dsquaref{O}{C'}{B\oplus B'}{V}{}{}{h_u}{k_u}\right)\]
%		is a distinguished square. The remaining subdiagrams are obvious. 
		\item Triangle (4').
		
		\begin{equation*}
		\begin{tikzcd}
		P\ar[r, >->,"\theta"] & Q \ar[dr,phantom,"\square"] \ar[r, >->,"1\oplus C\oplus  C'"] & Q\oplus C\oplus C' \\
		&	O \ar[r, >->] \ar[u, {Circle[open]}->]& C\oplus C' \ar[u, {Circle[open]}->,"Q\oplus 1"] \\
		&& C\oplus C'  \ar[u, {Circle[open]}->] 
		\end{tikzcd}\quad \begin{tikzcd}
		B\oplus B' \ar[r, >->,"1"] & B\oplus B' \ar[r, >->,"h_u"]  \ar[dr,phantom,"\square"]  & V\\
		&	O\ar[r, >-> ]\ar[u, {Circle[open]}->]& C\oplus C' \ar[u, {Circle[open]}->,"k_u	"] \\
		&& C\oplus  C' \ar[u, {Circle[open]}->] 
		\end{tikzcd}
		\end{equation*}
	\end{itemize}
\end{proof}


\subsection{Permutation Lemma} This section isolates a permutation lemma that clarifies how direct sums interact with reordering; this will play a key role in the study of admissible triples in Section~\ref{sec:admtriples}.

\begin{lemma}[Permutation Lemma]\label{lem:permute} Setup: 
	\begin{itemize}
		\item Let $X_1,\dots, X_m$ be 1-simplices in $G\calC$ of the form
		\[ X_j:=\left( \,\,\dsquaref{O}{Z}{A_j}{B_j}{ }{ }{\alpha_j}{\beta_j}\quad,\quad \dsquaref{O}{Z}{A'_j}{B'_j}{ }{ }{\alpha'_j}{\beta'_j}\,\,\right), \] 	% \qquad j\in\{0,\dots,m\}
		so that all have the same quotient object $Z$. 
		\item For any permutation $\sigma\in S_m$, define
		\[ \widetilde{X}_{\sigma(j)}:=\left( \,\,\dsquaref{O}{Z}{A_j}{B_j}{ }{ }{\alpha_j}{\beta_j}\quad,\quad \dsquaref{O}{Z}{A'_{\sigma(j)}}{B'_{\sigma(j)}}{ }{ }{\alpha'_{\sigma(j)}}{\beta'_{\sigma(j)}}\,\,\right),\]
		i.e. the LHS square is fixed while the RHS is permuted by $\sigma$.
	\end{itemize}	
	\underline{Then}, there is a homotopy between
	$$X_1\oplus \dots \oplus X_m \quad \text{and} \quad \widetilde{X}_{\sigma(1)}\oplus \dots \oplus \widetilde{X}_{\sigma(m)}.$$
\end{lemma}
\begin{proof} We first treat the case for adjacent swaps, before extending to arbitrary permutations.
	
	
	\subsubsection*{Step 0: Adjacent Swaps} Assume $m=2$ and $\sigma=(12)$. Let $\calW\subseteq  G\calC\times G\calC$ be the simplicial subset whose higher $n$-simplices are pairs 
	$$(\alpha_1,\alpha_2)\in G\calC\times G\calC ([n])$$
sharing the same 0th-face.\footnote{In other words, $\calW:=GG\calC$, the $G$-construction applied twice to $S\calC$.} % $G\calC$ really is a pair of $n+1$-simplices in $S\calC$ with common $O$ base-point, and the 0 faces agree. Now we have a pair of pairs of such $n+1$-simplices.
%\footnote{For clarity: no conditions are placed on the vertices of $\calW$.}  
Explicitly, for any $n>0$, if 
	$$\alpha_r=\begin{pmatrix} O \rtail \doubleunderline{A_{r,0}\rtail \dots \rtail A_{r,n}}\\ O \rtail A'_{r,0}\rtail \dots \rtail A'_{r,n} \end{pmatrix}\qquad r\in \{1,2\},$$
	then we require all quotient index diagrams to agree:
	$$	\begin{tikzcd}
\frac{	A'_{1,j}}{A'_{1,i}} \ar[r,>->,""] \ar[dr,phantom,"\square"]& \frac{	A'_{1,j}}{A'_{1,l}} 
	\\
\frac{	A'_{1,k}}{A'_{1,i}} \ar[r,>->] \ar[u, {Circle[open]}->] & \frac{	A'_{1,k}}{A'_{1,l}} \ar[u, {Circle[open]}->] 	\end{tikzcd} = \begin{tikzcd} \frac{	A_{1,j}}{A_{1,i}} \ar[r,>->,""] \ar[dr,phantom,"\square"]& \frac{	A_{1,j}}{A_{1,l}} 
\\
\frac{	A_{1,k}}{A_{1,i}} \ar[r,>->] \ar[u, {Circle[open]}->] & \frac{	A_{1,k}}{A_{1,l}} \ar[u, {Circle[open]}->] 	\end{tikzcd} =\begin{tikzcd}
\frac{	A_{2,j}}{A_{2,i}} \ar[r,>->,""] \ar[dr,phantom,"\square"]& \frac{	A_{2,j}}{A_{2,l}} 
\\
\frac{	A_{2,k}}{A_{2,i}} \ar[r,>->] \ar[u, {Circle[open]}->] & \frac{	A_{2,k}}{A_{2,l}} \ar[u, {Circle[open]}->] 	\end{tikzcd} = \begin{tikzcd}
\frac{	A’_{2,j}}{A’_{2,i}} \ar[r,>->,""] \ar[dr,phantom,"\square"]& \frac{	A'_{2,j}}{A'_{2,l}} 
\\
\frac{	A’_{2,k}}{A’_{2,i}} \ar[r,>->] \ar[u, {Circle[open]}->] & \frac{	A'_{2,k}}{A'_{2,l}} \ar[u, {Circle[open]}->] 	\end{tikzcd},$$
for all $0\leq i\leq  l\leq  k\leq  j\leq n$.\footnote{Use the obvious identifications whenever the indices are equal -- e.g. when $j=l$, set $A'_{1,j}/A'_{1,l}=O$.} In particular, any pair of 1-simplices $(X_1,X_2)\in G\calC\times G\calC$ with common quotient belongs to $\calW$.
	\begin{comment} In other words, the $n$-simplices only differ on the topline; everything starting from the $\M$-filtration just below the top-line will be identical.
	
Remark: for the permutation argument to work, it is not enough to just require that their quotient objects agree:
		$$\frac{A'_{1,k}}{A'_{1,0}}=\frac{A_{1,{k}}}{A_{1,0}}=\frac{A_{2,k}}{A_{2,0}}=\frac{A'_{2,k}}{A'_{2,0}}.$$
		
To see why $\calW$ is a simplicial subset, one can check by hand that it is stable under face and degeneracy maps.			
	\end{comment}
	
	We now set up the main construction of the proof. For a monotone map $\beta\colon [n] \to [1]$, set
	$$i:=\max\{j\mid \beta(j)=0\},$$
	with the convention $i=-1$ if $\beta$ is constantly 1. Then, define the map
	\begin{align*}
	h\colon \calW \times [1] ([n]) & \longrightarrow G\calC ([n]) \\
	\left(\left(\alpha_1,\alpha_2\right)\,,\,\beta\right) &\longmapsto \begin{pmatrix} O \rtail \doubleunderline{A_{1,0}\oplus A_{2,0}\rtail \dots \rtail A_{1,i}\oplus A_{2,i} \rtail A_{1,i+1}\oplus A_{2,i+1}\rtail \dots \rtail A_{1,n}\oplus A_{2,n}}\\ O \rtail A'_{2,0}\oplus A'_{1,0}\rtail \dots \rtail A'_{2,i}\oplus A'_{1,i} \rtail A'_{1,i+1}\oplus A'_{2,i+1}\rtail \dots \rtail A'_{1,n}\oplus A'_{2,n} \end{pmatrix},
	\end{align*}
	where $(\alpha_1,\alpha_2)$ is an $n$-simplex of $\calW$.
	
	In English: the map $h$ defines a simplicial homotopy between $\alpha_1+\alpha_2$ and the permuted sum. When $i=-1$, $h$ gives $\alpha_1+\alpha_2$. When $i\geq 0$, $h$ modifies $\alpha_1+\alpha_2$ by the following rule:
	\begin{itemize}
		\item[$\diamond$] \textbf{Stages $j\leq i$.} Apply the obvious isomorphisms to permute the summands of the relevant $\M$ and $\E$-morphisms (the quotient index diagrams are left unchanged);
		\item[$\diamond$] \textbf{Stages $j\geq i+1$.} Leave the $\M$-morphisms as is, and permute the $\E$-morphisms accordingly to make a simplex in $G\calC$. % (informally, we need to absorb the permutations used in stages $j\leq i$).
	\end{itemize}
Using Lemma~\ref{lem:DirectSum}, one checks that $h$ is a simplicial map. In particular, $h$ defines a homotopy 
		$$X_1\oplus X_2 \sim  \widetilde{X}_{\sigma(1)} \oplus \widetilde{X}_{\sigma(2)},$$
		where $X_1,X_2$ are 1-simplices with common quotient $Z$.
	\begin{comment} Details. 
	
We need to check that the assignment by $h$ defines an $n$-simplex, and that it respects faces and degeneracies (and is thus a simplicial map). Let's examine the case of a 3-simplex. The main idea is that we permute things in stages; the main technicality is to adjust the $\E$-morphisms to ensure that we still have distinguished squares.


\begin{itemize}
\item $i=-1$. We get the obvious sum:
	\[\begin{tikzcd}
A_{1,0}\oplus A_{2,0} \ar[r, >->,"\alpha_{1,0}\oplus\alpha_{2,0}"] & A_{1,1}\oplus A_{2,1} \ar[dr,phantom,"\square"] \ar[r, >->,"\alpha_{1,1}\oplus\alpha_{2,1}"] & A_{1,2}\oplus A_{2,2} \ar[rr,>->,"\alpha_{1,2}\oplus\alpha_{2,2} "] \ar[drr,  "\square",phantom] && A_{1,3}\oplus A_{2,3}\\
O \ar[u,{Circle[open]}->] \ar[r,>->] &  \frac{B}{A}\oplus\frac{B}{A}\ar[r, >->,"v\oplus v" ] \ar[u, {Circle[open]}->,"\beta_{1,1}\oplus\beta_{2,1}"] & \frac{C}{A}\oplus \frac{C}{A}  \ar[u, {Circle[open]}->,swap," \beta_{1,2}\oplus\beta_{2,2}"] \ar[rr,>->,"w\oplus w"] &&  \frac{D}{A}\oplus\frac{D}{A} \ar[u,{Circle[open]}->,swap,"\beta_{1,3}\oplus\beta_{2,3}"] \\
&& \frac{C}{B}\oplus\frac{C}{B} \ar[u,{Circle[open]}->,swap,"\zeta\oplus\zeta"] \ar[rr,>->] && \frac{D}{B}\oplus\frac{D}{B} \ar[u,{Circle[open]}->,swap,"\zeta'\oplus\zeta'"] 
\end{tikzcd}\]
	\[\begin{tikzcd}
A'_{1,0}\oplus A'_{2,0} \ar[r, >->,"\alpha'_{1,0}\oplus\alpha'_{2,0}"] & A'_{1,1}\oplus A'_{2,1} \ar[dr,phantom,"\square"] \ar[r, >->,"\alpha'_{1,1}\oplus\alpha'_{2,1}"] & A'_{1,2}\oplus A'_{2,2} \ar[rr,>->,"\alpha'_{1,2}\oplus\alpha'_{2,2} "] \ar[drr,  "\square",phantom] && A'_{1,3}\oplus A'_{2,3}\\
O \ar[u,{Circle[open]}->] \ar[r,>->] &  \frac{B}{A}\oplus\frac{B}{A}\ar[r, >->,"v\oplus v" ] \ar[u, {Circle[open]}->,"\beta'_{1,1}\oplus\beta'_{2,1}"] & \frac{C}{A}\oplus \frac{C}{A}  \ar[u, {Circle[open]}->,swap," \beta'_{1,2}\oplus\beta'_{2,2}"] \ar[rr,>->,"w\oplus w"] &&  \frac{D}{A}\oplus\frac{D}{A} \ar[u,{Circle[open]}->,swap,"\beta'_{1,3}\oplus\beta'_{2,3}"] \\
&& \frac{C}{B}\oplus\frac{C}{B} \ar[u,{Circle[open]}->,swap,"\zeta\oplus\zeta"] \ar[rr,>->] && \frac{D}{B}\oplus\frac{D}{B} \ar[u,{Circle[open]}->,swap,"\zeta'\oplus\zeta'"] 
\end{tikzcd}\]



\item $i=0.$ Since the first diagram will always remain unchanged, from here on out we just include the second flag diagram.
	\[\begin{tikzcd}
A'_{2,0}\oplus A'_{1,0} \ar[r, >->,"\tau\circ \alpha'_{2,0}\oplus\alpha'_{1,0}"] & A'_{1,1}\oplus A'_{2,1} \ar[dr,phantom,blue,"\square"] \ar[r, >->,"\alpha'_{1,1}\oplus\alpha'_{2,1}"] & A'_{1,2}\oplus A'_{2,2} \ar[rr,>->,"\alpha'_{1,2}\oplus\alpha'_{2,2} "] \ar[drr, red,"\square",phantom] && A'_{1,3}\oplus A'_{2,3}\\
O \ar[u,{Circle[open]}->] \ar[r,>->] &  \frac{B}{A}\oplus\frac{B}{A}\ar[r, >->,"v\oplus v" ] \ar[u, {Circle[open]}->,"\tau\circ \beta'_{2,1}\oplus\beta'_{1,1}"] & \frac{C}{A}\oplus \frac{C}{A}  \ar[u, {Circle[open]}->,swap,"\tau\circ \beta'_{2,2}\oplus\beta'_{1,2}"] \ar[rr,>->,"w\oplus w"] &&  \frac{D}{A}\oplus\frac{D}{A} \ar[u,{Circle[open]}->,swap,"\tau\circ \beta'_{2,3}\oplus\beta'_{1,3}"] \\
&& \frac{C}{B}\oplus\frac{C}{B} \ar[u,{Circle[open]}->,swap,"\zeta\oplus\zeta"] \ar[rr,>->] && \frac{D}{B}\oplus\frac{D}{B} \ar[u,{Circle[open]}->,swap,"\zeta'\oplus\zeta'"] 
\end{tikzcd}\]

Let us justify why we get distinguished squares:
\begin{itemize}
\item The leftmost square is by horizontal composition
\[\begin{tikzcd}
A'_{2,0}\oplus A'_{1,0} \ar[r,>->,"\alpha'_{2,0}\oplus\alpha'_{1,0}"] & A'_{2,1}\oplus A'_{1,1} \ar[r,>->,"\tau"]& A'_{1,1}\oplus A'_{2,1}\\
O \ar[r,>->] \ar[u,{Circle[open]}->] & \frac{B}{A}\oplus\frac{B}{A} \ar[r,>->,"1"] \ar[u,{Circle[open]}->,"\beta'_{2,1}\oplus\beta'_{1,1}"]  & \frac{B}{A}\oplus\frac{B}{A} \ar[u,{Circle[open]}->,"\tau\circ \beta'_{2,1}\oplus\beta'_{1,1}"] 
\end{tikzcd}\] 
The left above square is obvious [to be precise, from Lemma~\ref{lem:DirectSum}], the second is because distinguished squares are closed under somorphisms.
\item The blue indicated square is by vertical composition:
\[\begin{tikzcd}
A'_{1,1}\oplus A'_{2,1} \ar[r,>->,"\alpha'_{1,1}\oplus\alpha'_{2,1}"]& A'_{1,2}\oplus A'_{2,2}\\
A'_{2,1}\oplus A'_{1,1} \ar[r,>->,"\alpha'_{2,1}\oplus\alpha'_{1,1}"] \ar[u,{Circle[open]}->,"\tau "]  & A'_{2,2}\oplus A'_{2,1} \ar[u,{Circle[open]}->,"\tau"]  \\
\frac{B}{A}\oplus\frac{B}{A} \ar[r,>->,"v\oplus v"] \ar[u,{Circle[open]}->," \beta'_{2,1}\oplus\beta'_{1,1}"]  & \frac{C}{A}\oplus\frac{C}{A} \ar[u,{Circle[open]}->," \beta'_{2,2}\oplus\beta'_{1,1}"] 
\end{tikzcd}\] 
The bottom square is given by Lemma~\ref{lem:DirectSum}; the top square is due to permutations commuting with direct sums (Lemma~\ref{lem:DirectSum})  + the fact the distinguished squares are closed under isomorphisms, regardless of whether $\E$ or $\E^{\opp}\subseteq \calC$.
\item The red square is distinguished by a similar logic as the blue square. 
\end{itemize}

\item $i=1$,

\[\begin{tikzcd}
A'_{2,0}\oplus A'_{1,0} \ar[r, >->,"\alpha'_{2,0}\oplus\alpha'_{1,0}"] & A'_{2,1}\oplus A'_{1,1} \ar[dr,phantom,blue,"\square"] \ar[r, >->,"\tau\circ \alpha'_{2,1}\oplus\alpha'_{1,1}"] & A'_{1,2}\oplus A'_{2,2} \ar[rr,>->,"\alpha'_{1,2}\oplus\alpha'_{2,2} "] \ar[drr, red,"\square",phantom] && A'_{1,3}\oplus A'_{2,3}\\
O \ar[u,{Circle[open]}->] \ar[r,>->] &  \frac{B}{A}\oplus\frac{B}{A}\ar[r, >->,"v\oplus v" ] \ar[u, {Circle[open]}->,"\beta'_{2,1}\oplus\beta'_{1,1}"] & \frac{C}{A}\oplus \frac{C}{A}  \ar[u, {Circle[open]}->,swap,"\tau\circ \beta'_{2,2}\oplus\beta'_{1,2}"] \ar[rr,>->,"w\oplus w"] &&  \frac{D}{A}\oplus\frac{D}{A} \ar[u,{Circle[open]}->,swap,"\tau\circ \beta'_{2,3}\oplus\beta'_{1,3}"] \\
&& \frac{C}{B}\oplus\frac{C}{B} \ar[u,{Circle[open]}->,swap,"\zeta\oplus\zeta"] \ar[rr,>->] && \frac{D}{B}\oplus\frac{D}{B} \ar[u,{Circle[open]}->,swap,"\zeta'\oplus\zeta'"] 
\end{tikzcd}\]

	Notice that the blue indicated square is distinguished because it is the composition of 
\[\begin{tikzcd}
A'_{2,1}\oplus A'_{1,1} \ar[r,>->,"\tau\circ \alpha'_{2,1}\oplus \alpha'_{1,1} "]& A'_{1,2}\oplus A'_{2,2}\\
A'_{2,1}\oplus A'_{1,1} \ar[u,{Circle[open]}->,"1"] \ar[r,>->,"\alpha'_{2,1}\oplus \alpha'_{1,1}"]& A'_{2,2}\oplus A'_{2,1} \ar[u,{Circle[open]}->,"\tau"] \\
\frac{B}{A}\oplus \frac{B}{A} \ar[u,{Circle[open]}->,"\beta'_{2,1}\oplus \beta'_{1,1} "] \ar[r,>->,"v\oplus v"]& \frac{C}{A}\oplus\frac{C}{A} \ar[u,{Circle[open]}->,"\beta'_{2,2}\oplus \beta'_{2,1} "]
\end{tikzcd}\]
The bottom square is distinguished by Lemma~\ref{lem:DirectSum} (v). The top square is distinguished by Lemma~\ref{lem:DirectSum} (iii). The red square is distinguished as above. 


\item $i=2$


\[\begin{tikzcd}
A'_{2,0}\oplus A'_{1,0} \ar[r, >->,"\alpha'_{2,0}\oplus\alpha'_{1,0}"] & A'_{2,1}\oplus A'_{1,1} \ar[dr,phantom, "\square"] \ar[r, >->,"\alpha'_{2,1}\oplus\alpha'_{1,1}"] & A'_{2,2}\oplus A'_{1,2} \ar[rr,>->,"\tau\circ \alpha'_{2,2}\oplus\alpha'_{1,2} "] \ar[drr, red,"\square",phantom] && A'_{1,3}\oplus A'_{2,3}\\
O \ar[u,{Circle[open]}->] \ar[r,>->] &  \frac{B}{A}\oplus\frac{B}{A}\ar[r, >->,"v\oplus v" ] \ar[u, {Circle[open]}->,"\beta'_{2,1}\oplus\beta'_{1,1}"] & \frac{C}{A}\oplus \frac{C}{A}  \ar[u, {Circle[open]}->,swap,"\beta'_{2,2}\oplus\beta'_{1,2}"] \ar[rr,>->,"w\oplus w"] &&  \frac{D}{A}\oplus\frac{D}{A} \ar[u,{Circle[open]}->,swap,"\tau\circ \beta'_{2,3}\oplus\beta'_{1,3}"] \\
&& \frac{C}{B}\oplus\frac{C}{B} \ar[u,{Circle[open]}->,swap,"\zeta\oplus\zeta"] \ar[rr,>->] && \frac{D}{B}\oplus\frac{D}{B} \ar[u,{Circle[open]}->,swap,"\zeta'\oplus\zeta'"] 
\end{tikzcd}\]
The red square is distinguished by a similar argument as above.


\item $i=3$

\[\begin{tikzcd}
A'_{2,0}\oplus A'_{1,0} \ar[r, >->,"\alpha'_{2,0}\oplus\alpha'_{1,0}"] & A'_{2,1}\oplus A'_{1,1} \ar[dr,phantom, "\square"] \ar[r, >->,"\alpha'_{2,1}\oplus\alpha'_{1,1}"] & A'_{2,2}\oplus A'_{1,2} \ar[rr,>->,"\alpha'_{2,2}\oplus\alpha'_{1,2} "] \ar[drr, "\square",phantom] && A'_{2,3}\oplus A'_{1,3}\\
O \ar[u,{Circle[open]}->] \ar[r,>->] &  \frac{B}{A}\oplus\frac{B}{A}\ar[r, >->,"v\oplus v" ] \ar[u, {Circle[open]}->,"\beta'_{2,1}\oplus\beta'_{1,1}"] & \frac{C}{A}\oplus \frac{C}{A}  \ar[u, {Circle[open]}->,swap,"\beta'_{2,2}\oplus\beta'_{1,2}"] \ar[rr,>->,"w\oplus w"] &&  \frac{D}{A}\oplus\frac{D}{A} \ar[u,{Circle[open]}->,swap,"\beta'_{2,3}\oplus\beta'_{1,3}"] \\
&& \frac{C}{B}\oplus\frac{C}{B} \ar[u,{Circle[open]}->,swap,"\zeta\oplus\zeta"] \ar[rr,>->] && \frac{D}{B}\oplus\frac{D}{B} \ar[u,{Circle[open]}->,swap,"\zeta'\oplus\zeta'"] 
\end{tikzcd}\]
\end{itemize}

The higher $n$-simpices are defined analogously.

It remains to check that $h$ respects face and degeneracy maps (= defines a natural transformation).
\begin{itemize}
\item For degeneracy maps, we modify either before $i$ or after $i$, it's easy to check everything is OK. To illustrate in the case of $i=1$, suppose:
\[\begin{tikzcd}
A'_{2,0}\oplus A'_{1,0} \ar[r, >->,"\alpha'_{2,0}\oplus\alpha'_{1,0}"] & A'_{2,1}\oplus A'_{1,1} \ar[dr,phantom,blue,"\square"] \ar[r, >->,"\tau\circ \alpha'_{2,1}\oplus\alpha'_{1,1}"] & A'_{1,2}\oplus A'_{2,2} \ar[rr,>->,"\alpha'_{1,2}\oplus\alpha'_{2,2} "] \ar[drr, red,"\square",phantom] && A'_{1,3}\oplus A'_{2,3}\\
O \ar[u,{Circle[open]}->] \ar[r,>->] &  \frac{B}{A}\oplus\frac{B}{A}\ar[r, >->,"v\oplus v" ] \ar[u, {Circle[open]}->,"\beta'_{2,1}\oplus\beta'_{1,1}"] & \frac{C}{A}\oplus \frac{C}{A}  \ar[u, {Circle[open]}->,swap,"\tau\circ \beta'_{2,2}\oplus\beta'_{1,2}"] \ar[rr,>->,"w\oplus w"] &&  \frac{D}{A}\oplus\frac{D}{A} \ar[u,{Circle[open]}->,swap,"\tau\circ \beta'_{2,3}\oplus\beta'_{1,3}"] \\
&& \frac{C}{B}\oplus\frac{C}{B} \ar[u,{Circle[open]}->,swap,"\zeta\oplus\zeta"] \ar[rr,>->] && \frac{D}{B}\oplus\frac{D}{B} \ar[u,{Circle[open]}->,swap,"\zeta'\oplus\zeta'"] 
\end{tikzcd}\] 
A degeneracy map will yield something like (before $i$)

\[\begin{tikzcd}
A'_{2,0}\oplus A'_{1,0} \ar[r, >->,"\alpha'_{2,0}\oplus\alpha'_{1,0}"] & A'_{2,1}\oplus A'_{1,1} \ar[r,>->,"1\oplus 1"]& A'_{2,1}\oplus A'_{1,1}  \ar[dr,phantom,blue,"\square"] \ar[r, >->,"\tau\circ \alpha'_{2,1}\oplus\alpha'_{1,1}"] & A'_{1,2}\oplus A'_{2,2} \ar[rr,>->,"\alpha'_{1,2}\oplus\alpha'_{2,2} "] \ar[drr, red,"\square",phantom] && A'_{1,3}\oplus A'_{2,3}\\
O \ar[u,{Circle[open]}->] \ar[r,>->] &  \frac{B}{A}\oplus\frac{B}{A}\ar[r, >->,"1\oplus 1" ] \ar[u, {Circle[open]}->,"\beta'_{2,1}\oplus\beta'_{1,1}"] &  \frac{B}{A}\oplus\frac{B}{A}\ar[r, >->,"v\oplus v" ] \ar[u, {Circle[open]}->,"\beta'_{2,1}\oplus\beta'_{1,1}"] & \frac{C}{A}\oplus \frac{C}{A}  \ar[u, {Circle[open]}->,swap,"\tau\circ \beta'_{2,2}\oplus\beta'_{1,2}"] \ar[rr,>->,"w\oplus w"] &&  \frac{D}{A}\oplus\frac{D}{A} \ar[u,{Circle[open]}->,swap,"\tau\circ \beta'_{2,3}\oplus\beta'_{1,3}"] \\
&&& \frac{C}{B}\oplus\frac{C}{B} \ar[u,{Circle[open]}->,swap,"\zeta\oplus\zeta"] \ar[rr,>->] && \frac{D}{B}\oplus\frac{D}{B} \ar[u,{Circle[open]}->,swap,"\zeta'\oplus\zeta'"] 
\end{tikzcd}\] 

The fact that e.g. $1\oplus 1\colon A\oplus B\to A\oplus B$ is the identity follows from Lemma~\ref{lem:DirectSum} (ii). The fact that the new square is distinguished follows from distinguished squares being closed under isomorphisms.


Or, if after $i$, yield something like 
\[\begin{tikzcd}
A'_{2,0}\oplus A'_{1,0} \ar[r, >->,"\alpha'_{2,0}\oplus\alpha'_{1,0}"] & A'_{2,1}\oplus A'_{1,1} \ar[dr,phantom,blue,"\square"] \ar[r, >->,"\tau\circ \alpha'_{2,1}\oplus\alpha'_{1,1}"] & A'_{1,2}\oplus A'_{2,2} \ar[rr,>->,"1\oplus 1 "] \ar[drr, red,"\square",phantom] &&A'_{1,2}\oplus A'_{2,2} \ar[rr,>->,"\alpha'_{1,2}\oplus\alpha'_{2,2} "]  && A'_{1,3}\oplus A'_{2,3}\\
O \ar[u,{Circle[open]}->] \ar[r,>->] &  \frac{B}{A}\oplus\frac{B}{A}\ar[r, >->,"v\oplus v" ] \ar[u, {Circle[open]}->,"\beta'_{2,1}\oplus\beta'_{1,1}"] & \frac{C}{A}\oplus \frac{C}{A}  \ar[u, {Circle[open]}->,swap,"\tau\circ \beta'_{2,2}\oplus\beta'_{1,2}"] \ar[rr,>->,"1\oplus 1"] &&\frac{C}{A}\oplus \frac{C}{A}  \ar[u, {Circle[open]}->,swap,"\tau\circ \beta'_{2,2}\oplus\beta'_{1,2}"] \ar[rr,>->,"w\oplus w"] &&  \frac{D}{A}\oplus\frac{D}{A} \ar[u,{Circle[open]}->,swap,"\tau\circ \beta'_{2,3}\oplus\beta'_{1,3}"] \\
&& \frac{C}{B}\oplus\frac{C}{B} \ar[u,{Circle[open]}->,swap,"\zeta\oplus\zeta"] \ar[rr,>->]&& \frac{C}{B}\oplus\frac{C}{B} \ar[u,{Circle[open]}->,swap,"\zeta\oplus\zeta"] \ar[rr,>->] && \frac{D}{B}\oplus\frac{D}{B} \ar[u,{Circle[open]}->,swap,"\zeta'\oplus\zeta'"] 
\end{tikzcd}\] 
\item For the 0th face map, all 0 faces are identical and $h$ leaves the 0th face untouched (besides adding them). For the non-zero face maps, face maps act by forgetting a column, which amounts to composing the squares. But we know that: (1) distinguished squares compose horizontally, so their composition is also distinguished [and so, this face map is well-defined in $G\calC$] (2) We know that, for $\M$-morphisms, direct sums and composition commute -- this was already verified in Lemma~\ref{lem:DirectSum}. Hence, the $h$-map also respects face maps. 
\end{itemize}

	\end{comment}
	\subsubsection*{Step 1: Arbitrary Permutations} Let $m$ be any positive integer. The case for $m=1$ is trivial. As for $m>1$, any permutation $\sigma\in S_m$ is a finite product of adjacent transpositions. For each adjacent transposition, say $(k\,\, k+1)$ for some $0\leq k <m$, apply Step 0 to get 
		$$X_k\oplus X_{k+1} \sim  \widetilde{X}_{\sigma(k)} \oplus \widetilde{X}_{\sigma(k+1)},$$
		before extending to get
				$$X_1\oplus \dots \oplus X_k\oplus X_{k+1}\oplus \dots \oplus X_m \sim  X_1\oplus \dots \oplus \widetilde{X}_{\sigma(k)} \oplus \widetilde{X}_{\sigma(k+1)} \oplus \dots \oplus X_m;$$
	this uses the fact that the $H$-space addition on $G\calC$ is homotopy associative. Hence, concatenating all the homotopies produced by the adjacent transpositions, deduce the desired homotopy for $\sigma$. 
\end{proof}


\begin{comment} \begin{remark} The proof of Lemma~\ref{lem:permute} establishes a permutation result, not just for 1-simplices sharing the same quotient objects, but also certain families of $n$-simplices. Notice however the conditions placed on them via our simplicial subset $\calW$ are quite strong. Nonetheless, the same argument should give a direct proof that involution is the inverse of $H$-addition on $G\calC$, for arbitrary $n$-simplices by proving that $\alpha+\alpha^{-1}$ (for arbitrary $n$-simplex $\alpha$) is diagonalisable.
\end{remark}
\end{comment}


\subsection{Admissible Triples}\label{sec:admtriples} This section proves Lemma~\ref{lem:admtriple}, which characterises the free homotopy class of admissible triples under mild assumptions. The result appears as Corollary~\ref{cor:admtrip}, following a key lemma proved below. As before, recall Convention~\ref{conv:K1} that $K_1(\calC) = \pi_1|G\calC^o|$, where $G\calC^o$ is the base-point component.

We begin by clarifying the group structure on $K_1(\calC)$ before turning to the main argument. The standard group action is given by concatenation of loops, but since $G\calC$ is an $H$-space, we can say more.

\begin{observation}\label{obs:K1DirectSum} $K_1(\calC)$ is generated by 1-simplices of $G\calC^o$. The group action is equivalent to taking their direct sums. In particular, given any 1-simplex  
		$$	X:=	\begin{pmatrix}  &O\rtail & \doubleunderline{A \xrtail{\alpha} B}\\ &O  \rtail & A'\xrtail{\alpha'}B'\end{pmatrix},
	$$
its corresponding inverse is given by swapping the top and bottom rows 
	$$	Y:=	\begin{pmatrix}  &O\rtail & \doubleunderline{A' \xrtail{\alpha'} B'}\\ &O  \rtail & A\xrtail{\alpha}B\end{pmatrix}.$$
\end{observation}
\begin{proof} That $K_1(\calC)$ is generated by 1-simplices in $G\calC^o$ is Remark~\ref{rem:piGC}. Next, by the usual Eckmann-Hilton argument, the standard group action of $\pi_1|G\calC^o|$ is homotopic to the $H$-space action induced by the $G$-construction -- which, by Theorem~\ref{thm:Gconstruction}, corresponds to taking direct sums. It remains to verify the claim about inverses. Taking the sum of $X$ and $Y$ above, we get
	$$X + Y = \begin{pmatrix}  &O\rtail & \doubleunderline{A\oplus A'\xrtail{\alpha\oplus \alpha'}B\oplus B' }\\ &O  \rtail & A'\oplus A\xrtail{\alpha'\oplus \alpha}B'\oplus B \end{pmatrix}\sim \begin{pmatrix}  &O\rtail & \doubleunderline{A\oplus A'\xrtail{\alpha\oplus \alpha'}B\oplus B'}\\ &O  \rtail & A\oplus A'\xrtail{\alpha\oplus \alpha'}B\oplus B' \end{pmatrix},$$
with the obvious quotients. By Permutation Lemma~\ref{lem:permute}, we can permute the summands on the bottom row of $X+Y$ without changing its homotopy class; hence, $X+Y$ is homotopic to a diagonal square. Since the canonical loop of a diagonal square bounds a 2-simplex, conclude that $\lrangles{X+Y}=0$ in $K_1$, as desired.\footnote{ {\em Notes for the cautious reader.} The choice of maximal spanning tree $T$ in Remark~\ref{rem:piGC} determines a canonical loop for each 1-simplex $f$ in $G\calC^o$, so $\lrangles{f}$ is well-defined in $K_1$. Explicitly, let us build $T$ by first including $\{(O,O)\to(A,A')\}$ with $A$ the chosen common quotient, before extending to a maximal tree by Zorn’s Lemma. In particular, diagonal squares still vanish in $K_1$ since their canonical loops bound a 2-simplex (see Theorem~\ref{thm:NenSurj}, Relation~(N1), proved independently of Lemma~\ref{lem:admtriple}).}\end{proof}

\begin{comment} A previous attempt involved explicitly constructing a sequence of 2-simplices to connect the canonical loops of $X+Y$ with the canonical loop of the diagonal square, but all the obvious ways of assembling the diagrams together produced an unwanted extra twist on the $\E$-morphism, which we were unable to get rid of.
\end{comment}

\begin{comment} Why does homotopy equivalence of 1-simplices $f\sim f'$ imply $\lrangles{f}=\lrangles{f'}$?

Given a maximal spanning tree $T$, for each vertex $v$, choose the unique tree path $p_v$ from the base-point to $v$.

For a 1-simplex $f\colon v_0\to v_1$, set $\mu(f)=p_{v_1}fp^{-1}_{v_0}$. If $f$ and $f'$ are simplicially homotopic, the induced homotopy descends to a homotopy of the images $q(f)$ and $q(f')$ in the quotient $X/T$. Since $X\to X/T$ is a homotopy equivalence (as we are quotienting by a contractible subspace), $q_\ast$ is an isomorphism on $\pi_1$, and so $l_T(f)=l_T(f')\in \pi(X,\ast)$. Thus homotopic 1-simplices define the same class in $K_1$. 
\end{comment}

\begin{construction} Let $\calT$ be a triangle contour, presented as in Diagram~\eqref{eq:tricontour}. Given any vertex $(A,A')$ in $G\calC$, one can construct a new triangle contour $(A,A')\oplus \mathcal{T}$ by formal direct sum
	\begin{equation}
	\begin{tikzcd}
	&(P_1\oplus A,P'_1\oplus A') \ar[dr,"e_1\oplus (A\text{,}A')"]\\
	(P_0\oplus A,P'_0\oplus A') \ar[ur,"e_0\oplus (A\text{,}A')"] \ar[rr,"e_2\oplus (A\text{,}A')"] && (P_2\oplus A,P'_2\oplus A')
	\end{tikzcd} 
	\end{equation}
That $(A,A')\oplus \mathcal{T}$ is well-defined follows from Lemma~\ref{lem:ADTsquares}.
\end{construction}

\begin{lemma}[Key Lemma] Let $\calT=(e_0,e_1,e_2)$ be an admissible triple, where each $e_i$ is a double exact square. Following the notation of ~\eqref{eq:tricontour}, this assembles into a pair of flag diagrams, presented below.
	\begin{equation}\label{eq:admissibleDES}
	\begin{tikzcd}
	P_0 \ar[r, >->,"\alpha_{0,1}"] & P_1 \ar[dr,phantom,"\square"] \ar[r, >->,"\alpha_{1,2}"] & P_2\\
	&	P_{1/0} \ar[r, >->,"\alpha_{1/0,2/0}"] \ar[u, {Circle[open]}->,"\alpha_{1/0,1}"]& P_{2/0} \ar[u, {Circle[open]}->,swap,"\alpha_{2/0,2}"] \\
	&& P_{2/1} \ar[u, {Circle[open]}->,swap,"\alpha_{2/1,2/0}"] 
	\end{tikzcd}\quad \begin{tikzcd}
	P_0 \ar[r, >->,"\alpha'_{0,1}"] & P_1 \ar[dr,phantom,"\square"] \ar[r, >->,"\alpha'_{1,2}"] & P_2\\
	&	P_{1/0} \ar[r, >->,"\alpha'_{1/0,2/0}"] \ar[u, {Circle[open]}->,"\alpha'_{1/0,1}"]& P_{2/0} \ar[u, {Circle[open]}->,swap,"\alpha'_{2/0,2}"] \\
	&& P_{2/1} \ar[u, {Circle[open]}->,swap,"\alpha'_{2/1,2/0}"] 
	\end{tikzcd}
	\end{equation}		
Let $l(\calT)$ be the double exact square associated to $\calT$, and $\mu(l(\calT))$ be its canonical loop.
	\underline{Then} the loop $\calT=e_0e_1e_2^{-1}$ is freely homotopic to 
	$(P_2,P_2)\oplus \mu(l(\calT)).$
\end{lemma}

\begin{proof}\hfill
	\subsubsection*{Step 0: Setup}
	 Start by taking repeated restricted pushouts
	
	\begin{equation}\label{eq:admtripRP}
	\!\!\!\!\!\!\!\!\!\!\!\!\!\!\!\!\!\!\!\!\!\!\!\!\!\!\!\!\!\!\!\!\!\!\!\!\!\!\!\!\small{\begin{tikzcd} & O \ar[r,>->] \ar[d,>->]& P_{1/0} \ar[d,>->] \ar[r,>->,"\alpha_{1/0,2/0}"] & P_{2/0} \ar[d,>->]\\
		P_0 \ar[d,>->,swap,"\alpha_{0,1}"] \ar[r,>->,"\alpha_{0,2}"] & P_2\ar[r,>->,"1\oplus P_{1/0}"] \ar[d,>->]& P_2\oplus P_{1/0} \ar[d,>->] \ar[r,>->,"1\oplus\alpha_{1/0,2/0}",yshift=3.5]&  P_{2}\oplus P_{2/0} \ar[d,>->]\\
		P_1  \ar[r,>->] \ar[d,>->,swap,"\alpha_{1,2}"] & Q \ar[r,>->] \ar[d,>->] & Z_0 \ar[d,>->] \ar[r,>->]& Z_1 \ar[d,>->]\\
		P_2 \ar[r,>->] & W_0\ar[r,>->]  & W_1 \ar[r,>->] & W_2\\
		\end{tikzcd}} \,\,\quad \small{\begin{tikzcd}& O \ar[r,>->] \ar[d,>->]& P_{1/0} \ar[d,>->] \ar[r,>->,"\alpha'_{1/0,2/0}"] & P_{2/0} \ar[d,>->] \\
		P_0 \ar[d,>->,swap,"\alpha'_{0,1}"] \ar[r,>->,"\alpha'_{0,2}"] & P_2\ar[r,>->,"1\oplus P_{1/0}"] \ar[d,>->]& P_2\oplus P_{1/0} \ar[d,>->] \ar[r,>->,"1\oplus\alpha'_{1/0,2/0}",yshift=3.5]&  P_{2}\oplus P_{2/0} \ar[d,>->]\\
		P_1  \ar[r,>->] \ar[d,>->,swap,"\alpha'_{1,2}"] & Q' \ar[r,>->] \ar[d,>->] & Z'_0 \ar[d,>->] \ar[r,>->]& Z'_1 \ar[d,>->]\\
		P_2 \ar[r,>->] & W'_0\ar[r,>->]  & W'_1 \ar[r,>->] & W'_2\\
		\end{tikzcd}},
	\end{equation}
	%where commutativity of the diagrams is given by Fact~\ref{facts:restrictedpushouts}.
	 For readability, we have renamed the restricted pushouts so that, e.g. $Q:= P_2\star_{P_0} P_1$ is the restricted pushout induced by $P_1\xltail{\alpha_{0,1}}P_0 \xrtail{\alpha_{0,2}} P_2$. Notice: all pairs of $\M$-morphisms in Diagram~\eqref{eq:admtripRP} form 1-simplices. %\footnote{Why? The $\M$-morphism pairs corresponding to the admissible triple are required to be 1-simplices by assumption; the rest follows from the fact that restricted pushouts preserve quotients.} 
	This fact, combined with Corollary~\ref{cor:RPtrick}, gives a roadmap for how to connect the vertices of the two loops.
	
\subsubsection*{Step 1: Reformulation} The restricted pushouts in Step 0 assemble into the diagram of 1-simplices below.
	\begin{equation*}
	\!\!\!\!\!\!\!\!\!\!\!\!\!\!\!\!\!\!\!\small{\begin{tikzcd}
		&& \ones{W_2}{W'_2}  \\ 
		\ones{P_2\oplus P_{1/0}}{P_2\oplus P_{1/0}}\ar[rr,blue,"\gamma_1"]  \ar[urr,"a"{name=Y6},violet] && 	\ones{P_2\oplus P_{2/0}}{P_2\oplus P_{2/0}} \ar[u,"b"{name=Y7},violet] \ar[from=Y6, to=Y7,phantom,"(1)", yshift=-5]\\
		& 	\ones{P_2}{P_2} \ar[ur,blue,"\gamma_2",""{name=E2}] \ar[ul,"\gamma_0",swap,blue,""{name=E1}] \end{tikzcd}}\qquad \begin{tikzcd}
	&&&& \ones{W_2}{W_2'}   \\
	\ones{Q}{Q'}  \ar[rrrru,"e"{name=U1},violet] \ar[rrrr, "h"{name=Y0}]&&&& \ones{W_0}{W_0'} \ar[u, violet,"f"{name=U0}] \ar[from=U0,to=U1,phantom,"(1')",yshift=-10] \\
	&\ones{P_1}{P_1} \ar[rr, "e_1"{name=Y0},red] \ar[ul,""{name=Y1}] \ar[urrr,bend left=5,""{name=Y3}]&& \ones{P_2}{P_2} \ar[ur,""{name=Y4}] \ar[from=Y3, to=Y1,phantom,"(2')", xshift=-15]\ar[from=Y3, to=Y4,phantom,"(3')",,yshift=-5]
	\\
	&& \ones{P_0}{P_0} \ar[uull, bend left=35,""{name=Z3}] \ar[uurr, bend right=35,""{name=Z2}]\ar[d,"e_2"] \ar[ul,red,swap,"e_0"{name=Z6}] \ar[ur,red,"e_2"{name=Z7}]
	\ar[from=Z6,to=Z3,phantom,"(4')"]\ar[from=Z7,to=Z2,phantom,"(5')"]
	\\
	&& \ones{P_2}{P_2} \ar[uuurr, bend right=30,violet,swap, "g",""{name=Z0}] 
	\ar[from=Z2, to=Z0, phantom, "(7')",xshift=-50,yshift=-30]  
	\ar[uuull, bend left=30,violet,"d"{name=Z1}]
	\ar[from=Z1, to=Z3, phantom, "(6')",xshift=50,yshift=-30]
	\end{tikzcd}
	\end{equation*}
	
\noindent Here, $(P_2,P_2)\oplus \mu(l(\calT))$ is presented as the blue edges $\gamma_0 \gamma_1 \gamma_2^{-1}$ in the LHS diagram above, while the loop $\calT=e_0e_1e_2^{-1}$  as the red edges in the RHS diagram. Some preliminary observations:
\begin{enumerate}[label=(\alph*)]
	\item On the RHS diagram: the loop $dhg^{-1}$ is obtained by taking restricted pushouts of 
	$$\ones{P_2}{P_2}\leftarrow \ones{P_0}{P_0}\to \ones{P_1}{P_1}\qquad \ones{P_2}{P_2}\leftarrow \ones{P_0}{P_0}\to \ones{P_2}{P_2}\quad \text{and}\quad \ones{Q}{Q'}\leftarrow \ones{P_1}{P_1}\to \ones{P_2}{P_2}.$$
	By Corollary~\ref{cor:RPtrick}, Triangles (2') - (7') are all 2-simplices. It is also easy to see that they assemble into the diagram above. The only subtlety is verifying Triangles (3') and (5') share the same face $(P_2,P'_2)\to (W_0,W'_0)$ -- but this follows from our hypothesis that $\calT$ is an {\em admissible} triple. Conclude that $\calT$ is freely homotopic to the loop $dhg^{-1}$.
	\item Triangles (1) and (1') are obtained by taking restricted pushouts of 
	$$\ones{W_1}{W'_1}\leftarrow \ones{P_2\oplus P_{1/0}}{P_2\oplus P_{1/0}}\to \ones{P_2\oplus P_{2/0}}{P_2\oplus P_{2/0}}\qquad \text{and}\quad \ones{Z_1}{Z_1'}\leftarrow \ones{Q}{Q'}\to \ones{W_0}{W'_0}$$
	respectively, before applying Corollary~\ref{cor:RPtrick}. Conclude that $\calT$ is freely homotopic to the outer loop $def^{-1}g^{-1}$, and $(P_2,P'_2)\oplus \mu(l(\calT))$ is freely homotopic to $\gamma_0ab^{-1}\gamma_2^{-1}$.
	\item The concatenation of 1-simplices $\gamma_2 b$ is homotopic to $gf$ by applying Corollary~\ref{cor:RPtrick} to the restricted pushout
	\[\begin{tikzcd}
	\ones{P_2}{P_2} \ar[rr,"g"] \ar[d,swap,"\gamma_2"] \ar[drr] && \ones{W_0}{W_0'} \ar[d,"f"]\\
	\ones{P_2\oplus P_{2/0}}{P_{2}\oplus P_{2/0}} \ar[rr,swap,"b"]&& \ones{W_2}{W_2'}
	\end{tikzcd}.\]
	\item It is worth recalling Warning~\ref{warning:compose}: given two composable 1-simplices 
	$$z_0\colon (A,A')\to (B,B') \qquad\text{and}\qquad z_1\colon (B,B')\to (C,C')$$
with sequences $A\rtail B\rtail C$ and $A'\rtail B'\rtail C'$ in $\M$, their composite 
$$\overline{z_0z_1} \colon (A,A')\to (C,C')$$
need not satisfy
$$\lrangles{z_0}+\lrangles{z_1}=\lrangles{\overline{z_0z_1}}\qquad\text{in} \,\, K_1(\calC),$$
{\em unless} the three 1-simplices are related by a 2-simplex. In the case of Item (c), such a 2-simplex exists, which yields
\begin{equation}\label{eq:compSPLIT}
\lrangles{g}+\lrangles{f}=\lrangles{\overline{gf}} = \lrangles{\overline{\gamma_2 b}} =\lrangles{\gamma_2}+\lrangles{b}.
\end{equation}
\end{enumerate}

In sum: items (a) and (b) establishes the free homotopies 
$$\calT\sim def^{-1}g^{-1} \qquad\text{and}\qquad  (P_2,P'_2)\oplus \mu(l(\calT)) \sim \gamma_0ab^{-1}\gamma_2^{-1}.$$
Hence, to prove the lemma, it suffices to show $def^{-1}g^{-1}$ and $\gamma_0ab^{-1}\gamma_2^{-1}$ are freely homotopic to each other. This is equivalent\footnote{The setup here is more general than in Lemma~\ref{lem:Nenclosedloop}: a canonical loop for a 1-simplex may involve a zig-zag back to $(O,O)$. Nevertheless, the same cancellation argument applies, so the decomposition of loops into formal sums of 1-simplices still holds.} to showing 
$$ \lrangles{\gamma_0} + \lrangles{a} - \lrangles{b} -\lrangles{\gamma_2} = \lrangles{d} + \lrangles{e} -\lrangles{f} -\lrangles{g} \qquad \text{in $K_1$}.$$
Since $\lrangles{g}+\lrangles{f}=\lrangles{\gamma_2} +\lrangles{b}$ by item (d), this reduces to showing 
\begin{equation}\label{eq:step2ID}
\lrangles{\gamma_0}+\lrangles{a}-\lrangles{d}-\lrangles{e}=0 .
\end{equation}


\subsubsection*{Step 2: A New 2-Simplex} Recall from Observation~\ref{obs:K1DirectSum} that addition in $K_1$ corresponds to taking direct sums of 1-simplices, and subtraction to swapping the top and bottom rows. Hence, with Step 1 in mind, we construct a 2-simplex whose boundary corresponds to the concatenation of 1-simplices $\lrangles{\gamma_0}-\lrangles{d}$ and $\lrangles{a}-\lrangles{e}$ (up to permutation of summands). 

Explicitly, this 2-simplex is presented by
$$	\begin{pmatrix}  &O\rtail & \doubleunderline{\,P_2\oplus P_2 \rtail  P_2\oplus P_{1/0}\oplus Q'\rtail W_2\oplus W'_2\, }\\ &O  \rtail & P_2\oplus P_2 \rtail  Q\oplus P_2\oplus P_{1/0}\rtail W_2\oplus W'_2\,\end{pmatrix},
$$
which corresponds to the flag diagrams
\begin{equation*}
\!\!\!\!\!\!\!\!\!\!\!\!\!\!\!\!\!\!\!\!\!\!\!\!\!\!\!\!\!\small{\begin{tikzcd}
P_2\oplus P_2\ar[r, >->] & P_2\oplus P_{1/0}\oplus Q'  \ar[dr,phantom,"\square"] \ar[r, >->] & W_2\oplus W'_2\\
&	P_{1/0}\oplus P_{1/0} \ar[r, >->] \ar[u, {Circle[open]}->]& P_{2/0}\oplus P_{2/0}\oplus P_{2/0}\oplus P_{2/0}\ar[u, {Circle[open]}->] \\
&& P_{2/0}\oplus P_{2/1}\oplus P_{2/0}\oplus P_{2/1}\ar[u, {Circle[open]}->] 
\end{tikzcd}\!\!\begin{tikzcd}
P_2\oplus P_2\ar[r, >->] & Q\oplus P_2\oplus P_{1/0} \ar[dr,phantom,"\square"] \ar[r, >->] & W_2\oplus W'_2\\
&	P_{1/0}\oplus P_{1/0} \ar[r, >->] \ar[u, {Circle[open]}->]& P_{2/0}\oplus P_{2/0}\oplus P_{2/0}\oplus P_{2/0}\ar[u, {Circle[open]}->] \\
&& P_{2/0}\oplus P_{2/1}\oplus P_{2/0}\oplus P_{2/1}\ar[u, {Circle[open]}->] 
\end{tikzcd}}
\end{equation*}

Step 2 constructs this 2-simplex in stages. We continue with the $\M$-morphisms in Step 0, now explicitly labelled as needed. % In addition, quotient representatives arising from restricted pushouts are fixed according to Convention \ref{conv:restrPO-quotient} unless stated otherwise. \MING{Actually, this is not necessary. We are pretty explicit about our choice of quotients, and this may need mild adjustments anyway, given the footnote in Claim~\ref{claim:step-2b}}.

\subsubsection*{Step 2a} By Axiom (PQ), we have the following exact squares
\begin{equation}\label{eq:Step3a}
\dsquaref{O}{ P_{1/0}}{P_2}{Q}{}{}{q_1}{\beta_{1/0}} \qquad\text{and}\qquad  \dsquaref{O}{ P_{1/0}}{P_2}{Q'}{}{}{q'_1}{\beta'_{1/0}}.
\end{equation}
Applying Lemma~\ref{lem:DirectSum}, construct the following pair of distinguished squares
\begin{equation}\label{eq:3a}
\begin{tikzcd}
O \ar[rr,>->] \ar[d,{Circle[open]}->] \ar[drr,phantom,"\square"]&& P_{1/0}\oplus P_{1/0}  \ar[d,{Circle[open]}->,"t_0"]\\
P_2\oplus P_2 \ar[rr,>->,"s_0"]&& P_2\oplus P_{1/0}\oplus Q'
\end{tikzcd} \qquad 
\begin{tikzcd}
O \ar[rr,>->] \ar[d,{Circle[open]}->]\ar[drr,phantom,"\square"]&& P_{1/0}\oplus P_{1/0}  \ar[d,{Circle[open]}->," t_1"]\\
P_2\oplus P_2 \ar[rr,>->,"s_1"]&& Q\oplus P_2\oplus P_{1/0}
\end{tikzcd}
\end{equation}
\begin{equation*}
s_0=\begin{pmatrix}
1 & 0\\
0 & 0\\
0 & q'_1
\end{pmatrix},\,\, s_1=\begin{pmatrix}
q_1 & 0\\
1 & 0\\
0 & 0
\end{pmatrix},\,\, t_0=\begin{pmatrix}
0 & 0\\
1 & 0\\
0 & \beta'_{1/0}
\end{pmatrix},\,\, t_1=\begin{pmatrix}
\beta_{1/0} & 0\\
0 & 0\\
0 & 1
\end{pmatrix}.
\end{equation*}
\subsubsection*{Step 2b} In this stage, we show that the quotients of the restricted pushouts in Step 0 can be chosen to align with those of the admissible triple $\calT$. The following claim makes this precise.

\begin{claim}\label{claim:step-2b} Fix the morphisms of the given admissible triple $\calT=(e_0,e_1,e_2)$, as well as the $\M$-morphisms in Step 0. \underline{Then}, one can construct a pair of flag diagrams
		\begin{equation}\label{eq:3bWARMUP}
	\begin{tikzcd}
	P_0 \ar[r, >->,"\alpha_{0,1}"] & P_1 \ar[dr,phantom,"\square"] \ar[r, >->,"\alpha_{1/2}"] & P_2\\
	&	P_{1/0} \ar[r, >->,"\alpha_{1/0,2/0}"] \ar[u, {Circle[open]}->,"\alpha_{1/0,1}"]& P_{2/0} \ar[u, {Circle[open]}->,swap,"\alpha_{2/0,2}"] \\
	&& P_{2/1} \ar[u, {Circle[open]}->,swap,"\alpha_{2/1,2/0}"] 
	\end{tikzcd}\quad \begin{tikzcd}
	P_2 \ar[r, >->,"q_1"] & Q \ar[dr,phantom,"\square"] \ar[r, >->,"w_0"] & W_0\\
	&	P_{1/0} \ar[r, >->,"\beta_{1/0,2/0}"] \ar[u, {Circle[open]}->,"\beta_{1/0}"]& P_{2/0} \ar[u, {Circle[open]}->,swap,"\beta_{2/0}"] \\
	&& P_{2/1} \ar[u, {Circle[open]}->,swap,"\beta_{2/1,2/0}"] 
	\end{tikzcd}
	\end{equation}
satisfying the identities
	$$\alpha_{1/0,2/0}=\beta_{1/0,2/0} \qquad \text{and} \qquad \alpha_{2/1,2/0}=\beta_{2/1,2/0}\;.$$
\end{claim}
\begin{proof}[Proof of Claim] Apply Lemma~\ref{lem:quotFilt} to the filtrations
\begin{equation}\label{eq:2b-filtration}
P_0\rtail P_1\rtail P_2 \qquad\text{and}\qquad P_{2}\rtail Q\rtail W_0.
\end{equation}
We now make the corresponding choices of quotients explicit. By Axiom (PQ) and Convention~\ref{conv:restrPO-quotient}, choose quotients of $W_0$ such that the following diagrams commute in the ambient pCGW category $\calC$. %\footnote{To avoid confusion: the $\M$-morphisms here are exactly those in Step 0, except now labelled to facilitate analysis.} 
\begin{equation}\label{eq:3bPOproperty}
\begin{tikzcd}
P_1 \ar[r,>->,"\alpha_{1,2}"] \ar[d,>->,swap,"q_1"] & P_2 \ar[dr,phantom,"\circlearrowleft"]\ar[d,>->,swap,"q_2"] & P_{2/1} \ar[l,swap,"\alpha_{2/1,2}",{Circle[open]}->] \ar[d,>->,"="]\\
Q \ar[r,>->,swap,"w_0"]& W_0 & P_{2/1} \ar[l,swap,swap,"\beta_{2/1}",{Circle[open]}->]
\end{tikzcd} \qquad\text{and}\qquad  \begin{tikzcd}
P_0\ar[r,>->,"\alpha_{0,2}"] \ar[d,>->,swap,"\alpha_{0,2}"] & P_2 \ar[dr,phantom,"\circlearrowleft"]\ar[d,>->,swap,"q_2"] & P_{2/0} \ar[l,swap,"\alpha_{2/0,2}",{Circle[open]}->] \ar[d,>->,"="]\\
P_2 \ar[r,>->,swap,"q_2"]& W_0 & P_{2/0} \ar[l,swap,swap,"\beta_{2/0}",{Circle[open]}->]
\end{tikzcd} \quad .
\end{equation}
Hence, let the morphisms $\alpha_{1/0,1}$, $\alpha_{2/0,2}$, $\beta_{1/0}$, and $\beta_{2/0}$ be the obvious quotients associated to~\eqref{eq:2b-filtration}. Finally, in the notation of Diagram~\eqref{eq:3bWARMUP}, choose the quotients of $\M$-morphisms $\alpha_{1/0,2/0},\beta_{1/0,2/0}\colon P_{1/0}\rtail P_{2/0}$ so that the following $\E$-squares commute:
\begin{equation}\label{eq:3bsquare}
\begin{tikzcd}
P_{2/1} \ar[d,{Circle[open]}->,swap,"\alpha_{2/1,2/0}"]\ar[r,{Circle[open]}->,"="] \ar[dr,phantom,"\circlearrowleft"]& P_{2/1} \ar[d,{Circle[open]}->,"\alpha_{2/1,2}"]\\
P_{2/0}\ar[r,{Circle[open]}->,swap,"\alpha_{2/0,2}"]& P_2
\end{tikzcd} \qquad \begin{tikzcd}
P_{2/1} \ar[d,{Circle[open]}->,swap,"\beta_{2/1,2/0}"]\ar[r,{Circle[open]}->,"="] \ar[dr,phantom,"\circlearrowleft"]& P_{2/1} \ar[d,{Circle[open]}->,"\beta_{2/1}"]\\
P_{2/0} \ar[r,{Circle[open]}->,swap,"\beta_{2/0}"]& W_0
\end{tikzcd}\,\,.
\end{equation}	

\begin{comment} The morphism $\alpha_{2/1,2}$ is not explicitly mentioned in the statement of the lemma, where we only gave the flag diagrams. However, this data is given by the admissible triple, and the notation of these $\E$-morphism is alluded to in the reference to \ref{eq:tricontour} within the lemma statement.
\end{comment}


\begin{comment} Previous notes:

Apply the functor $k^{-1}\colon \Ar_{\triangle}\M\to \Ar_{\square}\E$ to the filtrations
\begin{equation}\label{eq:2b-filtration}
P_0\rtail P_1\rtail P_2 \qquad\text{and}\qquad P_{2}\rtail Q\rtail W_0.
\end{equation}
This produces a diagram analogous to Diagram~\eqref{eq:3bWARMUP}, except the quotients chosen by $k^{-1}$ need not coincide with those required here. By Axiom (K), however, all choices of quotients are canonically related by codomain-preserving isomorphisms. Since distinguished squares are closed under isomorphisms, we may, without loss of generality, choose the quotient representative most natural for our context (cf. Footnote~\ref{fn:INV}).


Functor $c$ yields the canonical quotient $Z_0:=Q/P_2$. This is not necessarily $P_{1/0}$, but this is isomorphic to $P_{1/0}$, so the relevant distinguished square can always be modified. The same is true for $Z_1:=W_0/P_2$, so we have the top right square by Axiom (K)

\[\begin{tikzcd}
P_2 \ar[r,>->]& Q \ar[r,>->]& W_0 \ar[r,>->,"1"]& W_0\\
O \ar[u,{Circle[open]->},swap]\ar[r,>->]& Z_0 \ar[r,>->,"\zeta"]\ar[u,{Circle[open]->}] & Z_1\ar[u,{Circle[open]->},"\xi",swap] \ar[r,>->] & P_{2/0}\ar[u,{Circle[open]->},swap,,"\beta_{2/0}"]\\
O\ar[u,{Circle[open]->}] \ar[r,>->]& P_{1/0} \ar[u,{Circle[open]->},"\gamma"] \ar[r,>->,"\zeta\circ \gamma"] & Z_1 \ar[u,{Circle[open]->},"1",swap] \ar[r,>->]& P_{2/0}\ar[u,{Circle[open]->},swap]
\end{tikzcd}\]

This arugment works for any choice of quotients of $P_2\rtail Q$ and $P_2\rtail W_0$, in particular the choice of pushout quotients following Convention~\ref{conv:restrPO-quotient}.


After composing, we obtain the diagram

\[\begin{tikzcd}
P_2 \ar[r,>->,"q_1"]& Q \ar[r,>->,"w_0"]& W_0  \\
O \ar[u,{Circle[open]->},swap]\ar[r,>->]& P_{1/0} \ar[r,>->,"\zeta"]\ar[u,{Circle[open]->},"\beta_{1/0}"] & P_{2/0}\ar[u,{Circle[open]->},"\beta_{2/0}",swap] 
\end{tikzcd}\]

We can then apply Axiom (K) to get the exact square
\[\dsquaref{O}{Z_2}{P_{1/0}}{P_{2/0}}{ }{ }{\zeta}{\rho}\]
And so we get that $\beta_{2/0}\circ \rho$ is a quotient of $w_0$.  We also know that $\beta_{2/1}$ is a formal quotient of $w_0$, and so there exists an isomorphism $Z_2\xrightarrow{\rho'} P_{2/1}$ such that 
\[\begin{tikzcd}
W_0 \ar[r,>->,"1"] & W_0\\
Z_2 \ar[u,{Circle[open]->},"\beta_{2/0}\circ\rho"]  \ar[r,>->,"\rho'"] & P_{2/1} \ar[u,{Circle[open]->},"\beta_{2/1}"]
\end{tikzcd}\]
So we can always precompose $\rho$ with $\rho'^{-1}$. The same argument works for the other side.
\end{comment}

Our claim then follows by translating Diagrams~\eqref{eq:3bPOproperty} and \eqref{eq:3bsquare} into commutative diagrams in $\calC$, and performing a diagram-chase. There are two cases to check.
	\begin{itemize}
		\item \textbf{Case 1:} $\E\subseteq \calC$. In which case, we get the identity
		$$\beta_{2/0}\circ \beta_{2/1,2/0}=\beta_{2/1}=q_2\circ \alpha_{2/1,2}.$$
		Since 
		$$\alpha_{2/1,2}=\alpha_{2/0,2}\circ \alpha_{2/1,2/0},$$
		this yields 
		$$  \beta_{2/0}\circ \beta_{2/1,2/0}=q_2\circ \alpha_{2/0,2}\circ \alpha_{2/1,2/0}.$$
		In particular, notice $\beta_{2/0}=q_2\circ \alpha_{2/0,2}$
		by Diagram~\eqref{eq:3bPOproperty}. Hence, since $\E$-morphisms are monic in $\calC$ by Axiom (M), deduce that  
		$$  \beta_{2/1,2/0}=\alpha_{2/1,2/0}.$$
In other words, $\alpha_{1/0,2/0}$ and $\beta_{1/0,2/0}$ are formal kernels of the same $\E$-morphism, and are thus equivalent up to codomain-preserving isomorphism by Axiom (K). Hence, without loss of generality,\footnote{Explicitly, one may insert an isomorphism $\kappa\colon P_{1/0}\to P_{1/0}$ into the relevant distinguished squares so that $\alpha_{1/0,2/0}$ and $\beta_{1/0,2/0}$ coincide. This corresponds to readjusting the chosen quotient for $q_1\colon P_2\rtail Q$ in Step 2a by an isomorphic representative; the reader can check this does not affect our ability to construct the 2-simplex.} we may assume
	$$\alpha_{1/0,2/0}=\beta_{1/0,2/0}.$$
	
\begin{comment} Previous footnote:
\footnote{For the reader interested in working out the details: remember that we are aiming to construct a 2-simplex, and we can always adjust our choice of quotient for $q_1\colon P_{2}\rtail Q$ in Step 2a to ensure this.}	
\end{comment}	
	
	\begin{comment} 
Details. We now know that the two $\M$ morphisms $\alpha_{1/0,2/0},\beta_{1/0,2/0}$ have the same quotient, and so the $\M$-morphisms are related by codomain-preserving isomorphism. 
	Let us denote this isomorphism as $\delta$. We therefore get the following vertical composition of distinguished squares
	
\[\begin{tikzcd}
P_{1/0}  \ar[r,>->,"\beta_{1/0,2/0}"]& P_{2/0}\\
P_{1/0}  \ar[u,{Circle[open]->},"\delta"] \ar[r,>->,"\alpha_{1/0,2/0}"]& P_{2/0} \ar[u,{Circle[open]->},swap,"1"]\\
O  \ar[u,{Circle[open]->}] \ar[r,>->]  & P_{2/1} \ar[u,{Circle[open]->},swap,"\alpha_{2/1,2/0}"]
\end{tikzcd}\]	


Now we can compose this with our previous diagram to get:
\[\begin{tikzcd}
P_2 \ar[r,>->,"q_1"]& Q \ar[r,>->,"w_0"]& W_0  \\
O \ar[u,{Circle[open]->},swap]\ar[r,>->]& P_{1/0} \ar[r,>->,"\alpha_{1/0,2/0}"]\ar[u,{Circle[open]->},"\beta_{1/0}\circ \delta"] & P_{2/0}\ar[u,{Circle[open]->},"\beta_{2/0}",swap] \\
& O \ar[u,{Circle[open]->}] \ar[r,>->]  & P_{2/1} \ar[u,{Circle[open]->},swap,"\alpha_{2/1,2/0}"]
\end{tikzcd}\]

There is an unwanted twist in $\beta_{1/0}\circ\delta$. However, we have the freedom to choose whatever quotient of $P_2\rtail Q$ we want. The only important thing is that we get a 2-simplex in the end, which involves ensuring the quotient index diagrams agree.

So we may assume without loss of generality that the $\beta_{1/0}$ in Step 2a is in fact $\beta_{1/0}\circ\delta$ -- the argument still goes through, except at a notational cost.
	\end{comment}
	
		\item \textbf{Case 2:} $\E^\opp\subseteq \calC$. Dual to Case 1.
		\begin{comment}
		We now get the identity 
		$$\beta_{2/1,2/0}\circ \beta_{2/0}\circ q_2=\beta_{2/1}\circ q_2=\alpha_{2/1,2}= \alpha_{2/1,2/0}\circ \alpha_{2/0,2}=\alpha_{2/1,2/0}\circ \beta_{2/0}\circ q_2.$$
		Since $\beta_{2/0}\circ q_2=\alpha_{2/0,2}$ and $\E$-morphisms are epi in $\calC$ by Axiom (M), conclude that $\beta_{2/1,2/0}=\alpha_{2/1,2/0}$. Hence $\alpha_{1/0,2/0},\beta_{1/0,2/0}$ agree up to codomain-preserving isomorphism, and we may WLOG identify them, by the same argument as in Case 1.
		\end{comment}
	\end{itemize}
\end{proof}

\subsubsection*{Step 2c} Leveraging Step 2b, we construct two pairs of distinguished squares by a similar procedure.
\begin{itemize}%[label=$\diamond$]
	\item \textbf{First Pair.} Apply Axiom (DS) to the restricted pushouts of the spans 
	$$P_2\oplus P_{2/0}\ltail P_2\rtail W_0 \quad \text{and}\quad P_2\oplus P_{2/0}\ltail P_2\rtail W'_0, $$
and obtain distinguished squares
	\[\dsquaref{P_{2/0}}{P_{2/0}\oplus P_{2/0}}{W_0}{W_2}{}{\beta_{2/0}}{}{v_0}\qquad \dsquaref{P_{2/0}}{P_{2/0}\oplus P_{2/0}}{W'_0}{W'_2}{}{\beta'_{2/0}}{}{v_1}.\]
Composing with the distinguished squares constructed by Claim~\ref{claim:step-2b}, we obtain 
	\begin{equation}\label{eq:3c1}
\dsquaref{P_{1/0}}{P_{2/0}\oplus P_{2/0}}{Q}{W_2}{u_0}{\beta_{1/0} }{ }{ v_0} \qquad \dsquaref{P_{1/0}}{P_{2/0}\oplus P_{2/0}}{Q'}{W'_2}{u_1}{ \beta'_{1/0}}{}{ v_1 } 
\end{equation}
where the top maps are given by
\[
u_0=\begin{pmatrix}\alpha_{1/0,2/0}\\[4pt]0\end{pmatrix},\qquad
u_1=\begin{pmatrix}\alpha'_{1/0,2/0}\\[4pt]0\end{pmatrix}.
\]
We omit an explicit description of the $\E$-morphisms \(v_0,v_1\), since they play no further role in the argument.

\item \textbf{Second Pair.} Construct the following diagram pair
\[\begin{tikzcd}
P_{1/0}\oplus P_{2/0} \ar[dr,phantom,"\square",blue]\ar[d,{Circle[open]->}] &P_{1/0} \ar[l,>->,"1\oplus P_{2/0}",swap ] \ar[d,{Circle[open]->}] \ar[dr,phantom,"\square",red]\ar[r,>->,"\alpha_{1/0,2/0}"] & P_{2/0} \ar[d,{Circle[open]->}] \\
W_1 & P_{2}\oplus P_{1/0}\ar[r,>->]\ar[l,>->] & P_2\oplus P_{2/0} 
\end{tikzcd}\qquad \begin{tikzcd}
P_{1/0}\oplus P_{2/0} \ar[dr,phantom,"\square",blue]\ar[d,{Circle[open]->}] &P_{1/0} \ar[l,>->,"1\oplus P_{2/0}",swap] \ar[d,{Circle[open]->}] \ar[dr,phantom,"\square",red]\ar[r,>->,"\alpha'_{1/0,2/0}"] & P_{2/0} \ar[d,{Circle[open]->}] \\
W'_1 & P_{2}\oplus P_{1/0}\ar[r,>->]\ar[l,>->] & P_2\oplus P_{2/0} 
\end{tikzcd}\,\,;\]
the red squares are constructed by Lemma~\ref{lem:DirectSum} (v), the blue squares by applying Axiom (DS) to the restricted pushouts of the spans
$$P_{2}\oplus P_{1/0}\ltail P_{2}\rtail W_0 \qquad\text{and}\qquad P_{2}\oplus P_{1/0}\ltail P_{2}\rtail W'_0.$$
\begin{comment} Details. Apply Axiom (DS) to
$$P_2\oplus P_{1/0}\ltail P_2 \rtail W_0$$
and obtain
$$\begin{tikzcd}
O \ar[d,{Circle[open]->}] \ar[dr,phantom,"\square"]\ar[r,>->] & P_{1/0} \ar[dr,phantom,"\square"] \ar[d,{Circle[open]->}] \ar[r,>->]& P_{1/0}\oplus P_{2/0}\ar[d,{Circle[open]->}]  \\
P_{2}\ar[r,>->] & P_2\oplus P_{1/0} \ar[r,>->]& W_1
\end{tikzcd},$$
this gives the blue square. 
\end{comment}
Applying Axiom (DS) once more to the above diagram pair, we obtain
\begin{equation}\label{eq:3c2}
\dsquaref{P_{1/0}}{P_{2/0}\oplus P_{2/0}}{P_{2}\oplus P_{1/0}}{W_2}{u_0}{P_2\oplus 1 }{ }{ w_0} \qquad \dsquaref{P_{1/0}}{P_{2/0}\oplus P_{2/0}}{P_2\oplus P_{1/0}}{W'_2}{u_1}{ P_2\oplus 1}{}{ w_1 }.
\end{equation}
\end{itemize}
We then play the same game as in Step 2a. Applying Lemma~\ref{lem:DirectSum} (v), combine Diagrams~\eqref{eq:3c1} and \eqref{eq:3c2} to construct 
\begin{equation}\label{eq:3cDS}
\!\!\!\!\!\!\!\!\!\dsquaref{P_{1/0}\oplus P_{1/0}}{P_{2/0}\oplus P_{2/0}\oplus P_{2/0}\oplus P_{2/0}}{P_{2}\oplus P_{1/0}\oplus Q'}{W_2\oplus W'_2}{u_3}{t_0}{ }{z_0 } \, \dsquaref{P_{1/0}\oplus P_{1/0}}{P_{2/0}\oplus P_{2/0}\oplus P_{2/0}\oplus P_{2/0}}{Q\oplus P_{2}\oplus P_{1/0}}{W_2\oplus W'_2}{u_3}{t_1 }{ }{z_1 }
\end{equation}
whereby
\begin{equation*}
u_3=u_0\oplus u_1, \qquad z_0=w_0\oplus v_1, \qquad z_1=v_0\oplus w_1,
\end{equation*}
and $t_0,t_1$ are defined as in Step 2a.
\subsubsection*{Step 2d} We now construct our 2-simplex. Start by horizontally composing Diagrams~\eqref{eq:3a} and ~\eqref{eq:3cDS}. Then, vertically compose Diagram~\eqref{eq:3cDS} with the following (identical) pair of exact squares, obtained by applying Axiom (K):
\begin{equation}
\small{\dsquaref{O}{P_{2/0}\oplus P_{2/1}\oplus P_{2/0}\oplus P_{2/1}}{P_{1/0}\oplus P_{1/0}}{P_{2/0}\oplus P_{2/0}\oplus P_{2/0}\oplus P_{2/0}}{}{ }{u_3}{} }\quad \small{\dsquaref{O}{P_{2/0}\oplus P_{2/1}\oplus P_{2/0}\oplus P_{2/1}}{P_{1/0}\oplus P_{1/0}}{P_{2/0}\oplus P_{2/0}\oplus P_{2/0}\oplus P_{2/0}}{}{ }{u_3}{}}\quad.
\end{equation}
And we are done.


\subsubsection*{Step 3: Finish} We now prove $(P_2,P_2)\oplus \mu(l(\calT))$ is freely homotopic to $\calT$. Summarising our work above:
\begin{itemize}
	\item \textbf{Step 1:} $\calT\sim def^{-1}g^{-1}$.
	\item \textbf{Step 1:} $(P_2,P_2)\oplus \mu(l(\calT)) \sim \gamma_0ab^{-1}\gamma_2^{-1}.$
	\item \textbf{Step 1:} $gf\sim \gamma_2 b$. In fact, by Equation~\eqref{eq:compSPLIT}, their composites split:
	$$\lrangles{g}+\lrangles{f} = \lrangles{\overline{gf}}=\lrangles{\overline{\gamma_2b}}=\lrangles{\gamma_2}+\lrangles{b}.$$
	\item \textbf{Step 2:} We have the following 2-simplex  
$$	\begin{pmatrix}  &O\rtail & \doubleunderline{\,P_2\oplus P_2 \rtail  P_2\oplus P_{1/0}\oplus Q'\rtail W_2\oplus W'_2\, }\\ &O  \rtail & P_2\oplus P_2 \rtail  Q\oplus P_2\oplus P_{1/0}\rtail W_2\oplus W'_2\,\end{pmatrix}.
$$
\end{itemize}
By Step 1, our problem reduces to showing that the concatenation $de$ is homotopic to $\gamma_0 a$. Presented as generators of $\pi_1|G\calC ^o|$, this amounts to showing
$$\lrangles{\gamma_0}+\lrangles{a}-\lrangles{d} -\lrangles{e}=0.$$
By Step 2, the 2-simplex yields the equation\footnote{This step implicitly applies the extension of Proposition~\ref{prop:baseline} noted in Remark~\ref{rem:piGC}, allowing all 1-simplices in $G\calC^o$ rather than only double exact squares.}
\begin{align}\label{eq:step4}
\!\!\!\!\!\!\!\!\!\!\left\langle 	\begin{pmatrix}  &O\rtail & \doubleunderline{\,P_2\oplus P_2 \rtail  P_2\oplus P_{1/0}\oplus Q' \, }\\ &O  \rtail & P_2\oplus P_2 \rtail  Q\oplus P_2\oplus P_{1/0} \,\end{pmatrix} \right\rangle &+ \left\langle 	\begin{pmatrix}  &O\rtail & \doubleunderline{\,  P_2\oplus P_{1/0}\oplus Q' \rtail W_2\oplus W'_2 \, }\\ &O  \rtail & Q\oplus P_2\oplus P_{1/0} \rtail W_2\oplus W'_2\,\end{pmatrix} \right\rangle \nonumber  \\ 
&= \left\langle 	\begin{pmatrix}  &O\rtail & \doubleunderline{\,P_2\oplus P_2 \rtail  W_2\oplus W'_2 \, }\\ &O  \rtail & P_2\oplus P_2 \rtail  W_2\oplus W'_2 \,\end{pmatrix} \right\rangle . 
\end{align} 


Let us start by analysing the first term. Compute
$$\!\!\!\!\!\! \!\left\langle 	\begin{pmatrix}  &O\rtail & \doubleunderline{\, {\color{purple}P_2}\oplus {\color{blue} P_2} \rtail  {\color{purple} P_2\oplus P_{1/0}}\oplus {\color{blue} Q'} \, }\\ &O  \rtail & {\color{blue} P_2} \oplus {\color{purple}  P_2} \rtail {\color{blue} Q}\oplus  {\color{purple} P_2\oplus P_{1/0}} \,\end{pmatrix} \right\rangle  = \left\langle 	\begin{pmatrix}  &O\rtail & \doubleunderline{\, {\color{purple}P_2}\oplus {\color{blue} P_2} \rtail  {\color{purple} P_2\oplus P_{1/0}}\oplus {\color{blue} Q'} \, }\\ &O  \rtail & {\color{purple} P_2} \oplus {\color{blue}  P_2} \rtail {\color{purple} P_2\oplus P_{1/0}} \oplus  {\color{blue} Q} \,\end{pmatrix} \right\rangle   = \lrangles{\gamma_0} - \lrangles{d}; $$
the first equality applies Permutation Lemma~\ref{lem:permute}, the second equality applies Observation~\ref{obs:K1DirectSum}. The same permutation argument can be applied to the other terms of Equation~\eqref{eq:step4}. This assembles to yield
$$\lrangles{\gamma_0} - \lrangles{d} + \lrangles{a} - \lrangles{e} = \lrangles{\overline{\gamma_2b}} -\lrangles{\overline{gf}} = 0,$$
where the final equality is justified by Step 1, Equation~\eqref{eq:compSPLIT}, completing the proof.
\end{proof}


\begin{comment} 

For the second term, here's one way to show the desired identity (one can also show this directly, without appealing to diagonal squares):

\begin{align*}
\!\!\!\!\!\!\!\!\!\!\lrangles{a}-\lrangles{e} & -\left\langle 	\begin{pmatrix}  &O\rtail & \doubleunderline{\,P_2\oplus P_{1/0 } \oplus Q' \rtail  W_2\oplus W'_2\, }\\ &O  \rtail & Q\oplus P_2\oplus P_{1/0 } \rtail  W_2\oplus W'_2\,\end{pmatrix} \right\rangle \nonumber  \\
& =\left\langle \begin{pmatrix}  &O\rtail & \doubleunderline{\,{\color{purple}P_2\oplus P_{1/0}\oplus Q' }\oplus {\color{blue} Q\oplus P_2\oplus P_{1/0 } } \rtail  {\color{purple} W_2\oplus W'_2 }\oplus {\color{blue} W_2\oplus W'_2} \, }\\ &O  \rtail & {\color{blue} P_2\oplus P_{1/0} \oplus Q} \oplus {\color{purple} P_2\oplus P_{1/0 } \oplus Q' }\rtail  {\color{blue} W'_2\oplus W_2}\oplus {\color{purple} W_2\oplus W'_2}\,\end{pmatrix}\right\rangle \\
& =\left\langle \begin{pmatrix}  &O\rtail & \doubleunderline{\,{\color{purple}P_2\oplus P_{1/0}\oplus Q' }\oplus {\color{blue} Q\oplus P_2\oplus P_{1/0 } } \rtail  {\color{purple} W_2\oplus W'_2 }\oplus {\color{blue} W_2\oplus W'_2} \, }\\ &O  \rtail &  {\color{purple} P_2\oplus P_{1/0 } \oplus Q' }\oplus {\color{blue} P_2\oplus P_{1/0} \oplus Q} \rtail  {\color{purple} W_2\oplus W'_2} \oplus {\color{blue} W'_2\oplus W_2}\,\end{pmatrix}\right\rangle \\
& =\underbrace{\left\langle \begin{pmatrix}  &O\rtail & \doubleunderline{\,{\color{purple}P_2\oplus P_{1/0}\oplus Q' }\rtail  {\color{purple} W_2\oplus W'_2 }\, }\\ &O  \rtail &  {\color{purple} P_2\oplus P_{1/0 } \oplus Q' } \rtail  {\color{purple} W_2\oplus W'_2} \,\end{pmatrix}\right\rangle }_{=0} + \left\langle \begin{pmatrix}  &O\rtail & \doubleunderline{\, {\color{olive} Q} \oplus {\color{teal}  P_2\oplus P_{1/0 } } \rtail  {\color{olive} W_2}\oplus {\color{teal} W'_2} \, }\\ &O  \rtail &  {\color{teal} P_2\oplus P_{1/0}} \oplus {\color{olive}  Q} \rtail  {\color{teal} W'_2} \oplus {\color{olive} W_2}\,\end{pmatrix}\right\rangle \\
& =  \underbrace{\left\langle \begin{pmatrix}  &O\rtail & \doubleunderline{\, {\color{olive} Q} \oplus {\color{teal}  P_2\oplus P_{1/0 } } \rtail  {\color{olive} W_2}\oplus {\color{teal} W'_2} \, }\\ &O  \rtail & {\color{olive}  Q}\oplus  {\color{teal} P_2\oplus P_{1/0}}   \rtail  {\color{olive} W_2} \oplus  {\color{teal} W'_2} \,\end{pmatrix}\right\rangle}_{=0} = 0 
\end{align*}

For the last term, one approach would be to argue that it is diagonal. It is certainly true the objects match up, and so do the $\M$-morphisms (by the restricted pushout construction), but it is less clear why the $\E$-morphisms match up, since they may be different. The permutation argument makes things clearer: 

$$ \left\langle \begin{pmatrix}  &O\rtail & \doubleunderline{\,{\color{purple} P_2}\oplus {\color{blue} P_2} \rtail  {\color{purple} W_2}\oplus {\color{blue}  W'_2} \, }\\ &O  \rtail & {\color{blue} P_2}\oplus {\color{purple}  P_2} \rtail  {\color{blue} W_2} \oplus {\color{purple} W'_2} \,\end{pmatrix} \right\rangle =  \left\langle \begin{pmatrix}  &O\rtail & \doubleunderline{\,{\color{purple} P_2}\oplus {\color{blue} P_2} \rtail  {\color{purple} W_2}\oplus {\color{blue}  W'_2} \, }\\ &O  \rtail & {\color{purple} P_2}\oplus {\color{blue}  P_2} \rtail  {\color{purple} W'_2} \oplus {\color{blue} W_2} \,\end{pmatrix} \right\rangle =\lrangles{\overline{\gamma_2b}} - \lrangles{\overline{gf}}$$
\end{comment}



\begin{comment} Another detail: why do we know the third term in the 2-simplex actually corresponds to 
$$\lrangles{\overline{\gamma_2 b}}-\lrangles{\overline{gf}}?$$
Notice in Claim~\ref{claim:step-2b}, our flag diagram chose quotients such that the composition equals $\alpha_{2/1,2}$.

In essence: Step 0 simply took the restricted pushouts of the $\M$-morphisms. We left the choice of quotients open, and simply appealed to Corollary A.2 to argue that this construct the diagram of 1-simplices depicted in Step 1. 

In Step 2, we filled in the natural choices of the quotients. The quotients of the restricted pushouts can be anything. Our only real restriction is $\gamma_1$, or the $\M$-morphism of the associated double exact square of the triple. Step 2 was careful in preserving this.
\end{comment}

\begin{corollary}\label{cor:admtrip} If an admissible triple $\calT$ features only double exact squares, then its loop is freely homotopic to $\mu(l(\calT))$. 
\end{corollary}
\begin{proof}
	
	Consider the operation
	\begin{equation}
	(A,A')\oplus\argu \colon G\calC\to G\calC,
	\end{equation}
	which adds a vertex $(A,A')$ to all the nodes of an $n$-simplex of $G\calC$ (in the sense of Lemma~\ref{lem:DirectSum}). Given any edge $(A,A')\to (B,B')$, this induces a simplicial homotopy %\footnote{How? Change the direct summands of the filtration one level at a time. The simplices/distinguished squares are constructed coordinate-wise, relying on Lemma~\ref{lem:DirectSum} .} 
	between the maps 
	\begin{equation}
	(A,A')\oplus\argu \longrightarrow (B,B')\oplus\argu.
	\end{equation} 
	In particular, if the loop $\calT$ features only double exact squares, then there exists an edge $(O,O)\to (P_2,P_2)$. By the above, this implies $(P_2,P_2)\oplus \mu(l(\calT))$ is homotopic to $\mu(l(\calT))$. 
	\begin{comment} Alternatively: notice that $\gamma_0$ and $\gamma_2$ are diagonal double exact squares, and so their homotopy class in $K_1$ trivialises. One can then construct by hand the obvious 2-simplices that assemble into a free homotopy between $(P_2\oplus P_{1/0},P_2\oplus P_{1/0})\to (P_2\oplus P_{2/0},P_2\oplus P_{2/0})$ and $(  P_{1/0},  P_{1/0})\to (  P_{2/0},  P_{2/0})$.
	\end{comment}
\end{proof}


%\newpage 
\bibliography{biblio}

\end{document}
