\usepackage{lmodern}
\usepackage[T1]{fontenc}
\usepackage[utf8]{inputenc}

\usepackage{geometry}
\usepackage{extpfeil}
\usepackage[new]{old-arrows}
\usepackage{comment}
\usepackage[shortlabels]{enumitem}

\geometry{verbose}
\setcounter{secnumdepth}{2}
\setcounter{tocdepth}{1}
\setlength{\parskip}{\smallskipamount}
\setlength{\parindent}{0pt}
\usepackage[usenames,dvipsnames,svgnames,table]{xcolor}

\usepackage[maxbibnames=99, style=numeric]{biblatex}
\addbibresource{main.bib}
% \newbibmacro*{regularcite}{\usebibmacro{cite}}
% \renewbibmacro*{cite}{{\iffieldundef{shorthand}{\usebibmacro{regularcite}}{\usebibmacro{cite:shorthand}}}}


\usepackage{mathtools}
\usepackage{palatino}
\usepackage{mathpazo}
% \usepackage{eulervm}
%------doesn't run on overleaf-----%
%\usepackage{mbboard}
%----------------------------------%
\usepackage{mathrsfs}

%\usepackage{refstyle}
%QB: This conflicts with amsmath. Also we are just using cleverref.

\usepackage{amsmath}
\usepackage{amsthm}
\usepackage{amssymb}
\usepackage{mathdots}
\usepackage{todonotes}
\usepackage{debate}
\usepackage{nicefrac}
\usepackage{graphicx}
\usepackage{caption}
\usepackage{float}
\DeclareFontFamily{U}{min}{}
\DeclareFontShape{U}{min}{m}{n}{<-> udmj30}{}
\newcommand\yo{\!\text{\usefont{U}{min}{m}{n}\symbol{'207}}\!}
\usepackage[all]{xy}
\usepackage[unicode=true,pdfusetitle,
 bookmarks=true,bookmarksnumbered=false,bookmarksopen=false,
 breaklinks=false,pdfborder={0 0 0},pdfborderstyle={},backref=false,colorlinks=true]{hyperref}

\usepackage{graphicx}
\newcommand{\heart}{\ensuremath\heartsuit}
%for the heart symbol

\hypersetup{
    colorlinks, linkcolor=OliveGreen,
    citecolor=OliveGreen, urlcolor=OliveGreen
}
 
\usepackage[nameinlink,capitalise,noabbrev]{cleveref}


\usepackage{bookmark}
\bookmarksetup{numbered}


\makeatletter

% *** quiver ***
% A package for drawing commutative diagrams exported from https://q.uiver.app.
%
% This package is currently a wrapper around the `tikz-cd` package, importing necessary TikZ
% libraries, and defining a new TikZ style for curves of a fixed height.
%
% Version: 1.2.0
% Authors:
% - varkor (https://github.com/varkor)
% - AndréC (https://tex.stackexchange.com/users/138900/andr%C3%A9c)

\NeedsTeXFormat{LaTeX2e}
% \ProvidesPackage{quiver}[2021/01/11 quiver]

% `tikz-cd` is necessary to draw commutative diagrams.
\RequirePackage{tikz-cd}
% `amssymb` is necessary for `\lrcorner` and `\ulcorner`.
\RequirePackage{amssymb}
% `calc` is necessary to draw curved arrows.
\usetikzlibrary{calc}
% `pathmorphing` is necessary to draw squiggly arrows.
\usetikzlibrary{decorations.pathmorphing}

% A TikZ style for curved arrows of a fixed height, due to AndréC.
\tikzset{curve/.style={settings={#1},to path={(\tikztostart)
    .. controls ($(\tikztostart)!\pv{pos}!(\tikztotarget)!\pv{height}!270:(\tikztotarget)$)
    and ($(\tikztostart)!1-\pv{pos}!(\tikztotarget)!\pv{height}!270:(\tikztotarget)$)
    .. (\tikztotarget)\tikztonodes}},
    settings/.code={\tikzset{quiver/.cd,#1}
        \def\pv##1{\pgfkeysvalueof{/tikz/quiver/##1}}},
    quiver/.cd,pos/.initial=0.35,height/.initial=0}

% TikZ arrowhead/tail styles.
\tikzset{tail reversed/.code={\pgfsetarrowsstart{tikzcd to}}}
\tikzset{2tail/.code={\pgfsetarrowsstart{Implies[reversed]}}}
\tikzset{2tail reversed/.code={\pgfsetarrowsstart{Implies}}}
% TikZ arrow styles.
\tikzset{no body/.style={/tikz/dash pattern=on 0 off 1mm}}

%%%%%%%%%%%%%%%%%%%%%%%%%%%%%% Textclass specific LaTeX commands.
\newtheoremstyle{ctheorem}{}{}{}{}{\color{black}\bfseries}{}{ }{}
\theoremstyle{ctheorem}
\theoremstyle{definition}
\newtheorem{thm}{Theorem}[section]
\newtheorem{defn}[thm]{Definition}
\newtheorem{notation}[thm]{Notation}
\newtheorem{convention}[thm]{Convention}
\newtheorem{construction}[thm]{Construction}
\newtheorem{recollection}[thm]{Recollection}
\newtheorem{conjecture}[thm]{Conjecture}
\theoremstyle{definition}
\newtheorem{prop}[thm]{Proposition}
\theoremstyle{definition}
\newtheorem{lem}[thm]{Lemma}
\theoremstyle{definition}
\newtheorem{obsv}[thm]{Observation}
\theoremstyle{definition}
\newtheorem{rem}[thm]{Remark}
\theoremstyle{definition}
\newtheorem{example}[thm]{Example}
\theoremstyle{definition}
\newtheorem{cor}[thm]{Corollary}
\theoremstyle{definition}
\newtheorem{war}[thm]{Warning}
\theoremstyle{definition}
\newtheorem{que}[thm]{Question}
\theoremstyle{definition}
\newtheorem{ter}[thm]{Terminology}
\theoremstyle{definition}
\newtheorem{question}[thm]{Question}
\newtheorem{var}[thm]{Variant}
\theoremstyle{definition}


\usepackage{mymacros}
\usepackage{citationscheme}
\author{Yuxuan Hu\thanks{Northwestern University.}}

