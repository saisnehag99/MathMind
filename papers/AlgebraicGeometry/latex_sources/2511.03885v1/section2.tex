\section{Birational sequences}\label{sec2}
For $k,n\in \Z^+$, it is known that there is an identification of $\Gr\left(k,n\right)$ with $SL_n/P_k$, where $SL_n$ is the special linear complex group and $P_k$ is the parabolic subgroup of upper triangular block matrices of size $k\times k$ and $(n-k)\times (n-k)$ (cf. \cite[Chapter 5]{LakshmibaiBrown2015}). Consider the special linear Lie algebra $\sl_n$ and the exponential map $\exp:\sl_n\rightarrow SL_n$. Given the $\mathbf{A}_{n-1}$ type root system $\Phi$, let $\Phi^+=\lbrace \epsilon_i-\epsilon_j:(i<j)\in I_{2,n}\rbrace$ be the set of positive roots. For any $\beta=\epsilon_i-\epsilon_j\in \Phi^+$, define $f_{\beta}$ the $n\times n$ matrix with 1 in the entry $(i,j)$ and 0 in the others. Consider $x_{\beta}\in \C$ and $\exp\left(x_{\beta}f_{\beta}\right)\in SL_n$. Let
\begin{equation*}
    U_{\beta} :=\left\lbrace \mathbf{1}_{n\times n} + x_{\beta}f_{\beta}:x_{\beta}\in \C\right\rbrace\subset U^+,
\end{equation*} with $U^+\subset SL_n$ the subgroup of upper triangular matrices with 1s along the diagonal.

\begin{definition}[\cite{FANG2017107,Bossinger:2021}]\label{birseqs}
    Given $\beta_1,\dots,\beta_N\in \Phi^+$, the sequence $S=\left(\beta_1,\dots,\beta_N\right)$ is \emph{birational} for $\Gr\left(k,n\right)$ if the multiplication map
    \begin{equation*}
        \psi_S: U_{\beta_N}\times \dots \times U_{\beta_1}\rightarrow U^+
    \end{equation*} has image birational to $\Gr\left(k,n\right)$.
\end{definition}

Consider the following example.
\begin{ex}[PBW sequence]
    Let $\Phi_k^+ := \left\lbrace \epsilon_i-\epsilon_j:1\leq i\leq k<j\leq n\right\rbrace$ and let $S=\left(\beta_1,\dots,\beta_N\right)$ be any ordering of the elements of $\Phi_k^+$. This is called a \emph{PBW sequence}. The image of the multiplication map has elements of the form
    \begin{equation*}
        A = \begin{pmatrix}
            \mathbf{1}_{k\times k} & * \\
            \mathbf{0} & \mathbf{1}_{(n-k)\times (n-k)}
        \end{pmatrix},
    \end{equation*}  and the span of the first $k$ rows gives the required birational map. It is a straightforward computation to show that the ordering of the roots in this example does not modify the birational map.
\end{ex}

It will be seen that certain PBW sequences can be considered as belonging to the special subclass of \emph{iterated sequences}. In order to define this class, the following Lemma (cf. \cite[Lemma 1]{Bossinger:2021}) is needed.

\begin{lem}\label{iterated}
    Let $\beta_1,\dots,\beta_N\in \Phi^+$ (in $\sl_n$) and let $S' = \left(\beta_1,\dots,\beta_N\right)$ be a birational sequence for $\Gr\left(k,n\right)$. Let $i_1,\dots,i_k\in [n]$ be pairwise different indices. Then 
    \begin{equation*}
        S = \left(\epsilon_{i_1}-\epsilon_{n+1},\dots,\epsilon_{i_k}-\epsilon_{n+1},\beta_1,\dots,\beta_N\right)
    \end{equation*} is a birational sequence for $\Gr\left(k,n+1\right)$.
\end{lem}

This lemma helps construct new birational sequences which are not necessarily PBW. Furthermore, the proof gives a detailed construction of the birational map, describing both the map and its pullback. It is now possible to define the special class of \emph{iterated sequences}.

\begin{definition}\label{itseqs}
    Let $S'$ be a birational sequence for $\Gr\left(k,k+1\right)$. Obtain a birational sequence $S$ for $\Gr\left(k,n\right)$ by applying \cite[Lemma 1]{Bossinger:2021} $n-k-1$ times to $S'$. Then $S$ is called an \emph{iterated sequence}.
\end{definition}

\subsection*{Lowest-term valuations}
Now that the birational sequences have been defined and the examples of PBW and iterated sequences have been presented, it is possible to define a valuation on $A_{k,n}\setminus \lbrace 0\rbrace$.

Fix a birational sequence $S= \left(\beta_1,\dots,\beta_N\right)$ for $\Gr\left(k,n\right)$ and denote $\phi_S:\im\left(\psi_S\right)\dashrightarrow \Gr\left(k,n\right)$ the birational map. Consider the polynomial ring $\C\left[x_{\beta}:\beta\in S\right]$. Fix an identification
\begin{equation}\tag{*}
    x_{\beta_1}^{a_1}\dots x_{\beta_N}^{a_N} \quad \longleftrightarrow \quad \left(a_1,\dots,a_N\right). 
\end{equation} and let $<_{lex}$ be the lexicographic order on $\Z^N$.

Recall that a valuation is a function $\V_S:\C\left[x_{\beta}:\beta\in S\right]\setminus \lbrace 0\rbrace \rightarrow \left(\Z^N,\prec_S\right)$, with $\prec_S$ a monomial order on $\Z^N$, that satisfies: for all $f,g\in \C\left[x_{\beta}:\beta\in S\right]\setminus \lbrace 0\rbrace$ and $r\in \C^*$ it holds
\begin{equation*}
    \min_{\prec_S}\lbrace \V_S(f),\V_S(g)\rbrace\preceq_S \V_S\left(f+g\right),\qquad \V_S\left(fg\right) = \V_S\left(f\right) + \V_S\left(g\right), \qquad \V_S\left(rf\right) = \V_S\left(f\right).
\end{equation*}

In order to define the function $\V_S$, start by defining the \emph{height function} $\Ht:R^+\rightarrow \Z^+$ and the \emph{height-weighted function} $\Psi_S:\Z^N\rightarrow \Z^+$ as
\begin{equation*}
    \Ht\left(\epsilon_i-\epsilon_j\right) := j-i,\qquad \Psi_S\left(m_1,\dots,m_N\right) := \sum_{l=1}^N m_l\Ht(\beta_l).
\end{equation*}

The $\Psi_S$\emph{-weighted reverse lexicographic order} $\preceq_{S}$ is the monomial order defined as follows: $\x^{\a}\preceq_{\Psi_S} \x^{\b}$ if and only if $\Psi_S\left(\a\right)< \Psi_S\left(\b\right)$ or ($\Psi_S\left(\a\right) = \Psi_S\left(\b\right)$ and  $\a \geq_{lex} \b$).

Let $f=\sum_{\a\in \N^N}c_{\a}\x^{\a}$, with only finitely many $c_{\a}\neq 0$. Consider the function  defined by
\begin{equation*}
    \V_S:\C\left[x_{\beta}:\beta\in S\right]\setminus\lbrace 0\rbrace \rightarrow \left(\Z^N,\prec_S\right),\qquad \V_S\left(f\right) := \min_{\prec_S}\left\lbrace \a\in \N^N:c_{\a}\neq 0 \right\rbrace.
\end{equation*}
This function is, in fact, a valuation on $\C\left[x_{\beta}:\beta\in S\right]\setminus \lbrace 0\rbrace$. It is extended to $\C\left(x_{\beta}:\beta\in S\right)\setminus \lbrace 0\rbrace$ as $\V_S\left(\frac{f}{g}\right):=\V_S\left(f\right)-\V_S\left(g\right)$. Using the birational map $\phi_S$, define it on $\C\left(\Gr(k,n)\right)\setminus \lbrace 0\rbrace$ by $\V_S\left(h\right) :=\V_S\left(\phi_S^*(h)\right)$, and restrict to $A_{k,n}\setminus \lbrace 0\rbrace$. This is called the \emph{lowest term valuation} associated to the biration sequence $S$.

\subsection*{Weight homogeneity}
Given the definition of the lowest term valuation associated to a birational sequence $S$ for $\Gr\left(k,n\right)$, the next step is computing the valuations of Plücker coordinates in order to get the weighting matrix $M_{\V_S}$. This might, at first, seem not so straightforward since first the values of $\Psi_S$ must be computed and compared, and in the case of a tie, procede with the lexicographic order. 

It turns out that if $k=3$ and $S$ is iterated, this computation requires only the comparison using the lexicographic order. In order to prove this, the following lemma is necessary.

\begin{lem}[Weight Homogeinity]\label{homogeneidad}
    Let $S$ be an iterated sequence for $\Gr\left(3,n\right)$, $I=(i_1,i_2,i_3)\in I_{3,n}$ and $\overline{p_I}\in A_{3,n}$. Let $\a\in \N^{3(n-3)}$ be such that $\overline{\xa}\in \supp\left(\phi_S^*(\overline{p_I})\right)$. Then
    \begin{equation}
        \Psi_S\left(\a\right)=i_1+i_2+i_3-6.
    \end{equation}
\end{lem}
\begin{proof}
    Without loss of generality, let $(\beta_1,\beta_2,\beta_3)$ be a PBW sequence and $S$ the iterated sequence
    \begin{equation*}
        S=\left(\epsilon_{l_1}-\epsilon_n,\epsilon_{l_2}-\epsilon_n, \epsilon_{l_3}-\epsilon_n,\dots, \beta_1,\beta_2,\beta_3\right).
    \end{equation*} 

    Procede by induction: let $I=(i_1,i_2,i_3)\in I_{3,4}\setminus I_{3,3}$ and $j\in [3]$ be such that $I=\left((1,2,3)\setminus j\right)\cup 4$. Then $\supp\left( \phi_S^*\overline{p_I})\right)=\left\lbrace x_{j,4}\right\rbrace=\left\lbrace \x^{\a_{j,4}}\right\rbrace$ and
    \begin{equation*}
        \Psi_S\left(\a_{j,4}\right) = 4 - j = 1 + 2 + 3 + 4 -j -6 = i_1 + i_2 + i_3 - 6.
    \end{equation*}

    Assume that for all $J=(j_1,j_2,j_3)\in I_{k,n-1}$ and all $\a\in \Z^{3(n-3)}$ with $\xa\in \supp(\phi_S^*(\overline{p_I}))$, the following equality holds
    \begin{equation*}
        \Psi_S(\a)= j_1+j_2+j_3 - 6.
    \end{equation*}

    Let $I=(i_1,i_2,i_3)\in I_{3,n}\setminus I_{3,n-1}$. By the proof of \cite[Lemma 1]{Bossinger:2021}, the following equality holds
    \begin{equation*}
        \supp\left(\phi_S^*(\overline{p_I})\right)=\left\lbrace x_{l,n}\x^{\b}: \x^{\b}\in \supp(\phi_S^*(\overline{p_{I\setminus n\cup l}})),l\in \lbrace l_1,l_2,l_3\rbrace \wedge (l\not \in I\setminus n)\right\rbrace.
    \end{equation*}

    Let $l'\in (l_1,l_2,l_3)$ be such that $x_{l',n}\x^{\b}\in \supp(\phi_S^*(\overline{p_I}))$, with $\x^{\b}\in \supp\left(\phi_S^*(p_{I\setminus n\cup l'})\right)$. By induction, it holds
    \begin{equation*}
        \Psi(\b)= i_1 + i_2+i_3-n+l' -6. 
    \end{equation*} Let $\a\in \Z^{3(n-3)}$ be such that $\xa = x_{l',n}$. Notice that $\x^{\a+\b}=x_{l',n}\xb$. Then
    \begin{equation*}
        \Psi(\a+\b)=n-l' + i_1+i_2+i_3-n+l'-6 = i_1 + i_2+i_3-6.
    \end{equation*}
\end{proof}

The direct consequence of this lemma is Corollary \ref{only-depends-on-lex}, which significantly reduces the complexity of computations via the Algorithm \ref{algorithm}.

\begin{cor}\label{only-depends-on-lex}
    Let $S$ be an iterated sequence for $\Gr\left(3,n\right)$, $I=(i_1,i_2,i_3)\in I_{3,n}$ and $\overline{p_I}\in A_{3,n}$. Then
    \begin{equation*}
        \V_S\left(\overline{p_I}\right) = \max_{<_{lex}} \left\lbrace \mathbf{m}\in \Z^d:\x^{\mathbf{m}}\in \supp (\phi_S^*(\overline{p_I}))\right\rbrace.
    \end{equation*}
\end{cor}

In order to describe Algorithm \ref{algorithm}, used to compute the lowest term valuations, the following notation will be used: if $\lbrace i_1,i_2,i_3 \rbrace\subset [r-1]$ is the set used to iterate from a sequence for $\Gr\left(k,r-1\right)$ to a sequence for $\Gr\left(k,r\right)$, then $\beta_{i_j,r} := \epsilon_{i_j}-\epsilon_r$ for $j=1,2,3$. Furthermore, $f_l\in \Z^{3(n-3)}$ will denote the unitary vector with 1 in the $l$-th entry and 0 in the others.

\begin{algorithm}[H]
\caption{Computation of \emph{lowest term valuation} $\V_S$ of a Plücker coordinate}
\begin{algorithmic}[1]
\Require An iterated sequence $S$ and a multiindex $I\in I_{3,n}$.
\Statex
\State \textbf{Start:} $r=0$, $\mathbf{m}=\mathbf{0}\in \Z^{3(n-3)}$.
\While{$n-r\geq 4$}
    \State Compute $\V_S\left(\overline{p_I}\right)$ by adding unitary vectors and modifying the multiindex $I$.
    \If{$n-r\not\in I$}
        \State $r = r+1$.
    \Else
        \While{$n-r\in I$}
        \State $j=1$ and $\beta_{i_j,n-r}$
        \If{$i_j\not\in I$}
            \State $I= I\setminus (n-r)\cup i_{j}^{n-r}$, $\mathbf{m} = \mathbf{m}+f_{r+j}$, $r=r+1$.
        \Else 
        \State $j = j+1$.
        \EndIf
        \EndWhile
    \EndIf
\EndWhile
\Ensure $\mathbf{m} = \V_S(\overline{p_I})$.
\end{algorithmic}
\label{algorithm}
\end{algorithm}

Besides its usefulness in the computational implementation, Algorithm \ref{algorithm} also implies the following fact: the lowest term valuation $\V_S:\C\left[\Gr(3,n)\right]\setminus \lbrace 0\rbrace \rightarrow \Z^{3(n-3)}$ is a full-rank valuation (cf. Proof of \cite[Theorem 1]{Bossinger:2021}). To prove this it is enough to find a submatrix of the weighting matrix which is triangular with 1s along the diagonal. For example, when considering an iterated PBW sequence $S$ for the Grassmannian $\Gr\left(3,6\right)$ written as
\begin{equation*}
    S = \left(\beta_{1,6},\beta_{2,6},\beta_{3,6},\beta_{1,5},\beta_{2,5},\beta_{3,5},\beta_{1,4},\beta_{2,4},\beta_{3,4}\right).
\end{equation*}
Computing the lowest term valuation of the following Plücker coordinates and arranging them, in the exact order, in the weighting matrix, gives the matrix with 1s along the diagonal:
\begin{equation*}
    \overline{p_{(4,5,6)}} , \overline{p_{(1,5,6)}}, \overline{p_{(1,2,6)}}, \overline{p_{(3,4,5)}}, \overline{p_{(1,4,5)}}, \overline{p_{(1,2,5)}}, \overline{p_{(2,3,4)}}, \overline{p_{(1,3,4)}}, \overline{p_{(1,2,4)}}.
\end{equation*} 








































