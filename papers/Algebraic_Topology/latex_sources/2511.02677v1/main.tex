\documentclass[english]{article}
\usepackage{latexsym}
\usepackage{amscd}
\usepackage{amsmath}
\usepackage{amssymb}
\usepackage[mathscr]{euscript}
\usepackage{graphicx}
\usepackage{geometry}
\usepackage{mathtools}
\usepackage[all]{xy}
\usepackage[dvipsnames]{xcolor}
\usepackage[colorlinks,final,hyperindex]{hyperref}
\usepackage{tikz}
\usepackage{tikz-cd}
\usetikzlibrary{decorations.pathmorphing,decorations.markings,arrows,calc,shapes.geometric,arrows.meta,positioning}
\usepackage{enumerate}
\usepackage{multirow}

\usepackage[bb=dsserif]{mathalpha}
\usepackage{bm}

\newcommand{\QQ}{\ensuremath{\mathbb{Q}}}
\newcommand{\DD}{\ensuremath{\mathbb{D}}}
\newcommand{\CC}{\ensuremath{\mathbb{C}}}
\newcommand{\RR}{\ensuremath{\mathbb{R}}}
\newcommand{\ZZ}{\ensuremath{\mathbb{Z}}}
\newcommand{\NN}{\ensuremath{\mathbb{N}}}

\newcommand{\g}{\ensuremath{\mathfrak{g}}}
\renewcommand{\SS}{\ensuremath{\mathbb{S}}}

\newcommand{\id}{\ensuremath{\mathrm{id}}}

\makeatletter

\newcommand{\bbone}{\mathbb{1}}

\newcommand{\calA}{\mathcal{A}}
\newcommand{\calB}{\mathcal{B}}
\newcommand{\calC}{\mathcal{C}}
\newcommand{\calD}{\mathcal{D}}
\newcommand{\calE}{\mathcal{E}}
\newcommand{\calF}{\mathcal{F}}
\newcommand{\calG}{\mathcal{G}}
\newcommand{\calH}{\mathcal{H}}
\newcommand{\calI}{\mathcal{I}}
\newcommand{\calJ}{\mathcal{J}}
\newcommand{\calK}{\mathcal{K}}
\newcommand{\calL}{\mathcal{L}}
\newcommand{\calM}{\mathcal{M}}
\newcommand{\calN}{\mathcal{N}}
\newcommand{\calO}{\mathcal{O}}
\newcommand{\calP}{\mathcal{P}}
\newcommand{\calQ}{\mathcal{Q}}
\newcommand{\calR}{\mathcal{R}}
\newcommand{\calS}{\mathcal{S}}
\newcommand{\calT}{\mathcal{T}}
\newcommand{\calU}{\mathcal{U}}
\newcommand{\calV}{\mathcal{V}}
\newcommand{\calW}{\mathcal{W}}
\newcommand{\calX}{\mathcal{X}}
\newcommand{\calY}{\mathcal{Y}}
\newcommand{\calZ}{\mathcal{Z}}
\newcommand{\kk}{\Bbbk}

\newcommand{\scrY}{\mathscr{Y}}
\newcommand{\scrV}{\mathscr{V}}
\newcommand{\scrP}{\mathscr{P}}

\newcommand{\Rep}{\mathop{\mathrm{Rep}}\nolimits}
\newcommand{\alg}{\mathop{\mathrm{alg}}\nolimits}
\newcommand{\Span}{\mathop{\mathrm{Span}}\nolimits}
\newcommand{\proj}{\mathop{\mathrm{proj}}\nolimits}
\newcommand{\Hom}{\mathop{\mathrm{Hom}}\nolimits}
\newcommand{\PHom}{\mathop{\mathrm{PHom}}\nolimits}
\newcommand{\Tw}{\mathop{\mathrm{Tw}}\nolimits}
\newcommand{\Dim}{\mathop{\mathrm{Dim}}\nolimits}
\newcommand{\Imm}{\mathop{\mathrm{Im}}\nolimits}
\newcommand{\Aus}{\mathop{\mathrm{Aus}}\nolimits}
\newcommand{\Ann}{\mathop{\mathrm{Ann}}\nolimits}
\newcommand{\APC}{\mathop{\mathrm{APC}}\nolimits}
\newcommand{\Barr}{\mathop{\mathrm{Bar}}\nolimits}
\newcommand{\Cone}{\mathop{\mathrm{Cone}}\nolimits}
\newcommand{\modu}{\mathop{\mathrm{mod}}\nolimits}
\newcommand{\dgMod}{\mathop{\mathrm{dgMod}}\nolimits}
\newcommand{\soc}{\mathop{\mathrm{soc}}\nolimits}
\newcommand{\dg}{\mathop{\mathrm{dg}}\nolimits}
\newcommand{\red}{\mathop{\mathrm{red}}\nolimits}
\newcommand{\Map}{\mathop{\mathrm{Map}}\nolimits}
\newcommand{\inj}{\mathop{\mathrm{inj}}\nolimits}
\newcommand{\op}{\mathop{\mathrm{op}}\nolimits}
\newcommand{\Ker}{\mathop{\mathrm{Ker}}\nolimits}
\newcommand{\Tor}{\mathop{\mathrm{Tor}}\nolimits}
\newcommand{\Tot}{\mathop{\mathrm{Tot}}\nolimits}
\newcommand{\Ext}{\mathop{\mathrm{Ext}}\nolimits}
\newcommand{\sg}{\mathop{\mathrm{sg}}\nolimits}
\newcommand{\Sl}{\mathop{\mathfrak{sl}}\nolimits}
\newcommand{\nc}{\mathop{\mathrm{nc}}\nolimits}
\newcommand{\Tr}{\mathop{\mathrm{Tr}}\nolimits}
\newcommand{\Irr}{\mathop{\mathrm{Irr}}\nolimits}
\newcommand{\HC}{\mathop{\mathrm{HC}}\nolimits}
\newcommand{\HH}{\mathop{\mathrm{HH}}\nolimits}
\newcommand{\THH}{\mathop{\mathrm{TH}}\nolimits}
\newcommand{\Perf}{\mathop{\mathrm{Perf}}\nolimits}
\newcommand{\del}{\partial}
\newcommand{\ev}{\mathop{\mathrm{ev}}\nolimits}
\newcommand{\co}{\mathop{\mathrm{co}}\nolimits}
\newcommand{\pt}{\mathop{\mathrm{pt}}\nolimits}
\newcommand{\wt}{\mathop{\mathrm{wt}}\nolimits}
\newcommand{\pCY}{\mathop{\mathrm{pCY}}\nolimits}
\newcommand{\gr}{\mathop{\mathrm{gr}}\nolimits}
\newcommand{\coker}{\mathop{\mathrm{coker}}\nolimits}


\newcommand{\Top}{\mathop{\mathrm{Top}}\nolimits}
\newcommand{\Set}{\mathop{\mathrm{Set}}\nolimits}
\newcommand{\Vect}{\mathop{\mathrm{Vect}}\nolimits}
\newcommand{\Mod}{\mathop{\mathrm{Mod}}\nolimits}
\newcommand{\Ob}{\mathop{\mathrm{Ob}}\nolimits}
\newcommand{\Ind}{\mathop{\mathrm{Ind}}\nolimits}
\newcommand{\Sym}{\mathop{\mathrm{Sym}}\nolimits}
\newcommand{\Alt}{\mathop{\mathrm{Alt}}\nolimits}
\newcommand{\End}{\mathop{\mathrm{End}}\nolimits}
\newcommand{\Ch}{\mathop{\mathrm{Ch}}\nolimits}


\newcommand{\Ass}{\mathop{\mathrm{Ass}}\nolimits}
\newcommand{\Com}{\mathop{\mathrm{Com}}\nolimits}
\newcommand{\Lie}{\mathop{\mathrm{Lie}}\nolimits}
\newcommand{\colim}{\mathop{\mathrm{colim}}\nolimits}

\newcommand{\Op}{\mathop{\mathsf{Operads}}\nolimits}
\newcommand{\Diop}{\mathop{\mathsf{Dioperads}}\nolimits}
\newcommand{\PDiop}{\mathop{\mathsf{PDioperads}}\nolimits}
\newcommand{\Coop}{\mathop{\mathsf{Cooperads}}\nolimits}
\newcommand{\Codiop}{\mathop{\mathsf{Codioperads}}\nolimits}
\newcommand{\PCodiop}{\mathop{\mathsf{PCodioperads}}\nolimits}

\newcommand{\antish}{\ensuremath{\mbox{!`}}}


\tikzset{->-/.style={decoration={
			markings,
			mark=at position #1 with {\arrow{>}}},postaction={decorate}}}
\tikzset{-w-/.style={decoration={
			markings,
			mark=at position #1 with {\arrow{Stealth[fill=white,scale=1.4]}}},postaction={decorate}}}
\tikzset{->-/.default=0.65}
\tikzset{-w-/.default=0.65}
\tikzstyle{bullet}=[circle,fill=black,inner sep=0.5mm]
\tikzstyle{circ}=[circle,draw=black,fill=white,inner sep=0.5mm]
\tikzstyle{vertex}=[circle,draw=black,thick,inner sep=0.5mm]
\tikzstyle{dot}=[draw,circle,fill=black,minimum size=0.5mm,inner sep = 0mm, outer sep = 0mm]
\tikzset{darrow/.style={double distance = 4pt,>={Implies},->},
	darrowthin/.style={double equal sign distance,>={Implies},->},
	tarrow/.style={-,preaction={draw,darrow}},
	qarrow/.style={preaction={draw,darrow,shorten >=0pt},shorten >=1pt,-,double,double
		distance=0.2pt}}
\newcommand\tikcirc[1][2.5]{\tikz[baseline=-#1]{\draw[thick](0,0)circle[radius=#1mm];}}
\newcommand{\tikzfig}[1]{\begin{tikzpicture}[auto,baseline={([yshift=-.5ex]current bounding box.center)}]#1\end{tikzpicture}}
\usetikzlibrary{decorations.pathreplacing}

\usepackage{amsthm}
\usepackage{thmtools}
\usepackage[noabbrev,capitalize]{cleveref}
\newtheorem{theorem}{Theorem}
\newtheorem{lemma}[theorem]{Lemma}
\newtheorem{proposition}[theorem]{Proposition}
\newtheorem{corollary}[theorem]{Corollary}
\newtheorem{conjecture}[theorem]{Conjecture}
\newtheorem{Claim}[theorem]{Claim}

\newtheorem*{theorem*}{Theorem}

\theoremstyle{definition}
\newtheorem{definition}{Definition}
\newtheorem{example}{Example}
\newtheorem{cexample}{Counterexample}
\newtheorem{exercise}{Exercise}
\newtheorem{question}{Question}
\newtheorem{remark}{Remark}

\addtolength{\textheight}{1.75cm}
\addtolength{\voffset}{-0.5cm}
\addtolength{\footskip}{+0.6cm}
\geometry{margin=1.2in}


\usepackage[backend=biber, maxbibnames=99, bibencoding=utf8, doi=false, isbn=false, url=false, style=alphabetic]{biblatex}
\addbibresource{refs.bib}


\begin{document}
\title{Proper kernels in microlocal sheaf theory}
\maketitle
\begin{abstract}
Let $X$ and $Y$ be real analytic manifolds and let $\Lambda \subseteq T^*X$ and
$\Sigma \subseteq T^*Y$ be closed conic subanalytic singular isotropics.
Given a sheaf $K \in \Sh_{-\Lambda \times \Sigma}(X \times Y)$ microsupported
in $-\Lambda \times \Sigma$, consider the convolution functor
\(
(-) \ast K \colon \Sh_{\Lambda}(X) \rightarrow \Sh_{\Sigma}(Y)
\)
from sheaves microsupported in $\Lambda$ to sheaves microsupported in $\Sigma$.
We show that the convolution functor $(-) \ast K$ preserves compact objects
if and only if for each $x \in X$, the restriction
$K|_{\{x\} \times Y} \in \Sh_\Sigma(Y)$ is a compact object.
By a result of Kuo-Li \cite{kuo-li-duality},
the functor sending a sheaf kernel $K$ to the convlution functor $(-) \ast K$ is
an equivalence
between the category $\Sh_{-\Lambda \times \Sigma}(X \times Y)$ of sheaves
microsupported in $-\Lambda \times \Sigma$ and the category of cocontinuous functors
from $\Sh_\Lambda(X)$ to $\Sh_\Sigma(Y)$.
We therefore classify all cocontinuous functors that preserve compact objects
between the two categories.
Our approach is entirely categorical and requires minimal input from geometry:
we introduce the notion of a proper object in a compactly
generated stable $\infty$-category and study its properties under
strongly continuous localizations to obtain the result.
The main geometric input is the analysis of compact and proper objects of
the category of $P$-constructible sheaves for a triangulation $P$ of a manifold $Z$
via the exit path category $\Exit(Z, P) \simeq P$.
Along the way, we show that a sheaf $F \in \Sh_\Lambda(X)$
is proper if and only if it has perfect stalks,
which is equivalent to a result of Nadler.
\end{abstract}
\tableofcontents
\section{Introduction}
\subsection{Motivation and background}
Let $X$ be a real analytic manifold and let $\Lambda \subseteq T^*X$ be a closed conic subanalytic singular isotropic.
The work of Kuo-Li \cite{kuo-li-duality} shows that the category
$\Sh_\Lambda(X)$ of sheaves microsupported in $\Lambda$ is a \emph{dualizable}
stable $\oo$-category (cf. \cref{defn:dualizable}).
More precisely, let $Y$ be a real analytic manifold and $\Sigma$ be a closed conic subanalytic singular isotropic,
it is shown there \cite[Theorem 1.1, Theorem 1.2, Corollary 1.7]{kuo-li-duality} that
the dual of $\Sh_\Lambda(X)$ is given by $\Sh_{-\Lambda}(X)$:
\[
	\Sh_\Lambda(X)^\vee \simeq \Sh_{-\Lambda}(X),
\]
the Kunneth formula holds:
\[
	\Sh_\Lambda(X) \otimes \Sh_\Sigma(Y) \simeq \Sh_{\Lambda \times \Sigma}(X \times Y),
\]
and that the equivalence
\[
	\Sh_{-\Lambda\times \Sigma}(X\times Y) \simeq \Sh_\Lambda(X)^\vee \otimes \Sh_\Sigma(Y) \simeq \Fun^L(\Sh_\Lambda(X), \Sh_\Sigma(Y))
\]
is given by the assignment
\[
	K \mapsto (-) \ast K,
\]
where
\[
	(-) \ast K \coloneqq \pi_{2,!}(\pi_1^* (-) \otimes K)
\]
is the convolution functor. Here $\Fun^L(-,-)$ denotes the category of cocontinuous functors, and $\pi_1 \colon X\times Y \rightarrow X$ and
$\pi_2 \colon X \times Y \rightarrow Y$ are the projections.

In conclusion, cocontinuous functors between the sheaf categories
$\Sh_\Lambda(X)$ and $\Sh_\Sigma(Y)$ are classified by sheaf kernels
$K \in \Sh_{-\Lambda\times \Sigma}(X\times Y)$ on $X \times Y$ microsupported
in $-\Lambda \times \Sigma$.

On the other hand, previous works \cite{nadler-zaslow, nadler-microlocal-branes, gps3} have drawn parallels between
microlocal sheaf theory and the theory of
Fukaya categories in various flavors.
In particular, assuming $\Lambda \subseteq T^*X$ contains the zero section $0_X$,
it is shown in \cite{gps3} that there is an equivalence
between the category of sheaves of $\ZZ$-modules microsupported in $\Lambda$ and
the ind-completion of the partially wrapped Fukaya category of $T^*X$
stopped at $-\Lambda_\infty$:
\[
	\Sh_\Lambda(X; \ZZ\mathhyphen\Mod) \simeq \Ind\Wc(T^*X, -\Lambda_\infty).
\]
Here $\Lambda_\infty$ is the projection of $\Lambda - 0_X$ to the
cosphere bundle $S^*X$.
Taking compact objects on both sides,
we have
\[
	\Sh_\Lambda(X;\ZZ\mathhyphen\Mod)^\omega \simeq \Perf \Wc(T^*X,-\Lambda_\infty),
\]
where $\Perf \Wc(T^*X, -\Lambda)$ is the idempotent completion of $\Wc(T^*X,-\Lambda_\infty)$.\footnote{Strictly speaking, it is only possible to compare the category on
the left and that on the right \emph{up to Morita equivalence},
since the identification of $R$-linear stable $\oo$-categories, dg-categories,
and $A_\infty$-categories requires the Morita model structure.
In this sense, the equivalence only exists up to some replacement
in the model category in the first place, and it is technically redundant to
explictly mention idempotent completion.}
In \cite{nadler-wrapped}, Nadler first introduced the notion of
\emph{wrapped sheaves}, which are by definition, compact objects in $\Sh_\Lambda(X)$.
In \cite{kuo-wrapped-sheaves}, Kuo showed that the category of wrapped sheaves
can also be realized geometrically, without alluding to compactness
in the ambient category.

Consequently, one can argue that studying compact objects
in $\Sh_\Lambda(X)$ is not a purely academic pursuit,
but is of intrinsic geometric interest from the viewpoint of Floer theory.
Given Liouville manifolds $M$ and $N$,
and a Lagragian correspondence $\Lc \subseteq M^{-} \times N$,
Gao \cite{gao-correspondence} showed that, if $\Lc \rightarrow N$ is proper,
under some genericity conditions,
there is an induced $A_\infty$-functor
\[
	\Theta_{\Lc} \colon \Wc(M) \rightarrow \Wc(N),
\]
which on objects is given by geometric composition of Lagrangians:
\[
	M \supseteq L \mapsto L \circ \Lc \subseteq N.
\]
A natural question to ask is what the sheaf-theoretic incarnation
of the above functor is.
Note that the category 
\[
	\Fun^\ex(\Sh_\Lambda(X)^\omega, \Sh_\Sigma(Y)^\omega)
\]
of exact functors is equivalent to the subcategory of
\[
	\Fun^L(\Sh_\Lambda(X), \Sh_\Sigma(Y)) \simeq \Sh_{-\Lambda \times \Sigma}(X \times Y)
\]
spanned by functors that preserve compact objects.
Therefore, an equivalent question to ask is:
\begin{question}
Under what conditions on the sheaf kernel
$K \in \Sh_{-\Lambda \times \Sigma}(X\times Y)$, does the convolution functor
\[
	(-) \ast K \colon \Sh_{\Lambda}(X) \rightarrow \Sh_{\Sigma}(Y)
\]
preserve compact objects?
\end{question}
Our main result provides a complete answer to this question. As a consequence, we can verify the following special case, which was conjectured by Ganatra-Kuo-Li-Wu (see \cref{example:conj-ganatra}).

\begin{conjecture}[Ganatra-Kuo-Li-Wu]\label{conj-ganatra}
Let $K \in \Sh_{-\Lambda \times \Sigma}(X \times Y)$ be a sheaf kernel. If $\ssupp(K) \rightarrow T^*Y$ is proper and $K$ has perfect stalks, assuming $Y$ is compact,
then $(-) \ast K$ preserves compact objects.
\end{conjecture}

\begin{rem}
Ganatra-Kuo-Li-Wu pursue a geometric approach to this question in \cite{Ganatra-Kuo-Li-Wu}, using techniques based on wrappings. Their work actually addresses the following more general conjecture.
\end{rem}
\begin{conjecture}[Ganatra-Kuo-Li-Wu]\label{conj-complete}
Let $L \in \Sh(X \times Y)$ be a constructible sheaf with perfect stalks,
not necessarily microsupported in ${-\Lambda \times \Sigma}$.
Certain geometric constraints on $\ssupp(L)$ guarantee that $\mathfrak{M}^+_{-\Lambda \times \Sigma}(L) \ast (-)$ preserves compact objects. \footnote{Here $\mathfrak{M}^{+}_{-\Lambda \times \Sigma}$ is the positive wrapping functor introduced in \cite{kuo-wrapped-sheaves}, which is equivalent to the localization functor $\iota^*_{-\Lambda \times \Sigma} \colon \Sh(X) \rightarrow \Sh_{-\Lambda \times \Sigma}(X \times Y)$ in our notation.}
\end{conjecture}
\subsection{Main results and overview}
In this section, fix real analytic manifolds $X$ and $Y$
and closed conic subanalytic singular isotropics $\Lambda \subseteq T^*X$
and $\Sigma \subseteq T^*Y$.
Our main result is the following.
\begin{thm}[\cref{main-thm}]
Let $K \in \Sh_{-\Lambda\times \Sigma}(X\times Y)$ be a sheaf kernel.
The convolution functor
\[
	(-) \ast K\colon \Sh_\Lambda(X) \rightarrow \Sh_\Sigma(Y)
\]
preserves compact objects if and only if for every $x \in X$,
the restriction $K|_{\{x\} \times Y} \in \Sh_{\Sigma}(Y)$ is a compact object.
Consequently, there is an equivalence
\[
	\Pc \simeq \Fun^{\ex}(\Sh_\Lambda(X)^\omega,\Sh_\Sigma(Y)^\omega),
\]
where $\Pc \subseteq \Sh_{-\Lambda \times \Sigma}(X \times Y)$
is the full subcategory spanned by such sheaf kernels.
\end{thm}
From this, we can deduce a sufficient condition.
\begin{cor}[\cref{perfect-stalk-and-compact-support}]
Let $K \in \Sh_{-\Lambda \times \Sigma}(X \times Y)$ be a sheaf kernel.
If $K$ has perfect stalks and $\supp(K|_{\{x\} \times Y})$ is compact for every $x \in X$,
then convolution with $K$ preserves compact objects.
\end{cor}
\begin{rem}\label{example:conj-ganatra}
If $Y$ is compact, then $\supp(K|_{\{x\} \times Y})$ is always compact.
In this case, if $K$ has perfect stalks, then convolution with $K$ preserves
compact objects. In particular, \cref{conj-ganatra} is true.
\end{rem}
The core to our argument is the notion of a
\emph{proper} object
in a compactly generated stable $\oo$-category.
\begin{defn}[\cref{defn:proper-object}]
Let $\Cc$ be a presentable stable $\infty$-category.
We say $c\in\Cc$ is \emph{proper} if the functor
\[
\map_\Cc(-,c) \colon \Cc^{\omega,\op} \to \Sp
\]
factors through $\Sp^\omega \subseteq \Sp$.
\end{defn}
The following result on proper objects in $\Sh_\Lambda(X)$ can be also
seen as a special case of the main theorem.
\begin{thm}[\cref{proper-objects-in-shv}]
A sheaf $F \in \Sh_\Lambda(X)$ is proper if and only if $F$ has perfect stalks.
\end{thm}
The result above, albeit stated in a slightly different setting,
was first proved as \cite[Theorem 3.21]{nadler-wrapped} using
arborealization,
and later proved again as \cite[Corollary 4.24]{gps3} with
a more direct argument.
The main point of this paper is that,
once we distill the essential ideas in its proof
to categorical terms, the argument can be
further simplified and adapted to a relative setting,
allowing us to prove the main theorem of this paper.
\subsection{Notations and conventions}
For the sake of brevity and clarity,
we will work exclusively with sheaves of spectra, unless otherwise specified.
\begin{rem}
All our arguments work mutatis mutandis if
we replace the $\infty$-category $\Sp$ of spectra with
any compactly generated rigid monoidal $\infty$-category $\Vc$,
and argue in the context of $\Vc$-enriched categories instead.
Much of the theory of $\Vc$-enriched category is developed in
\cite{gepner-haugseng, hinich18, hinich21, heine-equivalence, benmoshe-naturality}.
For a quick review on the theory of presentable and dualizable categories
in the enriched setting directly applicable to this paper, see \cite[\S 1]{ramzi2024dualizable}.
\end{rem}
\begin{notation}
Let $\Cc$ be a stable $\oo$-category.
Throughout this paper,
$\map_\Cc(-,-)$ denotes the mapping spectrum,
while $\Map_\Cc(-,-)$ denotes the mapping space.
\end{notation}
\subsection*{Acknowledgements}
I am grateful to Haosen Wu for bringing this problem to my attention and sharing an early draft of the joint work \cite{Ganatra-Kuo-Li-Wu}, and to Wenyuan Li for answering many questions about that work.
This project began as an attempt to give a categorical proof of a geometric result pursued in \cite{Ganatra-Kuo-Li-Wu}.
While the theorem we prove here is more general than originally anticipated, it addresses a different (though related) question than the one studied there.
I thank Qingyuan Bai, Peter Haine, and Gus Schrader for reading an early draft of this paper, and Peter Haine in particular for very detailed comments.
I was supported by the NSF grant DMS 2302624.

\section{Preliminaries}
\subsection{Exodromy}
Consider a topological space $X$ with a stratification $P$,
or simply a \emph{stratified space} $(X,P)$.
\begin{recollection}
A \emph{stratified space} $(X,P)$ is a topological space $X$,
together with a continuous map $X \rightarrow P$, where $P$ is a poset
equipped with the Alexandroff topology.
\end{recollection}
The \emph{exodromy equivalence}
\begin{comment}
was first proposed by MacPherson,
and studied by Treumann \cite{treumann2009exit}.
It has since been genralized by Lurie \cite[Appendix A]{HA},
Lejay \cite{lejay2021constructible}, Porta-Teyssier \cite{porta-teyssier},
and Haine-Porta-Teyssier \cite{haine2024exodromy}.
\end{comment}
(see \cite{treumann2009exit, HA, lejay2021constructible, porta-teyssier, haine2024exodromy})
states that under suitable assumptions
on $(X,P)$ and $\Vc$, the category
\[
	\Cons_P(X;\Vc)
\]
of $P$-constructible sheaves \footnote{Technically, to state the most general result,
one has to consider hyper-constructible hypersheaves.
However, the distinction between hyper-constructible hypersheaves
and constructible sheaves disappears when everything is hypercomplete,
as is the case if $X$ is a finite-dimensional manifold and all the strata
are submanifolds.}
on $X$ valued in $\Vc$
is equivalent to the category
\[
	\Fun(\Exit(X,P), \Vc)
\]
of functors from the \emph{exit-path $\oo$-category}
$\Exit(X,P)$ to $\Vc$.
While the construction of $\Exit(X,P)$ is quite involved in general,
the only result we need in this paper is the following.
\begin{prop}
Let $P$ be a triangulation of a manifold $X$. Then $\Exit(X,P) \simeq P$.
\end{prop}
\begin{proof}
This is a special case of \HA{Theorem}{A.6.10}.
\end{proof}
\begin{cor}\label{cons-p-fun-p}
Let $P$ be a triangulation of a manifold $X$.
Then $ \Cons_P(X) \simeq \Fun(P, \Sp)$.
\end{cor}

\subsection{Dualizable stable $\infty$-categories}
We give a quick recap of the theory of dualizable stable $\oo$-categories.

The $\infty$-category $\Prlst$ of presentable stable $\infty$-categories and left adjoints admits
a closed symmetric monoidal structure given by tensor product in $\Prl$.
The internal hom is given by $[\Cc, \Dc] = \Fun^L(\Cc, \Dc)$.

The $\infty$-category $\Catperf$ of small idempotent complete stable $\infty$-categories
has a symmetric monoidal structure where the tensor product $\Cc_0 \otimes \Dc_0$
classifies bi-exact functors out of $\Cc_0 \times \Dc_0$.
It is again a closed monoidal category with internal homs given by
$[\Cc_0, \Dc_0] = \Fun^{\ex}(\Cc_0, \Dc_0)$.

The ind-construction
\[
\Ind \colon \Catperf \to \Prlst
\]
makes $\Catperf$ a wide subcategory of $\Prlst$, with
its essential image spanned by compactly generated stable $\infty$-categories
and left adjoints that preserve compact objects.

Moreover, $\Ind$ is a symmetric monoidal fuctor.
The ind construction and taking compact objects give inverse equivalences
\[
	\Ind \colon \Catperf \leftrightarrows {\Pr}^{L,\omega}_\st \colon (-)^\omega,
\]
where $\Pr^{L,\omega}_\st$ is the $\oo$-category of compactly generated
stable $\oo$-categories and left adjoints that preserve compact objects.
\begin{defn}\label{defn:dualizable}
A presentable stable $\infty$-category $\Cc$ is \emph{dualizable} if
there exists $\Cc^\vee$ so that the functor
\[
- \otimes \Cc \colon \Prlst \to \Prlst.
\]
is left adjoint to
\[
- \otimes \Cc^\vee \colon \Prlst \to \Prlst.
\]
\end{defn}
\begin{example}[{{\SAG{Proposition}{D.7.2.3}}}]
If $\Cc$ is a compactly generated stable $\oo$-category, then $\Cc$ is
dualizable. The dual is given by $\Cc^\vee \simeq \Ind(\Cc^{\omega,\op})$,
and the evaluation map $\ev: \Cc^\vee \otimes \Cc \to \Sp$ is given by
the left Kan extension of
\[
  \map_\Cc(-, -) \colon \Cc^{\omega,\op} \otimes \Cc^\omega \to \Sp.
\]
along $\Cc^{\omega,\op} \otimes \Cc^\omega \to \Cc^\vee \otimes \Cc \simeq \Ind(\Cc^{\omega,\op} \otimes \Cc^\omega)$.
\end{example}
\begin{recollection}
A left adjoint functor between presentable $\oo$-categories is called
an \emph{internal left adjoint} in $\Pr^L$, if it is the left adjoint of
an adjunction in the $(\infty,2)$-category $\mathbf{Pr}^L$ of presentable $\oo$-categories and left adjoints.
Being an internal left adjoint is equivalent to being a \emph{strongly cocontinuous functor}: a functor whose right adjoint admits a further right adjoint.
\end{recollection}
\begin{rem}\label{internal-left-adjoint-between-compactly-generated}
A left adjoint functor between compactly generated presentable stable $\infty$-categories
is an internal left adjoint if and only if it preserves compact objects.
Therefore, $\Catperf \simeq \Pr^{L,\omega}_\st$ is equivalent to
the $\oo$-category of compactly generated stable $\oo$-categories and internal
left adjoints.
\end{rem}
\begin{defn}
The $\infty$-category $\Pr^{\dual}_\st$ of dualizable stable $\infty$-categories
is the wide subcategory of $\Prlst$ spanned by dualizable objects
and internal left adjoints in $\Prlst$.
\end{defn}
\begin{rem}
By \cref{internal-left-adjoint-between-compactly-generated}, $\Catperf \simeq \Pr^{\dual}_\st$ is the full subcategory of $\Pr^\dual_\st$ spanned
by compactly generated stable $\oo$-categories.
\end{rem}
\begin{thm}[{{\SAG{Proposition}{D.7.3.1}}}]
  A presentable stable $\infty$-category is dualizable
  if and only if it is the retract of a compactly
  generated one in $\Prlst$.
\end{thm}
\begin{comment}
\begin{obsv}
Let $\Cc$ and $\Dc$ be compactly generated presentable stable $\infty$-categories
and $K \in \Cc \otimes \Dc$.
Under the equivalences
\[
\Cc \otimes \Dc \simeq \Fun^L(\Cc^\vee, \Dc) \simeq \Fun^{\ex}(\Cc^{\omega,\op}, \Dc),
\]
the object $K \in \Cc \otimes \Dc$
corresponds to the composition
\[
\Cc^{\omega, \op}
\xrightarrow{(\id,  K)} \Cc^{\omega, \op} \times \Cc \otimes \Dc
\xrightarrow{\map_\Cc(-, -) \otimes \id} \Dc,
\]
where $\map_\Cc(-,-)\otimes \id$ is the functor
classifying the functor
\[
\Cc^{\omega, \op} \times \Cc \times \Dc \xrightarrow{\map_\Cc(-,-) \otimes -} \Dc
\]
cocontinuous in the second and third variables.
\end{obsv}
We can therefore conclude the following.
\begin{prop}\label{left-proper-kernel-preserves-compact}
If $K \in \Cc \otimes \Dc$ lies in
the essential image of $\Cc^\pr \otimes \Dc^\omega \rightarrow \Cc \otimes \Dc$,
then the induced functor $\Cc^\vee \rightarrow \Dc$ preserves compact objects.
\end{prop}
\begin{proof}
Under the equivalences
\[
\Fun^L(\Cc^\vee, \Dc) \simeq \Fun^\ex((\Cc^\vee)^\omega, \Dc) \simeq \Fun^\ex(\Cc^{\omega,\op}, \Dc),
\]
a functor $\Cc^\vee \rightarrow \Dc$ preserves compact objects if and only if
the corresponding functor $\Cc^{\omega,\op} \rightarrow \Dc$ factors through
$\Dc^\omega \subseteq \Dc$.
By assumption on $K$,
we have a commutative diagram
% https://q.uiver.app/#q=WzAsNyxbMiwxLCJcXENjXntcXG9tZWdhLFxcb3B9XFx0aW1lc1xcQ2NcXG90aW1lc1xcRGNeXFxvbWVnYSJdLFs1LDEsIlxcRGMiXSxbMSwxLCJcXENjXntcXG9tZWdhLFxcb3B9Il0sWzIsMCwiXFxDY157XFxvbWVnYSxcXG9wfVxcdGltZXMgXFxDY15cXHFwclxcb3RpbWVzXFxEY15cXG9tZWdhIl0sWzAsMF0sWzUsMCwiXFxEY15cXG9tZWdhIl0sWzQsMCwiXFxTcF5cXG9tZWdhXFxvdGltZXNcXERjXlxcb21lZ2EiXSxbMCwxLCJcXG1hcF9cXENjKC0sLSlcXG90aW1lcyAtIl0sWzIsMCwiKFxcaWQsIEspIl0sWzMsMCwiIiwxLHsic3R5bGUiOnsidGFpbCI6eyJuYW1lIjoiaG9vayIsInNpZGUiOiJ0b3AifX19XSxbMiwzLCIiLDAseyJzdHlsZSI6eyJib2R5Ijp7Im5hbWUiOiJkYXNoZWQifX19XSxbNSwxLCIiLDEseyJzdHlsZSI6eyJ0YWlsIjp7Im5hbWUiOiJob29rIiwic2lkZSI6InRvcCJ9fX1dLFszLDYsIlxcbWFwX1xcQ2MoLSwtKVxcb3RpbWVzXFxpZCJdLFs2LDVdXQ==&macro_url=https%3A%2F%2Fgist.githubusercontent.com%2Fcasteryh%2F4e19ae618815e6d0f61c98f9121c3c1a%2Fraw%2F916a7aadc97e24b907d7bb38739ad684758f2464%2Fmacros.txt
% https://q.uiver.app/#q=WzAsNyxbMiwxLCJcXENjXntcXG9tZWdhLFxcb3B9XFx0aW1lc1xcQ2NcXG90aW1lc1xcRGMiXSxbNSwxLCJcXERjIl0sWzEsMSwiXFxDY157XFxvbWVnYSxcXG9wfSJdLFsyLDAsIlxcQ2Nee1xcb21lZ2EsXFxvcH1cXHRpbWVzIFxcQ2NeXFxxcHJcXG90aW1lc1xcRGNeXFxvbWVnYSJdLFswLDBdLFs1LDAsIlxcRGNeXFxvbWVnYSJdLFs0LDAsIlxcU3BeXFxvbWVnYVxcb3RpbWVzXFxEY15cXG9tZWdhIl0sWzAsMSwiXFxtYXBfXFxDYygtLC0pXFxvdGltZXMgLSJdLFsyLDAsIihcXGlkLCBLKSJdLFszLDAsIiIsMSx7InN0eWxlIjp7InRhaWwiOnsibmFtZSI6Imhvb2siLCJzaWRlIjoidG9wIn19fV0sWzIsMywiIiwwLHsic3R5bGUiOnsiYm9keSI6eyJuYW1lIjoiZGFzaGVkIn19fV0sWzUsMSwiIiwxLHsic3R5bGUiOnsidGFpbCI6eyJuYW1lIjoiaG9vayIsInNpZGUiOiJ0b3AifX19XSxbMyw2LCJcXG1hcF9cXENjKC0sLSlcXG90aW1lc1xcaWQiXSxbNiw1XV0=&macro_url=https%3A%2F%2Fgist.githubusercontent.com%2Fcasteryh%2F4e19ae618815e6d0f61c98f9121c3c1a%2Fraw%2F916a7aadc97e24b907d7bb38739ad684758f2464%2Fmacros.txt
\[\begin{tikzcd}[cramped]
	{} && {\Cc^{\omega,\op}\times \Cc^\pr\otimes\Dc^\omega} && {\Sp^\omega\otimes\Dc^\omega} & {\Dc^\omega} \\
	& {\Cc^{\omega,\op}} & {\Cc^{\omega,\op}\times\Cc\otimes\Dc} &&& \Dc
	\arrow["{\map_\Cc(-,-)\otimes\id}", from=1-3, to=1-5]
	\arrow[from=1-3, to=2-3]
	\arrow[from=1-5, to=1-6]
	\arrow[hook, from=1-6, to=2-6]
	\arrow[dashed, from=2-2, to=1-3]
	\arrow["{(\id, K)}", from=2-2, to=2-3]
	\arrow["{\map_\Cc(-,-)\otimes -}", from=2-3, to=2-6]
\end{tikzcd}.\]
\end{proof}

\begin{lem}\label{tensor-in-catperf-preserves-fully-faithfulness}
  If $\Cc \rightarrow \Dc$ is a fully faithful functor in $\Catperf$,
  then $\Cc \otimes \Ec \rightarrow \Dc \otimes \Ec$ is fully faithful
  for any $\Ec \in \Catperf$.
\end{lem}
\begin{proof}
  Since $\Ind(\Ec)$ is dualizable in $\Prlst$,
  it follows that $- \otimes \Ind(\Ec) \simeq \Fun^L(\Ind(\Ec^\op), -)$
  preserves fully-faithfulness.
  In particular,
  $\Ind(\Cc) \otimes \Ind(\Ec) \rightarrow \Ind(\Dc) \otimes \Ind(\Ec)$
  is fully faithful.
  Taking compact objects, we get
  $\Cc \otimes \Ec \rightarrow \Dc \otimes \Ec$ is fully faithful.
\end{proof}

\begin{lem}\label{tensor-with-proper-is-fully-faithful}
Let $\Cc_0 \subseteq \Cc$ be a small idempotent complete stable full subcategory,
and $\Dc$ a compactly generated presentable stable $\infty$-category.
Assume further that $\Dc$ is proper.
Then the functor
\[
\Cc_0 \otimes \Dc^\omega \rightarrow \Cc \otimes \Dc
\]
is fully faithful.
\end{lem}
\begin{proof}
We can identify the left hand side with compact objects in
\[
\Ind(\Cc_0 \otimes \Dc^\omega) \simeq \Fun^{\ex}(\Dc^{\omega, \op}, \Ind(\Cc_0)).
\]
Similarly, the right hand side with
\[
\Fun^{\ex}(\Dc^{\omega, \op}, \Cc).
\]
It suffices to show that
$\Cc_0 \otimes \Dc^\omega \rightarrow \Fun^{\ex}(\Dc^{\omega, \op}, \Cc)$
factors through
$\Fun^{\ex}(\Dc^{\omega, \op}, \Cc_0)$.
This follows from the fact that $\Dc$ is proper, and thus
for any $c \in \Cc_0$ and $d \in \Dc^\omega$,
\[
c \otimes d \colon \Dc^{\omega, \op} \rightarrow \Ind(\Cc_0), \quad d' \mapsto \map_\Dc(d', d) \otimes c \in \Cc_0 \subseteq \Ind(\Cc_0).
\]

\end{proof}
\end{comment}
\section{Compactness and properness}
In this section, we discuss \emph{compact} and \emph{proper} objects
in presentable stable $\infty$-categories.
We will focus on how these objects behave under
the inclusion of reflective and co-reflective subcategories.
\begin{recollection}
Let $\mathcal{C}$ be a presentable $\infty$-category.
An object $c \in \mathcal{C}$ is \emph{compact} if the functor
$\Map_{\Cc}(c, -) \colon \mathcal{C} \to \Spc$ commutes with filtered colimits.
If $\Cc$ is stable, this is equivalent to that
$\map_{\Cc}(c, -) \colon \Cc \to \Sp$ commutes with filtered colimits.
\end{recollection}
\begin{defn}\label{defn:proper-object}
Let $\Cc$ be a presentable stable $\infty$-category.
We say $c\in\Cc$ is \emph{proper} if the functor
\[
\map_\Cc(-,c) \colon \Cc^{\omega,\op} \to \Sp
\]
factors through $\Sp^\omega \subseteq \Sp$.
Denote by $\Cc^{\text{pr}} \subseteq \Cc$ the full subcategory of proper objects.
\end{defn}
\begin{rem}
If $\Cc$ is compactly generated and every compact object in $\Cc$ is proper,
then $\Cc$ is a proper stable $\infty$-category in the sense of \SAG{Definition}{11.1.0.1}.
\end{rem}
\begin{prop}
Let $\Cc$ be a presentable stable $\oo$-category.
Then $\Cc^{\text{pr}}$ is an idempotent complete stable subcategory of $\Cc$.
\end{prop}
\begin{proof}
Note that $\Sp^\omega \subseteq \Sp$ is closed under finite (co)limits and retracts,
and for any $x \in \Cc^\omega$,
the functor $\map_\Cc(x,-) \colon \Cc \rightarrow \Sp$ preserves finite (co)limits and retracts.
It follows that $\Cc^\pr$ is closed under finite (co)limits and retracts.
\end{proof}
\begin{prop}
Let $\Cc$ be a compactly generated stable $\oo$-category.
The spectral Yoneda embedding
\[
	c  \mapsto \map_\Cc(-,c)
\]
restricts to an equivalence
\[
	\Cc^\pr \xrightarrow{\simeq} \Fun^\ex(\Cc^{\omega,\op}, \Sp^\omega).
\]
Here $\Fun^\ex(-,-)$ denotes the category of exact
functors between two stable $\oo$-categories.
\end{prop}
\begin{proof}
Since $\Cc$ is compactly generated, there are equivalences
\[
	\Cc \xrightarrow{\simeq} \Ind(\Cc) \simeq \Fun^{\lex}(\Cc^{\omega,\op},\Spc)
	\simeq \Fun^{\ex}(\Cc^{\omega,\op},\Sp).
\]
Here $\Fun^{\lex}(-,-)$ denotes the category of left exact\footnote{Recall that a functor is called \emph{left exact} if it preserves finite limits.} functors.
By definition, the composite functor is the spectral Yoneda embedding.
Restricting to proper objects gives the desired equivalence.
\end{proof}
\begin{cor}
If $\Cc$ is a compactly generated stable $\oo$-category,
then $\Cc^\pr$ is a small $\oo$-category.
\qed
\end{cor}
Compact and proper objects behave in a very controllable way, under
\emph{strongly cococontinuous localizations}, which we now introduce.
\begin{recollection}
A functor between presentable $\infty$-categories is called \emph{strongly cocontinuous},
if its right adjoint admits a further right adjoint.
Alternatively, it is an internal left adjoint in $\Pr^L$.
\end{recollection}
\begin{obsv}
Let $\Cc$ be a presentable $\oo$-category, and $\iota_* \colon \Dc \hookrightarrow \Cc$
be a full subcategory.
Suppose that $\Dc$ is closed under small limits and colimits.
By the $\oo$-categorical reflection theorem \cite{infinity-categorical-reflection},
the inclusion $\iota_*$ participates in a triple adjunction:
% https://q.uiver.app/#q=WzAsMixbMCwwLCJcXERjIl0sWzEsMCwiXFxDYyJdLFswLDEsIlxcaW90YV8qIiwxLHsic3R5bGUiOnsidGFpbCI6eyJuYW1lIjoiaG9vayIsInNpZGUiOiJ0b3AifX19XSxbMSwwLCJcXGlvdGFeXFxmbGF0IiwwLHsib2Zmc2V0IjotM31dLFsxLDAsIlxcaW90YV4qIiwyLHsib2Zmc2V0IjozfV1d&macro_url=https%3A%2F%2Fgist.githubusercontent.com%2Fcasteryh%2F4e19ae618815e6d0f61c98f9121c3c1a%2Fraw%2F916a7aadc97e24b907d7bb38739ad684758f2464%2Fmacros.txt
\[\begin{tikzcd}[cramped]
	\Dc & \Cc
	\arrow["{\iota_*}"{description}, hook, from=1-1, to=1-2]
	\arrow["{\iota^\flat}", shift left=3, from=1-2, to=1-1]
	\arrow["{\iota^*}"', shift right=3, from=1-2, to=1-1]
\end{tikzcd}.\]
In particular, the localization functor $\iota^*$ is strongly cocontinuous.
\end{obsv}
\begin{notation}\label{bireflective-subcategory}
In the above situation, we say $\iota^* \colon \Cc \rightarrow \Dc$
is a \emph{strongly cocontinuous localization}, and $\iota_* \colon \Dc \hookrightarrow \Cc$
is a \emph{bi-reflective subcategory}. In this case, we always use
$\iota^* \dashv \iota_* \dashv \iota^\flat$ to refer to the adjoint triple.
\end{notation}
\begin{prop}
\label{localization-compactness}
Let $\Cc$ be a compactly generated presentable $\oo$-category.
Suppose $\iota_* \colon \Dc \hookrightarrow \Cc$ is a bi-reflective subcategory.
Then:
\begin{enumerate}[(1)]
   \item The inclusion $\iota_*$ detects compact objects: if $\iota_* d$ is compact in $\Cc$, then $d$ is compact in $\Dc$.
   \item The $\infty$-category $\Dc$ is compactly generated by $\iota^* \Cc^\omega$.
   \item The $\Dc^\omega$ is the smallest replete subcategory of $\Dc$
	   containing $\iota^* \Cc^\omega$ and closed under finite colimits
	   and retracts.
\end{enumerate}
\end{prop}
\begin{proof}
To prove point (1), note that there is a natural equivalence
\[
\Map_\Dc(d, -) \simeq \Map_\Cc(\iota_*d, \iota_* -)
\]
and that $\iota_*$ preserves all colimits, in particular filtered colimits.
Therefore if $\iota_*d$ is compact in $\Cc$, then $\Map_\Dc(d, -)$ commutes with filtered colimits, and hence $d$ is compact in $\Dc$.

To prove points (2) and (3), first note that $\iota^*$ preserves compact objects:
indeed, its right adjoint $\iota_*$ preserves all colimits, in particular filtered colimits \HTT{Proposition}{5.5.7.2}.

Now let $\Dc'$ be the full subcategory of $\Dc$ generated under colimits
by $\iota^*\Cc^\omega$.
Consider the full subcategory
\[
  (\iota^*)^{-1} \Dc' \defeq \{ c \in \Cc \mid \iota^* c \in \Dc' \} \subseteq \Cc.
\]
Since $\iota^*$ preserves colimits, it is immediate that $(\iota^*)^{-1}(\Dc')$ is closed under colimits.
However, by definition of $\Dc'$ we have
\[
\Cc^\omega \subseteq (\iota^*)^{-1}\iota^* \Cc^\omega \subseteq (\iota^*)^{-1}\Dc'.
\]
As $\Cc^\omega$ generates $\Cc$ colimits, we must have $(\iota^*)^{-1}\Dc' = \Cc$.
Therefore, we obtain $\Dc' = \Dc$: for any $d \in \Dc$, we have
$d \simeq \iota^* \iota_* d \in \iota^* (\Cc) = \iota^*(\iota^*)^{-1}\Dc' \subseteq \Dc'$.
Consequently, $\iota^*\Cc^\omega$ generates $\Dc$ under colimits.
\end{proof}
\begin{prop}\label{properness-localization}
Let $\Cc$ be a compactly generated stable $\oo$-category.
Let $\iota_* \colon \Dc \hookrightarrow \Cc$ be a bi-reflective subcategory.
Then $d \in \Dc$ is proper in $\Dc$
if and only if $\iota_*d \in \Cc$ is proper in $\Cc$.
\end{prop}
\begin{proof}
Fix $d$ so that $\iota_*d$ is proper in $\Cc$.
We need to show that $\map_\Dc(-, d)$ sends $\Dc^\omega$ to $\Sp^\omega$.
Consider the full subcategory
\[
\Dc' \defeq \{x \in \Dc \mid \map_\Dc(x, d) \in \Sp^\omega\} \subseteq \Dc.
\]
Clearly, $\Dc'$ is closed under finite (co)limits and retracts.
For any $y \in \Cc^\omega$, we have
\[
\map_\Dc(\iota^*y, d) \simeq \map_\Cc(y, \iota_* d) \in \Sp^\omega.
\]
Therefore, $\iota^*\Cc^\omega \subseteq \Dc'$.
By \cref{localization-compactness},
$\Dc^\omega$ is generated by $\iota^*\Cc^\omega$,
so $\Dc^\omega \subseteq \Dc'$ as $\Dc'$ is closed under retracts and finite colimits.
Thus by definition of $\Dc'$, $d$ is proper in $\Dc$.

Conversely, assume $d$ is proper in $\Cc$.
Then for any $y \in \Cc^\omega$, we have
\[
\map_\Dc(\iota^*y, d) \simeq \map_\Cc(y, \iota_* d) \in \Sp^\omega
\]
and thus $\iota_* d$ is proper in $\Cc$
\end{proof}


\section{$\Vc$-properness}
Throughout this section, let $\Vc$ be a compactly generated stable $\oo$-category.
\begin{notation}
Let $\Cc$ be a dualizable stable $\oo$-category.
Denote by
\[
	e \colon \Cc \otimes \Vc \xrightarrow{\simeq} \Fun^L(\Cc^\vee, \Vc)
\]
the evaluation functor.
\end{notation}
\begin{defn}
Let $\Cc$ be a compactly generated stable $\oo$-category.
An object $F \in \Cc \otimes \Vc$ is \emph{$\Vc$-proper}, if the functor
\[
	e(F) \colon \Cc^\vee \rightarrow \Vc
\]
sends $(\Cc^\vee)^\omega \simeq \Cc^{\omega,\op}$ to $\Vc^\omega$.
\end{defn}
\begin{rem}
An object $x \in \Cc$ is proper in the sense of \cref{defn:proper-object},
precisely if $x \in \Cc \simeq \Cc \otimes \Sp$ is $\Sp$-proper.
\end{rem}
\begin{example}
Let $x \in \Cc$ be a proper object and $v \in \Vc$ a compact object.
Then $x \boxtimes v \in \Cc \otimes \Vc$ is $\Vc$-proper.
Here $- \boxtimes - \colon \Cc \times \Vc \rightarrow \Cc \otimes \Vc$ is
the universal bi-cocontinuous functor.
\end{example}
\begin{obsv}
Let $\Cc$ be a compactly generated stable $\oo$-category.
Suppose $\iota_* \colon \Dc \hookrightarrow \Cc$ is a bi-reflective subcategory (cf. \cref{bireflective-subcategory}):
  % https://q.uiver.app/#q=WzAsMixbMCwwLCJcXERjIl0sWzEsMCwiXFxDYyJdLFswLDEsIlxcaW90YV8qIiwxLHsic3R5bGUiOnsidGFpbCI6eyJuYW1lIjoiaG9vayIsInNpZGUiOiJ0b3AifX19XSxbMSwwLCJcXGlvdGFeXFxmbGF0IiwwLHsib2Zmc2V0IjotM31dLFsxLDAsIlxcaW90YV4qIiwyLHsib2Zmc2V0IjozfV1d&macro_url=https%3A%2F%2Fgist.githubusercontent.com%2Fcasteryh%2F4e19ae618815e6d0f61c98f9121c3c1a%2Fraw%2F916a7aadc97e24b907d7bb38739ad684758f2464%2Fmacros.txt
  \[\begin{tikzcd}[cramped]
    \Dc & \Cc
    \arrow["{\iota_*}"{description}, hook, from=1-1, to=1-2]
    \arrow["{\iota^\flat}", shift left=3, from=1-2, to=1-1]
    \arrow["{\iota^*}"', shift right=3, from=1-2, to=1-1]
  \end{tikzcd}.\]
Since $\iota^* \dashv \iota_*$ is an adjunction internal to $\Pr^L$,
tensoring with $\Vc$ gives rise to a bi-reflective subcategory
% https://q.uiver.app/#q=WzAsMixbMCwwLCJcXERjIFxcb3RpbWVzIFxcVmMiXSxbMiwwLCJcXENjIFxcb3RpbWVzIFxcVmMiXSxbMCwxLCJcXGlvdGFfKiBcXG90aW1lcyBcXFZjIiwxLHsic3R5bGUiOnsidGFpbCI6eyJuYW1lIjoiaG9vayIsInNpZGUiOiJ0b3AifX19XSxbMSwwLCJcXGlvdGFeXFxmbGF0X1xcVmMiLDAseyJvZmZzZXQiOi0zfV0sWzEsMCwiXFxpb3RhXiogXFxvdGltZXMgXFxWYyIsMix7Im9mZnNldCI6M31dXQ==&macro_url=https%3A%2F%2Fgist.githubusercontent.com%2Fcasteryh%2F4e19ae618815e6d0f61c98f9121c3c1a%2Fraw%2F916a7aadc97e24b907d7bb38739ad684758f2464%2Fmacros.txt
\[\begin{tikzcd}[cramped]
	{\Dc \otimes \Vc} & {\Cc \otimes \Vc}
	\arrow["{\iota_*^\Vc}"{description}, hook, from=1-1, to=1-2]
	\arrow["{\iota^\flat_\Vc}", shift left=3, from=1-2, to=1-1]
	\arrow["{\iota^*_\Vc}"', shift right=3, from=1-2, to=1-1].
\end{tikzcd}\]
Here $\iota^*_\Vc = \iota^* \otimes \Vc$, $\iota_*^\Vc = \iota_* \otimes \Vc$, and $\iota^\flat_\Vc$ is the right adjoint to $\iota_* \otimes \Vc$.
\end{obsv}
\begin{prop}[cf. \cref{properness-localization}]
\label{v-properness-localization}
In above situation, an object $F \in \Dc \otimes \Vc$ is $\Vc$-proper
if and only if $\iota_*^\Vc(F)$ is $\Vc$-proper.
\end{prop}
\begin{proof}
By \HA{Proposition}{4.8.1.17},
there is a commutative diagram
% https://q.uiver.app/#q=WzAsNSxbMSwwLCJcXFByciJdLFszLDAsIlxcUHJyIl0sWzEsMSwiXFxQcl57TCxcXG9wfSJdLFszLDEsIlxcUHJeUiJdLFswLDBdLFswLDEsIi0gXFxvdGltZXMgXFxWYyJdLFswLDIsIlxcc2ltZXEiLDJdLFsyLDMsIlxcRnVuXntSfSgtXlxcb3AsXFxWYykiXSxbMSwzLCIiLDAseyJsZXZlbCI6Miwic3R5bGUiOnsiaGVhZCI6eyJuYW1lIjoibm9uZSJ9fX1dXQ==&macro_url=https%3A%2F%2Fgist.githubusercontent.com%2Fcasteryh%2F4e19ae618815e6d0f61c98f9121c3c1a%2Fraw%2F916a7aadc97e24b907d7bb38739ad684758f2464%2Fmacros.txt
\[\begin{tikzcd}[cramped]
	{} & \Prr && \Prr \\
	& {\Pr^{L,\op}} && {\Pr^R}
	\arrow["{- \otimes \Vc}", from=1-2, to=1-4]
	\arrow["\simeq"', from=1-2, to=2-2]
	\arrow[equals, from=1-4, to=2-4]
	\arrow["{\Fun^{R}(-^\op,\Vc)}", from=2-2, to=2-4]
\end{tikzcd}.\]
Restricting the domain of the functors
to compactly generated stable $\oo$-categories and compact preserving left adjoints
(resp. their right adjoints),
we have
\[
	- \otimes \Vc \simeq \Fun^R(-^\op,\Vc) \simeq \Fun^\ex((-)^{\omega,\op},\Vc).
\]
Under this equivalence, the right adjoint $\iota_*^\Vc$ corresponds to
\[
	- \circ \iota^* \colon \Fun^{\ex}(\Dc^{\omega,\op},\Vc) \rightarrow \Fun^{\ex}(\Cc^{\omega,\op},\Vc).
\]
Suppose $F \colon \Dc^{\omega,\op} \rightarrow \Vc$ factors through $\Vc^\omega$.
It follows that $\iota_*^\Vc(F) = F \circ \iota^*$ also factors through $\Vc^\omega$.
On the other hand, suppose $\iota_*^\Vc(F)$ factors through $\Vc^\omega$,
and let $\mathcal{K} \subseteq \Dc^\omega$ be the full subcategory spanned
by objects $d \in \Dc^\omega$ such that $F(d) \in \Vc^\omega$.
By assumptions on $\iota_*^\Vc(F)$,  we have
$F(\iota^*c) = \iota_*^\Vc(F)(c) \in \Vc^\omega$,
and therefore $\iota^*\Cc^\omega \subseteq \mathcal{K}$.
Because $\mathcal{K}$ is closed under finite colimits and retracts,
\cref{localization-compactness} shows that $\mathcal{K} = \Dc^\omega$.
\end{proof}
\section{Proper objects in functor categories}
Let $P$ be a triangulation of a manifold $X$.
By the exodromy equivalence (cf. \cref{cons-p-fun-p}), we have
\[
\Cons_P(X) \simeq \Fun(P, \Sp).
\]
In this section, we study proper objects in $\Cons_P(X) \simeq \Fun(P,\Sp)$.
First we note that, if $P$ is a triangulation of $X$,
then $P$ as a poset is \emph{locally finite}.
\begin{defn}
A poset $P$ is \emph{locally finite} if for every $p \in P$, the poset $P_{p/}$ is finite.
\end{defn}
\begin{prop}[{\cite[Lemma 4.4.10]{bh25}}]\label{compact-objects-in-fun-p}
Let $P$ be a locally finite poset.
Then $F$ is compact in $\Fun(P, \Sp)$ if and only if
$F$ is finitely supported and $F(p)$ is a finite spectrum for every $p \in P$.
\end{prop}
\begin{comment}
\begin{thm}\label{pointiwse-finite-is-proper}
Let $P$ be a locally finite poset.
Then $F$ is proper in $\Fun(P, \Sp)$ if and only if $F(p)$ is a finite spectrum
for every $p \in P$.
\end{thm}

\begin{proof}
Suppose $F(p)$ is finite for every $p \in P$.
We claim that $F$ is proper in $\Fun(P, \Sp)$.
By \cite[Proposition 2.2.3]{LurieRotation}, $\Fun(P,\Sp)$ is compactly generated
by the the stable coYoneda images $\yo(p) = \Sigma^\infty_+\Map_P(p, -)$ for $p \in P$.
It therefore suffices to show that for every $p \in P$,
the spectrum
\[
\map_{\Fun(P, \Sp)}(\yo(p), F)
\]
is finite.
Now note that $\yo(p)$ is left Kan extended along $P_{p/} \into P$,
and thus
\begin{align*}
\map_{\Fun(P, \Sp)}(\yo(p), F)
&\simeq \map_{\Fun(P_{p/}, \Sp)}(\yo(p), F|_{P_{p/}}) \\
&\simeq \map_{\Fun(P_{p/}, \Sp)}(\underline{\SS}, F|_{P_{p/}}) \\
&\simeq \lim_{k \in P_{p/}} F(k) \\
&\simeq F(p)
\end{align*}
is a finite, where the last equivalence holds because $p \in P_{p/}$ is
an initial object.
The other direction also follows immediately from the above calculation.

For the other direction, assume $F \in \Fun(P, \Sp)$ is proper.
We wish to show that $F(p)$ is finite for each $p \in P$.
Replacing $P$ by $P_{p/}$ if necessary, it suffices to prove the following:
\begin{center}
  ($\ast$) Let $P$ be a finite poset. If $F \in \Fun(P, \Sp)$ is proper, then $F(p)$ is finite for each $p \in P$.
\end{center}
We prove ($\ast$) by induction on the Krull dimension of $P$.
If $\dim(P) = 0$, then $P$ is discrete, and the result is clear.
Now assume $\dim(P) > 0$, and let $p \in P$.
If $p$ is not minimal, then $\dim(P_{p/}) < \dim(P)$, and $F|_{P_{/p}}$ is proper.
By induction, $F(p)$ is finite.
If $p$ is minimal, replacing $P$ by $P_{/p}$ if necessary, we can write
$P = Q^\triangleleft$ for a some poset $Q$ and $p = - \infty$.
Write $j: Q \hookrightarrow P$ for the inclusion and consider the fiber sequence
\[
F' \rightarrow F \rightarrow j_* j^* F.
\]
Note that $F'(q) \simeq 0$ for all $q \in Q$ and $F'(-\infty) \simeq F(-\infty)$,
since $j^*F \simeq j^*j_*j^*F$.
Applying
\[
\map_{\Fun(P, \Sp)}(\yo(-\infty), -) \simeq \lim_{P}(-)
\] to this sequence,
we get another fiber sequence
\[
F'(-\infty) \simeq \lim_P F' \rightarrow \lim_P F \rightarrow \lim_P j_* j^* F \simeq \lim_{Q} j^*F.
\]
Now that $j^*F$ is proper, by induction, $\lim_{Q} j^*F$ is a finite limit of
finite spectra, hence finite. By properness of $F$,
$\map_{\Fun(P, \Sp)}(\yo(-\infty), F) \simeq \lim_P F$ is finite.
It follows that $F'(-\infty) \simeq F(-\infty)$ is also finite, as desired.
\end{proof}
\end{comment}
\begin{thm}\label{v-proper-objects-in-fun-p}
Let $P$ be a locally finite poset. Then $F \in \Fun(P,\Sp) \otimes \Vc \simeq \Fun(P,\Vc)$ is $\Vc$-proper if and only if $F(p) \in \Vc^\omega$ for every $p \in P$.
\end{thm}
\begin{proof}
Write
\begin{align*}
	\yo \colon P^\op & \rightarrow \Fun(P, \Spc) \rightarrow \Fun(P, \Sp) \\
	p & \mapsto \Map_P(p, -) \mapsto \Sigma^\infty_+\Map_P(p, -)
\end{align*}
for the stable (co)Yoneda embedding.

By \cite[Proposition 2.2.3]{LurieRotation}, $\Fun(P,\Sp)$ is compactly generated
by the collection $\{\yo(p)\}_{p \in P}$.
Recall that the equivalence
\[
	\Fun(P, \Sp) \otimes \Vc \simeq \Fun(P, \Vc)
\]
can be obtained as the composite
\[
	\Fun(P, \Sp) \otimes \Vc \simeq \Fun^R(\Fun(P,\Sp)^\op, \Vc)
	\simeq \Fun^R(\Fun(P,\Spc)^\op, \Vc) \simeq \Fun(P, \Vc).
\]
It follows that under the above equivalence,
the continuos functor
\[
	\tilde{e}(F) \colon \Fun(P,\Sp)^\op \rightarrow \Vc
\]
classified by an object $F \in \Fun(P,\Vc)$
sends $\yo(p)$ to $F(p) \in \Vc$.

On the other hand, unwinding definitions
(cf. the proof of \SAG{Proposition}{D.7.2.3}),
the evaluation
\[
	e(F) \colon \Fun(P,\Sp)^\vee \rightarrow \Vc
\]
is the ind extension of
\[
	\tilde{e}(F)|_{\Fun(P,\Sp)^{\omega,\op}} \colon \Fun(P,\Sp)^{\omega,\op} \rightarrow \Vc.
\]
Since $\yo(p)$ is compact in $\Fun(P,\Sp)$,
it follows that
\[
	e(F)(\yo(p)) \simeq \tilde{e}(F)(\yo(p)) \simeq F(p).
\]
As $\{\yo(p)\}_{p \in P}$ generate $\Fun(P,\Sp)^\omega$
under small colimits and retracts, the result follows.
\end{proof}
\begin{cor}\label{properness-of-fun-p}
  If $P$ is a locally finite poset, then $\Fun(P, \Sp)$ is a proper stable $\infty$-category:
  every compact object is proper.
  The compact objects are precisely those functors that factor through $\Sp^\omega$
  and are supported on a finite subset of $P$.
  \qed
\end{cor}

\section{Proper objects in sheaf categories}
In this section, we prove the main theorem.
We do this by reducing to the category of $P$-constructible sheaves
for a triangulation $P$, studied in the previous section.
\begin{comment}
\begin{convention}
All manifolds are real analytic. All Lagrangians are subanalytic.
In particular, exodromy holds and all Lagrangians are in the conormal of some triangulation.
\end{convention}
\end{comment}

To this end, we first recall some facts on the geometry of stratifications
and microlocal sheaf theory.
\begin{notation}
Let $P$ be a $C^1$-stratification of a $C^1$-manifold $X$.
We write
\[
	N^*P \coloneqq \cup_{p \in P} N^*X_p \subseteq T^*X
\]
for the union of the conormals of the strata in $X$.
\end{notation}
\begin{recollection}[{{\cite[Corollary 8.3.22]{ks90}, \cite{czapla2012definable}}}]
Let $X$ be a real analytic manifold and $\Lambda \subseteq T^*X$ be a closed conic subanalytic singular isotropic.
There is a $C^\infty$ Whitney stratification $S$ of $X$ so that $\Lambda \subseteq N^*S$. Moreover, $S$ can be refined to a $C^p$ Whi:etney
triangulation $P$ for any $p \geq 1$.
\end{recollection}
\begin{recollection}[{{\cite[Proposition 8.4.1]{ks90}}}]
Let $P$ be a $C^1$ Whitney stratification of $X$.
Then
\[
	\Cons_P(X) \simeq \Sh_{N^*P}(X).
\]
\end{recollection}
\begin{recollection}[{{\cite[Proposition 3.4]{guillermou-viterbo}}}]
Let $\Lambda \subseteq T^*X$ be a closed conic isotropic.
The $\oo$-category $\Sh_\Lambda(X)$ is closed under limits and colimits in $\Sh(X)$.
\end{recollection}
The results mentioned so far can be combined and summarized as follows.
\begin{obsv}
Let $X$ be a real analytic manifold and $\Lambda \subseteq T^*X$ a closed conic
subanalytic isotropic.
Then there exists a $C^1$ Whitney triangulation $P$ of $X$, so that $\Lambda \subseteq N^*P$.
And the inclusion
\[
	\iota_{\Lambda, P, *} \colon \Sh_\Lambda(X) \hookrightarrow \Sh_{N^*P}(X) \simeq \Cons_P(X) \simeq \Fun(P, \Sp)
\]
is closed under both limits and colimits, \emph{i.e.}, a bi-reflective subcategory
(cf. \cref{bireflective-subcategory}).
In particular, it participates
in an adjoint triple
$\iota^*_{\Lambda,P} \dashv \iota_{\Lambda,P,*} \dashv \iota_{\Lambda,P}^\flat$.
\end{obsv}
For the rest of this section, fix $X$, $\Lambda$, and $P$ as above.
\begin{cor}\label{compacts-in-shv}
If $F \in \Sh_\Lambda(X)$ is compactly supported and has perfect stalks,
then $F$ is compact.

\end{cor}
\begin{proof}
By \cref{compact-objects-in-fun-p}, $F$ is compact in $\Cons_P(X)\simeq \Fun(P,\Sp)$ and
thus also compact in $\Sh_\Lambda(X)$ by \cref{localization-compactness}.
\end{proof}
\begin{thm}\label{proper-objects-in-shv}
A sheaf $F \in \Sh_\Lambda(X)$ is proper if and only if $F$ has perfect stalks.
\end{thm}
To prove this, we need a lemma about calculating stalks in $\Cons_P(X)$.
\begin{lem}\label{stalk-in-cons-p}
	Let $(X, P)$ be an exodromic stratified space, and $\Vc$ a dualizable stable $\oo$-category\footnote{Here the dualizability condition is assumed to ensure that we have\[
	\Cons_P(X;\Vc) \simeq \Cons_P(X;\Spc) \otimes \Vc \simeq \Fun(\Exit(X,P),\Vc).\] See {\cite[\S 4]{haine2024exodromy}} for detaield discussions on the exodromy equivalence with coefficients.}
Then the stalk functor at $x$
\[
(-)_x \colon \Cons_P(X;\Vc) \rightarrow \Vc
\]
is canonically equivalent to
the evaluation functor
\[
\ev_x \colon \Fun(\Exit(X, P), \Vc) \rightarrow \Vc.
\]
at $x \colon [0] \rightarrow \Exit(X, P)$.
\end{lem}
\begin{proof}
Taking stalks at $x$ is by definition the pullback along
$x: \ast \rightarrow X$, \emph{i.e.}
\[
F_x \simeq x^* F \in \Sh(\ast; \Vc) \simeq \Vc.
\]
By the functoriality of the exodromy equivalence,
this is equivalent to evaluation at $x \colon [0] \rightarrow \Exit(X, P)$.
\footnote{Here we distinguish the topological space $\ast$ consisting of a single point
from its homotopy type $[0]$.}
\end{proof}

\begin{proof}[Proof of \cref{proper-objects-in-shv}]
By \cref{properness-localization}, a sheaf $F$ is proper in $\Sh_\Lambda(X)$
if and only if it is proper in $\Cons_P(X) \simeq \Fun(P, \Sp)$.
By \cref{properness-of-fun-p}, this is equivalent to $F$ taking values in $\Sp^\omega$,
which in turn is equivalent to $F$ having perfect stalks by \cref{stalk-in-cons-p}.
\end{proof}
The exact same argument using \cref{v-properness-localization} can be used prove the analogous statement for $\Vc$-properness.
\begin{thm}\label{v-proper-object-in-shv}
Let $\Vc$ be a compactly generated stable $\oo$-category.
An object $F \in \Sh_\Lambda(X) \otimes \Vc$ is $\Vc$-proper if and only
if viewed as an object in $\Sh(X;\Vc)$, it has stalks valued in $\Vc^\omega$.
\end{thm}
\begin{proof}
By \cref{v-properness-localization}, $F \in \Sh_\Lambda(X) \otimes \Vc$ is
$\Vc$-proper if and only if
\[
	\iota_{\Lambda, P, *}^\Vc (F) \in \Cons_P(X) \otimes \Vc \simeq \Fun(P,\Sp) \otimes \Vc \simeq \Fun(P, \Vc)
\]
is $\Vc$-proper.
By \cref{v-proper-objects-in-fun-p},
this is equivalent to $F(p) \in \Vc^\omega$ for every $p \in P$,
which in turn is equivalent to $F$ has stalks valued
in $\Vc^\omega$ when viewed as an object in $\Sh(X;\Vc)$
by \cref{stalk-in-cons-p}.
\end{proof}
The above result, together with the following results of Kuo-Li \cite{kuo-li-duality},
leads to our main theorem.
\begin{recollection}[{{\cite[Theorem 1.1, Theorem 1.2, Corollary 1.7]{kuo-li-duality}}}]
Let $X$ and $Y$ be real analytic manifolds, $\Lambda \subseteq T^*X$ and
$\Sigma \subseteq T^*Y$ closed conic subanalytic istropics.
Then the Kunneth formula holds:
\[
	\Sh_\Lambda(X) \otimes \Sh_\Sigma(Y) \simeq \Sh_{\Sigma \times \Sigma}(X \times Y).
\]
The dual of $\Sh_\Lambda(X)$ is $\Sh_{-\Lambda}(X)$.
And the equivalence
\[
	\Sh_{-\Lambda\times \Sigma}(X\times Y) \simeq \Sh_\Lambda(X)^\vee \otimes \Sh_\Sigma(Y) \simeq \Fun^L(\Sh_\Lambda(X), \Sh_\Sigma(Y))
\]
is given by the assignment
\[
	K \mapsto (-) \ast K.
\]
\end{recollection}
\begin{thm}\label{main-thm}
Let $K \in \Sh_{-\Lambda\times \Sigma}(X\times Y)$ be a sheaf kernel.
The convolution functor
\[
- \ast K\colon \Sh_\Lambda(X) \rightarrow \Sh_\Sigma(Y)
\]
preserves compact objects if and only if for every $x \in X$,
the restriction $K|_{\{x\} \times Y} \in \Sh_{\Sigma}(Y)$ is a compact object.
\end{thm}
\begin{proof}
By definition, the convolution preserves compact objects precisely
if $K \in \Sh_{-\Lambda \times \Sigma}(X \times Y) \simeq \Sh_{-\Lambda}(X) \otimes \Sh_\Sigma(Y)$ is a $\Sh_\Sigma(Y)$-proper object.
By \cref{v-proper-object-in-shv},
this is equivalent to
$K$ having compact stalks when viewed as a sheaf on $X$ valued in $\Sh_\Sigma(Y)$.
In light of the naturality of the Kunneth formula, the stalk of said sheaf
at $x \in X$ is equivalent to
$K|_{\{x\} \times Y} \in \Sh_{\Sigma}(Y)$, whence the result.
\end{proof}
\begin{comment}
\begin{thm}
$K \in \Sh_{-\Lambda\times \Sigma}(X\times Y)$,
if $Y$ is compact and $K$
has perfect stalks,
then the convolution functor $- \ast K\colon \Sh_\Lambda(X) \rightarrow \Sh_\Sigma(Y)$
preserves compact objects.
\end{thm}

\begin{proof}
Let $(X, P)$ and $(Y, Q)$ be triangulations of $X$ and $Y$ respectively,
such that $-\Lambda \subseteq N^*P$ and $\Sigma \subseteq N^*Q$.
By \cref{localization-compactness}, it suffices to show that the
functor $- \ast K$ takes compact objects in $\Sh_\Lambda(X)$ to compact objects
in $\Cons_Q(Y)$.
By \cref{left-proper-kernel-preserves-compact}, it suffices to show that
if $K \in \Sh_{-\Lambda \times N^*Q}(X \times Y)$ has perfect stalks,
then $K$ lies in the essential image of
\[
\Sh_{-\Lambda}(X)^\pr \otimes \Cons_Q(Y)^\omega
\rightarrow \Sh_{-\Lambda}(X) \otimes \Cons_Q(Y)
\simeq \Sh_{-\Lambda \times N^*Q}(X \times Y).
\]
By \cref{tensor-with-proper-is-fully-faithful},
the above functor is fully faithful.
Since $Q$ is a finite poset, by \HTT{Proposition}{5.3.5.15}, we have
\begin{align*}
\Sh_{-\Lambda}(X)^\pr \otimes \Cons_Q(Y)^\omega
& \simeq (\Ind\Sh_{-\Lambda}(X)^\pr \otimes \Fun(Q, \Sp))^\omega \\
& \simeq (\Fun(Q, \Ind\Sh_{-\Lambda}(X)^\pr))^\omega \\
& \simeq \Fun(Q, \Sh_{-\Lambda}(X)^\pr),
\end{align*}
while
\[
\Sh_{-\Lambda}(X) \otimes \Cons_Q(Y) \simeq \Fun(Q, \Sh_{-\Lambda}(X)).
\]
It follows that $K$ lies in the image of
$\Sh_{-\Lambda}(X)^\pr \otimes \Cons_Q(Y)^\omega$
precisely when the corresponding
map $Q \rightarrow \Sh_{-\Lambda}(X)$
factors through $\Sh_{-\Lambda}(X)^\pr$,
which is equivalent to $K$ having perfect stalks
by \cref{stalk-in-cons-p}.
\end{proof}
\end{comment}
%The following geometric criteria of sheaf kernels preseving compact objects
%will appear in the forthcoming work of Ganatra-Kuo-Li-Wu \cite{Ganatra-Li-Kuo-Wu}.
\begin{cor}\label{perfect-stalk-and-compact-support}
Let $K \in \Sh_{-\Lambda \times \Sigma}(X \times Y)$ be a sheaf kernel.
If $K$ has perfect stalks and $\supp(K|_{\{x\} \times Y})$ is compact for every $x \in X$,
then convolution with $K$ preserves compact objects.
\end{cor}
\begin{proof}
Combine \cref{compacts-in-shv} and \cref{main-thm}.
\end{proof}

\printbibliography[keyword=alpha]
\printbibliography[notkeyword=alpha, heading=none]
\BiblatexSplitbibDefernumbersWarningOff

\end{document}
