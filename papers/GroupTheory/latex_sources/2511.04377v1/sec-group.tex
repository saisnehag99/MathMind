\section{Power maps on groups}\label{sec:group-power}
In this section, we consider the power map on the group $\GL_n(\C)$, defined by
$    x\mapsto x^M,
$
where $M\geq 2$ is an integer. 
These maps have been studied extensively for the last couple of decades; for example, Chatterjee and Steinberg independently proved that Let $G$ be a connected semisimple algebraic group over an algebraically closed field of characteristic $p$, and a positive integer $n$ relatively prime to $p$; the power map on $G$, a connected semisimple algebraic group over an algebraically closed field whose characteristic exponent is $p$ sending $x$ to $x^n$ is surjective if and only if $n$ is relatively prime to the ``bad'' primes for $G$ (possibly $2,3,5$) as well as to the primes dividing the order of the center of $G$, see \cite{Chatterjee2003,Steinberg2003}.
These maps are also interesting in their own right; they produce one of the examples of words $\omega$ such that a finite non-abelian simple group $G$ may have $\omega(G)^2\neq G$.
This is why the results of \cite{LST11} are optimal.
These maps have been studied for several groups, notably in a series of papers for several finite groups of Lie type, see \cite{KunduSingh22, PanjaSinghSympOrth22, PanjaSinghUnitary23}, for general linear groups over local principal ideal rings of length two in \cite{panjaRoySingh2025}. 
They have interesting applications as well, for example, see \cite{panja2023roots,panja2024fibers,panja2024imageRatio}.
We now state and prove the main result of this section.
\begin{theorem}\label{thm:FJ-word-map}
Let $M\geq 2$ be an integer, and $w:\GL_n(\C)\longrightarrow\GL_n(\C)$ be the word map $w(x)=x^M$. Then an element $X\in\GL_n(\C)$ is in $\F_w(\GL_n(\C))$ if and only if $\alpha_i>1$ for all $1\leq i\leq n$, where $\alpha_1,\cdots,\alpha_n$ are eigenvalues of $X$, counted with multiplicity.
\end{theorem}
\begin{proof}
Recall that a family $\mathscr F$ of holomorphic functions on an open set $U\subseteq\GL_n(\C)$ is normal if and only if every sequence of functions from $\mathscr F$ contains a subsequence that converges locally uniformly on $U$ (to a holomorphic function $f: U\longrightarrow\GL_n(\C)$). 
Note that $\overline{\GL_n(\C)}=\M_n(\C)$, which makes the behavior of the family of functions different, as compared to the case of functions $f_\alpha: U\longrightarrow \M_n(\C)$.
We endow $\GL_n(\C)$ with the induced metric from $\M_n(\C)$.
We consider three cases for our proof.

\underline{Case 1: $\rho(X)<1$} Then $X^{M^{nk}}\longrightarrow0$ as $k\longrightarrow\infty$, where $0\in\M_n(\C)$ is the zero matrix.
Hence, this sequence has no subsequence converging to a holomorphic map $f: U\longrightarrow\GL_n(\C)$ for any open set containing $X$.
Hence $X\in \J_w(\GL_n(C))$.
This also shows that if $X\in\GL_n(\C)$ has at least one eigenvalue with modulus strictly less than $1$, then $X\in \J_w(\GL_n(\C))$.
In what follows we will assume that $\mu(X)=\min\{|\lambda|:\lambda\in\sigma(X)\}\geq 1$.

\underline{Case 2: $\rho(X)=1$} We start with the case when $X$ is not semisimple.
Hence, in its Jordan form, it has a Jordan block of dimension $\geq 2$, say $J_{\lambda,t}$ for some $2\leq t\leq n$.
Then $J_{\lambda,t}\in\J_{x^M}(\M_t(\C))$, by \Cref{thm:FJ-polynomial-map}. 
Hence $X\in \J_w(\GL_n(\C))$.

We now discuss the case when $X$ is semisimple and $\rho(X)=1$.
Then, without loss of generality, we may assume $X=\operatorname{diag}(\alpha_1,\cdots,\alpha_n)$, following the same reasoning as in \Cref{lem:red-jord}.
For $m\geq 1$, consider the sequence 
\begin{align*}
    X_m=\operatorname{diag}(\alpha_1,\cdots,\alpha_n)-\dfrac{1}{m}I\longrightarrow X,
\end{align*}
as $m\longrightarrow\infty$. 
Given any open set $U$ containing $X$, there exists an $m_0$ such that for all $t\geq m_0$, one has $X_t\in U$. 
Then $\{w^k(X_{m_0+1})\}$ converges to $0\in\M_n(\C)\setminus\GL_n(\C)$, which shows that $X\in\J_p(\GL_n(\C))$.

Similar to before if $X\in\GL_n(\C)$ has an eigenvalue $1$, then $X\in\J_w(\GL_n(\C))$. 
We now discuss the final case.

\underline{Case 3: $\mu(X)>1$}
Then $\rho(X^{-1})<1$. 
Let $U$ be an open set on $\GL_n(\C)$ containing $X$.
Since $\GL_n(\C)$ is open in $\M_n(\C)$, the set $U$ is open in $\M_n(\C)$.
Thus all eigenvalues of $X^{-1}$ are in $\F_p(\mathbb C)$.
Since $\GL_n(\C)$ is a Lie group, $g\mapsto g^{-1}$ is a homeomorphism; hence on $U^{-1}$ the family of functions $\{w^m\}$ is normal, by\Cref{thm:FJ-polynomial-map} (converging uniformly to $0$), which implies that $\{w^m\}$ diverges uniformly to $\infty$ on $U$. Hence $X\in \F_p(\GL_n(\C))$.
\end{proof}
