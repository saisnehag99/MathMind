\subsection{Buildings in symplectizations}\label{subsec:buildings}
The moduli space of genus zero buildings have a natural stratification modeled on trees. We recall some basics and develop relevant notations for moduli spaces of genus zero buildings. The dual trees underlying our buildings without levels take the following form.

\begin{definition}\label{de:decorated-tree} A \emph{decorated tree (or forest)} is a directed tree (or forest) $T$ with internal and external (respectively finite and exterior) edges together with the data of 
	\begin{itemize}
		\item a class $\beta_v \in H_2(Y,\{\gamma_{e}\}_{e\in E_v})$ for each $v \in V$,
		\item $\gamma\cl E(T)\to \cP_1$ associating to an edge a Reeb orbit,
	\end{itemize}
    We call a vertex $v \in V(T)$ \emph{trivial} if $\beta_v = 0$ and $v$ has at most two adjacent edges. A decorated tree $T$ is \emph{stable} if it has no trivial vertices.
\end{definition}

\vspace*{-1.5cm}
\begin{figure}[h]
    \centering
    \incfig{0.95}{Untitled-2}
    \caption{Morphisms of a tree}
    \label{fig:tree_morph}
\end{figure}

\noindent Define the category $\scS$ (and $\scS^\bullet$) to have as objects decorated trees (respectively forests) as in Definition~\ref{de:decorated-tree} with morphisms given by contractions $p \cl T\to T'$ so that 
\begin{enumerate}
	\item for each non-contracted edge $e\in E(T)$ we have $\gamma'_{p(e)} = \gamma_{e}$,
	\item for each $v' \in V(T')$ we have $\beta_{v'} = \#_{p(v) = v'}\beta_v$.
\end{enumerate}
See Figure~\ref{fig:tree_morph} for an example.
The composition of two such morphisms is the obvious one. The category $\scS$ is the category $\scS_I$ in \cite{Par19}. We denote by $\Aut(T/T')$ the group of automorphisms of $T$ that leave $p$ invariant. We call a tree $T$ \emph{maximal} if any morphism $T\to T'$ is an isomorphism. Note for maps to a symplectization, any maximal decorated tree is a \emph{corolla}, at tree with a unique vertex and no interior edges.


\begin{notation*}\label{} Given a Riemann surface $C$ and a point $z \in C$, we denote by $$S_zC \coloneqq( T_zC\sm \{0\})/\bR_{> 0}$$ 
the boundary circle at infinity.
\end{notation*}



\begin{definition}\label{de:holomorphic-building} Let $T$ be a decorated tree. A \emph{pseudoholomorphic building of type $T$} consists of the following data
	\begin{enumerate}[label = (\roman*),leftmargin=20pt,ref= \roman*]
		\item for each $v \in V(T)$ a closed connected, possibly nodal, genus zero Riemann surface $(C_v,\fj_v)$ together with a set of pairwise distinct points $\{z_{v,e}\}_{e\in E_v}$;
		\item for each $v \in V(T)$ a smooth map $u_v \cl \dot{C}_v \coloneqq C_v\sm \{z_{v,e}\}\to \wh Y$, which  
		\begin{itemize}[leftmargin=15pt]
			\item is $C^0$-convergent to the positive trivial cylinder over $\gamma_{e}$ at $z_{v,e}$ for an incoming edge $e \in E_v^+$,
			\item is $C^0$-convergent to the negative trivial cylinder over $\gamma_{e}$ at $z_e$ for an outgoing edge $e \in E_v^-$
			\item represents $\beta_v$;
			\item satisfies $\delbar_{J,\fj_v}u_v = 0$. 
		\end{itemize}
		\item for each $e\in E^{\normalfont\text{ext}}_v$ an \emph{asymptotic marker} $\wt b_{e}\in S_{z_{v,e}}C_v$,
		\item for every interior edge $e$ from $v$ to $v'$, a \emph{matching isomorphism} $m_e \cl S_{z_{v,e}}C_v \to S_{z_{v',e}}C_{v'}$ intertwining $d{u_v}_Y$ and $d{u_{v'}}_Y$.
	\end{enumerate}
	We call such a building \emph{stable} if $T$ is stable.
    An \emph{isomorphism} of two such buildings consists of a collection $\{\iota_v \cl C_v \cong C'_v\}$ of biholomorphisms with $\iota_v(z_{v,e}) = z'_{v,e}$, preserving the matching isomorphisms so that $u_v = u'_v\g \iota_v$.
\end{definition}


\begin{definition}[Moduli spaces of buildings] Let $\wt\cM^J(T)$ be the moduli space of maps as in Definition~\ref{de:holomorphic-building} up to isomorphism. It admits a free $\bR^{V(T)}$-action given by translating the map on the corresponding sub-curve in the $\bR$-factor of $\wh Y$. The \emph{moduli space of $ J$-holomorphic buildings of type $T$} is
    $$\cM^J(T) \coloneqq \wt\cM^J(T)/\bR^{V(T)}.$$ 
    We define its \emph{SFT compactification} to be
	$$\Mbar^J(T) \coloneqq \djun{[T'\to T]}\cM^J(T')/\Aut(T'/T)$$
    equipped with the Gromov topology, \cite{Par19}.
\end{definition}

If $T$ is a corolla with incoming and outgoing exterior edges labeled by $\Gamma^+ = \{\gamma^+\}$ and $\Gamma^- = \{\gamma^-\}$, respectively, we write
    $$\Mbar^J(\Gamma^+,\Gamma^-;\beta) \coloneqq\Mbar^J(T).$$

We call $T$ \emph{effective} if $\Mbar^J(T)\neq \emst$.
By \cite[Theorem~10.1]{BEH03}, respectively \cite[Theorem 2.27]{Par19}, the moduli space $\Mbar^J(T)$ is compact for any $T$ and there exist only finitely many isomorphism classes of morphisms $T' \to T$ so that $T'$ is effective. Given $T$, define 
\begin{equation}\label{eq:tree-reeb-orbits}
    \cP(T) \coloneqq \union{T' \to T}{\{\gamma_e \mid e\in E(T')\}},
\end{equation}
to be the set of Reeb orbits which buildings in $\Mbar^J(T)$ can be asymptotic to, either at “internal" puncture or at an input or output puncture. Due to the compactness of $\Mbar^J(T)$, the set $\cP(T)$ is finite. In particular, for a fixed set of input and output Reeb orbits, the energy equations in \cite[\S 5.8]{BEH03} show that the Hofer energy is a constant that is linearly dependent on the action of the Reeb orbits.


\subsection{Leveled buildings}\label{subsec:leveled-buildings}
We now add level structures to our buildings by endowing the decorated trees defined in the last section with level structures. The derived orbifold of pseudo-holomorphic buildings with levels will be later used to construct the morphism spaces of the flow category in \textsection\ref{sec:sft-flow}. 

\begin{definition}\label{de:leveled-treee}
    A \emph{leveled tree} $(T,\ell)$ consists of a decorated tree $T$ as in Definition \ref{de:decorated-tree} equipped with a \emph{level function} $\ell : V(T) \to \bN$ satisfying 
    \begin{itemize}[leftmargin=20pt]
        \item if $\ell(v) = 1$, then $v$ has an incoming exterior edge,
        \item $\ell (w) \geq \ell(v)+1$ if $(v,w) \in E(T)$. 
     \end{itemize}
     We call $(T,\ell)$ \emph{stable} if each \emph{level} $\ell\inv(\{j\})$ is nonempty for $j \leq \max \ell$. The \textit{size} of a leveled tree is the maximal value of the level function. A \textit{morphism} of a leveled tree is obtained by simultaneous contraction of edges between adjacent levels.
\end{definition}


    Any leveled tree is from now on assumed to be stable.


\begin{lemma}\label{lem:pre-level-exists}
    Each decorated tree $T$ as in Definition~\ref{de:decorated-tree} admits a unique minimal level function, its \emph{pre-level function} $p\ell : V(T) \to \bN$ given by 
    \begin{itemize}[leftmargin=20pt]
        \item $p\ell(v) = 1$ if $v\in V^i$, the set of vertices with an incoming exterior edge,
        \item $p\ell (v) = \max \{ d(v,v_i) \mid v_i \in V^i \}$ 
     \end{itemize}
      where $d$ is the (edge-)distance function, which is well defined since $T$ is a tree.\qed
\end{lemma}



\begin{definition}
    A vertex $v$ in a decorated tree is \emph{trivial} if it has exactly two adjacent edges, both labeled by the same Reeb orbit, and if it carries the $0$ homology class.
\end{definition}

\begin{remark}
    Given a leveled tree $(T,\ell)$, we can construct a tree $T_\ell$ that contracts onto $T$ by adding a chain of $\ell(w)-\ell(v)-1$ trivial vertices between the vertices $v,w \in V(T)$ with $(v,w) \in E(T)$ and $\ell(w) > 1+ \ell(v)$. This recovers the usual notion of the underlying tree of SFT buildings.
\end{remark}

If the pre-level function is injective, it is the only level function on $T$. The simplest occurrence of non-injectivity is when the tree $T$ has three vertices (as in Figure \ref{fig:tree_morph}) with a pre-level containing two vertices. In this case, the fiber of the forgetful map 
\begin{equation}\label{eq:forget-levels}\Mbar_{\sft}^J (\Gamma^+,\Gamma^-;\beta)\to \Mbar^J(\Gamma^+,\Gamma^-;\beta)\end{equation}
is homeomorphic to a closed interval $[0,1]$. In particular, the interval $(0,1)$ records the relative height between the curves corresponding to each of the vertices, and the boundary $\{0,1\}$ of the fiber consists of the breaking of the pre-level $j$ into two levels and the appearance of trivial cylinder components.

\begin{figure}[H]
    \centering
    \incfig{1}{leveled_tree}
    \caption{Three different level functions for an unleveled tree. We do not draw the trivial vertices for the sake of clarity; one can recover the trivial vertices uniquely by the differences of levels between adjacent vertices.}
    \label{fig:leveled_tree}
\end{figure}

\begin{convention}
    We will denote the non-negative real line by $\bR_+ = [0,\infty).$
\end{convention}


\begin{definition}
    A leveled tree $(T,\ell)$ is \emph{maximally leveled} if $\#\ell\inv(j) \leq 1$ for each $j \in \bN$.
\end{definition}
    
\begin{proposition}\label{prop:maximally-leveled-tree}
 Given any decorated tree $T$, there exists a number, $N_T$, of maximally leveled trees $(T,\ell)$ over $T$, where $N_T$ is uniquely determined by the underlying tree.
\end{proposition}

The proof is by induction, for which we need another definition.

\begin{definition}
    A leveled tree $(T,\ell)$ is \emph{$k$-maximally leveled} if 
    \begin{itemize}
        \item $\#\ell\inv(j) \leq 1$ for $j \le k$;
        \item $\ell$ is injective on the set $p\ell\inv(\{1,\dots,k\})$.
    \end{itemize}
\end{definition}


\begin{lemma}\label{lem:increasing-maximality}
    Given a $k$-maximally leveled tree $(T,\ell)$, there exists $\#\ell\inv(k+1)!$-many leveled trees $(T,\ell')$ that are $(k+1)$-maximally leveled and contract onto $(T,\ell)$ and which are minimal with respect to this property.
\end{lemma}


\begin{proof} Set $N \coloneqq \#\ell\inv(k+1)$. Unraveling definitions, the conditions mean that we have to find $N$ level functions $\ell_1,\dots,\ell_N \cl V(T)\to \bN$ that satisfy
\begin{enumerate}[(a),leftmargin=25pt,ref=\alph*]
    \item\label{i:maximally-leveled} $\ell_i$ is $(k+1)$-maximal;
    \item\label{i:extends} $\ell_i$ agrees with $\ell$ on $\ell\inv(\{1,\dots,k\})$;
    \item\label{i:remains} for any $v,w \in V(T)$ that can be connected by a (directed) path, we have 
    \begin{equation}\label{eq:same-difference} \ell_i(v)-\ell_i(w) = \ell(v)-\ell(w) \end{equation}
    if $\ell(v)$, $\ell(w) \ge k+1$.
\end{enumerate}
    Condition~\eqref{i:extends} means that $\ell_i$ is determined on $\ell\inv(\{1,\dots,k\})$, while Condition~\eqref{i:remains} implies that the values of $\ell_i$ on $\ell\inv(\{m \geq k+2\})$ are determined by the values of $\ell_i$ on $\ell\inv(k+1)$, due to the minimality requirement. The $(k+1)$-maximality together with the requirement that $(T,\ell_i)$ be stable means that we have $N$ elements that have to distributed over $N$ levels. Thus, we have $N!$ options for extending $\ell$.
\end{proof}



\begin{proof}[Proof of Proposition~\ref{prop:maximally-leveled-tree}] The leveled tree $(T,p\ell)$ is tautologically $0$-maximal. Now we apply Lemma~\ref{lem:increasing-maximality} inductively until we obtain $\#V(T)$-corollas. The minimality of Lemma~\ref{lem:increasing-maximality} ensures that each $\#V(T)$-corolla can only be obtained through a unique path of $k$-corollas for $k \leq \# V(T)$. Thus, the number of such trees is determined inductively by the values $\#\ell\inv(k+1)!$ by Lemma~\ref{lem:increasing-maximality}.
\end{proof}


\begin{definition}\label{de:buildings-with-levels}
   Let $(T,\ell)$ be a leveled tree with underlying decorated tree $T$. A \emph{$J$-holomorphic building } $(u,C,b_*,m_*)$ \emph{of type $(T,\ell)$} is a $J$-holomorphic building of type $T_\ell$ as in Definition~\ref{de:holomorphic-building}. An isomorphism between two such buildings is defined as before. 
\end{definition}

\begin{remark}
    Note that for any such building $u$ and any trivial vertex $v$ of $T$, the map $u_v$ is forced to be a trivial cylinder.
\end{remark}

We write $\cM^J(T,\ell)$ for the space of leveled buildings of type $(T,\ell)$ up to reparametrization and translation, and we define the \emph{SFT moduli space of buildings of type at least $(T,\ell)$} to be 
   \begin{equation}
       \Mbar^J_{\sft}(T,\ell) \coloneqq \djun{(T',\ell')\to (T,\ell)}\cM^J(T',\ell')/\Aut(T'_{\ell'}/T_\ell)
   \end{equation}
   equipped with the Gromov topology as defined in \cite[\textsection7.3]{BEH03}.


\subsection{Buildings in exact symplectic cobordisms}
We now discuss the trees, which stratify the moduli space of buildings in exact symplectic cobordisms. 

\begin{definition}\label{de:exact-cobordism}
    An \emph{exact symplectic cobordism} from $(Y_+,\lambda_+)$ to $(Y_-,\lambda_-)$ is an exact symplectic manifold $(\wh X,\omega = d\lambda)$ together with embeddings
    \begin{gather*}
        \Theta^+\cl (-N,\infty)\times Y^+\to \wh X\\
         \Theta^-\cl (-\infty,N)\times Y^-\to \wh X
    \end{gather*} 
    so that $\wh X\sm \im(\Theta^+)\cup \im(\Theta^-)$ is compact and $(\Theta^\pm)^*\lambda = e^{s}\lambda^\pm$.
\end{definition}

An almost complex structure $J$ on a symplectic cobordism $(X,\omega)$ is \emph{compatible} if it is $\omega$-compatible and ${\Theta^\pm}^*J$ is translation invariant. In particular, this implies that ${\Theta^\pm}^*J$ is an adapted almost complex structure.

\begin{ex}\label{ex:interpolating-between-contact-structures}
    Suppose $(Y,\xi)$ is a contact manifold with two contact forms $\lambda^\pm$ and two adapted almost complex structures $J^\pm$. Since the space of contact forms is contractible, we can find a smooth path $\{\lambda_s\}_{s\in \bR}$ of contact structures so that $\lambda_s= \lambda^-$ for $s \leq -m$ and $\lambda_s = \lambda^+$ for $s \ge m$. We can then find a path $\{J_s\}_{s\in \bR}$ that agrees with $J^-$ on $\bR_{\leq -m}$ and $J^+$ on $\bR_{\ge m}$ and so that $J_s$ is $\lambda_s$-adapted for each $s$. By taking a $m\gg0$ and rescaling the homotopy $\{\lambda_s\}_{s\in \bR}$ we can ensure that $\partial_s\lambda_s$ is small, whence $d(e^s\lambda_s)$ is symplectic. Then, $J$ is a compatible almost complex structure on the symplectic cobordism $(\bR\times Y,\omega = d(e^s\lambda_s))$.
\end{ex}


\begin{notation*}
    Given an exact symplectic cobordism $(\wh X,\omega)$ from $(Y^+,\lambda^+)$ to $(Y^-,\lambda^-)$, we abbreviate
    $$X^{00} := \bR \times Y^+, \quad  X^{01}:=X, \quad X^{11}:= \bR \times Y^-.$$
    We set $\cP_{i} := \cP(X^{ii})$ and $\cP_{01} := \cP_0\sqcup \cP_1$.
\end{notation*}

\begin{definition}\label{de:decorated-cob-tree} A \textit{decorated cobordism tree (or forest)} is a tree (or forest) $T_c$ equipped with maps 
\begin{equation*}
    \ast \cl E(T_c)\to \{0,1 \}\qquad \qquad \ast_\pm\cl V(T_c) \to \{0,1\}
\end{equation*}
so that 
\begin{itemize}[leftmargin=20pt]
    \item $\ast(E^{\text{ext},+}) = 0$ and $\ast(E^{\text{ext},-}) = 1$,
    \item $\ast_+\leq \ast_-$,
    \item for any exterior edge $e\in E^{\text{ext},\pm}$ adjacent to the vertex $v$ we have $\ast(e) = \ast_\pm(v)$,
\end{itemize}
together with the data of
\begin{itemize}[leftmargin=20pt]
    \item a Reeb orbit $\gamma_e\in \cP_{\ast(e)}$ for each $e\in E(T_c)$,
    \item  a homology class $\beta_v\in H_2(X^{\ast_-(v)\ast_+(v)},\{\gamma_e\}_{e\in E_v})$ for each $v \in V(T_c)$.
\end{itemize}
We call $v$ a \emph{symplectisation vertex} if $\ast_+(v )= \ast_-(v)$. The cobordism tree is \emph{stable} if none of the symplectization vertices are trivial.
\end{definition}

Similar to the category $\scS$, there is a category of decorated cobordism trees $\scS^{c}$ whose objects are decorated cobordism trees and whose morphisms are contractions $T' \xrightarrow[\pi]{} T$ such that $\ast_+(\pi(v)) \leq \ast_+(v), \ast_-(\pi(v)) \geq \ast_-(v)$ and $\ast(\pi(e))= \ast(e)$ for any non-contracted edge $e$.
\vspace*{-1cm}
\begin{figure}[H]
    \centering
    \incfig{1.1}{cobtreemorph}
    \vspace*{-0.5cm}
    \caption{Morphisms of a cobordism tree}
    \label{fig:cob_tree_morph}
\end{figure}


\begin{definition}\label{de:leveled-cob-tree}
    A \emph{leveled cobordism tree (or forest)} $(T_c,\ell_c)$ is a decorated cobordism tree (or forest) with a level function $\ell_c$ such that $(T_c,\ell_c)$ is a leveled tree as in Definition \ref{de:leveled-treee} and there is an integer $\mathfrak{c}>0$ so that $\ell_c\inv(\mathfrak{c}) = \ast_+\inv(0) \cap \ast_-\inv(1)$ and $(\ast_+(v),\ast_-(v))=(0,0)$ if and only if $\ell(v) < \mathfrak{c}$, while $(\ast_+(v),\ast_-(v))=(1,1)$ if and only if $\ell(v) > \mathfrak{c}$.
\end{definition}


\begin{definition}
    A leveled cobordism tree is \textit{maximally levelled} if $\ell\inv(j)\leq 1$ for all $j\in \bN \sm \{ \mathfrak{c}\}.$
\end{definition}

The proof of Proposition~\ref{prop:maximally-leveled-tree} carries over to leveled cobordism trees.

\begin{proposition}\label{prop:maximally-leveled-cob-tree}
 Given any decorated cobordism tree $T_c$, there exists a number $N_{T_c}$ of maximally leveled trees $(T_c,\ell)$ over $T_c$, where $N_{T_c}$ is uniquely determined by the underlying tree.\qed
\end{proposition}



\begin{definition}\label{de:cob-holomorphic-building} Let $T_c$ be a decorated cobordism tree. A \emph{pseudoholomorphic building of type $T_c$} consists of the same data as in Definition \ref{de:holomorphic-building} but with the property that the target of the map $u_v$ is $X^{\ast_-(v)\ast_+(v)}$.
	We call such a building \emph{stable} if $T_c$ is stable.\par
    An \emph{isomorphism} of two such buildings consists of a collection $\{\iota_v \cl C_v \cong C'_v\}$ of biholomorphisms with $\iota_v(z_{v,e}) = z'_{v,e}$ so that $u_v = u'_v\g \iota_v$.
\end{definition}

\begin{definition}[Moduli space of cobordism buildings] Let $\wt{\cM}^J(T_c)$ be the moduli space of maps as in Definition \ref{de:cob-holomorphic-building} up to isomorphism. It admits a free $\bR^{V_s(T)}$-action, given by translations of maps on each symplectization vertex. The \textit{moduli space of $J$-holomorphic cobordism buildings of type $T_c$} is $$\cM^J(T_c):= \wt\cM^J(T_c)/ \bR^{V_s(T)}.$$
We define 
$$\ol\cM^J(T_c) \coloneqq \bigsqcup_{[T'_c \to T_c]} \cM^J(T'_c)/\Aut(T'_c/T_c)$$ equipped with the Gromov topology as in \cite{BEH03}.
\end{definition}
