\subsubsection{Base space with levels}\label{subsec:base-with-levels}
In this subsection, we construct a generalized blow-up of $\cBR$ that will serve as the base space for the global Kuranishi chart of $\Mbar_{\sft}^{\, J}(\Gamma^+,\Gamma^-;\beta)$. For this, we use the generalized blow-up of \cite{KM15} and the stratification of $\cBR$ by decorated trees.
\vspace*{-1pt}
\subsubsection*{Generalized blow-up} We give a quick recap of the generalized blow-up of a manifold $X$ with corners as defined in \cite{KM15}. We denote the set of closures of connected boundary (codimension 1) components by $\cM_1(X)$, assuming that no $F\in \cM_1(X)$ self-intersects. We can then define the set of codimension-$k$ faces $\cM_k(X)$ to consist of intersections $F_I = F_{i_1}\cap \dots\cap F_{i_k}$ of $k$ distinct elements of $\cM_1(X)$. We now associate the following combinatorial data to $X$. 
\begin{itemize}[leftmargin=17pt]
    \item  To each face $F \in \cM_k(X)$, we associate the freely generated monoid $$\sigma_F \coloneqq \bigoplus_{\substack{H\in \cM_1{X},\\ F \sub H}} \bN e_H$$
    \item  The \emph{monoidal complex} $\cP_X$ of $X$ consists of the collection of monoids $\sigma_F$ for every face $F$ of codimension 1 and higher, together with the canonical maps $i_{GF} : \sigma_G \to \sigma_F$ induced by inclusion of the faces $F\sub G$.
\end{itemize}

\begin{definition}[{\cite[Definition~2.2]{KM15}}]
    A \emph{refinement} $\cR_\sigma$ of a single monoid $\sigma$ is a collection $\cR_\sigma = \{\tau\mid \tau \subset\sigma\}$ of submonoids such that 
\begin{enumerate}[label=\roman*),leftmargin=22pt,ref=\roman*]
    \item\label{maximal-monoids-suffice} if $\tau\in \cR_\sigma$ and $\tau'\sub \tau$, then $\tau'\in \cR_\sigma$,
    \item for any $\tau_1,\tau_2\in \cR_\sigma$, the intersection $\tau_1\cap \tau_2$ is a face of both $\tau_1$ and $\tau_2$,
    \item $\text{span}_{\bR_+}(\sigma) = \union{\tau\in \cR_\sigma}{\text{span}_{\bR_+}(\tau)}$.
\end{enumerate}
\end{definition}
Note that \eqref{maximal-monoids-suffice} implies that a refinement is uniquely determined by the maximal monoids it contains. A classical example of a refinement of a monoid is given by subdivison. We refer to \cite[\textsection2]{KM15} for more details and examples.\par 
The more general notion of the \emph{refinement of monoidal complex} $\cR_Q$ amounts to a collection $\cR_Q = \{\cR_\sigma\mid \sigma\in Q\}$ of refinements of the monoids of $Q$ together with compatibilities between these refinements. We refer the reader to \cite[Definition 4.7]{KM15} for the precise definition.

The main construction in \cite{KM15} can be paraphrased as follows.

\begin{theorem}[{\cite[Theorem A]{KM15}}]\label{thm:corner-blowup}
    For any smooth refinement $\cR \to \cP_X$, there is a manifold $Y = [ X ; \cR] $ with corners with a blow-down map $b : Y \to X$ such that $\cP_Y  = \cR$ and the blow-down map induces the refinement $\cR \to \cP_X$. 
\end{theorem}

We refer the reader to the excellently written \cite{KM15} for the proof and just add some description of the exceptional divisors, which is missing from \cite{KM15}. This will be useful for understanding the difference between real-oriented blow-ups and generalized blow-ups as well as\textsection\ref{subsec:leveled-gluing}, which shows that the generalized blow-up yields the `correct' base space.\par 
Recall that for the real-oriented blow $\text{Bl}_D(X)$ of a smooth quasi-projective variety along a normal crossing divisor $D = \{D_i\}_i$ in $X$, we intuitively replace $D_i$ by the
\emph{spherical projectivization} 
\[\bP_{> 0}(N_{D_i/X}) \coloneqq (N_{D_i/X}\sm 0)/\bR_{>0}.\]
Given now a smooth manifold $Y$ with corners and two embedded codimension-$1$ boundary strata $Z_1$ and $Z_2$ intersecting in $Z_{12}$, the generalized blow-up of $Y$ along $Z_{12}$ replaces $Z_{12}$ by the \emph{positive part} 
\begin{equation}\label{eq:positive-part}
    \bP_+(N_{Z_{12}/Y}) = \set{(y,[v])\in \bP_{> 0}(N_{Z_{12}/X})\mid v  =v_1\oplus v_2 \text{ with } v_i\in N_{Z_i/Y}\text{ inward pointing}}.
\end{equation}
of the spherical projectivization, where we use that $N_{Z_{12}/Y}$ splits as the direct sum $N_{Z_1/Y}\oplus N_{Z_2/Y}$. In the picture below we show the simplest case.

\begin{figure}[h]
    \centering
    \incfig{0.6}{blo2woT}
    \caption{Corner blowup of $\bR_+^2$.}
    \label{fig:blowup1}
\end{figure}

\begin{remark}\label{rem:neighbourhood-of-exceptional-boundary}
    Given a Riemannian metric $g$ on $Y$, we can identify $\bP_+(N_{Z_i/Y})$ with a subset of the sphere normal bundle $SN_{Z_i/Y}$ of $Z_i$ and extend this isomorphism to an isomorphism 
    \begin{equation*}
        [0,\epsilon)\times \bP_+(N_{Z_{12}/Y})\cong U \sub \wt Y,
    \end{equation*}
    where $\wt Y$ is the blow-up. Then, the blow-down map $\beta$ becomes the map
    \begin{equation*}
        [0,\epsilon)\times \bP_+(N_{Z_{12}/Y})\to Y : (t,y,[v]) \mapsto \exp_y(t\wt v),
    \end{equation*}
    where $\wt v$ is the unique lift of $[v]$ to the sphere normal bundle.
\end{remark}

\noindent Suppose $D$ is a normal crossing divisor in $X$ with two irreducible smooth components $D_1,D_2$. Applying the above observation to the case of $Y = \text{Bl}_D(X)$ with $Z_i$ the preimage of $D_i$ under the blow-down map, we have that the normal bundle of (the interior of) $Z_i$ is canonically isomorphic to the hyperplane line bundle 
\begin{equation}
    L_i\,\coloneqq \, \cO_{\bP_{>0}(N_{D/X})}(1) \coloneqq \bP_{>0}(N_{D/X})\times_D N_{D/X}.
\end{equation}
On the other hand, the (interior of the) intersection $Z_{12}$ is canonically identified with 
\begin{equation}\label{eq:intersection-and-projectivization}
    Z_{12}\,\cong\,\lbr{\bP_{+}(N_{Z_1/X})\times \bP_{+}(N_{D_2/X})}|_{D_{12}}
\end{equation}
with normal bundle corresponding to 
$$N_{Z_{12}/Y} \;\cong\; L_{1}|_{Z_{12}}\,\oplus\, L_{2}|_{Z_{12}}$$
under the identification~\eqref{eq:intersection-and-projectivization}. The generalized blow-up of $Y$ along $Z_{12}$ now replaces $Z_{12}$ with $\bP_{+ }(L_{1}|_{Z_{12}}\oplus L_{2}|_{Z_{12}})$. Hence, a point $p$ in $\beta\inv(Z_{12})$ corresponds to a tuple 
\begin{equation}\label{eq:point-in-blow-up}
p = \lbr{y,[v_1],[v_2],[v'_1\oplus v'_2]},\end{equation}
where $y \in D_{ij}$ and $v_i,v'_i \in (N_{D_i/X})_y$ with $[v_i'] = [v_i]$. The brackets denote the equivalence class under the $\bR_{>0}$-action on the respective bundle. The key point of~\eqref{eq:point-in-blow-up} is that the ``added data'' of $[v'_1\oplus v'_2]$ yields a ratio obtained by choosing a Riemannian metric on $X$ and lifting $v_i$ and $v_1'\oplus v_2'$ to unit vectors with respect to that metric. Another consequence is that the normal bundle of the embedding $\beta\inv(Z_{12})\hkra \beta\inv(Z_1)$ is exactly the pullback of $L_{2}|_{Z_{12}}$.


\subsubsection{Blow-up of $\cBR$}\label{subsubsec:blowup_to_leveled_base} We now return to our situation at hand, the real-oriented blow-up $\cBR \coloneqq\cBR_{n^+,n^-}(d)$ of $\cB_{n^+,n^-}(d)$. The construction of the generalized blow-up of \cite{KM15} as written assumes that the codimension one boundary components are embedded. In our situation we do not have this property due to the fact that the contraction maps $T'\to T$ of trees in $\scS$ can be fixed by a nontrivial automorphism of $T'$, see e.g., \cite[Figure~8]{Par19}. We therefore split up the construction into two steps.
\begin{itemize}[leftmargin=15pt]
     \item Fix a neighborhood $U$ of the closure of $\cBR_2$, the union of all codimension 2 strata in $\cBR$, such that every boundary (closure of codimension $1$ points) component in $U$ is embedded. 
    \item Replace $U$ with a corner blow-up $[U;\cR]$ corresponding to a suitable refinement of the monoidal complex $\cR \to \cP_{U} $.
\end{itemize}


The first step, the choice of $U$ we can do immediately. Let us now define the refinement $\cR$, recalling that the strata of $\cBR$ are indexed by decorated trees.

\begin{definition}\label{de:leveled-monoid}
    Given a decorated tree $T$ and an enumeration $E(T) = \{e_1,\dots,e_n\}$ with associated monoid $\sigma_T = \bN\lspan{e_1,\dots,e_n}$ and a maximal level function $\ell$ on $T$, the associated monoid is 
    \begin{equation}\label{} \sigma_{T,\ell} = \bN\lspan{e'_1,\dots,e'_n},\end{equation}
    where 
    \begin{equation}
        e'_i \coloneqq e_i + \s{\substack{p\ell(e_j)\leq p\ell(e_i)\\e\ell(e_j) < e\ell(e_i)}}{e_j},
    \end{equation}
	with $e\ell((v,w)) = \ell(w)$ for any edge $e = (v,w)$.
\end{definition}


\begin{definition}\label{} Given a decorated tree $T$, we let $L_T$ be the set of maximally leveled level functions on $T$. The \emph{refinement} $\cR(\sigma_T)$ of $\sigma_T$ is the refinement generated by $\set{\sigma_{T,\ell}\mid \ell\in L_T}$.
\end{definition}

\begin{lemma}\label{} $\cR(\sigma_T)$ is a smooth refinement, and the refinements $\cR(\sigma_T)$ form a refinement $\cR$ of the monoidal complex $\cP_U$.
\end{lemma}

\begin{proof} 
It suffices to show that each $\sigma_{T,\ell}$ is a smooth monoid in the sense of \cite[]{KM15}. Since $\sigma_T$ is smooth, it suffices to show that $e'_1,\dots,e'_n$ are linearly independent in $\bR\lspan{e_1,\dots,e_n}$. This follows from the definition of a level function. The second claim is a direct consequence of the construction.
\end{proof}


\begin{definition}\label{de:sft-base-sapce} We define the \emph{leveled base space}  to be
	\begin{equation}\label{}\cBS = \cBS_{n^+,n^-}(d) \coloneqq \frac{[U;\cR]\sqcup (\cBR\sm\cBR_2)}{\sim}\end{equation} 
	where we identify the interior of $[U;\cR]$ with $U \sm\cBR_2$ via the blow-down map $[U;\cR]\to U$.
\end{definition}


\begin{lemma}
The space $\cBS_{n^+,n^-}(d)$ is a smooth oriented manifold with corners whose codimension-$k$ boundary strata correspond to $(k+1)$-leveled trees.
\end{lemma}

\begin{proof}
The proof will be based on \cite[ Proposition 3.2]{KM15} which states that the dimension of a monoid $\tau$ in a refinement is equal to the codimension of a face $F_\tau$ corresponding to the monoid in the blow-up. We know that in the complement of $\cBR_2$, the codimension-$1$ strata correspond to trees with exactly two vertices, and there are no strata of higher codimension. Thus the result follows easily in $\cBR\sm\cBR_2.$ Definition \ref{de:sft-base-sapce} shows that it remains to consider $[U;\cR]$.
It follows from Definition \ref{de:leveled-monoid} that every monoid of dimension $k$ in the smooth refinement corresponds to a leveled tree with $(k+1)$-many levels. The result then follows directly from \cite[Proposition~3.2]{KM15}.
\end{proof}

\begin{ex}We briefly describe the preimage of a (simple) corner stratum of $\cBR$ under the blow-down map. Let $T_k$ be the tree with a unique vertex, one input and $k$ outputs, and let $\cBR_{T_k}$ be the stratum corresponding to $T_k$. For the standard simplex $\Delta^n $, we define the \emph{maximally blown-up simplex} $\wt \Delta^n$ by iteratively blowing up the faces of the simplex, starting with the zero-dimensional faces and ending at a $(n-2)$-dimensional face. Thus, we obtain a sequence 
    \begin{equation}\label{eq:max-blown-up-simplex} \Delta^n \xleftarrow[\text{ blow up vertices}]{\pi_1}\Delta^n_1  \xleftarrow[\text{ blow up edges  }]{\pi_2} \Delta^n_2  \xleftarrow[\text{ blow up $2$-faces}]{\pi_3} \dots \xleftarrow[\text{ blow up } (n-2) \text{-faces}]{\pi_{n-2}} \wt \Delta^n \end{equation}
    of real-oriented and generalized blow-ups. We write $\wt\pi \cl \wt\Delta^n \to \Delta^n$
    for the composition of the maps in \eqref{eq:max-blown-up-simplex}.
Now, a small enough neighborhood of $\cBR_{T_{k}}$ is diffeomorphic to $\cBR_{T_{k}} \times [0,1)^k$. The corner blow-up replaces $\cBR_{T_{k}} \times [0,1)^k$ with $\cBR_{T_{k}} \times \wt{\Delta}^{k-1} \times [0,1).$ Explicitly, in the case of the tree $T_2$ we have 

\begin{figure}[h]
    \centering
    \incfig{1}{blo2}
    \vspace*{-1cm}
    \caption{Corner blowup of $\bR_+^2$ corresponding to the tree with 3 vertices.}
    \label{fig:blowup2}
\end{figure}

\end{ex}




 \subsubsection{Base space for disconnected domains}\label{subsec:base-for-disconnected} We now construct the base space for leveled buildings with disconnected domains, where each component has a unique incoming vertex. While the base space for disconnected Pardon buildings is simply given by the product of base space for (connected) Pardon buildings, the construction of $\cBS$ is more subtle because the level structure is defined for the whole forest, not each tree separately. The first lemma reduces this to the connected setting.

\begin{lemma}\label{lem:from-disconnected-to-connected}
    There exists a functor $\Phi$ from the category $\scS^\bullet$ of forests with $k$ components, each with one incoming edge, to the full subcategory $\scS_{k,\gamma_0}\sub \scS$ of decorated trees with one incoming edge labeled by a fixed Reeb orbit $\gamma_0$, so that the root vertex has energy zero and $k$ outgoing edges.
\end{lemma}

\begin{proof} The category $\scS^\bullet$ splits into a disjoint union of full subcategories, indexed by the Reeb orbits at the incoming exterior edges. Fix one such subcategory $\scS^\bullet_{\lc{\gamma}}$, labeled by Reeb orbits $\gamma_1,\dots,\gamma_k$. Let $T_0$ be the decorated corolla with zero energy, one incoming exterior edge, labeled by an arbitrary Reeb orbit $\gamma_0$, and $k$ outgoing exterior edges, labeled by $\gamma_1,\dots,\gamma_k$. Then, the functor 
\begin{gather*}
     \scS^\bullet_{\lc{\gamma}}\;\to\; \scS_{k,\gamma_0}\\
     T = T_ 1\sqcup \dots \sqcup T_k \;\mapsto\; T_0 \#_{i} T_i
\end{gather*}
is well-defined and an isomorphism by inspection.
\end{proof}

Intuitively, one can think of the isomorphism by introducing a \emph{ghost vertex} to which all incoming exterior edges of the components of the forest are connected. We write $e_j$ for the edge connecting $T_j$ to the ghost vertex in the construction of Lemma~\ref{lem:from-disconnected-to-connected}.
The functor $\Phi$ can be upgraded to a functor 
\begin{equation}\label{eq:upgraded-functor}
    P \cl \scS^\bullet_k\times (\{1,\dots,k\},\super)\to \scS
\end{equation}
by letting $P(T,I)$ be the tree obtained from $\Phi(T)$ by contracting all edges $e_j$ with $j \notin I$.\\

To make the blow-up construction concrete, fix Reeb orbits $\Gamma^+$ and $\Gamma^-$ and let $\Lambda\cl \Gamma^-\to \Gamma^+$ be a function. It induces a partition of $\Gamma^-$ into the sets $\Lambda_\gamma := \Lambda\inv(\{\gamma\})$. Fix for each $\gamma\in \Gamma^+$ a corolla $T_\gamma$ with a unique positive edge labeled by $\gamma$ and negative edges labeled by $\Lambda_\gamma$. Suppose the objects in the category $\scS_{/T_\gamma}$ come equipped with a function $d_\gamma \cl V(T'_\gamma)\to \bZ_{> 0}$ that is additive under contractions of edges. Set $d_\gamma \coloneqq d_\gamma(T_\gamma)$. Then, the product 
$$\cBR_\Lambda \coloneq \p{\gamma\in\Gamma^+}{\cBR_{\gamma,\Lambda_\gamma}(d_\gamma)}$$ 
is going to be the base space for the construction of the global Kuranishi chart for Pardon buildings with disconnected domains stratified by forests that contract onto the forest $\sqcup_\gamma T_\gamma$. Define the category $\wt\scS$ of leveled decorated forests similarly.

\begin{construction}\label{con:disconnected-leveled-base} Fix sequences $\Gamma^\pm$ of Reeb orbits and a partition $\Lambda\cl \Gamma^+\to \Gamma^-$. Then, we define the base space $\cBS_{\Lambda}$ as follows. Equip $[0,1)^{|\Gamma^+|}\times \cBR_\Lambda$ with the product stratification. Let $U \sub \cBR_\Lambda$ be a neighborhood of the strata of codimension at least $2$, so that each hyperplane in $U$ is embedded. The strata of $U$ are indexed by a decorated forest $T$ and a subset $I\sub \{1,\dots,k\}$, where $k = |\Gamma^+|$. Then, the monoid $\sigma_{T,I}$ associated to the stratum $S_{T,I}$ of $U$ is given by 
$$\sigma_{T,I} \coloneqq \bigoplus_{e\in E(T)} \bN e \;\oplus\;\bigoplus\limits_{i \in I}\,\bN e_i \;=\bigoplus_{e\in E(P(T,I))} \bN e.$$
The second equality holds by the definition of $P(T,I)$. We may now apply the algorithm of Proposition~\ref{prop:maximally-leveled-tree} along with Definition~\ref{de:leveled-monoid} for $\sigma_{T,I} = \sigma_{P(T,I)}$ to obtain a refinement $\wt\cR$ of the monoidal complex of $[0,1)^{|\Gamma^+|}\times U$. Then, we define $\wt U$ to be the pushout
\begin{equation*}\label{eq:disconnected-refinement}\begin{tikzcd}
			(U\sm  S_2(\cBR_{\lc{d}}))\times [0,1)^{|\Gamma^+|} \arrow[r,""] \arrow[d,""]&{[}U;\wt\cR{]} \arrow[d,""]\\ 
            {[0,1)^{|\Gamma^+|}} \times \cBR_\Lambda\arrow[r,""] & \wt U\end{tikzcd}\end{equation*}
           where the upper horizontal map is the inverse of the blow-down map, restricted to the complement of the blown-up locus. This admits a canonical smooth structure. Moreover, $\wt U$ is equipped with a canonical smooth map $\wt\beta\cl \wt U \to [0,1)^{|\Gamma^+|}$. We define the \emph{base space of disconnected buildings}
       \begin{equation}\label{eq:disconnected-base}
           \cBS_{\Lambda} \coloneqq \wt\beta\inv(\{0\}^{|\Gamma^+|})
       \end{equation}
       to be the preimage of the stratum of highest codimension in $[0,1)^{|\Gamma^+|}$.
\end{construction}

\begin{lemma}\label{lem:disconnected-base-well-defined}
    $\cBS_{\Lambda}$ is a smooth manifold with corners of dimension $\dim (\cBR_{\Lambda})+|\Gamma^+|-1$. 
\end{lemma}

\begin{proof}
By the universal property of the generalized blow-up, we can do the blow-up iteratively. Another way to see this is that the refinement we obtain can be realized as a sequence of iterated star subdivisions; then it follows from \cite[Corollary 7.3]{KM15}. By Construction~\ref{con:disconnected-leveled-base}, the first step is to blow-up $[0,1)^{|\Gamma^+|}$ to obtain a space $Y$, while the second step is to blow-up $Y\times\cBR_\Lambda$ according to the refinement described above. Using the explicit description in \cite[\textsection3]{KM15}, we see that the preimage $Z\sub Y$ of $\{0\}^{|\Gamma^+|}\sub [0,1)^{|\Gamma^+|}$ under the blow-down map is a codimension-$1$ face of $Y$ and thus a manifold with corners. Then, $\cBS_{\lc{d}}$ is the preimage of $Z$ under the generalized blow-up, or equivalently, the blow-up of $Z\times \cBR_\Lambda$ according to the given refinements. Thus, it is a smooth manifold with corners by \cite[Theorem~A]{KM15} and has dimension $\dim (\cBR_\Lambda)+|\Gamma^+|-1$.
\end{proof}

