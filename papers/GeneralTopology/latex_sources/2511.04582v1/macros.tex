%-----------------------------------------------------------
% Packages
%-----------------------------------------------------------

\usepackage{amsmath,amssymb,amsthm,amsfonts,amscd,mathrsfs,mathtools}
\usepackage[english]{babel}
\usepackage{graphicx} % Required for inserting images
\usepackage[utf8]{inputenc}
\usepackage{tikz,tikzscale}
\usepackage{tikz-cd}
\usepackage{enumitem}
\usepackage{geometry}
\usepackage{amssymb}
\usepackage{floatrow}
\usepackage{hyperref}
\usepackage{multicol}
\usepackage{rotating}
\usepackage[sort=def,symbols,nomain,numberline]{glossaries}
\usepackage[autostyle]{csquotes} % For quotation marks
\newcommand{\listofsymbolsname}{List of Symbols} 
%\makenoidxglossaries
%\usepackage{spectralsequences}
%\usepackage[rgb]{xcolor}

%-----------------------------------------------------------
% Formatting
%-----------------------------------------------------------
% FONT
\usepackage{libertine}
\usepackage{libertinust1math}
\usepackage[mathscr]{euscript}
\linespread{1.08}

% COLOR
\definecolor{darkred}{RGB}{139,0,0}
\definecolor{darkblue}{RGB}{0,0,139}
\definecolor{darkgreen}{RGB}{0,100,0}

\hypersetup{
	linktoc=all,
	colorlinks   = true, 
	urlcolor     = darkred, 
	linkcolor    = darkred, 
	citecolor   = darkgreen}



%-----------------------------------------------------------
% Tikz Diagram 
%-----------------------------------------------------------
\usetikzlibrary{backgrounds}
\usetikzlibrary {arrows,arrows.meta,automata,positioning}
\usetikzlibrary{shapes,shapes.geometric,shapes.misc}
\usetikzlibrary{matrix,calc,decorations.pathreplacing,decorations.pathmorphing}
\usetikzlibrary{calc,babel} % <--- IMPORTANT!
% this style is applied by default to any tikzpicture included via \tikzfig
\tikzstyle{tikzfig}=[baseline=-0.25em,scale=0.5]

% standard layers used in .tikz files
\pgfdeclarelayer{edgelayer}
\pgfdeclarelayer{nodelayer}
\pgfsetlayers{background,edgelayer,nodelayer,main}

% style for blank nodes
\tikzstyle{none}=[inner sep=0mm]

% Edge styles
\tikzstyle{open}=[Parenthesis-Parenthesis,darkred]
\tikzstyle{left closed} = [Bracket-Parenthesis,darkred]
\tikzstyle{right closed} = [Parenthesis-Bracket,darkred]
\tikzstyle{arrow} = [{Hooks[right]}-{stealth},darkblue]


% A TikZ style for curved arrows of a fixed height, due to AndréC.
\tikzset{
  curve/.style={
    settings={#1},
    to path={
      (\tikztostart)
      .. controls ($(\tikztostart)!\pv{pos}!(\tikztotarget)!\pv{height}!270:(\tikztotarget)$)
      and ($(\tikztostart)!1-\pv{pos}!(\tikztotarget)!\pv{height}!270:(\tikztotarget)$)
      .. (\tikztotarget)\tikztonodes
    },
  },
  settings/.code={%
    \tikzset{quiver/.cd,#1}%
    \def\pv##1{\pgfkeysvalueof{/tikz/quiver/##1}}%
  },
  quiver/.cd,
  pos/.initial=0.35,
  height/.initial=0,
}


%-----------------------------------------------------------
% Environments
%-----------------------------------------------------------
\newtheorem{bigthm}{Theorem}
\newtheorem{bigcor}[bigthm]{Corollary}
\newtheorem{biglem}[bigthm]{Lemma}
\newtheorem{bigprop}[bigthm]{Proposition}
\newtheorem{thm}{Theorem}[section]
\newtheorem*{nthm}{Theorem}
\newtheorem{lemma}[thm]{Lemma}
\newtheorem{proposition}[thm]{Proposition}
\newtheorem{corollary}[thm]{Corollary}
\newtheorem{conjecture}[thm]{Conjecture}
\newtheorem{claim}{Claim}

\theoremstyle{definition}
\newtheorem{definition}[thm]{Definition}
\newtheorem{notation}[thm]{Notation}
\newtheorem*{nconvention}{Convention}
\newtheorem{convention}[thm]{Convention}
\theoremstyle{remark}
\newtheorem{example}[thm]{Example}
\newtheorem*{nexample}{Example}
\newtheorem{remark}[thm]{Remark}
\newtheorem*{nrem}{Remark}
\newtheorem{construction}[thm]{Construction}
\newtheorem{ass}[thm]{Assumption}

\renewcommand{\thebigthm}{\Alph{bigthm}}

%-----------------------------------------------------------
% Setminus 
%-----------------------------------------------------------
\newsavebox{\mybox}
\sbox{\mybox}{%
  \tikz{\draw[line width=0.5pt,line cap=round] (3pt,0) -- (0,6pt);}%
}


\newcommand{\sminus}{\mkern2mu\mathord{\usebox{\mybox}}\mkern2mu}

\newcommand{\wedg}{\mathop{\scalebox{1.2}[1.5]{$\vee$}}}

%-----------------------------------------------------------
% Shortcuts
%-----------------------------------------------------------

%Categories
\newcommand{\cat}[1]{\mathrm{#1}} % category
\newcommand{\topcat}[1]{\mathsf{#1}} % topological category
\newcommand{\icat}[1]{\mathscr{#1}} % Single infinity category
\newcommand{\twocat}[1]{\mathbf{#1}} % Single (infinity,2) category
\renewcommand{\top}[1]{\mathsf{#1}} % topological things (e.g.functors)

%Operations
\newcommand{\lsh}[1]{{#1}_{!}} %(-)_{!} lower shriek
\newcommand{\ust}[1]{{#1}^{\ast}} %(-)^{*} upper star
\newcommand{\lst}[1]{{#1}_{\ast}} %(-)_{*} lower star

\newcommand{\tensor}{\otimes}


% Topological Categories
\newcommand{\Top}{\mathsf{Top}}
\newcommand{\sSet}{\mathsf{sSet}}
\newcommand{\Kan}{\mathsf{Kan}}
\newcommand{\tC}{\mathsf{C}} % a topological category
\newcommand{\catTop}{\mathsf{Cat(Top)}} % category of topological categories


% (Infinity,1) Categories
\newcommand{\iC}{\mathscr{C}} % An infinity category
\newcommand{\iD}{\mathscr{D}}
\newcommand{\iE}{\mathscr{E}}
\newcommand{\iM}{\mathscr{M}}
\newcommand{\iS}{\mathscr{S}} % Infinity category of Spaces
\newcommand{\pS}{\mathscr{S}_{\ast}} % Infinity category of pointed spaces
\newcommand{\Psh}{\mathrm{Psh}} % Infinity category of Presheaves
\newcommand{\Fun}{\mathrm{Fun}} % Infinity category of Functors
\newcommand{\infCat}{\mathscr{C}\mathrm{at}_{\infty}} % Infinity category of small infnity categoryies
\newcommand{\infLCat}{\mathscr{C}\mathrm{AT}_{\infty}} % Infinity category of large infinity categories  
\newcommand{\inftwoCat}{\mathscr{C}\mathrm{at}_{(\infty,2)}} % Infinity category of infinity,2 categories
\newcommand{\Spct}{\mathscr{S}\mathrm{p}}

\newcommand{\Mfld}{\mathscr{M}\mathrm{fld}}
\newcommand{\Disc}{\mathscr{D}\mathrm{isc}}
\newcommand{\Discn}{{\mathscr{D}\mathrm{isc}}_{\leq n}}
\newcommand{\DiscnN}{{{\mathscr{D}\mathrm{isc}}_{\leq n}}_{/N}}
\newcommand{\PnS}{\mathscr{P}_{n}\mathscr{S}_\ast} % Goodwillie approximation for pointed spaces
\newcommand{\Bord}{\mathscr{B}\mathrm{ord}}
\newcommand{\ncBord}{\mathrm{nc}\mathscr{B}\mathrm{ord}}
\newcommand{\coSpan}{\mathrm{coSpan}}


\newcommand{\cO}{\mathscr{O}} % The poset of open subsets
\newcommand{\uPnS}{{\mathscr{P}_{n}\mathscr{S}_{\ast}}^{\partial_c /}}
\newcommand{\uS}{\mathscr{S}^{\partial_c /}}
\newcommand{\uMfld}{\mathscr{M}\mathrm{fld}_{\partial}}
\newcommand{\uMfldN}{{\mathscr{M}\mathrm{fld}_{\partial}}_{/N}}
\newcommand{\uDisc}{{\mathscr{D}\mathrm{isc}}_{\partial}}
\newcommand{\uDiscn}{{\mathscr{D}\mathrm{isc}}_{\partial,\leq n}}
\newcommand{\uDiscnN}{{{\mathscr{D}\mathrm{isc}}_{\partial,\leq n}}_{/N}}
\newcommand{\cP}{\mathscr{P}} % The poset of power set

% (Infinity,2) Categories
\newcommand{\twoinfCat}{\mathbf{Cat}_{\infty}} % (Infinity,2) category of Infinity categories

% Basic Math Symbols
\newcommand{\R}{\mathbb{R}}
\newcommand{\Z}{\mathbb{Z}}
\newcommand{\N}{\mathbb{N}}
\renewcommand{\S}{\mathbb{S}}

% Font operator 
\newcommand{\ul}[1]{\underline{#1}}
\newcommand{\ol}[1]{\overline{#1}}

% Math operator
\newcommand{\Gp}{\mathrm{Gp}}

\newcommand{\Hom}{\mathrm{Hom}}
\newcommand{\Aut}{\mathrm{Aut}}
\newcommand{\Map}{\mathrm{Map}}

\newcommand{\iso}{\cong}
\newcommand{\eq}{\simeq}
\newcommand{\set}[1]{\{#1\}}

\newcommand{\id}{\mathrm{id}}
\newcommand{\const}{\mathrm{const}}
\newcommand{\pr}{\mathrm{pr}}
\newcommand{\diag}{\mathrm{diag}}
\newcommand{\coh}{\mathrm{coh}}
\newcommand{\inc}{\mathrm{inc}}
\newcommand{\ob}{\mathrm{ob}}
\newcommand{\mor}{\mathrm{mor}}
\newcommand{\ev}{\mathrm{ev}}
\newcommand{\res}{\mathrm{res}}

\newcommand{\Imm}{\mathrm{Imm}}
\newcommand{\emb}{\mathrm{Emb}}
\newcommand{\bdemb}{\mathrm{Emb}_{\partial}}
\newcommand{\ce}{\mathrm{CE}}
\newcommand{\Diff}{\mathrm{Diff}}
\newcommand{\B}{\mathrm{B}}
\newcommand{\BAut}{\mathrm{BAut}}
\newcommand{\BDiff}{\mathrm{BDiff}}
\newcommand{\Th}{\mathrm{Th}}

\newcommand{\nSigma}{\Sigma_n^\infty}
\newcommand{\nOmega}{\Omega_n^\infty}
\newcommand{\nmnSigma}{\Sigma_{n,n-1}^{\infty}}
\newcommand{\inmn}{\iota_{n,n-1}}
\newcommand{\ynmn}{{y_{n}^{n-1}}}
\newcommand{\ovph}{\overline{\varphi}}
\newcommand{\oph}{\overline{\phi}}
\newcommand{\ops}{\overline{\psi}}
\newcommand{\icpt}{\iota_{\mathrm{cpt}}}
% Objects
\newcommand{\kcirc}{\vee^{k} S^1}

\DeclareMathOperator*{\colim}{colim}
\DeclareMathOperator*{\fib}{fib}
\DeclareMathOperator*{\cofib}{cofib}
\DeclareMathOperator*{\holim}{holim}
\DeclareMathOperator*{\tot}{tot}
\DeclareMathOperator*{\im}{im}
\DeclareMathOperator*{\coker}{coker}
\DeclareMathOperator*{\op}{op}
\DeclareMathOperator*{\lax}{lax}
% \geometry{left=1in, right=1in, top=1in, bottom=1in}
% \newcommand{\squarebox}{\ensuremath{\square}}
% \newlist{todolist}{enumerate}{1}
% \setlist[todolist,1]{label=\arabic*., left=0pt, labelsep=3mm, , font=\squarebox\;\bfseries}
% \newlist{todolistt}{enumerate}{1}
% \setlist[todolistt,1]{label=\arabic*., left=10pt, labelsep=1mm, , font=\squarebox\;\bfseries}
\newcommand{\todo}[1]{{\color{blue}#1}}
\newcommand{\done}{\hspace{-0.95cm}$\checkmark$\hspace{.659cm}}
\newcommand{\donet}{\hspace{-0.75cm}$\checkmark$\hspace{.5cm}}

