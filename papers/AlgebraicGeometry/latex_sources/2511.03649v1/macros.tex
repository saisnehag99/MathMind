% vim: redrawtime=10000 et sw=2

% Some more packages used for macro defintions
\usepackage{suffix}  % To define starred commands
\usepackage{ifthen}
\usepackage{xspace}

% Switch the style for 1-categorical dg constructions
\newif\ifdgcal
\dgcaltrue    % Use caligraphic uppercase and italic lowercase.
% \dgcalfalse     % Use roman.

%%%%%%%%%%%%%%%%%%%%%%%%
% Standardize notation %
%%%%%%%%%%%%%%%%%%%%%%%%
\newcommand\basecat{{\cat V}}       % The base category for our constructions.
\newcommand\refined{refined }
\let\simeq\cong
\newcommand\dg{DG\xspace}
\newcommand\Dg{DG\xspace}

%%%%%%%%%%%%%%%%%%%
% Useful notation %
%%%%%%%%%%%%%%%%%%%
\newcommand\isoto{\xrightarrow{\sim}}
\newcommand{\xtwoheadrightarrow}[2][]{%   Extensible two-headed arrow (https://tex.stackexchange.com/a/260563/83)
  \xrightarrow[#1]{#2}\mathrel{\mkern-14mu}\rightarrow
}
\mathchardef\mhyphen="2D    % hyphen in math environments (used in some macros below)


%%%%%%%%%%%%%%%%%%%
% Basic rings     %  
%%%%%%%%%%%%%%%%%%%
\newcommand\kk{{\mathbb{k}}}            % base field
\newcommand\NN{{\mathbb{N}}}
\newcommand\ZZ{{\mathbb{Z}}}
\newcommand\CC{{\mathbb{C}}}

%%%%%%%%%%%%%%%%%%%
% Category theory %
%%%%%%%%%%%%%%%%%%%
\newcommand\cat\mathcal             % 1-categories
\newcommand\bicat\mathbf            % 2-categories

\DeclareMathOperator{\obj}{Ob}
\newcommand\Hom{\operatorname{Hom}}
\newcommand\End{\operatorname{End}}

\newcommand\id{\operatorname{id}}
\let\iden\id
\let\Id\id

\newcommand\Tr{\operatorname{Tr}}                   % Serre trace

\newcommand\Cat{{\bicat{Cat}}}                      % bicategory of ordinary categories
\newcommand\catfVect{\mathrm{Vect^f_{\kk}}}         % abelian cat of fd vector spaces
\newcommand\catgrVect{\mathrm{gr\mhyphen Vect_{\kk}}} % cat of graded vector spaces

\DeclareMathOperator\colim{colim}
\DeclareMathOperator\Kar{Kar}                       % Karoubi completion
\newcommand\addsym{\sym}                            % symmetric power of additive category


%%%%%%%%%%%%%%%%%%%%%
% Algebra           %
%%%%%%%%%%%%%%%%%%%%%
\newcommand\Ind{\operatorname{Ind}}     % induction functor
\newcommand\Res{\operatorname{Res}}     % restriction functor
\newcommand\dual{\vee}                  % dual object
\DeclareMathOperator\Sym{Sym}           % Symmetric powers of a vector space
\newcommand\SymGrp[1]{S_{#1}}           % Symmetric group
\newcommand\sgn{\operatorname{sgn}}     % sign of an element of the symmetric group
\newcommand\triv{{\mathrm{triv}}}       % trivial representation
\newcommand\sign{{\mathrm{sign}}}       % sign representation
\newcommand\SL{\mathrm{SL}}             % special linear group
\newcommand\stimes{\vee}                % symmetric product times symbol
\newcommand{\normord}[1]{:\mathrel{#1}:}
\newcommand\hochhom{\mathrm{HH}}
\newcommand\hochcx{\mathrm{HC}}

\newcommand{\vac}{\vert 0 \rangle}      %vacuum vector
\newcommand{\fix}{{\mathrm{Fix}}}       % fixed point set
\newcommand{\cts}{{\mathrm{cts}}}    % Reduction to continuous chains map on Hochschild complex
\newcommand{\redcts}{{\mathrm{\underline{Rdc}}}}    % Reduction to continuous chains map on Hochschild complex

\newcommand{\chrc}{{\mathrm{char}}}    % Reduction to continuous chains map on Hochschild complex


%%%%%%%%%%%%%%%%%%%%%%
% DG category theory %
%%%%%%%%%%%%%%%%%%%%%%
\newcommand\DGCat{{\bicat{dgCat}}}                  % bicategory of dg categories
\newcommand\DGCatone{{\bicat{dgCat}^1}}             % bicategory of dg categories
\newcommand\DGCatdg{{\bicat{dgCat}^{dg}}}           % bicategory of dg categories
\newcommand\DGModCat{{\bicat{dgModCat}}}            % bicategory of dg module categories
\newcommand\DGBiMod{{\bicat{dgMor}}}                % bicategory of dg bimodules
% \DeclareMathOperator{\hmtpy}{{Ho}}                % unused?
\newcommand\HoDGCat{{\bicat{Ho}(\DGCat)}}           % homotopy cat of dg categories
\newcommand\HoDGCatone{{\bicat{Ho}(\DGCatone)}}     % homotopy cat of dg categories
% \DeclareMathOperator{\Morita}{{Mo}}               % unused?
\newcommand\MoDGCat{{\bicat{Mor}(\DGCat)}}          % dg cats up to Morita equivalence
\newcommand\MoDGCatdg{{\bicat{Mor}(\DGCat)^{dg}}}   % dg cats up to Morita equivalence
\newcommand\MoDGCatone{{\bicat{Mor}(\DGCatone)}}    % dg cats up to Morita equivalence
\def\EnhCat{{\bicat{EnhCat}}}                       % enhanced triangulated categories
\def\EnhCatone{{\bicat{EnhCat}^1}}                  
\def\EnhCatKC{{\bicat{EnhCat}_{kc}}}                % karoubi complete enhanced triangulated categories
\def\EnhCatKCdg{{\bicat{EnhCat}_{kc}^{dg}}}         % karoubi complete enhanced triangulated categories
\newcommand{\PDGCat}{\mathcal{P}dg\mathcal{C}at}    % category of dg categories with tuples of subcategories

% 1-categories constructed from a given dg category
\ifdgcal
  \DeclareMathOperator{\modd}{\mathcal{M}\mkern-1.5mu\mathit{o\mkern-1mud}}                    % dg module category
  \DeclareMathOperator{\hperf}{\mathcal{H}\mathit{perf}}                 % h-projective perfect modules (1-category version)
  \DeclareMathOperator{\perf}{{\mathcal{P}\mkern-1.5mu\mathit{erf}}}
  \DeclareMathOperator{\hproj}{\mathcal{P}}
  \DeclareMathOperator{\pretriag}{{\mathcal{P}\mkern-1.5mu\mathit{re\mhyphen}\mkern-2mu\mathcal{T}\mkern-3mu\mathit{r}}}
  \DeclareMathOperator{\DGFun}{{\mathcal{DGF}\mkern-1.5mu\mathit{un}}}
  \DeclareMathOperator{\acyc}{\mathcal{A}\mathit{c}}
\else
  \DeclareMathOperator{\hperf}{Hperf}                 % h-projective perfect modules (1-category version)
  \DeclareMathOperator{\modd}{Mod}                    % dg module category
  \DeclareMathOperator{\perf}{Perf}
  \DeclareMathOperator{\hproj}{P}
  \DeclareMathOperator{\pretriag}{Pre\mhyphen Tr}
  \DeclareMathOperator{\DGFun}{DGFun}
  \DeclareMathOperator{\acyc}{Ac}
\fi

\DeclareMathOperator{\bihperf}{\bicat{Hperf}}               % h-projective perfect modules (2-category version)
\DeclareMathOperator{\bipretriag}{\bicat{Pre\mhyphen Tr}}   % pretriangulated hull (2-category version)
\newcommand{\lfrp}{\mathrm{lfrp}}                   % left flat, right perfect

\newcommand\catdgfVect{\mathrm{dg\mhyphen Vect^f_{\kk}}} % dg cat of fd vector spaces

\newcommand\sym{{\cat{S}}}                              % symmetric power of a dg category

\newcommand\numGgp[1]{K_0^{\mathrm{num}}\ifthenelse{\equal{#1}{}}{}{(#1)}}      % numerical Grothendieck group

% more module categories
\newcommand\modbar{{\overline{\modd}}}              % dg module bar category
% NOTE: the kerning in the following macros is for lmodern. Other fonts may need adaptations.
\newcommand\rightmod[1]{{\modd\mkern-2mu\mhyphen #1}}   % right modules
\newcommand\leftmod[1]{{#1\mhyphen\mkern-2.5mu\modd}}   % left modules
\newcommand\bimod[2]{{#1\mhyphen\mkern-2.5mu\modd\mkern-2mu\mhyphen #2}}   % bimodules
\newcommand\rightmodbar[1]{{\modbar\mhyphen #1}}        % right modules (no \mkern because of \overline)
\newcommand\leftmodbar[1]{{#1\mhyphen\modbar}}          % left modules
\newcommand\bimodbar[2]{{#1\mhyphen\modbar\mhyphen #2}} % bimodules


%%%%%%%%%%%%%%%%%%%%%%%%%%%
% Heisenberg-type objetcs %
%%%%%%%%%%%%%%%%%%%%%%%%%%%

% Heisenberg algebras
\newcommand\halg[1]{H_{#1}}              % Heisenberg algebra
\newcommand\chalg[1]{\underline{H}_{#1}} % classical (not idempotent-modified) Heisenberg algebra
\newcommand\chalga[1]{\underline{H}^{A}_{#1}} % classical (not idempotent-modified) Heisenberg algebra, A definition
\newcommand\chalgpq[1]{\underline{H}^{PQ}_{#1}} % classical (not idempotent-modified) Heisenberg algebra, PQ definition
\newcommand\falg[1]{F_{#1}}              % Fock space
\newcommand\cfalg[1]{\underline{F}_{#1}} % classical (not idempotent-modified) Fock space
\newcommand\walg[1]{\widehat{H}_{#1}}              % centrally extended Heisenberg algebra
\newcommand\cwalg[1]{\widehat{\underline{H}}_{#1}} % classical (not idempotent-modified) centrally extended Heisenberg algebra
\newcommand\valg[1]{V_{#1}}              % vertex algebra
\newcommand\cvalg[1]{\underline{V}_{#1}} % classical (not idempotent-modified) Vertex algebra

% The Heisenberg category
\newcommand\hcat[1]{\bicat{H}_{#1}}                     % The Heisenberg 2-category
\newcommand\vcat[1]{\bicat{V}_{#1}}                     % The vertex 2-category
\newcommand\wcat[1]{\widehat{\bicat{H}}_{#1}}                     % The Heisenberg 2-category
\WithSuffix\newcommand\hcat*[1]{\bicat{H}'_{#1}}        % The non-completed Heisenberg 2-category
\newcommand\hcatadd[1]{\bicat{H}^{\mathrm{add}}_{#1}}                       % The additive Heisenberg 2-category
\WithSuffix\newcommand\hcatadd*[1]{\bicat{H}^{\mathrm{add}\prime}_{#1}}     % The non-completed additiveHeisenberg 2-category
\newcommand\monhcat[1]{\underline{\bicat{H}}_{#1}}                          % The monoidal Heisenberg category
\WithSuffix\newcommand\monhcat*[1]{\underline{\bicat{H}}'_{#1}}             % The non-completed monoidal Heisenberg category
\newcommand\PP{{\mathsf{P}}}            % 1-morphism P
\newcommand\QQ{{\mathsf{Q}}}            % 1-morphism Q
\newcommand\RR{{\mathsf{R}}}            % 1-morphism R
\newcommand\hAA{{\mathsf{A}}}           % Hocschild chain A
\newcommand\hunit{{\mathbb{1}}}         % The empty string 1-morphism
\newcommand\ho{N}                       % an object in \hcat
\WithSuffix\newcommand\ho*{N'}          % another object in \hcat

% Fock category
\newcommand\fcat[1]{\bicat{F}_{#1}}                   % The Fock space 2-category
\WithSuffix\newcommand\fcat*[1]{\bicat{F}'_{#1}}      % The non-completed Fock space 2-category
\newcommand\fcatadd[1]{\bicat{F}^{\mathrm{add}}_{#1}}                         % The additive Fock space 2-category
\WithSuffix\newcommand\fcatadd*[1]{\bicat{F}^{\mathrm{add}\prime}_{#1}}       % The non-completed additive Fock space 2-category
\newcommand\starmap[1]{\star_{#1}}                      % Image of the Serre-star 2-morphism

% String diagrams
\tikzset{dot/.style={circle, fill, inner sep=0, minimum size=4pt}}
\tikzset{serre/.style={star, fill, star points=10, star point ratio = 2, scale=0.4}}
\tikzset{up/.style={color=blue}}
%\tikzset{up/.style={}}
\tikzset{down/.style={color=red}}
%\tikzset{down/.style={dashed}}
\tikzset{cc/.style={color=green!50!black}}
%\tikzset{cc/.style={dotted}}
\tikzset{desc/.style={fill=white}}
\tikzset{sym/.style={rectangle, draw, minimum width=1cm}}
\tikzset{many/.style={very thick}}
\newcommand\dcross[5]{\phi^{#1,#2}_{#3,#4,#5}}      % down-up double crossing
\newcommand\sdcross[3]{\psi^{#1}_{#2,#3}}           % symmetric down-up double crossing

%%%%%%%%%%%%%%%%%%%%%%
% Algebraic geometry %
%%%%%%%%%%%%%%%%%%%%%%
\newcommand\Spec{\operatorname{Spec}}   % spectrum of a ring
\newcommand\as[1]{\mathbb{A}^{#1}}      % affine space
\newcommand\ps[1]{\mathbb{P}^{#1}}      % projective space     
\newcommand\sheaf\mathcal               % sheaves
\newcommand\sO{{\sheaf O}}              % structure sheaf
\newcommand\catDQCoh[1]{\mathrm{D}_{\mathrm{qc}}(#1)}               % (triangulated) unbounded derived cat of qcoh sheaves
\newcommand\catDb[2][]{\mathrm{D}^{\mathrm{b}}_{#1}(#2)}
\newcommand\catDbCoh[1]{\mathrm{D}^{\mathrm{b}}_{\mathrm{coh}}(#1)} % (triangulated) derived cat of coh sheaves
\newcommand\catDGCoh[1]{\cat{I}(#1)}                                % dg enhanced derived cat of coh sheaves
\newcommand\catPerf[1]{\cat{Perf}(#1)}                              % (dg category) of perfect complexes

%%%%%%%%%%%%%%%%%%%%
% Timothy's macros %
%%%%%%%%%%%%%%%%%%%%
% TODO: sort into above\below categories

\DeclareMathOperator{\codim}{codim}
\DeclareMathOperator{\krn}{Ker}
\DeclareMathOperator{\adj}{Adj}
\DeclareMathOperator{\cok}{Coker}
\DeclareMathOperator{\chr}{char}
\DeclareMathOperator{\img}{Im}
\DeclareMathOperator{\homm}{Hom}
\DeclareMathOperator{\tor}{Tor}
%\DeclareMathOperator{\shhomm}{{\it\mathcal{H}om\rm}}
\DeclareMathOperator{\eend}{End}
%\DeclareMathOperator{\seend}{\it \mathcal{E}nd}
\DeclareMathOperator{\autm}{Aut}
%\DeclareMathOperator{\vectspaces}{\bf Vect}
\DeclareMathOperator{\cone}{Cone}
\DeclareMathOperator{\opp}{{opp}}
%\DeclareMathOperator{\fg}{{\it fg}}
%\DeclareMathOperator{\qrep}{\it \mathcal{Q}r}
%\DeclareMathOperator{\bhproj}{\bar{\mathcal{P}}}
%\DeclareMathOperator{\intmod}{{\it \mathcal{I}nt}}
%\DeclareMathOperator{\semifree}{\mathcal{S}\mathcal{F}}
%\DeclareMathOperator{\sffg}{\mathcal{S}\mathcal{F}_{\fg}}
\DeclareMathOperator{\triag}{{Tr}}
\DeclareMathOperator{\tria}{{Tria}}
\DeclareMathOperator{\hilb}{{Hilb}}
\DeclareMathOperator{\lder}{\bf L}
\DeclareMathOperator{\rder}{\bf R}
\DeclareMathOperator{\ldertimes}{\overset{\lder}{\otimes}}
%\DeclareMathOperator{\rderhom}{\rder\homm}
%\DeclareMathOperator{\rdershom}{\rder\shhomm}
\DeclareMathOperator{\DGMod}{\it{\mathcal{D}\mathcal{G}\mathcal{M}od}}
\DeclareMathOperator{\composition}{cmps}
\DeclareMathOperator{\action}{act}
\DeclareMathOperator{\unit}{unit}
\DeclareMathOperator{\counit}{counit}
\DeclareMathOperator{\eval}{eval}
\DeclareMathOperator{\pseudof}{psdfnc}
\DeclareMathOperator{\bimodapx}{{\underline{Apx}}}
\DeclareMathOperator{\alghh}{{AHH}}
\DeclareMathOperator{\algcat}{{{\mathcal{A}}lg}}
\DeclareMathOperator{\catalg}{{{\mathcal{C}}at}}
\DeclareMathOperator{\tensorfn}{{\underline{\otimes\vphantom{p}}}}  % "tensor functor" functor. (The \vphantom forces the underline the be at the same hight as for \bimodapx.)

\def\barperf{{\it\mathcal{P}\overline{er}f}}
\def\barperfA{{\barperf(\A)}}
\def\barperfB{{\barperf(\B)}}
\def\F{{\mathcal{F}}}
\def\L{{\mathcal{L}}}
\def\M{{\mathcal{M}}}
\def\N{{\mathcal{N}}}
\def\R{{\mathcal{R}}}
\def\T{{\mathcal{T}}}
\def\aA{\leftidx{_{a}}{\A}}
\def\bA{\leftidx{_{b}}{\A}}
\def\Aa{{\A_a}}
\def\ha{{h_a}}
\def\ah{\leftidx{_{a}}{h}{}}
\def\Ea{E_a}
\def\aE{\leftidx{_{a}}{E}{}}
\def\Eb{E_b}
\def\bE{\leftidx{_{b}}{E}{}}
\def\Fa{F_a}
\def\aF{\leftidx{_{a}}{F}{}}
\def\Fb{F_b}
\def\bF{\leftidx{_{b}}{F}{}}
\def\aM{\leftidx{_{a}}{M}{}}
\def\bM{\leftidx{_{b}}{M}{}}
\def\aN{\leftidx{_{a}}{N}{}}
\def\cN{\leftidx{_{c}}{N}{}}
\def\aMb{\leftidx{_{a}}{M}{_{b}}}
\def\bMa{\leftidx{_{b}}{M}{_{a}}}
\def\bMc{\leftidx{_{b}}{M}{_{c}}}
\def\aNb{\leftidx{_{a}}{N}{_{b}}}
\def\bNc{\leftidx{_{b}}{N}{_{c}}}
\def\rfRFa{\leftidx{_{\RF}}{\RF}{_{\A}}}
\def\aRFrf{\leftidx{_{\A}}{\RF}{_{\RF}}}
\def\biAMA{\leftidx{_{\A}}{M}{_{\A}}}
\def\biAMC{\leftidx{_{\A}}{M}{_{\C}}}
\def\biCMA{\leftidx{_{\C}}{M}{_{\A}}}
\def\biCMC{\leftidx{_{\C}}{M}{_{\C}}}
\def\biALA{\leftidx{_{\A}}{L}{_{\A}}}
\def\biALC{\leftidx{_{\A}}{L}{_{\C}}}
\def\biCLA{\leftidx{_{\C}}{L}{_{\A}}}
\def\biCLC{\leftidx{_{\C}}{L}{_{\C}}}
\def\Na{{N_a}}
\def\partn{{Part}}
\def\ordpartn{{OrdPart}}



%%%%%%%%%%%%%%%%%%%%%%%%%%%%
% Shortcuts for categories %
%%%%%%%%%%%%%%%%%%%%%%%%%%%%

% categories
\def\A{{\cat{A}}}
\def\B{{\cat{B}}}
\def\C{{\cat{C}}}
\def\D{{\cat{D}}}
\def\I{{\cat{I}}}
\def\Aopp{{\cat{A}^{\opp}}}
\def\Bopp{{\cat{B}^{\opp}}}
\def\Copp{{\cat{C}^{\opp}}}

% 2-categories
\def\biA{{\bicat{A}}}
\def\biB{{\bicat{B}}}
\def\biC{{\bicat{C}}}
\def\biD{{\bicat{D}}}
\def\biI{{\bicat{I}}}

% perfect hulls
\def\hperfA{{\hperf(\A)}}
\def\hperfB{{\hperf(\B)}}
\def\hperfC{{\hperf(\C)}}
\def\hperfAopp{{\hperf(\Aopp)}}
\def\hperfBopp{{\hperf(\Bopp)}}

% right modules
\def\modk{{\rightmod{\kk}}}
\def\modA{{\rightmod \A}}
\def\modB{{\rightmod B}}
\def\modC{{\rightmod C}}
\def\modD{{\rightmod D}}
\def\modbarA{{\overline{\modd}\mhyphen \A}}
\def\modbarAopp{{\overline{\modd}\mhyphen \Aopp}}
\def\modbarB{{\overline{\modd}\mhyphen \B}}
\def\modbarC{{\overline{\modd}\mhyphen \C}}
\def\modbarD{{\overline{\modd}\mhyphen \D}}
\def\modbarBopp{{\overline{\modd}\mhyphen \Bopp}}

% left modules
\def\Amod{{\leftmod \A}}

% bimodules
\def\AmodA{{\bimod{\A}{\A}}}
\def\AmodB{{\bimod\A\B}}
\def\BmodB{{\bimod\B\B}}
\def\BmodA{{\bimod\B\A}}
\def\AmodC{{\bimod\A\C}}
\def\AmodD{{\bimod\A\D}}
\def\BmodC{{\bimod\B\C}}
\def\BmodD{{\bimod\B\D}}
\def\CmodA{{\bimod\C\A}}
\def\CmodB{{\bimod\C\B}}
\def\CmodC{{\bimod\C\C}}
\def\CmodD{{\bimod\C\D}}
\def\DmodA{{\bimod\D\A}}
\def\DmodB{{\bimod\D\B}}
\def\DmodC{{\bimod\D\C}}
\def\DmodD{{\bimod\D\D}}
\def\AbimA{{\A\mhyphen\A}}
\def\AbimB{{\A\mhyphen\B}}
\def\AbimC{{\A\mhyphen\C}}
\def\AbimD{{\A\mhyphen \A}}
\def\BbimA{{\B\mhyphen \A}}
\def\BbimB{{\B\mhyphen \B}}
\def\BbimC{{\B\mhyphen \C}}
\def\BbimD{{\B\mhyphen \C}}
\def\CbimA{{\C\mhyphen \A}}
\def\CbimB{{\C\mhyphen \B}}
\def\CbimC{{\C\mhyphen \C}}
\def\CbimD{{\C\mhyphen \C}}
\def\DbimA{{\D\mhyphen \A}}
\def\DbimB{{\D\mhyphen \B}}
\def\DbimC{{\D\mhyphen \C}}
\def\DbimD{{\D\mhyphen \C}}
\def\AmodbarA{{\bimodbar\A\A}}
\def\AmodbarB{{\bimodbar\A\B}}
\def\AmodbarC{{\bimodbar\A\C}}
\def\AmodbarD{{\bimodbar\A\D}}
\def\BmodbarA{{\bimodbar\B\A}}
\def\BmodbarB{{\bimodbar\B\B}}
\def\BmodbarC{{\bimodbar\B\C}}
\def\BmodbarD{{\bimodbar\B\D}}
\def\CmodbarA{{\bimodbar\C\A}}
\def\CmodbarB{{\bimodbar\C\B}}
\def\CmodbarC{{\bimodbar\C\C}}
\def\CmodbarD{{\bimodbar\C\D}}
\def\DmodbarA{{\bimodbar\D\A}}
\def\DmodbarB{{\bimodbar\D\B}}
\def\DmodbarC{{\bimodbar\D\C}}
\def\DmodbarD{{\bimodbar\D\D}}

% others
\def\Aperf{{\A\mhyphen\perf}}
\def\Bperf{{\B\mhyphen\perf}}
\def\hprojA{{\hproj(\A)}}
\def\hprojB{{\hproj(\B)}}
\def\perfA{{\perf(\A)}}
\def\perfB{{\perf(\B)}}
\def\barA{{\bar{\A}}}
\def\barhom{{\mathrm H\overline{\mathrm{om}}}}
\def\barend{{\overline{\eend}}}
\def\bartimes{\mathrel{\overline{\otimes}}}
\def\Ainfty{{A_{\infty}}}
\def\degzero{{\mathrm{deg.\,0}}}

% symmetric powers
\newcommand\symbc[1]{{\sym^{#1}\basecat}}           % symmetric powers of \basecat (dg setting)
\newcommand\addsymbc[1]{\addsym^{#1}\basecat}       % symmetric powers of \basecat (additive setting)
\def\symbcn{{\symbc{\ho}}}
\def\symbcnmone{{\symbc{\ho-1}}}
\def\symbcnmtwo{{\symbc{\ho-2}}}
\def\symbcnmk{{\symbc{\ho-k}}}
\def\symbcnpone{{\sym^{\ho+1}\basecat}}


% Hochschild stuff

\def\euler{{\mathrm{eu}}}
\def\strace{{\mathrm{strace}}}



