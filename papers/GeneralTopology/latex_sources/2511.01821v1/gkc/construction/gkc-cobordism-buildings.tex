\subsection{Kuranishi charts for buildings in symplectic cobordisms}\label{subsec:gkc-for-cobordism} 
In this subsection we construct global charts for moduli spaces of (leveled) buildings in symplectic cobordisms. Fix thus an exact symplectic cobordism $(\wh X,\omega = d\lambda)$ from $(Y^-,\lambda^-)$ to $(Y^+,\lambda^+)$ as well as open embeddings 
\begin{gather*}\label{eq:contact-ends} 
\Theta^+ \cl (N,\infty) \times Y^+\to \wh X\\ 
	\Theta^- \cl (-\infty,-N) \times Y^-\to \wh X
\end{gather*}
for some $N \gg 0$, so that $(\Theta^\pm)^*\lambda = e^s \lambda^\pm$ and $X \coloneqq \wh X\sm (\im(\Theta^+)\cup \im(\Theta^-))$ is compact.\par 
Fix an $\omega$-compatible almost complex structure $\wh J$ on $\wh X$ whose pullback under $\Theta^\pm$ is an adapted almost complex structure $J^\pm$ on $\wh Y^\pm$.

\subsubsection{Base space}\label{subsec:cobordism-base} The construction of the base space for buildings in symplectic cobordisms is similar.

\begin{definition}\label{de:type-for-exact-cobordism} Given a stable map $\varphi \cl C\to \bP^d$ of genus zero whose domain has a unique node $x$, we say that $x$ is of \emph{type $0$} if it is non-separating or if it separates $C$ into irreducible components $C_0$ and $C_1$ of degree $d_0$, respectively $d_1$ so that 
\begin{equation}\label{eq:degree-comparison-cobordism} 
\mathrm{d}_x:=(d_0-p^+\s{z^+_i \in C_0}{d_i^+}+p^-\s{z^-_j \in C_0}{d_j^-}) - (d_1-p^+\s{z^+_i \in C_1}{d_i^+}+p^-\s{z^-_j \in C_1}{d_j^-}) = 0.
\end{equation}
We say $x$ is of \emph{type $1$} with order $|\mathrm{d}_x|$ otherwise.\end{definition}

We let $\cB'_c$ be the real-oriented blow-up at the normal crossing divisor with irreducible components given by the divisors corresponding to curves with a unique node that has type $1$. Then, we define $\cBR_c$ to be the total space of the torus bundle over $\cB'_c$ obtained by adding asymptotic markers at the marked points.

Recall from Lemma \ref{lem:stratification-base-space} that there is a stratification $P\cl  \cBR \to \scS^o$ which assigns to a map the tree type of its domain. Here, we can upgrade this to a stratification that keeps track of the targets by using the convention discussed in Remark \ref{rem:distinguish-from-degree}. For a vertex $v\in V(T)$ let $D_{C_v}\sub C_v$ be the divisor of special points on $C_v$. Then, define $\ast^\pm(v)$ by setting

\begin{itemize}[leftmargin=20pt]
    \item  $\ast^\pm(v) = 0$ if \[ p^+ \mid \;
\underbracket[0.8pt]{\left(\deg(\varphi\!\mid_{C_v})-\deg(\omega_{C_v}(D_{C_v}))\right)}_{\neq 0}
\]
    % \underbrace{ (\deg(\varphi|_{C_v})-\deg(\omega_{C_v}(D_{C_v})))}_{\neq0}\]

    \item $\ast^\pm(v) = 1$ if \[ p^- \mid \underbracket[0.8pt]{\left(\deg(\varphi\!\mid_{C_v})-\deg(\omega_{C_v}(D_{C_v}))\right)}_{\neq 0}
\]

    \item if $(\deg(\varphi|_{C_v})-\deg(\omega_{C_v}(D_{C_v}))) = 0$, then \begin{equation}\ast^\pm(v)=
        \begin{cases}
            1 & \text{ if }p^-\mid |\mathrm{d}_x|\\
            0 & \text{ if }p^+\mid |\mathrm{d}_x|,
        \end{cases}
    \end{equation}
    \item if $p^\pm \not \mid  (\deg(\varphi|_{C_v})-\deg(\omega_{C_v}(D_{C_v})))$, then $\ast^+(v)=0$ and $\ast^-(v) = 1$.
\end{itemize}


\begin{remark}
    Notice that the discussion above does not lift the stratification to  $\scS^c$ since not all morphisms of $\scS$ induce morphisms in $\scS^c$. Moreover, at this stage there is not always a function $\ast$ on $E(T)$ for which $(T, \ast, \ast^\pm)$ is a cobordism tree.
    This issue can be resolved by a corner-blow-up construction as explained below.
\end{remark}

In order to obtain the leveled base space $\cBS_c$, we need a preliminary definition. We call $(T,\ast_\pm)$ a \emph{pre-cobordism tree (or forest)} if the functions $\ast_\pm\cl V(T)\to \{0, 1\}$ satisfy the conditions in Definition \ref{de:decorated-cob-tree}. Similarly, $(T,\ast_\pm,\ell)$ is a \emph{leveled pre-cobordism tree} if the level function $\ell$ satisfies the conditions in Definition \ref{de:leveled-cob-tree}.\par
Then, we define the \emph{leveled cobordism base space} $\cBS_c$ to be the space obtained by applying the generalized blow-up on $\cBR_c$ associated to the refinement arising from maximally leveled pre-cobordism trees. This construction is similar to the construction for base space for leveled buildings. We may assume that $\cBS_c$ carries a stratification by leveled cobordism trees by removing the locus corresponding to leveled pre-cobordism trees, which do not support a leveled cobordism tree as in Definition~\ref{de:leveled-cob-tree}.


\subsubsection{Framings}\label{subsec:cobordism-framings} In order to obtain framings of buildings in $\wh X$, we use a similar construction as in \textsection\ref{subsec:framings}. However, we vary it somewhat to ensure that (in most cases) we can already see from the base space which `part' of the cobordism an irreducible component is mapped to.
 
 
\begin{definition}\label{de:approximation-cobordism}
    Let  $\cF = \cF^+\sqcup \cF^-$ be a finite set of Reeb orbits of $\lambda^+$ and $\lambda^-$. We call a $1$-form $\wt\lambda$ on $\wh X$ an \emph{$\cF$-integral approximation} if 
\begin{enumerate}[label=(\roman*),leftmargin=25pt,ref=\roman*]
	\item $(\Theta^\pm)^*\wt\lambda = e^s\wt\lambda^\pm$, where $\wt\lambda^\pm$ is an $\cF^\pm$-integral approximation,
    \item $\forall \Gamma^\pm \sub \cF\,:\, \cA_\lambda(\Gamma^+)-\cA_\lambda(\Gamma^-) > 0 \,\rimp \, \cA_{\wt\lambda}(\Gamma^+)-\cA_{wt\lambda}(\Gamma^-) > 0$.
\end{enumerate}
\end{definition}

The existence of $\wt \lambda$ follows from the same argument as in Lemma \ref{lem:contact-approximation}. Fix two prime numbers $p^\pm$ so that 
\begin{equation}\label{eq:primes-for-base}
    p^- > \s{\gamma\in \scF^+}{\cA_{\wt\lambda}(\gamma)}\qquad \text{and}\qquad p^+ \gg p^-
\end{equation}
Given a smooth stable building $u = (u_v,C_v,z_{v,*}^\pm)$ with dual graph $T$, recall that the vertices of $T$ are decorated with a pair of symbols $*^\pm(v)\in \{+,-\}^2$\footnote{In Definition \ref{de:decorated-cob-tree} we had that $\ast^\pm$ took values in $\{ 0,1\}$. We abuse notation in this subsection by identifying $0\sim +, 1\sim -$ to reduce clutter in \eqref{eq:cob-bundle}.}
so that $u_v$ maps to $Y^{*^\pm(v)}$ if the two symbols agree and to $\wh X$ if they disagree. Define for $v \in V(T)$ the line bundle 
\begin{equation}\label{eq:cob-bundle} 
L_{u,v} \coloneqq \cO_{C_v}\lbr{p^{*^+(v)}\s{e \in E_v^+}{\int\gamma_e^*\wt\lambda}-p^{*^-(v)}\s{e \in E_v^-}{\int\gamma_e^*\wt\lambda}}. 
\end{equation}
As before, this yields a holomorphic line bundle 
\begin{equation}
    \fL_u \coloneqq \omega_C(D) \otimes L_u
\end{equation} 
on $C$, where $D$ is the divisor of marked points, which is unique up to holomorphic isomorphism. By the definition of the $\cF$-integral approximation and the choice of $p^\pm$, $\deg(\fL_u|_{C'}) > 0$ for each irreducible component $C'$ of $C$. Thus, we obtain a $\PGL_{d+1}(\bC)$-orbit in $\cB_{\Gamma^+,\Gamma^-}(d)$ as before, where $d = \deg(\fL_u)$.

\begin{remark}\label{rem:distinguish-from-degree}
    The numbers $p^\pm$ were chosen in this specific way so that one can see from the framing which irreducible components are mapped to the ends $\wh Y^\pm$ and which are mapped to $\wh X$ proper. Concretely, given a framing $\varphi \cl C\hkra \bP^d$ associated to the line bundle $\fL_u$, an irreducible component $C'\sub C$ is mapped to $\wh Y^\pm$ if and only if  
    \[ p^\pm \mid \underbracket{ \deg(\varphi|_{C'})-\deg(\omega_C|_{C'})}_{\neq0}\]
    \noindent while in the case where $\deg(\varphi|_{C'})-\deg(\omega_{C}|_{C'})=0$, the component $C'$ is mapped to $\wh Y^\pm$ if and only if $p^\pm \mid |\mathrm{d}_x|$ where $x$ is any type 1 node in $C'$.
\end{remark}

\subsubsection{Families and local models}\label{subsec:families-and-local-models} We can now define the analogue of the infinite-dimensional spaces $\cZ$ of \textsection\ref{subsec:domain-metrics}. Fix a decorated corolla $T$ in $\scS^c$ and let $\wt\lambda$ be a $\cP(T)$-integral approximation, where we can define $\cP(T)$ for a cobordism tree as in Equation~\eqref{eq:tree-reeb-orbits}. Let $\Gamma^\pm$ be the Reeb orbits labeling the positive and negative edges of $T$, and set $L =\cA_\lambda(\Gamma^+)$. Fix a constant $\kappa_L > 0$ so that any non-trivial $J$-holomorphic map between Reeb orbits of action at most $L$ has $\omega$-energy at least $2\kappa_L$.

\begin{definition}\label{de:family-of-cobordism-buildings}
We define $\cZ^c = \cZ^c_{\wt\lambda}(T)$ to consist of tuples $(\varphi,T',u)$ of the form 
\begin{enumerate}[label=\roman*),leftmargin=20pt,ref=\roman*]
    \item $\varphi\in \cBR$ with image $\wch\varphi\in \cB$;
    \item $T'\in \scS^c_{/T}$ is mapped to $T_\varphi$ by the forgetful functor $\scS^c\to \scS$ and $\wch\varphi\in \cB$ satisfies 
    \begin{equation*}
        \deg(\wch\varphi_v) = |D_v|-2 + p^{\star^+(v)}\s{\gamma\in E^+_v}{\int \gamma_e^*\wt\lambda} -  p^{\star^-(v)}\s{\gamma\in E^-_v}{\int \gamma_e^*\wt\lambda}
    \end{equation*}
    for each $v \in V(T_\varphi)$, where $D_v$ is the divisor of special points on $C_v$;
    \item $u = (u_v)_{v\in V(T')}$ is a collection of maps where 
    \begin{enumerate}[leftmargin=15pt]
        \item $u_v$ is an equivalence class of smooth maps $\dot{C}_v \to \wh Y^+$ up to translation if $*^\pm(v) = 0$,
        \item $u_v$ is an equivalence class of smooth maps $\dot{C}_v \to \wh Y^-$ up to translation if $*^\pm(v) = 1$,
        \item $u_v$ is a smooth map $\dot{C}_v \to \wh X$ whenever $v$ is a nontrivial vertex with $\ast^+(v) = 0$ and $*^-(v) =1$ and a trivial cylinder over the associated Reeb orbit if $v$ is trivial, 
    \end{enumerate}
    such that 
	\begin{itemize}[leftmargin=20pt]
    \setlength\itemsep{2.5pt}
	\item $u_v$ is $J$-holomorphic near the punctures of $\dot{C}_v$;
	\item if $x \in C_v$ is a positive/negative node of type $1$, then $u_v$ is positively/negatively asymptotic to a Reeb orbit $\gamma\in \cP^\pm$ near $x\in C$;
    \item $\int_{C_v}{u_v}^*\omega \geq 0$;
	\item if $C_v$ is unstable, then $\int_{C_v}u^*_v\omega \geq \kappa_L$.
	\end{itemize}
    \end{enumerate}
     We write $\cZ^c_{\delbar}$ for the locus of $J$-holomorphic elements of $\cZ^c$.
\end{definition}

\begin{remark}
    $V(T')$ is only bigger than $V(T_\varphi)$ if $\wh X$ is the trivial cobordism from $(Y,\lambda)$ to itself. In this case, the base does not capture the full stratification of $\cZ^c$ and the resulting thickening will be a rel--$C^1$ manifold with boundary (in the fibers). This will be important in \textsection\ref{subsec:stable-complex}. 
\end{remark}

\begin{lemma}\label{lem:stratification-cobordism-family}
    $\cZ^c$ carries a canonical topology, which is stratified by $\scS^c$.
\end{lemma}

\begin{proof}
 We equip $\cZ^c$ with the following topology. We use $\cC_{\ge \epsilon}$ as defined in \eqref{eq:curlyCnbd}. Given $(\varphi,T', u)\in \cZ^c$, define $\cN_\epsilon(\varphi,T',u)$ to be the subset of points $(\varphi',T'',u')$ such that
\begin{itemize}[leftmargin=20pt]
\setlength\itemsep{1pt}
	\item $d_\cB(\varphi,\varphi') < \epsilon$;
        \item 
        $T' \xrightarrow{\pi_{\textbf{g}}}T''$ for $\textbf{g}$ in an $\epsilon$-neighborhood of $0$ in $G^c_{T'/}$
	\item the (orbits of the) graphs satisfy 
    \begin{equation*}
    d_{\text{H}}\lbr{\bR\cdot\graph(\varphi_v,u_v)|_{\cC_{\ge\epsilon}},\bR\cdot\graph(\varphi'_{\pi_\textbf{g}(v)},u'_{\pi_\textbf{g}(v)})|_{\cC_{\ge\epsilon}}} < \epsilon
	\end{equation*} 
	in the Gromov-Hausdorff metric for every vertex $v$ such that $\pi_\textbf{g}(v)$ is a symplectization vertex, and we choose any representatives of the classes $[u_v]$ and $[u'_{\pi_\textbf{g}(v)}]$; 
    \item the graphs satisfy
    $$d_{\text{H}}\lbr{\graph(\varphi_v,\pi_g\circ u_v)|_{\cC_{\ge\epsilon}},\graph(\varphi'_{\pi_\textbf{g}(v)},u'_{\pi_\textbf{g}(v)})|_{\cC_{\ge\epsilon}}} < \epsilon$$
    for vertices $v$ such that $\ast(\pi_g(v)) = 01$,
	\item for $e \in E(T'')$ with associated Reeb orbit $\gamma_e$ and corresponding node $x_e \in \cC_\varphi$ we have $$d_Y(u'_Y(z),\gamma_e) < \epsilon$$ 
	for any $z \in \cC_{\varphi'}$ with $d_\cC(z,x_e) \leq \epsilon$.
\end{itemize}
Similar to $G_{\text{II}}$ in \cite[\S 2.5]{Par19}, the gluing parameter space 
\begin{equation}\label{eq:local-model-cobordisms}
G^c_{T/}:= 
\left\{
\bigl(\{g_{e}\}_{e}, \{g_{v}\}_{v}\bigr) \in (0,\infty]^{E^{\mathrm{int}}(T)} \times (0,\infty]^{E_{00}(T)}
\;\middle|\;
g_{v} = g_{e} + g_{v'} \;\; \text{for }e= (v,v') \text{ with } v \in E_{00}(T)
\right\}
\end{equation}
 models neighborhoods of cobordism buildings. In the equation above, we interpret $g_{v'}=0$ if $v'\not \in E_{00}(T)$. There is a natural stratification of $G^c_{T/} \to (\scS^c)_{T/}$ obtained by sending $(\{ g_e\}, \{ g_v\})$ to the map $\pi: T \to T'$
such that
\begin{itemize}
    \item the edge $e$ is contracted if $g_e< \infty$
    \item $\ast(v)=00$ is changed to $\ast(v)=01$ if $g_v<\infty$.
\end{itemize}
Conceptually, we view $G^c$ as the parameter space of gluing the \textit{target} of the cobordism buildings. Any element $\textbf{g}\in G^c_{T/}$ naturally induces a collection of maps $$\{ \pi_* : \wh{X}_v \to \wh{X}_{\pi(v)} \}_{v\in V(T)}$$ 
where \begin{itemize}
    \item maps of the type  $\wh Y^\pm \to \wh Y^\pm$ are allowed to be
any $\bR$-translation
\item maps of the type $\wh X  \to \wh X$ must be the identity
\item maps of the type $\wh Y^\pm \to \wh X$ are the pre-composition of the relevant boundary collar identifying the ends of $\wh X$ with $\wh Y^\pm$ with any $\bR-$translation of $\wh Y^\pm$.
\end{itemize}
We denote by $0$ the unique element in $G^c$ corresponding to performing no gluing. An $\epsilon$-neighborhood of $0$ is defined via the natural identification $[0,1) \cong (1,\infty]$ via $t \mapsto 1/t$.
 It follows from the construction of $\cZ^c$ that this topology is equipped with a natural stratification by $\scS^c$.
\end{proof}

\noindent The following properties are shown by the same arguments as Lemma~\ref{lem:palais-proper} and Lemma~\ref{lem:zero-locus-correct}.

\begin{lemma}
    The $G_\bC$-action on $\cZ^c$ is Palais proper.\qed
\end{lemma}

\begin{lemma}
The induced map $$\psi\cl \modulo{\cZ^c_{\delbar}}{\PGL_{d+1}(\bC)} \to \Mbar^{\wh X,\,J}(T)$$ is an isomorphism.\qed
\end{lemma}

\subsubsection{Construction}\label{subsec:construction-cobordism} Recall that we have fixed a corolla $T$ with degree $\beta$ and positive/negative edges labeled by $\Gamma^+$ and $\Gamma^-$ respectively. We will write $\Mbar^{\wh X,J}(\Gamma^+,\Gamma^-;\beta) = \Mbar^{\wh X,J}(T)$ from now on to make the step to leveled buildings in Corollary~\ref{cor:leveled-gkc-cobordism} notationally easier.

\begin{definition} A \emph{pre-perturbation datum} $\fD = (\wt\lambda,\conn,p^\pm)$ for $\Mbar^{\wh X,J}(\Gamma^+,\Gamma^-;\beta)$
consists of 
	\begin{itemize}[leftmargin=20pt]
		\item a $\cP(\Gamma^+,\Gamma^-)$-integral approximation $\wt\lambda$ of $\lambda$ as in Definition~\ref{de:approximation-in-symplectisation};
		\item a $J$-linear connection $\conn$ on $T\wh X$ so that $(\Theta^\pm)^*\conn$ is translation invariant;
		\item primes $p^\pm$ satisfying \eqref{eq:primes-for-base}.
        \end{itemize}
 \end{definition}       
Given this, we set 
\begin{equation}\label{eq:auxiliary-degree-cobordism}
    d' \coloneqq p^+\s{\gamma\in \Gamma^+}\cA_{\wt\lambda}(\gamma)-p^-\s{\gamma\in \Gamma^-}\cA_{\wt\lambda}(\gamma)
\end{equation}
and $d \coloneqq d' +|\Gamma^+|+|\Gamma^-|-2$. Let $\cBR_c$ be the smooth manifold with corners defined in \textsection\ref{subsec:cobordism-base} and set $G := \PU(d+1)$ and $\cG := \PGL_{d+1}(\bC)$. 
We let $\cZ^c =\cZ^c_{\wt\lambda}(T)\to \cBS_c$ be the family of buildings of Definition~\ref{de:family-of-cobordism-buildings} with $T$ being the corolla with positive/negative edges labeled by $\Gamma^\pm$.
\begin{definition}\label{de:auxiliary-datum-cobordism}  A \emph{perturbation datum} extending $\fD$ is a tuple $\alpha= (\fD,\cU,\lambda,E,\mu,\mu^\pm)$, consisting of
        \begin{itemize}[leftmargin=20pt]
        \item a good covering $\cU := \{(U_i,\sigma_{ij},D_{ij},\chi_i)\}_{i\in I}$ of $\cZ^c$ and a $\cG$-equivariant map 
        \[\zeta\cl \cB^{\text{st}}_{d'}(d)/S_{d'}\to \cG/G \]
        yielding a $\cG$-equivariant map $\zeta_\cU \cl \cZ\to \fs\fu(d+1)$ (by Lemma~\ref{lem:map-to-lie-algebra});
	\item a \emph{joint perturbation space} $(E,\mu,\mu^\pm)$ consisting of a finite-dimensional $G$-representation $E$ equipped with $G$-equivariant linear maps
    $$\mu\cl E\to C^\infty_c(\cC\inn\times\wh X,\Lambda^{*,0,1}_{\cC\inn/\cBR_d}\otimes_\bC T\wh X)$$
    and 
    $$\mu^\pm\cl E\to C^\infty_c(\cC\inn\times\wh Y^\pm,\Lambda^{*,0,1}_{\cC\inn/\cBR_d}\otimes_\bC T\wh Y^\pm)^\bR$$
    so that $(\Theta^\pm)^*\mu(e)$ agrees with $ \mu^\pm(e)$ restricted to the respective end in $\cC\inn\times \wh Y^\pm$. We require that the map 
    \begin{equation}
        E\to \coker(D_u) : e\mapsto [\mu(e)|_{\graph(\varphi,u)}]
    \end{equation}
    is surjective for any $(\varphi,u)\in \cZ^c_{\delbar}$ with $\zeta_\cU(\varphi,u) = 0$.
\end{itemize}
\end{definition}



\begin{construction}\label{con:gkc-cobordism} Given a perturbation data $\alpha$, we define $$\cK^c_\alpha := (\bT^{\Gamma^+\sqcup \Gamma^-}\times G,\cT^c/\cBR,\cE,\fs)$$ 
by letting $\cT\sub \cZ\times E$ be the space of tuples $(\varphi,\wt T,u,w)$ such that 
\begin{enumerate}[label=\alph*),leftmargin=20pt,ref=\alph*]
    \item for each nontrivial vertex $v \in V(\wt T)$ the associated map $u_v$ (respectively a representative thereof) satisfies 
    \begin{equation}\label{eq:cobordism-perturbed-cr}
        \delbar_{\wh J} \, u_v + \mu_k^{\ast(v)}(w)|_{\graph(\varphi_v,u_v)} = 0,
    \end{equation}
    on $\dot{C}_v$
    \item the linearized operator of~\eqref{eq:cobordism-perturbed-cr} is surjective (without variation of the framing $\varphi$).
\end{enumerate}
    The obstruction bundle $\cE\to \cT$ is the trivial bundle $$\cE =E\oplus \fp\fu(d+1),$$ while the obstruction section $\fs$ is given by a mollification of $\wh\fs(\varphi,u,w) = (w,\lambda_\cU(\varphi,u))$ as in Lemma~\ref{lem:better-obstruction-section}.
\end{construction}

\begin{theorem}\label{thm:cobordism-gkc} Given an exact symplectic cobordism $(\wh X,d\lambda)$ from $(Y^+,\lambda^+)$ to $(Y^-,\lambda^-)$, a compatible almost complex structure $J$ on $\wh X$, and a tree $T$ in $\scS^c$, the following holds
\begin{enumerate}[\normalfont 1),leftmargin=15pt,ref=\arabic*]
		\item\label{cobordism-gkc-unobstructed-aux} The moduli space $\Mbar^{\wh X\,J}(T)$ of buildings in $\wh X$ admits perturbation data. Construction~\ref{con:gkc-cobordism} associates to each perturbation datum $\alpha$ a rel--$C^1$ global Kuranishi chart $\cK^c_\alpha$ with corners for $\Mbar^{\wh X\,J}(T)$.
		\item\label{cobordism-gkc-orientation} If the Reeb orbits $\Gamma^+$ and $\Gamma^-$ labeling the exterior edges of $T$ consist of good Reeb orbits, then there exists a canonical isomorphism $$\fo_{\cK^c_\alpha}\cong \bigotimes\limits_{\gamma\in \Gamma^+}\fo_\gamma\otimes\bigotimes\limits_{\gamma\in \Gamma^-}\fo\dul_{\gamma}$$ of orientation lines.
	\end{enumerate}    
\end{theorem}

As in the case of buildings in symplectizations, this yields a chart for leveled buildings.

\begin{corollary}\label{cor:leveled-gkc-cobordism}
    The pullback Kuranishi chart 
    \begin{equation}
        \cK^{c,\bR} \coloneqq \cBS_c\times_{\cBR_c}\cK^c_\alpha
    \end{equation}
    is a global Kuranishi chart for the moduli space $\Mbar^{\wh X,\,J}_{\sft}(\Gamma^+,\Gamma^-;\beta)$.
\end{corollary}

\begin{proof}
    The proof is analogous to the proof of Theorem~\ref{thm:leveled-gkc}.
\end{proof}

The proof of the first assertion is analogous to the proof of Theorem~\ref{thm:pardon-gkc}\eqref{gkc-unobstructed-aux} and \eqref{gkc-rel-smooth}. We simply have to replace Lemma~\ref{lem:invariant-approximations-exist} with the following definition and existence result. Then the arguments carry over verbatim. The proof of Claim~\eqref{cobordism-gkc-orientation} follows from the arguments of \textsection\ref{subsec:orientation}.

\begin{definition}
Suppose $V\to B$ and $E^\pm \to B^\pm$ are three smooth $G$-vector bundles and that $B^\pm$ admits a free $\bR$-action, which commutes with the $G$-action and lifts to $E^\pm$. Suppose there exist open $G$-invariant subsets $B^\pm_{\circ} \sub B^\pm$ whose orbit under the $\bR$-action covers all of $B^\pm$ and which admit smooth $G$-equivariant open embeddings $j^\pm \cl B^\pm_{\circ} \hkra B$ with disjoint image lifting to embeddings of vector bundles. Assume additionally that the quotients $B^\pm/\bR$ and $B\sm \im(j^+)\sqcup \im(j^-)$ are compact. Then, a \emph{joint finite-dimensional approximation scheme} of $(V,E^\pm)$ is a sequence $(E_\ell,\mu_\ell,\mu^\pm_\ell)$ of finite-dimensional $G$-representations together with $G$-equivariant linear maps 
$$\mu_\ell \cl E_\ell \hkra C^\infty(B,V)$$
and 
$$\mu^\pm_\ell \cl E_\ell \hkra C^\infty_c(B,V)^\bR := \{\eta\in C^\infty(B^\pm,E^\pm)^\bR\mid \supp(\eta)/\bR \text{ is compact}\}$$
satisfying 
\begin{enumerate}[label=\roman*),leftmargin=20pt]
    \setlength\itemsep{2pt}
     \item $E_\ell$ is a subrepresentation of $E_{\ell+1}$ with $\mu_{\ell+1}|_{E_\ell} = \mu_\ell$ and $\mu^\pm_{\ell+1}|_{E_\ell} = \mu^\pm_\ell$,
     \item $\union{}{\im(\mu^\pm_\ell)}$ is dense in $C^\infty_c(B^\pm,E^\pm)^\bR$ in the $C^\infty_{\text{loc}}$-topology,
     \item $\supp(\mu_\ell(v)|_{\im(j_*^\pm)}-j_*^\pm\mu_\ell^\pm(v)|_{B_{\circ}^\pm})$ is precompact in $B$ for any $\ell\ge 1$ and $v\in E_\ell$,
     \item $\union{}{\im(\mu_\ell-{j^+}_*\mu^+_\ell-{j^-}_*\mu^-_\ell)}$ is dense in $C^\infty_c(B,V)$ in the $C^\infty_{\text{loc}}$-topology.
 \end{enumerate}
 Note that the last property makes sense due to the third one.
\end{definition}


\begin{lemma} \label{lem:joint-fin-dim-scheme}
    Given finite approximation schemes $\mu^\pm_*$, there exists a choice of $\mu_*$ such that $(\mu^+,\mu, \mu^-)$ forms a joint finite-dimensional approximation scheme.
\end{lemma}

\begin{proof}
    We adapt the proof of \cite[Lemma~4.2]{AMS23}. Fix $G\times \bR$-invariant connections $\conn^\pm$ on $E^\pm$ and let $\conn$ be a $G$-invariant connection on $V$ so that ${j^\pm}^*\conn$ agrees with $\conn^\pm|_{B^\pm_{\circ}}$ away from a subset $K \sub B^\pm_\circ$, which is precompact in $B^\pm$.
    Let $A \coloneqq B\,\sm \;  (\im(j^+)\cup \im(j^-))$ and let $(A_n)_n$ be a countable exhaustion of $A$ so that each $A_n$ is a smooth $G$-invariant manifold with boundary. Fix an increasing sequence$(\rho_{n,k})_k$ of $G$-invariant smooth bump functions with support in $A_n$ and $A_n = \union{}{\rho_{n,k}\inv(1)}$ for each $n$.
    
    Similarly, let $(B^\pm_n)_n$ be a countable exhaustion of $B^\pm_{\circ}$ by $G$-invariant smooth manifolds with boundary, and let $\rho^\pm_n$ be a $G$-invariant smooth bump function that is identically $1$ on $B^\pm_n$ and supported in $B^\pm_{\circ}$.
    Then, let $\wch\conn^\pm$ be the induced connection on $E^\pm/\bR\to B^\pm/\bR$ and let $(\lambda_\ell^\pm)_\ell$ be the increasing sequence of non-negative eigenvalues of the Laplacian associated to $\wch\conn^\pm$. 
    
    Let $\wch W_\ell^\pm$ be the preimage of the space of eigenfunctions associated to the non-negative eigenvalues $\lambda_j^\pm$ with $j \leq \ell$ and let $W^\pm_\ell \sub C^\infty_c(B^\pm,V^\ell)^\bR$ be the preimage of $\wch{W}_\ell^\pm$. Define 
    $$E^\pm_\ell := \bigoplus\limits_{n\leq \ell} W^\pm_n$$ 
    and let $\mu^\pm_\ell \cl E^\pm_\ell \to C^\infty_c(B^\pm,E^\pm)^\bR$ be the inclusion on each summand. Then, define 
    $$\mu_\ell \cl E^\pm_\ell \to C^\infty(B,V)$$ 
    by 
    $$\mu_\ell((v_n)_n) = \s{n\leq \ell}{\rho_n^\pm\,j^\pm_*\mu^\pm_n(v_n)}.$$
    Finally, doubling $A_{n+1}$ and $V|_{A_{n+1}}$ and considering the eigenspaces of the Laplacian of the induced connection on the doubled vector bundle, we obtain for each $n$ a sequence of vector spaces $(E_{n,k}\inn)_k$ together with maps 
    $$\mu_{n,k}\cl E_{n,k}\inn\to C^\infty_c(B,V) : v \mapsto \rho_{n,k}\, v.$$
    Define for $\ell \ge 1$ the vector space 
    $$E\inn_\ell := \bigoplus\limits_{n,k\leq \ell} E_{n,k}\inn$$
    and let $\mu_\ell \cl E\inn_\ell \hkra C^\infty_c(B,V)$ be the canonical map induced by the maps $\mu\inn_{n,k}$. We finally define 
    $$E_\ell \coloneqq V^+_\ell \oplus E\inn_\ell \oplus V^-_\ell$$
    and let $\mu_\ell$ be given by the sum of the maps $\mu_\ell$ defined above. Extend $\mu^\pm_\ell$ to $E_\ell$ by letting it be $0$ on $E\inn_\ell\oplus V^\mp_\ell$.
\end{proof}


\subsubsection{Disconnected buildings in symplectic cobordisms}
Given a symplectic cobordism $(\wh X,\omega)$ as above and sequences $\Gamma^\pm$ of Reeb orbits of $\lambda^\pm$ as well as a partition $\Lambda\cl \Gamma^-\to \Gamma^+$ and a sequence $\beta (\beta_\gamma)_{\gamma\in \Gamma^+}$ of relative homology classes, we define the moduli space $\Mbar^{\wh X,J}_{\sft}(\Gamma^+,\Gamma^-;\beta)$ of disconnected leveled buildings in $\wh X$ exactly as in Definition~\ref{de:disconnected-domains} except that (one level of) the buildings now maps to the symplectization.\par 

Given $\gamma\in \Gamma^+$ let $\cK_\gamma^c$ be the global Kuranishi chart for $\Mbar^{\wh X,J}(\gamma,\Lambda_\gamma;\beta_\gamma)$ with base space $\cBS_{c,\gamma}$ given by Theorem~\ref{thm:cobordism-gkc}. Recall that $\cBS_{c,\gamma}$ was defined in \S\ref{subsec:cobordism-base} as the corner blow-up of $\cBR_{*}$ corresponding to refinement using maximally leveled pre-cobordism trees. Similarly, let $\cBS_{c,\Lambda}$ be the corner blow-up of $\p{\gamma\in \Gamma^+}{\cBR_{c,\gamma}}$ corresponding to refinements as in \textsection\ref{subsubsec:blowup_to_leveled_base} but using maximally leveled pre-cobordism forests.

\begin{proposition}\label{prop:disconnected-in-cobordims}
 The pullback global Kuranishi chart
   $$\cK^\bR_{c,\Lambda} \,\coloneqq\, \cBS_{c,\Lambda}\times_{\p{\gamma}{\cBR_{c,\gamma}}}\p{\gamma}{\cK^c_{\gamma}}$$ 
   is a global Kuranishi chart for $\Mbar^{\wh X, J}_{\sft}(\Gamma^+,\Gamma^-;\beta)_\Lambda$.
\end{proposition}

\begin{proof}
    The proof is analogous to the proof of Theorem~\ref{thm:leveled-gkc}.
\end{proof}