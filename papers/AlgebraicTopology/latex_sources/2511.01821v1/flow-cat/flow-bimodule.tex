\subsection{Flow bimodules from symplectic cobordisms}\label{subsec:flow-bimodule} 
In this subsection we show that exact symplectic cobordisms induce flow bimodules between the flow categories constructed in \S\ref{subsec:unstructured-flow-cat}. Let $(\wh X,\omega)$ be an exact symplectic cobordism from $(Y^+,\lambda^+)$ to $(Y^-,\lambda^-)$, equipped with an $\omega$-adapted almost complex structure $J$. Suppose we are given action bounds $L^\pm\ge 0$ and pre-perturbation data $\fD^\pm = (\wt\lambda^\pm,\nabla^\pm, p^\pm )$ for $\cP_{L^\pm}$, where 
\begin{itemize}
    \item $\wt \lambda ^\pm$ is a $\cP_{L^\pm}(Y^\pm)$-integral approximations of $\lambda^\pm$,
    \item $\wh \nabla^\pm$ are complex linear connections on $\wh X$ and $\wh Y^\pm$, respectively, such that  $\nabla_o$ restricts to $\wh \nabla^\pm$ on the corresponding end and the connections on $Y^\pm$ are translation invariant;
    \item prime numbers $p^+, p^-$ satisfying~\eqref{eq:primes-for-base}.
\end{itemize}


\begin{theorem}\label{thm:flo_bim}
    Given a pre-perturbation datum $\fD$ extending $\fD^\pm$, there exists a flow bimodule $\scN^{\wh X}$ from the flow category $\scM_{\leq L^-}^{Y^-}$ to the flow category $\scM_{\leq L^+}^{Y^+}$. The morphism space from an object $\Gamma^- \in \cP_{L^-}(Y^-)$ to $\Gamma^+\in\cP_{L^+}(Y^+)$ is
    \begin{equation*}\label{eq:morphisms-bimodule}
        \scN^{\wh X}(\Gamma^-,\Gamma^+) = \djun{\Lambda}\,\Mbar_{\sft}^{\wh X,\, J}(\Gamma^+,\Gamma^-;\beta)_\Lambda
    \end{equation*}
   respectively, a global Kuranishi chart of said moduli space.
\end{theorem}


\begin{proof}
    We only sketch the construction of $\scN^{\wh X}$ because of its similarity to the construction of the flow category in \ref{subsec:unstructured-flow-cat}. There is a natural extension of the definition of the `norm' of $\|(\Gamma^+ , \Gamma^-) \|$ for $\Gamma^\pm \in \cP_{L^\pm}(Y^\pm)$ given by the height of the tallest building in $\Mbar_{\sft}^{\wh X,\, J}(\Gamma^+,\Gamma^-;\beta)_\Lambda$ (if it were unobstructed). The inductive construction of perturbation data for the moduli spaces begins similarly as before. By the choice of perturbation datum $\fD^\pm$ and the constructions of \textsection\ref{subsec:unstructured-flow-cat}, we are given global Kuranishi charts for the morphism spaces between objects in the same symplectization. Hence, we only need to construct charts for the moduli of buildings from an object of $\scM_{\leq L^-}^{Y^-}$ to an object of $\scM_{\leq L^+}^{Y^+}$ so that its boundary strata are compatible with the already chosen charts in the construction of $\scM_{\leq L^\pm}^{Y^\pm,\lambda^\pm}.$ 
    
    We do this inductively as well. However, both due to the formalism and due to the geometry of the moduli spaces, we have to make an artificial choice: we let 
    \begin{equation*}\label{eq:trivial-cobordism}
        \scN^{\wh X}(\emptyset_{Y^+},\emptyset_{Y_-})\coloneqq *
    \end{equation*}  
    be the trivial global Kuranishi chart for a point. This choice is forced on us due to the following phenomenon in cobordisms: there can be a family of holomorphic planes in $\wh X$ that escape off to $\wh Y^+$, resulting in a two-leveled building, which has a holomorphic plane in $\wh Y^+ $ and an empty level in the symplectization. This phenomenon can occur whenever curves have no negative punctures. 
    
    Given a partition $\Lambda\cl \Gamma^+\to \Gamma^-$ of norm $0$, extend $\fD$ to an arbitrary perturbation datum $\alpha_{\Gamma^\pm}$ for $\Mbar^{\wh X,\, J}_{\sft}(\Gamma^+,\Gamma^-)_\Lambda$ as in Definition~\ref{de:auxiliary-datum}. Let 
 $$\scK_{\Lambda}^c = \scK^c_{\alpha_{\Lambda}} = (\wh G_{\Lambda},\scT_{\Lambda},\scE_{\Lambda},\fs_{\Lambda})$$ 
 be the associated global Kuranishi chart for $\Mbar_{\sft}^{\wh X,\, J}(\Gamma^+,\Gamma^-;\beta)_\Lambda$ as in Lemma~\ref{lem:embedding-inductive-step}, obtained from the chart $\cK_{\alpha_{\Lambda}}$ of Proposition~\ref{prop:disconnected-in-cobordims}. The proofs of the counterparts of Propositions \ref{prop:embedding-base-spaces} and \ref{lem:associativity-of-embeddings} are the same except for the added notational complexity required to keep track of the targets.\par
 An important observation is that the category of trees that stratifies $\scBS_c$ naturally carries the information of an order $\mathrm{d}_e $ for every edge $e$, obtained from Definitions~\ref{de:type-for-exact-cobordism} and~\ref{de:type-for-symplectisation}. The other change one has to make is that the concatenation $T_e\#T_c$, of a leveled forest $T_e$ labeling a stratum in $\scBS$ with a leveled forest $T_c$ labeling a stratum in $\scBS_c$, is a leveled forest, which labels a stratum in $\scBS_c$. For the cobordism counterparts of Lemma \ref{lem:embedding-inductive-step} and Proposition \ref{prop:embedding-gkc}, the only difference lies in the construction of the embedding maps between thickenings; see \S \ref{subsec:embeddings-thickening}. While choosing finite-dimensional approximation scheme $E_*$, we use Lemma \ref{lem:joint-fin-dim-scheme} to obtain a joint finite-dimensional approximation scheme. The rest of the construction follows similarly. 
\end{proof}


An important bimodule from a flow category $\scX$ to itself is the \emph{diagonal bimodule} $\Delta_\scX$, which can be thought of as the identity morphism. It has objects given by two copies $\cP^\pm$ of the symmetric sets $\cP$ of objects of $\scX$ and morphisms given by 
\begin{equation}\label{eq:morphisms-diagonal-bimodule}
    \Delta_\scX(x,y) = \begin{cases}
        \cD \scX(x,y) \quad & x \neq y \\
        * \quad & x= y,
    \end{cases}
\end{equation}
where $\cD X$ is the conic degeneration of an orbifold, \cite[\textsection 6.1]{AB24}. It comes with a natural map $d\cl\cD X\to X$, and we write $\cD\cK = (\cD\cT,d^*\cE,d^*\fs)$ for the conic degeneration of a derived orbifold. In particular, $\Delta_\scX(x,y)$ defines an object of the category $\dOrb_{/\cdot}$ defined in Definition~\ref{de:derived-orbifold-category}

\begin{lemma}\label{lem:trivial-cobordism-diagonal-bimodule}
    If $(\wh X,\omega ) = (\wh Y,d(e^s\lambda))$ is the trivial cobordism equipped with the almost complex structure $J$, then $\scN_{\leq L}^{\wh X}$ of Theorem~\ref{thm:flo_bim} is equivalent to the diagonal bimodule.
\end{lemma}

\begin{proof}
    We will use slightly different perturbation data to construct the flow bimodule in this case. The proof of equivalence of global Kuranishi charts in \cite[Proposition~6.1]{HS22} then shows that the associated flow bimodule is equivalent to the one constructed in Proposition~\ref{thm:flo_bim}. Here we say two flow bimodules are equivalent if they have the same objects and their morphism spaces are equivalent compatibly with structure maps.\par
    Let $\fD$ be the pre-perturbation datum chosen for the construction of $\scM_{\le L}^Y$ and let $\{\alpha_\Lambda\}_\Lambda$ be the collection of perturbation data constructed inductively in \textsection\ref{subsec:unstructured-flow-cat}. Then, $\fD$ is also a pre-perturbation datum for the symplectic cobordism $\wh X$ and $\wt\alpha_\Lambda\coloneqq \alpha_\Lambda$ defines a perturbation datum for the moduli space $\Nbar^J_{\sft}(\Gamma^+,\Gamma^-)$ of buildings in $\wh X$, denoted by $\Nbar$ instead of $\Mbar$ in order to distinguish it from moduli spaces of buildings in the symplectisation. By definition, $\scN^{\wh X}(\emst,\emst)$ is a point, while for a nonempty sequence $\Gamma$, the moduli space of trivial cylinders in $\wh X$ is regular and a point, whence $\scN^{\wh X}(\Gamma,\Gamma) = *$ as well. Suppose $\Gamma^-\neq \Gamma^+$ and let $\Lambda\cl \Gamma^-\to\Gamma^+$ be a partition. By Lemma~\ref{lem:auxiliary-thickening}, the thickening $\scT^c_\Lambda$  of the global Kuranishi chart $\scK^c_\Lambda$ for $\Nbar^{\,J}(\Gamma^-,\Gamma^+)_\Lambda$ admits a canonical equivariant rel--$C^1$ map $q\cl \scT^c_\Lambda\to \scT_\Lambda$, which is a fiber bundle of intervals. Moreover, $\scE^c_\Lambda = q^*\scE_\Lambda$ and the obstruction section is pulled back as well. Since $\text{Homeo}_+([0,1])$ is contractible, one can lift $q$ to a homeomorphism $\scT^c_\Lambda\to \scT_\Lambda\times [0,1]$. Using the explicit description of the composition maps, this shows that the forgetful map $\scT^c_\Lambda\to \scB_\Lambda$ factors through the conic degeneration $\cD\scB_\Lambda$ of $\scB_\Lambda$ and that $\scT^c_\Lambda = \cD\scB_\Lambda\times_{\scB_\Lambda}\scT_\Lambda$. Since the map $\scT^c_\Lambda\to \cD\scB_\Lambda$ is compatible with the bimodule structure maps, the claim follows.
\end{proof}

