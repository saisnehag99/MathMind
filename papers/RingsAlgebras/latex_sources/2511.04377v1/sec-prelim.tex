\section{Preliminary definitions and known results}\label{sec:prelim}
\begin{definition}
    Let $G$ be a complex Lie group and $w\in \F_r$ be an element from the free group on $r$ generators, say $x_1,x_2,\ldots,x_r$. 
    This defines a \emph{word map on $G$},
    $(g_1,g_2,\cdots,g_r)\mapsto w(g_1,g_2,\cdots,g_r)$, by plugging $g_i$ in place of $x_i$.
    Given $r$ words $w_1,\cdots,w_r$ one gets a map $\underline{w}:G^r\longrightarrow G^r$ by defining: $$\underline{g}=(g_1,g_2,\cdots,g_r)\mapsto({w_1(\underline{g}),\cdots,w_r(\underline{g})}).$$
    The \emph{Fatou set of the pair $(G,\underline{w}=(w_1,\ldots,w_r))$} consists of the set of points $x\in G^r$ which have a neighborhood $U$ such that $(\underline{w}^n|_U)_n$ is normal (recall that a family $\mathscr F$ of functions on a domain $U\subseteq \C^m$ is called normal if every sequence in $\mathscr F$ contains a subsequence that converges locally uniformly in $U$.). 
    Here $\underline{w}^1(x)=\underline{w}(x)$, $\underline{w}^2(x)=\underline{w}(\underline{w}(x))$ and more generally $\underline{w}^n(x)=\underline{w}(\underline{w}^{n-1}(x))$.
    
    With necessary modification, for a topological $\C$-algebra $A$ and $f_1,\cdots,f_r\in\C\langle x_1,x_2,\ldots,x_r\rangle$ the \emph{Fatou set of the pair $(A,\underline{f}=(f_1,\cdots,f_r))$} consists of the set of points $x\in A^r$ which have a neighborhood $U$ such that $(\underline{f}^n|_U)_n$ is normal.
    
    The \emph{Julia set} of the pair is defined to be the set-theoretic complement of the Fatou set.
\end{definition}
The Julia set for the pair $(G,w)$, of a group and a word, is denoted by $\J_w(G)$, and the Fatou set is denoted by $\F_w(G)$; similar notation is followed for the pair $(A,f)$, of an algebra and a polynomial.
Our definitions are motivated by the theory of several complex variables.
In this article, we compute the Fatou set (consequently Julia set) for the pairs $(\GL_n(\mathbb C),x^M)$ and $(\M_n(\mathbb C), x^M+\sum\limits_{i=0}^{M-1}b_ix^i)$, where $\GL_n(\C)$ is considered to be a topological group and $\M_n(\C)$ is considered to be a topological algebra; here $a_i\in \C$ for all $i=1,2,\ldots, M-1$.

In the spirit of the theory of a single complex variable, we define the \emph{filled Julia set} for the pair $(G,\und{w})$ to be 
\begin{align*}
    K_{\und{w}}(G)=\left\{x\in G^r:\left(\und{w}^n(x)\right)_{n\geq 0}\text{ is a bounded sequence}\right\}.
\end{align*}
For a polynomial function (of one variable) $p$ on $\C$, one has the Julia set $J_p(\C)=\partial K_p(\C)$; see \cite[Lemma 9.4]{Milnor2006book}.
However, this need not continue to hold in general, as we see in later sections.
So one needs to be careful while dealing with the case of several variables.
% \begin{definition}
%     Let $G$ be a complex Lie group (resp. $A$ be a $\C$-algebra). Given a word map (resp. polynomial map) $\chi$, the associated Fatou set is the disjoint union of its connected components, known as \emph{Fatou components}.
%     The Fatou components are permuted by $\chi$.
%     For a Fatou component $U\subseteq G^r$ (resp. $\subseteq A^r$), it is (a) \emph{periodic} if $\chi^p(U)=U$ for some $p>0$, (b) \emph{preperiodic} if $\chi^k(U)$ is periodic, and (c) \emph{wandering Fatou component} if the sets $\left\{\chi^k(U)\right\}_{k\geq 0}$ are pairwise disjoint.
% \end{definition}
For later use, we tally the Sullivan classification of Fatou Components.
\begin{lemma}\cite[Theorem 16.1]{Milnor2006book}\label{lem:suli-fatou-comp}
Let $f$ be a rational function on the Riemann sphere. 
If $f$ maps the Fatou component $U$ onto itself,
then there are just four possibilities, as follows: Either $U$ is the
immediate basin for an attracting fixed point or for one petal of a parabolic fixed point which has multiplier $\lambda=1$ or else $U$ is a Siegel disk or Herman ring.
\end{lemma}
Further by \cite[Lemma 9.4]{Milnor2006book}, all the bounded components of the Fatou set of a polynomial of degree $\geq 2$ are simply connected. 
Thus while analyzing the elements of the Fatou components of the polynomial $f$, one gets exactly three possibilities; Herman ring does not occur (as it is conformally isomorphic to some annulus).
To check whether the family $\{w^m\}_{m\geq0}$ is normal or not, we use the criterion developed by Gerado and Krantz which applies to complex manifold. 
For future reference we note it down here.
\begin{lemma}\cite[Theorem 3.1]{GeradoKrantz1991}\label{lem:normal-GK}
Let $\mathscr{M}\subseteq\C^n$ be a complex manifold. Let $N$ be a complete
complex Hermitian manifold of dimension $k$. 
Let $\mathscr{F}=\left\{f_\alpha\right\}_{\alpha\in A}\subseteq \mathrm{Hol}(\mathscr{M},N)$.
The family $\mathscr{F}$ is not normal if and only if there exist (a) a compact set $K\subseteq\subseteq \mathscr{M}$, (b) a sequence $\{p_j\}\subseteq K$, (c) a sequence $\{f_j\}\subseteq \mathscr{F}$, (d) a sequence of reals $\{\rho_j\}$ satisfying $\rho_j>0$ and $\rho_j\longrightarrow 0^+$ and (e) a sequence of Euclidean vectors $\{\varepsilon_j\}\subseteq \C^n$ such that
\begin{align*}
g_j(\zeta)=f_j(p_j+\rho_j\varepsilon_j\zeta),\,\,\,\zeta\in \C,
\end{align*}
converges uniformly on compact subsets of $\C$ to a \emph{nonconstant} entire function $g$.
\end{lemma}