\begin{convention}
    \begin{itemize}
        \item In this section, we do not identify $\GL_2(\RR)$ with $\GU(J_2)^{\prime}(\RR)$ automatically. 
        \item In this section, we choose measures as in Convention \ref{convention: measure}. 
        \item Let $n = a + b + r + s$.
    \end{itemize}
\end{convention}


\subsection{The zeta integral of Gelbart and Piatetski-Shapiro}
\label{SS: global zeta integral}
In this subsection, we recall the zeta integral constructed in \cite{Gelbart&PS84} and modified in \cite{PS18}.

\begin{convention}
\begin{itemize}
    \item Let $dt_{\infty}$ be the Lebesgue measure on the additive group $\RR$. If $v$ is a nonarchimedean place of $\QQ$, let $dt_{v}$ be the Haar measure on $\QQ_{v}$ such that $\ZZ_{v}$ has volume one.
    \item 
    Let $d^{\times}t_{v}$ be the measure on $\QQ_{v}^{\times}$ by 
    \begin{equation*}
        d^{\times}t_{v} = \begin{cases}
                             & \frac{dt_{v}}{|t_v|} \quad \text{if} \ v \ \text{is archimedean}, \\
                             & \frac{p}{p - 1}\frac{dt_{v}}{|t_v|} \quad \text{if} \ v \ \text{is nonarchimedean}.  
                          \end{cases}
    \end{equation*}
    \item 
    Let $dt$ (resp., $d^{\times}t$) denote the restricted product measure $\prod_{v}dt_{v}$ on $\AAA$ (resp., the restricted product measure $\prod_{v}d^{\times}t_{v}$ on $\AAA^{\times}$). 
\end{itemize}
\end{convention}

\begin{definition}
    Let $V_{2}$ be the standard representation of $\GL_2$ on row vectors and let $B_2 = T_2U_2$ be the standard Borel of $\GL_2$. Let $\Phi$ be a Schwartz-Bruhat function on $V_{2}(\AAA)$ and let $\nu = (\nu_1, \nu_2) \colon T(\AAA)/T(\QQ) \rightarrow \CC^{\times}$ be a character of the diagonal torus of $\GL_2$. We now define
    \begin{equation*}
        f(g, \Phi, \nu, s) = \nu_{1}(\det(g))|\det(g)|^{s} \int_{\GL_1(\AAA)} \Phi((0, t)g) \nu_1\nu_2^{-1}(t) |t|^{2s} dt
    \end{equation*}
    and set the Eisenstein series on $\GL_2(\AAA)$ to be 
    \begin{equation*}
        E(g, \Phi, \nu, s) = \sum\limits_{\gamma \in B(\QQ)\backslash GL_2(\QQ)} f(\gamma g, \Phi, \nu,s). 
    \end{equation*}
    We write $f(g_1, \Phi, \nu, s)$ and $E(g_1, \Phi, \nu, s)$ as functions on $H(\AAA)$ via the projection $H(\AAA) \twoheadrightarrow \GL_2(\AAA)$. 
\end{definition}
\begin{remark}
    \begin{itemize}
        \item The general theorem \cite[Lemma 4.1]{Langlands76} implies that the Eisenstein series 
        \begin{equation*}
            E(g, \Phi, \nu, s) = \sum\limits_{\gamma \in B_{2}(\QQ)\backslash \GL_2(\QQ)} f(\gamma g, \Phi, \nu, s)
        \end{equation*}
        is absolutely convergent for $\mathrm{Re}(s)$ big enough and satisfies a functional equation. 
        \item By the explicit formula for $f(g, \Phi, \nu, s)$, we can see that the central character is $\nu_1\nu_2$.
    \end{itemize}
\end{remark}

\begin{definition}[Global zeta integral]
\label{def: Zeta_integral}
    For an irreducible cuspidal automorphic representation $\pi$ of $G(\AAA)$ and a cusp form $\varphi$ in the space of $\pi$, we define the global zeta integral as follows: 
    \begin{equation*}
        I(\varphi, \Phi, \nu, s) = \int_{H(\QQ)Z(\AAA) \backslash H(\AAA)} \varphi(g) E(g_1, \Phi, \nu, s) dg. 
    \end{equation*}
\end{definition}

\subsection{Comparison of the integral and the zeta integral}
\label{SS: comparison of integral with zeta integral}

In this subsection, we express the integral 
\begin{equation*}
    \int_{H(\QQ) Z(\AAA) \backslash H(\AAA)} \Xi_{m, t}(\phi_f)(g) dg
\end{equation*}
in Proposition \ref{prop: explicit_pairing} in terms of the zeta integral in Definition \ref{def: Zeta_integral}. 

\begin{proposition}
\label{Proposition_comp_pair_w_zeta_int}
Let $\nu_1, \nu_2 \colon \QQ^{\times} \backslash \AAA^{\times} \rightarrow \CC^{\times}$ be continuous characters and let $s \in \CC$. Let $\chi_{\nu_1, \nu_2,s}$ be the character of $B_{2}(\AAA)$ defined by 
\begin{equation*}
    \chi_{\nu_1, \nu_2, s}(\begin{pmatrix}
                                a & b \\
                                  & d 
                           \end{pmatrix}) = \nu_1(a) \nu_2(d) \left|\frac{a}{d}\right|^{s}. 
\end{equation*}
Then for any Schwartz-Bruhat function $\Phi$ on $\AAA^{2}$, the function $f(g, \Phi, \nu, s)$ on $\GL_2(\AAA)$ defined by 
    \begin{equation}
    \label{eq: def of f}
        f(g, \Phi, \nu, s) = \nu_1(\det(g)) |\det(g)|^{s} \int_{\GL_1(\AAA)} \Phi((0, t)g) \nu_1\nu_2^{-1}(t) |t|^{2s} dt
    \end{equation}
    belongs to $\Ind_{B_2(\AAA)}^{\GL_2(\AAA)}\chi_{\nu_1, \nu_2, s}$, where $\Ind$ means unnormalized induction. 
\end{proposition}

\begin{proof}
    The proposition follows from the following computations:
    \begin{equation*}
         f(\begin{pmatrix}
            1 & b \\
              & 1 \\
          \end{pmatrix}g, \Phi, \nu, s) = f(g, \Phi, \nu, s), 
    \end{equation*}
    \begin{align*}
        f(\begin{pmatrix}
              a_1 &     \\
                  & d_1 \\
           \end{pmatrix}g, \Phi, \nu, s) &= \nu_1(a_1d_1) |a_1d_1|^{s} \nu_1\nu_2^{-1}(d_1^{-1}) |d_1|^{-2s}  f(g, \Phi, \nu, s) \\
                                         &= \nu_1(a_1) \nu_2(d_1)  \left|\frac{a_1}{d_1}\right|^{s} f(g, \Phi, \nu, s). 
    \end{align*}
\end{proof}

\begin{proposition}
\label{Prop: choice of v1,v2}
    Let $\nu_1$ be the Hecke character  $||^{-n}\nu$ and $\nu_2$ be the trivial Hecke character, where $\nu$ is a finite-order Hecke character of sign $(-1)^n$. Then the following statements hold.
    \begin{itemize}
        \item The nonarchimedean part $\chi_{f}$ of $\chi_{\nu_1, \nu_2, 1 + n}$ satisfies 
        \begin{equation*}
            \chi_{f}(\begin{pmatrix}
                        a\alpha & b \\
                                & d\delta \\
                    \end{pmatrix}) = a^{-1}d^{n + 1} \nu(\alpha), 
        \end{equation*}
        for any $a, d \in \QQ_{+}^{\times}$ and $\alpha, \delta \in \widehat{\ZZ}^{\times}$. 
        \item The restriction of the archimedean part $\chi_{\infty}$ of $\chi_{\nu_1, \nu_2, 1 + n}$ to the identity component of the diagonal maximal torus of $\GL_2(\RR)^{+}$ is identified with $\lambda(n + 2, -n)$ (see notation \ref{Notation: character on GL2}) through the isomorphism $\GL_2(\RR)^{+} \cong \GU({J_2})^{\prime}(\RR)^{+}$. 
    \end{itemize}
\end{proposition}

\begin{proof}
    For the nonarchimedean part, we have 
    \begin{equation*}
         \chi_{f}(\begin{pmatrix}
                        a\alpha & b \\
                                & d\delta \\
                    \end{pmatrix}) = |a|_{f}^{-n}\nu(\alpha) |\frac{a}{d}|_{f}^{1 + n} = a^{-1}d^{n + 1} \nu(\alpha).
    \end{equation*}
    For the archimedean part, for $\begin{pmatrix}
                                     a & \\
                                       & a^{-1}v \\
                                  \end{pmatrix} \in \GL_2(\RR)^{+}$, 
    we have 
    \begin{equation*}
        \chi_{\infty}(\begin{pmatrix}
                        a &  \\
                                & a^{-1}v \\
                    \end{pmatrix}) = |a|_{\infty}^{-n}|a^{2}v^{-1}|_{\infty}^{1 + n} = a^{n + 2} v^{\frac{-n - (n + 2)}{2}}.
    \end{equation*}
    Hence, the action of $\chi_{\infty}$ is $\lambda(n + 2, -n)$ via the isomorphism $\GL_2(\RR)^{+} \cong \GU({J_2})^{\prime}(\RR)^{+}$. 
\end{proof}

\begin{proposition}
\label{Prop: identify f with phi}
    Let $\nu_1$ be the Hecke character  $||^{-n}\nu$ and $\nu_2$ be the trivial Hecke character, where $\nu$ is a finite-order Hecke character of sign $(-1)^n$.
    Let $\Phi$ be a factorizable Schwartz-Bruhat function $\Phi = \otimes_{v}^{\prime} \Phi_{v}$ on $V_{2}(\AAA)$ such that 
    \begin{enumerate}
        \item The function $\Phi_{f} = \otimes_{v < \infty}^{\prime} \Phi_{v}$ is $\overline{\QQ}$-valued, 
        \item The function $\Phi_{\infty}(x, y)$ is defined by
        \begin{equation*}
                  \Phi_{\infty}(x, y) := (ix - y)^{\frac{n - k}{2}}(ix + y)^{\frac{n + k}{2}} e^{-\pi(x^2 + y^2)}. 
        \end{equation*}
    \end{enumerate}
    Then there exists a $\phi_f \in \mathcal{B}_{n, \overline{\QQ}}$ such that for any $g \in \GL_{2}(\AAA)$, we have 
    \begin{equation*}
        (\phi_{k}^{n} \otimes \phi_f)(g) = (-1)^{\frac{a + b + r + s - k}{2}} 2i \pi^{1 +a + b + r + s} \Gamma(1 + a + b + r + s)^{-1} f(g, \Phi, \nu, 1 + a + b + r +s), 
    \end{equation*}
    where the $\phi_{k}^{n}$ on the left-hand side the pullback of the function $\phi_{k}^{n}$ defined in Definition \ref{def: function phi} through the isomorphism $\GL_2(\RR) \cong \GU(J_2)^{\prime}(\RR)$ and $f$ is defined in equation (\ref{eq: def of f}). 
\end{proposition}

\begin{proof}
    \par First, we can factor $f(g, \Phi, \nu, s)$ into an Euler product of local Tate integrals 
    \begin{equation*}
        f(g, \Phi, \nu, s) = \otimes_{v}^{\prime} f(g_{v}, \Phi_{v}, \nu_{v}, s), 
    \end{equation*}
    where 
    \begin{equation*}
        f(g_{v}, \Phi_{v}, \nu_{v}, s) = \nu_{1, v}(\det(g_{v}))|\det(g_{v})|^{s} \int_{\GL_1(\QQ_{v})} \Phi((0, t_v)g_{v}) (\nu_{1, v}\nu_{2, v})^{-1}(t_v) |t_v|_{v}^{2s} dt_v. 
    \end{equation*}
    \par Second, it follows from Proposition \ref{Proposition_comp_pair_w_zeta_int}, Proposition \ref{Prop: choice of v1,v2} and Defintion \ref{Def: In(v)} that 
         \begin{equation*}
             f^{\infty}(g, \Phi, \nu, 1 + n) = \otimes_{v < \infty}^{\prime} f(g_{v}, \Phi_{v}, \nu_{v}, 1 + n)
         \end{equation*}
         belongs to $I_n(\nu) \subset \mathcal{B}_{n, \overline{\QQ}}$ for $g = (g_{v})_{v < \infty} \in G(\AAA_f)$. Hence, there exists a $\phi_f \in \mathcal{B}_{n, \overline{\QQ}}$ such that 
        \begin{equation*}
            \phi_f = f^{\infty}(g, \Phi, \nu, 1 + n)
        \end{equation*}
        for $g \in G(\AAA_f)$. 
    \par Third, it can seen from the fact that $\phi^{n}_{k}$ (see Proposition \ref{prop: phi equivariant}) and $\Phi_{\infty}$ (see Lemma \ref{Lemma: arch_weight of Phi}) has $\U(1)(\RR)$-weight $-k$ and $\Ind_{B_2(\RR)^{+}}^{\GL_2(\RR)^{+}} \lambda(2 + n, -n)$ is $\U(1)(\RR)$-multiplicity-free that there is a $c \in \CC$ such that for any $g \in G(\RR)$, 
    \begin{equation*}
        \phi^{n}_{k}(g) = c \cdot f(g, \Phi_{\infty}, \nu_{\infty}, 1 + n),
    \end{equation*}
    where $\nu_{\infty} = (\nu_{1, \infty} = ||^{-1}\mathrm{sgn}^{n}, \nu_{2, \infty} = 1)$ is the archimedean component of $\nu = (\nu_1, \nu_2)$. 
    Since $\phi_{k}^{n}(1) = 2i$ (see Proposition \ref{def: function phi}) and
    \begin{equation*}
        f(1, \Phi_{\infty}, \nu_{\infty}, 1+n) = (-1)^{\frac{a + b + r + s - k}{2}} \pi^{-1 -a - b - r - s} \Gamma (1 + a + b + s)
    \end{equation*}
    (see Lemma \ref{Lemma: arch f formula}), 
    we get
    \begin{equation*}
        c = (-1)^{\frac{a + b + r + s - k}{2}} 2i \pi^{1 +a + b + r + s} \Gamma(1 + a + b + r + s)^{-1}. 
    \end{equation*}
\end{proof}

\begin{corollary}
\label{corollary: identify integral of Xi and I}
    There exists a $\phi_f \in \mathcal{B}_{n, \overline{\QQ}}$ such that the integral
    \begin{equation*}
        \int_{H(\QQ) Z(\AAA) \backslash H(\AAA)} \Xi_{m, k}(\phi_f)(g) dg
    \end{equation*}
    defined in notation \ref{notation: Xi} is equal to 
    \begin{equation*}
        (-1)^{\frac{a + b + r + s - k}{2}} 2i \pi^{1 +a + b + r + s} \Gamma(1 + a + b + r + s)^{-1} I(\varphi, \Phi, \nu, 1 + a + b + r + s),
    \end{equation*}
    where $\varphi = X^{m}_{(1, -1, 0)} \Psi$ and the choice of $\Phi$ and $\nu$ is explained in Proposition \ref{Prop: identify f with phi}. 
\end{corollary}

\begin{proof}
    This is direct from Proposition \ref{Prop: identify f with phi}. 
\end{proof}


\subsection{The unfolding}
\label{SS: unfolding}
In this subsection, we unfold the zeta integral defined in Subsection \ref{SS: global zeta integral} and factor it into the product of the standard automorphic $L$-function and a local integral of Whittaker functions. 

\begin{definition}
\label{def: Whittaker function}
    \begin{enumerate}
        \item (Global Whittaker functionals) We define a non-zero global Whittaker functional $\Lambda \colon \pi \rightarrow \CC$ on an irreducible cuspidal automorphic representation $\pi$ by the following integral: for $\varphi \in \pi$, 
        \begin{equation*}
            \Lambda(\varphi) := \int_{U_{B}(\QQ)\backslash U_B(\AAA)} \chi^{-1}(u) \varphi(u) du. 
        \end{equation*}
        Here, $U_{B}$ is the unipotent radical of the upper-triangular Borel subgroup of $G$ and define the character $\chi \colon U_{B}(\QQ) \backslash U_{B}(\AAA) \rightarrow \CC^{\times}$ by $\chi(u) = \psi(\Tr_{E/\QQ}(\delta^{-1}u_{23}))$, where $\psi$ is the standard additive character $\psi \colon \QQ\backslash \AAA \rightarrow \CC^{\times}$ and $u_{23}$ is the entry of $u$ in the position of the second row and the third column. 
        \item (Global Whittaker functions) For a cusp form $\varphi$ in the space of $\pi$, denote by
        \begin{equation*}
             W_{\varphi}(g) = \Lambda(\pi(g) \varphi) = \int_{U_{B}(\QQ)\backslash U_B(\AAA)} \chi^{-1}(u) \varphi(ug) du
        \end{equation*}
        the global Whittaker function associated to $\varphi$. 
        \item (Local Whittaker functions)
        We assume that $\varphi = \otimes_{v}^{\prime} \varphi_{v}$ is a pure tensor in the decomposition $\pi = \otimes_{v}^{\prime} \pi_{v}$ and such that $\Lambda(\varphi) = W_{\varphi}(1) \neq 0$. Then for each place $v$ of $\QQ$, we define a local Whittaker functional $W_{\varphi, v}\colon G(\QQ_{v}) \rightarrow \CC$ by setting $W_{\varphi, v}(g_{v}) = \Lambda(\pi(g_{v}) \varphi) / \Lambda(\varphi)$ for $g_v \in G(\QQ_v)$ and $\varphi \in \pi$. We then have 
        \begin{equation*}
            W_{\varphi}(g) = W_{\varphi}(1) \prod_{v} W_{\varphi, v}(g_v), 
        \end{equation*}
        where the product runs over all the places ${v}$ of $\QQ$. 
    \end{enumerate}
\end{definition}

\begin{convention}
    \begin{itemize}
        \item Let $\omega_{\pi}\colon Z_{G}(\QQ) \backslash Z_{G}(\AAA) \rightarrow \CC^{\times}$ be the central character of $\pi$. For later sections, we assume that $\omega_{\pi}|_{Z(\AAA)} = (\nu_1 \nu_2)^{-1}$. 
        \item When $\varphi$ is clear, we use $W_v$ to denote $W_{\varphi, v}$. 
    \end{itemize}
\end{convention}

\begin{definition}[Local zeta integral]
\label{def: local zeta integral}
    Let $U_2$ be the upper-triangular unipotent subgroup $\{\begin{pmatrix}
                                                               1 & * \\
                                                                 & 1
                                                            \end{pmatrix}\}$ of $\GL_2$. For a Whittaker function $W_{v}$ on $G(\QQ_{v})$ and a Schwartz-Bruhat function $\Phi_{v}$ on $V_{2}(\QQ_v)$, 
    we define the local zeta integral as
    \begin{align}
    \label{eqn: def of local I}
        I_{v}(W_{v}, \Phi_{v}, \nu_v, s) &:= \int_{Z(\QQ_{v})U_{2}(\QQ_v) \backslash H(\QQ_v)} f(g_{1, v}, \Phi_{v}, \nu_{v}, s) W_{v}(g_v) dg_{v}, \\
                                         &= \int_{U_{2}(\QQ_v) \backslash H(\QQ_v)} \nu_1(\det(g_{1, v})) \Phi_{v}((0,1)g_{1,v}) W_{v}(g_v) |\det(g_{1,v})|_{v}^{s}dg_{v}, 
    \end{align}
    where $g_{1,v}$ is projection of $g_v$ onto $\GL_2(\QQ_v)$ and 
    \begin{equation*}
        f(g_{1,v}, \Phi_{v}, \nu_{v}, s) = \nu_{1}(\det(g_{1,v}))|\det(g_{1,v})|^{s} \int_{\GL_1(\QQ_{v})} \Phi((0, t)g_{1,v}) (\nu_1\nu_2)^{-1}(t) |t|_{v}^{2s} dt. 
    \end{equation*}
\end{definition}

We now recall the definition of the standard automorphic $L$-function attached to $\pi$. 

\begin{definition}[Standard automorphic $L$-function]
\label{def: L-function}
\begin{itemize}
    \item The Langlands dual group of $G$ is $\widehat{G} = \Gm(\CC) \times \GL_3(\CC)$, with an action of the non-trivial element $\sigma \in \Gal(E/\QQ)$ given by 
    \begin{equation*}
        \sigma \colon (x, g) \in \Gm(\CC) \times \GL_3(\CC) \mapsto (x \det g, \Phi_{3}^{-1} {^{t}g^{-1}} \Phi_{3}) \in \Gm(\CC) \times \GL_3(\CC),
    \end{equation*}
    where 
    \begin{equation*}
        \Phi_{3} = \begin{pmatrix}
                        0 & 0 & 1 \\
                        0 & -1 & 0 \\
                        1 & 0 & 0
                   \end{pmatrix}. 
    \end{equation*}
    The $L$-group of $G$ is ${^{L}G = \widehat{G} \rtimes \Gal(E/\QQ)}$. 
    \item Define $G^{\prime} = \Res_{E/\QQ}(\Gm \times \GL_3)$. The Langlands dual group of $G^{\prime}$ is $\widehat{G^{\prime}} = \widehat{G} \times \widehat{G}$ with an action of the non-trivial element $\sigma \in \Gal(E/\QQ)$ given by 
    \begin{equation*}
        \sigma \colon (g, h) \in \widehat{G^{\prime}} \mapsto (\sigma(h), \sigma(g)) \in \widehat{G^{\prime}}, 
    \end{equation*}
    where $\sigma(g)$ and $\sigma(h)$ is the action of $\sigma$ on $\widehat{G}$ defined before. Define the $L$-group of $G^{\prime}$ as ${^{L}G^{\prime}} = \widehat{G^{\prime}} \rtimes \Gal(E/\QQ)$. 
    \item It can be seen from the definition that we have the diagonal embedding of Langlands dual groups $\widehat{G} \rightarrow \widehat{G^{\prime}}$, and we can extend the map to the $L$-groups ${^{L}G} \rightarrow {^{L}G^{\prime}}$. We define the standard representation $\mathrm{std} \colon {^{L}G^{\prime}} \rightarrow \GL_6(\CC)$ as 
    \begin{align*}
        & ((x_1, g_1), (x_2, g_2)) \rtimes 1 \mapsto \begin{pmatrix}
                                             x_1 g_1 & \\
                                                     & x_2 \det(g_2) \Phi_3 {^{t}g_{2}^{-1} \Phi_{3}^{-1}}
                                           \end{pmatrix}, \\
        & 1 \rtimes \sigma \mapsto \begin{pmatrix}
                                     & 1_{3} \\
                                     1_{3} &       \\
                                   \end{pmatrix}. 
    \end{align*}
    \item For unramified $\pi_{p}$, using the Satake transform of Cartier \cite{Cartier79} or Haines\textendash Rostami \cite{Haines-Rostami10}, the well-defined local standard $L$-function $L_{p}(s, \pi_{p}, \mathrm{std})$ is a meromorphic function of $s \in \CC$. 
    \item For a Hecke character $\nu \colon \QQ^{\times} \backslash \AAA^{\times} \rightarrow \CC^{\times}$, we define the twisted standard $L$-function $L_{p}(s, \pi_{p} \times \nu_{p}, \mathrm{std})$ of $\pi \times \nu$ as the standard $L$-function of $(\nu \circ \mu) \pi$, where $(\nu \circ \mu) \pi$ is the product of the representation $\pi$ by the automorphic character $\nu \circ \mu$ of $G$, where $\mu$ is the similitude character of $G$. 
    \end{itemize}
\end{definition}

We now state a result regarding twisted base change that can be used to define $L$-factors at all places. 

\begin{proposition}[\cite{Roga90}, \S 13]
\label{Prop: twisted base change}
    Let $\pi$ be a cuspidal automorphic representation of $G$ over $\QQ$ such that $\pi_{\infty}$ is a discrete series. Then there exists a unique automorphic representation $\tau$ on $\Res_{E/\QQ}(\Gm \times \GL_3)$ that is compatible with $\pi$ at every place $v$ of $\QQ$ such that 
    \begin{itemize}
        \item $v$ splits in $E$, or
        \item $v$ is inert in $E$ and $\pi_v$ is unramified, or 
        \item $v$ is ramified in $E$ and $\pi_ v$ has a vector fixed by $G(\Zp)$. 
    \end{itemize}
\end{proposition}
\begin{remark}
    \begin{itemize}
        \item  The results in \cite{Roga90} are only for the ordinary unitary group. Results for the similitude setting are also constructed in \cite{Morel_Coh10} and \cite{Shin_BC14}. 
        \item Since $\tau$ is uniquely determined by $\pi$, one can use $\tau$ to make a well-defined $L$-factor at \textit{all} places by using the $L$-group representation in Definition \ref{def: L-function} and the known local $L$-parameters of representations of $\Res_{E/\QQ}(\Gm \times \GL_3)$. 
    \end{itemize}
\end{remark}



\begin{proposition} \textnormal{\cite{Gelbart&PS84, Baruch97} \cite[Proposition 3.11.]{PS18}}
\label{Prop:factorize}
With the factorizable data and notation as above,  we have 
\begin{equation*}
    I(\varphi, \Phi, \nu, s) = W_{\varphi}(1) \prod_{v} I_{v}(W_{v}, \Phi_{v}, \nu_v, s). 
\end{equation*}
For all finite places $p$ of $\QQ$, the local integral is absolutely convergent for $\mathrm{Re}(s) \gg 0$ and has meromorphic continuation to a rational function of $p^{-s}$.
If the finite place $p$ is not $2$, $p$ is unramified in $E$, $\pi_{p}$ and $\nu_{p}$ are unramified, $\varphi$ is fixed by the maximal compact subgroup $G(\ZZ_{p})$ and $\Phi_{p}$ is the characteristic function of $V_2(\ZZ_{p})$, then 
\begin{equation*}
    I_{p}(W_{p}, \Phi_{p}, \nu_{p}, s) = L_{p}(s, \pi_{p} \times \nu_{1, p}, \mathrm{std}).
\end{equation*}
If we let $S$ be the completement set of the above finite places, then we have 
\begin{align*}
    I(\varphi, \Phi, \nu, s) &= W_{\varphi}(1) \prod_{p \in S}  I_{p}(W_{p}, \Phi_{p}, \nu_{p}, s) \times I_{\infty}(W_{\infty}, \Phi_{\infty}, \nu_{\infty}, s) \times \prod_{p \notin S, p < \infty} L_{p}(s, \pi_{p} \times \nu_{1, p}, \mathrm{std}) \\
                             &= W_{\varphi}(1) \prod_{p \in S} \frac{I_{p}(W_{p}, \Phi_{p}, \nu_{p}, s)}{L_{p}(s, \pi_{p} \times \nu_{1, p}, \mathrm{std})} 
                             \times I_{\infty}(W_{\infty}, \Phi_{\infty}, \nu_{\infty}, s) \times L(s, \pi \times \nu_{1}, \mathrm{std}).
\end{align*}

\end{proposition}

\subsection{The nonarchimedean local zeta integral}
\label{SS: the nonarch integral}
In this subsection, we recall results about algebraicity of the nonarchimedean local zeta-integral $I_{p}(W_{p}, \Phi_{p}, \nu_{p}, s)$ and the nonarchimedean local $L$-factor $L_{p}(s, \pi_{p} \times \nu_{1, p}, \mathrm{std})$.

\begin{proposition}[\cite{PS18}, Proposition 7.1]
\label{prop: PS18 proposition 7.1}
    Suppose that the Schwartz-Bruhat function $\Phi_p$ on $\Qp^{2}$ and the Whittaker function $W_p$ are both $\overline{\QQ}$-valued. Then the local integral $I_{p}(W_{p}, \Phi_{p}, \nu_{p}, s)$ has a meromorphic continuation to a rational function $\frac{P(X)}{Q(X)}$ of $X = p^{-s}$, where $P$ and $Q$ have algebraic coefficients. \\ 
    In particular, the meromorphic continuation of $I_{p}(W_{p}, \Phi_{p}, \nu_{p}, s)$ to any $s \in \QQ$ is $\overline{\QQ}$-valued if it is finite. 
\end{proposition}

We now address the hypothesis that $W_{p}$ is $\overline{\QQ}$-valued in the previous proposition. In order to use it in the global setting later, we work in the general setting. 

\begin{definition}
\label{def: field of def of a rep}
    Assume that $\pi$ is a representation of a group $G$ on a $\CC$-vector space $V$. We say that $\pi$ is defined over $L \subset \CC$ if there exists a representation $\pi_{0}$ on a $L$-vector space $V_{0}$ such that $V_{0} \otimes_{L} \CC$ is isomorphic to $\pi$. We call $V_{0}$ a model of $V$ over $L$. 
\end{definition}

\begin{proposition}[\cite{HS18}, Proposition 3.3.2] 
\label{prop: rational whittaker}
    Suppose that a representation $\pi$ of a group $G$ is defined over $L$ and let $V$ and $V_{0}$ be as in Definition \ref{def: field of def of a rep}. Assume that $\Lambda\colon V \rightarrow \CC$ is a non-zero functional on $V$ that has a left transformation property with respect to a subgroup $H$ and a character $\psi \colon H \rightarrow L^{\times}$ that characterizes it uniquely up an element of $\CC^{\times}$. (For instance, $\Lambda$ can be a Whittaker functional.) Then there exists a functional $\Lambda_{0} \colon  V \rightarrow \CC$ with $\mathrm{Im}(\Lambda_{0}|_{V_{0}}) \subseteq L$. 
\end{proposition}

\begin{notation}
    Let $\Lambda: \pi_{p} \rightarrow \CC$ be a fixed Whittaker functional of $\pi_p$. If $\pi_{p}$ is defined over a number field, we can assume that $\Lambda$ is $\overline{\QQ}$-valued  by Proposition \ref{prop: rational whittaker}. For any $\varphi \in \pi_{p}$, we define the Whittaker function $W_{\varphi}$ by 
    \begin{equation*}
        W_{\varphi}(g) = \Lambda(g \varphi),
    \end{equation*}
    for any $g \in G(\QQ_{p})$. 
    (Note that this is slightly different from Definition \ref{def: Whittaker function}, which we use since we will vary our choice of $\varphi$.)
\end{notation}

\begin{proposition}
\label{Prop:algebracity}
Let $\pi_{p}$ be a representation of $G(\Qp)$ defined over a number field $L$. For any $\overline{\QQ}$-valued Schwartz-Bruhat function $\Phi_{p}$ on $V_{2}(\Qp)$ and $\varphi_{p}$ in a model of $\pi_p$ over $L$ such that $\Lambda(\varphi_{p}) \neq 0$,  there exists a function $\eta$ in the Hecke algebra $C_{c}^{\infty}(G(\Qp), \QQ)$ such that 
\begin{equation*}
     I_{p}(W_{\pi_{p}(\eta)\varphi_p}, \Phi_p, \nu_p, s) \in \overline{\QQ}^{\times},
\end{equation*}
for any $s \in \QQ$. 
\end{proposition}

\begin{proof}
The algebraicity follows from Proposition \ref{prop: PS18 proposition 7.1} and the nonvanishing is from \cite[Lemma 7.4]{PS18}. 
\end{proof}

\begin{definition}
    For any finite place $p$, we could also define the local $L$-factor $L_{p}(s, \pi_{p} \times \nu_{1, p}, \mathrm{std})$ to be the greatest common divisor of the zeta integrals $I_{p}(W_{p}, \Phi_{p}, \nu_{p}, s)$ as $\Phi_{p}$ and $W_{p}$ vary, whose existence has been proved in \cite{Baruch97}. 
\end{definition}

The following corollary is a direct consequence of Proposition \ref{Prop:algebracity}. 

\begin{corollary}
\label{corollary: local L-factor algebraicity}
The local $L$-factor $L_{p}(s, \pi_{p} \times \nu_{1, p}, \mathrm{std})$ is equal to 
\begin{equation*}
    L_{p}(s, \pi_{p} \times \nu_{1, p}, \mathrm{std}) = \frac{1}{Q(p^{-s})},
\end{equation*}
where $Q(X)$ is a polynomial with $\overline{\QQ}$-coefficients and constant $1$. 
Hence, for any $s \in \QQ$, we have 
\begin{equation*}
    L_{p}(s, \pi_{p} \times \nu_{1, p}, \mathrm{std}) \in \overline{\QQ}^{\times}. 
\end{equation*}
\end{corollary}

\subsection{The archimedean local zeta integral}
\label{SS: arch local integral}
In this subsection, we compute the archimedean zeta integral for some ``nice'' test vectors. 

\begin{convention}
    \begin{itemize}
        \item  Let $\nu_{1, \infty}, \nu_{2, \infty} \colon \RR^{\times} \rightarrow \CC^{\times}$ be the characters defined by 
        \begin{align*}
            & \nu_{1, \infty}(x) = |x|^{-n} \sgn(x)^{n}, \\
            & \nu_{2, \infty}(x) = 1,
        \end{align*}
        and let $\nu_{\infty} = (\nu_{1, \infty}, \nu_{2, \infty})$. 
        \item For $(x, y) \in V_{2}(\RR)$, we let  
              \begin{equation}
                  \Phi_{\infty}(x, y) := (ix - y)^{\frac{n - k}{2}}(ix + y)^{\frac{n + k}{2}} e^{-\pi(x^2 + y^2)}. 
              \end{equation}
        \item Let $\pi_{\infty}$ be a discrete series representation of $G(\RR)^{+}$ with Blattner parameter 
            \begin{equation*}
                (1+s-r+b, -1-a+s-r, s-r). 
            \end{equation*}
        \item Let $\tilde{\pi}_{\infty}$ be the contragredient of $\pi_{\infty}$, which is a discrete series representation with Blattner parameter $\mu = (1 + a + r -s, -1 -b + r - s, r - s)$. 
        \item Let $\Psi_{\infty} \in \tilde{\pi}_{\infty}$ be a vector of weight $(-1 -b + r - s, 1 + a + r - s, r - s; b + 2s -r -2)$, which is the lowest weight vector of the minimal $K_{G}$-type $\tau_{(1 + a + r -s, -1 - b + r  + s, r - s)}$ of $\tilde{\pi}_{\infty}$. 
    \end{itemize}
\end{convention}


\begin{lemma}
\label{Lemma: arch_weight of Phi}
    The function $\Phi_{\infty}(x, y)$ has $\U(1)(\RR)$-weight $-k$. 
\end{lemma}

\begin{proof}
    For any $g = \begin{pmatrix}
                    a & b \\
                    -b & a 
                  \end{pmatrix} \in \U(1)(\RR) \hookrightarrow \GL_2(\RR)$, 
    we have 
    \begin{equation*}
        (x, y) g = (ax - by, bx + ay). 
    \end{equation*}
    We can see from this that: 
    \begin{equation*}
        \Phi_{\infty}((x, y) \begin{pmatrix}
                                a & b \\
                                -b & a
                             \end{pmatrix}) = (a + ib)^{-k} \Phi_{\infty}(x, y). 
    \end{equation*}
    Hence, the function $\Phi_{\infty}(x, y)$ has $\U(1)(\RR)$-weight $-k$. 
\end{proof}

\begin{lemma}
\label{Lemma: arch f formula}
    For any $g = \begin{pmatrix}
                    a & b \\
                    c & d 
                \end{pmatrix} \in \GL_2(\RR)$, the section $f(g, \Phi_{\infty}, \nu_{\infty}, z)$ is equal to 
    \begin{equation*}
        \sgn(\det(g))^{n}|\det(g)|^{z - n} (ic - d)^{\frac{n - k}{2}}(ic + d)^{\frac{n + k}{2}}\pi^{-z}(c^2 + d^2)^{-z} \Gamma(z)
    \end{equation*}
    for $z \in \CC$. 
\end{lemma}

\begin{proof}
     It follows from the identity  $(0, t) \begin{pmatrix}
                                                a & b \\
                                                c & d 
                                          \end{pmatrix} = (ct, dt)$ that 
                            \begin{align*}
                                \Phi_{\infty}(ct, dt) &= (ict - dt)^{\frac{n - k}{2}}(ict + dt)^{\frac{n + k}{2}}e^{-\pi(c^2 + d^2)t^2}\\
                                                      &= (ic - d)^{\frac{n - k}{2}}(ic + d)^{\frac{n + k}{2}}t^ne^{-\pi(c^2 + d^2)t^2},
                            \end{align*}
            so we have 
            \begin{align*}
                f(g, \Phi_{\infty}, \nu_{\infty}, z) &= \sgn(\det(g))^{n}|\det(g)|^{z - n}(ic - d)^{\frac{n - k}{2}}(ic + d)^{\frac{n + k}{2}} \int_{\RR^{\times}}
                e^{-\pi(c^2 + d^2)t^2}  t^{n} |t|^{2z - n} \sgn(t)^{n} \frac{dt}{t} \\
                                                &= 2\sgn(\det(g))^{n}|\det(g)|^{z - n}(ic - d)^{\frac{n - k}{2}}(ic + d)^{\frac{n + k}{2}} \int_{0}^{\infty} e^{-\pi(c^2 + d^2)t^2} t^{2z} \frac{dt}{t}. 
            \end{align*}
            It follows from a change of variables that 
            \begin{equation*}
                2  \int_{0}^{\infty} e^{-\pi(c^2 + d^2)t^2} t^{2z} \frac{dt}{t} = \left(\frac{1}{\pi(c^2 + d^2)}\right)^{z} \Gamma(z), 
            \end{equation*}
            where $\Gamma(z)$ is the Gamma function. 
            So we get
            \begin{equation*}
                f(g, \Phi_{\infty}, \nu_{\infty}, z) = \sgn(\det(g))^{n} |\det(g)|^{z - n} (ic - d)^{\frac{n - k}{2}}(ic + d)^{\frac{n + k}{2}}\pi^{-z}(c^2 + d^2)^{-z} \Gamma(z). 
            \end{equation*}
\end{proof}

\begin{notation}
    \begin{itemize}
        \item If we use the convention in \cite[page 965]{Koseki-Oda95}, we could write the Blattner parameter of $\tilde{\pi}_{\infty}$ as 
             \begin{equation*}
                 \mu = (1 + a, -1 - b). 
             \end{equation*}
        \item Let $d = 2 + a + b$. 
        \item It follows from the Iwasawa decomposition $G(\RR) = U_{B}(\RR)T_{B}(\RR) K_{G}$ that the Whittaker function with a fixed $K_{G}$-type is determined by its value on 
        \begin{equation*}
           T_{B}(\RR) =  \begin{pmatrix}
                            t & 0 & 0 \\
                            0 & 1 & 0 \\
                            0 & 0 & t^{-1}
                        \end{pmatrix}, 
        \end{equation*}
        for all $t \in \RR^{\times}$. 
        \item Recall that we have uniqueness of Whittaker models for $G(\RR)$ (see \cite[\S 3.2]{Koseki-Oda95}). Let $C_{m}^{(1+a, -1-b)}$ be the Whittaker function defined  on $\RR^{\times}$ in \textnormal{\cite[P974]{Koseki-Oda95}} for $0 \le m \le d$, which is associated to the vector $\nu_{d - m}$ in the $K_{G}$-type $\tau_{\mu}$ (see subsubsection \ref{SSS: repn of compact Lie group}). 
    \end{itemize}
\end{notation}

\begin{definition}
\label{Def: classical Whittaker function}
    For an integer $l$, we define the classical Whittaker function $W_{0, l}$ on $\RR^{\times}$ by 
             \begin{equation}
             \label{def:Whittaker_function}
                 W_{0, l}(x) = \frac{e^{-\frac{1}{2}x}}{\Gamma(\frac{1}{2} + |l|)} \int_{0}^{\infty} t^{-\frac{1}{2} + |l|}(1 + \frac{t}{x})^{-\frac{1}{2} + |l|} e^{-t} dt. 
             \end{equation}
\end{definition}

\begin{remark}
    We have the following identity: 
    \begin{equation*}
        W_{0, l}(x) = \pi^{-\frac{1}{2}} x^{1/2} K_{l}(\frac{x}{2}), 
    \end{equation*}
    where $K_{l}(\cdot)$ is the modified $K$-Bessel function. 
\end{remark}

\begin{lemma}
\label{Lemma: relation C and W}
    The function $C_{m}^{(1+a, -1-b)}$ is related to a classical Whittaker function by
             \begin{equation*}
                 C_m^{(1+a, -1-b)}(t) = (-i)^{2 + a + b -m}t^{a + b + \frac{7}{2}} W_{0, m-(1+a)}(8\sqrt{2} \pi D^{-\frac{3}{4}}t), t \in \RR^{\times}
             \end{equation*}
\end{lemma}

\begin{proof}
    In this section, we use the Hermitian form $J$ to define the unitary group, but in \cite{Koseki-Oda95} the Hermitian form $J^{\prime}$ is used. The corresponding transform is 
    \begin{equation*}
        g \in \GU(J^{\prime})(\RR) \mapsto CgC^{-1} \in \GU(J)(\RR), 
    \end{equation*}
    where $C = \begin{pmatrix}
                  \frac{D^{\frac{1}{4}}}{\sqrt{2}} & 0 & \frac{D^{\frac{1}{4}}}{\sqrt{2}} \\
                   0 & 1 & 0 \\
                   -\frac{D^{\frac{1}{4}}}{\sqrt{-2}} & 0 & \frac{D^{\frac{1}{4}}}{\sqrt{-2}}
              \end{pmatrix}$.
    \par Recall that in the setup of \cite{Koseki-Oda95}, $E_{2, +} = \begin{pmatrix}
                                                                     0 & -1 & 0 \\
                                                                     1 & 0 & -1 \\
                                                                     0 & -1 & 0
                                                                    \end{pmatrix}$ and $E_{2, -} = \begin{pmatrix}
                                                                                                 0 & -i & 0 \\
                                                                                                 -i & 0 & i \\
                                                                                                 0 & -i & 0
                                                                                                \end{pmatrix}$.
    So in our setting, we have 
    \begin{align*}
        & E_{2, +} = C \begin{pmatrix}
                          0 & -1 & 0 \\
                          1 & 0 & -1 \\
                          0 & -1 & 0
                     \end{pmatrix} C^{-1} = \begin{pmatrix}
                                                0 & -\sqrt{2}D^{\frac{1}{4}} & 0 \\
                                                0 & 0 & -\sqrt{-2}D^{-\frac{1}{4}} \\
                                                0 & 0 & 0
                                             \end{pmatrix}, \\
         & E_{2, -} = C \begin{pmatrix}
                          0 & -i & 0 \\
                          -i & 0 & i \\
                          0 & -i & 0
                     \end{pmatrix} C^{-1} = \begin{pmatrix}
                                                0 & -\sqrt{2}D^{\frac{1}{4}} & 0 \\
                                                0 & 0 & -\sqrt{2}D^{-\frac{1}{4}} \\
                                                0 & -\sqrt{2}D^{\frac{1}{4}} & 0
                                             \end{pmatrix}.                                    
    \end{align*}
    It follows from Definition \ref{def: Whittaker function} that 
    \begin{align*}
        & \chi(E_{2, +}) = \psi(\Tr_{E/\QQ}(\delta^{-1}(-\sqrt{-2}D^{-\frac{1}{4}}))) = -4\pi{i}\sqrt{2}D^{-\frac{3}{4}}, \\
        & \chi(E_{2, +}) = \psi(\Tr_{E/\QQ}(\delta^{-1}(-\sqrt{2}D^{-\frac{1}{4}}))) = 0. 
    \end{align*}
    \par Plugging these into the formulas in \cite[Theorem (4.5), Page 977]{Koseki-Oda95}, we have 
    \begin{align*}
        & \eta_{\pm} = \chi(E_{2, +}) \pm \sqrt{-1} \chi(E_{2, -}) = -4\sqrt{2} \pi{i} D^{-\frac{3}{4}}, \\
        & b = -\eta_{+}\eta_{-} = 32\pi^{2} D^{-\frac{3}{2}}, \gamma_m = (-i)^{2 + a + b - m},
    \end{align*}
    and 
    \begin{equation*}
           C_m^{(1+a, -1-b)}(t) = (-i)^{2 + a + b -m}t^{a + b + \frac{7}{2}} W_{0, m-(1+a)}(8\sqrt{2} \pi D^{-\frac{3}{4}}t). 
    \end{equation*}
\end{proof}


\begin{lemma}
\label{Lemma:Mellin_Trans_Whittaker}
        The Mellin transform of the Whittaker function $C_{m}^{(1+a, -1-b)}$ at $l \in \CC$ is 
              \begin{align*}
                  \int_{0}^{\infty} C_{m}^{(1+a, -1-b)}(t) t^{l} d^{\times}t &= (-i)^{2 + a + b - m} \frac{1}{2} (\frac{1}{2\sqrt{2}D^{-\frac{3}{4}}})^{l + a + b + \frac{7}{2}}\pi^{-(l + a + b + 4)} \Gamma(\frac{1}{2}(l + m + b + 3) ) \\
                  & \times \Gamma(\frac{1}{2}(l + b - m + 5) + a). 
              \end{align*}
\end{lemma}

\begin{proof}
     It follows from Lemma \ref{Lemma: relation C and W} that 
     \begin{equation*}
         \int_{0}^{\infty} C_{m}^{(1 + a, -1 - b)}(t) t^{l} d^{\times}t = (-i)^{2 + a + b - m} \int_{0}^{\infty} t^{l + a + b + \frac{7}{2}} W_{0, m-(1 + a)}(8\sqrt{2} \pi D^{-\frac{3}{4}}t) d^{\times}t.  
     \end{equation*}
     If we change variables by $t \mapsto \frac{t}{8\sqrt{2}\pi D^{-\frac{3}{4}}}$, then this integral is 
     \begin{equation*}
         (-i)^{2 + a + b - m} (\frac{1}{8\sqrt{2}\pi D^{-\frac{3}{4}}})^{l + a + b + \frac{7}{2}}  \int_{0}^{\infty} t^{l + a + b + \frac{7}{2}} W_{0, m-(1 + a)}(t) d^{\times}t. 
     \end{equation*}
     Using the formula in \cite[\S 5.2, P979]{Koseki-Oda95}, we find it is equal to 
     \begin{equation*}
         (-i)^{2 + a + b - m} \frac{1}{2} (\frac{1}{2\sqrt{2}D^{-\frac{3}{4}}})^{l + a + b + \frac{7}{2}}\pi^{-(l + a + b + 4)} \Gamma(\frac{1}{2}(l + m + b + 3) ) \Gamma(\frac{1}{2}(l + b - m + 5) + a). 
     \end{equation*}
\end{proof}

Now we compute the archimedean zeta integral with specific chosen test vectors. 

\begin{convention}
    \begin{itemize}
        \item Let $\Psi_{\infty} \in \tilde{\pi}_{\infty}$ be a vector of weight $(-1 - b + r - s, 1 + a + r - s, r - s; b + 2s - r)$, which is the lowest weight vector of the minimal $K_{G}$-type $\tau_{\mu}$.
        \item We write the basis $\nu_0, \nu_1, \ldots, \nu_d$ in the $K_{G}$-type $\mu$ defined in subsubsection \ref{SSS: repn of compact Lie group} as $\nu_0^{\mu}, \nu_1^{\mu}, \ldots, \nu_d^{\mu}$. 
        \item We identify $\Psi_{\infty}$ with the vector $\nu_{0}^{\mu}$, so $X^{m}_{(1, -1, 0)}\Psi_{\infty}$ is $m! \nu_{m}^{\mu}$. 
        \item By the uniqueness of Whittaker models (see \cite[\S 3.2]{Koseki-Oda95}), we choose $W_{\infty}$ to be the Whittaker function corresponding to $X_{(1, -1, 0)}^{m} \Psi_{\infty}$ such that $W_{\infty}(1) = 1$.
        \item Because $W_{\infty}(1) = 1$ and $C_{d - m}^{(1 + a, -1-b)}(1) = (-i)^{m}W_{0, 1 + b - m}(8\sqrt{2}\pi D^{-\frac{3}{4}})$ (see Lemma \ref{Lemma: relation C and W}), it can be seen from uniqueness of Whittaker models that if we let 
        \begin{equation*}
            C = (-i)^{m}W_{0, 1 + b - m}(8\sqrt{2}\pi D^{-\frac{3}{4}}),
        \end{equation*}
        then we have 
        \begin{equation}
        \label{eq: infinite Whittaker and C}
            W_{\infty}(t) = C^{-1} C_{d - m}^{(1+a, -1-b)}(t),
        \end{equation}
        for all $t \in \RR^{\times}$\footnote{Here we view $W_{\infty}$ as a function on $\RR^{\times}$.}. 
        \item Let $K_{\infty} = \U(1)(\RR)$.
    \end{itemize}
    
\end{convention}

\begin{lemma}
\label{Lemma: integral I1}
If $m = 1+ a + r - s$, then the weight of $W_{\infty}$ is 
\begin{equation*}
(a - b + 2(r - s), 0, r-s).
\end{equation*}
In this case, the integral 
\begin{equation*}
    I_1 = \int_{\GL_1(\RR)} t^{2z - n - 2} W_{\infty}( \begin{pmatrix}
                                                                        t & 0 & 0 \\
                                                                        0 & 1 & 0 \\
                                                                        0 & 0 & t^{-1} 
                                                                      \end{pmatrix}) dt
\end{equation*}
is equal to 
\begin{align*}
    C^{-1} (-i)^{1 + a - r - s} (\frac{1}{2\sqrt{2}D^{-\frac{3}{4}}})^{(2z -r - s + \frac{3}{2})}\pi^{-(2z - r - s + 2)}\Gamma(z - r + 1 + \frac{1}{2}(b - a)) \Gamma(z - s + 1 - \frac{1}{2} (a + b)). 
\end{align*}
    
\end{lemma}

\begin{proof}
    First, we have 
            \begin{align*}
                I_1 &= \int_{\GL_1(\RR)} t^{2z - n - 2} W_{\infty}( \begin{pmatrix}
                                                                        t & 0 & 0 \\
                                                                        0 & 1 & 0 \\
                                                                        0 & 0 & t^{-1} 
                                                                      \end{pmatrix}) dt \\
                    &= \int_{\RR^{\times}_{>0}} t^{2z - n - 2} W_{\infty}( \begin{pmatrix}
                                                                        t & 0 & 0 \\
                                                                        0 & 1 & 0 \\
                                                                        0 & 0 & t^{-1} 
                                                                      \end{pmatrix}) dt + \int_{\RR^{\times}_{<0}} t^{2z - n - 2} W_{\infty}( \begin{pmatrix}
                                                                        t & 0 & 0 \\
                                                                        0 & 1 & 0 \\
                                                                        0 & 0 & t^{-1} 
                                                                      \end{pmatrix}) dt. 
            \end{align*}
        By a change of variables and the fact that $W_{\infty}$ has weight $(a - b + 2(r - s), 0, r-s)$, we get 
        \begin{equation*}
            \int_{\RR^{\times}_{>0}} t^{2z - n - 2} W_{\infty}( \begin{pmatrix}
                                                                        t & 0 & 0 \\
                                                                        0 & 1 & 0 \\
                                                                        0 & 0 & t^{-1} 
                                                                            \end{pmatrix}) dt =  \int_{\RR^{\times}_{<0}} t^{2z - n - 2} W_{\infty}( \begin{pmatrix}
                                                                                    t & 0 & 0 \\
                                                                                    0 & 1 & 0 \\
                                                                                    0 & 0 & t^{-1} 
                                                                                  \end{pmatrix}) dt. 
        \end{equation*}
        Hence, we get 
        \begin{equation*}
            I_1 = 2  \int_{\RR^{\times}_{>0}} t^{2z - n - 2} W_{\infty}( \begin{pmatrix}
                                                                        t & 0 & 0 \\
                                                                        0 & 1 & 0 \\
                                                                        0 & 0 & t^{-1} 
                                                                        \end{pmatrix}) dt, 
        \end{equation*}
        which is a Mellin transform of a Whittaker function. 
        \par Second, it follows from equation (\ref{eq: infinite Whittaker and C}) that 
        \begin{equation*}
            I_1 = 2 C^{-1} \int_{0}^{\infty} t^{2z - n - 2} C_{1 + b - r + s}^{(1+a, -1-b)}(t) d^{\times}t. 
        \end{equation*}
        \par Finally, using the formula in Lemma \ref{Lemma:Mellin_Trans_Whittaker}, we have 
        \begin{align*}
             I_1 = C^{-1} (-i)^{1 + a - r - s} (\frac{1}{2\sqrt{2}D^{-\frac{3}{4}}})^{(2z -r - s + \frac{3}{2})}\pi^{-(2z - r - s + 2)}\Gamma(z - r + 1 + \frac{1}{2}(b - a)) \Gamma(z - s + 1 - \frac{1}{2} (a + b)). 
        \end{align*}
\end{proof}

\begin{proposition}
\label{Prop:Arch_Zeta}
        \begin{enumerate}
            \item If $m \neq 1 + a + r - s$,  then the archimedean local zeta integral 
                    \begin{equation*}
                        I_{\infty}(W_{\infty}, \Phi_{\infty}, \nu_{\infty}, z) = \int_{Z(\RR)U_{2}(\RR) \backslash H(\RR)} f(g_{1, \infty}, \Phi_{\infty}, \nu_{\infty}, z) W_{\infty}(g_{\infty}) dg_{\infty}
                    \end{equation*} is equal to $0$. 
            \item If $m = 1 + a + r - s$, then the zeta integral $I_{\infty}(W_{\infty}, \Phi_{\infty}, \nu_{\infty}, 1 + a + b + r + s)$ is only non-zero when the $K_{\infty}$-type of $\Phi_{\infty}$ is $-(a + r -b - s)$.  In this case,  $I_{\infty}(W_{\infty}, \Phi_{\infty}, \nu_{\infty}, 1 + a + b + r + s)$ is equal to
            \begin{align*}
                &  (-1)^{a} \pi^{-(3a + 3b + 2r + 2s + 5)} W_{0, b + s - a - r}(8\sqrt{2}\pi D^{-\frac{3}{4}})^{-1}\left(\frac{1}{2\sqrt{2}D^{-\frac{3}{4}}}\right)^{2a + 2b + r + s + \frac{7}{2}} \\
                & \times \Gamma(a + b + r + s + 1) \Gamma\left( \frac{1}{2}a + \frac{3}{2}b + s + 2\right) \Gamma\left(\frac{1}{2}a + \frac{1}{2} b + r + 2\right), 
            \end{align*}
            where $W_{0,b + s - a - r}$ is the classical Whittaker function on $\RR$ defined in Definition \ref{Def: classical Whittaker function} and $\Gamma$ is the Gamma function. 
        \end{enumerate}
\end{proposition}
\begin{proof}
   Recall that $H = \GL_2 \boxtimes E^{\times} $ and $Z(\RR) = \left \{(\begin{pmatrix}
                                                        z & 0 \\
                                                        0 & z \\ 
                                                    \end{pmatrix}, z) | z \in \RR \right \}$, so it follows from the Iwasawa decomposition that  
    \begin{equation*}
        Z(\RR) U_2(\RR) \backslash H(\RR) / K_{\infty} \cong \GL_1(\RR) \times \mathbb{S}^{1} = \left \{ (\begin{pmatrix}
                                                                                                    t & 0 \\
                                                                                                    0 & t^{-1} 
                                                                                                \end{pmatrix}, c) | t \in \GL_1(\RR), c \in \mathbb{S}^{1} \right \}, 
    \end{equation*}
    where $\mathbb{S}^{1}$ is the unit circle $\{ z \in \CC | z\bar{z} = 1 \}$. There is an embedding 
   $\iota \colon \GL_1(\RR) \times \mathbb{S}^{1} \hookrightarrow G(\RR)$ that maps $(\begin{pmatrix}
                                                                                    t & 0 \\
                                                                                    0 & t^{-1} 
                                                                                 \end{pmatrix}, c) $ to $\left(\begin{matrix}
                                                                                                            t & 0 & 0 \\
                                                                                                            0 & c & 0 \\
                                                                                                            0 & 0 & t^{-1} 
                                                                                                            \end{matrix}   \right)$. 
    \par Since the $K_{\infty}$-weight of $\Phi_{\infty}$ is $-k$ and the $K_{\infty}$-weight of $W_{\infty}$ is $m - 1 - b$, we have
    \begin{equation*}
        I_{\infty}(W_{\infty}, \Phi_{\infty}, \nu_{\infty}, z) = 0
    \end{equation*}
    unless $k = m - 1 -b$. In this case, the action of $K_{\infty}$ on $f(g_{1, \infty}, \Phi_{\infty}, \nu_{\infty}, z)W_{\infty}(g_{\infty})$ is trivial, so
    \begin{align}
        & I_{\infty}(W_{\infty}, \Phi_{\infty}, \nu_{\infty}, z) \\ = & \int_{Z(\RR)U_{2}(\RR) \backslash H(\RR)} f(g_{1, \infty}, \Phi_{\infty}, \nu_{\infty}, z) W_{\infty}(g_{\infty}) dg_{\infty} \\ 
        = & \pi^{-z}\Gamma(z) \int_{\GL_1(\RR) \times \mathbb{S}^{1}} (-t^{-1})^{\frac{n-k}{2}} (t^{-1})^{\frac{n+k}{2}} t^{2z} W_{\infty}(\iota( \begin{pmatrix}
            t & 0 \\
            0 & t^{-1} 
        \end{pmatrix}, c)) |t|^{-2} dt dc  \\
        = & \pi^{-z}\Gamma(z) (-1)^{\frac{n - k}{2}}  \int_{\GL_1(\RR) \times \mathbb{S}^{1}}t^{2z - n - 2} W_{\infty}( \begin{pmatrix}
                                                                                                                            t & 0 & 0 \\
                                                                                                                            0 & c & 0 \\
                                                                                                                            0 & 0 & t^{-1} 
                                                                                                                        \end{pmatrix}) dt dc. \\ 
        \label{eqn: Arch_Zeta_Int} = & \pi^{-z}\Gamma(z) (-1)^{\frac{n - k}{2}} \int_{\GL_1(\RR)} t^{2z - n - 2} W_{\infty}( \begin{pmatrix}
                                                                                                    t & 0 & 0 \\
                                                                                                    0 & 1 & 0 \\
                                                                                                    0 & 0 & t^{-1} 
                                                                                                  \end{pmatrix}) dt, 
         \times \int_{\mathbb{S}^{1}} c^{1 + a - m + r - s} dc                                                                                                                                                                      
    \end{align}
    where the last identity holds since
    \begin{equation*}
        W_{\infty}(\begin{pmatrix}
                     t & 0 & 0 \\
                     0 & c & 0 \\
                     0 & 0 & t^{-1} 
                    \end{pmatrix}) =  W_{\infty}(\begin{pmatrix}
                                                     t & 0 & 0 \\
                                                     0 & 1 & 0 \\
                                                     0 & 0 & t^{-1} 
                                                  \end{pmatrix} \begin{pmatrix}
                                                                     1 & 0 & 0 \\
                                                                     0 & c & 0 \\
                                                                     0 & 0 & 1
                                                                \end{pmatrix}) = c^{1 + a - m + r - s} W_{\infty}(\begin{pmatrix}
                                                                                                                     t & 0 & 0 \\
                                                                                                                     0 & 1 & 0 \\
                                                                                                                     0 & 0 & t^{-1} 
                                                                                                                    \end{pmatrix}). 
    \end{equation*}
    Now we set $I_1 = \int_{\GL_1(\RR)} t^{2z - n - 2} W_{\infty}( \begin{pmatrix}
                                                                        t & 0 & 0 \\
                                                                        0 & 1 & 0 \\
                                                                        0 & 0 & t^{-1} 
                                                                      \end{pmatrix}) dt$ and $I_2 = \int_{\mathbb{S}^{1}} c^{1 + a - m + r - s} dc$, 
    and will compute the integral $I_1$ and $I_2$, respectively. 

    \par The integral $I_2$ is computed as follows. 
         We have 
         \begin{align*}
             I_2 & = \int_{\mathbb{S}^{1}} c^{1 + a - m + r - s} dc \\
                 & = \int_{0}^{1} e^{2\pi{i}(1 + a - m + r - s)x} dx \\
                 & = \begin{cases}
                        1 & \text{If}\ m = 1+ a + r - s, \\
                        0 & \text{If}\ m \neq 1+ a + r - s. 
                     \end{cases}
         \end{align*}
    When $m = 1 + a + r - s$, the integral $I_1$ is computed in Lemma \ref{Lemma: integral I1}. 
    
        \par Finally, plugging the formulas for $I_1$ and $I_2$ into equation (\ref{eqn: Arch_Zeta_Int}), we have: 
        \begin{itemize}
            \item if $m \neq 1 + a + r - s$, then
                  \begin{equation*}
                      I_{\infty}(W_{\infty}, \Phi_{\infty}, \nu_{\infty}, z) = 0; 
                  \end{equation*}
            \item if $m = 1 + a + r - s$, then 
            \begin{align*}
                & I_{\infty}(W_{\infty}, \Phi_{\infty}, \nu_{\infty}, z) \\ = & (-1)^{a + r} C^{-1} (-i)^{1 + a - r - s} (\frac{1}{2\sqrt{2}D^{-\frac{3}{4}}})^{(2z -r - s + \frac{3}{2})}\pi^{-(3z - r - s + 2)} \\ & \times \Gamma(z - r + 1 + \frac{1}{2}(b - a)) \Gamma(z - s + 1 - \frac{1}{2} (a + b)) \Gamma(z).
            \end{align*}
            Plugging $z = 1 + a + b + r + s$ and $C = (-i)^{1 + a + r - s}W_{0, b + s - a - r}(8\sqrt{2}\pi D^{-\frac{3}{4}})$ into the formula, we have
            \begin{align*}
                &  I_{\infty}(W_{\infty}, \Phi_{\infty}, \nu_{\infty}, 1 + a + b + r + s) \\
                = (-1)^{a}& \pi^{-(3a + 3b + 2r + 2s + 5)} W_{0, b + s - a - r}(8\sqrt{2}\pi D^{-\frac{3}{4}})^{-1}\left(\frac{1}{2\sqrt{2}D^{-\frac{3}{4}}}\right)^{2a + 2b + r + s + \frac{7}{2}} \\
                \times &\Gamma(a + b + r + s + 1) \Gamma( \frac{1}{2}a + \frac{3}{2}b + s + 2) \Gamma(\frac{1}{2}a + \frac{1}{2} b + r + 2). 
            \end{align*}
        \end{itemize}
    \end{proof}

\begin{remark}
\label{remark: arch_zeta_integral}
    \begin{enumerate}
        \item If we let $m = 1 + a + r - s = 1 + a + 2b -2j + r + s$ as in Proposition \ref{prop: explicit_pairing}, then $j = b + s$. So we can see from Corollary \ref{corollary: identify integral of Xi and I} only one term in the summation of equation (\ref{eq:pairing}) is left. So the pairing $\langle \omega_{\Psi},  [\rho] \rangle_{B}$ is Eulerian by our previous discussion of zeta integral. 
        \item The integer $j = b + s$ satisfies the condition $A \le j \le B$ in Proposition \ref{prop: explicit_pairing}. 
    \end{enumerate}
 
\end{remark}



