\newpage

%=====================================================================
% 8. Global balance of exponents  (revised version)
%=====================================================================
\section{Global balance of exponents}\label{sec:balance}

We assemble the local estimates from \ref{sec:broad3}--\ref{sec:narrow} and show that the cumulative
exponents in~$\lambda$ and $D=\lambda^{1/12}$ are negative. Hence the
traditional factor $\lambda^{\varepsilon}$ is not needed, and the final result remains
$\varepsilon$-free.\footnote{The contribution of each block is taken exactly in the versions
recorded in §§\,\ref{sec:kernel}--\ref{sec:narrow}. The \emph{Robust Kakeya} (high angular density)
and \emph{Tube packing} regimes are used mutually exclusively.}

%--------------------------------------------------------------------
\subsection{Summary table}\label{subsec:balance-table-final}
%--------------------------------------------------------------------
\[
  \sigma_{\lambda} := \sum_{\text{blocks}}\!\!\bigl(\text{exponent in }\lambda\bigr),
  \qquad
  \sigma_{D} := \sum_{\text{blocks}}\!\!\bigl(\text{exponent in }D\bigr).
\]

\begin{table}[h] \label{tab:global-balance}
\centering
\caption{Summary balance of exponents (regime \S\ref{sec:tube-pack} \emph{off})}
\begin{tabular}{lcc}
\toprule
Block & \(\lambda\)–exponent & \(D\)–exponent \\
\midrule
Broad geometry (Broad--3) & \(+\tfrac{5}{36}\) & \(0\)\textsuperscript{\(*\)} \\
Kernel (12 IBP $+$ Schur $+$ $TT^{*}$) & \(-\tfrac{9}{2}\) & \(-3\) \\
Robust\,Kakeya (threshold $>c_{*}D$) & \(+\tfrac{1}{12}\) & \(+1\) \\
Algebraic “shell” & \(-\tfrac{1}{12}\) & \(-1\) \\
Narrow cascade (double $7/8$ $+$ globalization) & \(-\tfrac{5}{64}\) & \(0\) \\
\midrule
\(\sigma_{\lambda}\) & \(-\tfrac{2557}{576}\approx-4.44\) & --- \\
\(\sigma_{D}\)       & --- & \(-3\) \\
\bottomrule
\end{tabular}

\smallskip
\raggedright\textsuperscript{\(*\)}The $D$–factor from the Cauchy–Schwarz step (via \( \|F_m\|_{L^2}\)) is entirely
attributed to the Robust–Kakeya block; see the remark in §\ref{sec:broad3}.
\end{table}

\noindent\emph{For reference: the scenario with \S\ref{sec:tube-pack} \textbf{instead of} Robust\,Kakeya.}
By the mutual exclusivity rule we remove the RK contribution and insert that of \S\ref{sec:tube-pack}:
\[
  \sigma_\lambda \;\mapsto\; -\frac{2557}{576}\;-\;\frac{7}{6}\;-\;\frac{1}{12}
  \;=\; -\frac{3277}{576}\approx -5.69,
  \qquad
  \sigma_D \;\mapsto\; (-3)\;-\;1\;+\;\frac14 \;=\; -\frac{15}{4}.
\]
Both versions yield negative sums; the option “add \S\ref{sec:tube-pack} on top of RK’’
is methodologically not used. % See also the tabular summary: \S\ref{sec:broad3}, \S\ref{sec:kernel}, \S\ref{sec:kakeya}, \S\ref{sec:narrow}.

%--------------------------------------------------------------------
\subsection{Final $\varepsilon$–free estimate}\label{subsec:main-theorem}
%--------------------------------------------------------------------
\begin{theorem}\label{thm:main}
Let $F=\sum_{\Theta} F_{\Theta}$ be a cap decomposition of radius
$r=\lambda^{-2/3}$ ($\lambda\ge2$), and let the cylinder
$Q_{\lambda}\subset\mathbb{R}_{t}\times\mathbb{R}^{3}_{x}$ be given by~\eqref{eq:def-cylinder}.
Denote by $\mathcal I=\{\Theta:\,\|F_\Theta\|_{L^6(Q_\lambda)}\neq0\}$ the set of active caps.
Assume that
\[
  \#\mathcal I \ \ge\ c_\star\,\lambda^{4/3}\quad\text{\emph{(“many caps’’ regime).}}
\]
Then
\[
  \|F\|_{L^{6}(Q_{\lambda})}
  \;\le\;
  C_{\varepsilon}\,
  \lambda^{\sigma_{\lambda}}\,
  D^{\sigma_{D}}\,
  \Bigl(\sum_{\Theta}\|F_{\Theta}\|_{L^{6}(Q_\lambda)}^{2}\Bigr)^{1/2},
  \qquad
  \sigma_\lambda=-\frac{2557}{576}\approx-4.44,\quad
  \sigma_D=-3,
\]
where $D=\lambda^{1/12}$. Without rounding:
\[
  \|F\|_{L^6(Q_\lambda)} \;\lesssim_\varepsilon\;
  \lambda^{\sigma_\lambda+\varepsilon}\,D^{\sigma_D+\varepsilon}
  \Bigl(\sum_{\Theta}\|F_\Theta\|_{L^6(Q_\lambda)}^2\Bigr)^{1/2}.
\]
\end{theorem}

where $\varepsilon>0$ is fixed, and the factors $\lambda^\varepsilon$, $D^\varepsilon$
arise from replacing logarithmic losses by power losses with exponent $\varepsilon$
as in \S\ref{subsec:log-budget}.

\begin{remark}[Branching by the number of caps]
The “many caps’’ condition is placed in the hypothesis of Theorem~\ref{thm:main}.
The small number of caps case is covered by Lemma~\ref{lem:local-small-caps} below; taken together,
the two regimes yield a global estimate on $Q_\lambda$.
Sequential application of the blocks in \S\ref{subsec:balance-table-final}
gives the exponents $\sigma_\lambda,\sigma_D$ in the “large’’ regime; the accounting of $\lambda^{\varepsilon}$, $D^{\varepsilon}$
is as indicated above.
\end{remark}

\begin{lemma}[Local $L^6$ estimate for a small number of caps]
\label{lem:local-small-caps}
Let $F=\sum_{\Theta\in\mathcal{I}}F_\Theta$, where 
$\#\mathcal{I} \le c_\star\,\lambda^{4/3}$.
Then
\[
  \|F\|_{L^6(Q_\lambda)}
  \ \lesssim\
  \Bigl(\sum_{\Theta\in\mathcal{I}}\|F_\Theta\|_{L^6(Q_\lambda)}^2\Bigr)^{1/2}.
\]
\end{lemma}

\begin{proof}
Cover $Q_\lambda$ by a fixed number of anisotropic blocks,
and use bounded overlap of the contributions of $F_\Theta$ to obtain
\[
\Big\| \sum_{\Theta\in\mathcal{I}} f_\Theta\Big\|_{L^6} 
 \lesssim \Big(\sum_{\Theta\in\mathcal{I}} \|f_\Theta\|_{L^6}^2\Big)^{1/2}.
\]
Details are standard: almost-orthogonality in $\Theta$ in $L^6$ on the window $Q_\lambda$.
\end{proof}

\begin{remark}[Reference on a modification of the kernel block]
If in the kernel block (§\,\ref{sec:kernel}) one uses five integrations in $x'$
instead of six, then
\[
   \|K\|_{L^{2}\to L^{2}}\ \lesssim\ \lambda^{-25/6}D^{-5/2},
\]
and the exponent $\sigma_{\lambda}$ increases by $+\tfrac{1}{3}$,
remaining negative. The statement of
Theorem~\ref{thm:main} remains unchanged.
\end{remark}
