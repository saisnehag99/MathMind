\subsection{Motivation}
\label{SS: motivation}
The analytic class number formula is a deep relationship between an analytic invariant and an arithmetic invariant of a number field $F$. It relates the leading Taylor coefficient at $0$ of the Dedekind zeta-function $\zeta_{F}(s)$ of $F$ to the Dirichlet regulator 
\begin{equation*}
    r\colon O_F^{\times} \rightarrow \RR^{r_1 + r_2},
\end{equation*}
where $O_F^{\times}$ is the unit group and $r_1$ (resp., $r_2$) is the number of real (resp., complex) places of the number field $F$.  In the 1980s, Beilinson made deep conjectures about special values of motivic $L$-functions generalizing the classical analytic class formula \cite{Beilinson85}. In order to do so, Beilinson replaced the units $O_{K}^{\times}$ by the motivic cohomology $\mathrm{H}^{i}_{M}(X, \QQ(n))$ where $X$ is a smooth $\QQ$-scheme and replaced the Dirichlet regulator by the higher regulator map 
\begin{equation*}
    r_{H}\colon \mathrm{H}^{i}_{M}(X, \QQ(n)) \rightarrow \mathrm{H}_{H}^{i}(X, \RR(n))
\end{equation*}
from motivic cohomology to absolute Hodge cohomology. For an introduction to Beilinson's conjectures, the interested reader may consult \cite{Nekovar94}. 

\par In this paper, we establish a connection between motivic classes in motivic cohomology with non-trivial coefficients of Picard modular surfaces and the motivic $L$-functions of some cuspidal automorphic representations of the unitary similitude group $\GU(2, 1)$, through which we show that the motivic classes are non-trivial. It answers a question raised in \cite{LSZ22}, in which the authors constructed an Euler system for $\GU(2, 1)$ from the motivic classes. Along the way, we also prove a vanishing on the boundary result for the motivic classes. We hope this work could shed some light on proving more general relationships between special values of motivic $L$-functions associated to automorphic representations and mixed motives. 

\subsection{Statement of the main results}

Now we introduce the main results. These can be divided into two parts. In the first part, we construct motivic classes $\mathcal{E}is_{M}(\phi_f)$ in the motivic cohomology with non-trivial coefficients of Picard modular surfaces $\mathrm{H}^{3}_{M}(S, V(2))$ and then we prove the classes are in a ``nice" subspace of the cohomology.  Along the way, we also prove the regulators of the classes $\mathcal{E}is_{H}(\phi_f)$ are in some ``nice" subspace of the absolute Hodge cohomology $\mathrm{H}^{3}_{M}(S, V(2))$. In the second part, we prove a relation between the image of the motivic class and a non-critical value of the motivic $L$-function associated to some ``nice" irreducible cuspidal automorphic representation $\pi$. The relation resembles Beilinson's conjectures on special values of $L$-functions.

\par Now, we state the results more precisely. Let $G = \GU(2, 1)$ and let $H$ be a subgroup of $G$ that can be embedded into $G$ and is an extension of the group $\GL_2$. You can find the precise definition of $H$ in Definition \ref{defn:G_H}. The relation between the groups $\GL_2$, $H$ and $G$ can be summarized in  following diagram: 
\[
    \begin{tikzcd}
        H \ar[r, hook] \ar[d, twoheadrightarrow] & G  \\
        \GL_2  &  
    \end{tikzcd}. 
\]
These maps of algebraic groups induce morphisms of the corresponding Shimura varieties\footnote{In the introduction, we ignore the level structures. Rigorously speaking, $S$ (resp., $M$) is defined to be the Weil restriction to $\QQ$ of $\Sh_{G}$ (resp., $\Sh_{H}$) (see Convention \ref{Convention: M_{Q} and S_{Q}} for more details). But in the introduction, we ignore this technical detail.} : 
\[
    \begin{tikzcd}
        M = \Sh_H \ar[r, hook, "\iota"] \ar[d, rightarrow, "p"] & S=\Sh_G  \\
        \Sh_{\GL_2}  &  
    \end{tikzcd}. 
\]
Let $V$ be an algebraic representation of $G$ (see Definition \ref{def:rep_G}), $V_2$ be the standard representation of $\GL_{2}$ and $W$ be an algebraic representation of $H$ (see Definition \ref{def:repn_H}). They correspond to motivic sheaves $V$ on $\Sh_{G}$, $V_2$ on $\Sh_{\GL_2}$ and $W$ on $\Sh_{H}$ (see Lemma \ref{lemma: Ancona} and Proposition \ref{Prop: Jin}). Then we have the corresponding morphisms of motivic cohomologies: 
\[
    \begin{tikzcd}
        \mathrm{H}_{M}^{1}(M, W(1)) \ar[r, "\iota^{*}"] & H^{3}_{M}(S, V(2))  \\
        \mathrm{H}_{M}^{1}(\Sh_{\GL_2}, \Sym^{n}V_2(1))  \ar[u, "p^{*}"] &  
    \end{tikzcd}. 
\]

Beilinson constructed the \emph{Eisenstein symbol} \cite[\S 3]{Beilinson_Modular_Curve}. For each nonnegative integer $n$, the Eisenstein Symbol is a $\QQ$-linear map
\[
    Eis^{n}_{M} \colon \mathcal{B}_{n} \rightarrow \mathrm{H}^{1}_{M}(\Sh_{\GL_2},\mathrm{Sym}^{n} V_2(1)), 
\]
where the source $\mathcal{B}_n$ is the space of locally constant $\QQ$-valued functions on $\GL_2(\AAA_{f})$ satisfying some equivariant properties (see Section \ref{SS: Eis symbol and Hodge realization}). 

\par If we compose the Eisenstein symbol with the pullback $p^{*}\colon \mathrm{H}^{1}_{M}(M,\mathrm{Sym}^{n} V_2(1)) \rightarrow \mathrm{H}^{1}_{M}(M, W(1))$ and then with the Gysin map $\iota_{*}\colon \mathrm{H}^{1}_{M}(M, W(1)) \rightarrow \mathrm{H}^{3}_{M}(S, V(2))$,  we get the map $\mathcal{E}is^{n}_{M}\colon \mathcal{B}_{n} \rightarrow \mathrm{H}^{3}_{M}(S, V(2))$. Hence, for $\phi_{f} \in \mathcal{B}_{n}$, we have a motivic class $\mathcal{E}is_{M}^{n}(\phi_f) \in \mathrm{H}^{3}_{M}(S, V(2))$. In other words, we have the following diagram: 
\[
    \begin{tikzcd}[row sep = 0]
        \mathcal{B}_n \ar[r, "{Eis^n_M}"] & \mathrm{H}^{1}_{M}(\Sh_{\GL_2}, \mathrm{Sym}^{n}V_2(1))  \ar[r, "{p^{*}}"] & \mathrm{H}^{1}_{M}(M, W(1)) \ar[r, "{\iota_{*}}"] & \mathrm{H}^{3}_{M}(S,V(2)) \\
        \phi_f \ar[r, mapsto] & Eis^n_M(\phi_f) \ar[rr, mapsto] & & \mathcal{E}is^{n}_{M}(\phi_f) 
    \end{tikzcd}. 
\]
This construction has been given in \cite[Definition 9.2.3]{LSZ22}. 
\par Applying the Beilinson regulator $r_{H}$, we have the following diagram of absolute Hodge cohomology: 
\begin{equation*}
         \begin{tikzcd}[row sep = 0]
        \mathcal{B}_{n, \RR} \ar[r, "{Eis^n_H}"] & \mathrm{H}^{1}_{H}(\Sh_{\GL_2}, \mathrm{Sym}^{n}V_2(1))  \ar[r, "{p^{*}}"] & \mathrm{H}^{1}_{H}(M, W(1)) \ar[r, "{\iota_{*}}"] & \mathrm{H}^{3}_{H}(S,V(2)) \\
        \phi_f \ar[r, mapsto] & Eis^n_H(\phi_f) \ar[rr] & & \mathcal{E}is^{n}_{H}(\phi_f) 
    \end{tikzcd}.
\end{equation*}
\subsubsection{Geometric results}
Our first geometric theorem states that the Hodge class $\mathcal{E}is^{n}_{H}(\phi_f)$ is in a ``nice" subspace of $\mathrm{H}^{3}_{H}(S,V(2))$, whose proof follows from the method used in \cite{LemmaI15}: 
\begin{theorem}[Theorem \ref{Thm: Hdg vanish on the boudary}]
\label{Thm: intro_Hodge_vanish_on_the_boundary}
For $V = V^{a, b}\{r, s\}$ as in Definition \ref{def:rep_G}, under certain conditions (see Theorem \ref{Thm: Hdg vanish on the boudary}), 
the map $\mathcal{E}is_{H}^n\colon \mathcal{B}_{n,\RR} \rightarrow \mathrm{H}^{3}_{H}(S, V(2))$ factors through the inclusion 
\[
    \Ext^{1}_{\mathrm{MHS}_{\RR}^{+}}(\mathbf{1}, \mathrm{H}^{2}_{B,!}(S, V(2))_{\RR}) \hookrightarrow \mathrm{H}^{3}_{H}(S, V(2))_{\RR}, 
\]
where $\mathrm{MHS}_{\RR}^{+}$ is the abelian category of mixed $\RR$-Hodge structures, ${\mathbf{1}}$ denotes the trivial Hodge structure that is the unit of $\mathrm{MHS}_{\RR}^{+}$ and $\mathrm{H}^{2}_{B,!}(S, V(2))$ is the interior Betti cohomology with coefficients in $V$ defined by
\begin{equation*}
    \mathrm{H}^{2}_{B,!}(S, V(2))_{\RR} := \mathrm{Im}(\mathrm{H}^{2}_{B,c}(S, V(2))_{\RR} \rightarrow \mathrm{H}^{2}_{B}(S, V(2))_{\RR}). 
\end{equation*}
\end{theorem}

\begin{remark}
    The interior Betti cohomology $\mathrm{H}^{2}_{B,!}(S, V(2))_{\CC}$ is the cohomology group containing the ``cohomological" cuspidal automorphic representations $\pi$ of $G(\AAA)$ (see Proposition \ref{Prop:coh_identity} and its proof). Hence, Theorem \ref{Thm: intro_Hodge_vanish_on_the_boundary} tells us that the regulator of the motivic class $\mathcal{E}is^{n}_{H}(\phi_f)$ is in a ``nice" subspace that is related to automorphic representations. It gives one hope that the class $\mathcal{E}is^{n}_{H}(\phi_f)$ can be related to the motivic $L$-function of some automorphic representation. 
\end{remark}

Our second geometric theorem is a ``motivic lifting'' of the first theorem that says that the motivic class $\mathcal{E}is^{n}_{M}(\phi_f)$ is in a ``nice" subspace of $\mathrm{H}^{3}_{M}(S,V(2))$: 
\begin{theorem}[Theorem \ref{Thm: mot vanish on the boudary}]
\label{Thm: intro_motivic_vanish_on_the_boundary}
Under the same conditions of the previous theorem, 
the map $\mathcal{E}is_{M}^n \colon \mathcal{B}_{n} \rightarrow \mathrm{H}^{3}_{M}(S, V(2))$ factors through the inclusion 
\[
    \mathrm{H}_{M}^{3}(\mathrm{Gr_{0}}M_{gm}(V(2)), \QQ(0)) \hookrightarrow \mathrm{H}^{3}_{M}(S, V(2)), 
\] 
where $\mathrm{H}_{M}^{3}(\mathrm{Gr_{0}}M_{gm}(V(2)), \QQ(0))$ (see Definition \ref{def: motivic cohomology of interior motives}) is the motivic cohomology of the interior motive $\mathrm{Gr_{0}}M_{gm}(S, V(2))$ (see Definition \ref{def: intersection motive and interior motive}).  
\end{theorem}

\begin{remark}
    \begin{itemize}
        \item The interior Betti cohomology is the realization of the interior motive $\mathrm{Gr_{0}M_{gm}}(S, V(2))$ through the Betti realization functor. 
        \item Under the Beilinson regulator $r_{H}$, the image of $\mathrm{H}_{M}^{3}(\mathrm{Gr_{0}}M_{gm}(V(2)), \QQ(0))$ lies in 
            \begin{equation*}
                \Ext^{1}_{\mathrm{MHS}_{\RR}^{+}}(\mathbf{1}, \mathrm{H}^{2}_{B,!}(S, V(2))_{\RR}).
            \end{equation*}
        Hence, Theorem \ref{Thm: intro_motivic_vanish_on_the_boundary} can be viewed as a ``motivic lifting'' of Theorem \ref{Thm: intro_Hodge_vanish_on_the_boundary}. 
        \item Theorem \ref{Thm: intro_motivic_vanish_on_the_boundary} tells us that the motivic class $\mathcal{E}is^{n}_{M}(\phi_f)$ is in a ``nice" subspace that is related to the Grothendick motive associated to some automorphic representation \cite[Theorem 5.6]{Wild_15}. It gives one hope to relate the motivic class $\mathcal{E}is^{n}_{M}(\phi_f)$ to the motivic $L$-function of some automorphic representation.
        \item In \cite{Kings98}, the author raised the question of proving the motivic version of vanishing on the boundary for Eisenstein class on Shimura varieties. In the case of Hilbert modular surfaces, a result of motivic vanishing on the boundary has been proved in \cite[Corollary 3.14]{Wild_09_HMS} using a different formalism from us. 
    \end{itemize}
\end{remark}

\subsubsection{Results on special values of $L$-functions}

Let $\pi = \pi_f \otimes \pi_{\infty}$ be some ``cohomological" irreducible cuspidal automorphic representation $\pi$ of $G(\AAA)$, where we denote by $E(\pi_f)$ its rational field (see Section \ref{Sec: main result}). 
Let $M(\pi_f, V(2))$ be the Grothendieck motive associated to $\pi$ \cite[Theorem 5.6]{Wild_15} after a Tate twist and $M_{B}(\pi_f, V(2))$ be its Betti realization. Then the absolute Hodge cohomology of $M_{B}(\pi_f, V(2))$ is a rank-$1$ $E(\pi_f) \otimes_{\QQ} \RR$-module. In it, we define two rational structures $\mathcal{K}(\pi_f, V(2))$ and $\mathcal{D}(\pi_f, V(2))$, where the rational structure $\mathcal{K}(\pi_f, V(2))$ (maybe trivial) is defined using the motivic class $\mathcal{E}is_{M}(\phi_f)$ and $\mathcal{D}(\pi_f, V(2))$ (non-trivial) is defined using the comparison between the Betti and de Rham realizations (see Section \ref{SS: Poincare duality}). Our theorem says that the difference between the two rational structures can be measured using a non-critical value of the motivic $L$-function $L(M_{\text{\'et}}(\pi_f, V(2)), s)$ (see Definition \ref{Def: mot L-function}) of $\pi$: 
\begin{theorem}[Theorem \ref{Them_mot_L_function}]
\label{Thm:intro_mot_L_value}
    Under the conditions in the statement of Theorem \ref{Them_mot_L_function}, 
    the relation between $\mathcal{K}(\pi_f,V(2))$ and $\mathcal{D}(\pi_f,V(2))$ is as follows: 
    \begin{equation*}
            \mathcal{K}(\pi_f,V(2)) = C \cdot L(M_{\text{\'et}}(\pi_f, V(2)), 0)\mathcal{D}(\pi_f, V(2)), 
    \end{equation*}
    where $C \in (E(\pi_f) \otimes_{\QQ} \CC)^{\times}$ and $C$ can be expressed explicitly in terms of a comparison of Whittaker periods (see Defintion \ref{def: Whittaker period}) and Deligne periods (see Definition \ref{def: Deligne period}).
\end{theorem}

\begin{remark}
    \begin{itemize}
        \item That the motivic $L$-function $L(M(\pi_f, V(2)), s)$ is being evaluated at $s = 0$ is exactly what Beilinson conjectured. \cite[Conjecture (6.1)]{Nekovar94} 
        \item The $C$ in Theorem \ref{Thm:intro_mot_L_value} should be in $(E(\pi_f) \otimes_{\QQ} \overline{\QQ})^{\times}$ according to Beilinson's conjectures.
        \item Theorem \ref{Thm:intro_mot_L_value} is a weight $w \le -3$ counterpart of the main theorem proved in \cite{PS18}, where the authors constructed a motivic class $\xi$ in the motivic cohomology with trivial coefficients $\mathrm{H}_{M}^{3}(S, \QQ(2))_{\overline{\QQ}}$ and proved a relation between $\xi$ and a non-critical $L$-value of some cuspidal automorphic representation $\pi$\footnote{There is a gap in the proof of \cite{PS18}, which seems could be fixed by the tool of tempered currents developed recently in \cite{BCLRJ24}}. 
        \item A direct corollary of Theorem \ref{Thm:intro_mot_L_value}  is that the map $\mathcal{E}is_{M}$ is non-trivial (see Corollary \ref{corollary: nontrivial_class}). This answers a question raised in \cite{LSZ22}, where the authors needed the non-triviality of the motivic classes $\mathcal{E}is(\phi_f)$ for $\phi_f \in \mathcal{B}_{n}$ as an input to construct an Euler system for $\GU(2,1)$. 
        \item Theorem \ref{Thm:intro_mot_L_value} is deduced from an automorphic version of the theorem (see Theorem \ref{Thm: auto_L-value}) and a comparison between automorphic $L$-functions and motivic $L$-functions after many normalizations (see Proposition \ref{Prop:rel_L_ftn}). 
    \end{itemize}
\end{remark}

\subsection{An overview of the proof}

\subsubsection{An outline of the proof of the geometric results}

We first give an outline of the proof of vanishing on the boundary of absolute Hodge cohomology (Theorem \ref{Thm: intro_Hodge_vanish_on_the_boundary}, Theorem \ref{Thm: Hdg vanish on the boudary}). 

\par For each of the two Shimura varieties $X = \{M, S\}$, we denote by $X^{*}$ its Baily{\textendash}Borel compactification (see Section \ref{SS: Baily-Borel compactification of PMF} for more details) and let $\partial{X} = X^{*} - X$ be the cusps of the compactification. Hence, we have the following commutative diagram: 
\[
    \begin{tikzcd}
        M \ar[r, "j^{\prime}"] \ar[d, "{\iota}"] & M^{*} \ar[d, "p"] & \partial{M} \ar[l, "i^{\prime}" above] \ar[d, "q"] \\
        S \ar[r, "j"] & S^{*} & \partial{S} \ar[l, "i" above], 
    \end{tikzcd}
\]
where $j$ and $j^{\prime}$ are open immersions, $i$ and $i^{\prime}$ are  closed immersions and $\iota$, $p$ and $q$ are closed immersions\footnote{It holds after a careful choice of level structures.}. 

\par By properties of the derived category of real algebraic mixed Hodge modules (see \cite[Definition A.2.4]{HW98}), we have the following exact sequence (see Proposition \ref{Prop: Hodge exact sequence}):

\begin{equation}
\label{Intro: Hodge exact sequence}
    0 \rightarrow \Ext^{1}_{\mathrm{MHS}_{\RR}^{+}}(\mathbf{1}, \mathrm{H}^{2}_{B,!}(S, V(2))_{\RR}) \rightarrow \mathrm{H}^{3}_{H}(S,V(2)) \rightarrow \mathrm{H}^{1}_{H}(\partial{S}, i^{*}j_{*}V(2)).
\end{equation}

By the functoriality, we can get the following commutative diagram (see Proposition \ref{Prop: Hodge commutative diagram}): 
\begin{equation}
\label{Intro: Hodge commutative diagram}
    \begin{tikzcd}
        \mathcal{B}_{n, \RR} \ar[d] & \\
        \mathrm{H}^{1}_{H}(M,W(1)) \ar[r] \ar[d] & \mathrm{H}^{0}_{H}(\partial{M}, i^{\prime*}j^{\prime}_{*}W(1)) \ar[d, "\theta^{H}"] \\
        \mathrm{H}^{3}_{H}(S, V(2)) \ar[r] & \mathrm{H}^{1}_{H}(\partial{S}, i^{*}j_{*}V(2)). 
    \end{tikzcd}
\end{equation}

By exact sequence (\ref{Intro: Hodge exact sequence}) and commutative diagram (\ref{Intro: Hodge commutative diagram}), in order to prove that the map $\mathcal{E}is_{H}^n \colon \mathcal{B}_{n,\RR} \rightarrow \mathrm{H}^{3}_{H}(S, V(2))$ factors through the inclusion 
\[
    \Ext^{1}_{\mathrm{MHS}_{\RR}^{+}}(\mathbf{1}, \mathrm{H}^{2}_{B,!}(S, V(2))_{\RR}) \hookrightarrow \mathrm{H}^{3}_{H}(S, V(2)), 
\]
it suffices to prove that the map $\theta^{H}$ is zero, which is carried out in Proposition \ref{Prop: Hdg theta is zero}. The key input of the proof of that fact is the explicit computation of degeneration of Hodge structures over $M$ and $S$ (see Lemma \ref{lemma: BW_H}, Lemma \ref{lemma: BW_G}), which is based on a theorem of Burgos-Wildeshaus (see Theorem \ref{theorem: BW04}) and Kostant's theorem (see Theorem \ref{thm: Kostant_theorem}). Finally, we remark that the degeneration of Hodge structures over $S$ has been computed in \cite{Anc17}. \\


\par We now outline the proof of vanishing on the boundary of motivic cohomology (Theorem \ref{Thm: intro_motivic_vanish_on_the_boundary}, Theorem \ref{Thm: mot vanish on the boudary}). The proof is parallel to the proof in the absolute Hodge cohomology case. However, we need to translate everything into the language of motives. 

\par First, we have a motivic version of the exact sequence (\ref{Intro: Hodge exact sequence}) (see Proposition \ref{Prop: mot exact sequence}): 
\begin{equation}
\label{Intro: Mot exact sequence}
    0 \rightarrow   \mathrm{H}_{M}^{3}(\mathrm{Gr_{0}}M_{gm}(V(2)), \QQ(0)) \rightarrow \mathrm{H}^{3}_{M}(S,V(2)) \rightarrow \mathrm{H}^{3}_{M}(\partial{S}, i^{*}j_{*}V(2)).
\end{equation}

\par Second, by functoriality of the triangulated category of mixed motives, we have the motivic version of the commutative diagram (\ref{Intro: Hodge exact sequence}) (see Proposition \ref{Prop: mot commutative diagram}): 
\begin{equation}
\label{Intro: Mot commutative diagram}
    \begin{tikzcd}
        \mathcal{B}_n \ar[d] & \\
        \mathrm{H}^{1}_{M}(M,W(1)) \ar[r] \ar[d] & \mathrm{H}^{1}_{M}(\partial{M}, i^{\prime*}j^{\prime}_{*}W(1)) \ar[d, "\theta^{M}"] \\
        \mathrm{H}^{3}_{M}(S, V(2)) \ar[r] & \mathrm{H}^{3}_{M}(\partial{S}, i^{*}j_{*}V(2)). 
    \end{tikzcd}
\end{equation}

\par Finally, similar to the absolute Hodge cohomology case, in order to prove that the map $\mathcal{E}is_{M}^n\colon \mathcal{B}_{n} \rightarrow \mathrm{H}^{3}_{M}(S, V(2))$ factors through the inclusion 
\[
     \mathrm{H}_{M}^{3}(\mathrm{Gr_{0}}M_{gm}(V(2)), \QQ(0)) \hookrightarrow \mathrm{H}^{3}_{M}(S, V(2)), 
\] 
it suffices to prove that $\theta^{M}$ is zero, whose proof is based on the fact that $\theta^{H}$ is zero and weight conservativity of Hodge realization functor for Voevodsky motives of Abelian type \cite[Theorem 1.12.]{Wild_15} (see Proposition \ref{Prop: mot theta is zero}). 

\subsubsection{An outline of the proof of the $L$-value results}

In order to prove the relation 
\begin{equation*}
    \mathcal{K}(\pi_f, V(2)) = C^{\prime} \cdot \mathcal{D}(\pi_f, V(2)), 
\end{equation*}
in the rank $1$ $E(\pi_f) \otimes_{\QQ} \RR$-module $\Ext^{1}_{\mathrm{MHS_{\RR}^{+}}}(\RR(0), M_{B}(\pi_f, V(2))_{\RR})$ for some $C^{\prime} \in (E(\pi_f) \otimes_{\QQ} \RR)^{\times}$, it suffices to find an $E(\pi_f) \otimes_{\QQ} \RR$-linear form $\psi \colon M_{B}(\pi_f, V(2))^{-}_{\RR}(-1) \rightarrow E(\pi_f) \otimes_{\QQ} \CC$ which is trivial on $F^{0}M_{dR}(\pi_f, V(2))_{\RR}$ such that 
\begin{equation*}
    \psi(\tilde{v}_{K}) = C^{\prime} \cdot \psi(\tilde{v}_{D}), 
\end{equation*}
where $v_{K}$ (resp., $v_{D}$) is a $E(\pi_f)$-generator of $\mathcal{K}(\pi_f, V(2))$ (resp., $\mathcal{D}(\pi_f, V(2))$) and $\tilde{v}_{K}$ (resp., $\tilde{v}_{D}$) is a lifting of $v_{K}$ (resp., $v_{D}$) through the exact sequence (\ref{exact seq: Ext^1}) (see Lemma \ref{Lemma: linear form and rational structure}).

\par By the Poincar\'{e} duality pairing 
\begin{equation*}
        \langle \cdot, \cdot \rangle_{B} \colon \mathrm{H}^{2}_{B, !}(S, V(2)) \otimes \mathrm{H}^{2}_{B, !}(S, D) \rightarrow \QQ(0),
\end{equation*}
where $D$ is the contragredient representation of $V$ and the construction of a differential form $\Omega \in M(\tilde{\pi}_f|\mu|^{-2}, D)_{\CC}^{+}$, which is carried out in Section \ref{SS: the test vector}, we construct a linear form $\Psi$ and we get 
\begin{equation*}
    \mathcal{K}(\pi_{f}, V(2)) = \frac{\langle \Omega, \tilde{v}_{K} \rangle_{B}}{\langle \Omega, \tilde{v}_{D} \rangle_{B}} \mathcal{D}(\pi_f, V(2)). 
\end{equation*}
Hence, we are left to compute $\langle \Omega, \tilde{v}_{K} \rangle_{B}$ and $\langle \Omega, \tilde{v}_{D} \rangle_{B}$.  

\par The way to compute the pairing $\langle \Omega, \tilde{v}_{K} \rangle_{B}$ is as follows. We first map the class $v_{K}$ to Deligne{\textendash}Beilinson cohomology through the natural map $r_{H \rightarrow D}$ from absolute Hodge cohomology to Deligne{\textendash}Beilinson cohomology (see definition of Deligne{\textendash}Beilinson cohomology in Proposition \ref{Prop: DB-coh_temperd currents}), in which the class will be represented by a pair of tempered currents (see Remark \ref{remark: Eis_D class})
\begin{equation*}
    (\iota_{*}T_{p^{*}Eis_{H}^{n}(\phi_f)}, \iota_{*} T_{p^{*}Eis_{B}^{n}(\phi_f)})
\end{equation*}
and this can be used to compute the pairing $\langle \Omega, \tilde{v}_{K} \rangle_{B}$ (see Lemma \ref{Lemma: tempered current associated to Eis classes} for details). Finally, we express the pairing in terms of an integral of differential forms over Shimura varieties $\Sh_{H}$ (see Proposition \ref{Prop: pairing to integration of diff form}), which is expressed explicitly in Propositin \ref{prop: explicit_pairing}. 

\begin{remark}
    \begin{itemize}
        \item The tool of tempered currents defined in \cite{BCLRJ24} is a key ingredient of this step. Hence, we recall the key definitions in Section \ref{SS: DB cohomology}. 
        \item In order to apply the tool of Deligne{\textendash}Beilinson cohomology, we need to apply ``Liebermann’s trick" (see Definition \ref{Def: DB-coh with coefficients}) because we are working with non-trivial coefficients. 
    \end{itemize}
\end{remark}

\par By a careful choice of test vectors (see Proposition \ref{Proposition_comp_pair_w_zeta_int}), we can express the above explicit integral as a zeta integral $I(\varphi, \Phi, \nu, s)$ (see Definition \ref{def: Zeta_integral}) constructed by Gelbart and Piatetski-Shapiro, which gives an integral repersentation of the twisted standard automorphic $L$-function of the contragredient $\tilde{\pi}$ of $\pi$. We then prove the algebraicity of the nonarchimedean local Zeta integral at rational points in Propositon \ref{Prop:algebracity} and compute the archimedean Zeta integral explicitly in Proposition \ref{Prop:Arch_Zeta}. Combining all of these with an identity (see Lemma \ref{Lemma: relate L-function of contra of pi to L-function of pi}) between automorphic $L$-functions of an automorphic representation and it congragredient, we can express the pairing $\langle \Omega, \tilde{v}_{K} \rangle_{B}$ in terms of an automorphic $L$-value of $L(s, \pi, \mathrm{std})$ (see Definition \ref{def: L-function}) at non-critical points and Whittaker periods (see Definition \ref{def: Whittaker period}).  

\par By the definition of Deligne periods and Deligne $E(\pi_f)$-structure (see Definition \ref{def: Delign-rational structure}), we express the pairing $\langle \Omega, \tilde{v}_{D} \rangle_{B}$
in terms of Deligne periods (see Proposition \ref{prop: periods}). 

\par Combining the above computation of $\langle \Omega, \tilde{v}_{K} \rangle_{B}$ and $\langle \Omega, \tilde{v}_{D} \rangle_{B}$, we prove a relation 
\begin{equation*}
    \mathcal{K}(\pi_f, V(2)) = C^{\prime} \cdot \mathcal{D}(\pi_f, V(2)), 
\end{equation*}
where $C^{\prime}$ is expressed in terms of a non-critical value of $L(s, \pi, \mathrm{std})$ at $s = 0$, Whittaker periods and Deligne periods (see Theorem \ref{Thm: auto_L-value}). Finally, we relate the automorphic $L$-function $$L(s, \pi, \mathrm{std})$$ to the motivic $L$-function $L(M_{\text{\'et}}(\pi_f, V(2)), s)$ (see Proposition \ref{Prop:rel_L_ftn}). The fact that $C^{\prime} \in (E(\pi_f) \otimes_{\QQ} \CC)^{\times}$ follows from $L(M_{\text{\'et}}(\pi_f, V(2)), 0) \neq 0$ (see the proof of Corollary \ref{corollary: nontrivial_class}). 

\subsection{Structure of the paper}
In Section \ref{SS:algebraic groups}-Section \ref{SS: Lie groups}, we define certain algebraic groups, Lie groups and their representations. In Section \ref{SS:Shimura varieties}-Section \ref{SS: cohomology of Picard modular surfaces}, we define the corresponding Shimura varieties and recall the needed properties of cohomology of Picard modular surfaces. In Section \ref{SS:discrete series}-Section \ref{SS: Hodge decomposition}, we compute the $L$-packet of discrete series at the archimedean place and compute the Hodge decomposition of corresponding motives associated to automorphic representations. In Section \ref{SS: relative motive}-Section \ref{SS: Beilinson regulator}, we recall the definition of motivic cohomology, absolute Hodge cohomology and Beilinson regulators. In Section \ref{SS: motivic class}, we recall the construction of the motivic classes in \cite{LSZ22} and their image under the Beilinson regulator, which we call the Hodge classes. In Section \ref{SS: Baily-Borel compactification of PMF}-Section \ref{SS: the proof of the Hodge vanishing}, we prove that the Hodge classes belong to a ``nice'' subspace of absolute Hodge cohomology. In Section \ref{SS: weight structures}-Section \ref{Sec: motivic vanishing}, we prove that the motivic classes belong to a ``nice'' subspace of motivic cohomology. In Section \ref{SS: Poincare duality}-Section \ref{SS: res of the diff form}, we explain how to associate a differential form to the contragredient automorphic representation and use it and Poincar\'{e} duality to construct a linear form.  In Section \ref{SS: DB cohomology}, we recall the definition of Deligne{\textendash}Beilinson cohomology using tempered currents. In Section \ref{SS: DB-cohomology with coeffients}-Section \ref{SS the pairing}, we give an explicit expression of the classes in Deligne{\textendash}Beilinson cohomology and an explicit expression of its pairing with the above differential form. In Section \ref{SS: global zeta integral}-Section \ref{SS: comparison of integral with zeta integral}, we express the above integral in terms of a zeta integral. In Section \ref{SS: unfolding}-Section \ref{SS: arch local integral}, we unfold the zeta integral and recall an algebraicity result of nonarchimedean zeta integrals and compute the archimedean zeta integral explicitly. In Section \ref{SS: motivic periods}, we compute the pairing of the Deligne rational structure with the diffferential form. In Section \ref{SS: Whittaker period}, we recall the definition of the Whittaker period. In Section \ref{SS: connection to auto L-function}-Section \ref{SS: conenction to motivic L-function}, we prove the main results of the paper. 

\subsection*{Acknowledgments}
This paper is based on the author’s doctoral dissertation at the University of Connecticut. He would like to thank his Ph.D. advisor, Liang Xiao, for introducing him to the subject and for many insightful discussions during his graduate studies. He is especially grateful to his postdoctoral mentors, David Loeffler and Sarah Zerbes, for their continued interest in this work and for many valuable suggestions and comments that have improved the paper. He also thanks Keith Conrad for carefully correcting typographical and linguistic errors. Part of this work was carried out during a visit to the Morningside Center of Mathematics in Beijing in 2023–2024, generously hosted by Yongquan Hu, whose hospitality is warmly appreciated.
