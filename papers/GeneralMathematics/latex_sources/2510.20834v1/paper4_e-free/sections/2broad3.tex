\newpage

%------------------------------------------------------------------
\section{Broad rank-3 geometry and its role in Kakeya--BCT}%
\label{sec:broad3}
%------------------------------------------------------------------

Throughout this section we fix
\[
   r=\lambda^{-2/3},\quad
   D=\lambda^{1/12},\quad
   \alpha:=c_{0}\,rD^{1/2}=c_{0}\,\lambda^{-5/8},\qquad
   c_{0}>0,\;\;\lambda\ge2 .
\]
Angles and norms are always measured in the spherical (angular) metric.

%------------------------------------------------------------------
\subsection{Bilipschitzness of the map \texorpdfstring{$\xi\mapsto n(\xi)$}{xi->n(xi)}}%
\label{subsec:lip}

Angles are measured in the spherical (angular) metric on unit spheres.

Let \(|\xi|\sim\lambda\). For the paraboloid normal
\[
  n(\xi)=\frac{(-2\xi,\,1)}{\sqrt{\,1+4|\xi|^{2}\,}}
\]
there exists an absolute constant \(c_\ast\in(0,1)\) such that for any
\(\xi,\eta\) with \(|\xi|,|\eta|\sim\lambda\) one has
\begin{equation}\label{eq:lip}
   c_\ast\,\angle(\xi,\eta)
   \;\le\;
   \angle\!\bigl(n(\xi),n(\eta)\bigr)
   \;\le\;
   c_\ast^{-1}\,\angle(\xi,\eta).
\end{equation}
In particular, for sufficiently large \(\lambda\) one may take \(c_\ast=\tfrac12\).

\begin{proof}[Sketch]
From the exact formula for \(n(\xi)\) we get, for \(|\xi|\sim\lambda\),
\[
  (1+4|\xi|^{2})^{-1/2}
  \;=\;
  \frac{1}{2|\xi|}\,\Bigl(1-\frac{1}{8|\xi|^{2}}+O(|\xi|^{-4})\Bigr),
\]
and hence the uniform asymptotics for \(|\xi|\sim\lambda\)
\[
  n(\xi)=\frac{(-\xi,\,1/2)}{|\xi|}\,\Bigl(1+O(\lambda^{-2})\Bigr).
\]
With \(u=\xi/|\xi|\), \(v=\eta/|\eta|\), we obtain
\[
  n(\xi)=\bigl(-u,\,\tfrac{1}{2|\xi|}\bigr)+O(\lambda^{-2}),\qquad
  n(\eta)=\bigl(-v,\,\tfrac{1}{2|\eta|}\bigr)+O(\lambda^{-2}).
\]
The radial part contributes only an \(O(\lambda^{-2})\) correction (since \(|\xi|,|\eta|\sim\lambda\)),
therefore
\[
  |n(\xi)-n(\eta)|
  \;=\;
  |u-v|\,\bigl(1+O(\lambda^{-2})\bigr)+O(\lambda^{-2}).
\]
For unit vectors \(u,v\) we have \(|u-v|\asymp \angle(u,v)=\angle(\xi,\eta)\), and also
\(|n(\xi)-n(\eta)|\asymp \angle(n(\xi),n(\eta))\). Absorbing the small \(O(\lambda^{-2})\) terms into the constants,
we obtain the two-sided estimate \eqref{eq:lip}.
\end{proof}

\begin{remark}
Writing the main term as
\[
n(\xi)=\frac{(-\xi,\,1/2)}{|\xi|}\bigl(1+O(\lambda^{-2})\bigr)
\]
is consistent with formula~(\ref{eq:norm}) and the subsequent expansion in Appendix~\ref{app:A1};
this normalization (without “losing” a factor \(2\) in the spatial component)
is used later in controlling minors and Gram matrices.
\end{remark}

%------------------------------------------------------------------
\subsection{The functional \texorpdfstring{$\mathrm{Broad}_{3}$}{Broad3}}%
\label{subsec:broad-def}
%------------------------------------------------------------------

For functions \(F_{1},\dots,F_{6}\) on \(\mathbb{R}^{4}\) set
\[
   \mathrm{Broad}_{3}(x):=
   \min_{i<j<k}
   \frac{|F_{i}(x)\,F_{j}(x)\,F_{k}(x)|^{1/3}}
        {\|n_{i}\wedge n_{j}\wedge n_{k}\|^{1/3}},
   \qquad
   Q(x):=\prod_{m=1}^{6}|F_{m}(x)|^{1/2}.
\]
By Lemmas~\ref{lem:A2}–\ref{lem:A4} in combination with the bilipschitzness of the map
\S\ref{subsec:lip}, there exists a triple \((i,j,k)\) with
\(\|n_{i}\wedge n_{j}\wedge n_{k}\|\ge\tfrac{1}{16}\,\alpha^{2}\).
Hence
\begin{equation}\label{eq:broad-upper}
   \mathrm{Broad}_{3}(x)\ \le\
   \frac{|F_{i}F_{j}F_{k}|^{1/3}}{(1/16)^{1/3}\,\alpha^{2/3}}.
\end{equation}

Since $\binom{6}{3}=20$ and each $F_m$ appears in exactly $\binom{5}{2}=10$ triples, we have
\[
\min_{i<j<k}|F_iF_jF_k|^{1/3}
\le\Bigl(\prod_{i<j<k}|F_iF_jF_k|^{1/3}\Bigr)^{1/20}
=\prod_{m=1}^6 |F_m|^{10\cdot(1/3)\cdot(1/20)}=Q^{1/3}.
\]

Therefore $\|{\rm Broad}_3\|_{L^2}\lesssim \alpha^{-2/3}\|Q^{1/3}\|_{L^2}$ and, consequently, in the trilinear insertion
there appear the factor $\alpha^{-2/9}=\lambda^{5/36}$ (since $\alpha=c_0\lambda^{-5/8}$) and the exponent $1/18$
on $\prod_m\|F_m\|_{L^2}$ in (\ref{eq:BCT-final}).



\paragraph{Choice of functions \(F_m\).}
Split the family of caps into six disjoint subsets
\(\mathcal{C}_{1},\dots,\mathcal{C}_{6}\) so that the centers of their normals
satisfy the conditions of Lemma~\ref{lem:A4}, and set
\[
   F_{m}:=\sum_{\Theta\in\mathcal{C}_{m}} F_{\Theta},\qquad m=1,\dots,6.
\]

\paragraph{Local choice of six classes.}
The global partition yields $M = O(D)$ classes (Lemma~\ref{lem:AD-colouring}; see also Appendix~\ref{app:C2}).
Covering $Q_\lambda$ by a fixed family of $O(1)$ cells (at the scale of \S\ref{subsec:scales})
and applying in each cell a greedy selection of six disjoint classes, we obtain
subfamilies $\mathcal{C}_{1},\dots,\mathcal{C}_{6}$ for which the conditions of Lemma~\ref{lem:A4}
hold. The overlap of the covering is bounded by a constant, so summation over
cells does not introduce additional powers of~$\lambda$ or~$D$.
Hence we work with
\[
F_{m} := \sum_{\Theta\in\mathcal{C}_{m}} F_{\Theta}, \quad m = 1,\dots,6.
\]

%------------------------------------------------------------------
\subsection{Insertion into Kakeya–BCT}%
\label{subsec:bct-insert}
%------------------------------------------------------------------

\paragraph{Model and dynamics.}
From now on we work in one of the following equivalent contexts:
\begin{itemize}
\item[(i)] $F=Eg$, where $E$ is the extension operator for the paraboloid surface
$\{(\tau,\xi):\ \tau=|\xi|^2\}$ and $g$ is a function on frequency space;
\item[(ii)] $F$ is a solution of the Schrödinger equation $(\partial_t - i\Delta_x)F=0$
in $Q_\lambda$ with frequency support $|\xi|\sim\lambda$.
\end{itemize}
In both cases the decomposition by $\alpha$-caps $\Theta$ in frequency space
leads to wave localization of $F_\Theta$ in a family of tubes $\mathcal T_\Theta$
of radius $\lambda^{-1/2}$ and length $\lambda^{-3/2}$:
\begin{equation}\label{eq:wave-packet}
  \|F_\Theta - \mathbf 1_{\mathcal T_\Theta}F_\Theta\|_{L^2(Q_\lambda)}
  \;\le\; C_N\,\lambda^{-N}\,\|F_\Theta\|_{L^2(\mathbb R\times\mathbb R^3)},
  \quad\forall N\gg 1.
\end{equation}
This localization is used when passing from $f_m$ to the tube-localized form
in the \emph{Robust--Kakeya} block (§\ref{subsec:kakeya-robust}).

\paragraph{Applicability of BCT.}
The constructed triple \((i,j,k)\) from §\,\ref{subsec:broad-def} ensures
transversality of normals at level \((1/16)\,\alpha^{2}\), as required in the
trilinear Bennett–Carbery–Tao theorem. In the outer multiplicity count
we use \emph{only} the high angular density region.

\begin{lemma}[Multiplicity compensation: high-density version]\label{lem:compD3}
Let \(C_1,C_2,C_3\) be three independent angular classes
(see the partition into \(O(D)\) classes in Lemma~\ref{lem:AD-colouring}),
and set \(f_m:=\sum_{\Theta\in C_m}F_\Theta\), \(m=1,2,3\).
Define
\[
M(t,x):=\sum_{\Theta}\mathbf 1_{\mathcal T_\Theta}(t,x),\qquad
f_m^{\mathrm{hi}}:=f_m\cdot \mathbf 1_{\{M\ge cD\}}.
\]
Then
\[
   \|f_m^{\mathrm{hi}}\|_{L^2(Q_\lambda)}
   \ \lesssim\ D^{1/2}\,
   \Bigl(\sum_{\Theta\in C_m}\|F_\Theta\|_{L^2(Q_\lambda)}^2\Bigr)^{1/2}.
\]
Consequently, one Cauchy–Schwarz step in the trilinear insertion
over the set \(\{M\ge cD\}\) produces a factor of \(D^{3/2}\)
(respectively \(D^{3}\) at the level of the squared \(TT^{\!*}\)).
\end{lemma}

\begin{proof}[Idea of the proof]
This is a direct application of the clustered version of the Robust--Kakeya
estimate \eqref{eq:kakeya-factor} from §\,\ref{subsec:kakeya-robust}, when $G$ consists
of caps within one $\alpha$-cap; in this case $\sup M \lesssim D$, and
passing from the pointwise estimate (4.3) to the integral one yields a factor $D$.
The correspondence $F_\Theta \leftrightarrow \mathcal T_\Theta$ is guaranteed by standard
wave packet localization:
$\|F_\Theta-\mathbf 1_{\mathcal T_\Theta}F_\Theta\|_{L^2(Q_\lambda)}
\le C_N\lambda^{-N}\|F_\Theta\|_{L^2}$ (stationary phase on the $Q_\lambda$ scale),
so replacing $F_\Theta$ by $F_\Theta\,\mathbf 1_{\mathcal T_\Theta}$ does not affect
the exponents. Summation over $\alpha$-caps in the full configuration is
with bounded multiplicity (Lemmas~\ref{lem:AD-colouring} and~\ref{lem:A8}),
so no additional powers of~$\lambda$ or~$D$ appear.
\end{proof}

\medskip
\noindent
The passage from $g=\sum_\Theta F_\Theta\,\mathbf 1_{\mathcal T_\Theta}$ (as in §\,\ref{subsec:kakeya-robust})
to $f_m=\sum_{\Theta\in C_m}F_\Theta$ (in §\,\ref{subsec:bct-insert}) is carried out
via wave packet localization: on $Q_\lambda$
\[
   \|F_\Theta-\mathbf 1_{\mathcal T_\Theta}F_\Theta\|_{L^2(Q_\lambda)}
   \ \le\ C_N\,\lambda^{-N}\,\|F_\Theta\|_{L^2}
\]
for any $N\gg 1$. Therefore the estimates of §\,\ref{subsec:kakeya-robust} apply
to $f_m$ with indicator $\mathbf 1_{\{M\ge cD\}}$.
This legitimizes the use of \eqref{eq:kakeya-factor} inside the trilinear insertion.
\medskip



\noindent
\emph{Trilinear insertion at high multiplicity.}
For \(F_{m}=\sum_{\Theta\in C_{m}}F_{\Theta}\), \(m=1,\dots,6\),
the trilinear Bennett–Carbery–Tao theorem, after one Cauchy–Schwarz step and
Lemma~\ref{lem:compD3}, gives
\[
   \Bigl\|\Bigl(\prod_{m=1}^{6}F_{m}\Bigr)\mathbf 1_{\{M\ge cD\}}\Bigr\|_{L^{6}}
   \ \le\
   C\,D^{3/2}\,\|\mathrm{Broad}_{3}\|_{L^{2}}^{\,1/3}\,
   \Bigl(\sum_{m=1}^{6}\|F_{m}\|_{L^{2}}^{2}\Bigr)^{1/2}.
\]
Using \eqref{eq:broad-upper} and Hölder’s inequality
\(
\|Q^{1/3}\|_{L^{2}}\le
(\prod_{m=1}^{6}\|F_{m}\|_{L^{2}})^{1/6}
\) for \(Q=\prod_{m=1}^{6}|F_{m}|^{1/2}\),
we obtain
\begin{equation}\label{eq:BCT-final}
   \Bigl\|\Bigl(\prod_{m=1}^{6}F_{m}\Bigr)\mathbf 1_{\{M\ge cD\}}\Bigr\|_{L^{6}}
   \ \le\
   C\,D^{3/2}\,\lambda^{5/36}\,
   \Bigl(\prod_{m=1}^{6}\|F_{m}\|_{L^{2}}\Bigr)^{1/18}\,
   \Bigl(\sum_{m=1}^{6}\|F_{m}\|_{L^{2}}^{2}\Bigr)^{1/2}.
\end{equation}
The sign \(+\tfrac{5}{36}\) in \(\lambda\) comes from \(\alpha=c_{0}\lambda^{-5/8}\).

\begin{remark}[Accounting for the $D$–exponent in Broad–BCT]\label{rem:D-account}
The factor \(D^{3/2}\) arising from \(\|f_m^{\mathrm{hi}}\|_{2}\)
belongs entirely to the Robust–Kakeya block (§\,\ref{subsec:kakeya-robust})
in the global balance; the row “Geometry (Broad–3)” in Table~\ref{subsec:balance-table-final}
has $D$–exponent \(0\).
On the complement \(\{M<cD\}\) one uses an alternative regime
(§\,\ref{sec:tube-pack} or §\,\ref{sec:narrow}); these regimes are
mutually exclusive.
\end{remark}

%------------------------------------------------------------------
\subsection{Contribution to the balance of exponents}%
\label{subsec:balance-table}
%------------------------------------------------------------------

% Local safe version of \captionof (no extra packages required)
\makeatletter
\providecommand{\captionof}[1]{\def\@captype{#1}\caption}
\makeatother

% “Nonfloating” table: placed here immediately after the header
\begin{minipage}{\linewidth}
  \centering
  \captionof{table}{Summary balance of exponents (without using the contribution of §\,\ref{sec:tube-pack})}
  \label{tab:balance}
  \begin{tabular}{lcc}
    \toprule
    Block & $\lambda$–exponent & $D$–exponent \\
    \midrule
    Geometry (Broad--3) & $+\tfrac{5}{36}$ & $0$ \\
    Kernel (12 IBP)     & $-\tfrac{9}{2}$  & $-3$ \\
    Robust--Kakeya      & $+\tfrac{1}{12}$ & $+1$ \\
    Algebraic shell     & $-\tfrac{1}{12}$ & $-1$ \\
    Narrow cascade      & $-\tfrac{5}{64}$ & $0$ \\
    \midrule
    $\Sigma_{\lambda}$  & $-\tfrac{2557}{576}\approx -4.44$ & --- \\
    $\Sigma_{D}$        & --- & $-3$ \\
    \bottomrule
  \end{tabular}
\end{minipage}

\medskip
\noindent\textit{Remark (on the shortened sum).}
By the “simplified/shortened sum” in \(\lambda\) we mean the sum in which the rows of §\,\ref{sec:kakeya} (Robust--Kakeya: $+\tfrac{1}{12}$) and §\,\ref{sec:alg-skin} (“shell”: $-\tfrac{1}{12}$) are omitted
as mutually compensating in the main branch. In this shortened sum, upon replacing
the optimal exponent of block §\,\ref{sec:tube-pack} $-\tfrac{7}{6}$ by the more conservative
$-\tfrac{1}{2}$, only one row changes, and
\[
  \Sigma_{\lambda}
  = \tfrac{5}{36} - \tfrac{9}{2} - \tfrac{5}{64} - \tfrac{1}{2}
  = -\tfrac{2845}{576}\approx -4.94.
\]
For the “full” balance in the scenario “§\,\ref{sec:tube-pack} instead of §\,\ref{sec:kakeya}” while keeping §\,\ref{sec:alg-skin} we get
$\Sigma_{\lambda}=-\tfrac{2893}{576}\approx -5.02$, and with the optimal $-\tfrac{7}{6}$ —
$\Sigma_{\lambda}=-\tfrac{3277}{576}\approx -5.69$ (see §\,\ref{subsec:balance-table-final}).
