\section{Discussion}

In this work, we presented a vectorized framework for computing WECFs
(including ECFs and the WECT).
%
This framework allows for efficient computation of these functions using tensor
operations, which are well-suited for modern GPU hardware.
%
Furthermore, our method works in full generality for weighted simplicial
complexes (or cubical complexes) of arbitrary dimension with no structural
constraints.
%
Our framework is implemented in a publicly available Python package
called \pyect.
%
Experimentally, \pyect demonstrates a significant speedup over
existing methods across a variety of image datasets.
%
The implementation is also incredibly concise
as the core vectorized operations leverage \texttt{PyTorch},
a well-maintained and optimized library for tensor computation.
%
The core WECT algorithm in the \pyect implementation is only about
$25$ lines of code, which although we acknowledge it is not a perfect metric,
it is a testament to the simplicity of the approach.

The \pyect package also contains a variety of tools for constructing and
working with simplicial/cubical complexes.
%
Our package includes fully vectorized methods for building weighted Freudenthal
complexes and weighted cubical complexes from images.
%
Additionally, \pyect has dedicated methods for the specific cases of computing
ECFs of images and WECTs of weighted geometric simplicial complexes in $\R^n$.
%
These two special cases are implemented as \texttt{PyTorch} neural network
modules.

Another benefit of the approach presented in this work is that vectorized
operations support automatic gradient computations.
%
Potential future work includes exploring the use of this framework in
topological machine learning applications and, in particular, incorporating
topological transforms into neural network architectures.
%
Additionally, we believe that this work will encourage further investigation
into vectorized methods for computing other topological functions and
transforms.
