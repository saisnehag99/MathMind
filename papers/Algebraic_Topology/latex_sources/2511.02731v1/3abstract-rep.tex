\section{Abstract representation theory} \label{sec:abstract-rep}

In this section, we set the framework for a representation theory with coefficients in abstract stable homotopy theories.

This article is mainly concerned with representations of quivers. By a \emph{quiver} we mean a quadruple $Q=(Q_0,Q_1,s,t)$ consisting of a set of vertices $Q_0$, a set of arrows $Q_1$, and maps $s,t:Q_1 \to Q_0$ corresponding to the source and target, respectively. A quiver is called \emph{finite} if both $Q_0$ and $Q_1$ are finite, \emph{acyclic} if it contains no oriented cycles, and a \emph{tree} if it contains no unoriented cycles. Classically, one identifies a quiver $Q$ with the small category freely generated by it, and \emph{representations} in a category $\mathcal{C}$ are nothing but functors $Q \to \mathcal{C}$. We recall that in the special case that $\mathcal{C} = \mathsf{Mod}\, R$ is a category of modules over a ring, the category of representations is (equivalent to) that of modules over the \emph{path algebra} $RQ$.

Since the emergence of tilting theory \cite{AngHapKra07}, representation theorists have heavily focused on understanding derived categories of representations and their exact equivalences. Originally, results were stated over a field on which they might depend, however, some of them are of such a combinatorial nature that they hold for arbitrary rings, or even abelian categories ---this is for example the case of reflection functors, c.f. \cref{sec:reflections}. We focus on this type of results and abstract the derived category of the coefficients. Our claim is that these results are actually formal consequences of stability \cite{GroSto18b}, and so they should hold over any stable homotopy theory.

\subsection{Homotopy coherent representations} 

Let $Q$ be a quiver and $\mathcal{A}$ an abelian category (e.g. modules or sheaves). We study derived representations by forming the derived category $\mathsf{D}(\mathcal{A}^Q)$, which is a localization of the category of complexes $\mathsf{Ch}(\mathcal{A}^Q)$. Via the canonical equivalence $\mathsf{Ch}(\mathcal{A}^Q) \simeq \mathsf{Ch}(\mathcal{A})^Q$, an object of $\mathsf{D}(\mathcal{A}^Q)$ can be identified with a representation of complexes in $\mathcal{A}$. However, an important observation is that such an identification cannot be made on morphisms:
\begin{equation} \label{eq:coh-diags}
\mathsf{D}(\mathcal{A}^Q) \not\simeq \mathsf{D}(\mathcal{A})^Q.
\end{equation}
This is a distinction between (homotopy) coherent and incoherent diagrams of complexes, respectively. More generally, if $\mathcal{T}$ is a triangulated category, then $\mathcal{T}^Q$ is no longer triangulated. Somehow the process of forming functor categories into a triangulated category forgets about the common homotopical behavior occurring in these categories, and this needs to be fixed. We replace $\mathcal{T}$ by an $\infty$-categorical enhancement $\mathcal{C}$, so that $\mathcal{T} = h\mathcal{C}$, and consider (homotopy coherent) diagrams of shape $Q$ in $\mathcal{C}$.

From now on, we identify a quiver $Q$ with the small $\infty$-category freely generated by it, that is, the nerve of the small category freely generated by it.

\begin{nota}
Given a small $\infty$-category $K$ and a stable $\infty$-category $\mathcal{C}$, we denote the (stable) $\infty$-category of \emph{homotopy coherent representations} by $\mathcal{C}^K = \Fun(K,\mathcal{C})$.
\end{nota}

We emphasize that, unlike the case of triangulated categories, the $\infty$-category $\mathcal{C}^K$ is still stable whenever $\mathcal{C}$ is (\Cref{prop:closure-stable}). Moreover, linearity is also inherited, as shown in \Cref{prop:closure-linear}.

\begin{rema}[Quivers inside simplicial sets] \label{rema:sset-quivers}
A quiver $Q$ can also be regarded as a $1$-dimensionsal simplicial set by considering the pushout
$$\begin{tikzcd} 
\coprod\limits_{\alpha \in Q_1} \partial\Delta^1 \arrow[r,hook] \arrow[d,"{(s,t)}"'] \ar[rd,phantom, "\PO", pos=0.3] &[-0.7em] \coprod\limits_{\alpha \in Q_1} \Delta^1 \arrow[d] \\ \coprod\limits_{v \in Q_0} \Delta^0 \arrow[r] & Q_\bullet 
\end{tikzcd}$$
in the category $\sSet$. Fortunately, there is no difference in terms of representations between $Q_\bullet$ and the $\infty$-category $N(\mathsf{free}(Q))$ freely generated by $Q$. Indeed, the evident comparison map $Q_\bullet \to N(\mathsf{free}(Q))$, sending a vertex to itself and an arrow to the corresponding path of length 1, is inner anodyne (see \cite[Proposition 1.5.7.3]{Lur25}). In particular, it induces a trivial fibration
$$\Fun(N(\mathsf{free}(Q)),\mathcal{C}) \xrightarrow{\ \simeq \ } \Fun(Q_\bullet,\mathcal{C}),$$
for any $\infty$-category $\mathcal{C}$ (see \cite[Theorem 1.5.7.1]{Lur25}). \qed 
\end{rema}

As opposed to \eqref{eq:coh-diags}, one of the key advantages of considering homotopy coherent diagrams is that the derived $\infty$-category of representations can be recovered as an $\infty$-category of representations, as shown in the following result.

\begin{prop} \label{prop:derived-infty-cat}
Let $\mathcal{G}$ be a Grothendieck abelian category and $J$ a small category. There is a canonical equivalence $\D{\mathcal{G}}^J \simeq \D{\mathcal{G}^J}$.

In particular, for any ring $R$ and any finite quiver $Q$, there is a canonical equivalence $\D{R}^Q \simeq \D{RQ}$. Moreover, if $Q$ is acyclic, then $\mathcal{D}^\mathsf{per}(R)^Q \simeq \mathcal{D}^\mathsf{per}(RQ)$.
\end{prop}
\begin{proof}
The first equivalence follows from \Cref{theo:models}. The last one follows from the first by taking compacts, where the acyclic hypothesis is necessary for the compact functors to equal the compact-valued functors (see \cite[Proposition 2.8]{Aok23}). 
\end{proof}

In fact, one can think of representations as modules in much more general situations. We give here a new definition of spectral path algebras.

\begin{defi} \label{def:path-algebra}
Let $A$ be an $\mathbb{E}_1$-ring and $K$ an $\infty$-category with finitely many objects. The \emph{spectral path algebra} $AK$ is the spectral endomorphism\footnote{Any stable $\infty$-category $\mathcal{C}$ is naturally \enquote{enriched} over spectra via the equivalence $\Fun^\mathsf{lex}(\mathcal{C}^\mathrm{op},\Spc) \simeq \Fun^\mathsf{lex}(\mathcal{C}^\mathrm{op},\Sp)$, see e.g. \cite[Corollary 1.4.2.23]{Lur17}.} algebra of $\bigoplus_{k\in K} k_!A$ as an object of $(\mathsf{Mod}_A)^K$, where $k_!$ denotes left Kan extension along $k:\Delta^0\to K$.
\end{defi} 

\begin{rema}
The objects $k_!A \in (\mathsf{Mod}_A)^K$ are analogues of indecomposable projectives at a vertex $k$.
\end{rema}

The above definition is justified by the following major theorem.

\begin{theo}[Schwede-Shipley, Lurie]
If $\mathcal{C}$ is a stable $\infty$-category compactly generated by an object $x$, then it is equivalent to $\mathsf{Mod}_R$ for the $\mathbb{E}_1$-ring $R = \mathsf{End}_\mathcal{C}(x)$.
\end{theo}
\begin{proof}
See \cite[Theorem 7.1.2.1 and Remarks 7.1.2.2 and 7.1.2.3]{Lur17}.
\end{proof}

\begin{coro}
There is a canonical equivalence $(\mathsf{Mod}_A)^K \simeq \mathsf{Mod}_{AK}$.
\end{coro}
\begin{proof}
We only need to note that if $K$ has finitely many objects and $\mathcal{C}$ is compactly generated by $x$, then $\bigoplus_{k\in K} k_!(x)$ is a compact generator of $\mathcal{C}^K$. Indeed, a finite sum of compacts is compact, and for the generation, taking $Y \in \mathcal{C}^K$:
$$0 = \Hom{h\mathcal{C}^K}(\Sigma^n(\textstyle\bigoplus\limits_{k\in K} k_!(x)),Y) \cong \textstyle\bigoplus\limits_{k\in K} \Hom{h\mathcal{C}}(\Sigma^nx,k^*Y), \text{ for all $n \in \mathbb{Z}$},$$
implies $Y_k = 0$ for all $k\in K$, and hence, $Y=0$.
\end{proof} 

Let $R$ be an $\mathbb{E}_\infty$-ring, so that $\mathsf{Mod}_R$ is presentably symmetric monoidal. Then $\mathsf{Mod}_{RQ}$ is the (presentable) $R$-linearization of $Q^\mathrm{op}$ in the following sense:

\begin{prop}
For any $R$-linear $\infty$-category $\mathcal{D}$, there is a canonical equivalence
$$\Fun^\mathsf{L}_R(\mathsf{Mod}_{RQ},\mathcal{D}) \xrightarrow{ \ \simeq \ } \Fun(Q^\mathrm{op},\mathcal{D}).$$
\end{prop}
\begin{proof}
One has the following canonical equivalences:
\begin{align*}
\Fun^\mathsf{L}_R(\mathsf{Mod}_{RQ},\mathcal{D}) &\simeq \Fun^\mathsf{L}_R(\mathsf{Mod}_R \otimes \Fun(Q,\Spc),\mathcal{D}) \\ &\simeq \Fun^\mathsf{L}(\Fun(Q,\Spc),\mathcal{D}) \simeq \Fun(Q^\mathrm{op},\mathcal{D})  
\end{align*}
which follow from \Cref{prop:closure-linear} and \cite[Theorem 5.1.5.6]{Lur09}.
\end{proof}

\begin{coro} \label{coro:bimodules-reps}
There is a canonical equivalence $\mathsf{Mod}_R^{Q^\mathrm{op}\times Q} \simeq {}_{RQ}\mathsf{BMod}_{RQ}(\mathsf{Mod}_R)$.
\end{coro}
\begin{proof}
There are equivalences:
$$\mathsf{Mod}_R^{Q^\mathrm{op}\times Q} \simeq \mathsf{Mod}_{RQ}^{Q^\mathrm{op}} \simeq \Fun^\mathsf{L}_R(\mathsf{Mod}_{RQ},\mathsf{Mod}_{RQ}) \simeq {}_{RQ}\mathsf{BMod}_{RQ}(\mathsf{Mod}_R)$$
where the last one is Eilenberg-Watts (\Cref{example:Eilenberg-Watts}).
\end{proof}

When passing to representations over arbitrary presentable stable $\infty$-categories, the universality of spectra allows certain reductions to spectral representations. We have found the following valuable description which uses Lurie's tensor product of presentable $\infty$-categories (\Cref{cons:tensor-presentable}).

\begin{prop} \label{prop:univ-eq}
Let $K$ be a small $\infty$-category and $\mathcal{C}$ a presentable stable $\infty$-category. Then there is a canonical equivalence $\mathcal{C}^K \simeq \Sp^K \otimes \mathcal{C}$. 
\end{prop}
\begin{proof}
One has the following canonical equivalences:
\begin{align*}
\Fun(K,\mathcal{C}) &\simeq \Fun(K^\mathrm{op},\mathcal{C}^\mathrm{op})^\mathrm{op} \\
&\simeq \Fun^\mathsf{L}(\Fun(K,\Spc),\mathcal{C}^\mathrm{op})^\mathrm{op} \\
&\simeq \Fun^\mathsf{L}(\Sp(\Fun(K,\Spc)),\mathcal{C}^\mathrm{op})^\mathrm{op} \\
&\simeq \Fun^\mathsf{L}(\Fun(K,\Sp),\mathcal{C}^\mathrm{op})^\mathrm{op} \\
&\simeq \Fun^\mathsf{R}(\Fun(K,\Sp)^\mathrm{op},\mathcal{C}) \simeq \Fun(K,\Sp) \otimes \mathcal{C} 
\end{align*}
where we have used (in order) \cite[Theorem 5.1.5.6]{Lur09}, \Cref{prop:stabilization}, and \cite[Remark 1.4.2.9 and Proposition 4.8.1.17]{Lur17}.
\end{proof}

\begin{coro} \label{coro:sp-equiv}
Let $K$ and $K'$ be small $\infty$-categories. Any equivalence $\Sp^K \simeq \Sp^{K'}$ induces natural equivalences $\mathcal{C}^K\simeq \mathcal{C}^{K'}$ for all presentable stable $\infty$-categories $\mathcal{C}$.
\end{coro}

When restricting to finite acyclic quivers\footnote{This also works for finite directed categories, as mentioned in the proof of \Cref{prop:univ-eq-small}.}, there is a \enquote{small} version of the previous results:

\begin{prop} \label{prop:univ-eq-small}
Let $Q$ be a finite acyclic quiver and $\mathcal{C}$ any stable $\infty$-category. Then there is a canonical equivalence $\mathcal{C}^Q \simeq \Fun^\mathsf{ex}((\mathsf{fSp}^Q)^\mathrm{op},\mathcal{C})$. 
\end{prop}
\begin{proof}
Let $\Sp^{Q,\, \mathrm{fin}}$ denote the smallest stable subcategory of $\Sp^Q$ containing the image of the spectral Yoneda functor $Q^\mathrm{op} \to \Sp^Q$ (i.e. Yoneda composed with $\Sigma^\infty$). This has the following universal property: composition with $Q^\mathrm{op} \to \Sp^{Q,\, \mathrm{fin}}$ induces an equivalence $\Fun^\mathsf{ex}(\Sp^{Q,\, \mathrm{fin}},\mathcal{C}) \isoarrow \Fun(Q^\mathrm{op},\mathcal{C})$, see \cite[Proposition 2.2.7]{Lur15}. But for $Q$ at least a finite directed category, and in particular for finite acyclic quivers, it follows that $\Sp^{Q,\, \mathrm{fin}} = \fSp^Q = (\Sp^Q)^c$ (see the proof of \cite[Proposition 2.2.6]{Lur15} and \cite[Proposition 2.8]{Aok23}).
Hence one has the following canonical equivalences:
\begin{equation*}
\Fun(Q,\mathcal{C}) \simeq \Fun(Q^\mathrm{op},\mathcal{C}^\mathrm{op})^\mathrm{op}
\simeq \Fun^\mathsf{ex}(\fSp^Q,\mathcal{C}^\mathrm{op})^\mathrm{op}
\simeq \Fun^\mathsf{ex}((\fSp^Q)^\mathrm{op},\mathcal{C}) \qedhere
\end{equation*}
\end{proof}

\begin{coro}\label{coro:sp-equiv-quivers}
Let $Q$ and $Q'$ be finite acyclic quivers. Any equivalence $\Sp^Q \simeq \Sp^{Q'}$ induces natural equivalences $\mathcal{C}^Q\simeq \mathcal{C}^{Q'}$ for all stable $\infty$-categories $\mathcal{C}$.
\end{coro}
\begin{proof}
Simply note that equivalences $\Sp^Q \simeq \Sp^{Q'}$ correspond to equivalences of their compacts $(\Sp^Q)^c = \fSp^Q$, for $Q$ finite acyclic (see \cite[Proposition 2.8]{Aok23}).
\end{proof}

\subsection{Stably equivalent shapes} 

As we already mentioned, one of our main motivations comes from tilting theory and its generalizations. We say that two quivers $Q$ and $Q'$ are \emph{derived equivalent} over a field $k$ if there is an exact equivalence
$$\mathsf{D}^b(kQ) \stackrel{\triangle}{\simeq} \mathsf{D}^b(kQ')$$
between their derived categories. When such an equivalence is of combinatorial nature, it can often lead to derived equivalences of representations over arbitrary abelian categories ---these are called \emph{universal derived equivalences} by Ladkani \cite{Lad07,Lad08}. In the same spirit, Groth and Stovicek \cite{GroSto18b} define stably equivalent shapes in the setting of abstract representation theory.

\begin{defi} \label{def:stable-eq}
Let $K$ and $K'$ be two small $\infty$-categories. A \emph{stable equivalence} between them is a natural equivalence of (ordinary) functors $$\varphi: (-)^K \simeq (-)^{K'}: h\infCAT^\mathsf{ex} \to h\infCAT.$$ 
This means a collection of equivalences $\varphi_\mathcal{C}: \mathcal{C}^K \isoarrow \mathcal{C}^{K'}$, one for each stable $\infty$-category $\mathcal{C}$, such that for each exact functor $f:\mathcal{C}\to\mathcal{D}$, the square
$$\begin{tikzcd} \mathcal{C}^K \ar[r,"\varphi_\mathcal{C}"] \ar[d,"f_*"'] & \mathcal{C}^{K'} \ar[d,"f_*"] \\
\mathcal{D}^K \ar[r,"\varphi_\mathcal{D}"] & \mathcal{D}^{K'}
\end{tikzcd}$$
homotopy commutes. When it exists, we say $K$ and $K'$ are \emph{stably equivalent}.
\end{defi}

\begin{example}
For finite acyclic quivers $Q$ and $Q'$, any equivalence $\Sp^Q \simeq \Sp^{Q'}$ induces a stable equivalence between the two shapes (\Cref{coro:sp-equiv-quivers}).
\end{example}

The interest in finding stably equivalent shapes is double. First and most obvious, they vastly generalize classical derived equivalences, and could provide more conceptual explanations of those. Second, any such equivalence exhibits a potentially interesting \emph{symmetry in abstract stable homotopy theory}. For example, the reader might appreciate a new perspective on May's axioms for $\otimes$-triangulated categories in terms of an equivalence between the quiver $D_4$ and the commutative square \cite[sec. 10]{GroSto18b}, or an application of the abstract representation theory of Dynkin quivers of type $A_n$ to the construction of higher triangulations \cite[sec. 13]{GroSto16}. 

\begin{rema}
One could also ask for a stronger version of \Cref{def:stable-eq} by requiring an equivalence of functors $(-)^K \simeq (-)^{K'}: \infCAT^\mathsf{ex} \to \infCAT$, which means requiring all higher naturality conditions. However, it is often too difficult to prove such naturality when constructing new stable equivalences, and also not so common to use it. Moreover, in practice, one can build an equivalence over spectra and extend it to all other stable $\infty$-categories using \Cref{coro:sp-equiv-quivers}, which gives all higher naturality.
\end{rema}

For finite acyclic quivers we can already give a complete answer to the questions of when two of them are stably equivalent; this is thanks to Happel's foundational work \cite{Hap88} together with the recent construction of abstract BGP reflection functors by Dyckerhoff, Jasso and Walde \cite{DycJasWal21} (c.f. also \cref{sec:reflections}). Here we denote $\mathbb{Z}Q$ the so-called \emph{repetitive quiver} of $Q$ (see a definition in \cref{sec:equiv-repet}).

\begin{theo} \label{theo:equiv-quivers}
For $Q$ and $Q'$ finite acyclic quivers, the following are equivalent:
\begin{enumerate}
    \item $Q$ and $Q'$ are related by a sequence of source or sink reflections.
    \item $\mathbb{Z}Q \cong \mathbb{Z}Q'$ as translation quivers.
    \item $Q$ and $Q'$ are derived equivalent over a field $k$.
    \item $Q$ and $Q'$ are equivalent over spectra, i.e. $\Sp^Q \simeq \Sp^{Q'}$.
    \item $Q$ and $Q'$ are stably equivalent.
\end{enumerate}
\end{theo}
\begin{proof}
The equivalence between (1), (2) and (3) is known since Happel: (1) $\Leftrightarrow$ (2) is \cite[Lemma 5.7]{Hap88}, (3) $\Rightarrow$ (2) is \cite[Corollary 5.7]{Hap88}, and (1) $\Rightarrow$ (3) holds since reflection functors are represented by tilting modules. The implication (5) $\Rightarrow$ (4) is clear, and for (4) $\Rightarrow$ (3) we use \Cref{coro:sp-equiv-quivers} with $\mathcal{C} = \Dd{b}{kQ}$ and take homotopy categories. Finally, (1) $\Rightarrow$ (5) follows by the construction of abstract reflection functors in \cite[Corollary 2.6]{DycJasWal21} (c.f. also \cref{sec:reflections} for an alternative proof).
\end{proof}

\begin{coro}
Two oriented trees are stably equivalent if and only if they have the same underlying (unoriented) graph.
\end{coro}

Given \Cref{theo:equiv-quivers}, we are moved to the problem of trying to classify all stable equivalences between quivers. 
In this line, one would like to compute the groups of autoequivalences\footnote{If $Q$ and $Q'$ are stably equivalent and we fix $\varphi:(-)^Q\simeq(-)^{Q'}$ (e.g. reflection functors), then composition with $\varphi$ identifies equivalences $(-)^Q \simeq (-)^{Q'}$ and autoequivalences of $(-)^Q$.} (up to natural equivalence) of abstract representations $\mathcal{C}^Q$, and particularly the group of autoequivalences of $\Sp^Q$. This is the so-called \emph{spectral Picard group} of $Q$, that we understand as a group of \emph{universal symmetries} in stable homotopy theory (c.f. \Cref{coro:sp-equiv-quivers}). Inspired by work of Happel (see \Cref{theo:equiv-quivers}(2), for instance) and of Miyachi and Yekutieli \cite{MiyYek01}, we look in the next few sections for interesting autoequivalences in the shape of the repetitive quiver and more generally of the Auslander-Reiten quiver. Important applications to the computation of spectral Picard groups are given in \cref{sec:picard}.