\newpage

%==================================================================
%  Appendix A.  Technical lemmata for Section “Broad geometry”
%==================================================================

\appendix
\section{Details used in~\texorpdfstring{\S\ref{sec:broad3}}%
{Section~\ref{sec:broad3}}}%
\label{app:det}

In \S\ref{sec:broad3} (broad rank–3 geometry) we use several
geometric/algebraic facts. For convenience of the reader, we collect them
here; numbering starts at~\ref{app:A1} and proceeds in strict order.

%------------------------------------------------------------------
\subsection{Asymptotics of the normal}\label{app:A1}
%------------------------------------------------------------------

\begin{lemma}\label{lem:A1}
Let \( |\xi| \sim \lambda \) and \(s:=|\xi|\). Then
\[
   n(\xi)
   \;=\;
   \frac{(-\xi,\,1/2)}{s}
   \;-\; \frac{1}{8s^{3}}\Bigl(-\xi,\,\tfrac{1}{2}\Bigr)
   \;+\; R(\xi),
   \qquad |R(\xi)| \;\le\; C\,s^{-5}\lesssim C\,\lambda^{-5}.
\]
\end{lemma}

\begin{corollary}\label{cor:A1}
Set \(a(\xi):=\dfrac{(-\xi,\,1/2)}{|\xi|}\) and \(\rho(\xi):=n(\xi)-a(\xi)\).
Then uniformly for \(|\xi|\sim\lambda\),
\[
  |\rho(\xi)|\;\lesssim\;\lambda^{-2},
  \qquad
  |\rho_{\!\perp}(\xi)|\;\lesssim\;\lambda^{-2},
\]
where \(\rho_{\!\perp}\) is the projection of \(\rho\) onto the tangent plane to the sphere at \(\xi/|\xi|\).
\end{corollary}

\begin{proof}[Sketch of the lemma]
From the exact formula \(n(\xi)=(-2\xi,1)/\sqrt{1+4s^{2}}\) and the expansion
\[
(1+4s^{2})^{-1/2}
= \frac{1}{2s}\Bigl(1-\frac{1}{8s^{2}}+\frac{3}{128s^{4}}+O(s^{-6})\Bigr)
\]
we obtain the asserted asymptotics. The bound \(|\rho|+|\rho_{\!\perp}|\lesssim s^{-2}\)
follows immediately.
\end{proof}

\emph{Remark.}
In the first two terms of the expansion we do \emph{not} replace \(s\) by \(\lambda\); all estimates
are uniform for \(|\xi|\sim\lambda\). If finer radial localization is required,
the substitution \(s\mapsto\lambda\) is absorbed into the remainder \(R(\xi)\).

%------------------------------------------------------------------
\subsection{Packing of six directions}\label{app:A2}
%------------------------------------------------------------------

\begin{lemma}[$4$ out of $6$]\label{lem:A2}
For any six points \(\xi_{1},\dots,\xi_{6}\in\mathbb{S}^{2}\) one can
choose four whose pairwise angles are at least
\(\alpha=c_{0}rD^{1/2}\). Moreover, the number of “dense’’ (i.e., $<\alpha$) angles
among the sextuple is at most three.
\end{lemma}

\begin{proof}
A spherical cap of radius \(\alpha\) has area \(\asymp\alpha^{2}\).
If there were \(\ge4\) dense pairs, double counting of their caps would cover
the whole sphere; a contradiction.
\end{proof}

\begin{corollary}\label{cor:A2}
For the matrix
\(B=[\xi_{2}-\xi_{1}\;\xi_{3}-\xi_{1}\;\xi_{4}-\xi_{1}]\)
one has
\(|\det B|\gtrsim(\lambda rD^{1/2})^{3}\).
\end{corollary}



%------------------------------------------------------------------
\subsection{Mixed minors}\label{app:A3}
%------------------------------------------------------------------

\begin{lemma}\label{lem:A3}
Let \(a_{j}:=a(\xi_{i_{j}})=\dfrac{(-\xi_{i_{j}},\,1/2)}{|\xi_{i_{j}}|}\) and
\(\rho:=\rho(\xi_{k})\) be as in Cor.~\ref{cor:A1}. Then
\[
  \bigl|\det[a_{1}\;a_{2}\;a_{3}\;\rho]\bigr|
  \;\le\;
  |a_{1}\wedge a_{2}\wedge a_{3}|\;\,|\rho_{\!\perp}|
  \;\lesssim\; C\,\lambda^{-3-\frac{2}{3}}.
\]
\end{lemma}

\begin{proof}
Decompose \(\rho=\rho_{\parallel}+\rho_{\perp}\). The contribution of \(\rho_{\parallel}\) yields zero determinant
(linear dependence of columns). For \(\rho_{\perp}\) we have
\(\bigl|\det[a_{1}\;a_{2}\;a_{3}\;\rho_{\perp}]\bigr|
\le |a_{1}\wedge a_{2}\wedge a_{3}|\;|\rho_{\!\perp}|\).
By Cor.~\ref{cor:A1}, \(|\rho_{\!\perp}|\lesssim\lambda^{-2}\). The estimate
\(|a_{1}\wedge a_{2}\wedge a_{3}|\lesssim (2\lambda)^{-3}\cdot \lambda r^{2}D^{1/2}\)
follows from angular sparsity (Lemma~\ref{lem:A2}) and a standard Gram control;
together this even gives the stronger upper bound \(\lesssim \lambda^{-16/3}D^{1/2}\).
In the statement of the lemma it suffices to record the coarser consequence \(\lesssim \lambda^{-3-2/3}\),
which is the one used later.
\end{proof}

%------------------------------------------------------------------
\subsection{Gram control for crowding}\label{app:A4}
%------------------------------------------------------------------

\begin{lemma}\label{lem:A4}
If exactly $m\le3$ angles among
$a_{1},a_{2},a_{3}$ do not exceed~$\alpha$, then
\[
  \det G\ge 1-m\alpha^{2}\ge c\,D^{-3/2},
\]
where $G=(a_{i}\!\cdot a_{j})_{1\le i,j\le3}$.
\end{lemma}

\begin{proof}
Use the identity
$\det G=
  1-\sum_{a<b}\!\sin^{2}\theta_{ab}
  +2\prod_{a<b}\!\sin\theta_{ab}$.
Since $\alpha\ll1$,
the second term is $\le2\alpha^{m}$; for $m\le3$
we obtain the desired lower bound.
\end{proof}

%------------------------------------------------------------------
\subsection{Summary for the $4\times4$ minors}\label{app:A5}
%------------------------------------------------------------------

From Cor.~\ref{cor:A1}–\ref{cor:A2} and
Lemmas~\ref{lem:A3}–\ref{lem:A4} it follows: the principal
$4\times4$ minor of the six normals $\{n(\xi_{j})\}$ satisfies
\[
  \operatorname{Vol}_{4}\bigl(n_{1},\dots,n_{6}\bigr)\;
  \gtrsim\;\lambda^{-3}D^{3/2},
\]
while the “mixed’’ minors are $\le C\lambda^{-3-\frac23}$, with crowding loss
no worse than $D^{3/2}$.

\subsection{Partition into $O(D)$ classes}
\label{app:A6}
\begin{lemma}[Angular partition into $O(D)$ classes]\label{lem:AD-colouring}
A family $\{\Theta\}$ of caps of radius $r$ with centers $|\xi|\sim\lambda$ can be partitioned into $\le C D$ disjoint
classes $C_1,\dots,C_M$ such that for any two caps in the same class one has $\angle(\xi_\Theta,\xi_{\Theta'})\ge\alpha$.
Here $D=\lambda^{1/12}$, $\alpha=c_0 r D^{1/2}$, and $C>0$ is an absolute constant.
\end{lemma}

\begin{proof}[Idea of proof]
Consider the graph with vertices $\{\xi_\Theta\}$, joining by an edge pairs with angle $<\alpha$.
Each vertex has at most $C'(\alpha/r)^2\sim C'D$ neighbors (area of the angular annulus versus cap area).
Hence the maximum degree $\Delta\lesssim D$, and a greedy coloring yields at most $\Delta+1\le C D$ colors.
\end{proof}

%------------------------------------------------------------------
\subsection{Tubular estimate in $\mathbb{R}^4$}\label{app:A7}
%------------------------------------------------------------------

\begin{lemma}[Tubular measure (Wongkew at the scale $Q_\lambda$)]\label{lem:poly-sublevel-4d}
Let $P:\mathbb{R}^4\to\mathbb{R}$ be a polynomial of degree $d:=\deg P \le D^{1/4}$ and
$Z(P):=\{(t,x):\,P(t,x)=0\}$. Introduce the anisotropic mapping
\[
S_\lambda(t,x):=(\lambda^{3/2}t,\ \lambda^{1/2}x),
\qquad
\operatorname{dist}_\lambda\bigl((t,x),Z\bigr)
:=\operatorname{dist}\bigl(S_\lambda(t,x),\,S_\lambda(Z)\bigr),
\]
and set for a small absolute constant $c>0$
\[
\beta:=\frac{c\,D^{-1}}{d},\qquad
\mathcal N_\beta(P):=\Bigl\{(t,x)\in Q_\lambda:\ \operatorname{dist}_\lambda\bigl((t,x),Z(P)\bigr)<\beta\Bigr\}.
\]
Then there exists an absolute constant $C>0$ such that
\[
\bigl|Q_\lambda\cap \mathcal N_\beta(P)\bigr|\ \le\ C\,D^{-1}\,|Q_\lambda|.
\]
\end{lemma}

\begin{proof}[Outline]
Let $P_\lambda:=P\circ S_\lambda^{-1}$. Then $S_\lambda\!\bigl(\mathcal N_\beta(P)\bigr)$ is a Euclidean $r$–tube
around $Z(P_\lambda)$ of thickness $r=\beta$ in a unit–scale region.
By Wongkew’s theorem \cite{Wongkew2003} in $\mathbb{R}^4$ the volume of such a tube is $\lesssim d\,r$.
Taking $r=\beta=cD^{-1}/d$ and returning to the original variables, we get
\(
|Q_\lambda\cap \mathcal N_\beta(P)|\lesssim \beta\,|Q_\lambda|\lesssim D^{-1}|Q_\lambda|.
\)
\end{proof}

\begin{remark}
Classical sublevel estimates of the form $\mathrm{meas}\{|P|<\beta\}\lesssim C(d)\,\beta^{1/d}$ do not give the desired $D^{-1}$
when $d\le D^{1/4}$ and $\beta=D^{-1/4}$. The tubular formulation with the choice $\beta=cD^{-1}/d$ and the use
of Wongkew’s estimate yields the precise $D^{-1}$ factor at the scale of $Q_\lambda$.
\end{remark}



%------------------------------------------------------------------
\subsection{Packing in angular rings}\label{app:A8}
%------------------------------------------------------------------

\begin{lemma}[Packing in $k$–rings]\label{lem:A8}
Let $\{\Theta\}$ be a family of caps of radius $r=\lambda^{-2/3}$ with centers
$\xi_\Theta\in\mathbb{S}^2$, and let $\alpha=c_0\,r\,D^{1/2}$ with $D=\lambda^{1/12}$.
Then an angular ring of width $\delta\sim k\alpha$ contains at most $C\,k^2 D$ caps,
where $C>0$ is an absolute constant.
\end{lemma}

\begin{proof}
The area of a spherical angular ring of width $k\alpha$ on $\mathbb{S}^2$ is
$\asymp (k\alpha)^2$, while the area of one cap is $\asymp r^2$.
Hence
\[
  \#\{\Theta\ \text{in the ring}\}
  \;\lesssim\;
  \Bigl(\tfrac{k\alpha}{r}\Bigr)^2
  \;=\; k^2\Bigl(\tfrac{\alpha}{r}\Bigr)^2
  \;=\; k^2 D,
\]
since $\alpha/r=D^{1/2}$.
\end{proof}

\begin{remark}[Thin ring]\label{rem:A11}
If we consider a \emph{thin} ring of radius $\sim k\alpha$ and fixed thickness $\alpha$
(rather than $k\alpha$), then the area of such a ring is $\asymp k\,\alpha^2$, and therefore
\[
  \#\{\Theta\ \text{in a thin ring of thickness } \alpha\}\;\lesssim\;
  \frac{k\,\alpha^2}{r^2}\;=\;kD.
\]
This form ($\lesssim kD$) is precisely what is used in the dyadic angular decomposition in \S\ref{subsec:tube-L2}.
\end{remark}
