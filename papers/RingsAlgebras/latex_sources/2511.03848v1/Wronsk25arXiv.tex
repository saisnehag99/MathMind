\documentclass[a4paper,12pt,reqno]{amsart}%{bjp}%twoside
%\documentclass[12pt]{article}
%\documentclass[12pt]{iopconfser}
\usepackage{amssymb,amsfonts,amsmath,amscd,amsthm,latexsym,euscript}%,textcomp,amscd,
%\usepackage{graphicx}
\usepackage{cite}
\usepackage{times}

% Edited last by: AVK
% Date: 5 November 2025

\usepackage[bookmarks=false]{hyperref}
\hypersetup{%
    colorlinks=true,        % false: boxed links; true: colored links
    linkcolor=blue,          % color of internal links (change box color with linkbordercolor)
    citecolor=blue,         % color of links to bibliography
    urlcolor=blue           % color of external links
    }

%\usepackage{underscore}

%%%%%%%%%%%%%%%%%%
%\usepackage{geometry}
% \geometry{
% a4paper,
% total={112mm,186mm},
% left=49mm,
% top=55mm,
% }

%\usepackage{biblatex}
%\addbibresource{bibliography.bib}

%\bibliographystyle{iopart-num-long}
%\usepackage{cite}
%\newcommand{\BibTeX}{Bib\TeX}
%\newcommand{\REVTeX}{REV\TeX}

%\usepackage[a4paper, total={6in, 8.5in}]{geometry} %makes margins smaller
%\usepackage{hyperref} %makes sections clickable
%\usepackage{makecell} %Can use \thead{...} in table in order to have line breaks in cells

%\usepackage{amsmath,amssymb,amsthm}
%\usepackage{stmaryrd}
%\usepackage{wasysym}
%\usepackage{multicol}
%\usepackage{multirow}
%\usepackage{color}
%\usepackage{dirtytalk}
\usepackage{url}
%\usepackage{fontawesome} 
%\usepackage{csquotes}
%%%%%%%%%%%%%%%%%%%%%%%%%%%%%%%%%%%%%%%%

%\setlength{\footskip}{40pt} %moves page number down CHANGE FOR ARXIV

% PAGE SIZES
%--------------------------------------------------------------------%
\setlength{\headheight}{32pt}
\setlength{\headsep}{29pt}
\setlength{\footskip}{28pt}
\setlength{\textwidth}{444pt}
\setlength{\textheight}{636pt}
\setlength{\marginparsep}{7pt}
\setlength{\marginparpush}{7pt}
\setlength{\oddsidemargin}{4.5pt}
\setlength{\marginparwidth}{55pt}
\setlength{\evensidemargin}{4.5pt}
\setlength{\topmargin}{-15pt}
\setlength{\footnotesep}{8.4pt}
%--------------------------------------------------------------------%
\setcounter{tocdepth}{1}

%%%%%%%%%%%%%%%%%%%%%%%%%%%%%%%%%%%%%%%%

\newtheorem{theor}{Theorem}%[section]
\newtheorem{claim}[theor]{Claim}
\newtheorem*{claimNo}{Claim}
%%%
\theoremstyle{definition}
\newtheorem{simp}{Simplification}
\newtheorem*{theorNo}{Theorem}
\newtheorem*{convention}{Convention}
\newtheorem{prop}[theor]{Proposition}%[section]
\newtheorem{proposition}[theor]{Proposition}%[section]
\newtheorem{lem}[theor]{Lemma}%[section]
\newtheorem{cor}[theor]{Corollary}%[section]
\newtheorem{conjecture}[theor]{Conjecture}%[section]
\newtheorem{define}{Definition}%[section]
\newtheorem*{defNo}{Definition}
\newtheorem*{prob}{Problem formulation}
\newtheorem*{comment}{Comment}
\newtheorem{condition}{Condition}
\newtheorem*{observation}{Observation}
\newtheorem{notation}{Notation}
\newtheorem{problem}{Problem}%[section]
%\newtheorem{open}[problem]{Open problem}
\newtheorem{open}{Open problem}
\newtheorem{discussion}{Discussion}
\newtheorem{property}{Property}%[section]
\newtheorem{ex}{Example}%[section]
\newtheorem{nonex}[ex]{Non-example}%[section]
\newtheorem{counterexample}[ex]{Counterexample}%[section]
\newtheorem{exercise}{Exercise}%[section]
\newtheorem{implement}{Implementation}%[section]
\newtheorem{method}{Method}%[section]
\newtheorem{ax}{Axiom} %added later
%%%
\theoremstyle{remark}
\newtheorem{rem}{Remark}%[section]
\newtheorem*{remNo}{Remark}
\newtheorem*{Finalrem}{Final remark}
\theoremstyle{definition}
\newtheorem{idea}{Idea}%[section]
%\newtheorem*{ideaNo}{Idea}
\theoremstyle{definition}
\newtheorem{optimisation}{Optimisation point}%[section]

\def\MB{\mathbf}
\def\MC{\mathcal}
\def\MR{\mathrm}
\def\BB{\mathbb}
\def\R{\BB{R}}
\def\C{\BB{C}}

\def\wh{\widehat}
\def\ol{\overline}
\def\id{\operatorname{\MR{id}}\nolimits}
\def\Tr{\operatorname{\MR{Tr}}\nolimits}
%\usepackage{bm}

\newcommand{\BBR}{\mathbb{R}}\newcommand{\BBC}{\mathbb{C}}
\newcommand{\BBF}{\mathbb{F}}\newcommand{\BBN}{\mathbb{N}}
\newcommand{\BBS}{\mathbb{S}}\newcommand{\BBT}{\mathbb{T}}
\newcommand{\BBZ}{\mathbb{Z}}\newcommand{\BBE}{\mathbb{E}}
\newcommand{\BBP}{\mathbb{P}}
\newcommand{\EuA}{{{\EuScript A}}}
\newcommand{\EuX}{{{\EuScript X}}}
\newcommand{\cA}{{{\EuScript A}}}%{\mathcal{A}}
\newcommand{\bcA}{\boldsymbol{\mathcal{A}}}
\newcommand{\mcA}{\mathcal{A}}
\newcommand{\bcP}{{\boldsymbol{\mathcal{P}}}}
\newcommand{\bcQ}{{\boldsymbol{\mathcal{Q}}}}
\newcommand{\cB}{\mathcal{B}}
\newcommand{\cC}{\mathcal{C}}\newcommand{\tcC}{\smash{\widetilde{\mathcal{C}}}}
\newcommand{\cD}{\mathcal{D}}
\newcommand{\tcX}{\smash{\widetilde{\mathcal{X}}}}
\newcommand{\cE}{\mathcal{E}}\newcommand{\tcE}{\smash{\widetilde{\mathcal{E}}}}
\newcommand{\cEL}{\mathcal{E}_{\IL}}
\newcommand{\cEEL}{{\cE}_{\text{\textup{EL}}}}
\newcommand{\cEKdV}{{\cE}_{\text{\textup{KdV}}}}
\newcommand{\cELiou}{{\cE}_{\text{\textup{Liou}}}}
\newcommand{\cEToda}{{\cE}_{\text{\textup{Toda}}}}
\newcommand{\cF}{\mathcal{F}}
\newcommand{\cH}{\mathcal{H}}
\newcommand{\cN}{\mathcal{N}}
\newcommand{\cI}{\mathcal{I}}
\newcommand{\cJ}{\mathcal{J}}
\newcommand{\cL}{\mathcal{L}}
\newcommand{\cO}{\mathcal{O}}
\newcommand{\cP}{\mathcal{P}}\newcommand{\cR}{\mathcal{R}}
\newcommand{\cQ}{\mathcal{Q}}
\newcommand{\cU}{\mathcal{U}}
\newcommand{\cV}{\mathcal{V}}
\newcommand{\cW}{\mathcal{W}}
\newcommand{\cX}{{\EuScript X}}    %{\mathcal{X}}
\newcommand{\cY}{{\EuScript Y}}    %{\mathcal{Y}}
\newcommand{\cZ}{{\EuScript Z}}    %{\mathcal{Y}}
\newcommand{\boldb}{{\boldsymbol{b}}}
\newcommand{\Bone}{{\boldsymbol{1}}}
\newcommand{\bc}{{\mathbf{c}}}
\newcommand{\ba}{{\boldsymbol{a}}}
\newcommand{\bb}{{\boldsymbol{b}}}
\newcommand{\bbD}{{\boldsymbol{\mathrm{D}}}}
\newcommand{\bbf}{{\boldsymbol{f}}}
\newcommand{\bi}{{\boldsymbol{i}}}
\newcommand{\bn}{{\boldsymbol{n}}}
\newcommand{\bp}{{\boldsymbol{p}}}
\newcommand{\bq}{{\boldsymbol{q}}}
\newcommand{\br}{{\boldsymbol{r}}}
\newcommand{\bs}{{\boldsymbol{s}}}
\newcommand{\bu}{{\boldsymbol{u}}}
\newcommand{\bv}{{\boldsymbol{v}}}
\newcommand{\bw}{{\boldsymbol{w}}}
\newcommand{\bx}{{\boldsymbol{x}}}
\newcommand{\bX}{{\mathbf{x}}}
\newcommand{\bby}{{\boldsymbol{y}}}
\newcommand{\bz}{{\boldsymbol{z}}}
\newcommand{\bA}{{\boldsymbol{A}}}
\newcommand{\bD}{{\boldsymbol{D}}}
\newcommand{\bF}{{\boldsymbol{F}}}
\newcommand{\bH}{{\boldsymbol{H}}}
\newcommand{\bL}{{\boldsymbol{L}}}
\newcommand{\bN}{{\boldsymbol{N}}}
\newcommand{\bP}{{\boldsymbol{P}}}
\newcommand{\bQ}{{\boldsymbol{Q}}}
\newcommand{\bbU}{{\boldsymbol{U}}}
\newcommand{\bV}{{\boldsymbol{V}}}
\newcommand{\bE}{\mathbf{E}}
\newcommand{\bR}{\mathbf{r}}
\newcommand{\bS}{{\boldsymbol{S}}}
\newcommand{\bal}{{\boldsymbol{\alpha}}}
\newcommand{\bpi}{{\boldsymbol{\pi}}}
\newcommand{\bpsi}{{\boldsymbol{\psi}}}
\newcommand{\bxi}{{\boldsymbol{\xi}}}
\newcommand{\bet}{{\boldsymbol{\eta}}}
\newcommand{\bom}{{\boldsymbol{\omega}}}
\newcommand{\bOm}{{\boldsymbol{\Omega}}}
\newcommand{\bPhi}{{\boldsymbol{\Phi}}}
\newcommand{\bU}{\mathbf{U}}
\newcommand{\binfty}{\pmb{\infty}}
\newcommand{\BOne}{{\boldsymbol{1}}}
\newcommand{\BTwo}{{\boldsymbol{2}}}
\newcommand{\bsquare}{\pmb{\square}}
\newcommand{\bun}{\mathbf{1}}
\newcommand{\ga}{\mathfrak{a}}
\newcommand{\gothe}{\mathfrak{e}}
\newcommand{\gf}{\mathfrak{f}}
\newcommand{\hgf}{\smash{\widehat{\mathfrak{f}}}}
\newcommand{\gh}{\mathfrak{h}}
\newcommand{\hgh}{\smash{\widehat{\mathfrak{h}}}}
\newcommand{\gm}{\mathfrak{m}}
\newcommand{\gothg}{\mathfrak{g}}
\newcommand{\gotht}{\mathfrak{t}}
\newcommand{\gu}{\mathfrak{u}}
\newcommand{\gA}{\mathfrak{A}}
\newcommand{\gB}{\mathfrak{B}}
\newcommand{\gN}{\mathfrak{N}}
\newcommand{\gM}{\mathfrak{M}}
\newcommand{\veps}{\varepsilon}
\newcommand{\vph}{\varphi}
%\newcommand{\dd}{\partial}
\newcommand{\Id}{{\mathrm d}}
\newcommand{\ID}{{\mathrm D}}
\newcommand{\IL}{{\mathrm L}}
\newcommand{\rmi}{{\mathrm i}}
\newcommand{\rP}{{\mathrm P}}
\newcommand{\fnh}{{\text{\textup{FN}}}}
\newcommand{\rme}{{\mathrm{e}}}
\newcommand{\rmN}{{\mathrm{N}}}
\newcommand{\uu}{{\underline{u}}}
\newcommand{\uv}{{\underline{v}}}
\newcommand{\uw}{{\underline{w}}}
\newcommand{\sft}{{\mathsf{t}}}
\newcommand{\vx}{{\vec{\mathrm{x}}}}
\newcommand{\vy}{{\vec{\mathrm{y}}}}
\newcommand{\vz}{{\vec{\mathrm{z}}}}
\newcommand{\bvx}{{\vec{\mathbf{x}}}}
\newcommand{\vt}{{\vec{\mathrm{t}}}}
\newcommand{\vgdd}{{\vec{\mathfrak{d}}}}

\newcommand{\bbx}{{\boldsymbol{x}}}
\newcommand{\bbu}{{\boldsymbol{u}}}
\newcommand{\bbv}{{\boldsymbol{v}}}
\newcommand{\bbP}{{\boldsymbol{P}}}
\newcommand{\ld}{{\text{d}}}
\newcommand\eqdef{\stackrel{\mathclap{\normalfont\mbox{\tiny{def}}}}{=}}
\newcommand\eqdot{\stackrel{\mathclap{\normalfont\mbox{\tiny{\textbf{.}}}}}{=}}

\DeclareMathOperator{\Diffeo}{Diffeo}
\DeclareMathOperator{\Van}{Van}
\DeclareMathOperator{\Aut}{Aut}
\DeclareMathOperator{\Aff}{Aff}
\DeclareMathOperator{\sgn}{sgn}
\DeclareMathOperator{\Arg}{Arg}
\DeclareMathOperator{\Span}{span}
\DeclareMathOperator{\Sol}{Sol}
\DeclareMathOperator{\img}{im}
\DeclareMathOperator{\dom}{dom}
%\DeclareMathOperator{\id}{id}
\DeclareMathOperator{\rank}{rank}
\DeclareMathOperator{\sym}{sym}
\DeclareMathOperator{\cosym}{cosym}
\DeclareMathOperator{\pt}{pt}
\DeclareMathOperator{\arcsinh}{arcsinh}
\DeclareMathOperator{\coker}{coker}
\DeclareMathOperator{\Hom}{Hom}
\DeclareMathOperator{\End}{End}
\DeclareMathOperator{\Der}{Der}
\DeclareMathOperator{\Mat}{Mat}
\DeclareMathOperator{\poly}{poly}
\DeclareMathOperator{\CDiff}{\mathcal{C}Diff}
\DeclareMathOperator{\Diff}{Diff}
\DeclareMathOperator{\ord}{ord}
\DeclareMathOperator{\volume}{vol}
\DeclareMathOperator{\dvol}{d%\,
vol}
\DeclareMathOperator{\diag}{diag}
\DeclareMathOperator{\ad}{ad}
\DeclareMathOperator{\Ber}{Ber}
\DeclareMathOperator{\tr}{tr}
\newcommand{\Free}{\text{\textsf{Free}}\,}
\newcommand{\Sl}{\mathfrak{sl}}
\newcommand{\Gl}{\mathfrak{gl}}
\DeclareMathOperator{\const}{const}

\DeclareMathOperator{\Jac}{Jac}
%\DeclareMathOperator{\Jac}{\rm \textsf{Jac}}
\DeclareMathOperator{\Edge}{Edge}
\DeclareMathOperator{\Assoc}{Assoc}

\DeclareMathOperator{\GH}{gh}
\DeclareMathOperator*{\bigotimesk}{{\bigotimes\nolimits_{\Bbbk}}}
\DeclareMathOperator{\supp}{supp}

\newcommand{\td}{\widetilde{d}}
\newcommand{\tu}{\widetilde{u}}
\newcommand{\tv}{\widetilde{v}}
\newcommand{\tV}{\widetilde{V}}
\newcommand{\hxi}{\widehat{\xi}}
\newcommand{\lshad}{[\![}
\newcommand{\rshad}{]\!]}
\newcommand{\ov}{\overline}
\newcommand{\nC}{{\text{\textup{nC}}}}
\newcommand{\KdV}{{\text{KdV}}}
\newcommand{\ncKdV}{{\text{ncKdV}}}
\newcommand{\mKdV}{{\text{mKdV}}}
\newcommand{\pmKdV}{{\text{pmKdV}}}
\newcommand{\EL}{{\text{EL}}}
\newcommand{\Liou}{{\text{Liou}}}
\newcommand{\scal}{{\text{scal}}} 
\newcommand{\subout}{{\text{\textup{out}}}}
\newcommand{\subin}{{\text{\textup{in}}}}

\newcommand{\ib}[3]{ \{\!\{ {#1},{#2} \}\!\}_{{#3}} }
\newcommand{\schouten}[1]{\lshad {#1} \rshad}

\newcommand{\by}[1]{\textrm{{#1}}}
\newcommand{\jour}[1]{\textit{{#1}}}
\newcommand{\vol}[1]{\textbf{{#1}}}
\newcommand{\book}[1]{\textit{{#1}}}

\hyphenation{Kon-tse-vich Nam-bu Pois-son Ja-co-bi Ja-co-bi-a-tor Bu-ring}

\def\oldvec{\mathaccent "017E\relax }
\newcommand{\Or}{{\rm O\oldvec{r}}}

%\pagestyle{headings}
%\allowdisplaybreaks

\begin{document}

%\pagestyle{plain} %CHANGE FOR ARXIV

\title[Jacobi identities for Wronskians in multidimension]{Jacobi identities for Wronskian determinants\\[3pt] over multidimension}

%\runningheads{A.V.\ Kiselev}{Jacobi identities for Wronskians in multidimension}

%\begin{start}{%
\author[A.\,V.\,Kiselev]{Arthemy V.\ Kiselev}%{1}
\thanks{\textit{Address}:\quad %\address
Bernoulli Institute for Mathematics, Computer Science \&\ %\:%and 
Artificial Intelligence, 
University of Groningen, P.O.\,Box\:407, 9700\,AK Groningen, 
The Netherlands.}

\dedicatory{\textup{Based on the talk given at the XIII International symposium on Quantum Theory and Symmetries -- QTS13 (Yerevan, Armenia, 28 July -- 1 August 2025).
}}

\subjclass[2010]{13D10, 15A15, 17A42, 17B01, 17B66}
% MSC codes: %05C31, 05C90, 18G85, 53D55, 68R10 [VI.]
%53D55 - Deformation quantization, star products (?);
% Explanations:
% 05C31 - Graph polynomials; 05C90 - Applications of graph theory; 18G85 - Graph complexes and graph homology;  68R10 - Graph theory (including graph drawing) in computer science (?)

\keywords{Jacobi identity, jet space, multivariate analysis, $N$-ary bracket,
%$\mathfrak{sl}(2)$, 
strongly homotopy Lie algebra, Wronskian determinant}

\date{5 November 2025}
% Insert date of submission
%}

\begin{abstract}
The generalised Wronskian of differential order $k\geqslant 1$ for $N$ functions $f_1$,\ $\ldots$,\ $f_N$ in $d\geqslant 1$ independent variables $x^1$,\ $\ldots$,\ $x^d$ is the determinant of the matrix with these functions' derivatives 
$\partial^{|\sigma_i|} f_j / \partial (x^1)^{\sigma_i^1}\cdots \partial (x^d)^{\sigma_i^d}$ (of orders $0 \leqslant |\sigma_i| \leqslant k$), where the multi\/-\/indices $\sigma_i$ mark (all or part of) fibre variables $u_{\sigma_i}$ in the $k$th jet space $J^k\bigl(\BBR^d\to\BBR\bigr)$.
We prove that these (in)complete Wronskians --\,provided that their lowest\/-\/order parts are complete at differential orders $\ell\leqslant 1$\,-- over the $d$-\/dimensional base satisfy the table of bi\/-\/linear, Jacobi\/-\/type identities for Schlessinger\/--\/Stasheff's strongly homotopy Lie algebras. %with zero differential.
\end{abstract}

%\begin{KEY}
%Jacobi identity, jet space, multivariate analysis, $N$-ary bracket,
   %$\mathfrak{sl}(2)$, 
%strongly homotopy Lie algebra, Wronskian determinant
%\end{KEY}
%Enter key words or phrases in alphabetical order, separated by commas.
%\end{start}

%\date{}

%\affil{Bernoulli Institute for Mathematics, Computer Science and Artificial Intelligence, University of Groningen, P.O. Box 407, 9700 AK Groningen, The Netherlands}

%\email{a.v.kiselev@rug.nl}
\maketitle

\section{Introduction}\label{SecIntroduction}
\noindent%
The Wronskian determinants are used to inspect linear (in)\/dependence 
of functions\footnote{\label{FootWronskBehaves}
The Wronskian of $N$ scalar functions has conformal weight $N(N-1)/2$, so %hence 
itself is not a scalar function if $N>1$. Likewise, the arguments $f_j$ can be not scalar functions (of conformal weight $0$) but coefficients of positive\/-\/order differential operators on~$\BBR$, hence themselves behave under coordinate changes on the base $\BBR\ni x$.}
%%%
$f_1$,\ $\ldots$,\ $f_N$ (differentiable enough many times on an interval in~$\BBR$): if their Wronskian,
\[%$
W^{0,1,\ldots,N-1}(f_1,\ldots,f_N)=\det\bigl(\partial^{i-1} f_j / \partial x^{i-1}
\bigr)%\not\equiv0,
\]%$, 
is not identically zero, 
then they are linearly independent.%\footnote{

\begin{ex}%\textbf{Example.} 
\label{FootPeanoCounterex}
$W^{0,1}(x,x^2) = \left| \begin{smallmatrix} x & x^2 \\ 1 & 2x 
\end{smallmatrix} \right| = x^2 \not\equiv 0$ on $[-1,1]\ni x$.
Still the vanishing of the Wronskian on an interval does not yet imply linear independence; here is Peano's counterexample:
$W^{0,1}( x^2, x|x| ) \equiv 0$ on $[-1,1]\ni x$, but the functions $x^2$ and $x|x|$ are linearly independent on any open neighbourhood of the origin.%}
\end{ex}

For differentiable functions $f_j\in C^k\bigl( \BBR^{d\geqslant 1} \to \BBR\bigr)$ in many variables $x^1$,\ $\ldots$,\ $x^d$, the concept of Wronskian was re\/-\/discovered over decades by many authors from various disciplines
(see \cite{LeVeque1956,Schmidt1980,Wolsson1989a}, also~\cite{ForKac} referring to 2002--3 or A.\,G.\,Khovanskii in 2003--4 
(private communication)).%\footnote{

To generalise the Wronskian determinants to spaces of functions on $\BBR^d$ of dimensions $d\geqslant 1$, fix the differential order $k\geqslant 1$ (and work with arguments $f_j\in C^k(\BBR^d\to\BBR)$). 
List all the (multi\/-\/indices of) derivatives\footnote{\label{FootDimJetFibre}
\textbf{Lemma.} The dimension of $k$th jet fibre in the jet bundle $J^k(\BBR^d\to\BBR)$, counting $\sigma=\varnothing$ as well, equals $\binom{d+k}{d}$ under the natural assumption that $u_{xy}=u_{yx}$, etc., for all~$u_\sigma$.}
%%%
$u_{\sigma_i}$ of intermediate orders: $0\leqslant |\sigma_i|\leqslant k$.
%%%
We say that in a fixed system of coordinates on the affine %space 
$\BBR^d$, the \emph{complete} $k$th differential order Wronskian $W^{d\geqslant 1}_{k\leqslant 1}$ of $N=\binom{d+k}{d}$ arguments $f_j$ viewed as functions from $C^k(\BBR^d\to\BBR)$ is defined by the formula
\begin{equation}\label{EqFullWronskD}
W^{d\geqslant 1}_{k\geqslant 1} \bigl(f_1,\ldots,f_N\bigr) =
\det\Bigl(\partial^{|\sigma_i|} f_j \bigl/ \partial (x^1)^{\sigma_i^1}\ldots\partial (x^d)^{\sigma_i^d} \Bigr),
\end{equation}
where $\sigma_i=(\sigma_i^1,\ldots,\sigma_i^d)=(\#x^1,\ldots,\#x^d)$ runs over the multi\/-\/indices of $k$th jet's fibre coordinates $u_{\sigma_i}$; the index $i$ enumerates rows and $j$ counts columns ($1\leqslant i,j\leqslant N$).

\begin{ex}[{cf.~\cite{ForKac}}]\label{ExTernaryBr}
Over the Cartesian plane $\BBR^{d=2}\ni(x,y)$ and for the order bound $k=1$, the complete Wronskian is $W^{d=2}_{k=1} = \mathbf{1}\wedge \partial/\partial x\wedge \partial/\partial y$, that is
\begin{equation}\label{EqTernaryBracket}
W^{d=2}_{k=1}\bigl(f,g,h\bigr) = \begin{vmatrix} f & g & h \\ f_x & g_x & h_x \\ f_y & g_y & h_y \end{vmatrix},\qquad f,g,h\in C^1(\BBR^2\to\BBR).
\end{equation}
This ternary operator is tri\/-\/linear %(over $\BBR$) 
and totally antisymmetric w.r.t.\ its arguments: 
\[%$
W^{d=2}_{k=1}\bigl(\pi(f),\pi(g),\pi(h)\bigr)
= (-)^\pi W^{d=2}_{k=1}\bigl(f,g,h\bigr)
\]%$ 
for any permutation $\pi\in\mathbb{S}_3$.
\end{ex}

\begin{rem}\label{FootPeano2D}
The (in)complete generalised Wronskians over dimensions $d\geqslant 1$, which we presently describe, are subject to the same reservations --\,about their (in)sufficience to show the linear (in)dependence of functions\,-- as in the classical case of $d=1$. For \textmd{example}, the three functions $f(x,y)=x^2 y^2$,\ $g(x,y)=x|x|\cdot y^2$,\ and $h(x,y)=x^2\cdot y|y|$ are linearly independent on the square $[-1,1]\times[-1,1]\ni(x,y)$, yet their complete first\/-\/order generalised Wronskian from Eq.~\eqref{EqTernaryBracket}, see above, %below, 
vanishes identically on the entire domain of definition: 
$W^{d=2}_{k=1} \bigl(x^2y^2$,$x|x|\cdot y^2$,$x^2\cdot y|y|\bigr)\equiv 0$.
Indeed, for $x\geqslant0$ (and any $y\in\BBR$) determinant's 1st and 2nd columns coincide; for $y\geqslant 0$ (and any $x\in\BBR$) the 1st and 3rd columns coincide, whereas on $x<0$ and $y<0$, the 2nd and 3rd columns equal minus the first.%}
\end{rem}

\begin{define}%By definition, 
The generalised Wronskian determinant $'W^{d\geqslant 1}_{\underline{k\geqslant 1}}$ is \emph{incomplete} if its list $\{\sigma_i \}$ lacks certain multi\/-\/indices; exclusion is allowed only for the \emph{highest}\/-\/order derivatives (with $|\sigma_i|=k$).%\footnote{
\end{define}

\begin{ex}\label{FootExIncompleteWdimD}
For dimension $d=2$ and order $k=2$, by excluding the last multi\/-\/index $\sigma_6 = yy=(0,2)$ of top order $|\sigma_6|=2$ from their full list $\{\varnothing$,$x$,$y$,$xx$,$xy$,$yy\}$ at all orders $0\leqslant\ell\leqslant k=2$, we obtain the incomplete Wronskian dererminant of size $5\times 5$. Clearly, if this determinant already is not identically zero for five given functions, they are linearly independent (irrespective of the values of their second partial derivatives in~$y$).%}
\end{ex}


%\noindent\textbf
\subsection*{Preliminaries: strongly homotopy Lie algebras}%\quad
Let us recall that the usual Wronskians (over dimension $d=1$, see \cite{Dzhuma2002})
and complete generalised Wronskians (over $d>1$ and of differential orders $k,\ell\geqslant 1$, see \cite{ForKac}) satisfy the two\/-\/parametric (as $k,\ell\in\BBN$) table of Jacobi\/-\/type identities, bilinear w.r.t.\ the $N$-ary structures of orders $k$ and $\ell$, for strongly homotopy Lie algebras with zero differential.\footnote{\label{FootRefLadaStasheffSH}
The reader is addressed to the notes \cite{LadaStasheff1993} for definitions and %for the 
physical context: how homotopy Lie algebras appear in various models, see literature references therein.} 
%%%
Namely, denote by $A\mathrel{{:}{=}} C^{r\gg 1}(\BBR^d\to\BBR)$ the algebra of `good' functions; let $\Delta\in\Hom_\Bbbk\bigl(\bigwedge^{N_\subout} A$,$A\bigr)$ and $\nabla\in\Hom_\Bbbk\bigl(\bigwedge^{N_\subin} A$,$A\bigr)$ be $\Bbbk$-\/linear totally antisymmetric operators on~$A$. By definition, the \emph{action} of $\Delta$ on $\nabla$ is %, as usual,
$\Delta[\nabla]\bigl(a_1,\ldots,a_{N_\subout+N_\subin -1}\bigr) =
  \bigl[ N_\subin! (N_\subout -1)! \bigr]^{-1} \cdot{} $
\[%\begin{multline*}   %% = \dfrac{1}{N_\subin! (N_\subout -1)!} 
\sum_{\tau\in\BBS_{N_\subin+N_\subout-1}}
(-)^\tau \Delta\bigl(\nabla\bigl( a_{\tau(1)},\ldots,a_{\tau(N_\subin)} \bigr),\\
a_{\tau(N_\subin+1)},\ldots,a_{\tau(N_\subin+N_\subout-1)} \bigr);
\]%\end{multline*}
%note that by construction, 
the %right\/-\/hand side 
$(N_\subin+N_\subout-1)$-ary operator $\Delta[\nabla]$
is totally antisymmetric in~%w.r.t.\ its %the 
%arguments 
$a_m\in A$.

Dzhumadil'daev proved in~\cite{Dzhuma2002} for $d=1$ (cf.\ \cite{ForKac} with any $d\geqslant 1$) that %the 
Wronskian determinants of arbitrary orders $k_\subout,\ell_\subin%\in\BBN
$ satisfy the table of Jacobi %\/-\/type 
identities,
\begin{equation}\label{EqJacobiTableFullWWdim1}
W^{d=1}_{k_\subout\geqslant 1} \bigl[ W^{d=1}_{\ell_\subin\geqslant 1} \bigr] = 0.
\end{equation}
In the recent work~\cite{PRG25Wronsk} we recall in which way the Jacobi identities of this specific type for $N$-ary structures, given on~$A$ by the Wronskians, appear in the course of homotopy deformation of the Lie algebra $\cX^1(\BBR^{d=1})$ of vector fields on a one\/-\/dimensional base manifold.


\section{Jacobi identities for (in)complete Wronskians}\label{SecGrowWronsk}
\noindent%
We now strengthen the result in~\cite{ForKac}, extending the table of Jacobi\/-\/type identities \eqref{EqJacobiTableFullWWdim1} (over $d=1$ and over $d>1$ for complete sets of top\/-\/order derivatives in either Wronskian) to the case of \emph{incomplete} Wronskians: they may lack subsets of derivatives in the highest orders $k,\ell>1$
over dimension~$d>1$.

\begin{condition}\label{CondOrd1Full}
In what follows (and in contrast with Counterexample~\ref{CounterExNonZeroJac} on p.~\pageref{CounterExNonZeroJac} below), the (in)complete Wronskians
$'W^{d\geqslant 1}_{\underline{s\geqslant 1}}$ are admissible only if their set of \emph{first}\/-\/order derivatives is complete,%\footnote{
%} 
%%%
i.e.\ every $\partial/\partial x^a$ shows up in $'W^{d\geqslant 1}_{\underline{s\geqslant 1}} = \mathbf{1}\wedge\partial_{x^1}\wedge\ldots\wedge\partial_{x^d}\wedge\ldots$; omission of multi\/-\/indice(s) can occur only in the highest order~$s>1$.
\end{condition}

\begin{ex}\label{FootExWadmissible}
Consider again the example ($d=2$,$k=2$) in footnote~\ref{FootExIncompleteWdimD} on p.~\pageref{FootExIncompleteWdimD}: admissible are, for instance, the incomplete Wronskians $\mathbf{1}\wedge\partial_x\wedge\partial_y\wedge\partial_{xx}$ or
$\mathbf{1}\wedge\partial_x\wedge\partial_y\wedge\partial_{xx}\wedge\partial_{yy}$, etc., but not $\mathbf{1}\wedge\partial_y\wedge\partial_{xx}\wedge\partial_{xy}\wedge\partial_{yy}$ which lacks $\partial_x=\partial/\partial x$ in order~$1$.
\end{ex}

We recall from Lemma in footnote~\ref{FootDimJetFibre} on p.~\pageref{FootDimJetFibre} that $N=\binom{d+k}{d}$ is the number of different (modulo $u_{xy}=u_{yx}$, etc.) partial derivatives of all orders $0$,$\ldots$,$k\geqslant 1$ w.r.t.\ the $d\geqslant1$ independent variables $x^1$,$\ldots$,$x^d$. 
The \emph{complete} generalised Wronskians
$W^{d\geqslant1}_{k\geqslant1} = \mathbf{1}\wedge\partial_{x^1}\wedge\ldots\wedge\partial_{x^d}\wedge\ldots\wedge\partial_{x^d}^k$ contain all the multi\/-\/indices of these derivatives, starting from $\varnothing$ in the leading wedge factor $\mathbf{1}$ till all of the derivations $\partial^{|\sigma|}/\partial\bx^\sigma$ with $|\sigma|=k$. In this case of complete sets, we proved
%in~\cite{ForKac} 
that Jacobi identities \eqref{EqJacobiTableFullWWdim1} extend from $d=1$ to arbitrary dimensions~$d\geqslant1$.

\begin{ex}[cf.~\cite{ForKac}]\label{FootExTernaryJacobiDim2}
%\textbf{Example \textmd{(cf.~\cite{ForKac}% and references therein
%)}.} 
Over $d=2$ at order $k=1$, %the 
ternary bracket~\eqref{EqTernaryBracket} satisfies the ternary Jacobi identity,
$\mathbf{1}\wedge\partial_x\wedge\partial_y \bigl[ \mathbf{1}\wedge\partial_x\wedge\partial_y \bigr] = 0$, which is verified by direct calculation.
\end{ex}

\begin{theor}[\cite{ForKac}]\label{ThJacobiTableFullWWdimD}
Over $d\geqslant 1$, the complete generalised
Wronskians satisfy the Jacobi identities 
$W^{d\geqslant 1}_{k_\subout\geqslant 1} \bigl[ W^{d\geqslant 1}_{\ell_\subin\geqslant 1} \bigr] = 0$ for all differential orders $k_\subout,\ell_\subin\in\BBN$.%\footnote{
%}
\end{theor}

\begin{proof}[Proof scheme \textup{(cf.~\cite[Prop.\,5]{PRG25Wronsk} for $d=1$ and~\cite{ForKac} for $d\geqslant1$)}]
By construction, the ope\-ra\-tor $W^d_{k}\bigl[ W^d_{\ell} \bigr]$ is totally antisymmetric w.r.t.\ its $N_\subin + N_\subout -1$ arguments; hence, to be nonzero, this Jacobiator must act by pairwise non\/-\/coinciding differentiations $\partial^{|\sigma|} / \partial\bx^\sigma$ on all of its arguments. The key idea is to estimate the overall sum of their differential orders (in other words, count all $\partial/\partial x^a$ at hand for whatever $a\in\{1$,$\ldots$,$d\}$).
Without loss of generality suppose $\ell_\subin = \max(k_\subout$,$\ell_\subin)\geqslant k_\subout$; the complete Wronskian $W^{d\geqslant 1}_{\ell\geqslant 1} = \mathbf{1}\wedge \partial/\partial{x^1}\wedge\ldots\wedge\partial^{\ell}/\partial(x^d)^{\ell}$ contains all the differentiations of all orders $0$,$\ldots$,$\ell_\subin$.
To have more derivatives %differentiations 
that would not repeat the previously considered ones, higher\/-\/order operators (of orders${}>\ell_\subin$) are needed for the second, \ldots, last arguments of the other Wronskian. 
The other Wronskian, %of order${}\leqslant\ell_\subin$
$\mathbf{1}\wedge\langle{}$terms of order${}\leqslant\ell_\subin\rangle$, contains the right number of positive\/-\/order terms, but each of those differential orders does not exceed $\ell_\subin < \ell_\subin+1$, contrary to the required.
Therefore, at least one differentiation repeats twice, and the antisymmetrisation cancels out the entire operator's action.%\footnote{
%}
%%%
\end{proof}

\begin{rem}\label{FootRhoNearW}
From the proof it is readily seen that Wronskians can be pre\/-\/multiplied by an arbitrary factor $\varrho(\bx)$ --\,which, via the Leibniz rule, can absorb part of the derivatives acting on the inner Wronskian when it becomes the first argument of the outer structure,\,-- still preserving the statement of %assertion in
Theorem~\ref{ThJacobiTableFullWWdimD}: all the Jacobiators vanish.
\end{rem}

The flaw of assertion self in Theorem~\ref{ThJacobiTableFullWWdimD} is that over big dimension $d>1$, the matrix size of either Wronskian determinant leaps from
$N(d,r)=\binom{d+r}{d}$ to $N(d,r+1)=\binom{d+r+1}{d} > N(d,r)+1$ as the order $r$ increments by~$+1$. We claim that whenever $k,\ell>1$, the conclusion (with basically the same proof) can be strengthened: having completed the Wronskians $W^{d\geqslant1}_{k_\subout\geqslant1}$ and $W^{d\geqslant1}_{\ell_\subin\geqslant 1}$ at the preceding differential orders, we can \emph{gradually} accumulate either Wronskian in the next order $k_\subout+1$ and $\ell_\subin+1$ by incorporating new derivatives one by one. 
%By doing this 
Along many intermediate scenarios (for choosing the subsets of multi\/-\/indices in the next, not yet complete differential order), the \emph{complete} Wronskians
$W^{d\geqslant1}_{k_\subout+1 > 1}$ and $W^{d\geqslant1}_{\ell_\subin+1 > 1}$ are attained.

In what follows we assume again that, without loss of generality, $\ell_\subin \geqslant k_\subout$ (otherwise, swap `in'${}\rightleftarrows{}$`out'). We stress that under Condition~\ref{CondOrd1Full}, both the (in)complete Wronskians
$'W^{d\geqslant1}_{\underline{k_\subout\geqslant1}}$ and $'W^{d\geqslant1}_{\underline{\ell_\subin\geqslant 1}}$ must contain the complete sets of \emph{first}\/-\/order derivations $\partial %/\partial 
_{x^1}\wedge\ldots\wedge\partial_{x^d}$.

\begin{theor}\label{ThJacobiTableFullInPartOut}
Suppose that in the senior order $\ell_\subin\geqslant k_\subout\geqslant 1$, the Wronskian $W^{d\geqslant 1}_{\ell_\subin\geqslant 1}$ is complete;
the other Wronskian $'W^{d\geqslant 1}_{\underline{k_\subout\geqslant 1}}$ can be either incomplete in its highest order $1 < k_\subout \leqslant \ell_\subin$ or complete of order $k_\subout=1$, $W^{d\geqslant 1}_{k_\subout=1}$ without any derivatives of order${}\geqslant2$.
Then the Jacobi identity holds: $'W^{d\geqslant 1}_{\underline{k_\subout\geqslant 1}} \bigl[ W^{d\geqslant 1}_{\ell_\subin\geqslant 1} \bigr] = 0$.
\end{theor}

%
%\footnote{\label{ProofThFullPart}
\begin{proof}
Here, the proof repeats --\,word by word%verbatim
\,-- that 
of Theorem~\ref{ThJacobiTableFullWWdimD}.%}%
\end{proof}
%%%

\begin{theor}\label{ThJacobiTablePartInEnoughPartOut}
Suppose that in its senior order $\ell_\subin > 1$ the Wronskian $'W^{d\geqslant 1}_{\underline{\ell_\subin>1}}$ is incomplete, still the (in)complete other Wronskian determinant $'W^{d\geqslant 1}_{\underline{k_\subout\geqslant 1}}$ of size $N_\subout\times N_\subout$ with $k_\subout \leqslant \ell_\subin$ is such that $N_\subout - 1 > \bigl($the number of highest, $\ell_\subin$th\/-\/order derivatives missing in the top of the incomplete senior\/-\/order
Wronskian $'W^{d\geqslant 1}_{\underline{\ell_\subin > 1}}\bigr)$.
Then the Jacobi identity holds: $'W^{d\geqslant 1}_{\underline{k_\subout\geqslant 1}} \bigl[ 'W^{d\geqslant 1}_{\underline{\ell_\subin > 1}} \bigr] = 0$.
\end{theor}

%
%\footnote{\label{ProofThPartEnoughPart}%Again, 
\begin{proof}
We only need to bound the (sum of) orders $|\sigma|$ of $\partial^{|\sigma|}/\partial \bx^\sigma$. Do `complete' the senior\/-\/order Wronskian by fictitiously moving %borrowing 
the lacking number of derivatives from the lower\/-\/order Wronskian --- 
neglecting any repetitions of multi\/-\/indices $\sigma$ and pretending that all %of 
the carried derivatives are senior order, $\ell_\subin$.
One Wronskian now completed, the other again does not attain the required dif\-fe\-ren\-ti\-al order $\ell_\subin +1$ in each of its second, $\ldots$, last remaining wedge factor.%}%
\end{proof}
%%%

\begin{theor}\label{ThJacobiTablePartInPartOutInsuff}
Suppose that in its senior order $\ell_\subin > 1$ the Wronskian $'W^{d\geqslant 1}_{\underline{\ell_\subin > 1}}$ is incomplete, and the (in)complete other Wronskian determinant $'W^{d\geqslant 1}_{\underline{k_\subout\geqslant 1}}$ of size $N_\subout\times N_\subout$ with $k_\subout \leqslant \ell_\subin$ is such that
$N_\subout - 1 \leqslant \bigl($the number of highest, $\ell_\subin$th\/-\/order derivatives missing in the top of the incomplete senior\/-\/order
Wronskian $'W^{d\geqslant 1}_{\underline{\ell_\subin > 1}}\bigr)$.
Then the Jacobi identity holds: $'W^{d\geqslant 1}_{\underline{k_\subout\geqslant 1}} \bigl[ 'W^{d\geqslant 1}_{\underline{\ell_\subin > 1}} \bigr] = 0$.
\end{theor}

%
%\footnote{\label{ProofThPartInsuffPart}
\begin{proof}
Indeed, the outer Wronskian $'W^{d\geqslant 1}_{\underline{k_\subout\geqslant 1}}$ cannot supply $N_\subout - 1$ derivatives of order $\ell_\subin > 1$ --\,to let the Jacobiator act by non\/-\/coinciding derivations on all of its arguments\,-- because
$'W^{d\geqslant 1}_{\underline{k_\subout\geqslant 1}}$ contains at least one lowest\/-\/order derivation $\partial/\partial x^a$, yet their full set is already present in
$'W^{d\geqslant 1}_{\underline{\ell_\subin > 1}}$ of higher order.%}%
\end{proof}
%%%

%Remark.
Only the last case, Theorem~\ref{ThJacobiTablePartInPartOutInsuff} explicitly relies on the assumption $\ell_\subin > 1$ and Condition~\ref{CondOrd1Full} that the set of first\/-\/order multi\/-\/indices is full in $'W^{d\geqslant 1}_{\underline{\ell_\subin > 1}}$. In fact, the outer Wronskian can then be incomplete of order~$1$\,!

\begin{counterexample}\label{CounterExNonZeroJac}
But this is what happens when the above assumption ($\ell_\subin > 1$) and Condition~\ref{CondOrd1Full} are ignored: over $d=2$, we have
\[
\mathbf{1}\wedge \partial/\partial y \bigl[ \mathbf{1}\wedge \partial/\partial x \bigr] = 2\cdot W^{d=2}_{r=1} \not\equiv 0,
\] 
i.e.\ the action of one incomplete low\/-\/order Wronskian on the other of same type recovers ternary bracket~\eqref{EqTernaryBracket}.
\end{counterexample}

\section{Conclusion}\label{SecConclDeformDary?}%In conclusion, 
\noindent%
We established the `no\/-\/gaps' set of Jacobi identities (from the entire table of identities for the strongly homotopy Lie algebra with zero differential): we are now free to increment the size of either Wronskian determinant by $+1$, that is without huge leaps to the dimension of the next, higher\/-\/order jet fibre. Through Condition~\ref{CondOrd1Full} (contrasted by Counterexample~\ref{CounterExNonZeroJac}), the Wronskians over multidimension $d>1$
--\,participating in the homotopy of \emph{unknown} Lie\/-\/algebraic 
object\,-- still reproduce the fact that over $d=1$, the original structure to\/-\/deform was the Lie algebra $\cX^1(\BBR^1)$ of vector fields on the line, whence the wedge $\mathbf{1}\wedge \partial/\partial x$ (from the commutator of vector fields on $\mathbb{R}^1$) was seen in every Wronskian $W^{0,1,\ldots,N-1} = \mathbf{1}\wedge \partial_x\wedge\ldots\wedge\partial_x^{N-1}$. The deformation of $\cX^1(\BBR^1)$ ran over higher\/-\/order differential operators $w_j(x)\,\smash{\partial_x^{N/2}}$ for even $N=2p\in\BBN$ and over still\/-\/unrecognised objects for $N$ odd. The nature of deformation's higher\/-\/order terms over $d>1$ is not yet identified. The contrast of three new Theorems~\ref{ThJacobiTableFullInPartOut}--\ref{ThJacobiTablePartInPartOutInsuff} with Counterexample~\ref{CounterExNonZeroJac} (when Condition~\ref{CondOrd1Full} is violated) indicates the $(d+1)$-\/arity of the first\/-\/order `commutator' $W^{d>1}_{r=1}$ --\,for the objects on $\BBR^{d>1}$\,-- which undergoes the homotopy deformation.
%%%
An open problem is to describe the integral object for the algebra with the bracket built of $W^{d>1}_{r=1}$ and its homotopy by~$'W^{d>1}_{\underline{s>1}}$.%
%%%
\footnote{\label{FootWhatIsDeformedDimD} 
Over $d=1$, the Lie algebra $\cX^1(M^1)$ integrated to $\Diffeo(M^1)$ with associative composition~$\circ$;
the $L_\infty$-structure from \cite{ForKac,Dzhuma2002,LadaStasheff1993} integrated to an $A_\infty$-deformation of $\Diffeo(M^1)$ (note that $\circ$ is binary as $2=d+1$ for $M^1$). What is the $(d+1)$-ary analogue of the binary composition of diffeomorphisms\,?}
%%%


\subsection*{Acknowledgements}
%\noindent\textbf{Acknowledgements.}\quad
The author thanks the organisers of the XIII International symposium on Quantum Theory and Symmetries -- QTS13 (Yerevan, Armenia, 28~July -- 1~August 2025) for a warm atmosphere during the meeting.
This work has been partially supported by the Bernoulli Institute (Groningen, NL) via project~110135.
The author thanks M.\,Kontsevich and V.\,Retakh for helpful discussions and advice.


\begin{thebibliography}{7}\normalsize

\bibitem{LeVeque1956}%MR0080682(18,283d)
\by{W.J.\ LeVeque} %, William Judson
(1956) ``\book{Topics in number theory}''. Vol.~%\vol{1}, 
\vol{2}.
Addison\/--\/Wesley Publ.\ %ishing 
Co., %Inc., 
Reading~MA, {pp.128--134}. %x+198 pp. and viii+270 pp.

\bibitem{Schmidt1980}%MR0568710 (81j:10038) %628 references.
\by{W.M.\ Schmidt} %, Wolfgang M.  
(1980) ``\book{Diophantine approximation}''.
\textit{Lect.\ Notes in Math.} vol.~\vol{785}. Springer, Berlin,
{pp.114--150}. %Ch.V. %x+299 pp. ISBN: 3-540-09762-7 %10Fxx (10-02)

\bibitem{Wolsson1989a}%MR0993032  % 15A03 (15A54)
\by{K.\ Wolsson} %, K. (1-FAIR2)
(1989)
Linear dependence of a function set of $m$ variables with vanishing generalized Wronskians.
\jour{Linear Algebra Appl.} \vol{117}, {73--80}\\
DOI: \href{https://doi.org/10.1016/0024-3795(89)90548-X}{https://doi.org/10.1016/0024-3795(89)90548-X}.
%%%
%Wolsson, K. Linear dependence of a function set of m variables with vanishing generalized Wronskians. Linear Algebra Appl. 117 (1989), 73?80. (Reviewer: Shao Kuan Li) 15A03 (15A54)
%%%
%MR0989712
%Wolsson, Kenneth A condition equivalent to linear dependence for functions with vanishing Wronskian. Linear Algebra Appl. 116 (1989), 1?8. (Reviewer: P. R. Krishnamoorthy) 34A30

\bibitem{ForKac}
\by{A.V.\ Kiselev} (2007)
On associative Schlessinger\/--\/Stasheff algebras and Wronskian determinants.
\jour{J.~Math.\ Sci.% (N.Y.)
} \vol{141} 1, 1016--1030\\
DOI: \href{https://doi.org/10.1007/s10958-007-0028-2}{https://doi.org/10.1007/s10958-007-0028-2}.
%\by{A.V.\ Kiselev} (2004) On the associative homotopy Lie algebras and the Wronskians.
(\jour{Preprint} \href{https://arxiv.org/abs/math/0410185}{\textrm{arXiv:math.RA/0410185}})%, pp.~1--18.
%%%
%MR2137432 (2006g:17024)
%Kisel�v, A. V. (RS-MOSC)
%On associative Schlessinger-Stasheff algebras and Wronskian determinants. (Russian. English, Russian summary)
%Fundam. Prikl. Mat. 11 (2005), no. 1, 159?180; translation in
%J. Math. Sci. (N.Y.) 141 (2007), no. 1, 1016?1030
%17B55 (16S32)
%%%
%MR2112935 (2005i:17027)
%Kiselev, Arthemy V. (RS-MOSCP)
%On homotopy Lie algebra structures in the rings of differential operators. (English summary)
%Note Mat. 23 (2004/05), no. 1, 83?110.
%17B55 (17B66 17C50 22E65)

\bibitem{Dzhuma2002}%17A42
\by{A.S.\ Dzhumadil'daev} % (KZ-SDU)
(2005) $n$-Lie structures generated by Wronskians.
%Sibirsk. Mat. Zh. 46 (2005), no. 4, 759?773; translation in
\jour{Siberian Math.~J.} \vol{46} % (2005), no. 
{4} {601--612}
DOI: \href{https://doi.org/10.1007/s11202-005-0061-7}{https://doi.org/10.1007/s11202-005-0061-7}.
(\textit{Preprint} \href{https://arxiv.org/abs/math/0202043}{\textrm{arXiv:math.RA/0202043}})
%%%
%MR2081725 (2005i:17002) 
%Dzhumadil?daev, A. S. (F-IHES)
%N-commutators. (English summary)
%Comment. Math. Helv. 79 (2004), no. 3, 516--553.
%17A42 (17B66)

\bibitem{LadaStasheff1993}%MR1235010 
\by{J.\ Stasheff, T.\ Lada} (1993)
Introduction to SH~Lie algebras for physicists.
\jour{Internat.\ J.~Theoret.\ Phys.} \vol{32} {7} {1087--1103} 
DOI: \href{https://doi.org/10.1007/BF00671791}{https://doi.org/10.1007/BF00671791}.
(\textit{Preprint} \href{https://arxiv.org/abs/hep-th/9209099}{\textrm{arXiv:hep-th/9209099}})

\bibitem{PRG25Wronsk}
\by{A.V. Kiselev} (2025)
\textrm{Wronskians as $N$-ary brackets in finite\/-\/dimensional analogues of~$\mathfrak{sl}(2)$.}
Presented at
\textit{XXIX Int.\ conf.\ on Integrable Systems \textsl{\&}\ Quantum Symmetries} %ISQS29 
(ISQS29, 7--11~July 2025, CVUT Prague, Czech Republic)
(\jour{Preprint} \href{https://arxiv.org/abs/2510.02145}{\textrm{arXiv:2510.02145}} [math.CO], pp.~1--8)

\end{thebibliography}

\end{document}


MR1640457 (99k:17050)
Vinogradov, Alexandre (I-SLRN); Vinogradov, Michael
On multiple generalizations of Lie algebras and Poisson manifolds. (English summary) Secondary calculus and cohomological physics (Moscow, 1997), 273?287,
Contemp. Math., 219, Amer. Math. Soc., Providence, RI, 1998.
17B99 (17A42 17B81 58F05)

%%%%%%%%%%%%%%%%%%%%%%%%%%%%%  Cited Wolsson(1989) Linear dependence of a function set of $m$ variables with vanishing generalized Wronskian.
MR4720573 Reviewed Morrison, Megan; Kutz, J. Nathan Solving nonlinear ordinary differential equations using the invariant manifolds and Koopman eigenfunctions. SIAM J. Appl. Dyn. Syst. 23 (2024), no. 1, 924?960. (Reviewer: Patrick Bonckaert) 34A34 (34C45 37M20)
Review PDF Clipboard Journal Article
MR3816502 Reviewed Das, Tushar; Fishman, Lior; Simmons, David; Urba?ski, Mariusz Extremality and dynamically defined measures, part I: Diophantine properties of quasi-decaying measures. Selecta Math. (N.S.) 24 (2018), no. 3, 2165?2206. (Reviewer: Lei Yang) 11J13 (11J83 28A75 37C45 37F35)
Review PDF Clipboard Journal Article 15 Citations
MR3810193 Reviewed Boumenir, Amin; Tuan, Vu Kim; Hoang, Nguyen The recovery of a parabolic equation from measurements at a single point. Evol. Equ. Control Theory 7 (2018), no. 2, 197?216. (Reviewer: Enno Pais) 35R30 (35K20)
Review PDF Clipboard Journal Article 2 Citations
MR3804677 Reviewed Kossovskiy, Ilya; Xiao, Ming On the embeddability of real hypersurfaces into hyperquadrics. Adv. Math. 331 (2018), 239?267. (Reviewer: Theoharis Theofanidis) 53C40 (32V30 32V40)
Review PDF Clipboard Journal Article 1 Citation
MR3717990 Reviewed Fishman, Lior; Kleinbock, Dmitry; Merrill, Keith; Simmons, David Intrinsic Diophantine approximation on manifolds: general theory. Trans. Amer. Math. Soc. 370 (2018), no. 1, 577?599. (Reviewer: Michel Laurent) 11J83 (11J13 37A17)
Review PDF Clipboard Journal Article 25 Citations
MR2964908 Reviewed Deckers, Elke; Bergen, Bart; Van Genechten, Bert; Vandepitte, Dirk; Desmet, Wim An efficient wave based method for 2D acoustic problems containing corner singularities. Comput. Methods Appl. Mech. Engrg. 241/244 (2012), 286?301. 76M25 (35J05 65Nxx)
Review PDF Clipboard Journal Article 10 Citations
MR2741244 Reviewed Walker, Ronald A. Regarding an adaptive algorithm for testing multivariate linear dependence. Linear Algebra Appl. 434 (2011), no. 2, 605?613. (Reviewer: Abbas Salemi) 15A03 (05A10 11P81 68W40)
Review PDF Clipboard Journal Article
MR2166859 Reviewed Walker, Ronald A. Linear dependence of quotients of analytic functions of several variables with the least subcollection of generalized Wronskians. Linear Algebra Appl. 408 (2005), 151?160. 15A15 (32A20)
Review PDF Clipboard Journal Article 5 Citations 


%%%%%%%%%%%%%%%%%%%%%%%%%%%%%

\bibitem{Nambu1973}%MR0455611
\by{Y. Nambu} %Yoichiro
(1973) 
\textrm{Generalized Hamiltonian dynamics.} \jour{Phys.\ Rev.} \vol{D%(3)
7} {2405--2412}
DOI: \href{https://doi.org/10.1103/PhysRevD.7.2405}{https://doi.org/10.1103/PhysRevD.7.2405}.

\bibitem{Ascona96}
\by{M. Kontsevich} (1997)
\textrm{Formality conjecture.} 
In: {D.\,%aniel 
Sternheimer, J.\,%ohn 
Rawnsley and S.\,%imone 
Gutt (eds)}
``\book{Deformation theory and symplectic geometry} (Ascona %June 17--21,
1996)''.
\textit{Math.\ Phys.\ Stud.} vol.~\vol{20}, {Kluwer Acad.\ Publ., Dordrecht}, 
{pp.~139--156}
%doi:
{ISBN 0-7923-4525-8}

%\bibitem{ascona}
%Kontsevich M 1997 Formality conjecture \textit{Deformation theory and symplectic geometry (Ascona 1996)}
%139--156

\bibitem{skew21}
\by{R. Buring,  D. Lipper,  A.\,V.\,Kiselev} (2022)
\textrm{The hidden symmetry of Kontsevich's graph flows on the spaces of Nambu\/-\/determinant Poisson brackets.}
\jour{Open Comm.\ Nonlin.\ Math.\ Phys.} \vol{2} {186--215}
DOI: \href{https://doi.org/10.46298/ocnmp.8844}{https://doi.org/10.46298/ocnmp.8844}.
(\jour{Preprint} \href{https://arxiv.org/abs/2112.03897}{\textrm{arXiv:2112.03897}} [math.SG])

\bibitem{skew23}
\by{R.\,Buring, A.\,V.\,Kiselev} (2023) 
\textrm{The tower of Kontsevich deformations for Nambu\/--\/Poi\-s\-son structures on~$\BBR^d$:  Dimension\/-\/specific micro\/-\/graph calculus.} 
\jour{SciPost Phys.\ Proc.} \vol{14} {020} 1--11
%%%
%Proc.\ 34th International colloquium on group theoretical methods in Physics: \textsc{Group34} (18--22 July 2022, Strasbourg),
%Paper~020, 11~p.\ 
%%%
DOI: \href{https://doi.org/10.21468/SciPostPhysProc.14.020}{https://doi.org/10.21468/SciPostPhysProc.14.020}.
(\jour{Preprint} \href{https://arxiv.org/abs/2212.08063}{\textrm{arXiv:2212.08063}} [math.CO])

%\bibitem{avk}
%\by{A.\,V.\,Kiselev, M.\,S.\,Jagoe Brown, F.\,Schipper} (2024)
%\textrm{Kontsevich graphs act on Nambu\/--\/Poisson brackets, I. New identities for Jacobian determinants.} 
%\jour{J.~Phys.: Conf.\ Ser.} \vol{2912} {012008} 1--12
%DOI: \href{https://doi.org/10.1088/1742-6596/2912/1/012008}{https://doi.org/10.1088/1742-6596/2912/1/012008}.
%(\jour{Preprint} \href{https://arxiv.org/abs/2409.18875}{\textrm{arXiv:2409.18875}} [q-alg])

\bibitem{MSJB}
\by{M.\,S.\,Jagoe Brown, F.\,Schipper, A.\,V.\,Kiselev} (2024)
\textrm{Kontsevich graphs act on Nambu\/--\/Poisson brackets, II. The tetrahedral flow is a coboundary in 4D.}
\jour{J.~Phys.: Conf.\ Ser.} \vol{2912} {012042} 1--8
DOI: \href{https://doi.org/10.1088/1742-6596/2912/1/012042}{https://doi.org/10.1088/1742-6596/2912/1/012042}.
(\jour{Preprint} \href{https://arxiv.org/abs/2409.12555}{\textrm{arXiv:2409.12555}} [q-alg])

%\bibitem{fs}
%Schipper F, Jagoe Brown M S and Kiselev A V 2024 Kontsevich graphs act on Nambu--Poisson brackets, III. Uniqueness aspects. Journal of Physics: Conference Series, Vol. 2912 Paper 012035, pp.~1--8.  
%\textit{arXiv:2409.15932 [q-alg]}  %https://arxiv.org/abs/2503.10926

\bibitem{IV}
\by{M.\,S.\,Jagoe Brown, A.\,V.\,Kiselev} (2025)
\textrm{Kontsevich graphs act on Nambu\/--\/Poisson brackets, IV. 
Resilience of the graph calculus in dimensional shift $d \mapsto d+1$.%
%When the invisible becomes crucial. %%% (v2) pp.~1--8.
}
Presented at
\textit{XXIX Int.\ conf.\ on Integrable Systems \textsl{\&}\ Quantum Symmetries} %ISQS29 
(ISQS29, 7--11~July 2025, CVUT Prague, Czech Republic)
(\jour{Preprint} \href{https://arxiv.org/abs/2503.10916v1}{\textrm{arXiv:2503.10916v1}} [math.CO], pp.~1--16)


%\label{\paperend}  % obligatory label
\end{thebibliography}

\end{document}