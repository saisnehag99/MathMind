\section{Definitions and notation}\label{sec1}
In this section, the general notation and definitions that will be used throughout this work will be introduced. Furthermore, a brief reminder on tropical geometry is presented in order to relate, in the next sections, the weighting matrices (obtained naturally from birational sequences) to weight vectors in the tropical Grassmannian.

Given a positive integer $n\in \Z^+$, define $\left[n\right]:=\left\lbrace 1,\dots,n\right\rbrace$. Given $k,n\in \Z^+$, it will be considered that $0<k<n$, unless otherwise stated. Let $I_{k,n}:=\left\lbrace \lbrace i_1<\dots<i_k\rbrace:(\forall l=1,\dots,k)(i_l\in [n])\right\rbrace$.

Given two vector spaces $V$ and $W$, write $V\leq W$ if $V$ is a subspace of $W$. For $k,n\in \Z^+$, the \emph{Grassmannian} of $k$-planes in $\C^n$ is defined as
\begin{equation*}
    \Gr\left(k,n\right) := \left\lbrace V\leq \C^n:\dim(V) = k \right\rbrace.
\end{equation*} 

The Grassmannian will be considered together with the Plücker embedding $\Gr\left(k,n\right)\hookrightarrow \P^{\binom{n}{k}-1}$. Let $L=(l_1,\dots,l_k)\in I_{k,n}$ and let $p_L$ be the Plücker variable defined for $V\in \Gr\left(k,n\right)$ as follows: given a basis $\lbrace v_i\rbrace_{i=1}^k$, let $M_V$ be the matrix with rows the vectors $v_i$ expressed in the standard basis of $\C^n$. Then 
\begin{equation*} 
p_L\left(M_V\right):=\det\left((M_V)_{l_1},\dots,(M_V)_{l_k}\right).
\end{equation*}
Let $I=(i_1<\dots i_{s-1}<i_{s+1}<\dots <i_k)\in I_{k-1,n}$ and $J\in (j_1<\dots <j_{s-1}<j<j_{s+1}<\dots j_{k+1})$. For $j\in I$, define $p_{I\cup j}:=0$. Otherwise, if $j\not\in I$, define
\begin{align*}
    I\cup j &:= \left(i_1<\dots < i_{s-1}<j<i_{s+1}<\dots< i_k\right).\\
    J\setminus j &:= \left(j_1<\dots<j_{s-1}<j_{s+1}<\dots <j_{k_1}\right). 
\end{align*} 

For $I\in I_{k-1,n}$ and $J\in I_{k+1,n}$, define the \emph{Plücker relation} $R_{I,J}$ as follows
\begin{equation}\tag{PR}\label{PR}
    R_{I,J} :=\sum_{j\in J}\left(-1\right)^{\#\lbrace i\in I:i<j\rbrace + \#\lbrace j'\in J:j<j' \rbrace}p_{I\cup j}p_{J\setminus j}.
\end{equation} The \emph{Plücker ideal} is the ideal generated by the Plücker relations
\begin{equation*}
    \I_{k,n} :=\left\langle R_{I,J}:I\in I_{k-1,n},J\in I_{k+1,n}\right\rangle\subset \C\left[p_K:K\in I_{k,n}\right],
\end{equation*} and the Grassmannian $\Gr\left(k,n\right)$ is the vanishing locus of this ideal. The homogeneous coordinate ring is $\C\left[\Gr(k,n)\right] = \C\left[p_K:K\in I_{k,n}\right]/\I_{k,n}$, and is denoted $A_{k,n}$. The images of the Plücker variables $p_I$ under the canonical projection $\C\left[p_K:K\in I_{k,n}\right]\rightarrow A_{k,n}$ are denoted by $\overline{p_I}$ and are called \emph{Plücker coordinates}.

\subsection*{Monomial orders, initial ideals, and tropicalization}
Given a polynomial ring $R=\C\left[x_1,\dots,x_n\right]$, $\Mon\left(R\right)$ denotes the set of \emph{monomials} of $R$, that is
\begin{equation*}
    \Mon\left(R\right) := \left\lbrace x_1^{a_1}\dots x_n^{a_n}: (\forall i=1,\dots,n)(a_i\in \N)\right\rbrace.
\end{equation*} For an element in $\Mon\left(R\right)$, write $\x^{\a}:= x_1^{a_1}\dots x_n^{a_n}$ with $\a\in \N^n$.

A total order $\leq$ on $\Mon\left(R\right)$ is called a \emph{monomial order} or \emph{term order} if it satisfies
\begin{itemize}
    \item $\forall \a\in \N^n$: $1\leq \x^a$;
    \item If $\x^{\a}\leq \x^{\b}$, then: $\forall \U\in \N^{n}$ it holds $\x^{\a+\U}\leq \x^{\b+\U}$.
\end{itemize}

Given a total order $\leq$ on $\Mon\left(R\right)$ and vector $\w\in \R^n$, called a \emph{weight vector}, a new order is defined as: $\x^{\a}\leq_{\mathbf{w}} \x^{\b}$ if and only if 
\begin{itemize}
    \item $\w\cdot\a < \w\cdot \b$, or
    \item $\w\cdot \a = \w\cdot \b$ and $\x^{\a}\leq \x^{\b}$.
\end{itemize}

Let $f=\sum_{\a\in \N^n}c_{\a}\x^{\a}\in R$ be an arbitrary element where only finitely many $c_{\a}\neq 0$, and $I\subset R$ an ideal. For the monomial order $\leq$, define the \emph{initial term} of $f$ and the \emph{initial ideal} of $I$ as
\begin{equation*}
    \In_{\leq} \left(f\right) := \min \left\lbrace c_{\a}\x^{\a}:c_{\a}\neq 0 \right\rbrace , \qquad \In_{\leq }\left(I\right) :=\left\langle \In_{\leq}(g):g\in I \right\rangle.
\end{equation*}

Let $w\in \R^n$ be a weight vector and $M\in \Z^{m\times n}$. Let $<$ and $\prec$ be the usual order on $\R$ and a monomial order on $\Z^m$, respectively. Define the \emph{initial form} of $f$ with respect to $w$ and $M$, respectively, as
\begin{equation*}
    \In_{w}\left(f\right) := \sum_{\a: \,w\cdot \a  = \min_{<}\lbrace w\cdot \b:c_{\b}\neq 0\rbrace}c_{\a}\x^{\a},\qquad \In_M \left(f\right) := \sum_{\a:\,M\a^t = \min_{\prec} \lbrace M\b^t: c_{\b}\neq 0\rbrace}c_{\a}\x^{\a},
\end{equation*} and in a similar fashion, define the \emph{initial ideals} as 
\begin{equation*}
    \In_w\left(I\right) := \left\langle \In_w(g):g\in I\right\rangle ,\qquad \In_M\left(I\right) :=\left\langle \In_M(g):g\in I\right\rangle.
\end{equation*}

By the definition of initial ideal with respect to a weight vector, there is a fan structure on $\R^n$, called the \emph{Gröbner fan} and denoted $GF\left(I\right)$. By \cite[Lemma 1.1]{sturmfels1996equationsdefiningtoricvarieties}, toric ideals are prime and \emph{binomial}, i.e., generated by binomials. It is, however, not always true that initial ideals with respect to arbitrary weight vectors $w\in \R^n$ are binomial. Thus one must consider a special subfan of $GF(I)$.

Consider now $\hat{R} = \C\left[x_1^{\pm 1},\dots , x_n^{\pm 1}\right]$, the ring of Laurent polynomials and $\hat{f}=\sum_{\a\in \Z^n}c_{\a}\x^{\a}\in \hat{R}$, with only finitely many $c_{\a}\neq 0$. The \emph{tropicalization} of $f$ (cf. \cite[Definitions 3.1.1/3.1.2]{maclagan2021introduction}) is defined  as the function $\hat{f}^{\trop}:\R^n\rightarrow \R$ given by 
\begin{equation*}
    \hat{f}^{\trop}\left(w\right) := \min \left\lbrace w\cdot \a:c_{\a}\neq 0\right\rbrace.
\end{equation*} Given the hypersurface $V(\hat{f})\subset \left(\C^*\right)^n$, the \emph{tropical hypersurface} $\trop\left(V(\hat{f})\right)$ is defined as
\begin{equation*}
    \trop\left(V(\hat{f})\right) :=\left\lbrace w\in \R^n:\quad \begin{matrix}
        \textrm{the minimum in }\hat{f}^{\trop}(w)\\
        \textrm{is achieved at least twice}
    \end{matrix}\right\rbrace,
\end{equation*} and for an ideal $I\subset \hat{R}$, the \emph{tropical variety} is defined as
\begin{equation*}
    \trop\left(V(I)\right) := \bigcap_{f\in I}\trop\left(V(f)\right).
\end{equation*}

If $I\subset R$, consider the ideal $\hat{I}:=I\hat{R}$. Then, the tropicalization of the variety $V\left(I\right)$ is defined as
\begin{equation*}
    \trop\left(V(I)\right) :=\trop\left(V(\hat{I})\right).
\end{equation*}

The following characterization of the tropicalization, regarded as the \emph{Fundamental Theorem of Tropical Geometry}, is stated as follows \cite[Theorem 3.2.3]{maclagan2021introduction}

\begin{thm*}
    Let $I\subset R$ be an ideal. Then
    \begin{equation*}
        \trop\left(V(I)\right) = \left\lbrace w\in \R^n: \In_w(I) \textrm{ is monomial free} \right\rbrace.
    \end{equation*}
\end{thm*}

By this characterization, there is a fan structure on $\trop\left(V(I)\right)$ defined by noticing that two vectors $w,u\in \trop\left(V(I)\right)$ belong to relative interior of the same cone if and only if $\In_u\left(I\right) = \In_w\left(I\right)$. Furthermore, this describes $\trop\left(V(I)\right)$ as a subfan of the Gröbner fan.








