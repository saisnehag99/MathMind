\section{Review of the results on $\ell$ and $ku$}
\label{chap:reviewthhkul}

We review in this section the results about $\THH_*(ku;H\Z_{(p)})$, $\THH_*(\ell;H\Z_{(p)})$, $\THH_*(\ell)$ and the periodic spectrum $\THH(KU)$ and $\THH(L)$. We first give a computation of $\THH_*(ku;H\Z_{(p)})$ using the Brun spectral sequence in \cref{section:thhkuhz}. Then the Bockstein spectral sequence \eqref{ss:l}, computing $\THH_*(\ell)$, is known from \cite{angeltveit2010topological}. We review this result in \cref{section:thhl}. 


Our $q$-cofibrant commutative $S$-algebra model for the connective complex $K$-theory spectrum $ku$ will be the one of theorem VII.4.3 of \cite{elmendorf1997rings}; notwithstanding, the $E_\infty$ structure on $ku$ can be seen to be unique (see \cite{baker2008uniqueness}). We fix a prime $p$ and write $ku$ for the $p$-localized connective complex $K$-theory and $\ell$ its Adams summand. We obtain an $S$-algebra structure on the localization using the result on Bousfield localization stated in proposition VIII.1.8 of \cite{elmendorf1997rings}.

\subsection{The periodic case}

The spectra $ku$ and $\ell$ are the connective cover of the spectra $KU$ and $L$, the (periodic) $p$-completed complex $K$-theory spectrum and its (periodic) Adams summand. Since we already defined the connective version, we will consider $KU$ and $L$ to be the spectra obtained by inverting the Bott element or $v_1$ and then $p$-completing. Inverting these elements is a smashing localization as stated in before theorem VIII.4.3 of \cite{elmendorf1997rings}. This can also be seen to be the localization of $ku$ and $\ell$ at the Johnson-Wilson spectrum $E(1)$. In either case, they have the structure of $S$-algebras. Moreover, what we proved earlier about smashing localization and THH applies.

The homotopy type of $p$-completed topological Hochschild homology of $L$ was computed in \cite{mcclure1993topological} (theorem 8.1):
\begin{equation}
  \THH(L)_p \simeq (L\vee \Sigma L_\Q)_p
\end{equation}
where the subscript $p$ denotes $p$-completion and the subscript $\Q$ denotes rationalization. The argument was extended in \cite{ausoni2005topological} (proposition 7.13) to a compatible splitting with $KU$:
\begin{equation}
  \THH(KU)_p \simeq (KU\vee\Sigma KU_\Q)_p .
\end{equation}

This periodic result allow us to prove the following important lemma on the structure of the connective case:
\begin{lemma}\label{prop:torsioncoincide}
  In $\THH_*(ku)_{(p)}$ and for any $p$ prime, the $p$-torsion elements and the $u$-torsion elements are the same. Here, the subscript $(p)$ denotes $p$-localization.
\end{lemma}
\begin{proof}
  We will work with the following commutative diagram where the maps are formally inverting the elements given:
  \begin{equation}
    \begin{tikzcd}
      \THH_*(ku)_{(p)} \rar["a"] \dar["b"] & \THH_*(ku)_{(p)}[u^{-1}] \dar["c"] \\
      \THH_*(ku)_{(p)}[p^{-1}] \rar["d"] & \THH_*(ku)_{(p)}[p^{-1},\,u^{-1}]
    \end{tikzcd}
  \end{equation}
  The kernel of $a$ is the $u$-torsion elements, the kernel of $b$ is the $p$-torsion elements. To prove our claim, we only have to prove that $c$ and $d$ are monomorphisms.

  In each degree, $\THH_*(ku)_{(p)}$ will be a $p$-local finitely generated abelian group; this can be seen from the $E^1$-page of the Bockstein spectral sequence \eqref{ss:u}. The structure theorem of finitely generated abelian groups implies that to check if a map is a monomorphism, it is sufficient to check if the induced map on $p$-completion is a monomorphism.

  $\THH_*(ku)_{(p)}[p^{-1}]$ is the rationalization $\THH_*(ku)_\Q$, which can be computed using the Künneth spectral sequence:
  \begin{equation}
    \operatorname{Tor}^{E_*A^e}(E_*A,E_*M) \Rightarrow E_*\THH^R(A;M).
  \end{equation}
  Here, $E=H\Q$, $A=M=ku$ and $R$ is the sphere spectrum, and we have:
  \begin{equation}
    \operatorname{Tor}^{ku_{\Q*}\otimes ku_{\Q*}}(ku_{\Q*},ku_{\Q*}) \Rightarrow \THH_*(ku)_\Q.
  \end{equation}

  $ku_{\Q*}$ has a resolution as a $ku_{\Q*}\otimes ku_{\Q*}$-module given by
  \begin{equation}
    0 \leftarrow ku_{\Q*} \leftarrow ku_{\Q*}\otimes ku_{\Q*} \{1\} \leftarrow ku_{\Q*}\otimes ku_{\Q*} \{\sigma u\} \leftarrow 0
  \end{equation}
  with $d(\sigma u) = 1\otimes u - u \otimes 1$, thus the spectral sequence collapses at the $E^2$-page with
  \begin{equation}
    \THH_*(ku)_\Q \cong ku_{\Q*}\otimes E(\sigma u)
  \end{equation}
  and $|\sigma u| =3$. This is sufficient to see that the map $d$ from the initial diagram is a monomorphism, and that
  \begin{equation}\label{eq:thhkumegainverse}
     \THH_*(ku)_{(p)}[p^{-1},\,u^{-1}] \cong KU_{\Q*}\otimes E(\sigma u).
  \end{equation}

  On the other side, inverting $u$ is a smashing localization (see lemma V.1.15 of \cite{elmendorf1997rings}), so that our \cref{prop:localizationthh} yields a weak equivalence
  \begin{equation}
    \THH_*(ku)_{(p)}[u^{-1}] \simeq \THH_*(KU)_{(p)}.
  \end{equation}
  The previous result on $p$-completed $\THH(KU)$ and equation \eqref{eq:thhkumegainverse} allow us to conclude that $c$ is also a monomorphism.
\end{proof}


\subsection{Topological Hochschild homology of $ku$ with coefficients in $H\Z_{(p)}$}
\label{section:thhkuhz}

In this section, we compute $\THH_*(ku;H\Z_{(p)})$ with $p$ an odd prime. When $p=2$, we have $ku = \ell$; results about $\ell$ are in \cref{section:thhl}. We use the Brun spectral sequence:
\begin{equation}
  E^2_{p,q}=\THH_p(H\Z_{(p)};H\pi_q(H\Z_{(p)}\wedge_{ku}H\Z_{(p)}))\Rightarrow \THH_{p+q}(ku;H\Z_{(p)}) \tag{$u_\Z$}\label{ss:uZ}.
\end{equation}

The Künneth spectral sequence can be used to compute the coefficients.
\begin{proposition}
  \begin{equation}
    \pi_*(H\Z_{(p)}\wedge_{ku}H\Z_{(p)}) \cong E(\sigma u)
  \end{equation}
  an exterior algebra over $\Z_{(p)}$ on the generator $\sigma u$ of degree 3.
\end{proposition}
\begin{proof}
  $\Z_{(p)}$ has a resolution as a free $ku_*$-module given by $E(\sigma u)$, with $\sigma u$ of bidegree $(1,2)$ and $d(\sigma u) = u$, so that $\operatorname{Tor}_{*,*}^{ku_*}(\Z_{(p)},\Z_{(p)})\cong E(\sigma u)$. Then the Künneth spectral sequence
  \begin{equation}
    E^2_{p,q}=\operatorname{Tor}_{p,q}^{ku_*}(\Z_{(p)},\Z_{(p)})\Rightarrow \pi_{p+q}(H\Z_{(p)}\wedge_{ku}H\Z_{(p)})
  \end{equation}
  collapses for bidegree reasons with no extensions possible.
\end{proof}

The $E^2$-page of our Brun spectral sequence will then be two copies of $\THH_*(H\Z_{(p)};H\Z_{(p)})=\THH_*(H\Z_{(p)})$. $\THH_*(H\Z)$ was computed by Bökstedt in \cite{bokstedtthhzzp}:
\begin{equation}
  \THH_k(H\Z)=
  \begin{cases}
      \Z & \mbox{if $k=0$} \\
      0 & \mbox{if $k\geq 2$ is even} \\
      \Z/n & \mbox{if $k=2n-1\geq 2$.} \\
    \end{cases}
  \end{equation}
  Since localization at $p$ is smashing, we have
  \begin{equation}
    \THH_k(H\Z_{(p)})=
    \begin{cases}
      \Z_{(p)} & \mbox{if $k=0$} \\
      0 & \mbox{if $k\geq 2$ is even} \\
      \Z/\nu(n) & \mbox{if $k=2n-1\geq 2$} \\
    \end{cases}
  \end{equation}
  where $\nu$ is the $p$-adic valuation.
  Let $\mu_{n}$ be a generator of the $\Z/\nu(n)$ in degree $2n-1$. If $n$ is not divisible by $p$, then $\mu_{n} = 0$. We will also use the convention $\mu_0 = 1$ in our formulas.

  \begin{proposition}
    When $p$ is an odd prime, the spectral sequence \eqref{ss:uZ} collapse at the $E^2$-page. There are no extension, and
    \begin{equation}
      \THH_*(ku; H\Z_{(p)}) \cong \THH_*(H\Z_{(p)}) \otimes E(\sigma u)
    \end{equation}
    over $\Z_{(p)}$ with $\sigma u$ in degree 3.
  \end{proposition}
  \begin{proof}
    For bidegree reason, the only possible non-zero differentials are the $d^4$ between $\mu_{n+2}$ and $\sigma u\mu_{n}$. But if $p\geq 3$ divide $n+2$, it cannot divide $n$, so that at least one of $\mu_{n+2}$ or $\sigma u \mu_{n}$ is zero, and the spectra sequence collapse.

    Each generator is alone in its degree so that there cannot be any extension.
  \end{proof}

\subsection{Topological Hochschild homology of $\ell$}
\label{section:thhl}

In this section, we will review the results of \cite{angeltveit2010topological} on $\THH_*(\ell)$, relative to any prime $p$.  The first spectral sequence, denoted \eqref{ss:lZ}, is a Brun spectral sequence:
\begin{gather}
  \THH_*(H\Z_{(p)}; H(H\Z_{(p)}\wedge_\ell H\Z_{(p)})_*) \cong \THH_*(H\Z_{(p)})\botimes E(\sigma v_1)  \notag \\
  \Rightarrow \THH_*(\ell;H\Z_{(p)}) \tag{$\ell_\Z$}\label{ss:lZ}
\end{gather}
The second spectral sequence, denoted \eqref{ss:l}, is a Bockstein spectral sequence:
\begin{equation}
     \THH_*(\ell;H\Z_{(p)})\botimes P(v_1) \Rightarrow \THH_*(\ell) \tag{$\ell$}. \label{ss:l}
\end{equation}
In both spectral sequences, we chose here to begin at the $E^1$-pages, so that the differentials have bidegrees $|d^r|=(-r-1,r)$. The generators have bidegrees
\begin{equation}
  \begin{aligned}
    & |\mu_{kp}| = (2kp-1,0) \mbox{, $k\geq 1$ the generators of $\THH_*(H\Z_{(p)})$} \\
    & |\sigma v_1| = (0,2p-1)\\
    & |v_1| = (0,2(p-1)).
  \end{aligned}
\end{equation}

When necessary, for formulas in some discrete $\mathcal{R}$-algebra $\mathcal{A}$, we will use $x\cdot y$ for the $\mathcal{R}$-action of $x\in \mathcal{R}$ on $y\in \mathcal{A}$, and $xy$ for the product of $x,\,y\in \mathcal{A}$.  From \cite{angeltveit2010topological}, proposition 3.4, which compute $\THH_*(\ell;H\Z_{(p)})$ we can deduce:
\begin{proposition}\label{prop:diffin4}
  All the differentials in \eqref{ss:lZ} are given by the formulas:
  \begin{equation}
    d^{2p-1}(\mu_{(k+1)p})=p^{\nu(k)}\cdot\sigma v_1\mu_{kp}
  \end{equation}
  up to a unit where $k\geq 1$ and $\nu$ is the $p$-adic valuation.

  There is an extension given by $p\mu_p = \sigma v_1$.
\end{proposition}

\eqref{ss:l} is also computed in \cite{angeltveit2010topological}. We will use the following notations:
\begin{equation}
  \THH_*(\ell;H\Z_{(p)}) \cong \Z_{(p)}\{1,\, \mu_p\}\oplus\bigoplus_{k\geq 2} \faktor{\Z}{p^{\nu(k)}}\{v_0\mu_{kp},\, \sigma v_1 \mu_{kp}\} .
\end{equation}
Here from the Brun spectral sequence \eqref{ss:lZ}  we have $\sigma v_1 = p\cdot \mu_p$ and $v_0\mu_{kp}$ is a class in $\THH$ represented by $p\cdot\mu_{kp}\in E^\infty$. As in \cite{angeltveit2010topological}, we differentiate between the multiplication by $p$ in the first spectral sequence \eqref{ss:lZ}, denoted by $v_0$, and multiplication by $p$ in the second spectral sequence \eqref{ss:l}, denoted by $p$. 

\begin{theorem}[Theorem 6.4 of \cite{angeltveit2010topological}] \label{prop:diffinell}
  The differentials in \eqref{ss:l} are given by the formula:
  \begin{equation}
    d^{p^{n+1}+\dots +p}(p^{n}\cdot v_0\mu_{(k+1)p^{n+1}}) = k v_1^{p^{n+1}+\dots+p}\sigma v_1\mu_{kp^{n+1}},\; k\geq 0,\; n\geq 0
  \end{equation}
  up to a unit and linearity with respect to multiplication by $v_1$.
\end{theorem}

There are extensions at the end of this spectral sequence. We now state the result with our notations:
\begin{theorem}[sections 6.2 and 6.3 of \cite{angeltveit2010topological}]
  $\THH_*(\ell)$ is a quotient of the $\Z_{(p)}[v_1]$-module
  \begin{multline}
    \Z_{(p)}[v_1]\{1,\,\sigma v_1,\, v_0^n\mu_{p^{n+1}},\,n\geq 0\} \\
    \oplus \Z_{(p)}[v_1]\{v_0^h\sigma v_1\mu_{ap^n},\, n\geq 2,\, a\geq 1,\, \mbox{$a$ not divisible by $p$},\, h \geq 0\}
  \end{multline}
  by the relations in the non-torsion part:
  \begin{itemize}
  \item $p\cdot \mu_p = \sigma v_1$,
  \item $p\cdot v_0^n\mu_{p^{n+1}} = v_1^{p^n}v_0^{n-1}\mu_{p^n}$ for any $n\geq 1$,
  \end{itemize}
  and the relations in the torsion part:
  \begin{itemize}
  \item $v_0^{h} \sigma v_1 \mu_{ap^n} = 0$ for any $a \geq 1$ and $n \geq 2$, $a$ not divisible by $p$, and $h \geq n -1$,
  \item $v_1^{p^{n-h-1}+p^{n-h-2}+\dots+p }\cdot v_0^h\sigma v_1 \mu_{ap^n} = 0$ for any $a \geq 1$ and $n \geq 2$, $a$ not divisible by $p$ and  $0 \leq h \leq n - 2$,
  \item $p\cdot \sigma v_1 \mu_{(bp+p-1)p^n} = v_0\sigma v_1\mu_{(bp+p-1)p^n}+v_1^{p^n+p^{n-1}+\dots+p}v_0^{\nu(b)}\sigma v_1 \mu_{bp^{n+1}}$ for any $b\geq 1$ and $n \geq 2$.
  \item $p\cdot v_0^h\sigma v_1 \mu_{ap^n} = v_0^{h+1}\sigma v_1 \mu_{ap^n}$ for any $a \geq 1$, $n \geq 2$, $a$ not divisible by $p$, and any $1 \leq h \leq n - 2 $, or $h=0$ not in the previous case.
  \end{itemize}
  
  The degrees are:
  \begin{equation}
    \begin{aligned}
      & |\mu_{kp}| = 2kp - 1 \\
      & |\sigma v_1| = 2p - 1 \\
      & |v_0| = 0 \\
      & |v_1| = 2(p-1)
    \end{aligned}
  \end{equation}
  and $\nu$ is the $p$-adic valuation.
\end{theorem}

In order to lift this computation to the Bockstein spectral sequence \eqref{ss:u}, computing $\THH_*(ku)$, one must find another way to compare the sequences than the map induced by the inclusion $\ell\rightarrow ku$, since $\sigma v_1\in \THH_{2p-1}(\ell;H\Z_p)$ should be compared to $u^{p-2}\sigma u$ which is not a class in $\THH_{2p-1}(ku;H\Z_p)$. A solution is to consider the cofiber of the multiplication by $v_1$:
\begin{equation}
  \begin{tikzcd}
    \Sigma^{2p-1}ku \ar[r, "v_1"] & ku \ar[r] & ku/v_1.
  \end{tikzcd}
\end{equation}
This is done in \cref{chap:thhkucomplet}. \Cref{chap:truncated} developed the dictionary used to compare the two Bockstein spectral sequence computing $\THH_*(ku)$, the first associated to $u$ and the second to $v_1$.


%
%
%
%
