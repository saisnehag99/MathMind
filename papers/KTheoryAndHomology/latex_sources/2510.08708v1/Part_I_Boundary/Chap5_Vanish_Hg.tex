\subsection{The Baily{\textendash}Borel compactification of Picard modular surfaces}
\label{SS: Baily-Borel compactification of PMF} 
In this subsection, we recall the basics of the The Baily{\textendash}Borel compactification of Picard modular surfaces.


\par The boundary of the Baily-Borel compactification of a Shimura variety is stratified by Shimura varieties associated to standard (containing the Borel) admissible parabolic subgroups \cite[4.5 Definition]{Pink1} of the underlying reductive group $G$.

\par Let us recall the structure of standard admissible parabolic subgroups in the cases that we are interested in. 

\begin{itemize}
    \item The reductive group is $H$ (Definition \ref{defn:G_H}): Recall that $H$ is $\GL_2 \boxtimes E^{\times}$\footnote{In the following sections, we use $E^{\times}$ to denote $\mathrm{Res}_{E/\QQ} \Gm$.}, where the notation $\boxtimes$ denotes a pair with the same norm. If we denote by $B_2 = T_{2}N^{'}$ the standard Borel of $\GL_2$ together with its Levi decomposition,  then the standard admissible parabolic subgroups $Q^{\prime}$ of $H$ are isomorphic to $B_2 \boxtimes E^{\times}$, whose Levi decomposition is denoted as $Q^{\prime} = M^{\prime}N^{\prime}$. The canonical normal subgroup $P_{1}^{\prime}$ of $Q^{\prime}$ defined in \cite[4.7]{Pink1} is 
    \begin{equation*}
        P_{1}^{\prime} = \{ \begin{pmatrix}
                            * & * \\
                            0 & 1
                        \end{pmatrix} \} \boxtimes E^{\times} \subseteq H, 
    \end{equation*}
    whose Levi is 
    \begin{equation*}
         M_{1}^{\prime} = \{  \begin{pmatrix}
                                    * & 0 \\
                                    0 & 1
                             \end{pmatrix} \} \boxtimes E^{\times} \subseteq H. 
    \end{equation*}
    \item The reductive group is $G$ (Definition \ref{defn:G_H}): The standard admissible parabolic subgroups $Q$ of $G$ is isomorphic to the standard Borel $B = \begin{pmatrix}
                                             * & * & * \\
                                             0 & * & * \\
                                             0 & 0 & * 
                                          \end{pmatrix}$ of $G$,  whose Levi decomposition is denoted by $Q = MN$ (see \cite[Proposition 3.3]{Anc17}).  The canonical normal subgroup $P_{1}$ of $Q$ defined in \cite[4.7]{Pink1} is 
    \begin{equation*}
        P_1 = \{ \begin{pmatrix}
                    * & * & * \\
                    0 & * & * \\
                    0 & 0 & 1 
                  \end{pmatrix} \} \subseteq G, 
    \end{equation*}
    whose Levi is 
    \begin{equation*}
        M_1 = \{ \begin{pmatrix}
                     * & 0 & 0 \\
                     0 & * & 0 \\
                     0 & 0 & 1
                 \end{pmatrix}\} \subseteq G 
    \end{equation*} (see \cite[Lemma 3.8]{Anc17}).
\end{itemize}



 \par For each of the two Shimura varieties $X = \{M, S\}$, we denote by $X^{*}$ its Baily{\textendash}Borel compactification and let $\partial{X} = X^{*} - X$ be the cusps of the compactification. Then we have the commutative diagram.
\begin{equation}
\label{eq: diagram of compactification}
        \begin{tikzcd}
            M \ar[r, "j^{\prime}"] \ar[d, "{\iota}"] & M^{*} \ar[d, "p"] & \partial{M} \ar[l, "i^{\prime}" above] \ar[d, "q"] \\
            S \ar[r, "j"] & S^{*} & \partial{S} \ar[l, "i" above]  
        \end{tikzcd},
\end{equation}
where $j$ and $j^{\prime}$ are open immersions, $i$ and $i^{\prime}$ are closed immersions and $p$ and $q$ are closed immersions induced by the closed immersion $\iota$ \footnote{For this to hold, we need to choose level structures carefully.}. We also have 
\begin{equation*}
    \dim \partial S = 0, \, \dim \partial M = 0. 
\end{equation*}


\subsection{Degeneration of Hodge structures}
\label{SS: thm of BW and Kostant}
In this subsection, we compute the degeneration of Hodge structures over Shimura varieties $M$ and $S$, which will be used in the proof of vanishing on the boundary.

\begin{notation}
In the remaining part of this section, we use $\mu$ (instead of $\mu_{H}$) to denote the tensor functor that associates an algebraic representation of a reductive group to the corresponding variation of Hodge structure on the corresponding Shimura variety.  
\end{notation}

\subsubsection{The theorem of Burgos-Wildehaus and Kostant}
We first recall the theorem of Burgos-Wildehaus in the setting of $M$ and $S$. 

\begin{theorem}[\cite{BW04}, Theorem  2.6, 2.9]
\label{theorem: BW04}
Let $E$ (resp., $E^{\prime}$) be an algebraic representation of $G$ (resp., $H$). In the derived category $\mathrm{D}^{b}(\mathrm{MHM}_{\RR}(S))$ (resp., $\mathrm{D}^{b}(\mathrm{MHM}_{\RR}(M))$), we have 
\begin{align*}
    & i^{*}j_{*} \mu(E) \cong \bigoplus_{n} \mathcal{H}^{n}i^{*}j_{*} \mu(E)[-n] \\ 
    (\text{resp.,} \, & i^{\prime, *}j_{*}^{\prime} \mu(E^{\prime}) \cong \bigoplus_{n} \mathcal{H}^{n}i^{\prime, *}j_{*}^{\prime} \mu(E^{\prime})[-n]). 
\end{align*}
Furthermore, we have 
\begin{align*}
    & \mathcal{H}^{n}i^{*}j_{*} \mu(E) \cong  \mu(\mathrm{H}^{n + 2}(N, E)) \\
   (\text{resp.,} \,  & \mathcal{H}^{n}i^{\prime, *}j_{*}^{\prime} \mu(E^{\prime}) \cong  \mu(\mathrm{H}^{n + 1}(N^{\prime}, E^{\prime})).
\end{align*}
\end{theorem}

\begin{remark}
    \begin{enumerate}
        \item The group $G_1$ (resp., $G_{1}^{\prime}$) acts on the group cohomology of the unipotent group $\mathrm{H}^{q}(N, E)$ (resp., $\mathrm{H}^{q}(N^{\prime}, E^{\prime})$) via its action both on $N$ (resp., $N^{\prime}$) and $E$ (resp., $E^{\prime}$), so the second isomorphism can be viewed an isomorphism of representations of $G_1$ (resp., $G_{1}^{\prime}$). 
        \item The theorem of Burgos-Wildehaus and Kostant works for general Shimura varieties. 
    \end{enumerate}
\end{remark}

Now, let us introduce the necessary notations for Kostant's theorem. 

\begin{notation}
    \begin{enumerate}
        \item 
        Recall that for any unipotent group $N$ and any representation $E$ of $N$, we have $\mathrm{H}^{q}(N, E) = \mathrm{H}^{q}(\lien, E)$ where $\lien$ is the Lie algebra of $N$ and the right hand side is Lie algebra cohomology. 
        \item Let $Q$ be a parabolic  subgroup of a reductive group $G$ with Levi decomposition $Q = MN$, and $\lieq = \lien \oplus \liem$ where $\lieq = \mathrm{Lie}(Q)$, $\lien = \mathrm{Lie}(N)$ and $\liem = \mathrm{Lie}(M)$. Let $\lieh$ be the Cartan subalgebra of $\lieg = \mathrm{Lie}(G)$ corresponding to a fixed Borel and $\mathrm{R}^{+}(\lieg, \lieh)$ be the set of positive roots. We also write $\mathrm{R}(\lien, \lieh)$ as the subset of $\mathrm{R}^{+}(\lieg, \lieh)$ appearing in $\lien$. Denote by $\rho$ the half-sum of positive roots, and by $\mathrm{W}(\lieg, \lieh)$ the Weyl group. For any $w \in W(\lieg, \lieh)$, we write 
        \begin{itemize}
            \item  $\mathrm{R}^{+}(w) = \{ \alpha \in \mathrm{R}^{+}(\lieg, \lieh) | w^{-1} \alpha \notin \mathrm{R}^{+}(\lieg, \lieh) \}$, 
            \item $\mathrm{l}(w) = |\mathrm{R}^{+}(w)|$,
            \item $\mathrm{W}^{\prime} = \{w \in \mathrm{W}(\lieg, \lieh) | \mathrm{R}^{+}(w) \subset \mathrm{R}(\lien, \lieh) \}$. 
        \end{itemize}
    \end{enumerate}
\end{notation}

\begin{theorem} [\cite{Vogan81}, Theorem 3.2.3]
\label{thm: Kostant_theorem}
    Let $E_{\lambda}$ be an irreducible representation of $\lieg$ of highest weight $\lambda$. Then 
    \begin{equation*}
        \mathrm{H}^{q}(\lien, E_{\lambda}) = \bigoplus_{\{ w \in \mathrm{W}^{'} | \mathrm{l}(w) = q \}} F_{w(\lambda + \rho) - \rho}
    \end{equation*}
    as representations of $G_1$, where $F_{\mu}$ is an irreducible representation of $\liem$ of highest weight $\mu$. 
\end{theorem}

\subsubsection{Degeneration of Hodge structures on $M$ and $S$}
Now we give the explicit description of degeneration of Hodge structures on the Shimura varieties $M$ and $S$. \\

\par We first consider the case of $M$. 

\begin{notation}
Let us denote by $\lambda^{\prime}(d_1, d_2)$ the algebraic representation of 
\begin{equation*}
    G_1^{\prime} = \{(x;z) \in \Gm \boxtimes E^{\times} | x = z \bar{z} \}
\end{equation*}
with (highest) weight $(x; z) \mapsto z^{d_1}\bar{z}^{d_2}$. 
\end{notation}

\begin{lemma}
\label{lemma: BW_H}
    Consider the diagram 
    \[
        \xymatrix{
            M \ar[r]^-{j^{\prime}}  & M^{*}  & \partial{M} \ar[l]_-{i^{\prime}},  
        }. 
    \]
    As variations of mixed Hodge structures on $\partial M$, we have 
    \begin{equation}
    \label{eq: H: deg of MHS}
        i^{\prime*}j_{*}^{\prime} \mu(W) \cong \bigoplus_{n = -1}^{0} \mathcal{H}^{n}i^{\prime*}j_{*}^{\prime} \mu(W)[-n], 
    \end{equation}
    where 
    \begin{align*}
        & \mathcal{H}^{-1}i^{\prime*}j_{*}^{\prime}\mu(W) = \mu(\lambda^{\prime}(a + b + r + s, a + b + r + s)), \\
        & \mathcal{H}^{0}i^{\prime*}j_{*}^{\prime}\mu(W) = \mu(\lambda^{\prime}(-1, -1)), \\
        & \mathcal{H}^{n}i^{\prime*}j_{*}^{\prime}\mu(W) = 0, \ \text{for} \ n < -1 \ \text{or} \ n > 0. 
    \end{align*}
    The Hodge type of the corresponding variation of mixed Hodge structures are listed below: 
     \begin{equation}
     \label{eq: H: Hodge type}
            \begin{array}{|c|c|}
            \hline \text{VMHS} & \text{ Hodge type} \\
            \hline \mu(\lambda^{\prime}(a + b + r + s, a + b + r + s)) & (-(a + b + r + s), -(a + b + r + s))) \\
            \hline \mu(\lambda^{\prime}(-1, -1)) & (1, 1), \\
            \hline 
            \end{array}, 
    \end{equation}
    where ``VMHS" stands for variations of mixed Hodge structures.    
\end{lemma}

\begin{proof}
    It follows from Theorem \ref{theorem: BW04} and Theorem \ref{thm: Kostant_theorem} that 
    \begin{equation*}
        \mathcal{H}^{n}i^{\prime*}j_{*}^{\prime}\mu(W) =  \begin{cases}
                                                \mu(\mathrm{H}^{n + 1}(N^{\prime}, W)) & \text{when} \, n \ge -1, \\ 
                                                0 & \text{when} \, n < -1.
                                           \end{cases} 
    \end{equation*}
    Moreover, in our case when $\mathrm{W}^{\prime} = \mathrm{W} = S_{2}$ and $\rho = (\frac{1}{2}, -\frac{1}{2}; 0)$, we have the following table of elements $w \in \mathrm{W}^{\prime}$:
    \[
        \begin{array}{|c|c|c|}
           \hline w  & \mathrm{l}(w) & w(\lambda + \rho) - \rho \\
           \hline 1 & 0 & (\lambda_1, \lambda_2; d) \mapsto (\lambda_1, \lambda_2; d) \\
           \hline (12) & 1 & (\lambda_1, \lambda_2; d) \mapsto (\lambda_2 - 1, \lambda_1 + 1; d) \\
           \hline
        \end{array}. 
    \]
    From this,  we can see that 
     \begin{equation*}
        \mathcal{H}^{n}i^{\prime*}j_{*}^{\prime}\mu(W) = 0, \, \, \text{for} \,\,  n > 0. 
    \end{equation*}
    For $W = \Sym^{n}V_{2}$ where $n = a + b + r + s$, we get as algebraic representations of $G_1^{\prime}$, 
    \begin{align*}
       & \mathrm{H}^{0}(N^{\prime}, W) = \lambda^{\prime}(a + b + r + s, a + b + r + s), \\
       & \mathrm{H}^{1}(N^{\prime}, W) = \lambda^{\prime}(-1, -1).
    \end{align*}
\end{proof}

\begin{remark}
    This computation is essentially the same as \cite[Lemma 4.4]{LemmaI15}. 
\end{remark}

Now, let us consider the case of $S$. 
\begin{notation}
    Let us denote by $\lambda(\mu_1, \mu_2; d)$ the algebraic representation of 
    \begin{equation*}
        G_1 = \{\begin{pmatrix}
                x & & \\
                & z & \\
                &  & 1 
               \end{pmatrix} \} \subseteq G
    \end{equation*}
    with (highest) weight $x^{\mu_1}z^{\mu_2}(z\bar{z})^{d}$. 
\end{notation}
\begin{lemma}
\label{lemma: BW_G}
    Consider the diagram 
    \[         \xymatrix{             
                    S \ar[r]^-{j}  & S^{*}  & \partial{S} \ar[l]_-{i}.        
                }     
    \]
    As variations of mixed Hodge structures on $\partial S$, we have 
    \begin{equation}
    \label{eq: G: deg of MHS}
        i^{*}j_{*} \mu(V) \cong \bigoplus_{n = -2}^{1} \mathcal{H}^{n}i^{*}j_{*} \mu(V)[-n], 
    \end{equation}
    where
    \begin{align*}
         \mathcal{H}^{-2}i^{*}j_{*}\mu(V) &= \mu(\lambda(a + r - s, r - s; 2s - r + b)), \\
         \mathcal{H}^{-1}i^{*}j_{*}\mu(V) &= \mu(\lambda(r - s - 1, a + r -s; 2s - r + b)) \\ &\oplus \mu(\lambda(a + r - s, r - s - b - 1; 2s - r + b)), \\
         \mathcal{H}^{0}i^{*}j_{*}\mu(V) &=  \mu(\lambda(r - s - b - 2, a + r - s + 1; 2s - r + b)) \\ &\oplus \mu(\lambda(r - s - 1, r - s - b - 1; 2s - r + b)), \\
        \mathcal{H}^{1}i^{*}j_{*}\mu(V) &= \mu(\lambda(r - s - b - 2, r - s; 2s - r + b), \\
        \mathcal{H}^{n}i^{*}j_{*}\mu(V) &= 0, \ \text{for} \ n < -2 \ \text{or} \ n > 1. 
    \end{align*}
    The Hodge type of the corresponding variation of mixed Hodge structures are listed as follows, where ``VMHS" stands for variations of mixed Hodge structures.
    \begin{equation}
    \label{eq: G: Hodge type}
            \begin{tabular}{|c|c|}
            \hline 
            \text{VMHS} & \text{ Hodge type} \\
            \hline 
            \multirow{2}*{$\mu(\lambda(a + r - s, r - s; 2s - r + b))$} & $(-(a + b + r), -(a + b + s))$ \\
            \cline{2-2}
            ~ & $(-(a + b + s), -(a + b + r))$ \\
            \hline 
            \multirow{2}*{$\mu(\lambda(r - s - 1, a + r -s; 2s - r + b))$} & $(-(a + b + r), -(b + s - 1))$ \\
            \cline{2-2}
            ~ & $(-(b + s - 1), -(a + b + r))$ \\
            \hline 
            \multirow{2}*{$\mu(\lambda(a + r - s, r - s - b - 1; 2s - r + b))$} & $(-(a + r - 1), -(a + b + s))$ \\ 
            \cline{2-2}
            ~ & $(-(a + b + s), -(a + r - 1))$\\
            \hline 
            \multirow{2}*{$\mu(\lambda(r - s - b - 2, a + r - s + 1; 2s - r + b))$} & $(-(a + r - 1), -(s - 2))$ \\ 
            \cline{2-2}
            ~ & $(-(s - 2), -(a + r - 1))$ \\
            \hline 
            \multirow{2}*{$\mu(\lambda(r - s - 1, r - s - b - 1; 2s - r + b))$} & $(-(r - 2), -(s + b - 1))$ \\
            \cline{2-2}
            ~ & $(-(s + b - 1), -(r - 2))$ \\
            \hline 
            \multirow{2}*{$\mu(\lambda(r - s - b - 2, r - s; 2s - r + b)$} & $(-(r - 2), -(s - 2))$ \\ 
            \cline{2-2}
            ~ & $(-(s - 2), -(r - 2))$ \\
            \hline 
            \end{tabular}
    \end{equation}
\end{lemma}

\begin{proof}
It follows from Theorem \ref{theorem: BW04} that 
\begin{equation*}
    \mathcal{H}^{n}i^{*}j_{*}\mu(V) = \mu(\mathrm{H}^{n + 2}(N, V)). 
\end{equation*}
It follows from the fact $\mathrm{R}(\lien, \lieh) = \mathrm{R}^{+}(\lieg, \lieh)$ that $\mathrm{W}^{'} = \mathrm{W}$. By $\rho = (1, 0, -1; 0)$, we have the following table about elements $w \in \mathrm{W}^{'}$: 
\[
\begin{array}{|c|c|c|}
\hline w & \mathrm{l}(w) & w(\lambda + \rho) - \rho \\ 
\hline 1 & 0 & (\lambda_1, \lambda_2, \lambda_3; d) \mapsto (\lambda_1, \lambda_2, \lambda_3; d) \\
\hline (12) & 1 & (\lambda_1, \lambda_2, \lambda_3; d) \mapsto (\lambda_2 - 1, \lambda_1 + 1, \lambda_3; d) \\
\hline (13) & 3 & (\lambda_1, \lambda_2, \lambda_3; d) \mapsto (\lambda_3 - 2, \lambda_2, \lambda_1 + 2; d) \\
\hline (23) & 1 & (\lambda_1, \lambda_2, \lambda_3; d) \mapsto (\lambda_1, \lambda_3 - 1, \lambda_2 + 1; d) \\
\hline (132) & 2 & (\lambda_1, \lambda_2, \lambda_3; d) \mapsto (\lambda_3 - 2, \lambda_1 + 1, \lambda_2 + 1; d) \\
\hline (123) & 2 & (\lambda_1, \lambda_2, \lambda_3; d) \mapsto (\lambda_2 - 1, \lambda_3 - 1, \lambda_1 + 2; d) \\
\hline
\end{array}. 
\]
So if $n < -2$ or $n > 1$, we have 
\begin{equation*}
    \mathcal{H}^{n}i^{*}j_{*}\mu(V) = 0. 
\end{equation*}
Therefore, for $V = V^{a, b} \{r, s \} = V(a + r - s, r - s, r - s - b; 2s - r + b) $ and $ -2 \le n \le 1$, we get the results listed in the proposition. 
\end{proof}

\begin{remark}
    The computation is carried out in \cite[\S 4]{Anc17}. Our parametrization is different from \cite{Anc17}, so we reproduce it here. 
\end{remark}

\subsection{A vanshing theorem for Betti cohomology}
\label{SS: interior and boundary cohomologies}
We recall the following vanishing theorem.

\begin{theorem}[\cite{Sap1}, Theorem 5]
\label{thm: Sap}
    \par Let $G$ be a reductive algebraic group defined over $\QQ$ and let $D$ be the associated symmetric space, which can be written as $D = G(\RR) / K_{\infty}A_G$, where $K_{\infty}$ is a maximal compact of $G(\RR)$ and $A_{G}$ is the identity component of $\RR$-points of a maximal $\QQ$-split torus in the center of $G$. Let $\Gamma \subset G(\QQ)$ be a neat arithmetic subgroup, $X = \Gamma \backslash D$ be the associated locally symmetric space and $E$ be the local system on $X$ associated to an irreducible algebraic representation of $G$.
    \par If $D$ is a Hermitian or equal-rank symmetric space and the highest weight of $E$ is \textit{regular}, then $\mathrm{H}^{i}(X, E) = 0$ for all $i < \frac{1}{2}\dim X$, where $\dim X$ is the real dimension of the locally symmetric space $X$. 
\end{theorem}

\begin{remark}
    This result is announced without a detailed proof in \cite{Sap1} and proved in the preprint \cite{Sap2}. 
\end{remark}

\subsection{Vanishing on the boundary for absolute Hodge cohomology}
\label{SS: the proof of the Hodge vanishing}
In this subsection, we state and prove the theorem of vanishing on the boundary for absolute Hodge cohomology. 

\begin{convention}
\label{convention: the proof ov Hodge vanishing}
    \begin{itemize}
        \item In this subsection, Betti cohomology, compactly supported cohomology and interior cohomology all use $\RR$-coefficients. We will not stress this in the notation used for  cohomology.
        \item For Betti and compactly supported cohomology, we do not write the subscript $B$ but for interior cohomology, we write the subscript $B$. 
        \item In this subsection, we ignore the notation $\mu$ and write $V$ (resp., $W$) instead of $\mu(V)$ (resp. $\mu(W)$). 
    \end{itemize}
\end{convention}

\subsubsection{The theorem}
We first give the statement of the theorem. 

\begin{theorem}
\label{Thm: Hdg vanish on the boudary}
Under the conditions
\begin{enumerate}
     \item $0 \le -r \le a$ and  $0 \le -s \le b$, 
     \item $a > 0$ and $b > 0$,
     \item $r \neq 0$ or $s \neq 0$, 
\end{enumerate}
the map $\mathcal{E}is_{H}^n \colon \mathcal{B}_{n,\RR} \rightarrow \mathrm{H}^{3}_{H}(S, V(2))$ factors through the inclusion 
\[
    \Ext^{1}_{\mathrm{MHS}_{\RR}^{+}}(\mathbf{1}, \mathrm{H}^{2}_{B,!}(S, V(2))) \hookrightarrow \mathrm{H}^{3}_{H}(S, V(2)), 
\]
where $\mathrm{MHS}_{\RR}^{+}$ is the abelian category of mixed $\RR$-Hodge structures and ${\mathbf{1}}$ denotes the trivial Hodge structure that is the unit of $\mathrm{MHS}_{\RR}^{+}$. 
\end{theorem}



\subsubsection{The proof the theorem}
We give a proof of Theorem \ref{Thm: Hdg vanish on the boudary} here and the proof is divided into three steps. \\

\noindent \textbf{Step I.} In the first step, the goal is to prove the exactness of the sequence in Proposition \ref{Prop: Hodge exact sequence}. 

\begin{lemma} 
\label{Lemma:``Hom = 0" preparation1.}
    We have 
    \begin{align*}
        & \Hom_{\mathrm{MHS_{\RR}^{+}}}(\RR(0), \mathrm{H}^{0}(\partial S,i^{*}j_{*}V(2))) = 0, \\
        & \Hom_{\mathrm{MHS_{\RR}^{+}}}(\RR(0), \mathrm{H}^{1}(\partial S,i^{*}j_{*}V(2))) = 0, \\
        & \Hom_{\mathrm{MHS_{\RR}^{+}}}(\RR(0), \mathrm{H}^{3}(S,V(2))) = 0. 
    \end{align*}
\end{lemma}

\begin{proof}
    \par First, by equation (\ref{eq: G: deg of MHS}) and $\dim \partial S = 0$, we have 
        \begin{align*}
            \Hom_{\mathrm{MHS_{\RR}^{+}}}(\RR(0), \mathrm{H}^{0}(\partial S,i^{*}j_{*}V(2))) = \Hom_{\mathrm{MHS_{\RR}^{+}}}(\RR(0), \mathcal{H}^{0}i^{*}j_{*}V(2)), \\
            \Hom_{\mathrm{MHS_{\RR}^{+}}}(\RR(0), \mathrm{H}^{1}(\partial S,i^{*}j_{*}V(2))) = \Hom_{\mathrm{MHS_{\RR}^{+}}}(\RR(0), \mathcal{H}^{1}i^{*}j_{*}V(2)).
        \end{align*}
        By Table (\ref{eq: G: Hodge type}) in Lemma \ref{lemma: BW_G}, we can see that there is no part with Hodge type $(0, 0)$ contained in $\mathcal{H}^{0}i^{*}j_{*}V(2)$ and $\mathcal{H}^{1}i^{*}j_{*}V(2)$. Hence, we get
        \begin{align*}
             & \Hom_{\mathrm{MHS_{\RR}^{+}}}(\RR(0), \mathrm{H}^{0}(\partial S,i^{*}j_{*}V(2))) = 0, \\
        & \Hom_{\mathrm{MHS_{\RR}^{+}}}(\RR(0), \mathrm{H}^{1}(\partial S,i^{*}j_{*}V(2))) = 0. 
        \end{align*}
        Second, by the exact sequence in equation (\ref{eq: les of singular cohomology}) and Theorem \ref{thm: Sap}, we have 
        \begin{equation*}
            \mathrm{H}^{3}(S,V(2)) \cong \mathrm{H}^{1}(\partial S,i^{*}j_{*}V(2))). 
        \end{equation*}
        Hence, we conclude that 
        \[
            \Hom_{\mathrm{MHS_{\RR}^{+}}}(\RR(0), \mathrm{H}^{3}(S,V(2))) = 0.    
        \]
\end{proof}

\begin{proposition}
\label{Prop: Hodge exact sequence}
We have the following exact sequence
\[
    \begin{tikzcd}
        0 \ar[r] & \Ext^{1}_{\mathrm{MHS}_{\RR}^{+}}(\mathbf{1}, \mathrm{H}^{2}_{B,!}(S, V(2))) \ar[r] &  \mathrm{H}^{3}_{H}(S,V(2)) \ar[r, "{\partial_{S}^{H}}"] & \mathrm{H}^{1}_{H}(\partial{S}, i^{*}j_{*}V(2)),
    \end{tikzcd} 
\]
where $i^{*}$ and $j_{*}$ denote the pullback and pushforward of the mixed Hodge module. 

\end{proposition}

\begin{proof}
Apply the choices $U = S$, $c = 2$, $Y = \partial S$ and $M = V(2)$ in the exact sequence in equation \eqref{eq: les of singular cohomology}, to get the exact sequence 
\begin{equation} 
    0 \rightarrow \mathrm{H}_{c}^{2}(S,V(2)) \rightarrow \mathrm{H}^{2}(S,V(2)) \rightarrow \mathrm{H}^{0}(\partial S,i^{*}j_{*}V(2)) \rightarrow \mathrm{H}_{c}^{3}(S, V(2)). 
\end{equation}

By Poincar\'{e} duality, we have 
\begin{equation*}
    \mathrm{H}_{c}^{3}(S, V(2)) \cong \mathrm{H}^{1}(S, V^{\vee}(-2)) = 0,
\end{equation*}
where the last equality holds since $V^{\vee}$ is regular. So we have the short exact sequence
\begin{equation*}
     0 \rightarrow \mathrm{H}_{B, !}^{2}(S,V(2)) \rightarrow \mathrm{H}^{2}(S,V(2)) \rightarrow \mathrm{H}^{0}(\partial S,i^{*}j_{*}V(2)) \rightarrow 0. 
\end{equation*}
Apply the functor $\Hom_{\mathrm{MHS_{\RR}^{+}}}(\RR(0), -)$ to this short exact sequence to get the exact sequence
\begin{align*}
     \Hom_{\mathrm{MHS_{\RR}^{+}}}(\RR(0), \mathrm{H}^{0}(\partial S,i^{*}j_{*}V(2))) & \rightarrow  \Ext^{1}_{\mathrm{MHS_{\RR}^{+}}}(\RR(0),\mathrm{H}_{B, !}^{2}(S,V(2))) \\ & \rightarrow 
      \Ext^{1}_{\mathrm{MHS_{\RR}^{+}}}(\RR(0), \mathrm{H}^{2}(S,V(2))) \rightarrow 
     \Ext^{1}_{\mathrm{MHS_{\RR}^{+}}}(\RR(0), \mathrm{H}^{0}(\partial S,i^{*}j_{*}V(2)).  
\end{align*}

Since $\Hom_{\mathrm{MHS_{\RR}^{+}}}(\RR(0), \mathrm{H}^{0}(\partial S,i^{*}j_{*}V(2))) = 0$ by Lemma \ref{Lemma:``Hom = 0" preparation1.}, we have the following exact sequence
\begin{align*}
     0  & \rightarrow \Ext^{1}_{\mathrm{MHS_{\RR}^{+}}}(\RR(0),\mathrm{H}_{B, !}^{2}(S,V(2))) \\ & \rightarrow \Ext^{1}_{\mathrm{MHS_{\RR}^{+}}}(\RR(0), \mathrm{H}^{2}(S,V(2))) \rightarrow 
     \Ext^{1}_{\mathrm{MHS_{\RR}^{+}}}(\RR(0), \mathrm{H}^{0}(\partial S,i^{*}j_{*}V(2)).   
\end{align*}

Also by Lemma \ref{Lemma:``Hom = 0" preparation1.} and the exact sequence (\ref{Eq: ses for abs Hodge cohomology}), we get 
\begin{equation}
\mathrm{H}_{H}^{3}(S, V(2)) \cong \Ext^{1}_{\mathrm{MHS_{\RR}^{+}}}(\RR(0), \mathrm{H}^{2}(S,V(2)))
\end{equation}
and 
\begin{equation}
    \mathrm{H}_{H}^{1}(\partial S, i^{*}j_{*}V(2)) \cong \Ext^{1}_{\mathrm{MHS_{\RR}^{+}}}(\RR(0), \mathrm{H}^{0}(\partial S,i^{*}j_{*}V(2)),
\end{equation}
so we get the desired exact sequence
\[
    \xymatrix{
     0 \ar[r] &
     \Ext^{1}_{\mathrm{MHS_{\RR}^{+}}}(\RR(0),\mathrm{H}_{B, !}^{2}(S,V(2))) \ar[r] &   
     \mathrm{H}_{H}^{3}(S, V(2)) \ar[r] & 
     \mathrm{H}_{H}^{1}(\partial S, i^{*}j_{*}V(2)).  
    }
\]
    
\end{proof}

\vspace{10pt}

\noindent \textbf{Step II.} In the second step, the goal is to prove the diagram in Proposition \ref{Prop: Hodge commutative diagram} commutes. 

\begin{proposition}
\label{Prop: Hodge commutative diagram}
The following diagram is commutative,
\[
    \begin{tikzcd}
         \mathcal{B}_{n, \RR} \ar[d, "p^{*}\circ Eis_{H}^{n}"] & \\
        \mathrm{H}^{1}_{H}(M,W(1)) \ar[r, "\partial_{M}^{H}"] \ar[d, "{\iota_{*}^{H}}"] & \mathrm{H}^{0}_{H}(\partial{M}, i^{\prime, *}j^{\prime}_{*}W(1))) \ar[d, "{\theta^{H}}"] \\
        \mathrm{H}^{3}_{H}(S, V(2)) \ar[r, "\partial_{S}^{H}"] & \mathrm{H}^{1}_{H}(\partial{S}, i^{*}j_{*}V(2))
    \end{tikzcd}, 
\]
where $\iota_{*}^{H}$ is the Gysin morphism defined in Proposition \ref{Prop: Gysin for abs Hdg cohomology}, $\partial_{M}^{H}$ and $\partial_{S}^{H}$ are the corresponding boundary maps and the map $\theta^{H}$ is induced from the map
\begin{equation}
\label{eq: Hodge theta in derived category.}
    (\partial \iota_{*}^{H} \colon q_{*}i^{\prime*}j^{\prime}_{*}W(1) \rightarrow i^{*}j_{*}V(2)[1]) \in \mathrm{D}^{b}(\mathrm{MHM}_{\RR}(\partial S/\RR)),
\end{equation}
which is degenerated from $\iota^{H}_{*}$. 
    
\end{proposition}

\begin{proof}
It follows from Proposition \ref{Prop: Gysin for abs Hdg cohomology} that the Gysin morphism $\iota_{*} \colon \mathrm{H}^{1}_{H}(M,W(1)) \rightarrow \mathrm{H}^{3}_{H}(S, V(2))$ is induced by 
\begin{equation*}
    (\iota_{*}W(1) \rightarrow V(2)[1]) \in \mathrm{D}^{b}(\mathrm{MHM}_{\RR}(S/\RR)). 
\end{equation*}
Apply the functor $j_{*}$ to the morphism, and by the fact that $j_{*}\iota_{*}W(1) = p_{*}j^{\prime}_{*}W(1)$, we get the map
\begin{equation*}
    (p_{*}j^{\prime}_{*}W(1) \rightarrow j_{*}V(2)[1]) \in \mathrm{D}^{b}(\mathrm{MHM}_{\RR}(S^{*}/\RR)). 
\end{equation*}
Then we apply the functor $i_{*}i^{*}$ to the morphism and have a morphism 
\begin{equation*}
    (i_{*}i^{*}p_{*}j^{\prime}_{*}W(1) \rightarrow i_{*}i^{*}j_{*}V(2)[1]) \in \mathrm{D}^{b}(\mathrm{MHM}_{\RR}(S^{*}/\RR)). 
\end{equation*}
From the equation $i^{*}p_{*} = q_{*}i^{\prime*}$ which follows from the proper base change theorem of mixed Hodge modules for the proper map $p$, we get the following map 
\begin{equation*}
    (i_{*}q_{*}i^{\prime*}j^{\prime}_{*}W(1) \rightarrow i_{*}i^{*}j_{*}V(2)[1]) \in \mathrm{D}^{b}(\mathrm{MHM}_{\RR}(S^{*}/\RR)). 
\end{equation*}
By functoriality of the natural transformation $\mathrm{id} \rightarrow i_{*}i^{*}$ from adjunction, we can see the following commutative diagram in $\mathrm{D}^{b}(\mathrm{MHM}_{\RR}(S^{*}/\RR))$: 
\[
    \begin{tikzcd}
        p_{*}j^{\prime}_{*}W(1) \ar[r] \ar[d] & i_{*}q_{*}i^{\prime*}j^{\prime}_{*}W(1) \ar[d] \\
        j_{*}V(2)[1] \ar[r] & i_{*}i^{*}j_{*}V(2)[1]
    \end{tikzcd}. 
\]
Finally, apply the functor $\Hom_{\mathrm{D}^{\mathrm{b}}(\mathrm{MHM}_{\RR}(S^{*}/\RR))}(\RR(0)_{S^{*}}, -[2])$ to the above  commutative diagram to get the commutative diagram 
\begin{equation*}
    \begin{tikzcd}
        \mathrm{H}^{1}_{H}(M, W(1)) \ar[r] \ar[d] & \mathrm{H}^{0}_{H}(\partial M, i^{\prime*}j^{\prime }_{*}W(1))  \ar[d] \\
        \mathrm{H}^{3}_{H}(S, V(2)) \ar[r] & \mathrm{H}^{1}_{H}(\partial S, i^{*}j_{*}V(2))
    \end{tikzcd}.
\end{equation*}
\end{proof}

\begin{remark}
    From the construction, the boundary map $\theta_{S}^{H}$ in Proposition \ref{Prop: Hodge commutative diagram} is the same as the $\theta_{S}^{H}$ in Proposition \ref{Prop: Hodge exact sequence}. Hence, we use the same notation to for them. 
\end{remark}

\vspace{10pt}

\noindent \textbf{Step III.} In the third step, the goal is to prove the map $\theta^{H}_{S}$ in Proposition \ref{Prop: Hodge commutative diagram} is zero, which is stated in Proposition \ref{Prop: Hdg theta is zero}. \\

\par We first recall some notation. 
\begin{notation}
\begin{itemize}
    \item 
        We have the following commutative diagram:
        \begin{equation*}
                    \begin{tikzcd}
                        M \ar[r, "j^{\prime}"] \ar[d, "{\iota}"] & M^{*} \ar[d, "p"] & \partial{M} \ar[l, "i^{\prime}" above] \ar[d, "q"] \\
                        S \ar[r, "j"] & S^{*} & \partial{S} \ar[l, "i" above]  
                    \end{tikzcd}. 
            \end{equation*}
    \item Recall that the group homomorphism underlying the morphism between the boundary components $q \colon \partial M \rightarrow \partial S$ is      the map 
           \begin{equation}
           \label{eq: group embedding boundary}
                \begin{tikzcd}[row sep = 0]
                    {\iota} \colon M_1^{\prime} = \Gm \boxtimes E^{\times} \ar[r] & M_1 \\ (x; z) \ar[r, mapsto] & \begin{pmatrix}
                                                                                                                x & &  \\
                                                                                                                & z & \\
                                                                                                                & & 1 
                                                                                                            \end{pmatrix}
                 \end{tikzcd}. 
            \end{equation}
    \item We denote by $\lambda(\mu_1, \mu_2; d)$ the algebraic representation of $M_{1}$ with weight 
    \begin{equation}
    \label{eq: rep of G_1}
        \begin{pmatrix}
            x & &  \\
            & z & \\
            & & 1 
        \end{pmatrix} \mapsto x^{\mu_1}z^{\mu_2}(z\bar{z})^{d}, 
    \end{equation} 
    and we use $\lambda^{\prime}(d_1, d_2)$ to represent the algebraic representation of $M_1^{\prime}$ with weight $(x; z) \mapsto z^{d_1}\bar{z}^{d_2}$.
    \item We let $n = a + b + r + s$. 
\end{itemize}
\end{notation}

\begin{lemma}
\label{lemma: Hodge: source of theta}
    The source $\mathrm{H}^{0}_{H}(\partial{M}, i^{\prime*}j^{\prime}_{*}W(1))$ of the map $\theta^{H}_{S}$ is isomorphic to 
    \begin{equation*}
        \mathrm{H}^{0}_{H}(\partial{M}, \mathcal{H}^{0}i^{\prime*}j^{\prime}_{*}W(1)).
    \end{equation*}
\end{lemma}

\begin{proof}
    It follows from Lemma \ref{lemma: BW_H} that 
    \begin{equation*}
        i^{*}j_{*} \mu(W) \cong \bigoplus_{n = -1}^{0} \mathcal{H}^{n}i^{*}j_{*} \mu(W)[-n]. 
    \end{equation*}
    Hence, we have 
    \begin{equation*}
        \mathrm{H}^{0}_{H}(\partial{M}, i^{\prime*}j^{\prime}_{*}W(1)) \cong \mathrm{H}^{0}_{H}(\partial{M}, \mathcal{H}^{0}i^{\prime*}j^{\prime}_{*}W(1)) \oplus \mathrm{H}^{1}_{H}(\partial{M}, \mathcal{H}^{-1}i^{\prime*}j^{\prime}_{*}W(1)). 
    \end{equation*}
    We can see 
    \begin{equation*}
        \mathrm{H}^{1}_{H}(\partial{M}, \mathcal{H}^{-1}i^{\prime*}j^{\prime}_{*}W(1)) = 0 
    \end{equation*}
    from the fact that $\mathcal{H}^{-1}i^{\prime*}j^{\prime}_{*}W(1)$ is regular and $\dim \partial M = 0$. 
\end{proof}
\begin{remark}
    From Lemma \label{lemma: Hodge: source of theta}, we can see that it suffices to prove the map
    \begin{equation*}
        \theta_{H} \colon \mathrm{H}^{0}_{H}(\partial{M}, \mathcal{H}^{0}i^{\prime*}j^{\prime}_{*}W(1)) \rightarrow  \mathrm{H}^{1}_{H}(\partial{S}, i^{*}j_{*}V(2))
    \end{equation*}
    is zero. 
\end{remark}

\begin{lemma}
\label{lemma: pullback of deg of MHS}
    We have the following pullback formulas in $\mathrm{D}^{b}(\mathrm{MHM}_{\RR}(\partial M/\RR))$: 
    \begin{align*}
         & q^{*} \mathcal{H}^{-2}i^{*}j_{*}V(2) = \mu(\lambda^{\prime}(a + b + r + 2, a + b + s + 2))[1],  \\
         & q^{*} \mathcal{H}^{-1}i^{*}j_{*}V(2) = \mu(\lambda^{\prime}(a + b + r + 2, b + s + 1))[1] \oplus \mu(\lambda^{\prime}(a + r + 1, a + b + s + 2))[1], \\
         & q^{*} \mathcal{H}^{0}i^{*}j_{*}V(2) = \mu(\lambda^{\prime}(a + r + 1 , s))[1] \oplus \mu(\lambda^{\prime}(r, b + s + 1))[1], \\
         & q^{*} \mathcal{H}^{1}i^{*}j_{*}V(2) = \mu(\lambda^{\prime}(r, s))[1]. 
    \end{align*}
\end{lemma}

\begin{proof}
    First,  it follows from \cite[Proposition 2.3.12]{Bl06} that for $-2 \le l \le 1$, we have 
    \begin{equation*}
        q^{*} \mathcal{H}^{l}i^{*}j_{*}V(2) = (q^{s})^{*} \mathcal{H}^{l}i^{*}j_{*}V(2)[1], 
    \end{equation*}
    where $(q^{s})^{*}$ is the pullback functor in the category of variation of Hodge structures. The desired formulas are now
    direct consequences of Lemma \ref{lemma: BW_G} and equation (\ref{eq: group embedding boundary}).
\end{proof}

\begin{lemma}
\label{Lemma: Hom = 0 by cohomology degree}
    For $-2 \le l \le 0$, we have the following vanishing: 
    \begin{align*}
        \Hom_{\mathrm{D}^{b}(\mathrm{MHM}_{\RR}(\partial M/\RR))}(\mathcal{H}^{0}i^{\prime*}j_{*}^{\prime}W(1), q^{*}\mathcal{H}^{l}i^{*}j_{*}V(2)[1-l]) = 0.  
    \end{align*}
\end{lemma}

\begin{proof}
    From Lemma \ref{lemma: BW_H} and Lemma \ref{lemma: pullback of deg of MHS}, there exist $X, Y \in \mathrm{Var}_{\RR}(M)$ such that 
    \begin{align*}
        & \Hom_{\mathrm{D}^{b}(\mathrm{MHM}_{\RR}(\partial M/\RR))}(\mathcal{H}^{0}i^{\prime*}j_{*}^{\prime}W(1), q^{*}\mathcal{H}^{l}i^{*}j_{*}V(2)[1-l]) \\
        = & \Hom_{\mathrm{D}^{b}(\mathrm{MHM}_{\RR}(\partial M/\RR))}(X, Y[2-l]). 
    \end{align*}
    From the geometry of $\partial M$ and the definition of the derived category, 
    \begin{align*}
        \Hom_{\mathrm{D}^{b}(\mathrm{MHM}_{\RR}(\partial M/\RR))}(X, Y[2-l]) 
    \cong \Hom_{\mathrm{D}^{b}(\mathrm{MHS}_{\RR}^{+})}(X, Y[2-l]) 
    \cong \Ext^{2-l}_{\mathrm{MHS_{\RR}^{+}}}(X, Y). 
    \end{align*}
    Then the lemma follows from the fact that the cohomological dimension of the category $\mathrm{MHS_{\RR}^{+}}$ is $1$. 
\end{proof}

\begin{lemma}
\label{Lemma: Hom = 0 by Hodge type}
We have the following vanishing result: 
\begin{align*}
    \Hom_{\mathrm{D}^{b}(\mathrm{MHM}_{\RR}(\partial M / \RR))}(\mathcal{H}^{0}i^{\prime*}j_{*}^{\prime}W(1), q^{*}\mathcal{H}^{-1}i^{*}j_{*}V(2)) = 0. 
\end{align*}
\end{lemma}

\begin{proof}
It follows from Lemma \ref{lemma: BW_H} and Lemma \ref{lemma: pullback of deg of MHS} that
\begin{align*}
    & \Hom_{\mathrm{D}^{b}(\mathrm{MHM}_{\RR}(\partial M / \RR))}(\mathcal{H}^{0}i^{\prime*}j_{*}^{\prime}W(1), q^{*}\mathcal{H}^{1}i^{*}j_{*}V(2))  \\ 
    = & \Hom_{\mathrm{D}^{b}(\mathrm{MHM}_{\RR}(\partial M / \RR))}(\mu(\lambda^{\prime}(0, 0)), \mu(\lambda^{\prime}(r, s))[1]) \\
    = & \Hom_{\mathrm{D}^{b}(\mathrm{MHS}_{\RR}^{+})}(\RR(0), \mu(\lambda^{\prime}(r, s))[1])  \\
    = & \Ext^{1}_{\mathrm{MHS}_{\RR}^{+}}(\RR(0), \mu(\lambda^{\prime}(r, s))). 
\end{align*}

The weight of $\mu(\lambda^{\prime}(r, s))$ is greater than zero,  so by \cite[Theorem A.2.10]{HW98}, 
\begin{equation*}
    \Ext^{1}_{\mathrm{MHS}_{\RR}^{+}}(\RR(0), \mu(\lambda^{\prime}(r, s))) = 0. 
\end{equation*}
\end{proof}
        
\begin{proposition}
\label{Prop: Hdg theta is zero}
    The map $\theta^{H}$ in Proposition \ref{Prop: Hodge commutative diagram} is zero. 
\end{proposition}
\begin{proof}
    \par From Proposition \ref{Prop: Hodge commutative diagram} and Lemma \ref{lemma: Hodge: source of theta}, in order to prove $\theta^{H}_{S} = 0$ it suffices to prove 
    \begin{equation*}
        \Hom_{\mathrm{D}^{b}(\mathrm{MHM}_{\RR}(\partial S/\RR))}(q_{*}\mathcal{H}^{0}i^{\prime*}j^{\prime}_{*}W(1), i^{*}j_{*}V(2)[1]) = 0. 
    \end{equation*}
    Since $q \colon \partial M \rightarrow \partial S$ is a morphism from a $0$-dimensional scheme to a $0$-dimensional scheme, it is both an open immersion and a closed immersion. Therefore, we have
    \begin{equation*}
        q_{!} = q_{*}, q^{!} = q^{*}, 
    \end{equation*}
    from which we get the following isomorphisms:
    \begin{align*}
        & \Hom_{\mathrm{D}^{b}(\mathrm{MHM}_{\RR}(\partial S/\RR))}(q_{*}\mathcal{H}^{0}i^{\prime*}j_{*}^{\prime}W(1), i^{*}j_{*}V(2)[1]) \\ 
    \cong & \Hom_{\mathrm{D}^{b}(\mathrm{MHM}_{\RR}(\partial S/\RR))}(q_{!}\mathcal{H}^{0}i^{\prime*}j_{*}^{\prime}W(1), i^{*}j_{*}V(2)[1]) \\
    \cong & \Hom_{\mathrm{D}^{b}(\mathrm{MHM}_{\RR}(\partial M/\RR))}(\mathcal{H}^{0}i^{\prime*}j_{*}^{\prime}W(1), q^{!}i^{*}j_{*}V(2)[1]) \\
    \cong & \Hom_{\mathrm{D}^{b}(\mathrm{MHM}_{\RR}(\partial M/\RR))}(\mathcal{H}^{0}i^{\prime*}j_{*}^{\prime}W(1), q^{*}i^{*}j_{*}V(2)[1]), 
    \end{align*}
    where the second isomorphism comes from the adjunction between $q_{!}$ and $q^{!}$. 
    It follows from Lemma \ref{lemma: BW_G} that 
    \begin{equation*}
        i^{*}j_{*}V(2) = \bigoplus_{l = -2}^{1} \mathcal{H}^{l}i^{*}j_{*}V(2)[-l]. 
    \end{equation*}
    Finally, by Lemma \ref{Lemma: Hom = 0 by cohomology degree} and Lemma \ref{Lemma: Hom = 0 by Hodge type}, we complete the proof of the proposition.
\end{proof}

The following corollary follows directly from Proposition \ref{Prop: Hdg theta is zero} and Lemma \ref{lemma: Hodge: source of theta}. 

\begin{corollary}
\label{Corollary: Hodge theta is zero}
    For any 
    \begin{equation*}
        f \in \mathrm{H}^{0}_{H}(\partial M, i^{\prime*}j^{\prime }_{*}W(1)) = \Hom_{\mathrm{D}^{\mathrm{b}}(\mathrm{MHM}_{\RR}(S^{*}/\RR))}(\RR(0)_{S^{*}}, i_{*}q_{*}i^{\prime*}j^{\prime}_{*}W(1)[2]), 
    \end{equation*}
    the element 
    \begin{equation*}
        i_{*} (\partial \iota_{*}^{H}) \circ f \in \Hom_{\mathrm{D}^{\mathrm{b}}(\mathrm{MHM}_{\RR}(S^{*}/\RR))}(\RR(0)_{S^{*}}, i_{*}i^{*}j_{*}V(2)[3]) = \mathrm{H}^{1}_{H}(\partial S, i^{*}j_{*}V(2))
    \end{equation*}
    is $0$. 
\end{corollary}

\begin{remark}
    Although we have proved that the map 
    \begin{equation*}
        \theta^{H}_{S} \colon \mathrm{H}^{0}_{H}(\partial{M}, \mathcal{H}^{0}i^{\prime*}j^{\prime}_{*}W(1)) \rightarrow \mathrm{H}^{1}_{H}(\partial{S}, i^{*}j_{*}V(2)) 
    \end{equation*}
    is zero, the source and the target may not be zero for the following reasons.
    \begin{itemize}
        \item It can be seen that 
            \begin{equation*}
                \mathrm{H}^{0}_{H}(\partial{M}, \mathcal{H}^{0}i^{\prime*}j^{\prime}_{*}W(1)) = \Hom_{\mathrm{MHS}_{\RR}^{+}}(\RR(0), \mathrm{H}^{0}(\partial M, \mathcal{H}^{0}i^{\prime*}j^{\prime}_{*}W(1))). 
            \end{equation*}
            Then by Lemma \ref{lemma: BW_H}, we get
            \begin{equation*}
                \mathrm{H}^{0}_{H}(\partial{M}, \mathcal{H}^{0}i^{\prime*}j^{\prime}_{*}W(1)) = \Hom_{\mathrm{MHS}_{\RR}^{+}}(\RR(0), \RR(0)) = \RR.
            \end{equation*}
            Hence, we have shown that $\mathrm{H}^{0}_{H}(\partial{M}, i^{\prime*}j^{\prime}_{*}W(1)) \neq 0$.  
        \item It follows from Lemma \ref{lemma: BW_G} that 
        \begin{equation*}
              \mathrm{H}^{1}_{H}(\partial{S}, i^{*}j_{*}V(2)) 
             = \bigoplus_{l = - 2}^{1} \mathrm{H}^{-l+1}_{H}(\partial{S}, \mathcal{H}^{i}i^{*}j_{*}V(2)). 
        \end{equation*}
         We can see that 
          \begin{align*}
             \mathrm{H}^{1}_{H}(\partial{S}, \mathcal{H}^{0}i^{*}j_{*}V(2)) = & \Hom_{\mathrm{D}^{b}(\mathrm{MHS}_{\RR}^{+})}(\RR(0), \mathrm{H}^{0}(\partial S, \mathcal{H}^{0}i^{*}j_{*}V(2))[1]) \\
             = & \Ext^{1}_{\mathrm{MHS}_{\RR}^{+}}(\RR(0), \mathrm{H}^{0}(\partial S, \mathcal{H}^{0}i^{*}j_{*}V(2))) \\
             = & \Ext^{1}_{\mathrm{MHS}_{\RR}^{+}}(\RR(0), H_{1}) \oplus \Ext^{1}_{\mathrm{MHS}_{\RR}^{+}}(\RR(0), H_{2}), 
         \end{align*}
         where $H_{1} \in \mathrm{MHS}_{\RR}^{+}$ has Hodge type $(-(a + r + 1), -s)$ and $(-s, -(a + r + 1))$, and $H_{2} \in \mathrm{MHS}_{\RR}^{+}$ has Hodge type $(-r, -(b + s + 1))$ and $(-(b + s + 1), -r)$. 
         It follows from \cite[Theorem A.2.10.]{HW98} that if $a + r + s > 0$, then $\Ext^{1}_{\mathrm{MHS}_{\RR}^{+}}(\RR(0), H_{1}) \neq 0$, 
         and if $b + r + s > 0$, then $\Ext^{1}_{\mathrm{MHS}_{\RR}^{+}}(\RR(0), H_{2}) \neq 0$.
         Hence, we conclude that when $V$ is sufficiently regular, $\mathrm{H}^{1}_{H}(\partial{S}, \mathcal{H}^{0}i^{*}j_{*}V(2)) \neq 0$.
    \end{itemize}
\end{remark}

Finally, by combining Proposition \ref{Prop: Hodge exact sequence}, Proposition \ref{Prop: Hodge commutative diagram} and Proposition \ref{Prop: Hdg theta is zero}, we complete the proof of Theorem \ref{Thm: Hdg vanish on the boudary}.  
