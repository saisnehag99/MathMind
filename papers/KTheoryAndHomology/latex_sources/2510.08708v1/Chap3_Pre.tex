\subsection{Algebraic groups, algebraic representations and branching laws} 
\label{SS:algebraic groups}
In this subsection, we define the reductive group $\GU(2,1)$ and a subgroup $H$ of $G$, together with their representations. 

\subsubsection{Algebraic groups}
\begin{notation}
    Let $E$ be an imaginary quadratic field of discriminant $-D$, and let $x \mapsto \bar{x}$ be the non-trivial Galois automorphism of $E$ over $\QQ$. Let $\mathcal{O}$ be the ring of integers of $E$. We fix an identification of $E \otimes_{\QQ} \RR$ with $\CC$ such that the imaginary part of $\delta = \sqrt{-D}$ is positive.
\end{notation}
\begin{definition}
\label{defn:G_H}
\begin{enumerate}
    \item Let $J \in \GL_3(E)$ be the Hermitian matrix 
        \[
            J = \begin{pmatrix} 
                0 & 0 & \delta^{-1} \\
                0 & 1  & 0 \\
                -\delta^{-1} & 0 & 0 
                \end{pmatrix}, \quad \mathrm{where} \ \delta = \sqrt{-D}, 
        \]
        and let $G = \GU(J)$ be the group scheme over $\mathbb{Z}$ such that for all $\mathbb{Z}$-algebras $R$, we have 
        \[
            G(R) = \{(g, \mu) \in \GL_3(\mathcal{O} \otimes R) \times R^{\times} | \prescript{t}{}{\bar{g}}Jg = \mu J\}. 
        \]
        This is a unitary similitude group and we denote by $\mu \colon G \rightarrow \Gm$ the similitude character. The center $Z_{G}$ of the group $G$ can be identified with $\Res_{\mathcal{O}/\ZZ}(\Gm)$ via 
        \begin{equation}
        \label{eq: Z_{G}}
           z \in \Res_{\mathcal{O}/\ZZ}(\Gm)(R) \mapsto  \begin{pmatrix}
                                                            z & 0 & 0 \\
                                                            0 & z & 0 \\
                                                            0 & 0 & z
                                                         \end{pmatrix} \in Z_{G}(R). 
        \end{equation}
        The group $G$ is quasi-split over $\QQ$ and is split when base change from $\QQ$ to $E$: 
        \begin{equation*}
            G_{E} \cong \GL_{3, E} \times \Gm_{,E}. 
        \end{equation*}
        We define $G_{0} = \ker(\mu)$ as the unitary group. Hence, we have 
        \begin{equation*}
            G(R) = Z_{G}(R) G_{0}(R),
        \end{equation*}
        for all $\mathbb{Z}$-algebras $R$. 
        Note that $G$ is reductive over $\ZZ_{l}$ for $l \nmid D$ (even for $l = 2$). 
    \item Let $H$ be the group scheme over $\mathbb{Z}$ such that for a $\mathbb{Z}$-algebras $R$, we have 
        \[
            H(R) = \{ (g,z) \in \GL_2(R) \times (\mathcal{O} \otimes R)^{\times} | \det(g) = z\bar{z} \}. 
        \]
        The center $Z_{H}$ of the group $H$ is 
        \begin{equation}
        \label{eq: Z_{H}}
            Z_{H}(R) = \{ (\begin{pmatrix}
                               t & 0 \\
                               0 & t 
                           \end{pmatrix}, z) \in \GL_2(R) \times (\mathcal{O} \otimes R)^{\times} | t^{2} = z\bar{z}\}. 
        \end{equation}
        The group $H$ is quasi-split and is split when base change from $\QQ$ to $E$: 
        \begin{equation*}
            H_{E} \cong \GL_{2, E} \times \Gm_{, E}. 
        \end{equation*}
    \item We have the following two maps:
            \[
                \iota \colon H \hookrightarrow G,  \quad (\begin{pmatrix} 
                                                        a & b \\
                                                        c & d \\
                                                     \end{pmatrix}, z) \mapsto (\begin{pmatrix}
                                                                                    a & 0 & b \\
                                                                                    0 & z & 0 \\
                                                                                    c & 0 & d 
                                                                                 \end{pmatrix}, z\bar{z})
            \]
        and 
                    \[p \colon H \twoheadrightarrow \GL_2, \quad (\begin{pmatrix} 
                                                                a & b \\
                                                                c & d \\
                                                             \end{pmatrix}, z) \mapsto \begin{pmatrix} 
                                                                                        a & b \\
                                                                                        c & d \\
                                                                                     \end{pmatrix}.           
        \]
        The relation between the groups $\GL_2$, $H$ and $G$ can be summarized in the diagram: 
        \[
            \begin{tikzcd}
                H \ar[r, hook] \ar[d, twoheadrightarrow] & G  \\
                \GL_2  &  
            \end{tikzcd}. 
        \]
        \item 
            Via the map $\iota$, we define the group scheme $Z$ as 
            \begin{equation}
            \label{eq: Z}
                Z(R) := Z_{G}(R) \cap H(R) = \{ (\begin{pmatrix}
                                                    t & 0 \\
                                                    0 & t 
                                                \end{pmatrix}, t) \in \GL_2(R) \times R^{\times}\},
            \end{equation}
            which is the common center of $G$ and $H$. 
\end{enumerate}
\end{definition}

The definition of $H$ is natural for the following reasons. 
\par Let $J_2 \in \GL_2(E)$ be the Hermitian matrix 
            \[
                J_2 = \begin{pmatrix} 
                     0 & \delta^{-1} \\
                    -\delta^{-1} & 0
                    \end{pmatrix}, \quad \mathrm{where} \ \delta = \sqrt{-D}, 
            \]
            and let $\GU(J_2)$ be the group scheme over $\mathbb{Z}$ such that for all $\mathbb{Z}$-algebras $R$, we have 
            \[
                G(R) = \{(g, \mu) \in \GL_2(\mathcal{O} \otimes R) \times R^{\times} | \prescript{t}{}{\bar{g}}J_2g = \mu J_2\}. 
            \]
Let $\GU(J_2)^{\prime}$ be the subgroup scheme of $\GU(J_2)$ defined by  
\begin{equation*}
        \GU(J_2)^{\prime} = \{ g \in \GU(J_2)| \det(g) = \mu(g) \}. 
\end{equation*}
\begin{fact}
    We have
    \begin{equation*}
        \GU(J_2)^{\prime} = \GL_2, 
    \end{equation*}
    through which the similitude character $\mu$ is identified with the determinant character $\det$. Hence, we use the notation $\mu$ to denote the determinant character for $H$ and $\GL_2$. 
    
\end{fact}
\begin{proof}
     For $g \in \GU(J_2)^{\prime}$, we have $^{t}{\bar{g}}J_2g = J_2$, and if we left multiply both sides by $\left( \begin{matrix}
                                                                                                                            \delta & 0 \\
                                                                                                                            0 & \delta 
                                                                                                                         \end{matrix} \right)$, 
    we have 
    \begin{equation*}
        ^{t}{\bar{g}}\begin{pmatrix}
                        0 & 1 \\
                        -1 & 0
                    \end{pmatrix}g = \mu \begin{pmatrix}
                                        0 & 1 \\
                                        -1 & 0
                                     \end{pmatrix}. 
    \end{equation*}
    Hence, for $g = \left( \begin{matrix}
                              a & b \\
                              c & d
                            \end{matrix} \right)$, we have 
    \begin{align*}
        \begin{pmatrix}
            -\bar{c} & \bar{a} \\
            -\bar{d} & \bar{b}
        \end{pmatrix}=\begin{pmatrix}
                        \bar{a} & \bar{c} \\
                        \bar{b} & \bar{d} 
                      \end{pmatrix} \begin{pmatrix}
                                        0 & 1 \\
                                        -1 & 0 
                                    \end{pmatrix} & = \mu \begin{pmatrix}
                                                            0 & 1 \\
                                                            -1 & 0 
                                                        \end{pmatrix} \begin{pmatrix}
                                                                        d & -b \\
                                                                        -c & a
                                                                      \end{pmatrix} \frac{1}{ad - bc} \\ 
        & = \begin{pmatrix}
                -c & a \\
                -d & b 
            \end{pmatrix} \frac{\mu}{ad - bc} = \begin{pmatrix}
                                                    -c & a \\
                                                    -d & b 
                                                 \end{pmatrix}. 
    \end{align*}
    So we have $a = \bar{a}, b = \bar{b}, c = \bar{c}$ and $d = \bar{d}$, which means $g \in \GL_2$. 
    \par For $g \in \GL_2$, the above process can be reversed. 
\end{proof}

\begin{remark}
The natural subgroup $\GU(J_2)^{\prime} \boxtimes \Res_{\mathcal{O}/\ZZ}(\Gm)$ of $G$, which is defined by 
\begin{equation*}
    \GU(J_2)^{\prime} \boxtimes \Res_{\mathcal{O}/\ZZ}(\Gm)(R) = \{(g, z) \in \GU(J_2)^{\prime}(R) \times (\mathcal{O} \otimes R)^{\times} | \mu(g) = z \bar{z} \}
\end{equation*}
for all $\ZZ$-algebras $R$, can be identified with $H$. 
\end{remark}

\begin{convention}
    When we consider $H$ as an algebraic group over $\QQ$, we are free to use the notation
    \begin{equation*}
        H = \GU(J_2)^{\prime} \boxtimes E^{\times} = \GL_2 \boxtimes E^{\times}. 
    \end{equation*}
\end{convention}

\subsubsection{Algebraic representations}

\begin{convention}
    In this subsubsection, the algebraic groups $G$, $H$ and $\GL_{2}$ are all defined over $\QQ$. 
\end{convention}

\par We first define algebraic representations of $G$. 

\begin{definition}
\label{def:rep_G}
    \begin{enumerate}
        \item Let $\chi_i,i = 1 ,\dots, 4$, be the four characters of the diagonal torus $T/E$ of $G$ such that 
        \begin{align*}
          &  \chi_1 \colon \begin{pmatrix}
                     x & 0 & 0 \\
                     0 & z & 0 \\
                     0 & 0 & \frac{z\bar{z}}{\bar{x}}
                   \end{pmatrix} \mapsto x,\  \chi_2 \colon \begin{pmatrix}
                                                     x & 0 & 0 \\
                                                     0 & z & 0 \\
                                                     0 & 0 & \frac{z\bar{z}}{\bar{x}}
                                                   \end{pmatrix} \mapsto \bar{x}, \\
          &  \chi_3 \colon \begin{pmatrix}
                     x & 0 & 0 \\
                     0 & z & 0 \\
                     0 & 0 & \frac{z\bar{z}}{\bar{x}}
                   \end{pmatrix} \mapsto \frac{\det}{\mu},\  \chi_4 \colon \begin{pmatrix}
                                                                        x & 0 & 0 \\
                                                                        0 & z & 0 \\
                                                                        0 & 0 & \frac{z\bar{z}}{\bar{x}}
                                                                  \end{pmatrix} \mapsto \frac{\overline{\det}}{\mu}.                                           
        \end{align*}
        Here, $x, \bar{x}, z, \bar{z}$ are $4$ independent variables. 
        \item In this paper, we also use the notation $(\mu_1, \mu_2, \mu_3; d)$ to denote the character of the diagonal torus $T$ such that 
              \begin{equation*}
                   t = \begin{pmatrix}
                         a & 0 & 0 \\
                         0 & b & 0 \\
                         0 & 0 & c 
                       \end{pmatrix} \mapsto a^{\mu_1} b^{\mu_2} c^{\mu_3} \mu(t)^{d}. 
              \end{equation*}
        We denote by $\lambda(\mu_1, \mu_2, \mu_3; d)$ the irreducible algebraic representation of $G$ with highest weight $(\mu_1, \mu_2, \mu_3; d)$. When the action of $\mu$ is clear, sometimes we write $(\mu_1, \mu_2, \mu_3)$ for $\lambda(\mu_1, \mu_2, \mu_3; d)$. 
        \item For $a_1, a_2 \ge 0$, let $V^{a_1, a_2}$ denote the representation of $G$ of highest weight $a_1\chi_1 + a_2\chi_2$.
        \item For each representation $V$ of $G$, we write $V\{a_3,a_4\}$ for its twists by $\chi_3^{a_3}\chi_4^{a_4}$.
    \end{enumerate}
\end{definition}
Each irreducible representation of $G$ is $V^{a_1,a_2}\{a_3,a_4\}$ for some $a_1, a_2, a_3, a_4 \in \mathbb{Z}$ with $a_1 \ge 0, a_2 \ge 0$. 

\begin{proposition}
    \begin{enumerate}
        \item Each irreducible representation of $G$ is $V^{a_1,a_2}\{a_3,a_4\}$ for some $a_1, a_2, a_3, a_4 \in \mathbb{Z}$ with $a_1 \ge 0, a_2 \ge 0$.
        \item The highest weight of $V^{a_1,a_2}\{a_3,a_4\}$ is $(a_1 + a_3 - a_4, a_3 - a_4, a_3 - (a_2 + a_4); 2a_4 + a_2 - a_3)$. 
        \item The contragredient representation $D^{a_1, a_2}$ of $V^{a_1, a_2}$ is $V^{a_2, a_1}\{-(a_1 + a_2), -(a_1 + a_2) \}$. 
    \end{enumerate}
\end{proposition}

\begin{proof}
  \begin{enumerate}
      \item The special unitary group $SU(J)$ is the subgroup of $G$ such that $\det = \mu = 1$, which is a semisimple algebraic group. It is enough to consider the algebraic representation of $\SU(J) \cong \SL_{3, E}$. For $\SL_{3, E}$, the fundamental weights are $\omega_1 = (1, 0)$ and  $\omega_2 = (1, 1)$. 
      For any $t = \begin{pmatrix} 
                             x & 0 & 0 \\
                             0 & z & 0 \\
                             0 & 0 & \frac{1}{xz}
                          \end{pmatrix}  \in T_{E}$, we have $\chi_1(t) = x = \omega_1(t)$ and $\chi_2(t) = \bar{x} = xz = \omega_2(t)$. 
        Hence, by highest weight theory, every irreducible algebraic representation of $\SU(J)_{E}$ is determined by 
        \begin{equation*}
            a_1 \omega_1 + a_2 \omega_2 = a_1 \chi_1 + a_2 \chi_2,
        \end{equation*}
        for $a_1 \ge 0$ and $a_2 \ge 0$. Hence, every irreducible algebraic representation of $G$ is of the form $V^{a_1,a_2}\{a_3,a_4\}$ for some $a_1, a_2, a_3, a_4 \in \mathbb{Z}$ with $a_1 \ge 0, a_2 \ge 0$.
      \item The highest weight of $V^{a_1,a_2}\{a_3,a_4\}$ is 
            \begin{align*}
                t = \left( \begin{pmatrix} 
                             x & 0 & 0 \\
                             0 & z & 0 \\
                             0 & 0 & \frac{z\bar{z}}{\bar{x}}
                          \end{pmatrix}, z\bar{z}\right) \in T_{E} \mapsto \, &x^{a_1} (\bar{x})^{a_2}  (\frac{xz}{\bar{x}})^{b_1}(\frac{\bar{x}\bar{z}}{x})^{b_2} \\
                          = &x^{a_1 + b_1 - b_2}\bar{x}^{a_2 + b_2 - b_1}z^{b_1} \bar{z}^{b_2} \\
                          = & x^{a_1 + b_1 - b_2} z^{b_1 - b_2} (\frac{z \bar{z}}{\bar{x}})^{b_1 - (a_2 + b_2)}(z\bar{z})^{2b_2 + a_2 - b_1}. 
            \end{align*}
      \item First, we have 
            \begin{align*}
                a_1 \omega_1 + a_2 \omega_2 &= a_1 (\frac{2}{3}, -\frac{1}{3}, -\frac{1}{3}) + a_2 (\frac{1}{3}, \frac{1}{3}, -\frac{2}{3}) \\
                                            &= (\frac{2a_1 + a_2}{3}, \frac{-a_1 + a_2}{3}, \frac{-a_1 - 2a_2}{3}). 
            \end{align*}
            Hence, when restrict to $\SU(J)$, the highest weight of $D^{a_1, a_2} = (V^{a_1, a_2})^{\vee}$ is 
            \begin{align*}
                (\frac{a_1 + 2a_2}{3},  \frac{a_1 - a_2}{3},  \frac{-2a_1 - a_2}{3}) = a_2 \omega_1 + a_1 \omega_2. 
            \end{align*}
          Second, for the $t$ in (2), we have 
          \begin{align*}
              \chi_1(t) = x & = x^{\frac{2}{3}}z^{-\frac{1}{3}}(\frac{z\bar{z}}{\bar{x}})^{-\frac{1}{3}} \cdot (x \cdot z \cdot (\frac{z\bar{z}}{\bar{x}}))^{\frac{1}{3}} \\
                            & = \omega_1(t) \chi_3(t)^{\frac{2}{3}} \chi_4(t)^{\frac{1}{3}}.
          \end{align*}
          Similarly, we have 
          \begin{equation*}
              \chi_2(t) = \omega_2(t) \chi_3(t)^{\frac{1}{3}} \chi_4(t)^{\frac{2}{3}}. 
          \end{equation*}
          So the highest weight of $D^{a_1, a_2}$ is $\omega_1^{a_2}\omega_2^{a_1} \chi_3^{\frac{-2a_1 - a_2}{3}} \chi_4^{\frac{-a_1 - 2a_2}{3}}$ which is 
          \begin{equation*}
              \chi_1^{a_2} \chi_2^{a_1} \chi_3^{-(a_1 + a_2)} \chi_4^{-(a_1 + a_2)}. 
          \end{equation*}
  \end{enumerate}
\end{proof}
\begin{remark}
\label{remark: parameter of repn transform}
    For later use, we record the highest weight and central character in the following table. 
    \begin{equation}
        \begin{array}{|c|c|c|}
            \hline  \text{Representation}& \text{Highest weight} & \text{central character} \\
            \hline V = V^{a, b} \{r, s\} & (a + r - s, r - s, r - s - b; 2s - r + b) &  a + b + r + s\\
            \hline D = V^{\vee} & (b + s - r, s - r, s - r- a; -2s + r - b) & -a -b - r -s \\
            \hline 
            \end{array}
    \end{equation}
\end{remark}

Now we define irreducible representations of $H$. 
\begin{definition}
\label{def:repn_H}
    \begin{enumerate}
        \item For each representation $W$ of $H$, we write $W\{a_3,a_4\}$ for its twists by $\chi_3^{a_3}\chi_4^{a_4}$ through $\iota \colon H \hookrightarrow G$. 
        \item In this paper, we also use the notation $(\mu_1, \mu_2; d)$ to denote the character of the diagonal torus
              \begin{equation*}
                   t = (\begin{pmatrix}
                         a & 0 \\
                         0 & b \\
                        \end{pmatrix}; z) \in H \mapsto a^{\mu_1} b^{\mu_2} z^{d}. 
              \end{equation*}
              We denote by $\lambda^{\prime}(\mu_1, \mu_2; d)$ the irreducible algebraic representation of $H$ with highest weight $(\mu_1, \mu_2; d)$.
    \end{enumerate}
\end{definition}

\begin{convention}
    When the action of $\mathrm{Res}_{E/\QQ}\Gm$ in $H$ is clear, we sometimes write $(\mu_1, \mu_2)$ instead of $(\mu_1, \mu_2; d)$. 
\end{convention}

\begin{remark}
    Along the map $\iota$, the restriction of $\chi_3$ and $\chi_4$ is 
    \begin{align*}
        & \chi_3 \colon (\begin{pmatrix}
                    a & b \\
                    c & d
                 \end{pmatrix}, z) \in H \mapsto z \\
         & \chi_4 \colon (\begin{pmatrix}
                    a & b \\
                    c & d
                 \end{pmatrix}, z) \in H \mapsto \bar{z}. 
    \end{align*}
    So the representation $W\{a_3,a_4\}$ is the representation $W$ with a twist by the character $z \mapsto z^{a_3}\bar{z}^{a_4}$. 
\end{remark}


Hence, we can get the following proposition: 
\begin{proposition}
    \begin{enumerate}
        \item  For $b_1 \ge 0$, let $W^{b_1}$ be the representation $\mathrm{Sym}^{b_1}V_{2}$ of $H$, where $V_{2}$ denotes the pullback of the standard representation of $\GL_2$ through the map $p \colon: H \twoheadrightarrow \GL_2$. \\
        Then every irreducible representation of $H$ has the form $W^{b_1}\{b_2,b_3\}$ for some $b_1, b_2,b_3 \in \mathbb{Z}$ with $b_1 \ge 0$.
        \item The contragredient of $W^{b_1}$ is $W^{b_1}\{-b_1, -b_1\}$. 
    \end{enumerate}
\end{proposition}

\subsubsection{Branching law}

\par The following is a corollary of the classical branching law for $\GL_2 \subset \GL_3$ \cite[Lemma 8.3.1]{Goodman09}.
\begin{proposition}
    For an irreducible representation $V^{a_1,a_2}\{b_1,b_2\}$ of $G$ satisfying: 
    \begin{align*}
        & 0 \le -b_1 \le a_1, \\
        & 0 \le -b_2 \le a_2, 
    \end{align*}
    we have an embedding in the category of representations of $H$:
    \[                
        W^{n} \hookrightarrow \iota^{*} V^{a_1,a_2}\{b_1,b_2\}, 
    \]
    where $n = a_1 + a_2 + b_1 + b_2$ and $\iota^{*}$ denotes the restriction of the representation of $G$ to $H$ through $\iota$. 
\end{proposition}
\begin{proof}
    The highest weight of $W^{n}$ is $(a_1 + a_2 + b_1 + b_2, 0)$ and the highest weight of $V^{a_1,a_2}\{b_1,b_2\}$ is 
    $(a_1 + b_1 - b_2 , b_1 - b_2, b_1 - (a_2 + b_2); 2b_2 + a_2 - b_1)$. 
    \par First, the common center of $G$ and $H$ is isomorphic to $\Gm$ via 
         \begin{equation*}
             z\in \Gm \mapsto (\begin{pmatrix}
                                z & 0 \\
                                0 & z
                              \end{pmatrix},z)\in H. 
         \end{equation*}
         Its action on $W^{n}$ is via
         \begin{equation*}
              z \mapsto z^{n}
         \end{equation*}
         and its action on $\iota^{*} V^{a_1,a_2}\{b_1,b_2\}$\ is 
         \begin{equation*}
             z \mapsto z^{a_1 + a_2 + b_1 + b_2}. 
         \end{equation*}
         Hence, their central actions are compatible. 
    \par Second, the highest weight of $W^{n}\{-(2b_2 + a_2 - b_1), -(2b_2 + a_2 - b_1)\}$ is $(a_1 + 2b_1 - b_2, b_1 -2b_2 - a_2)$. Up to conjugation, by the classical branching $\GL_2 \subset \GL_3$ \cite[Lemma 8.3.1]{Goodman09}, in order to have the branching 
    \begin{equation*}
         W^{n}\{-(2b_2 + a_2 - b_1), -(2b_2 + a_2 - b_1)\} \hookrightarrow (\iota^{*} V^{a_1,a_2}\{b_1,b_2\}) \{-(2b_2 + a_2 - b_1), -(2b_2 + a_2 - b_1)\}
    \end{equation*}
    we must have 
    \begin{equation*}
        a_1 + b_1 - b_2 \ge a_1 + 2b_1 - b_2 \ge b_1 - b_2 \ge b_1 -2b_2 -a_2 \ge b_1 - (a_2 + b_2), 
    \end{equation*}
    which is equivalent to 
    \begin{align*}
        & 0 \le -b_1 \le a_1, \\
        & 0 \le -b_2 \le a_2. 
    \end{align*}
\end{proof}

\begin{remark}
In this paper, we only consider the case $W^{n} \hookrightarrow \iota^{*} V^{a, b}\{r, s\}$ when $n = a + b + r + s$ and will write $W$ instead of $W^{n}$ and $V$ instead of $V^{a,b} \{r, s\}$ to simplify notations. We will also denote by $D$ the contragredient of $V$. 
\end{remark}

\begin{notation}
    \begin{itemize}
        \item Throughout the paper, we always let $n = a + b + r + s$. 
        \item Throughout the paper, we always use $V$ to denote the algebraic representation $V = V^{a, b} \{r, s\}$ of $G$, $D$ to denote the contragredient of $V$ and $W = W^{n}$ to denote the algebraic representation of $H$ except in the situation where we will explain the meaning. 
        \item For a representation $V$ of $G$, $H$ or $\GL_2$ and an integer $d$, we use the notation $V(d)$ denote the representation $V$ with a twist of the character $\mu^{d}$. 
    \end{itemize}
   
\end{notation}

\subsection{Lie groups and representations}
\label{SS: Lie groups}
\subsubsection{Lie groups and Lie algebras}
\begin{notation}
To describe the maximal compact subgroup of $G(\RR)$ easily, when considering representations of Lie groups, we use the Hermitian form 
\begin{equation*}
    J^{\prime} = \begin{pmatrix}
                    1 & 0 & 0 \\
                    0 & 1 & 0 \\
                    0 & 0 & -1 
                 \end{pmatrix} 
\end{equation*}
to define the group $G = \GU(J^{\prime})$ and use the Hermitian form 
\begin{equation*}
    J_2^{\prime} = \begin{pmatrix}
                    1 & 0 \\
                    0 & -1 
                 \end{pmatrix} 
\end{equation*}
to define the group $\GU(J_2^{\prime})^{\prime}$, and then define the group $H = \GU(J_2^{\prime})^{\prime} \boxtimes (\Res_{E/\QQ}(\Gm))$. 
The definition of $G$ (resp., $H$) is the same as in Definition \ref{defn:G_H} except that we use $J^{\prime}$ (resp., $J^{\prime}_{2}$) instead of $J$ (resp., $J_{2}$).
\end{notation}

\begin{proposition}
    \begin{enumerate}
        \item If we let 
              \begin{equation*}
                  C = \begin{pmatrix}
                         \frac{D^{\frac{1}{4}}}{\sqrt{2}} & 0 & \frac{D^{\frac{1}{4}}}{\sqrt{2}} \\
                         0 & 1 & 0 \\
                          -\frac{D^{\frac{1}{4}}}{\sqrt{-2}} & 0 & \frac{D^{\frac{1}{4}}}{\sqrt{-2}}
                      \end{pmatrix}, 
              \end{equation*}
              then we have the following isomorphism of real Lie groups: 
              \begin{equation*}
                  \begin{tikzcd}[row sep = 0]
                   \GU(J)(\RR)  \ar[r, "\cong"] & \GU(J^{\prime})(\RR) \\
                      g \ar[r, mapsto] & C^{-1}gC
                  \end{tikzcd}. 
              \end{equation*}
        \item If we let 
              \begin{equation*}
                  C_2 = \begin{pmatrix}
                         \frac{D^{\frac{1}{4}}}{\sqrt{2}} & \frac{D^{\frac{1}{4}}}{\sqrt{2}} \\
                          -\frac{D^{\frac{1}{4}}}{\sqrt{-2}} &  \frac{D^{\frac{1}{4}}}{\sqrt{-2}}
                      \end{pmatrix}, 
              \end{equation*}
              then we have the following isomorphism 
              \begin{equation*}
                  \begin{tikzcd}[row sep = 0]
                   \GU(J_{2})^{\prime}(\RR)  \ar[r, "\cong"] & \GU(J_{2}^{\prime})^{\prime}(\RR) \\
                      g \ar[r, mapsto] & C_{2}^{-1}gC_{2}
                  \end{tikzcd}. 
              \end{equation*}
    \end{enumerate}
\end{proposition}

\begin{proof}
    The proof follows from direct computation. 
\end{proof}

\begin{convention}
    In this paper, we write $G(\RR)$ (resp., $H(\RR)$ and $\GL_{2}(\RR)$) for $\GU(J^{\prime})(\RR)$ (resp., 
    $(\GU(J_{2}^{\prime})^{\prime} \boxtimes E^{\times})(\RR)$ and $\GU(J_{2}^{\prime})^{\prime}(\RR)$). 
\end{convention}

Let $K_{G}$ be the real Lie group $A_{G}(\U(2) \times \U(1))(\RR)$,  $K_{H}$ be the real Lie group $A_{H}(\U(1) \times \U(1))(\RR)$ and $K_{\GL_2}$ be the real Lie group $A_{\GL_{2}}\U(1)(\RR)$. Let $\lieg$ be the Lie algebra of $G(\RR)$, $\lieh$ be the Lie algebra of $H(\RR)$ and $\liegl_{2}$ be the Lie algebra of $\GL_2(\RR)$. Then by the 
Iwasawa decomposition, we have the following decomposition of complexified Lie algebra: 
\begin{align*}
    & \lieg_{\CC} = \liek_{G, \CC} \oplus \liep_{\CC}, \\
    & \lieh_{\CC} = \liek_{H, \CC} \oplus \liep_{H, \CC}, \\
    & \liegl_{2, \CC} = \liek_{H, \CC} \oplus \liep_{H, \CC}, 
\end{align*}
where $\liek_{\CC}$, $\liek_{H, \CC}$ and $\liek_{\GL_2, \CC}$ are the complexified Lie algebras of $K_{G}$, $K_{H}$ and $K_{\GL_{2}}$, which is
\begin{align*}
    & \liek_{\CC} = \{\begin{pmatrix}
                        * & * & 0 \\
                        * & * & 0 \\
                        0 & 0 & * 
                      \end{pmatrix}\} \subset \lieg_{\CC}, \\ 
    & \liek_{H, \CC} = \{\begin{pmatrix}
                            * & 0 \\
                            0 & * 
                        \end{pmatrix}\} \oplus \liegl_{1, \CC} \subset \lieh_{\CC}, \\
    & \liek_{\GL_{2}, \CC} = \{\begin{pmatrix}
                            * & 0 \\
                            0 & * 
                        \end{pmatrix}\} \subset \liegl_{2, \CC}; 
\end{align*}
and $\liep_{\CC}$, $\liep_{H, \CC}$ and $\liep_{\GL_{2}, \CC}$ are the sub Lie algebras of $\lieg_{\CC}$, $\lieh_{\CC}$ and $\liegl_{2, \CC}$: 
\begin{align*}
    & \liep_{\CC} = \{\begin{pmatrix}
                    0 & 0 & * \\
                    0 & 0 & * \\
                    * & * & 0 
                  \end{pmatrix}\} \subset \lieg_{\CC}, \\ 
    & \liep_{H, \CC} = \{\begin{pmatrix}
                            0 & * \\
                            * & 0 
                        \end{pmatrix}\} \subset \lieh_{\CC}, \\
    & \liep_{\GL_{2}, \CC} = \{\begin{pmatrix}
                            0 & * \\
                            * & 0 
                        \end{pmatrix}\} \subset \liegl_{2, \CC}. 
\end{align*}
We also have a natural embedding $\iota_{*} \colon \lieh_{\CC} \hookrightarrow \lieg_{\CC}$ induced by the map $\iota \colon H \hookrightarrow G$ that maps $\liep_{H, \CC}$ to $\liep_{\CC}$. Explicitly, the map is given by 
\begin{equation*}
    \begin{pmatrix}
            0 & A \\
            B & 0 
    \end{pmatrix} \in  \liep_{H, \CC}  \mapsto \begin{pmatrix}
                                                    0 & 0 & A \\
                                                    0 & 0 & 0 \\
                                                    B & 0 & 0 
                                                \end{pmatrix} \in \liep_{\CC}. 
\end{equation*}

\subsubsection{Representation Theory of compact Lie groups}
\label{SSS: repn of compact Lie group}
We recall some basic facts of representation theory of the compact Lie group $(\U(2) \times \U(1))(\RR)$.

\par The compact torus $T_c$ is $(\U(1) \times \U(1) \times \U(1))(\RR)$. Hence, the dominant weights of $T_c$ are parametrized by the triples $(\lambda_1, \lambda_2, \lambda_3)$ such that $\lambda_1, \lambda_2, \lambda_3 \in \ZZ$ and $\lambda_1 \ge \lambda_2 \ge \lambda_3$. By highest weight theory of representations of compact Lie groups, for each dominant weight $(\lambda_1, \lambda_2, \lambda_3)$, there is an unique irreducible representation $\tau_{(\lambda_1, \lambda_2, \lambda_3)}$ of $(\U(2) \times \U(1))(\RR)$ whose highest weight is $(\lambda_1, \lambda_2, \lambda_3)$. 

By the Weyl dimension formula, we have 
\begin{equation*}
    \dim \tau_{(\lambda_1, \lambda_2, \lambda_3)} = \lambda_1 - \lambda_2 + 1. 
\end{equation*}

We denote $d = \lambda_1 - \lambda_2 + 1$. 
Actually, there exists a basis $(\nu_s)_{0 \le s \le d}$ of $\tau_{(\lambda_1, \lambda_2, \lambda_3)}$ such that 
\begin{align}
\label{eqn: action on K-type}
    & \left( \begin{matrix}
                1 & 0 \\
                0 & 0 
             \end{matrix} \right) \nu_s = (s + \lambda_2) \nu_s, \\
     & \left( \begin{matrix}
                0 & 0 \\
                0 & 1 
             \end{matrix} \right) \nu_s = (-s + \lambda_1) \nu_s, \\ 
      & \left( \begin{matrix}
                0 & 1 \\
                0 & 0 
             \end{matrix} \right) \nu_s = (s + 1) \nu_{s + 1}, \\ 
      & \left( \begin{matrix}
                0 & 0 \\
                1 & 0 
             \end{matrix} \right) \nu_s = (d - s + 1) \nu_{s - 1}.   
\end{align}

\begin{convention}
    \begin{itemize}
        \item We will use $K_{G}$ (resp., $K_{H}$ and $K_{\GL_{2}}$) to denote $(\U(2) \times \U(1))(\RR)$ (resp., $(\U(1) \times \U(1))(\RR)$ and $\U(1)(\RR)$) when the action $A_{G}$ is clear.
        \item We will use the coordinate system $(\lambda_1, \lambda_2, \lambda_3)$ in subsection \ref{SS:discrete series} and \ref{SS: Hodge decomposition} to denote the Blattner parameter and Harish-Chandra parameter of a discrete series representation of a real reductive Lie group. 
    \end{itemize}
\end{convention}

\subsection{Shimura varieties}
\label{SS:Shimura varieties}
In this subsection, we define the Shimura varieties associated to $G$, $H$ and $\GL_{2}$. 

    
\begin{notation}
    Throughout the paper, we let $\mathbb{S} = \Res_{\CC/\RR} \Gm$ be the Deligne torus. 
\end{notation}

\begin{definition}
\label{def: Shimura variety}
    \begin{enumerate}
        \item Let $X_{G}$ be the $G(\RR)$-conjugacy class of the morphism $h \colon \mathbb{S} \rightarrow G_{\RR}$ given by 
              \begin{equation*}
                  z = x + iy \mapsto \begin{pmatrix}
                                        x & 0 & y \\
                                        0 & z & 0 \\
                                        -y & 0 & x
                                     \end{pmatrix}.
              \end{equation*}
              The pair $(G, X_{G})$ is a (pure) Shimura datum associated to $G$. 
              \par The Hodge cocharacter $\mu_{\CC}$ of $\Gm_{, \CC}$ associated to $h$ is given on $\CC$-points by 
              \begin{equation*}
                  z \mapsto \begin{pmatrix}
                                z & 0 & 0 \\
                                0 & z & 0 \\
                                0 & 0 & 1 
                            \end{pmatrix}.
              \end{equation*}
              Since it is not complex conjugation invariant, we can see that the reflex field of $(G, X_{G})$ is $E$ with a the fixed complex embedding of $E$ into $\CC$ \cite[Lemma 4.2., P15]{LR92}. 
              For any neat compact open subgroup $L$ of $G(\AAA_f)$, there is a smooth \textit{quasi-projective} $E$-scheme $\Sh_{G}(L)$ such that as complex analytic varieties, we have 
              \begin{equation*}
                  \Sh_{G}(L)_{\CC}^{an} = G(\QQ) \backslash (X_{G} \times G(\AAA_f) / L), 
              \end{equation*}
              where $\Sh_{G}(L)_{\CC}^{an}$ is the analytification of the base change of $\Sh_{G}(L)$ to $\CC$. 
              \item  Let $X_{H}$ be the $H(\RR)$-conjugacy class of the morphism $h \colon \mathbb{S} \rightarrow H_{\RR}$ given by 
              \begin{equation*}
                  z = x + iy \mapsto (\begin{pmatrix}
                                        x & y \\
                                        -y & x
                                     \end{pmatrix}, z). 
              \end{equation*}
              The pair $(H, X_H)$ is a (pure) Shimura datum associated to $H$. 
              \par The Hodge cocharacter $\mu_{\CC}$ of $\Gm_{, \CC}$ associated to $h$ is given on $\CC$-points by 
              \begin{equation*}
                  z \mapsto (\begin{pmatrix}
                                z & 0 \\
                                0 & 1 
                            \end{pmatrix}, z).
              \end{equation*}
              The reflex field of $(H, X_H)$ is $E$ with a fixed complex embedding of $E$ into $\CC$. 
              For any neat compact open subgroup $K$ of $H(\AAA_f)$, there is a smooth \textit{quasi-projective} $E$-scheme $\Sh_{H}(K)$ such that as complex analytic varieties, we have 
              \begin{equation*}
                  \Sh_{H}(K)_{\CC}^{an} = H(\QQ) \backslash (X_H \times H(\AAA_f) / K), 
              \end{equation*}
              where $\Sh_{H}(K)_{\CC}^{an}$ is the analytification of the base change of $\Sh_{H}(K)$ to $\CC$.
        \item Let $X_{\GL_2}$ be the $\GL_2(\RR)$-conjugacy class of the morphism $h \colon \mathbb{S} \rightarrow \GL_{2,\RR}$ that is given by 
              \begin{equation*}
                  z \mapsto \begin{pmatrix}
                                x & y \\
                                -y & x
                              \end{pmatrix},
              \end{equation*}
              The pair $(\GL_2, X_{\GL_2})$ is a (pure) Shimura datum associated to $\GL_2$. 
              \par The Hodge cocharacter $\mu_{\CC}$ of $\Gm_{, \CC}$ associated to $h$ is given on $\CC$-points by 
              \begin{equation*}
                  z \mapsto \begin{pmatrix}
                                z & 0 \\
                                0 & 1 
                            \end{pmatrix}.
              \end{equation*}
              The reflex field of $(\GL_2, X_{\GL_{2}})$ is $\QQ$. 
              For any neat compact open subgroup $K^{\prime}$ of $\GL_{2}(\AAA_f)$, there is a smooth \textit{quasi-projective} $\QQ$-scheme $\Sh_{\GL_{2}}(K^{\prime})$ such that as complex analytic varieties, we have 
              \begin{equation*}
                    \Sh_{\GL_2}(K^{\prime})_{\CC}^{an} = \GL_{2}(\QQ) \backslash (X_{\GL_2} \times \GL_2(\AAA_f) / K^{\prime}), 
              \end{equation*}
              where $\Sh_{\GL_2}(K^{\prime})_{\CC}^{an}$ means the analytification of the base change of $\Sh_{\GL_2}(K^{\prime})$ to $\CC$.
    \end{enumerate}
\end{definition}
\begin{remark}
    \begin{itemize}
        \item Our choice of Shimura datum is standard, bu is a little different from \cite[8.1.1]{LSZ22} (see the difference in \cite[Remark 8.1.1]{LSZ22}). 
        \item The maps $\iota \colon H \hookrightarrow G$ and $p \colon H \twoheadrightarrow \GL_2$ of algebraic groups induce the following $E$-morphisms of Shimura varieties for suitable neat compact open subgroup $L$, $K$ and $K^{\prime}$ of $G(\AAA_f)$, $H(\AAA_f)$ and $\GL_2(\AAA_f)$: 
        \begin{equation}
        \label{eq: morphism of Shimura varieties over E}
            \begin{tikzcd}
                \Sh_{H}(K) \ar[r, hook, "\iota"] \ar[d, rightarrow, "p"] & \Sh_{G}(L) \\
                \Sh_{\GL_2}(K^{\prime})_{E}  &  
            \end{tikzcd},
        \end{equation}
        where $\iota$ is a closed immersion and $\Sh_{\GL_2}(K^{\prime})_{E}$ is the base change of $\Sh_{\GL_2}(K^{\prime})$ to $E$. 
    \end{itemize}
\end{remark}

\begin{fact}
    \begin{itemize}
        \item The decomposition
              \begin{equation*}
                  \lieg_{\CC} = \liek_{G, \CC} \oplus \liep_{G, \CC}
              \end{equation*}
              can be further decomposed as 
              \begin{equation*}
                  \lieg_{\CC} = \liek_{G, \CC} \oplus \liep^{+}_{G, \CC} \oplus \liep^{-}_{G, \CC}, 
              \end{equation*}
              where $\liep^{+}_{G, \CC}$ is 
              \begin{equation*}
                  \{\begin{pmatrix}
                    0 & 0 & * \\
                    0 & 0 & * \\
                    0 & 0 & 0 
                  \end{pmatrix}\} \subset \lieg_{\CC}, 
              \end{equation*}
              on which $(z, \bar{z}) \in \mathbb{S}(\CC)$ acts by $\frac{z}{\bar{z}}$ and
              $\liep^{-}_{\CC}$ is 
              \begin{equation*}
                  \{\begin{pmatrix}
                    0 & 0 & 0 \\
                    0 & 0 & 0 \\
                    * & * & 0 
                  \end{pmatrix}\} \subset \lieg_{\CC}, 
              \end{equation*}
              on which $(z, \bar{z}) \in \mathbb{S}(\CC)$ acts by $\frac{\bar{z}}{z}$. 
        \item Similarly, we have the decomposition for $H$: 
              \begin{equation*}
                  \lieh_{\CC} = \liek_{H, \CC} \oplus \liep^{+}_{H, \CC} \oplus \liep^{-}_{H, \CC}, 
              \end{equation*}
              where $\liep^{+}_{H, \CC}$ is 
              \begin{equation*}
                   \{\begin{pmatrix}
                        0 & * \\
                        0 & 0 
                     \end{pmatrix}\} \subset \lieh_{\CC}
              \end{equation*} and $\liep^{-}_{H, \CC}$ is 
              \begin{equation*}
                   \{\begin{pmatrix}
                        0 & 0 \\
                        * & 0 
                     \end{pmatrix}\} \subset \lieh_{\CC}. 
              \end{equation*}
        \item We have a decomposition for $\GL_{2}$: 
              \begin{equation*}
                  \liegl_{2, \CC} = \liek_{\GL_{2}, \CC} \oplus \liep^{+}_{\GL_{2}, \CC} \oplus \liep^{-}_{\GL_{2}, \CC}, 
              \end{equation*}
              where $\liep^{+}_{\GL_{2}, \CC}$ is 
              \begin{equation*}
                   \{\begin{pmatrix}
                        0 & * \\
                        0 & 0 
                     \end{pmatrix}\} \subset \liegl_{2, \CC}
              \end{equation*} and $\liep^{-}_{\GL_{2}, \CC}$ is 
              \begin{equation*}
                   \{\begin{pmatrix}
                        0 & 0 \\
                        * & 0 
                     \end{pmatrix}\} \subset \liegl_{2, \CC}. 
              \end{equation*}
    \end{itemize}
\end{fact}

\begin{convention}
\label{Convention: M_{Q} and S_{Q}}
    \begin{enumerate}
        \item We use $M(K)$ (resp., $S(L)$) to denote $\Res_{E/\QQ}(\Sh_{H}(K))$ (resp., $\Res_{E/\QQ}(\Sh_{G}(L))$), where $\Res_{E/\QQ}$ is the Weil restriction. Hence, we have 
        \begin{align*}
            & S(L)_{\CC}^{an} = \Sh_{G}(L)_{\CC}^{an} \bigsqcup \overline{\Sh_{G}(L)_{\CC}^{an}}, \\
            & M(K)_{\CC}^{an} = \Sh_{H}(K)_{\CC}^{an} \bigsqcup \overline{\Sh_{H}(K)_{\CC}^{an}}, 
        \end{align*}
        where $\overline{\Sh_{G}(L)_{\CC}^{an}}$ (resp., $\overline{\Sh_{H}(K)_{\CC}^{an}}$) is the complex conjugate of $\Sh_{G}(L)_{\CC}^{an}$ (resp., $\Sh_{H}(K)_{\CC}^{an}$). 
        \item Since we compute complex regulators in this paper, an explicit choice of a level structure of a Shimura variety is not necessary. Hence, we will sometimes ignore the level structure in the notation related to Shimura varieties and write $S$, $M$ and $\Sh_{\GL_2}$ instead of $S(L)$, $M(K)$ and $\Sh_{\GL_2}(K^{\prime})$ when there is no confusion. If reader prefers, this can be explained as a Shimura variety with suitable neat level structure. 
        \item Applying the Weil restriction functor $\Res_{E/\QQ}$ to the diagram (\ref{eq: morphism of Shimura varieties over E}), we get the following $\QQ$-morphisms of Shimura varieties: 
        \begin{equation}
        \label{eq: morphism of Shimura varieties over Q}
            \begin{tikzcd}
                M \ar[r, hook, "\iota"] \ar[d, rightarrow, "p"] & S \\
                \Sh_{\GL_2}  &  
            \end{tikzcd},
        \end{equation}
        where $\iota$ is a closed immersion. 
    \end{enumerate}
\end{convention}

\subsection{Cohomology of Picard modular surfaces}
\label{SS: cohomology of Picard modular surfaces}

\begin{convention}
\label{convention: V^{Q} and W_{Q}}
    \begin{enumerate}
        \item For any $V \in \mathrm{Rep}_{E}(G)$ (resp., $W \in \mathrm{Rep}_{E} \in \mathrm{Rep}_{E}(H)$), we denote by $\overline{V}$ (resp., $\overline{W}$) the twist of $V$ (resp., $W$) by the non-trivial element of $\Gal(E/\QQ)$. 
        \item For any morphism $(f\colon V \rightarrow V^{\prime})\in \mathrm{Rep}_{E}(G)$ (resp., $(g\colon W \rightarrow W^{\prime}) \in \mathrm{Rep}_{E}(H)$), we denote by $\overline{f}\colon \overline{V} \rightarrow \overline{V^{\prime}}$ (resp., $\overline{g}\colon \overline{W} \rightarrow \overline{W^{\prime}}$) the twist of $f$ (resp., $g$) by the non-trivial element of $\Gal(E/\QQ)$. 
        \item We denote by $\mathrm{Rep}_{\QQ}(G)$ (resp., $\mathrm{Rep}_{\QQ}(H)$) the cateogry whose objects are $(V, \overline{V})$ (resp., $(W, \overline{W})$), where $V \in \mathrm{Rep}_{E}(G)$ (resp., $W \in \mathrm{Rep}_{E}(H)$); and whose morphisms are $(f, \overline{f})$ (resp., $(g, \overline{g})$), where $(f: V \rightarrow V^{\prime}) \in \mathrm{Rep}_{E}(G)$ (resp., $(g: W \rightarrow W^{\prime}) \in \mathrm{Rep}_{E}(H)$). 
        \item We denote by $V \in \mathrm{Rep}_{\QQ}(G)$ (resp., $W \in \mathrm{Rep}_{\QQ}(H)$) the pair $(V, \overline{V})$ (resp., $(W, \overline{W})$). When talking about weights of $V \in \mathrm{Rep}_{\QQ}(G)$ (resp., $W \in \mathrm{Rep}_{\QQ}(H)$), we offen mean the weights of $V \in \mathrm{Rep}_{E}(G)$ (resp., $W \in \mathrm{Rep}_{E}(H)$). 
        \item For any $V \in \mathrm{Rep}_{\QQ}(G)$ (resp., $W \in \mathrm{Rep}_{\QQ}(H)$), there is a canonical local system over $S(L)$ (resp., $M(K)$), which is also denoted by $V$ (resp., $W$). 
        \item We denote by $\mathrm{Rep}_{\CC}(G)$ (resp., $W \in \mathrm{Rep}_{\CC}(H)$ the category whose objects are $V \otimes_{\QQ} \CC$ (resp., $W \otimes_{\QQ} \CC$) such that $V \in \mathrm{Rep}_{\QQ}(G)$ (resp., $W \in \mathrm{Rep}_{\QQ}(H)$). 
    \end{enumerate}
\end{convention}

For any $V \in \mathrm{Rep}_{\CC}(G)$, let $\xi$ be the restriction of the inverse of the central character of $V$ to $A_{G}$ and let $L$ be an compact open subgroup of $G(\AAA_{f})$. Let $\mathcal{A}^{*}_{c}(S(L), V(2))$ be the de Rham complex of $C^{\infty}$-differential forms with compact support on $S(L)_{\CC}^{an}$ with values in the local system $V(2)$. Similarly, let $\mathcal{A}^{*}(S(L), V(2))$ be the usual de Rham complex and let $\mathcal{A}^{*}_{(2)}(S(L), V(2))$ be the complex of square integrable differential forms. If $\circ$ is the symbol $c$, $(2)$ or the empty symbol, we define 
\begin{equation*}
    \mathcal{A}^{*}_{\circ}(S, V(2)) = \varinjlim_{L} \mathcal{A}^{*}_{\circ}(S(L), V(2)). 
\end{equation*}

Let $\lieg$ be the Lie algebra of $G(\RR)$ and $K_{G} = A_{G}(\U(2) \times \U(1))(\RR)$ be the subgroup of $G(\RR)$ which is maximal modulo the center. For any $(\lieg_{\CC}, K_{G})$-module $M$, let $\mathcal{C}^{*}(\lieg_{\CC}, K_{G}; M)$ be the $(\lieg_{\CC}, K_{G})$-complex of $M$. Then we have $G(\AAA_{f})$-equivariant isomorphisms of complexes: \cite[VII \S 2]{Borel_Wallach}
\begin{align*}
    &  \mathcal{A}^{*}_{c}(S_{\QQ}, V(2)) \cong \mathcal{C}^{*}(\lieg_{\CC}, K_{G}; V(2) \otimes_{\CC} C_{c}^{\infty}(G(\QQ) \backslash G(\AAA), \xi)), \\
    &  \mathcal{A}^{*}(S_{\QQ}, V(2)) \cong \mathcal{C}^{*}(\lieg_{\CC}, K_{G}; V(2) \otimes_{\CC} C^{\infty}(G(\QQ) \backslash G(\AAA), \xi)) ,
\end{align*}
that are compatible with the inclusions $\mathcal{A}^{*}_{c}(S_{\QQ}, V(2)) \subset \mathcal{A}^{*}(S_{\QQ}, V(2))$ and 
\begin{equation*}
  \mathcal{C}^{*}(\lieg_{\CC}, K_{G}; V(2) \otimes_{\CC} C_{c}^{\infty}(G(\QQ) \backslash G(\AAA), \xi)) \subset \mathcal{C}^{*}(\lieg_{\CC}, K_{G}; V(2) \otimes_{\CC} C^{\infty}(G(\QQ) \backslash G(\AAA), \xi)). 
\end{equation*}

If we take cohomology on both sides, we obtain $G(\AAA_{f})$-equivariant isomorphisms 
\begin{align*}
    & \mathrm{H}^{*}_{dR, c}(S, V(2)) \cong \mathrm{H}^{*}(\lieg_{\CC}, K_{G}; V(2) \otimes_{\CC} C_{c}^{\infty}(G(\QQ) \backslash G(\AAA), \xi)), \\
    & \mathrm{H}^{*}_{dR}(S, V(2)) \cong \mathrm{H}^{*}(\lieg_{\CC}, K_{G}; V(2) \otimes_{\CC} C^{\infty}(G(\QQ) \backslash G(\AAA), \xi)), 
\end{align*}
which are compatible with the maps from cohomology with compact support to cohomology without support.  
\par Let $\mathrm{H}^{*}_{(2)}(S, V(2))$ be the cohomology of the complex $\mathcal{A}^{*}_{(2)}(S, V(2))$ which 
is the $L^{2}$ cohomology of $S$ with coefficients in $V(2)$. According to \cite{Borel_regularization83}, we have a $G(\AAA_f)$-equivariant isomorphism 
\begin{equation*}
    \mathrm{H}^{*}_{(2)}(S, V(2)) \cong \mathrm{H}^{*}(\lieg_{\CC}, K_{G}; V(2) \otimes_{\CC} C_{(2)}^{\infty}(G(\QQ) \backslash G(\AAA), \xi)), 
\end{equation*}
where $C_{(2)}^{\infty}(G(\QQ) \backslash G(\AAA), \xi)$ denotes the subspace of functions in $C^{\infty}(G(\QQ) \backslash G(\AAA), \xi)$ which are square integrable modulo the center of $G$.
We also use the notation $\mathrm{H}^{*}_{\text{cusp}}(S, V(2))$ to denote 
\begin{equation*}
     \mathrm{H}^{*}(\lieg_{\CC}, K_{G}; V(2) \otimes_{\CC} C_{\text{cusp}}^{\infty}(G(\QQ) \backslash G(\AAA), \xi)), 
\end{equation*}
where $C_{cusp}^{\infty}(G(\QQ) \backslash G(\AAA), \xi)$ denotes the subspace of functions in $C^{\infty}(G(\QQ) \backslash G(\AAA), \xi)$ which are cuspidal. 

By definition, we have the following maps: 
\begin{equation*}
    C_{\text{cusp}}^{\infty}(G(\QQ) \backslash G(\AAA), \xi) \rightarrow C_{c}^{\infty}(G(\QQ) \backslash G(\AAA), \xi) \rightarrow C_{(2)}^{\infty}(G(\QQ) \backslash G(\AAA), \xi) \rightarrow C^{\infty}(G(\QQ) \backslash G(\AAA), \xi), 
\end{equation*}
where the first map is defined by smooth truncation to a large set which is compact modulo the center, and the second and the third arrows are the natural inclusions. 
These maps induce maps of the corresponding cohomology groups:
\begin{equation*}
    \mathrm{H}^{*}_{\text{cusp}}(S, V(2)) \rightarrow \mathrm{H}^{*}_{dR, c}(S, V(2)) \rightarrow \mathrm{H}^{*}_{(2)}(S, V(2)) \rightarrow \mathrm{H}^{*}_{dR}(S, V(2)).  
\end{equation*}
Let 
\begin{equation*}
    \mathrm{H}^{*}_{dR, !}(S, V(2)) = \mathrm{Im}(\mathrm{H}^{*}_{dR, c}(S, V(2)) \rightarrow \mathrm{H}^{*}_{dR}(S, V(2)). 
\end{equation*}
be the interior de Rham cohomology. 
In general, by \cite[Theorem 7.5]{Borel_ENS74}, the map 
\begin{equation*}
    \mathrm{H}^{*}_{\text{cusp}}(S, V(2)) \rightarrow \mathrm{H}^{*}_{dR, !}(S, V(2))
\end{equation*}
is injective. 

\begin{proposition}
\label{Prop:coh_identity}
    If the algebraic representation $V$ is regular, we have 
    \begin{equation*}
        \mathrm{H}^{*}_{\text{cusp}}(S, V(2)) = \mathrm{H}^{*}_{(2)}(S, V(2)) = \mathrm{H}^{*}_{dR, !}(S, V(2)) = \mathrm{H}^{2}_{dR, !}(S, V(2)). 
    \end{equation*}
\end{proposition}

\begin{proof}
    For the equality besides the last one, by the fact that the map 
    \begin{equation*}
         \mathrm{H}^{*}_{\text{cusp}}(S, V(2)) \rightarrow \mathrm{H}^{*}_{dR, !}(S, V(2))
    \end{equation*}
    is injective, it suffices to prove 
    \begin{equation*}
        \mathrm{H}^{*}_{\text{cusp}}(S, V(2)) = \mathrm{H}^{*}_{(2)}(S, V(2)). 
    \end{equation*}
     By \cite[Theorem 4]{Borel_Bull80} (which applies for $\rank G = \rank K_{G}$), we have 
     \begin{equation*}
         \mathrm{H}^{*}_{dR, !}(S, V(2)) = \bigoplus_{\pi = \pi_{\infty} \otimes \pi_f} m(\pi) \mathrm{H}^{*}(\lieg_{\CC}, K_{G}; V(2) \otimes_{\CC} \pi_{\infty})\otimes \pi_f, 
     \end{equation*}
     where $\pi$ runs over the discrete spectrum of $L^{2}(G(\QQ) \backslash G(\AAA), \xi)$, and $\xi$ is the restriction of the central character of $V^{\vee}$ to $A_{G}$.
     Let $\pi = \pi_f \otimes \pi_{\infty}$ be such an automorphic representation. By \cite[Theorem 5.6]{VZ84}\footnote{Although $\pi_{\infty}$ is not unitary as a representation of $G(\RR)^{+}$, if 
     we restrict to the semisimple part we could make it unitary. Hence, we can use the theorem in \cite{VZ84}, because the authors assume the real Lie group is semisimple.},
     the assumption $\mathrm{H}^{*}(\lieg_{\CC}, K_{G}; V(2) \otimes_{\CC} \pi_{\infty}) \neq 0$ implies $\pi_{\infty} = A_{\lieq}(\lambda)$, which is a cohomological induction from a $\theta$-stable parabolic subalgebra $\lieq$. Since the highest weight of $V$ is regular, the only $\lieq$ that satisfies the condition in \cite[Theorem 5.6]{VZ84} is the Borel. In our case, the torus of the Borel is compact, so $A_{q}(\lambda)$ must be a discrete series. Discrete series are tempered, so by \cite[III, Corollary 5.2]{Borel_Wallach}, $\pi_{\infty}$ only contributes to middle degree, that is 
     \begin{equation*}
         \mathrm{H}^{*}(\lieg_{\CC}, K_{G}; V(2) \otimes_{\CC} \pi_{\infty}) = \mathrm{H}^{2}(\lieg_{\CC}, K_{G}; V(2) \otimes_{\CC} \pi_{\infty}). 
     \end{equation*}
     Finally, for $\pi$ appearing in the discrete spectrum, by Wallach's theorem \cite[Theorem 4.3]{Wallach84} $\pi$ must be cuspidal. 
\end{proof}

\begin{remark}
    The condition that $V$ is regular is very important for Proposition \ref{Prop:coh_identity} to hold. 
\end{remark}

\subsection{Discrete series L-packets}
\label{SS:discrete series}

\begin{convention}
    In this subsection, $V$ is any object in $\mathrm{Rep}_{\CC}(G)$. 
\end{convention}

The $((\lieg_{\CC}, K_G) \times G(\AAA_f)) $-module $C_{\text{cusp}}^{\infty}(G(\QQ) \backslash G(\AAA), \xi)$ can be decomposed into a direct sum 
\begin{equation*}
    C_{\text{cusp}}^{\infty}(G(\QQ \backslash G(\AAA), \xi) = \bigoplus_{\pi} m(\pi) \pi,
\end{equation*}
where the sum runs over the irreducible cuspidal representations $\pi$ of $G$ and $m(\pi)$ is the multiplicity of $\pi$. This induces a decomposition 
\begin{equation}
\label{eq: decomosition of Betti into (g,K)-cohomology}
    \mathrm{H}^{2}_{B, !}(S, V(2)) \cong \mathrm{H}^{2}_{dR, !}(S, V(2)) = \bigoplus_{\pi = \pi_{\infty} \otimes \pi_f} m(\pi) \mathrm{H}^{2}(\lieg_{\CC}, K_{G}; V(2) \otimes_{\CC} \pi_{\infty})\otimes \pi_f. 
\end{equation}

Recall that we have the following theorem about the multiplicity $m(\pi)$: 

\begin{theorem} \textnormal{\cite{Mok15}, \cite[\S 1.1]{CHL11}, \cite[Theorem 2.6.3]{LSZ22}}
\label{Thm: multiplicity 1}
    The multiplicity $m(\pi)$ of the cohomological $\pi$ (i.e., the $\pi$ with $\mathrm{H}^{*}(\lieg_{\CC}, K_{G}; V(2) \otimes_{\CC} \pi_{\infty}) \neq 0$, 
    for some cohomological degree $*$) is $1$. 
\end{theorem}

\begin{definition}
    The \textit{discrete series $L$-packet} $P(V(2))$ associated to $V(2)$ is the set of isomorphism classes of discrete series $\pi_{\infty}$ of $G(\RR)^{+}$ whose Harish-Chandra parameter and central character are opposed to the ones of $V(2)$. 
\end{definition}

\begin{proposition}
    The $\pi$ such that 
    \begin{equation*}
         \mathrm{H}^{2}(\lieg_{\CC}, K_{G}; V(2) \otimes_{\CC} \pi_{\infty}) \neq 0 
    \end{equation*}
    are those $\pi$ such that $\pi_{\infty}|_{G(\RR)^{+}} \in P(V)$ \footnote{The archimedean part $\pi_{\infty}$ of a smooth automorphic representation $\pi$ is a $(\lieg_{\CC}, K_{G})$-module. By the exponential map, $\pi_{\infty}$ is equivalent to an representation of $G(\RR)^{+}$.}. 
\end{proposition}

\begin{proof}
    See the proof of Proposition \ref{Prop:coh_identity}. 
\end{proof}

According to \cite[Chapter II, \S 3, Proposition 3.1]{Borel_Wallach}, for any representation $\pi_{\infty}$ of $G(\RR)^{+}$, 
the $(\lieg_{\CC}, K_{G})$-complex $\pi_{\infty} \otimes_{\CC} V(2)$ has zero differential map. Hence, we have 
\begin{equation*}
    \mathrm{H}^{2}(\lieg_{\CC}, K_{G}; V(2) \otimes_{\CC} \pi_{\infty}) = \Hom_{K_G}(\wedge^{2} \lieg_{\CC} / \liek_{G, \CC}, V(2) \otimes_{\CC} \pi_{\infty}). 
\end{equation*}

We have the decomposition: 
\begin{equation*}
    \wedge^{2} \lieg_{\CC} / \liek_{G, \CC} = \wedge^{2} \liep_{G, \CC}^{+} \oplus  \wedge^{2} \liep_{G, \CC}^{-} \oplus \liep_{G, \CC}^{+} \otimes \liep_{G, \CC}^{-}. 
\end{equation*}

And we have the following identities as representations of $K_G$\footnote{When considering $(\lieg_{\CC}, K_{G})$-cohomology, the action of $A_{G}$ is determined by the action of $K$ because the centeral character is fixed, hence, we ignore the action of $A_{G}$ in the notation of representations of $K_{G}$. } : 
\begin{align*}
     & \wedge^{2} \liep_{G, \CC}^{+} = \tau_{(1, 1, -2)},  \\
     & \wedge^{2} \liep_{G, \CC}^{-} = \tau_{(-1, -1, 2)},  \\
     & \liep_{G, \CC}^{+} \otimes \liep_{G, \CC}^{-} = \tau_{(1, -1, 0)} \oplus \tau_{(0, 0, 0)}. 
\end{align*}

We also have the following proposition:
\begin{proposition}
\label{Prop: (g, K)-cohomology = Hom}
    The dimension of the $\CC$-vector space 
    \begin{equation*}
         \mathrm{H}^{2}(\lieg_{\CC}, K_{G}; V(2) \otimes_{\CC} \pi_{\infty}) = \Hom_{K_G}(\wedge^{2} \lieg_{\CC} / \liek_{G, \CC}, V(2) \otimes_{\CC} \pi_{\infty})
    \end{equation*}
    is less than or equal to $1$.
\end{proposition}
\begin{proof}
    This is a direct consequence of \cite[II, Proposition 3.1. and Theorem 5.3.]{Borel_Wallach}. 
\end{proof}

Finally, we can compute explicitly the elements of $P(V(2))$. 
\begin{proposition}
\label{Prop: DS L-packet}
    If the highest weight of $V(2)$ is $(\lambda_1, \lambda_2, \lambda_3; d)$ such that $\lambda_1 \ge \lambda_2 \ge \lambda_3$, 
    we have 
    \begin{equation*}
        P(V(2)) = \{ \pi_1, \pi_2, \pi_3, \bar{\pi}_1, \bar{\pi}_2, \bar{\pi}_3 \},
    \end{equation*}
    where for $i = 1,2,3$, $\bar{\pi}_{i}$ is the complex conjugate of $\pi_{i}$ and the Harish-Chandra parameters and Blattner parameters\footnote{We ignore the action of similitude factor in the notation of Harish-chandra and Blattner parameters because it is determind by the fixed central character. }of $\pi_1, \pi_2$ and $\pi_3$ are listed in the following table.
     \begin{equation}
     \label{talble: DS L-packt}
            \begin{array}{|c|c|c|}
            \hline  & \text{Harish-Chandra parameters} & \text{Blattner parameters} \\
            \hline \pi_1 & (1 - \lambda_1, -\lambda_2, -1-\lambda_1) & (1 - \lambda_3, 1 - \lambda_2, -2 - \lambda_1) \\
            \hline \pi_2 & (1 - \lambda_3, -1 - \lambda_1, -\lambda_2) & (1 - \lambda_3, -1 - \lambda_1, -\lambda_2) \\
            \hline \pi_3 & (-\lambda_2, -1-\lambda_1, 1-\lambda_3) & (-1 - \lambda_2, -1 - \lambda_1, 2 - \lambda_3) \\
            \hline 
            \end{array}
    \end{equation}
\end{proposition}
\begin{proof}
    For a dominant root system with positive roots 
    \begin{equation*}
        \Delta^{+} = \{(1, -1, 0), (0, 1, -1), (1, 0, -1) \}, 
    \end{equation*}
     we have $\rho = (1, 0, -1)$. Hence, the infinitesimal character of $V(2)$ is $(\lambda_1 + 1, \lambda_2, \lambda_3 - 1)$. 
    By \cite[II, Proposition 3.1]{Borel_Wallach}, the infinitesimal character of the representation $\pi_{\infty}$ of $G(\RR)^{+}$ with 
    \begin{equation*}
        \mathrm{H}^{*}(\lieg_{\CC}, K_{G}; V(2) \otimes_{\CC} \pi_{\infty}) \neq 0
    \end{equation*}
    must be the negative of the infinitesimal character of $V(2)$. Hence, the representations in $P(V(2))$ must have infinitesimal character $(1-\lambda_3, -\lambda_2, -1-\lambda_1)$. Then by \cite[Theorem 9.20]{Knapp86}, we can write down all the discrete series with fixed infinitesimal character. 
\end{proof}

\begin{remark}
    The representations $\pi_1$, $\pi_2$ and $\pi_3$ contribute to $\mathrm{H}^{2}_{dR, !}(\Sh_{G}(L)_{\CC}^{an}, V(2))$ while the representations $\bar{\pi}_1$, $\bar{\pi}_2$ and $\bar{\pi}_3$ contribute to $\mathrm{H}^{2}_{dR, !}(\overline{\Sh_{G}(L)_{\CC}^{an}}, V(2))$
\end{remark}

\subsection{Hodge decomposition}
\label{SS: Hodge decomposition}
In this subsection, we compute the Hodge decomposition of the Betti realization of the pure motive associated to an automorphic representation $\pi$. 

\begin{convention}
    In this subsection, $V$ is any object in $\mathrm{Rep}_{\QQ}(G)$. We use $V_{\CC}$ to denote $V \otimes_{\QQ} \CC$. 
\end{convention}

Recall that we have the following theorem. 
\begin{theorem}[\cite{BHD94},Theorem 3.2.2 and (2.3.3)]
\label{Thm: rational field}
    For any cuspidal automorphic representation $\pi = \pi_{f} \otimes \pi_{\infty}$ such that $\pi_{\infty} \in P(V_{\CC}(2))$, $\pi_f$ is defined over a number field $E(\pi_f)$. 
\end{theorem}

\begin{convention}
    In this subsection, we use $\pi_{f}$ to denote its $E(\pi_f)$-model. 
\end{convention}

\begin{definition}
\label{def: M_B and M_dR}
We define
\begin{equation*}
    M_{dR}(\pi_{f}, V(2)) := \Hom_{\QQ[G(\AAA_f)]}(\Res_{E(\pi_f)/\QQ}\pi_f, \mathrm{H}_{dR, !}^{2}(S, V(2)))
\end{equation*}
and similarly 
\begin{equation*}
    M_{B}(\pi_{f}, V(2)) := \Hom_{\QQ[G(\AAA_f)]}(\Res_{E(\pi_f)/\QQ}\pi_f, \mathrm{H}_{B, !}^{2}(S, V(2))), 
\end{equation*}
where $\mathrm{H}_{dR, !}^{2}(S, V(2))$ and $\mathrm{H}_{B, !}^{2}(S, V(2))$ are the interior de Rham and Betti cohomology with coefficients in $V$. 
These are $\QQ$-vector spaces with a $\QQ$-linear action of $E(\pi_f)$. 
\end{definition}

\begin{remark}
    Since $M_{B}(\pi_{f}, V(2))$ is a $\QQ$-vector space, define its base change to $\RR$ and $\CC$: 
    \begin{align*}
        & M_{B}(\pi_{f}, V(2))_{\RR} := M_{B}(\pi_{f}, V(2)) \otimes_{\QQ} \RR, \\
        & M_{B}(\pi_{f}, V(2))_{\CC} := M_{B}(\pi_{f}, V(2)) \otimes_{\QQ} \CC. 
    \end{align*}
\end{remark}

After base change to $\CC$, we have the comparison isomorphism: 
\begin{equation*}
    \begin{tikzcd}
        I_{\infty}\colon M_{B}(\pi_{f}, V(2))_{\CC} \ar[r, "{\cong}"] &  M_{dR}(\pi_{f}, V(2))_{\CC} 
    \end{tikzcd}
\end{equation*}
We also have 
\begin{align*}
    \rank_{E(\pi_f) \otimes_{\QQ} \CC}M_{B}(\pi_{f}, V(2))_{\CC} = \rank_{E(\pi_f) \otimes_{\QQ} \CC}M_{dR}(\pi_{f}, V(2))_{\CC} = 6. 
\end{align*}

Finally, we state a proposition about the Hodge decomposition of $M_{B}(\pi_{f}, V)_{\CC}$. 

\begin{proposition}
\label{Prop: Hodge decomp for motives}
     If the highest weight of $V(2)$ is $(\lambda_1, \lambda_2, \lambda_3; d)$ such that $\lambda_1 \ge \lambda_2 \ge \lambda_3$, 
     we have 
     \begin{align*}
        & M_{B}({\pi}_f, V(2))_{\CC} \\
        \cong & M_{B}^{2-\lambda_2-\lambda_3 - d, -\lambda_1 - d} \oplus M_{B}^{1-\lambda_1 - \lambda_3 - d, 1 - \lambda_2 - d} \oplus M_{B}^{-\lambda_1 - \lambda_2 - d, 2 - \lambda_3 - d} \\
        \oplus & M_{B}^{-\lambda_1 - d, 2 - \lambda_2 - \lambda_3 - d} \oplus M_{B}^{1 - \lambda_2 -d, 1 - \lambda_1 - \lambda_3 - d} \oplus M_{B}^{2 - \lambda_3 - d, -\lambda_1 - \lambda_2 - d},  
    \end{align*}
    where $M_{B}^{s, t}$ is a $1$-dimensional $\CC$-subspace with Hodge type $(s, t)$. 
    So $M_{B}({\pi}_f, V(2))$ is a pure $\QQ$-Hodge structure with weight $2 - (\lambda_1 + \lambda_2 + \lambda_3 + 2d)$. 
    Moreover, we have 
    \begin{align*}
        & M_{B}^{2-\lambda_2-\lambda_3 - d, -\lambda_1 - d} = \bigoplus_{ \sigma: E(\pi_F) \hookrightarrow \CC } \mathrm{H}^{2}(\lieg_{\CC}, K_{G}; V_{\CC}(2) \otimes_{\CC} \pi_{1}), \\
        & M_{B}^{1-\lambda_1 - \lambda_3 - d, 1 - \lambda_2 - d} = \bigoplus_{\sigma: E(\pi_F) \hookrightarrow \CC } \mathrm{H}^{2}(\lieg_{\CC}, K_{G}; V_{\CC}(2) \otimes_{\CC} \pi_{2}), \\
        & M_{B}^{-\lambda_1 - \lambda_2 - d, 2 - \lambda_3 - d}  = \bigoplus_{\sigma: E(\pi_F) \hookrightarrow  \CC } \mathrm{H}^{2}(\lieg_{\CC}, K_{G}; V_{\CC}(2) \otimes_{\CC} \pi_{3}), 
    \end{align*}
    where 
    \begin{equation*}
        P(V_{\CC}(2)) = \{ \pi_1, \pi_2, \pi_3, \bar{\pi}_1, \bar{\pi}_2, \bar{\pi}_3 \}
    \end{equation*}
    is defined in Proposition \ref{Prop: DS L-packet}. 
\end{proposition}

\begin{proof}
    For $i = 1,2,3$, if the Hodge type of $\pi_{i}$ is $(s, t)$, then the Hodge type of $\bar{\pi}_i$ is $(t, s)$. So we only need to compute the Hodge type of $\pi_1, \pi_2, \pi_3$. 
    Since complexification of $h$ in the definition of the Shimura datum of $\Sh_{G}(L)$ is 
    \begin{equation*}
        h_{\CC}\colon (z, \bar{z}) \in \mathbb{S}(\CC) \mapsto \begin{pmatrix}
                                                    z & 0 & 0 \\
                                                    0 & z & 0 \\
                                                    0 & 0 & \bar{z} 
                                                \end{pmatrix}, 
    \end{equation*}
    the contribution from the local system $V$ only depends on $K_{G}$-types. Then we compute the Hodge type of $\pi_1, \pi_2, \pi_3$ case by case. 
    \begin{enumerate}
        \item For $\pi_1$: The positive roots in \cite[Theorem 9.20]{Knapp86} for $\pi_1$ are 
                           \begin{equation*}
                                \Delta^{+} = \{(1, -1, 0), (0, 1, -1), (1, 0, -1) \}. 
                           \end{equation*}
                           Hence the lowest weight of $V_{\CC}(2)$ is $(\lambda_3, \lambda_2, \lambda_1)$, the minimal $K_{G}$-type of $V_{\CC}(2)$ is $\tau_{(\lambda_2, \lambda_3, \lambda_1)}$. 
                           Since the minimal $K_{G}$-type of $\pi_1$ is $\tau_{(1 - \lambda_3, 1 - \lambda_2, -2 - \lambda_1)}$, and the minimal $K_{G}$-type of 
                           $V_{\CC}(2) \otimes_{\CC} \pi_1$ is $\tau_{(1, 1, -2)}$. Hence, only the $K_{G}$-type $\tau_{(1, 1, -2)}$ contributes to 
                           \begin{equation*}
                               \mathrm{H}^{2}(\lieg_{\CC}, K_{G}; V_{\CC}(2) \otimes_{\CC} \pi_{1}) = \Hom_{K_G}(\wedge^{2} \lieg_{\CC} / \liek_{G, \CC}, V_{\CC}(2) \otimes_{\CC} \pi_{1}). 
                           \end{equation*}
                           Since the contributed $K_{G}$-type of $V_{\CC}(2)$ is $\tau_{(\lambda_2, \lambda_3, \lambda_1)}$ and 
                           \begin{equation*}
                               \tau_{(1,1,-2)} \subset \wedge^{2}\liep_{G, \CC}^{+} \subset \wedge^{2} \lieg_{\CC} / \liek_{G, \CC}, 
                           \end{equation*}
                           the contributed Hodge type of $\pi_1$ is 
                           \begin{equation*}
                               (2 - \lambda_2 - \lambda_3 - d, -\lambda_1 - d).
                           \end{equation*}
        \item For $\pi_2$: The positive roots in \cite[Theorem 9.20]{Knapp86} for $\pi_2$ are 
                           \begin{equation*}
                                \Delta^{+} = \{(1, -1, 0), (0, -1, 1), (1, 0, -1) \}. 
                           \end{equation*}
                           Hence the lowest weight of $V_{\CC}(2)$ is $(\lambda_3, \lambda_1, \lambda_2)$, so the minimal $K_{G}$-type of $V_{\CC}(2)$ is $\tau_{(\lambda_1, \lambda_3, \lambda_2)}$. 
                           Since the minimal $K_{G}$-type of $\pi_2$ is $\tau_{(1 - \lambda_3, -1 - \lambda_1,  - \lambda_2)}$, the minimal $K_{G}$-type of 
                           $V_{\CC}(2) \otimes_{\CC} \pi_2$ is $\tau_{(1, -1, 0)}$. Hence, only the $K_{G}$-type $\tau_{(1, -1, 0)}$ contributes to 
                           \begin{equation*}
                               \mathrm{H}^{2}(\lieg_{\CC}, K_{G}; V_{\CC}(2) \otimes_{\CC} \pi_{2}) = \Hom_{K_G}(\wedge^{2} \lieg_{\CC} / \liek_{G, \CC}, V_{\CC}(2) \otimes_{\CC} \pi_{2}). 
                           \end{equation*}
                           Since the contributed $K_{G}$-type of $V_{\CC}(2)$ is $\tau_{(\lambda_1, \lambda_3, \lambda_2)}$ and 
                           \begin{equation*}
                               \tau_{(1,-1,0)} \subset \liep_{G, \CC}^{+} \otimes \liep_{G, \CC}^{-} \subset \wedge^{2} \lieg_{\CC} / \liek_{G, \CC}, 
                           \end{equation*}
                           the contributed Hodge type of $\pi_1$ is 
                           \begin{equation*}
                               (1 - \lambda_1 - \lambda_3 - d, 1 -\lambda_2 - d).
                           \end{equation*}
        \item For $\pi_3$: The positive roots in \cite[Theorem 9.20]{Knapp86} for $\pi_3$ are 
                           \begin{equation*}
                                \Delta^{+} = \{(-1,  0, 1), (0, -1, 1), (1, -1, 0) \}. 
                           \end{equation*}
                           Hence the lowest weight of $V_{\CC}(2)$ is $(\lambda_2, \lambda_1, \lambda_3)$, so the minimal $K_{G}$-type of $V_{\CC}(2)$ is $\tau_{(\lambda_1, \lambda_2, \lambda_3)}$. 
                           Since the minimal $K_{G}$-type of $\pi_3$ is $\tau_{(-1 - \lambda_2, -1 - \lambda_1,  2- \lambda_3)}$, the minimal $K_{G}$-type of 
                           $V_{\CC}(2) \otimes \pi_3$ is $\tau_{(-1, -1, 0)}$. Hence, only the $K_{G}$-type $\tau_{(-1, -1, 0)}$ contributes to 
                           \begin{equation*}
                               \mathrm{H}^{2}(\lieg_{\CC}, K_{G}; V_{\CC}(2) \otimes_{\CC} \pi_{3}) = \Hom_{K_G}(\wedge^{2} \lieg_{\CC} / \liek_{G, \CC}, V_{\CC}(2) \otimes_{\CC} \pi_{3}). 
                           \end{equation*}
                           Since the contributed $K_{G}$-type of $V_{\CC}(2)$ is $\tau_{(\lambda_1, \lambda_2, \lambda_3)}$ and 
                           \begin{equation*}
                               \tau_{(-1,-1,0)} \subset \wedge^{2} \liep_{G, \CC}^{-} \subset \wedge^{2} \lieg_{\CC} / \liek_{G, \CC}, 
                           \end{equation*}
                           the contributed Hodge type of $\pi_3$ is 
                           \begin{equation*}
                               ( -\lambda_1 - \lambda_2 - d, 2 -\lambda_3 - d).
                           \end{equation*}
    \end{enumerate}
\end{proof}

\begin{corollary}
\label{Corollary: Hodge decomp for motives}
If $V = V^{a, b} \{r, s\}$, we have 
\begin{align*}
        & M_{B}({\pi}_f, V(2))_{\CC} \\
        \cong & M_{B}^{-r, -2 - a - b - s} \oplus M_{B}^{-1 - a - r, -1 - b - s} \oplus M_{B}^{-2 - a - b - r, -s} \\
        \oplus & M_{B}^{ -2 - a - b - s, -r} \oplus M_{B}^{-1 - b - s, -1 - a - r} \oplus M_{B}^{-s, -2 -a -b - r}.  
\end{align*}
\end{corollary}
