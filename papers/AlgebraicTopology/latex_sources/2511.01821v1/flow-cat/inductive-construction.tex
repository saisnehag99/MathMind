\subsection{A symmetric flow category with bounded action}\label{subsec:unstructured-flow-cat}
We can now construct a flow category using the Kuranishi charts of the previous section. Throughout, $(Y,\lambda)$ is a closed contact manifold equipped with a non-degenerate contact form $\lambda$. We denote its Reeb vector field by $R$ and let $(\wh Y,\omega) = (\bR\times Y,d(e^s\lambda))$ be the symplectisation of $(Y,\lambda)$. 
Given $L > 0$, let 
$$\cP_L\coloneqq \{\Gamma = (\gamma_1,\dots,\gamma_k)\mid \gamma_i \text{ is a Reeb orbit, with action }\cA_\lambda(\gamma_i)\leq L\}$$ 
be the set of finite sequences of unparametrized Reeb orbits of action at most $L$.  Recall that given by a function $\Lambda\cl \Gamma^-\to \Gamma^+$ and a sequence $\beta =(\beta_\gamma)_{\gamma\in \Gamma^+}$ of relative homology classes, we defined the moduli space $\Mbar_{\sft}^{\, J}(\Gamma^+,\Gamma^-;\beta)_\Lambda$ of buildings with disconnected domains in Definition~\ref{de:disconnected-domains}.

\begin{theorem}\label{thm:flow-cat}
    Given $L > 0$ and a choice of $\lambda$-adapted almost complex structure $J$, there exists a symmetric flow category $\scM^{Y,\lambda}_{\leq L}$ of class rel--$C^1$ whose objects are elements of $\cP_L$ and whose morphism spaces are 
    \begin{equation*}\label{eq:morphisms}
        \scM_{\leq L}^{Y,\lambda}(\Gamma^-,\Gamma^+)\,\coloneqq \,\djun{\Lambda}\djun{\beta}\Mbar_{\sft}^{\, J}(\Gamma^+,\Gamma^-;\beta)_\Lambda
    \end{equation*}
    for any $\Gamma^+,\Gamma^-\in\cP_L(Y)$, where the disjoint unions range over functions $\Lambda\cl \Gamma^-\to \Gamma^+$ and sequences $\beta =(\beta_\gamma)_{\gamma\in \Gamma^+}$ of relative homology classes.
\end{theorem}

Observe that the order of $\Gamma^-$ and $\Gamma^+$ is opposite the usual one in the morphism space of the symmetric flow category. We do so to be compatible with the conventions of \cite{AB24}. We use the energy functions
$$E = E_{\Gamma^-\Gamma^+}\cl \scM^{Y,\lambda}_{\leq L}(\Gamma^- , \Gamma^+) \to \subset [0,\infty) : \scA_\lambda(\Gamma^+)- \scA_\lambda(\Gamma^-),$$ 
which are clearly additive under composition and proper by \cite{BEH03}. The composition is given by the canonical ``stacking'' of buildings, as shown in 
\begin{figure}[H]
    \centering
    \incfig{0.7}{flo-comp}
    \caption{Composition in $\scM_{\le L}^{Y,\lambda}$ } 
    \label{fig:flocomp}
\end{figure}

\begin{remark}
The bound by $L$ selects Reeb orbits with action less than $L$ but does not constrain the length (and action) of $\Gamma$. Any Reeb orbit with action greater than $L$ will not occur in the moduli space $\Mbar^{\, J}_{\sft}(\Gamma^+,\Gamma^-,\beta)_\Lambda$ if $\cA_\lambda(\gamma_i^\pm) \leq L$ for all $\gamma_i^+$. This is due to the fact that each connected component of the holomorphic buildings we consider has exactly one positive puncture. 
\end{remark}
 
In order to prove Theorem~\ref{thm:flow-cat}, we construct the global Kuranishi charts for the moduli spaces~\eqref{eq:morphisms} inductively as in \cite{BX22}. Fix a pre-perturbation datum $\fD$ consisting of
\begin{itemize}
    \item an action bound $L > 0$;
    \item a $\cP_L$-integral approximation $\wt\lambda$ of $\lambda$,
    \item a translation-invariant $J$-linear connection $\wh\conn$ on $T\wh Y$,
    \item a prime number $p \gg \max \{\cA_{\wt\lambda}(\gamma) \mid \gamma \in \cP_L \}$. 
\end{itemize}
Recall that $\Mbar_{\sft}^{\, J}(\Gamma^+,\Gamma^-;\beta)_\Lambda$ is the moduli space of disconnected leveled buildings, where each connected component has one positive puncture asymptotic to some $\gamma\in \Gamma^+$ and negative punctures asymptotic to the elements $\gamma'\in \Lambda_\gamma \coloneqq\Lambda\inv(\gamma)$ and of degree $\beta_\gamma$.
 
\noindent We will from now on drop the mention of degree to make the notation more tractable. Define a partial order $\prec$ on $\cP_L$ by  
$$\Gamma \prec \Gamma' \qquad \dimp \qquad\exists\Lambda: \Mbar^{\, J}_{\sft}(\Gamma',\Gamma)_\Lambda \neq \emst$$
and define the ``norm"
$$\norm{(\Gamma,\Gamma')} := \sup\set{k\in \bN_0\mid \exists \Gamma_0,\dots,\Gamma_k \in \cP_L: \Gamma = \Gamma_0 \prec \Gamma_1 \prec\dots\prec \Gamma_k \prec \Gamma'}$$
 on $\cP_L\times \cP_L$. Note that $\Gamma \nprec \Gamma'$ if and only if $\norm{(\Gamma,\Gamma')} = -\infty$.
\noindent We use induction on $\norm{(\Gamma^-,\Gamma^+)}$ to construct perturbation data for the moduli spaces $\Mbar_{\sft}^{\, J}(\Gamma^+,\Gamma^-)_\Lambda$. 
 If $\norm{(\Gamma^-,\Gamma^+)} = -\infty$, there is nothing to do. Given a pair $\Gamma^\pm$ of norm $0$ and a partition $\Lambda\cl \Gamma^-\to \Gamma^+$, extend $(\wt\lambda,\wh\conn,p)$ to an arbitrary perturbation datum $\alpha_{\Lambda}$ for $\Mbar^{\, J}(\Gamma^+,\Gamma^-)_\Lambda$. Let 
 $$\cK_\Lambda =  (\wh G_{\Lambda},\cT_{\Lambda},\cE_{\Lambda},\fs_{\Lambda})$$ 
 be the associated global Kuranishi chart given by Proposition~\ref{prop:disconnected-buildings}. Given permutations $\sigma^\pm$, we define $\cK_{\sigma^+\g \Lambda\g \sigma^-}$ to be the global Kuranishi chart obtained from $\cK_{\Lambda}$ by permuting the labels of the positive/negative marked punctures according to $\sigma^+$ and $\sigma^-$ respectively. Note that we also have to change the partition $\Lambda$.\par 
 Returning to $\cK_{\Lambda}$, recall that its thickening (and obstruction bundle) are rel--$C^1$ over the base space $\cBS_\Lambda$, constructed in \textsection\ref{subsec:base-for-disconnected}.
It admits canonical smooth maps
\begin{equation}\label{eq:base-total-degree}
    \cBS_\Lambda\to \p{\gamma\in \Gamma^+}{\cBR_{\gamma,\Lambda_{\gamma}}} \to \p{\gamma\in \Gamma^+}{\cB(d_{\Lambda_\gamma})}
\end{equation}
where $\cB(d) \sub \Mbar_0(\bP^d,d)$ was defined in \textsection\ref{subsec:base-space} and the degree is given by
\[d_{\Lambda_\gamma} = 1+\#\Lambda_\gamma-2 +  p\,\lbr{\cA_{\wt\lambda}(\gamma)-\s{\gamma'\in \Lambda_\gamma}{\cA_{\wt\lambda}(\gamma')}}.\]

In particular, the evaluations maps on the products of $\cB(d)$ induce smooth $\PU(d_{\Lambda_\gamma}+1)$-equivariant evaluation maps 
\begin{equation}\label{eq:evaluation-positive}
    \eva^+_\gamma\cl\cBR_{\gamma,\Lambda_{\gamma}}\to \bP^{d_{\Lambda_{\gamma}}}
\end{equation}
for $\gamma\in \Gamma^+$ and 
\begin{equation}\label{eq:evaluation-negative}
    \eva^-_{\gamma'}\cl \cBR_{\gamma,\Lambda_{\gamma}}\to \bP^{d_{\Lambda_{\gamma}}}
\end{equation}
for $\gamma'\in \Lambda_\gamma$.


\subsubsection{Embeddings of base spaces}\label{subsec:embeddings-base} We will lift the evaluation maps $\eva^\pm_{\gamma'}$ to smooth maps to the spheres $S^{2d_{\Lambda_\gamma}+1}$ as in \cite[Definition~5.2.1]{BX22} but via a different construction. Then, we will use these lifts to construct embeddings of base spaces that will induce the composition maps of the symmetric flow category later on. 
Fix thus $\gamma\in \Gamma^+$ and set $d \coloneqq d_{\Lambda_\gamma}$. Let $J_0$ be the standard complex structure on 
\begin{equation}\label{eq:symplectisation-sphere}
    \bR \times S^{2d+1}\,\cong\, \cO_{\bP}(-1)\sm 0 \,\cong\,\bC^{d+1}\sm\{0\},
\end{equation}
considered as the symplectization of the contact manifold $S^{2d+1}$ equipped with the round (Morse-Bott) contact form. We denote the moduli space of Pardon buildings in $\bR \times S^{2d+1}$ by $\Mbar^{J_0}_{\gamma\sqcup\Lambda_\gamma}(S^{2d+1})$ and we equip it with the Gromov topology as described in \cite{Par19}. Note that we do not fix the Reeb orbits the punctures have to be asymptotic to, nor any base point of the Reeb orbits. The quotient map $S^{2d+1}\to \bP^d$ induces a continuous map 
\begin{equation}\label{eq:map-to-stable-maps}
    \Mbar^{J_0}_{\gamma\sqcup\Lambda_\gamma}(S^{2d+1})\to \Mbar_{0,\gamma\sqcup\Lambda_\gamma}(\bP^d,d)
\end{equation}
and we denote by $\cBRB_{\gamma, \Lambda_\gamma}$ the preimage of $\cB_{\gamma\sqcup\Lambda_\gamma}(d)$ under \eqref{eq:map-to-stable-maps}.
By construction, the map~\eqref{eq:map-to-stable-maps} lifts to a continuous map $q\cl \cBRB_{\gamma, \Lambda_\gamma} \to \cBR_{\gamma,\Lambda_\gamma}$.


\begin{lemma}\label{lem:principal-bundle} The map
 $q\cl \cBRB_{\gamma, \Lambda_\gamma} \to \cBR_{\gamma, \Lambda_\gamma}$ is a principal $S^1$-bundle.  
 \end{lemma}

\begin{proof} The $S^1$-action on $S^{2d+1}$ induces a continuous $S^1$-action on $\cBRB_{\gamma, \Lambda_\gamma}$, with respect to which the evaluation $\eva^+_\gamma$ at the positive puncture is equivariant. Since $S^1$ acts freely on $S^{2d+1}$, the action on $\cBRB_{\gamma, \Lambda_\gamma}$ is free as well. The map $q$ is clearly $S^1$-invariant. Since $\cBRB_{\gamma, \Lambda_\gamma} \to \cBR_{\gamma, \Lambda_\gamma}$ is the pullback of the proper map~\eqref{eq:map-to-stable-maps}, it is proper, and thus so is $q$.\par
It remains to show that the descent of $q$ to the quotient is bijective. 
To see surjectivity, fix $[\varphi,C,b,m] \in \cBR_d$, where $b$ denotes the asymptotic markers and $m$ denotes the matching isomorphisms, and let $T$ be the underlying decorated graph. Recall that to every vertex $v$ with one incoming edge and $k$ outgoing edges we can associate numbers $d^+_v, d^-_{v,1}, \dots d^-_{v,k}$ such that 
\[\deg (\varphi_v)  = d^+_{v} - d^-_{v,1} - \dots d^-_{v,k} + (k+1)-2 = (d^+_{v}-1) - (d^-_{v,1}-1) - \dots (d^-_{v,k}-1).  \] 
Thus, there is a $\bC^*$-family of sections of $\varphi_v^* \cO(-1)$ with pole of order $d^+_v-1$ at $z_{v,e}\in \C_v$ for $e\in E^+_v$ and zeroes of order $d^-_{v,i}-1$ at $z_{v,e_i} \in \C_v$ for $e_i\in E^-_v$ with $i\in \{ 1, \dots ,k\}$. Note that the degrees $d_{v,i}^-$ and $d_v^+$ are always at least $3$ by our choice of integer $p$.
Using the identification \eqref{eq:symplectisation-sphere}, we can construct the lifts of $\varphi_v$ by choosing sections of $\varphi^*_v \cO(-1)$ satisfying the pole-zero arithmetic described above and then projecting to $S^{2d+1}$. Note that the choice of the meromorphic section over $\varphi_v^*\cO(-1)$ is only well defined up to the $\bC^*$-action given by scaling. Thus, writing $D_v$ for the divisor of punctures of $C_v$, a lift $\Phi_v  : B_{D_v} C_v  \to S^{2d+1}$ is only well defined up to the Hopf action. Fix a vertex $v \in V(T)$ and a lift $\Phi_v$. By using the matching condition at the punctures, we can find unique lifts $\Phi_{v'}$ for vertices $v'\neq v$ such that $\Phi \cl \cup_{v\in V(T)} C_v\sm D_v \to S^{2d+1}$ is a Pardon building. From the construction it is clear that there is an $S^1$ family of such lifts of $[\varphi,C,b,m]$. The injectivity of the descent of $q$ follows directly.
\end{proof}
 

\begin{lemma}\label{lem:sphere_pullback}
    For $p \in \{\gamma\}\sqcup \Lambda_\gamma$, let $\eva_{p}$ be the respective evaluation map of \eqref{eq:evaluation-positive} or \eqref{eq:evaluation-negative}. Then the following holds.
    \begin{enumerate}[label=\normalfont\arabic*),leftmargin=20pt,ref=\arabic*]
        \item $\cBRB_{\gamma,\Lambda_\gamma}$ is isomorphic as a topological principal $S^1$-bundle to $\eva_{\gamma'}^*S^{2d+1}$,
        \item If we equip $\cBRB_{\gamma,\Lambda_\gamma}$ with the smooth structure pulled back from $\eva_\gamma^*S^{2d+1}$, then there exists a $U(d+1)$-equivariant smooth submersion $\wt\eva_{\gamma'} \cl \cBRB_{\gamma,\Lambda_\gamma}\to S^{2d+1}$ so that 
        \begin{equation*}\begin{tikzcd}
		\cBRB_{\gamma,\Lambda_\gamma} \arrow[r,"\wt\eva_{\gamma'}"] \arrow[d,""]&S^{2d+1} \arrow[d,""]\\ 
        \cB_{\gamma,\Lambda_\gamma} \arrow[r,"\eva_{\gamma'}"] & \bP^d \end{tikzcd} \end{equation*}
        commutes for each $p \in \Lambda_\gamma$.
        \item The restriction of $\cBRB_{\gamma,\Lambda_\gamma}$ to \begin{equation}\label{eq:slice-of-base}
            \scBR_{\gamma,\Lambda_\gamma} := \eva_\gamma\inv(\{[1:0:\dots:0]\})
        \end{equation} is a trivializable principal bundle with a compatible $\normalfont\text{U}(d)$-action, where $\normalfont\text{U}(d) \hkra \PU(d+1)$ is the canonical embedding.
    \end{enumerate}
\end{lemma}

\begin{proof} The first assertion is a consequence of Lemma~\ref{lem:principal-bundle} and the fact that the induced map $\cBRB_{\gamma,\Lambda_\gamma}\to \eva_{\gamma'}^*S^{2d+1}$ is equivariant, hence a morphism of principal bundles. Taking $p = \gamma$, this allows us to pull back the smooth structure on $\eva_\gamma^*S^{2d+1}$ to $\cBRB_{\gamma,\Lambda_\gamma}$. Note that $\wt\eva_\gamma$ is smooth with respect to this smooth structure. Now, for $p \in \{\gamma\}\sqcup \Lambda_\gamma$, we obtain by \cite[Proposition~I.13]{mw09} a diffeomorphism $\psi_{\gamma'}\cl \cBRB_{\gamma,\Lambda_\gamma}\to \eva_{\gamma'}^*S^{2d+1}$, which is the canonical map if $p = \gamma$. By using an equivariant version of the Whitney approximation theorem, we can ensure that $\psi_{\gamma'}$ is $U(d+1)$-equivariant. Thus, we can define $\wt\eva_{\gamma'}$ to be the composition
\begin{equation}\label{eq:lift-of-evaluation}
    \cBRB_{\gamma,\Lambda_\gamma}\,\xra{\psi_{\gamma'}}\, \eva_{\gamma'}^*S^{2d+1}\,\to\, S^{2d+1}.
\end{equation}
Since the second map in~\eqref{eq:lift-of-evaluation} is a smooth submersion, the whole composition is a smooth submersion. It lifts $\eva_{\gamma'}$ by construction.
Now, let 
\[\wt{\eva}_\gamma\cl \cBRB_{\gamma,\Lambda_\gamma}\to S^{2d+1}\] 
be the evaluation map at the positive puncture. Then the restriction of $q$ to 
\[\wt{\eva}_\gamma\inv(\{(1,0,\dots,0)\}) \,\to \scBR_{\gamma,\Lambda_\gamma} \]
  is an isomorphism, whence the claim follows.  
\end{proof}

\begin{remark}
    Note that we do not use regularity of $\cBRB_{\gamma,\Lambda_\gamma}$ to obtain a smooth structure and that the lifted evaluation maps $\wt\eva_{\gamma'}$ might not agree with the natural evaluation maps on this moduli space of buildings. While we believe this to be true, we do not need and hence do not show it.
\end{remark}
 
\begin{corollary}[Spherical evaluation map]\label{cor:normalized-evaluation} 
There exists a lift of the evaluation map $\eva^-\cl \scBR_{d_{\Lambda_\gamma}}\to (\bP^{d_{\Lambda_\gamma}})^{\Lambda_\gamma}$ to a smooth $U(d_{\Lambda_\gamma})$-equivariant map 
\begin{equation}
    \wt\eva^-\cl \scBR_{\gamma,\Lambda_\gamma}\to (S^{2d_{\Lambda_\gamma}+1})^{\Lambda_\gamma}  .
\end{equation}
\end{corollary}

\begin{proof}
    We may assume without loss of generality that $\Lambda\inv(\gamma)\neq \emst$. Take $\wt\eva^-$ to be the product of the compositions
    \begin{equation*}
        \scBR_{\gamma,\Lambda_\gamma}\,\xra{\sim}\, \wt\eva_{\gamma'}\inv(\{(1,0,\dots,0)\}) \,\xra{\wt\eva_{\gamma'}}\, S^{2d_{\Lambda_\gamma}+1}
    \end{equation*}
    over all $p \in \Lambda_\gamma$. Since $\wt\eva_{\gamma'}$ is $U(d+1)$-equivariant, its restriction to $\wt\eva_{\gamma'}\inv(\{(1,0,\dots,0)\})$ is $U(d)$-equivariant. As the first isomorphism is the inverse of a $U(d)$-equivariant map, the resulting map is equivariant as desired.
\end{proof}

Set $\scBR_\Lambda\coloneqq\p{\gamma\in \Gamma^+}{\scBR_{\gamma,\Lambda_\gamma}}$
and let $\scBS_\Lambda = \cBS_\Lambda\times_{\cBR_\Lambda} \scBR_\Lambda$ be the pullback, where $\cBS_\Lambda$ and the map to $\cBR_\Lambda$ were constructed in \textsection\ref{subsec:base-for-disconnected}. Equivalently, $\scBS_{\Lambda}$ is a fiber of the (submersive) evaluation map $\cBS_{\Lambda}\to \p{\gamma\in \Gamma^+}{\bP^{d_{\Lambda_\gamma}}}.$ Abbreviate
\begin{equation}
    \cG_\Lambda \coloneqq \p{\gamma\in \Gamma^+}{\PGL_{d_{\Lambda_\gamma}+1}(\bC)_{[1:0:\dots:0]}}\qquad \qquad G_{\Lambda} \coloneqq \p{\gamma\in \Gamma^+}{\text{U}(d_{\Lambda_\gamma})}
\end{equation}
which acts smoothly on $\scBR_\Lambda$ and $\scBS_\Lambda$ via the embedding 
    $$A\in U(d) \;\mapsto\; \begin{pmatrix} 1 & 0 \\ 0 & A\end{pmatrix} \in \PU(d+1)$$
onto the stabilizer of $[1:0:\dots:0]\in \bP^d$. Since $\PU(d+1)$ acts transitively on $\bP^d$ for any $d\geq 1$, the inclusion induces an isomorphism 
\begin{equation}\label{eq:comparing-group-quotients}
    \cG_\Lambda/G_\Lambda \,\cong\, \p{\gamma\in \Gamma^+}{\PGL_{d_{\Lambda_\gamma}+1}(\bC)/\PU(d_{\Lambda}+1)}.
\end{equation}



\begin{proposition}\label{prop:embedding-base-spaces}
    Given a factorization $\Gamma^- \xra{\Lambda^0}\Gamma_0 \to \dots \to\Gamma_{k-1}\xra{\Lambda^k} \Gamma^+$, there exists a smooth embedding
\begin{equation}\label{eq:embedding-base-spaces}
    \Psi\cl \scBS_{\Lambda^0}\qtimes{\;\bT_{\Gamma_0}}\dots\qtimes{\;\bT_{\Gamma_{k-1}}}\scBS_{\Lambda^k}\,\hkra\, \scBS_{\Lambda},
\end{equation}
where $\scBS_{\Lambda^{j-1}}\qtimes{\;\;\bT_{\Gamma^{j-1}}}\scBS_{\Lambda^{j}}$ is the quotient of the product by the diagonal action of $\bT_{\Gamma^{j-1}}$. The embedding maps to a boundary stratum of codimension $k$ and is equivariant with respect to the block matrix inclusion
\begin{equation}\label{eq:inclusion-of-groups}
    G_{\Lambda^k}\times \dots \times G_{\Lambda^0}\hkra G_{\Lambda}.
\end{equation}
\end{proposition} 

% \begin{remark}
%     The proposition above is a generalization of \cite[Lemma~5.20-5.21 ]{BX22}. We obtain the same embedding as \cite{BX22} if we specialize to the case of $|\Gamma_i|=1$ for all $i$.
% \end{remark}

\begin{proof}
We construct this map first at the level of the base spaces $\cBR$ and then lift it to~\eqref{eq:embedding-base-spaces}. We only discuss the case of $k = 1$, the more general case follows analogously. 
Suppose thus $\Lambda^1 \cl\Gamma\to \Gamma^+$ and $\Lambda^0 \cl\Gamma^-\to \Gamma$ are two partitions with $\Lambda = \Lambda^1\g\Lambda^0$. We will consider the case where $\Gamma^+ =\{\gamma^+\}$; the case of more components in the top level is a straightforward generalization. Assume without loss of generality that $\Gamma\neq \emst$.\par 
For any $d_0,\dots,d_n \geq 1$ we have a smooth map 
\begin{gather}
    \notag F_i\cl  S^{2d+1} \times \bP^{d_i}\,\to\, \bP^{d_0 +\dots + d_n} \\
    (a,[z]) \mapsto [a_0z_0:\dots: a_{d_0}z_0:0:\dots:0:z_1:\dots:z_{d_i}:0:\dots:0],
\end{gather}
considering $S^{2d+1}$ as a subset of $\bC^{d+1}$ and inserting $z_1,\dots,z_{d_i}$ in the positions $(d_0+\dots+d_{i-1}+1)$ to $(d_0+\dots+d_i+1)$. Now, given $\varphi_{\gamma^+} = [\check{\varphi}_{\gamma^+},C_{\gamma^+},m_{\gamma^+}]\in \scBR_{{\gamma^+},\Gamma}$ and $\varphi_{\gamma} = [\check{\varphi}_{\gamma},C_{\gamma},m_{\gamma}]\in \scBR_{\gamma,\Lambda^0_{\gamma}}$ for $\gamma\in \Gamma$, we define 
$$\wt{\varphi}\cl C_{\gamma^+}\sqcup \djun{\gamma}C_{\gamma}\,\to\, \bP^{d_{\Lambda_{\gamma^+}}}$$
by 

\begin{equation} \wt\varphi|_{C_{\gamma}} = 
    \begin{cases}
        j\g \varphi_{\gamma^+} \quad & \gamma = \gamma^+\\
        F_{\gamma^+}(\wt\eva_{\gamma}(\varphi_{\gamma^+},-)\g \varphi_{\gamma}\quad & \gamma\in \Gamma,
    \end{cases}
\end{equation}
where $j\cl \bP^{d_{\Gamma}}\hkra \bP^{d_{\Lambda_{\gamma^+}}} :[z]\mapsto [z:0\dots:0]$ is the inclusion into the first homogeneous coordinates.
By Corollary~\ref{cor:normalized-evaluation} and the definition of $\scB$, this descends to a holomorphic map $\check\varphi\cl C\to \bP^{d_{\Lambda_{\gamma^+}}}$ on the curve $C$ obtained from clutching $C_0$ and the $C_{\gamma}$ at the respective marked point. Using the lift of the clutching map to the real-oriented blow-up, we obtain the map \eqref{eq:embeddings-pardon-base-for-stable-complex}. 
Since 
\[j^*\cO_{\bP^{d_{\Lambda_{\gamma^+}}}}(1) = \cO_{\bP^{d_{\Gamma}}}(1)\qquad\quad \qquad F_{\gamma}(a_{\gamma},\cdot)^*\cO_{\bP^{d_{\Lambda_{\gamma^+}}}}(1) =\cO_{\bP^{d_{\Lambda^0_{\gamma}}}}(1),\] 
the map $\Psi_{\Lambda^1,\Lambda^0}$ is a well-defined map 
\begin{equation}\label{eq:embedding-pardon-base}
    \scBR_{\Lambda^0}\qtimes{\,\bT_\Gamma}\scBR_{\Lambda^1}\to\scBR_\Lambda
\end{equation}It is smooth and equivariant with respect to the inclusion~\eqref{eq:inclusion-of-groups} by construction. To lift it, observe that the universal family $\cC\to \cB(\Lambda)$ is canonically embedded in the product $\cB(\Lambda)\times \p{\gamma\in \Gamma}{\bP^{d_{\Lambda_\gamma}}}$. Pulling back the Fubini--Study metric on projective space, we obtain for any asymptotic marker a canonical lift to the normal bundle of the respective divisor. Using the explicit lift description in the proof of Lemma~\ref{lem:disconnected-base-well-defined} and in the discussion below Theorem~\ref{thm:corner-blowup}, we can lift the map~\eqref{eq:embedding-pardon-base} to~\eqref{eq:embedding-base-spaces} by incorporating the lengths into the matching isomorphisms. Concretely, recall that the matching isomorphisms at the newly created level jump are given by the sequence $(m_\gamma)_{\gamma\in \Gamma} = ([b^0_{\gamma}\otimes b^1_{\gamma}])_{\gamma}$. The $b^i_\gamma\in (T_{z_\gamma}C^i\sm0)/\bR_{>0}$ are the respective asymptotic markers, and $[b^0_{\gamma}\otimes b^1_{\gamma}]$ indicates their image in the quotient by the $S^1$-action. Writing $\wt b^i_\gamma$ for the lift to $T_{z_\gamma}C^i$ and given $a \in \bR^{\Gamma}_{>0}/\bR_{>0}$, we define the `refined matching isomorphism' $\wt m$ to be the image of $(\wt b^0_\gamma\,\otimes\, a_\gamma\wt b^1_\gamma)_{\gamma\in \Gamma}$ in the quotient
$$\bP_{>0}\lbr{\bigoplus\limits_{\gamma\in \Gamma} \sigma_\gamma^*T_{\scC\inn_{\Lambda^0}/\scB_{\Lambda^0}}\otimes \sigma_\gamma^*T_{\scC\inn_{\Lambda^1}/\scB_{\Lambda^1}}}/\bT_\Gamma,$$
where $\bP_{>0}$ of a vector bundle was defined in Equation~\eqref{eq:positive-part}. This completes the proof.
\end{proof}

\begin{lemma}\label{lem:associativity-of-embeddings} 
    For any factorization $\Gamma^-\xra{\Lambda^0}\Gamma_0\xra{\Lambda^1}\Gamma_1\xra{\Lambda^2}\Gamma^+$ of $\Lambda$, the square
    \begin{equation}\begin{tikzcd}
    \scBS_{\Lambda^0}\qtimes{\bT_{\Gamma_0}} \scBS_{\Lambda^1} \qtimes{\bT_{\Gamma_1}} \scBS_{\Lambda^2} \arrow[d,"\Psi\times \ide"] \arrow[rr,"\ide\times \Psi"]&& \scBS_{\Lambda^0}\qtimes{\bT_{\Gamma_0}}\scBS_{\Lambda^{21}}\arrow[d,"\Psi"]\\ 
    \scBS_{\Lambda^{10}}\qtimes{\bT_{\Gamma_{10}}}\scBS_{\Lambda^2} \arrow[rr,"\Psi"] && \scBS_{\Lambda} 
    \end{tikzcd} \end{equation}
    commutes and is a pullback square.
\end{lemma}

\begin{proof}
    The argument is analogous to the proof of \cite[Proposition~5.22]{BX22}.
\end{proof}

Given a sequence $\Lambda^* = \{ \Lambda_i \}_{i=0}^k$ of partitions composing to $\Lambda \cl\Gamma^-\to \Gamma^+$, define 

\begin{equation}
    \cc{\scB}^\bR_{\Lambda^*}\,\coloneqq\, G_\Lambda\times_{G_{\Lambda^*}} \im(\Psi_{\Lambda^*}),
\end{equation}
where $G_{\Lambda^*} = \prod_{j = 0}^k G_{\Lambda^j}$. For the next result, we have to recall some definitions from \cite[\S 5.2.5]{BX22}. For all $d\geq 0$, set ${\bm Q}_d \coloneqq \tilde {\bm Q}_d/ \bR_{> 0}$, where
\begin{equation*}
     \tilde {\bm Q}_d:= \set{ \tilde h \in \bC^{(d+1)\times (d+1)}\ |\ \tilde h^\ast   = \tilde h,\ \tilde h_{00} \neq 0 }.
\end{equation*}
We use the convention that the indices of the Hermitian matrix $\tilde h \in \tilde {\bm Q}_d$ range from $0$ to $d$. The multiplicative group ${\bR}_{> 0}$ acts on $\tilde {\bm Q}_d$ by scalar multiplication on each entry. The $\bR_{> 0}$-orbit of $\tilde h \in \tilde {\bm Q}_d$ is denoted by $[\tilde h]$. We identify ${\bm Q}_d$ with
\begin{equation*}
{\bm Q}_d^*:= \Big\{ h \in \bC^{(d+1)\times (d+1)}\ |\ h^\ast = h,\ h_{00} = 0 \Big\} 
\end{equation*}

in the way that a Hermitian matrix $h$ with $h_{00} = 0$ is identified with the $\bR_{> 0}$-orbit of $\tilde h = I_{d+1} + h$. Then ${\bm Q}_d$ is a real vector space with dimension equal to $d^2 + 2d$. 

\begin{lemma}\label{lem:boundary-up-to-stabilisation} There exists a smooth $G_\Lambda$-vector bundle 
\begin{equation}
    Q_{\Lambda^*}\to \cc{\scB}^\bR_{\Lambda^*}
\end{equation}
so that the total space admits an equivariant diffeomorphism $Q_{\Lambda^*}\to \del_{\Lambda^*}\scB_\Lambda$ to the corner stratum of $\scB_\Lambda$ extending the embedding~\eqref{eq:emb-base},\qed\end{lemma}

\begin{proof}
    This is similar to {\cite[Proposition~5.24]{BX22}} but since we have several punctures and asymptotic markers at outgoing as well as incoming punctures, we describe how the proof has to be adapted. We make a few simplifying assumptions to keep the exposition clear while still highlighting the essential modifications needed. Thus we will deal with the case of $\Lambda^* = \{ \Lambda^0,\Lambda^1 \}$, hence $\Lambda = \Lambda^1 \circ \Lambda^0 $. We further assume $\Gamma^+ =\{\gamma^+\}$. The general result will follow from a similar but combinatorially more involved argument. Write $\Gamma = (\gamma_1, \dots, \gamma_m).$
    
    Define the bundle $Q_{\Lambda^*}$ to be the $G_{\Lambda}$-equivariantization of the $G_{\Lambda^*}$ bundle $\bm Q_{\Lambda^*} \times \im(\Psi_{\Lambda^*})$ where $\bm Q_{\Lambda^*}$ is a distinguished subspace of $\bm Q_{{1+d_0 + d_{10}\dots +d_{1m}}}$. The required equivariant diffeomorphism $Q_{\Lambda^*} \to \partial_{\Lambda^*}\scB_\Lambda$ is determined uniquely by the map $$\rho \cl \bm Q_{\Lambda^*} \times \im (\Psi_{\Lambda^*}) \to \partial_{\Lambda^*}\scB_\Lambda$$ 
    such that $$\rho (h,\varphi)= (Id+\rho_h)\varphi,$$ 
    where $\rho_h$ is the upper diagonal matrix corresponding to $h$. The proof of the map $Q_{\Lambda^*} \to \partial_{\Lambda^*}\scB_\Lambda$  being injective follows exactly by the same arguments as in \cite[Proposition 5.24]{BX22}. 

    In order to prove surjectivity, recall that a rational stable map $\varphi$ of degree $\diamond \le d$ curve in $\bP^d$ lies in a unique minimal linear $\bP(W)\sub \bP^d$ where $\dim W = \diamond +1$. We call $W$ the \emph{linear span} of $\varphi$ \footnote{ We are slightly abusing terminology since classically the linear span is defined as the projectivization of the vector space we consider here}. Thus, any curve $\varphi \in \partial_{\Lambda^*}\scB_\Lambda$, determines a system of linear projective spaces $\bP(W_0), \bP(W_{10}),\,\dots ,\,\bP(W_{1m})$ where $W_{*}\hkra \bC^{1+d_0 + d_{10}\dots +d_{1m}}$ is a subspace so that
    \begin{itemize}
        \item $W_0$ is the linear span of $\varphi_{\gamma^+}$ and $W_{1i}$ is the linear span of $\varphi_{\gamma_i}$,
        \item $\dim (W_0 \cap W_{1i}) =1$,
        \item $W_{1i} \cap W_{1j} = \emst $ unless $i=j$,
        \item $W_0 + W_{10} \dots + W_{1m} = \bC^{1+d_0 + d_{10}\dots +d_{1m}}.$
    \end{itemize}
    The above system of spaces can be viewed as a `tree' generalization of the fans defined in \cite[\S 5.2.5]{BX22}. Thus, we call such a system a \textit{t-fan}. Set $L_i \coloneqq W_0 \cap W_{1i}$. We say a t-fan is \textit{normal} if $W_0$ is orthogonal to the orthogonal complement of $L_i \hkra W_{1j}$ for all $i$. A similar argument as in \cite[Proposition 5.21]{BX22} proves that $\cc{\scB}^\bR_{\Lambda^*}$ is exactly the set of curves with normal t-fans.

   The \textit{t-flag} corresponding to the t-fan $W_*$ is defined as \begin{itemize}
        \item $V_{0}= W_{0}$,
        \item $V_{1i} = W_{0} + W_{1i}.$
    \end{itemize}

By using the Gram--Schmidt process, any t-flag can be mapped to a standard t-flag under an action of $g\in G_\Lambda$ where the standard t-flag is defined as
    \begin{itemize}
        \item     $V_{0}= \bC^{1+d_0}$,
        \item $V_{1j} = \bC^{1+d_0} \times \{0\}^{d_{11} +\dots d_{1j-1}}\times \bC^{d_{1j}} \times \{0\}^{d_{1j+1} +\dots d_{1m}}.$
    \end{itemize}
Thus, given some $\varphi \in \partial_{\Lambda^*} \scB_\Lambda$, we can assume that the t-flag corresponding to it is the standard one. In particular we can assume that that linear span of $\varphi_{\gamma^+}$ is $\bC^{1+d_0} \hkra \bC^{1+d_0 + d_{10}\dots +d_{1m}}$, i.e., is given by the first $1+d_0$ coordinates. Let $y_i=\wt {ev}_{\gamma_i}(\varphi_{\gamma_+}) \in \bC^{1+d_0}$ and choose vectors $w^{1i}_1,w^{1i}_2,\dots ,w^{1i}_{d_i}$ such that $W_{1i}$ has a basis $(y_i,w^{1i}_1,w^{1i}_2,\dots ,w^{1i}_{d_i}).$ 

Let $\bm Q_{\Lambda^*}$ denote the elements of $\bm Q_{1+d_0+\dots}$ which are of the form $Id + A +A^*$ where $A$ is of the following block matrix form,


\begin{equation}\label{matrxform}
A=\begin{bmatrix}
0 & A^0_1 & A^0_2 & \cdots & A^0_m \\
0 & 0 & 0 & \cdots & 0 \\
0 & 0 & 0 & 0 & \cdots \\
\vdots & & & \ddots & \\
\end{bmatrix}.  
\end{equation}
We will construct an element $h\in \bm Q^*_{1+d_0+\dots}$ such that $I+h^u$ takes the t-fan $W_*$ to a normal t-fan where $h^u$ is the upper triangular matrix corresponding to $h$. Denote the orthogonal complement of $W_{0} \cap W_{1i}$ in $W_{1i}$ as $W_i^o$. Note that $W_i^o$ can be viewed as a graph of a linear function $T^o_i:\bC^{d_i} \to \bC^{1+d_0}$. Thus we can find a matrix $A$ of the form as described in \eqref{matrxform} such that $I+A+A^*$ takes the t-fan $W_*$ to the normal t-fan 
\begin{itemize}
    \item $\scW_0 =\bC^{1+d_0}\times\{0\}^{d_{11}+ d_{12}\dots}$,
    \item $ \scW_i = \bC \cdot \langle y_i \rangle \oplus \{0\}^{1+d_0+d_{11}\dots d_{1i-1}}\times \bC^{d_{1i}} \times \{ 0\}^{d_{1i+1}+\dots}$.
\end{itemize}
This shows that $\varphi$ lies in the image of the map $Q_{\Lambda^*} \to \partial_{\Lambda^*}\scB_\Lambda$. 
\end{proof}

\subsubsection{Embeddings of families of buildings}\label{subsec:embeddings-thickening} We first discuss the inductive construction when $\norm{(\Gamma^+,\Gamma^-)} = 1$. Let $\Lambda\cl \Gamma^-\to \Gamma^+$ be a partition and $\Gamma^-\xra{\Lambda^0}\Gamma\xra{\Lambda^1}\Gamma^+$ be a factorization. We will do the construction for the moduli space associated to $\Lambda$ and then endow any other moduli space in the orbit of the symmetric action with the same global Kuranishi chart after permuting the labels of the punctures.\par
By the inductive hypothesis, we have constructed global Kuranishi charts $\scK_{\Lambda^0}$ and $\scK_{\Lambda^1}$ with base spaces $\scBR_{\Lambda^0}$, respectively $\scBR_{\Lambda^1}$ and by Proposition~\ref{prop:embedding-base-spaces}, there exists an embedding
\begin{equation}\label{eq:emb-base}
    \Psi\cl \scBS_{\Lambda^0}\times_{S^1} \scBS_{\Lambda^1}\hkra \scBS_{\Lambda}
\end{equation} 
whose image is contained in a boundary stratum $\scBS_{\Lambda^{01}}$ of $\scBS_{\Lambda}$. Define the `restricted family'
$$\scZ_{\Lambda} \coloneqq\scBS_{\Lambda}\times_{\cB_{\Lambda}} \cZ_{\cBR_{\Lambda}}$$
of buildings, where $\cZ_{\cBR_{\Lambda}}$ was defined in Definition~\ref{de:family-of-tree}. To lift the embeddings~\eqref{eq:emb-base} to maps between these families, we need the following definition.

\begin{definition}[Orbifold associated to Reeb orbit]\label{de:reeb-orbifold}
    Given an unparametrized Reeb orbit $\gamma$, let 
\begin{equation}\label{eq:parametrisations-reeb-orbit}
    E\gamma\coloneqq\set{\sigma\in C^\infty(S^1,Y)\mid \dot{\sigma} = \cA_\lambda(\gamma)\,R(\sigma),\,\im(\sigma) = \im(\gamma)}
\end{equation}
be the space of parametrized Reeb orbits lying over $\gamma$. Then, $S^1$ acts transitively on $E\gamma$ with isotropy $\bZ/m_\gamma$ and we define 
\begin{equation}
    B\gamma := [E\gamma/S^1].
\end{equation}
Note that we have an equivalence $B\bZ/{m_\gamma} \to B\gamma$, where $m_\gamma$ is the multiplicity of $\gamma$.
\end{definition}

\noindent The asymptotic markers yield rel--$C^1$ evaluation maps
\begin{equation}\label{eq:evaluation-to-reeb-orbit}
    p_\gamma\cl \ol\scT_\Lambda\coloneqq [\scT_\Lambda/\wh G_\Lambda] \to B\gamma 
\end{equation}
induced by 
$$(\varphi,u,w) \mapsto [\theta\mapsto (\wh u)_{z_\gamma}(\theta\cdot b_\gamma)],$$
where $\theta\in S^1$ and for each $\gamma$ labeling an exterior edge $b_\gamma$ denotes the asymptotic marker associated to $\varphi$ and $\gamma$. Given a sequence $\Gamma =(\gamma_1,\dots,\gamma_k)$ of Reeb orbits, we set $$B\Gamma\coloneqq \prod_{i=1}^kB{\gamma}_i.$$
Then, the (smooth local) embeddings~\eqref{eq:emb-base} lift to (continuous) embeddings 
\begin{equation}\label{eq:emb-families-with-maps}
   \Psi_{\Lambda^{01}}\cl  \scZ_{\Lambda^0}\ov{\times}_{B\Gamma}\,\scZ_{\Lambda^1}\coloneqq \lbr{\scZ_{\Lambda^0}\times_{B\Gamma}\,\scZ_{\Lambda^1}}/S^1\hkra \scZ_{\Lambda},
\end{equation}
where $S^1$ acts on the fiber product via the diagonal embedding $S^1\hkra \bT_{\Gamma}\times\bT_\Gamma$. The lift uses the fact that the map~\eqref{eq:emb-base} is covered by a canonical isomorphism 
\begin{equation}\label{eq:emb-family}
\pr_1^*\scC_{\Gamma_i,\Gamma^+}\inn\sqcup\pr_2^*\scC_{\Gamma^-,\Gamma_i}\inn\cong\Psi^*\scC\inn_{\Gamma^+,\Gamma^-}.
\end{equation}
Note that we have already fixed a pre-perturbation datum $\fD$ for $\Mbar_{\sft}^J(\Gamma^+,\Gamma^-)_\Lambda$. Suppose $\cU_\Lambda$ is a good covering as in Definition~\ref{de:good-covering} and $\zeta$ a smooth map as in \eqref{eq:reducing-structure-group} yielding a $\cG_\Lambda$-equivariant map
\begin{equation}\label{eq:slice-map}
\zeta_\Lambda\cl \cZ_\Lambda \to \fp\fu_\Lambda \coloneqq\bigoplus\limits_{\gamma\in \Gamma^+}{\text{Lie}(\PU({d_{\Lambda_\gamma}+1}))},
 \end{equation}
 due to the isomorphism~\eqref{eq:comparing-group-quotients}. The pullback of~\eqref{eq:slice-map} to $\scZ_\Lambda$ yields a $\cG_{\Lambda}$-equivariant map $\zeta_\Lambda\cl\scZ_\Lambda\to\fg_\Lambda$. Thus, we may demand that $\zeta_\Lambda$ restricts to $\zeta_{\Lambda^1}\times\zeta_{\Lambda^0}$ over the respective boundary stratum. Assume also that we have found a perturbation space $(E_\Lambda,\mu)$ so that $(E_{\Lambda^1},\mu_{\Lambda^1})\times (E_{\Lambda^1},\mu_{\Lambda^1})$ admits a linear equivariant embedding into the pullback of $(E,\mu)$ along~\eqref{eq:emb-family}. Recall that the perturbation space depends on the choice of $\zeta_\Lambda$, since the regularity condition has to be satisfied over the locus $\{(\varphi,u)\mid \delbar_{J}u = 0,\, \zeta_\Lambda(\varphi,u) = 0\}$. Then, $\alpha_{\Lambda} = (\fD,\cU_\Lambda,\zeta,E_{\Lambda},\mu_\Lambda)$ is a well-defined perturbation datum. Thus, Proposition~\ref{prop:disconnected-buildings} yields a global Kuranishi chart 
$$\cK_\Lambda = \lbr{\bT_{\Gamma^+}\times\bT_{\Gamma^-}\times G'_\Lambda,\cTR_\Lambda,\cE^\bR_\Lambda,\fs_\Lambda}$$
for $\Mbar^{\, J}_{\sft}(\Gamma^+,\Gamma^-)_\Lambda$ equipped with a rel--$C^1$ map $\cT_\Lambda\to \cBS_\Lambda$. We let $G_\Lambda\sub G'_\Lambda$ be the product of stabilizers of $[1:0:\dots:0]$ and set $\wh G_\Lambda := \bT_{\Gamma^+}\times\bT_{\Gamma^-}\times G_\Lambda$. Define the $\wh G_\Lambda$-invariant subspaces
\begin{equation}
    \scT_\Lambda \coloneqq \scBS_\Lambda\times_{\cBS_\Lambda}\,\cTR_\Lambda\qquad \qquad   \scE_\Lambda \coloneqq \scBS_\Lambda\times_{\cBS_\Lambda}\,\cE^\bR_\Lambda
\end{equation}
and denote the pullback of $\fs_\Lambda$ to $\scT_\Lambda$ by the same symbol. By construction, the embeddings \eqref{eq:emb-families-with-maps} induce rel--$C^1$ embeddings
\begin{gather}\label{eq:emb-thickenings}
   \notag\wt\Psi\cl \scT_{\Lambda^0}\ov{\times}_{B\Gamma}\, \scT_{\Lambda^1}\hkra \scT_{\Lambda} \\ 
   ((\varphi_0,u_0,w_0),(\varphi_1,u_1,w_1)) \mapsto (\Psi((\varphi_0,u_0),\varphi_1,u_1)),w_0\oplus w_1).
\end{gather}
They are covered by embeddings of obstruction bundles, defined as follows. The original obstruction bundle is the direct sum $\cE_\Lambda = E_\Lambda \oplus \fp\fu(d_\Lambda +1)$. The embeddings $E_{\Lambda^1_i}\oplus E_{\Lambda^0_i}\hkra E_\Lambda$ exist by assumption, while the embeddings of Lie algebras are by the inclusion~\eqref{eq:inclusion-of-groups} of covering groups. We define 
\begin{equation}
     \scK_{\Lambda^0}\ov{\times}_{B\Gamma}\,\scK_{\Lambda^1}\coloneqq \lbr{\bT_{\Gamma^-}\times G_{\Lambda^0}\times G_{\Lambda^1}\times \bT_{\Gamma^+},\scT_{\Lambda^0}\ov{\times}_{B\Gamma}\,\scT_{\Lambda^1},\scE_{\Lambda^0}\boxplus\scE_{\Lambda^1},\fs_{\Lambda^0}\boxplus\fs_{\Lambda^1}}
\end{equation}
and summarize the construction in the following lemma.

\begin{lemma}\label{lem:embedding-inductive-step}
    There exists a global Kuranishi chart
    \begin{equation}\label{eq:slice-gkc}
        \scK_{\Lambda} \;\coloneqq\; (\wh{G}_\Lambda,\scT_{\Lambda},\scE_{\Lambda}, \fs_{\Lambda}) 
    \end{equation} 
    for $\Mbar_{\sft}^{\, J}(\Gamma^+,\Gamma^-)_\Lambda$ admitting strong equivalences 
     \begin{equation}\label{eq:emb-gkc}\scK_{\Lambda^0}\ov{\times}_{B\Gamma}\,\scK_{\Lambda^1} \hkra \scK_{\Lambda}
    \end{equation}
    onto a boundary strata of $\scK_{\Lambda}$ for any factorization $\Gamma^-\xra{\Lambda^0}\Gamma\xra{\Lambda^1}\Gamma^+$ of $\Lambda$.
\end{lemma}

\begin{proof}
   The discussion above shows that it remains to construct a suitable good covering and a suitable perturbation space. To keep the notation tractable, we assume that $\Gamma^+ = \{\gamma\}$ and $\Lambda^1 = \{\gamma'\}$. The general case is a straightforward generalization.\\
   
   \noindent\textbf{Step 1:} The usual clutching maps induce embeddings
   \begin{equation}\label{eq:clutching-on-stable}
   \psi\cl \scB^{\text{st}}_{1+\#\Lambda^1+3d'_{\Lambda^1}}(d_{\Lambda^1})\times \scB^{\text{st}}_{1+\#\Lambda^0+3d'_{\Lambda^0}}(d_{\Lambda^0})\hkra \scB^{\text{st}}_{1+\#\Lambda+3d'_{\Lambda}}
   \end{equation}
   where the superscript indicates that we only consider stable maps. The maps $\psi$ are equivariant with respect to the inclusions $\cG_{\Lambda^1}\times\cG_{\Lambda^1}\hkra \cG_\Lambda$. By \cite[Lemma~4.13]{AMS23} taking $X = \{\Pt\}$ in Definition~4.6 op. cit., the $\cG_\Lambda$-action on $\cB^{\stb}_{1+\#\Lambda + 3d'_{\Lambda}}$ is Palais proper. Thus, \cite[Theorem~4.3.1]{Pal61} asserts that $\scB^{\text{st}}_{\Lambda,+3d'_{\Lambda}}$ admits a $\cG_\Lambda$-invariant Riemannian metric as well as $\cG_\Lambda$-invariant bump functions. First, we can extend the product of the $\zeta_i := \zeta_{\cU^i}$ to the canonical  $\cG_\Lambda$-equivariant function 
   \[\zeta_{01}\cl \im(\psi)^\sim\coloneqq\cG_\Lambda\times_{\cG_{\Lambda^1}\times\cG_{\Lambda^0}}\im(\psi)\,\to\, \fg_\Lambda.\]
   Then, choosing a $\cG_\Lambda$-equivariant tubular neighborhood of $\im(\psi)^\sim$, we can extend $\zeta_{01}$ to a $\cG_\Lambda$-equivariant function on an open neighborhood of $\im(\psi)^\sim$. If we have several embeddings of the form~\eqref{eq:clutching-on-stable}, their images are disjoint, even under the $\cG_\Lambda$-action, due to our assumption on $\Gamma^+$ and $\Gamma^-$. Thus we can choose the tubular neighborhoods to be disjoint. Then, using $\cG_\Lambda$-invariant cut-off functions, we can extend the thus obtained functions to a smooth $\cG_\Lambda$-equivariant function 
   \begin{equation}\label{eq:suitable-slice-function}
    \zeta \cl\scB^{\text{st}}_{\Lambda,+3d'_{\Lambda}} \to \fg_\Lambda.
   \end{equation}
   It remains to extend the good coverings $\cU_{\Lambda^1}$ and $\cU_{\Lambda^0}$. Recall that such a good covering consists of a finite collection $\{(U_i,\sigma_{i,j},D_{i,j},\chi_i)\}_{i,j}$ of 
    \begin{enumerate}[ref=\roman*,leftmargin=20pt,label=\roman*)]
        \item\label{i:intersection-nice} open $\cG_{\Lambda'}$-invariant subsets $U_i \sub \scZ_{\Lambda'}$ that cover $\scZ_{\delbar}$
        \item a smooth $\cG_{\Lambda'}$-equivariant section $\sigma_{i,j}\cl U_i \to \cC\inn|_{U_i}$ for $1 \leq j \leq 3d_{\Lambda'}$ and divisors $D_{i,j}\sub Y$ so that for any $(\varphi,u)\in U_i$, we have $u\pf D_i$ and $u(\sigma_{i,j}(\varphi,u))\in D_{i,j}$ and
        $$\# \,C_v\cap \{\sigma_{i,j}(\varphi,u)\}_j = \frac{3}{p}\deg(L_{u,v})$$
        \noindent for any irreducible components $C_v\sub C$, allowing for the stabilisation map to lift to 
        \begin{equation}
            \text{st}_{U_i}\cl U_i \to \cB_{1+\#\Lambda + 3d'_{\Lambda'}}(d_{\Lambda'})
        \end{equation}
        \item $\cG_{\Lambda'}$-invariant functions $\chi_i \cl Z_i \to [0,1]$ with support contained in $U_i$
    \end{enumerate}
    so that $\scZ_{\Lambda',\delbar}$ is contained in the support of $\s{}{\chi_i}$.\par
    \vspace{-5pt}
    \noindent Write $\cU_{\Lambda^r}= \{(D_{i,j}^r,U_i^r,\chi_i^r)\}_{i\in I^r,j}$ and choose for $i \in I^0\times I^r$ an open $\cG_{\Lambda}$-invariant subset $U_i$ of $\scZ_\Lambda$ so that $U_i \cap \im(\Psi) = \Psi(U_{i^1}\times U_{i^0})$ and $U_i$ does not intersect any other other boundary stratum. Shrinking $U_i$ if neccessary, we may ensure that for any $1 \leq j \leq 3d'_{\Lambda^r}$ the section $\sigma_{i^r,j}^r$ extends to a $\cG_\Lambda$-invariant section $U_i\to\cC\inn|_{U_i}$ with $u(\sigma_{i,j}^r(\varphi,u))\in D^r_{i^r,j}$ and $u\pf D_{i,j}$ near $\sigma_{i^r,j}(\varphi,u)$ for any $(\varphi,u)\in U_i$. This requires the divisor $D_{i^r,j}$ and the fact that transversality is an open condition. Note that we do not need the whole map $u$ to intersect $D^r_{i^r,j}$ transversely. 
    In particular, this construction yields a $\cG_{\Lambda}$-equivariant map $\stb_{U_i}\cl U_i \to \cB^{\stb}_{1+\# \Lambda + 3d'_{\Lambda}}(d_\Lambda)$ so that 
\begin{equation}\label{eq:square-for-coverings}\begin{tikzcd}
    U_{i^1}^1\times_{S^1} U_{i^0}^0 \arrow[rr,"\Psi"] \arrow[d,"\stb_{U_{i^1}^1}\times\stb_{U_{i^0}^0}"]&&  U_i\arrow[d,"\stb_{U_i}"]\\ \scB^{\stb}_{1+\#\Lambda^1+3d'_{\Lambda^1}}\times\scB^{\text{st}}_{1+\#\Lambda^0+3d'_{\Lambda^0}}\arrow[rr,""] && \scB^{\text{st}}_{1+\#\Lambda+3d'_{\Lambda}}
    \end{tikzcd} \end{equation}
     commutes. Now, we can use the Tietze extension theorem, applied to $\cZ_\Lambda/\cG_\Lambda$, to extend the cut-off functions $\chi_{i^1}^1\times\chi^0_{i^0}$ on $\Psi(U_{i^1}^1\times U_{i^0}^0)^\sim$ equivariantly to obtain invariant continuous functions $\chi'_{i}\cl U_i\to [0,1]$. Since $A_i  := \cG\cdot\text{supp}(\chi_{i^1}^1\times\chi^0_{i^0})$ is closed in $\scZ_\Lambda$ and contained in $U_i$, we can use the Tietze extension theorem again to find a $\cG$-invariant function $\rho_i \cl \scZ_\Lambda\to [0,1]$ which is identically $1$ on $A_i$ and supported in $U_i$. Thus, we can extend $\rho_i\chi'_i$ to a $\cG$-invariant function $\chi_i$ on all of $\cZ_\Lambda$, extending the cut-off function on the boundary. Now, we can complete $\cU'$ to an invariant open cover $\cU$ of $\scZ_{\Lambda,\delbar}$ so that any $U \in \cU\sm \cU'$ does not meet the images of the embeddings $\Psi_i$. This yields the desired good covering $\cU_\Lambda$, so that $\zeta_{\cU_\Lambda}$ is an extension of the functions $\zeta_{01}$.\\
     
     \noindent\textbf{Step 2:} First, note that we can rephrase a perturbation space $(E,\mu)$ as the trivial $G$-vector bundle $E\to \cC\inn\times \wh Y$ equipped with a $\bR\times G$-equivariant vector bundle morphism $E\to \Lambda^{0,1*}_{\cC\inn/\cB}\otimes_\bC T\wh Y$. Thus, we can extend the trivial $G_{\Lambda_i^1}\times G_{\Lambda^0_i}$-vector bundle $E_{\Lambda^1}\times_{S^1} E_{\Lambda^0}\to \scB_0\times_{S^1}\scB_1$ to a $G_\Lambda$-vector bundle 
 \begin{equation*}
    E'_{01} := G_\Lambda\times_{G_{\Lambda_i^1}\times G_{\Lambda^0_i}} (E_{\Lambda^1}\times_{S^1} E_{\Lambda^0})
 \end{equation*}
 and we can extend $\mu_{\Lambda^1}\times\mu_{\Lambda^0}$ uniquely to $G_\Lambda$-equivariant map. By \cite[Proposition~1.1]{Las79}, we can take the direct sum with a $G_\Lambda$-vector bundle $W\to \im(\Psi)^\sim$, equipped with the zero map to $\Lambda^{0,1*}_{\cC\inn/\cB}\otimes_\bC T\wh Y$, to obtain a $G_\Lambda$-perturbation space $(E_{01},\mu_{01})$ for $\im(\Psi)^\sim$ that extends the perturbation space of the boundary stratum. Using the map of Lemma~\ref{lem:boundary-up-to-stabilisation} and multiplying $\mu_{01}$ with a suitable cut-off function, we can pull back this perturbation space to a perturbation space on the whole boundary stratum. Using a (sufficiently small) collar of $\im(\Psi_{\Lambda^0,\Lambda^1})^\sim$ and a bump function, we can extend $(E_{01},\mu_{01})$ to a perturbation space $(E_{01},\mu_{\Lambda,01})$ on all of $\cB_\Lambda$. Since these perturbation spaces are sufficient to achieve transversality for maps in $$\fs_{\text{pre}}\inv(0) = \{(\varphi,u)\in \cZ_\Lambda\mid \zeta_\cU(\varphi,u) = 0\}$$ 
 that lie near some boundary stratum, we can extend the direct sum $\bigoplus\limits_{\Gamma_i}(E_{01},\mu_{\Lambda,01})$ to a perturbation space $(E_\Lambda,\mu_\Lambda)$ for $\fs_{\text{pre}}\inv(0)$ so that over a boundary stratum $(E_\Lambda,\mu_\Lambda)$ is of the form $(E_{01},\mu_{01})\oplus (W,0)$. This yields the desired perturbation datum $\alpha =(\fD,\cU,\zeta,E,\mu)$.\par
      By definition (and \cite[Lemma~3.5]{AMS23}) the global chart $\scK_{\Lambda}$ is equivalent to $\cK_{\Lambda}$. In particular, it is a global chart for $\Mbar_{\sft}^{\, J}(\Gamma^+,\Gamma^-)_\Lambda$. The embeddings~\eqref{eq:emb-thickenings} exist by our choice of perturbation space $E_{\Gamma^+,\Gamma^-}$ and they fit into a commutative square 
    \begin{equation}\begin{tikzcd}
    \scE_{\Lambda^0}\ov{\times}_{B\Gamma}\, \scE_{\Lambda^1} \arrow[r,""] \arrow[d,""]& \scE_{\Lambda} \arrow[d,""]\\ 
    \scT_{\Lambda^0}\ov{\times}_{B\Gamma}\, \scT_{\Lambda^0} \arrow[r,""] & \scT_{\Lambda} 
    \end{tikzcd} \end{equation}
    This completes the proof.
\end{proof}

\begin{remark}
    In the construction of the perturbation space, we have used a strategy of \cite{Rez22}, instead of the construction in \cite{BX22}. This allows us to see the embeddings of boundary strata immediately as strong equivalences, while \cite{BX22} has to use an outer-collaring as well as some gluing results (\cite[Proposition~5.64]{BX22}). 
\end{remark}

It remains to prove the inductive step.

\begin{proposition}\label{prop:embedding-gkc}
    There exists a system of global Kuranishi charts $\{\scK_{\Lambda}\}_{\Lambda\cl\Gamma^-\to\Gamma^+}$ for the moduli spaces $\Mbar_{\sft}^J(\Gamma^+,\Gamma^-)_\Lambda$ admitting strong equivalences 
    \begin{equation}\label{eq:emb-gkc-first-step}
        \scK_{\Lambda^0}\ov{\times}_{B\Gamma}\,\scK_{\Lambda^1}\hkra \scK_{\Lambda}
    \end{equation} 
    onto the codimension-$1$ boundary strata of $\scK_{\Lambda}$ for any factorization $\Lambda = \Lambda^1\g\Lambda^0$ so that any boundary stratum of $\scK_{\Lambda}$ is the image of~\eqref{eq:emb-gkc} up to stabilization for some factorization and the squares 
    \begin{equation}\begin{tikzcd}
    \scK_{\Lambda^0}\ov{\times}_{B\Gamma_0}\, \scK_{\Lambda^1}\ov{\times}_{B\Gamma_1}\, \scK_{\Lambda^2} \arrow[r,""] \arrow[d,""]&\scK_{\Lambda^0}\ov{\times}_{B\Gamma_0}\,\scK_{\Lambda^{21}}\arrow[d,""]\\  \scK_{\Lambda^{10}}\ov{\times}_{B\Gamma_1}\,\scK_{\Lambda^2} \arrow[r,""] & \scK_{\Lambda} 
    \end{tikzcd} \end{equation}
    are pullback squares.
\end{proposition}

\begin{proof} 
    The construction of the embeddings $\scT_{\Lambda^1}\ov{\times}_{B\Gamma}\,\scT_{\Lambda^0}\hkra \scT_{\Lambda}$ of thickenings for respective partitions of Reeb orbits follows by induction, using the arguments of \textsection\ref{subsec:embeddings-thickening} and Lemma~\ref{lem:embedding-inductive-step}. The same holds for the embeddings of obstruction bundles. The additional difficulty in the general case is to ensure that our extensions of good coverings and perturbation spaces can be done compatibly over the corner strata.  We first extend the maps of~\eqref{eq:suitable-slice-function} from the boundary strata to the interior. Extending $\zeta_{\Gamma^+,\Gamma'}$ and $\zeta_{\Gamma',\Gamma^+}$ and extending to $\cG$-equivariant maps, we obtain $\cG$-equivariant functions 
    $$\wt\zeta_{\Gamma'}\cl \del_{\Lambda^{01}} {\cBS}^{\stb}_\Lambda\,\longrightarrow\,\fg_{\Lambda} $$
    that agree over the corner strata. By Lemma~\ref{lem:extending-from-boundary}, we may extend them to a $\cG$-equivariant smooth function $\zeta' \cl {\cBS}^{\stb}_{\Gamma^+,\Gamma^-}\to \fg$. By the inductive hypothesis, we assume that the good coverings of the codimension-$1$ strata over their respective boundary strata are obtained from good coverings of the moduli spaces forming the codimension-$2$ strata. Thus, we may use the same argument as in the discussion before the square~\eqref{eq:square-for-coverings} to extend the good coverings of the boundary strata to a good covering of a neighborhood of the boundary. In particular, the maps $\stb_{U_i}$ and $\stb_{U_j}$ agree over the intersection of the boundary with $U_i \cap U_j$. Thus, we may choose arbitrary extensions of $\cG$-invariant bump functions and extend these data to a good covering as at the end of the proof of Lemma~\ref{lem:embedding-inductive-step}.\par 
    To construct the perturbation space $(E_{\Gamma^+,\Gamma^-},\mu_{\Gamma^+,\Gamma^-})$, we observe that the beginning of the construction in the proof of Lemma~\ref{lem:embedding-inductive-step} applied to the codimension-$1$ boundary strata of $\cT_{\Gamma^+,\Gamma^-}$ yields a $G$-representation $E'_\Lambda$, equipped with a a map $$\mu^\del_\Lambda\cl C^\infty_c(\cC\inn|_{\del\scBS_{\Gamma^+,\Gamma^-}}\times\wh Y,\Lambda^{0,1*}_{\cC\inn/\scBS_{\Gamma^+,\Gamma^-}}\otimes_\bC T\wh Y)^\bR.$$
    whose restrictions to the closures of the codimension-$1$ strata agree over the codimension-$2$ strata. Thus, we can extend $\mu^\del_\Lambda$ to a $G_\Lambda$-equivariant map $\mu'_\Lambda\cl E'_\Lambda\to C^\infty_c(\cC\inn\times\wh Y,\Lambda^{0,1*}_{\cC\inn/\scBS_{\Gamma^+,\Gamma^-}}\otimes_\bC T\wh Y)^\bR$ by \cite{Kott}. As this is sufficient to achieve transversality for curves with domain near the boundary, we may extend $(E'_\Lambda,\mu'_\Lambda)$ to a perturbation datum, where $(E_\Lambda,\mu_\Lambda) = (E'_\Lambda,\mu'_\Lambda)\oplus (E\inn,\mu\inn)$, where $\mu\inn(e)$ is supported away from $\cC|_{\del\scB_\Lambda}\times \wh Y$. 
    This proves the first claim. Any boundary stratum is covered by such an embedding since any boundary stratum of $\scB_\Lambda$ is a vector bundle over the image of some embedding $\scB_{\Lambda^1}\times\scB_{\Lambda^0}\to \scB_\Lambda$. The last assertion follows from Lemma~\ref{lem:associativity-of-embeddings} and the construction of the perturbation spaces.
\end{proof}


\begin{lemma}\label{lem:extending-from-boundary}
    Suppose $\cG$ acts properly on a smooth manifold $M$ with corners and for each boundary stratum $S\sub \del M$ there exists a $\cG$-equivariant function $f_S \cl S\to V$ to a common finite-dimensional $\cG$-representation so that the restrictions agree over the codimension-$2$ corner strata. Then, there exists a $\cG$-equivariant smooth function $f \cl M\to V$ that extends the functions $f_S$.
\end{lemma}

\begin{proof}
    We first apply \cite{Kott} to the functions $\{f_S\}$ to obtain a smooth extension $\wt f\cl M\to V$. Averaging over $G$, we may assume $\wt f$ to be $G$-invariant. By \cite{Pal61}, we can find a locally finite open cover $\cU$ of $M$ so that for each $U \in \cU$ there exists a $G$-invariant subset $S\sub U$ with $U\cong\cG\times_G S$. Moreover, we can find a $\cG$-invariant partition of unity $\{\varsigma_U\}_U$ subordinate to $\cU$. Define $f_U \cl U\to V$ by $f_U(g\cdot s) = g\cdot \wt f(s)$ for $(g,s)\in \cG\times S$ and set $f := \s{U}{\varsigma_U\, f_U}$ to obtain the desired extension.
\end{proof}


This completes the proof of Theorem~\ref{thm:flow-cat}.