%documentclass{article}
\documentclass[a4paper, reqno]{amsart}
\usepackage[utf8]{inputenc}
\usepackage{palatino,newpxmath} % il primo è il font del testo, il secondo è il font matematico
\linespread{1.20} % l'interlinea, 1 è normale
\usepackage[left=3cm,right=3cm,top=3cm,bottom=3cm]{geometry}
\usepackage[english]{babel}
\usepackage{natbib}
\usepackage{url}
\usepackage{amsmath}
\let\Bbbk\relax
\usepackage{amssymb}
\usepackage{amsthm}
\usepackage{mathtools}
\usepackage{amsfonts}
\usepackage{bbm}
\usepackage{caption}
%\usepackage{eufrak}
\usepackage{gensymb}
\usepackage{tikz-cd}
\usepackage{mathdots}
\usepackage{cancel}
\usepackage{multicol}
%\usepackage{multirow}
\usepackage[font=small,labelfont=bf]{caption}
\usepackage[colorlinks=true, urlcolor = blue]{hyperref} % qui puoi cambiare i colori, tipo mettendo citecolor = red ti mette i riferimenti alla bibliografia in rosso, urlcolor = green ti mette gli url (= quelli esterni al file) in verde, etc.
\usepackage[shortlabels]{enumitem}
\usepackage{paracol}
\setenumerate[0]{label=(\arabic*)}

%\usepackage{natbib}
%\usepackage{graphicx}

\usepackage[many]{tcolorbox}  %for colored boxes

%\usepackage{romannum}

\newtcolorbox{boxE}{
    enhanced, % for a fancier setting,
    boxrule = 0pt, % clearing the default rule
    borderline = {0.75pt}{0pt}{main}, % outer line
    borderline = {0.75pt}{2pt}{sub} % inner line
}

\newcommand{\red}[1]{\textcolor{red}{#1}}

\newcommand{\setm}{\smallsetminus}
\newcommand{\N}{\mathbb{N}}
\newcommand{\Z}{\mathbb{Z}}
\newcommand{\Q}{\mathbb{Q}}
\newcommand{\R}{\mathbb{R}}
\newcommand{\C}{\mathbb{C}}
\newcommand{\D}{\mathbb{D}}
\newcommand{\Ss}{\mathbb{S}}
\renewcommand{\H}{\mathbb{H}}
\renewcommand{\P}{\mathbb{P}}
\newcommand{\RP}{\mathbb{RP}}
\newcommand{\CP}{\mathbb{CP}}
\newcommand{\HP}{\mathbb{HP}}
\newcommand{\F}{\mathbb{F}}
%\newcommand{\G}{\mathbb{G}}
\newcommand{\G}{\mathcal{G}}
\newcommand{\B}{\mathcal{B}}
\newcommand{\DR}{\mathcal{DR}}


\renewcommand{\phi}{\varphi}
\newcommand{\rad}[1]{\sqrt{#1}}
\newcommand{\norm}[1]{\left\lVert#1\right\rVert}
\newcommand{\Qmatrix}[4]{\begin{pmatrix}#1&#2\\#3&#4\end{pmatrix}}
\newcommand{\qmatrix}[4]{\left(\begin{smallmatrix}#1&#2\\#3&#4\end{smallmatrix}\right)}

\newcommand{\g}{\mathfrak{g}}
\newcommand{\h}{\mathfrak{h}}
\DeclareMathAlphabet{\mathpzc}{OT1}{pzc}{m}{it} % k
\renewcommand{\k}{\mathpzc{k}}
\newcommand{\q}{\mathfrak{q}}
\renewcommand{\c}{\mathfrak{c}}
\newcommand{\gl}{\mathfrak{gl}}
\renewcommand{\sl}{\mathfrak{sl}}
\newcommand{\so}{\mathfrak{so}}
\renewcommand{\u}{\mathfrak{u}}
\newcommand{\su}{\mathfrak{su}}
\renewcommand{\sp}{\mathfrak{sp}}
\newcommand{\m}{\mathfrak{m}}
\newcommand{\aff}{\mathfrak{aff}}
\usepackage{mathrsfs}
%\newcommand{\F}{\mathscr{F}}
\newcommand{\X}{\mathfrak{X}}
\renewcommand{\L}{\mathscr{L}}
\renewcommand{\O}{\mathscr{O}}
\newcommand{\M}{\mathcal{M}}

\newcommand{\RN}[1]{%
  \textup{\uppercase\expandafter{\romannumeral#1}}%
}

\newcommand{\1}{\mathbbm{1}}
\newcommand{\J}{\mathbb{J}}
\newcommand{\A}{\mathbb{A}}
\newcommand{\z}{\mathfrak{z}}
\DeclareMathOperator{\Diff}{Diff}
\DeclareMathOperator{\Isom}{Isom}
\DeclareMathOperator{\Ad}{Ad}
\DeclareMathOperator{\ad}{ad}
\DeclareMathOperator{\Aut}{Aut}
\DeclareMathOperator{\aut}{aut}
\DeclareMathOperator{\End}{End}
\DeclareMathOperator{\sym}{Sym}
\DeclareMathOperator{\tr}{tr}
\DeclareMathOperator{\der}{der}
\DeclareMathOperator{\Hom}{Hom}
\DeclareMathOperator{\Mor}{Mor}
\DeclareMathOperator{\Ob}{Ob}
\DeclareMathOperator{\SL}{SL}
\DeclareMathOperator{\MCG}{MCG}
\DeclareMathOperator{\Out}{Out}
\DeclareMathOperator{\Homeo}{Homeo}
\DeclareMathOperator{\Tr}{Tr}
\DeclareMathOperator{\Rep}{Rep}
\DeclareMathOperator{\lambdap}{\boldsymbol{\lambda^+}}
\DeclareMathOperator{\lambdam}{\boldsymbol{\lambda^-}}
\DeclareMathOperator{\Llambda}{\boldsymbol{\Lambda}}
\DeclareMathOperator{\llambda}{\boldsymbol{\lambda}}
\DeclareMathOperator{\GL}{GL}
\DeclareMathOperator{\PGL}{PGL}
\DeclareMathOperator{\Con}{Con}
\renewcommand{\t}{^\intercal}
\DeclareMathOperator{\im}{im}
\DeclareMathOperator{\rk}{rk}
\newcommand{\Span}[1]{\langle #1\rangle}
\newcommand{\then}{\Rightarrow}
\newcommand{\sse}{\Longleftrightarrow}
\newcommand{\Conn}{\mathfrak{Con}^V_\Theta}
\newcommand{\Conc}{\overline{\mathfrak{Con}}^V_\Theta}



\DeclareMathOperator{\ric}{Ric}

\renewcommand{\d}{\mathrm{d}}
\newcommand{\dt}{\left.\frac{\d}{\d t} \right\rvert_{t = 0}}
\newcommand{\dpar}[2]{\frac{\partial#1}{\partial#2}}
\renewcommand{\Re}{\mathfrak{Re\,}}
\renewcommand{\Im}{\mathfrak{Im\,}}

\theoremstyle{definition}
\newtheorem*{defn}{Definition}
\newtheorem{es}{Example}
\newtheorem*{exe}{Esercizio}
\newtheorem{oss}{Remark}
\newtheorem*{notat}{Notation}
\newtheorem*{fact}{Fact}

\theoremstyle{plain}
\newtheorem{thm}{Theorem}[section]
\newtheorem{cor}[thm]{Corollary}
\newtheorem{prop}[thm]{Proposition}
\newtheorem{lemma}[thm]{Lemma}

\newtheorem{theoremA}{Theorem}
\renewcommand*{\thetheoremA}{\Alph{theoremA}}




\begin{document}

%\maketitle
%\vfill
%\begin{center}
%    \includegraphics{UNIRENNES_LOGOnoir.jpg}
%\end{center}
%\vfill

\title{}
\title[Moduli Space of Connections]{Moduli Space of Connections on\\ a Rational Irregular Curve}
\author{Mattia Morbello}

\begin{abstract}
   We study the compactification of the moduli space of a certain class of rank-two irregular connections on the Riemann sphere, presenting one double pole and two simple poles. To explicitly build the compactification, we identify a class of irregular connections with an irregular rational curve and an extra complex parameter. \\As a first step, we will inspire to the Deligne and Mumford's work to compactify the moduli space of such irregular rational curves, introducing the notion of irregular stable nodal curve.\\Secondly, we will understand the behaviour of the extra complex parameter to conclude the compactification.
\end{abstract}


\maketitle


\tableofcontents

%\begin{center}
 %   \includegraphics[width=7cm]{CPX GEOM.jpeg}
%\end{center}

\vfill
\begin{center}
    \includegraphics[width= 2cm]{CHL.png}\hskip 40pt
    \includegraphics[width= 3cm]{UNIRENNES_LOGOnoir_0.png}\hskip 30pt
    \includegraphics[width= 3cm]{IRIS-E_960_V3.png}\\
    \vskip20pt
    \includegraphics[width= 2cm]{logo-rennesmetropole.png}
    \includegraphics[width= 4cm]{LogoCDB.png}
    \includegraphics[width= 2cm]{logoEUR-CAPS-noir.png}\hskip 40pt
    \includegraphics[width= 4cm]{capes-cofecub.jpg}
\end{center}
\thispagestyle{empty}
\restoregeometry






\newpage
%\section*{Introduction1}
%In 1981, Jimbo, Miwa, and Ueno \cite{JIMBO1981306} discovered that the Painlevé equations parametrize the isomonodromic deformations of certain classes of rank-two meromorphic connections over the Riemann sphere with a polar divisor of degree four. These connections have been extensively studied in recent literature, such as in \cite{Ohyama_2006}, \cite{Saitovanderput}, \cite{saitotak}, \cite{geompanv}. Particular attention has been given to the Painlevé VI equation, which parametrizes the isomonodromic deformations of a class of meromorphic connections that only have logarithmic singularities. In particular the braid group action on character varieties for the Painlevé VI equations has been first defined by Dubrovin and Mazzocco for special parameters in \cite{dubmaz}, 

%The aim of this work, part of my PhD project at the University of Rennes under the supervision of Frank Loray, is to construct a compactification of the moduli space of meromorphic connections related to the Painlevé V equation. This is preparatory work for a forthcoming article, where I plan to study the behaviour of the isomonodromic foliation on the boundary components of this compactification, with the goal of later analysing its dynamics and carrying out a study similar to the one by S. Cantat and F. Loray in 2007 \cite{AIF_2009__59_7_2927_0} for the Painlevé VI equation, using the wild character varieties introduced by P. Boalch in \cite{boalch} and studied by E. Paul and J-P. Ramis in \cite{paulramis}.

%In the first section we recall some preliminary notions about meromorphic connections and gauge transformations. The goal is to fix the notations that we use all along the article. \\
%Next, we specialise to the case of our interest, the PV connections, that are, roughly speaking, some rank two meromorphic connections on the Riemann sphere with a double pole and two logarithmic poles. In this section we explicitly describe the link between connections and Riccati foliations. It will be one of the cornerstones for building the moduli space. We end this section with a deep study of confluence of singularities that will be crucial for the compactification. \\In the third section we describe the non-compact moduli space, introducing local coordinates and pointing out its structure of trivial vector bundle based on the moduli space of some particular punctured stable curves, where to points and an order one jet on the Riemann sphere are labelled.\\In the last section we explicitly build the compactification of the moduli space. The first two parts are consecrated to the study of the compactification of the moduli space of these stable nodal curves. Then, we study connections on those curves: they will be the connections arising on the boundary of the moduli space. Finally, the last two parts are devoted to the explicit construction and description of the compactification. 
%\vfill
%\textit{Acknowledgements:}\\
%I would like to thank my advisor Frank Loray for guiding me through this beautiful journey, and Matilde for her mathematical et emotional support.\\
%This work is part of my Phd project, financed by the Labex Centre Herny Lebesgue and Rennes university. I received also a financial support form Collège doctorale de Bretagne, Rennes Metropole, EUR caps and IRIS-E for my mobility in Rio de Janeiro which allowed me to work with Gabriel Calsamiglia, whom I thank for sharing his knowledge on stable nodal curves. 



\section*{Introduction}

%On 8 August 1900, David Hilbert presented in Paris a list of ten open problems. One of them, now known as the 21st Hilbert's problem, is the following:
%\[\textit{Existence of [complex, Fuchsian] linear differential equations having a prescribed monodromy group.}\]
%It has been solved thanks to the works of famous mathematicians as Plemelj (1908, 1964), Birkhoff (1913), Il'yashenko (1966), Dekkers (1978), Kostov (1992) and Bolibrukh (1990, 1992), who solved the problem, showing that for any rank, an irreducible monodromy group can be realised by a Fuchsian system.\\While the discussion around the solution of this problem was very active, several parallel questions and different generalizations have been explored. In particular Deligne (\cite{Del}, 1970) proved the famous Riemann-Hilbert correspondence that is, roughly speaking, a correspondence between connections (up to gauge equivalence) and monodromy data (up to conjugacy).\\
%The Riemann-Hilbert correspondence holds for any complex manifold $X$ and any rank vector bundle endowed with a (flat) connection. Working with rank 2 connections over a Riemann surface, say $\P^1$, offers the possibility to concretely describe the Riemann-Hilbert map, and the moduli spaces that are involved. On one hand it allows a strong geometrical interpretation to this problem, and on the other hand it gives the possibility to explore what happens for irregular connections, that are connections with poles of order greater than one. For logarithmic connections, one can look for instance at the works \cite{geompanv}, \cite{dubmaz} and \cite{cantlor}.

In 1979 K. Okamoto \cite{okamoto} studied the space of initial conditions for the Painlevé equations. Two years later M. Jimbo, T. Miwa, K. Ueno \cite{JIMBO1981306} established a link between the Okamoto's work and a very active research field of that time: the Riemann-Hilbert correspondence, proved by Deligne \cite{Del} in 1970. They interpreted the Okamoto spaces as moduli spaces of certain classes of rank-two meromorphic connections on the Riemann sphere, and demonstrated that the Painlevé equations parametrise the isomonodromic deformations (the fibers of the Riemann-Hilbert map) of such connections. %As a consequence, thanks to the Riemann-Hilbert correspondence, the Okamoto spaces turns out to be isomorphic to the moduli space of monodormies.
%the Painlevé equations parametrise the isomonodromic deformation of certain classes of meromorphic rank 2 connections on the Riemann sphere. In particular, the Riemann-Hilbert correspondence establishes an isomorphism between the Okamoto space and the moduli space of monodromies, and the fibers of the Riemann-Hilbert map are solutions of the Painlevé equation.
\begin{center}
    \includegraphics[width=8cm]{DISEGNI15.jpeg}
\end{center}
Our work lies within this geometrical framework of the irregular Riemann-Hilbert correspondence, following the recent developments in \cite{BOALCH2001137} and \cite{saitovanderput}. In particular, we study the moduli space of a certain class of rank-two meromorphic connection on the Riemann sphere with one double pole and two simple poles, which we refer as \textit{PV connections}. In the literature, they are sometimes called confluent Heun equations. %The goal is to understand their isomonodromic deformations, that are described by the \textit{fifth} Painlevé equation.
\medskip\\
We denote by $\Conn$ the moduli space of PV connections. This space consists of a family of Okamoto spaces indexed by the time variable $t\in \C^*$ of the \textit{fifth} Painlevé equation. The solutions of this equation define a one-dimensional foliation within $\Conn$, transverse to the Okamoto spaces.
Our main goal, presented in Theorem \ref{A}, is to explicitly build and compactify this moduli space, in particular when the time parameter $t$ tends to $0$ or $\infty$. In a future work currently being drafted we use this compactification to study the resulting foliation on the boundary components.

\medskip

The main idea is to identify a generic class of PV connection as a class of irregular rational curves \cite{irrcurv} and an accessory complex number $\hat p$. The irregular curve represents the position of the poles, where the pole of order 2 is represented not only by a point, but by a 1-jet, while the extra complex parameter $\hat p$ is related to the residual spectral datum of the apparent singularity. Those generic connections, for which the apparent singularity stays away from the poles, have been studied in \cite{Diarra}.
The main result is the following:
\begin{theoremA}\label{A}
    Given generic spectral residual data $\Theta=\{\kappa_0,\kappa_1,\kappa_\infty\}$ defined up to integer shifts, the compactified moduli space $\overline\Conn$ of PV connection with spectral residual data $\Theta$ is a three dimensional quasi projective variety birational to the total space of a $\P^1$-bundle $\overline{\O_{\overline\M}(D)}$, where $\overline\M$ is the Deligne-Mumford compactification of the moduli space of irregular rational curves $(\P^1, [0]+2[1]+[\infty]+[q])$ up to Moebius transformations.
\end{theoremA}
The proof of this result comes from different propositions proved in Sections \ref{S1} and \ref{compsect}. In particular, in Section \ref{compsect} we show that the line bundle $\O_{\overline\M}(D)$ is trivial along a special divisor. We then prove that $\overline{\Conn}$ is an open set inside the variety resulting by blowing up some trivialising sections above this divisor. 



In Section \ref{S1}, we show where does the apparent singularity comes from and we build the moduli space $\M$, which turns out to be a quasi-projective variety embedded in $\P^1\times\P^1$. Moreover, we give a very explicit description of the moduli space of those generic PV connections, producing the trivial bundle over $\M$, that is Zariski dense inside the moduli space $\Conn\supseteq\M\times\C$.\\
To compactify $\Conn$, it is natural to first build a compactification of $\M$ and try to extend the trivial bundle over the boundary components. In particular, we adapt the technics that Deligne and Mumford \cite{DelMun} used to compactify $\M_{0,5}$ to our case, explicitly stating the analogy. A surprising result is that $\overline \M$ is a singular surface (while $\overline{\M_{0,5}}$ is smooth) given by the contraction of the unique $(-2)$-curve in a weak Del Pezzo surface of degree 5, with hence strictly canonical singularities. Once $\overline\M$ is constructed we study the limit connections arising on stable nodal irregular curves, finding the new admissible extra complex parameter $\hat p$. This allows us to extend the trivial bundle to $\overline \M$ and define $\overline{\O_{\overline\M}(D)}$. A study of confluence of singularities concludes the proof of the theorem, producing $\overline\Conn$.

\medskip

The text is structured as follows: in Section \ref{secprelnot} we set the notations and we recall the basic definition of the objects involved in the following. A particular attention will be given to the definition of the residual spectral data of a connection. In Section \ref{PVconn} we define the main protagonist of this work, PV connections. We describe the Riccati foliation induced by a connection on the projectivised bundle and, thanks to it, we will give an accurate geometric interpretation of the confluence of (apparent) singularities in this setting. Section \ref{S1} is devoted to describe $\M$ and its link with PV connections. Section \ref{compsect} is the main core of the work, in which a compactification of the moduli space is proposed. 

\vfill
\textit{Acknowledgements:}\\
{\small This work is part of my Phd project, financed by the Centre Henri Lebesgue and Rennes university.} 

I would like to thank my advisor Frank Loray for guiding me through this beautiful journey, Gabriel Calsamiglia for having spent a lot of time discussing about stable nodal curves, and Matilde Maccan for her mathematical and emotional support.

I received also a financial support form Collège doctorale de Bretagne, Rennes Metropole, EUR caps, IRIS-E, and CAPES-COFECUB program for my mobility in Rio de Janeiro.





\newpage

%\section*{Introduction 2}
%On 8 August 1900, David Hilbert presented in Paris a list of ten open problems. One of them, now known as the 21st Hilbert's problem, is the following:
%\[\textit{Existence of [complex, Fuchsian] linear differential equations having a prescribed monodromy group.}\]
%It has been solved thanks to the works of famous mathematicians as Plemelj (1908, 1964), Birkhoff (1913), Il'yashenko (1966), Dekkers (1978), Kostov (1992) and Bolibrukh (1990, 1992), who solved the problem, showing that for any rank, an irreducible monodromy group can be realised by a Fuchsian system.\\While the discussion around the solution of this problem was very active, several parallel questions and different generalizations have been explored. In particular Deligne (\cite{Del}, 1970) proved the famous Riemann-Hilbert correspondence that is, roughly speaking, a correspondence between connections (up to gauge equivalence) and monodromy data (up to conjugacy).\\
%The Riemann-Hilbert correspondence holds for any complex manifold $X$ and any rank vector bundle endowed with a (flat) connection. Working with rank 2 connections over a Riemann surface, say $\P^1$, offers the possibility to concretely describe the Riemann-Hilbert map, and the moduli spaces that are involved. On one hand it allows a strong geometrical interpretation to this problem, and on the other hand it gives the possibility to explore what happens for irregular connections, that are connections with poles of order greater than one. For logarithmic connections, one can look for instance at the works \cite{geompanv}, \cite{dubmaz} and \cite{cantlor}.

%\medskip

%This work fits within this geometrical exploration of the irregular Riemann-Hilbert correspondence in the most elementary and studied case, where the polar divisor has degree four. The goal is indeed to explicitly build the moduli space of a certain class of meromorphic connections, that we will call \textit{PV} connections, find a suitable compactification exploring the limits objects that can arise and studying its birational geometry. It can be seen as a first step for studying the geometric Riemann-Hilbert map in this case. 

%\medskip

%The main result is the following:
%\begin{theoremA}
%    Given generic spectral residual data $\Theta=\{\kappa_0,\kappa_1,\kappa_\infty\}$ defined up to integer shifts, the compactified moduli space $\overline\Conn$ of PV connection with spectral residual data $\Theta$ is a three dimensional quasi projective variety birational to the total space of a $\P^1$-bundle $\overline{\O_{\overline\M}(D)}$, where $\overline\M$ is the Deligne-Mumford compactification of the moduli space of irregular rational curves $(\P^1, [0]+2[1]+[\infty]+[q])$ up to Moebius transformations.
%\end{theoremA}
%The proof of this result comes from different propositions proved in Sections \ref{S1} and \ref{compsect}. The main idea is to identify a generic class of PV connection as a class of irregular rational curves and an accessory complex number $\hat p$. The irregular curve represents the position of the poles and of an extra apparent singularity, while the extra complex parameter is related the residual spectral datum of the apparent singularity. Those generic connection, for which the apparent singularity stays away from the poles, have been studied in \cite{Diarra}. In Section \ref{S1}, we build the moduli space $\M$, which turns out to be a quasi-projective variety embedded in $\P^1\times\P^1$. Moreover, we give a very explicit description of the moduli space of those generic PV connections, producing the trivial bundle over $\M$, that is Zariski dense inside the moduli space $\Conn\supseteq\M\times\C$.\\
%It sounds then natural, to compactify $\Conn$, to first build a compactification of $\M$ and try to extend the trivial bundle over the boundary components. In particular, we adapt the technics that Deligne and Mumford \cite{DelMun} used to compactify $\M_{0,5}$ to our case, explicitly stating the analogy. A surprising result is that $\overline \M$ is a singular surface (while $\overline{\M_{0,5}}$ is smooth) given by the contraction of the unique $(-2)$-curve in a weak Del Pezzo surface of degree 5, with hence strictly canonical singularities. Once $\overline\M$ is constructed we study the limit connections arising on stable nodal irregular curves, finding the new admissible extra complex parameter $\hat p$. This allows us to extend the trivial bundle to $\overline \M$ and define $\overline{\O_{\overline\M}(D)}$. A study of confluence of singularities concludes the proof of the theorem, producing $\overline\Conn$.

%\medskip

%The text is structured as follows: in Section \ref{secprelnot} we set the notations and we recall the basic definition of the objects involved in the following. A particular attention will be given to the definition of the residual spectral data of a connection. In Section \ref{PVconn} we define the main protagonist of this work, PV connections. We describe the Riccati foliation induced by a connection on the projectivised bundle and, thanks to it, we will give an accurate geometric interpretation of the confluence of (apparent) singularities in this setting. Section \ref{S1} is devoted to describe $\M$ and its link with PV connections. Section \ref{compsect} is the main core of the work, in which a compactification of the moduli space is proposed. 

%\vfill
%\textit{Acknowledgements:}\\
%{\small This work is part of my Phd project, financed by the Centre Henri Lebesgue and Rennes university.} 

%I would like to thank my advisor Frank Loray for guiding me through this beautiful journey, Gabriel Calsamiglia for having spent a lot of time discussing about stable nodal curves, and Matilde Maccan for her mathematical and emotional support.

%I received also a financial support form Collège doctorale de Bretagne, Rennes Metropole, EUR caps, IRIS-E, and CAPES-COFECUB program for my mobility in Rio de Janeiro.


%\newpage




\section{Preliminary Notions}\label{secprelnot}
\subsection{Linear Meromorphic Connections}\label{intro1}
%A Riemann-Hilbert map is a transcendental map between two algebraic varieties: the moduli space of a certain class of connections and a corresponding character variety. This map associates with any connection its monodromy representation.\\
We are interested in the study of a special class of linear meromorphic connections, of PV type, but let us start by introducing what a meromorphic connection is in a general setting. 
\begin{defn}
    Let $X$ be a complex manifold. A linear meromorphic connection is a triple $(\nabla, E, D)$, where 
    \begin{itemize}
    \item $E\to X$ is a holomorphic vector bundle, 
        \item $D$ is an effective divisor of $X$,
        \item $\nabla$ is a $\C$-linear morphism of sheaves $\nabla\colon \mathcal E \to \mathcal E\otimes \Omega^1_X(D),$
    satisfying the Leibniz rule given by the $\O_X$-module structure of the sheaf $\mathcal E$ of holomorphic sections: $\nabla (f\cdot \sigma)=df\cdot\sigma+f\cdot\nabla\sigma$.
    \end{itemize} 
    The divisor $D$ is called \emph{polar divisor} and it describes the order and the position of the poles that the connection must have. We recall the notation $\Omega^1_X(D):=\Omega^1_X\otimes\O_X(D)$.
\end{defn}
\begin{oss}
    If $D=\emptyset$, we get a holomorphic connection.
\end{oss}
Even if some of the notions we introduce hold for any complex manifold and any vector bundle, for the following of the article we always suppose that $X=\P^1$ and $\mathrm{rk}E=2$. This significantly simplify notations. 
\begin{defn}
    A point $a\in \P^1$ is called a \textit{singularity} (or a \textit{pole}) for the connection $(E,\nabla, D)$  if $a\in |D|$. Its \textit{Poincaré rank} is the integer number $\deg (D_{|a})-1$. 
\end{defn}
\begin{defn}
    A singularity is said \textit{logarithmic} if its Poincaré rank is zero. Otherwise, it is said \textit{regular} if it can be reduced to a logarithmic singularity and \textit{irregular} if not.
\end{defn}
Let us study the simplest example of meromorphic connection.
\begin{es}[Euler System]\label{euler}
     Let us consider the connection $(\nabla, \O\oplus\O, [0]+[\infty])$ expressed in the local chart $U_0\subseteq\P^1$
     \[dY(x)=\Omega_0(x)\cdot Y(x), \;\;\;\;\;\;\Omega_0(x)=A_1^{(0)}\frac{dx}{x}, \;\;\;\;\;\;\;\;\; A_1^{(0)}=(a_{ij})\in \mathfrak{gl}_2(\C).\]
     It has polar divisor $D=[0]+[\infty]$, therefore both are logarithmic singularities. It is clear for $x=0$, while for $x=\infty$, we should look at the other chart $U_\infty\subseteq\P^1$: the system can be expressed in the coordinate $z=1/x$,  
     \[A_1^{(0)}\frac{dx}{x}=A_1^{(0)}zd(1/z)=A_1^{(0)}z(-1/z^2)dz=-A_1^{(0)}\frac{dz}{z}:=A_1^{(\infty)}\frac{dz}{z}\]
     and hence also $\infty$ is a pole of order 1 and then it is a logarithmic singularity with $A_1^{(\infty)}=-A_1^{(0)}$.
\end{es}
We can generalise a bit more this example via the following proposition.
\begin{prop}
    Any meromorphic connection $(\nabla, \O\oplus\O, D)$ on the trivial bundle can be globally expressed as 
    \[\nabla = d+\Omega\;\;\;\;\;\text{for some}\;\;\;\;\;\Omega\in \gl_2\big(\Omega^1(D)\big).\]
\end{prop}
Roughly speaking, we can see any connection on the trivial bundle as a rank 2 system of ODEs over the projective line. Indeed, if we write
\[\Omega=\begin{pmatrix}\omega_{1,1}&\omega_{1,2}\\\omega_{2,1}&\omega_{2,2}\end{pmatrix}\;\;\;\;\text{then} \;\;\;\;\nabla Y=(d+\Omega)Y=\begin{pmatrix}dy_1\\dy_2\end{pmatrix}+\begin{pmatrix}\omega_{1,1}&\omega_{1,2}\\\omega_{2,1}&\omega_{2,2}\end{pmatrix}\begin{pmatrix}y_1\\y_2\end{pmatrix}=0.\]
We wonder what happens when the vector bundle $E$ is not globally trivial. We still know that each holomorphic vector bundle is locally trivial on any contractible open set of $\P^1$, and then it is sufficient to consider the open cover $\{U_0, U_\infty\}$. In this setting, a connection is just the datum of two different system of ODEs on $\C$, that express via the formula $\nabla_0=d+\Omega_0$ and $\nabla_\infty=d+\Omega_\infty$, and the two connection matrices are related in the overlap $U_0\cap U_\infty$ by the formula
\[\Omega_0=g_{0,\infty}^{-1}\cdot\Omega_\infty\cdot g_{0,\infty}+g_{0,\infty}^{-1}\cdot dg_{0,\infty} \]
where $g_{0,\infty}$ is the cocycle of the vector bundle $E$. 
\begin{oss}
    One can easily prove this relation by noticing that if $Y_0$ is a local section in $U_0$ satisfying $\nabla_0Y_0=0$, then the section $Y_\infty=g_{0,\infty}Y_0$ defined in $U_0\cap U_\infty$ must satisfy $\nabla_\infty Y_\infty=0$.
\end{oss}
Sometimes, we will be interested in the behaviour of the connection in a neighbourhood of a singularity: we can then shrink a trivialising open set for $E$ in order to get a simply connected open set $U$ that contains at most one singularity at, for instance, $x=a$. On this open set $U$, the connection matrix has a very simple and explicit description: 
\[\text{if }\;\; E_{|U}\cong U\times\C^2 \;\;\text{ and }\;\;|D|\cap U=\{a\},\;\;\;\;\text{ then } \;\;\nabla_{U}=d+\Omega_U,\]
for some $\Omega_U\in \gl\big(\Omega^1(n[a])\big)$ that writes:
\[\Omega_U:=\sum_{k=1}^n A_k^{(a)}\frac{dx}{(x-a)^k} +\text{holomorphic terms}\]
where $A_k^{(a)}\in \gl_2(\C)$ are constant matrices and $n$ is the order of the pole in $x=a$. 
\begin{defn}
    We call $\sum_{k=1}^n A_k^{(a)}\frac{dx}{(x-a)^k}$ the \textit{principal part} of a connection $\nabla$ around the singularity $a\in \P^1$.
\end{defn}


\subsection{Horizontal Sections}
\begin{defn}
    A local section $Y$ satisfying $\nabla Y=0$ is called \textit{parallel} or \textit{horizontal} section for the connection $(\nabla,E,D)$. Equivalently, we say that $Y$ is a solution of the differential equation $\nabla=0$
\end{defn}
We are interested in studying the behaviour of such sections in a neighbourhood of a singularity $x=a$. 
\begin{thm}
    Let X be a Riemann surface and $(\nabla, E,D)$ a connection on it. Let $U\subseteq X\setminus \mathrm{Supp}(D)$ a simply connected open set trivialising $E$. Then:
    \begin{itemize}
        \item There exists a fundamental matrix of solutions $Y(x)$ defined everywhere over $U$.
        \item Any other matrix of solutions $Y'(x)$ differs from $Y(x)$ by a constant matrix factor $Y'(x)=Y(x)\cdot C$.
        \item For any $x_0\in U$, the evaluation map $Y(\cdot)\mapsto Y(x_0)$ is an isomorphism between the linear space of solutions of $\nabla_{|U}Y=0$ and $\C^2$.
    \end{itemize}
\end{thm}
\begin{proof}
    Theorem 14.1 and remark 19.2 of \cite{Ily}
\end{proof}
We can specify the theorem via the following facts coming from \cite[15.10, 20.20, 21.3]{Ily}.
\begin{fact}
    If $x=a$ is a logarithmic singularity, then the local solutions of $\nabla =0$ are meromorphic on $U$ in $x=a$. Moreover, a basis of solution can be expressed as $Y(x)=H(x)x^A$ where $A$ is a constant matrix and $H$ is a meromorphic invertible matrix function. 
\end{fact}
\begin{fact}
    If $x=a$ is an irregular singularity, then the local solutions of $\nabla =0$ can be expressed as $Y(x)=F(x)x^A$, where $A$ is a constant matrix and $F$ is a matrix function single valued and holomorphically invertible in $U^*$, but eventually having an essential singularity in $x=a$. 
\end{fact}
The description of solutions around an irregular singularity is much more complicated since the so called Stokes phenomena appear. In fact, for simplicity, assume that $U\cong \D$ is isomorphic to the unitary complex disk centred in the singularity. We cannot express a solution as a holomorphic function in $\D^*:=\D\setminus\{0\}$, but we are only allowed to define it on angular sectors that cover $\D^*$, whose amplitude depends on the order of the singularity. For example, in the case of an order two singularity, we have two sectors:
\[V^k:=\Big\{x\in\D^*\;\;\Big|\;\;\big|\arg(x)-(\theta+k\pi)\big|<\pi\Big\}\;\;\; \text{ for }\;\; k=0,1\]
for a $\theta$ that depends on the spectral data of $\Omega_\D$, and on $V^k$ the solutions are holomorphic. Moreover, a solution $Y(x)$ will grow exponentially fast on $V^0$ and decreases faster than any finite power $|x|^{N>0}$ in $V^1$. For more details, see paragraph 21.1 of \cite{Ily} and section 1.3 of \cite{RHrank2}.



\subsection{Equivalent Connections and Residual Spectral Data}\label{resspecd}
The goal of this work is to build the moduli space of PV type connections, that will be introduced in the next section. A moduli space is a topological space such that any point represents the equivalence class of one object under a suitable equivalence relation. Here we define the equivalence relation that we will use to identify different meromorphic connections. 


\textbf{Holomorphic Gauge Transformations.}
\begin{defn}
    Two connections $(\nabla, E, D)$ and $(\nabla', E, D)$ are said \textit{gauge equivalent} if there exists a holomorphic automorphism $\Phi\in \mathrm{Aut}(E)$ that sends $\nabla$-horizontal sections into $\nabla'$-horizontal sections. Equivalently, we have that 
    \[\nabla=\Phi^*\nabla'.\]
\end{defn}
There is a local description of the gauge action: if $\nabla_{|U}=d+\Omega$ and $\nabla'_{|U}=d+\Omega'$, for each trivialising set $U$, it holds that
\[\Omega=\Phi^{-1}\Omega'\Phi+\Phi^{-1}d\Phi.\]
%The set of gauge transformations over a vector bundle $E$ forms a group called the gauge group, often denoted by $\mathcal G(E)$. 
\begin{es}\label{gaugeloc}
    Let us consider a connection $(\nabla, E, D)$ expressed locally in $U$ around a double pole say centred in $x=0$. Its connection matrix has expression has expression
    \[\Omega=A^{(0)}_2\frac{dx}{x^2}+A^{(0)}_1\frac{dx}{x}+holomorphic \;terms.\]
    The first terms of a holomorphic gauge transformation are, for instance, $\Phi_{|U}=M_0+xM_1$, with $M_i\in GL_2(\C)$. The gauge action is
    \[\Big(M_0^{-1}A_2^{(0)}M_0\Big)\frac{dx}{x^2}+\Big(M_0^{-1}A_1^{(0)}M_0+M_0^{-1}A_2^{(0)}M_1-M_0^{-1}M_1M_0^{-1}A_2^{(0)}M_0\Big)\frac{dx}{x}+holomorphic \;terms.\]
\end{es}
We can state more generally that locally a holomorphic gauge transformation acts by conjugation on the highest term of a linear meromorphic connection. We finally remark that we can always diagonalise the principal part of a connection via a finite number of holomorphic gauge transformations.



\textbf{Meromorphic Gauge Transformations and Elementary Transformations.}
Holomorphic gauge transformations allow us to understand when two connections on the same bundle $E$ are equivalent. We would like to have a more general definition, allowing us to establish an equivalence of connections over different bundles of different degree. We should hence look into some birational bundle automorphism.
\begin{defn}
    A \textit{meromorphic gauge transformation} $\Psi\colon(\nabla,E,D)\to (\nabla',E',D')$ is a birational morphism of bundles $\Psi\colon E\to E'$ such that $\Psi^*\nabla'=\nabla$.
\end{defn}
\begin{oss}
    We do not define in this paper the notion of monodromy of a meromorphic connection, but it is important to know that meromorphic gauge transformations are exactly the monodromy preserving transformations.
\end{oss}
A particular and interesting class of meromorphic gauge transformations are elementary transformations. They are characterised by having a single pole or zero of order one.
\begin{defn}
    Let $E$ be a rank 2 vector bundle and $l$ be a line in $E_{x_0}$. We denote via $(E,l)$ the parabolic vector bundle. The \textit{elementary transformations} $\mathrm{elm}^\pm_{l,x_0}$ are meromorphic gauge transformations 
   %\[\mathrm{elm}^+_{x_0,l_0}\colon E \dashedrightarrow E^+\;\;\;\text{ and }\;\;\;\mathrm{elm}^-_{x_0,l_0}\]
   \begin{center}
       % https://tikzcd.yichuanshen.de/#N4Igdg9gJgpgziAXAbVABwnAlgFyxMJZABgBpiBdUkANwEMAbAVxiRBgYFsA9AagH1gAD35kGogL4AdKQGMIDAgAIAoiAml0mXPkIoAjOSq1GLNir7rNIDNjwEiAZiPV6zVonZd+3ALTDRUnFiaTkFZTUNLTtdIgAWFxN3cz8raJ0HFAAmRLczTxkcGCEcYCU6MCglCXVjGCgAc3giUAAzACcITiQyEBwIJEMQBiwwDxAoCBwiqDSQDq6kHL6BxGdh0fHJ6fraiSA
\begin{tikzcd}
{\mathrm{elm}^+_{l,x_0}\colon (E,l)} \to  (E^+,l^+) &\text{ and } & {\mathrm{elm}^-_{l,x_0}\colon (E,l)} \to  (E^-,l^-).
\end{tikzcd}
   \end{center}
   If we choose a local trivialisation $E_{|U}\cong U\times\C^2$ based in $x_0=0$ and such that $l=\C\cdot\begin{pmatrix}1\\0\end{pmatrix}$, then the elementary transformations have this local expression
   \begin{equation}\label{elemtransexpr}
   \mathrm{elm}^+_{l,x_0}=\begin{pmatrix}x&0\\0&1   \end{pmatrix}\colon \C^2\to \C^2\;\;\;\;\; \text{ and }\;\;\;\;\;\mathrm{elm}^-_{l,x_0}=\begin{pmatrix}1&0\\0&\frac{1}{x}  \end{pmatrix}\colon \C^2\to \C^2
   \end{equation}
\end{defn}
%The following proposition describes how $E$ and $D$ are modified after the action of a gauge transformation.
\begin{prop}
    Let us consider a point $x_0\in U\subseteq \P^1$ and the elementary transformation $\mathrm{elm}^\pm_{l,x_0}$. It holds that $\det(E^\pm)=\det(E)\otimes\O(\pm[x_0])$ and therefore $\deg E^\pm=\deg E\pm1.$
\end{prop}
\begin{proof}
    Section 6 of \cite{Machu2007}.
\end{proof}
%\begin{proof}
%    Let us denote by $\{g_{\alpha\beta}\}$ the cocycle defining $E$. Then $\det(E)$ is the line bundle induced by the cocycle $\{\det(g_{\alpha\beta})\}$, moreover we recall that the cocycle of a tensor product of bundles is the product of the cocycles. In particular we have that $\det(E^\pm)$ is induced by the cocycle 
    %\[\det(E^\pm)\longleftrightarrow\{\det(g^\pm_{\alpha\beta})\}=\{\det(\Psi^{\pm-1}_{U_\alpha} \circ g_{\alpha\beta}\circ \Psi^\pm_{U_\beta})\}=\{\det(g_{\alpha\beta})\cdot\det(\Psi^{\pm-1}_{U_\alpha}\cdot \Psi^{\pm}_{U_\beta})\}\longleftrightarrow\det(E)\otimes L,\]
    %and it remains to investigate $L$. We know $\Psi^\pm_{U_\alpha}$ is the identity for any $U_\alpha\neq U$, therefore $\det(\Psi^\pm_{U_\alpha}\cdot \Psi^{\pm-1}_{U_\beta})=1,x^{\pm1}$, in particular we are saying that $L$ is generated by a unique section that can admit a 0 or a pole in $x_0$, this implies, by definition, that $L\cong \O(\pm[x_0])$ and the thesis follows.
%\end{proof}
In order to better understand how the elementary transformations acts on the connection matrix, let us consider the following example.
\begin{es}
    Let us consider the following connection on the trivial bundle given by $(d+\Omega, \O\oplus\O, D)$ and a point $q\notin |D|$. Then
    \[\mathrm{elm}^-_{q}(d+\Omega, \O\oplus\O, D)=(d+\tilde\Omega, \O\oplus\O(1), D-[q]).\]
    Indeed here we can globally find a choice of coordinates on $\O\oplus\O$ to express the elementary transformation as a matrix, and it acts in the standard way on $\Omega$: 
    \begin{align*}
        \tilde\Omega=&\begin{pmatrix}1&0\\0&\frac{1}{x-q}  \end{pmatrix}^{-1}\begin{pmatrix}\omega_{1,1}&\omega_{1,2}\\\omega_{1,2}&\omega_{1,2} \end{pmatrix}\begin{pmatrix}1&0\\0&\frac{1}{x-q}  \end{pmatrix}+\begin{pmatrix}1&0\\0&\frac{1}{x-q}  \end{pmatrix}^{-1}\begin{pmatrix}0&0\\0&-\frac{dx}{(x-q)^2}  \end{pmatrix}=\\&\begin{pmatrix}\omega_{1,1}&(x-q)\omega_{1,2}\\\frac{1}{(x-q)}\omega_{2,1}&\omega_{1,2}-\frac{dx}{x-q} \end{pmatrix}.
    \end{align*}
\end{es}
%Any two meromorphic gauge transformations cannot be composed in general; in other words, the set of meromorphic gauge transformations does not form a group. %because we can not compose in general two meromorphic gauge transformations. 
\begin{oss}
    When we refer to the isomorphism class of a connection, we will implicitly mean under the action of meromorphic gauge transformations. Otherwise we will specify that we consider only holomorphic gauge equivalence.
\end{oss}
\textbf{Residual Spectral Data.}
Let us end this section by introducing the residual spectral data of a linear meromorphic connection. They will play a central role in the construction of the moduli space and in the study of its geometry.
\begin{defn}
We call $A_1^{(a)}=\mathrm{Res}_{x=a}\Omega_U$ the \textit{residual matrix} of the connection $(\nabla,E,D)$ at $x=a$. Sometimes we denote it just as $\mathrm{Res}_{x=a}\nabla$.
\end{defn}
We are interested in studying the spectrum of the residual matrix. By Example \ref{gaugeloc} we see that it is in general not invariant for gauge transformations for poles of order higher than one.
\begin{defn}
    We then refer to the \textit{residual spectral data} of a connection $(\nabla, E, D)$ as the data $\cup_{a\in |D|}\mathrm{spec}\big(\mathrm{Res}_{x=a}\widetilde\nabla\big):=\cup_{a\in |D|}\{\kappa_a^+, \kappa_a^-\}$, where $\widetilde\nabla$ is holomorphic gauge equivalent to $\nabla$ and it has diagonal principal part.
\end{defn}
We would like to have a formula to find the residual spectral data of a connection, at least for a pole of order 2. Let us consider the trace and the determinant of a connection matrix over a trivialising set centred in the singularity $a$ and let us call $\lambda, \mu\in \C((x-a))$. We have that 
\[\mathrm{tr}\Omega=\frac{a_2}{(x-a)^2}+\frac{a_1}{x-a}+holom=\lambda+\mu\;\;\;\text{ and }\;\;\;\det\Omega=\frac{b_4}{(x-a)^4}+\frac{b_3}{(x-a)^3}+\dots=\lambda\mu.\]
Moreover we have that $(\lambda+\mu)^2-4\lambda\mu=(\lambda-\mu)^2$. If we set $R(\Omega):=(\mathrm{tr}\Omega)^2-4\det\Omega$, we have that 
\[R(\Omega)=\frac{c_4}{(x-a)^4}+\frac{c_3}{(x-a)^3}+\dots=(\lambda-\mu)^2=\Big(\frac{\alpha_2}{(x-a)^2}+\frac{\alpha_1}{x-a}\Big)^2+\dots,\]
and then, since we are interested only in the residue:
\[\alpha_1=-\frac{c_3}{2\sqrt{c_4}}=\kappa_a^+-\kappa_a^-.\]
\begin{oss}\label{fuchs}
    The residual spectral data must satisfy the Fuchs' relation
    \[\sum_{a\in |D|}\kappa_a^++ \kappa_a^-=-\deg E\]
\end{oss}
\begin{defn}
    The residual spectral data of a connection are called \textit{generic} if the condition
    \[\sum_{a\in |D|}\kappa_a^{\epsilon_i}\notin\Z\]
    holds for all choices of $\epsilon_i\in\{+,-\}$.
\end{defn}
We always assume that the residual spectral data are generic. This implies that the connection is irreducible, that is, there does not exist a $\nabla$-invariant line bundle $L\subseteq E$.  
\begin{oss}
    Meromorphic gauge transformations may shift some residual data by an integer. The residual spectral data then satisfy the new Fuchs' relation \ref{fuchs}. This implies that the residual spectral data of an isomorphism class of a connection are well defined only up to integers. Since when some residual data are zero some pathological behaviours arise, it also explains why we asked the generic condition to be satisfied.
\end{oss}


%How to explain that there is a minimal divisor D??\\
%is the proposition false if x0 is in D??\\
%\begin{prop}
 %   Any meromorphic connection $(\nabla, E, D)$, with $\deg D=4$ and generic residual spectral data, is such that
 %   \begin{itemize}
 %       \item $E\cong\O(k)\oplus\O(k)$ or $\O(k-1)\oplus\O(k+1)$ if $\deg E = 2k$,
 %       \item $E\cong\O(k)\oplus\O(k+1)$ if $d=2k+1$.
 %   \end{itemize}
 %   In particular, if $\deg E=1$, then $E$ needs to be the bundle $\O\oplus\O(1)$.
%\end{prop}
%\begin{proof}
%    It is a consequence of the Fuchs' relation and the irreducibility of the connection (implied by the generic choice of the residual spectral data). See Proposition 2.7.4 of \cite{geompanv} and Section 3.1 of \cite{loray:hal-01133705}.
%\end{proof}
\begin{prop}\label{corO1}
    Any irreducible meromorphic connection $(\nabla, E, D)$ with $\deg D=4$ is meromorphic gauge equivalent to a connection $(\nabla', E', D')$ where $E'\cong\O(-1)\oplus\O$.
\end{prop}
\begin{proof}
     It is a consequence of the Fuchs' relation and the irreducibility of the connection (implied by the generic choice of the residual spectral data) as shown in Proposition 2.7.4 of \cite{geompanv}. An explicit description of such connections is given in Section 4.1 of \cite{Ohyama_2006}.
\end{proof}






\section{Connections of PV type}\label{PVconn}
We introduce now the main character of this work: connections of PV type. They are the first example of meromorphic connections with polar divisor of degree four, which present an irregular singularity. They are the first degeneracy of the very famous and studied Painlevé VI type connections, see for instance \cite{geompanv}. 
\begin{defn}
     A meromorphic connection $(\nabla, E, D)$ over $\P^1$ is of \textit{Painlevé type} if: $\mathrm{rank} E=2$, and the minimal polar divisor within its meromorphic gauge equivalence class has degree four.
\end{defn}
\begin{oss}
    A complete classification of Painlevé type connections can be found in Section 1.4 of \cite{Ohyama_2006}.
\end{oss}
\begin{defn}
    A Painlevé type connection is of \textit{PV type} if the minimal divisor within its meromorphic gauge equivalence class is equivalent to $[0]+2[1]+[\infty]$.
\end{defn}
    Painlevé type connections with fixed residual data are the easiest example of meromorphic connections with a non trivial moduli space (w.r.t. meromorphic gauge equivalence). Indeed, the dimension of the moduli space depends on the degree of the polar divisor and on the rank of the vector bundle. If we consider rank 2 connections, we need at least a polar divisor of degree four to have a moduli space of positive dimension. If we ask the residual spectral data to be fixed (up to integer shifts), the Euler system we presented in the Example \ref{euler} is indeed unique up to gauge equivalence (and up to a Moebius transformation that sends the two poles respectively to 0 and $\infty$). When the minimal degree of the polar divisor is three, we have the so called hypergeometric systems, and also in this case the moduli space (fixed spectral data up to integer shifts) corresponds to a point (see Section 1.3.5 of \cite{geompanv}).  




\subsection{Riccati Foliation}\label{riccati}
Any rank 2 meromorphic connection $(\nabla, E, D)$ over $\P^1$ induces a singular Riccati foliation on the $\P^1$-bundle $\P(E)$. 
\begin{oss}
Given a rank 2 vector bundle $E$ we can define the $\P^1$-bundle $\P(E)$. The fibers of $\P(E)$ are just the projectivisation of the fibers of $E$. The cocycle of $E$, namely $g_{ij}\colon E_{|U_i}\to E_{|U_j}$, descends to the quotient, since the action is linear. Finally, given an element $a\in \P^1$, a point $p=[1:p]\in F_a=\P(E_a)$ represents the vector $(1,p)\in E_a$.
\end{oss}
Since $\nabla$ is a $\C$-linear morphism, it descends to the projectivisation. Indeed, in any open set $U$ trivialising the connection, we have an expression in coordinates $\nabla=d+\Omega_U$ and let $Y=(y_1,y_2)$ an horizontal section. We know that $\P(E_{|U})\cong U\times\P^1$, and let $[1:y]=[y_1:y_2]$ be the coordinate on the fibers. Then:
\[d\begin{pmatrix}y_1\\y_2\end{pmatrix}+\begin{pmatrix}\omega_{1,1}&\omega_{1,2}\\\omega_{1,2}&\omega_{1,2} \end{pmatrix}\begin{pmatrix}y_1\\y_2\end{pmatrix}=0\;\;\;\implies \;\;\;dy=d\bigg(\frac{y_2}{y_1}\bigg)=\frac{dy_2\cdot y_1-dy_1\cdot y_2}{y_1^2}=\omega_{1,2}y^2+(\omega_{1,1}-\omega_{2,2})y-\omega_{2,1},\]
and the solutions of this Riccati differential equation define a Riccati foliation on $\P(E_{|U})$. \\As one can expect, the following result holds.
\begin{prop}
    Holomorphic gauge equivalent connections induce isomorphic Riccati foliation. 
\end{prop}
\begin{proof}
    Section 1.4.3 of \cite{geompanv}.
\end{proof}
%\begin{prop}
%    Two meromorphic connections $(\nabla, E, D)$ and $(\nabla', E', D')$ induce the same Riccati foliation if, and only if, $(\nabla, E, D)=(L,\zeta, D'')\otimes(\nabla', E', D')$ for some rank 1 meromorphic connection $(L,\zeta, D'')$.
%\end{prop}
%\begin{proof}
%    It holds that $(\nabla, E, D)=(L,\zeta, D'')\otimes(\nabla', E', D')$ if, and only if, $\P(E)=\P(E')$ and the two connections matrix are related by $\Omega= \Omega'+\zeta I$. In particular the two induced Riccati foliations are given by the same formula since
%    \[dy=\omega_{1,2}y^2+(\omega_{1,1}-\omega_{2,2})y-\omega_{2,1}=\omega_{1,2}'y^2+(\omega_{1,1}'+\cancel\zeta-\omega_{2,2}'-\cancel\zeta)y-\omega_{2,1}'.\]
%\end{proof}
%\begin{prop}
%     $(\nabla, E, D)=(L,\zeta, D'')\otimes(\nabla', E', D')$ if, and only if, there exist a composition of elementary transformation $\psi$ such that $\psi(\nabla, E, D)=(\nabla', E', D')$
%\end{prop}
%\begin{proof}
%\end{proof}
A fundamental consequence of this result and Proposition \ref{corO1} is that any meromorphic gauge class of Painlevé V connection can be represented as a Riccati foliation over $\F_1:=\P(\O(-1)\oplus\O)$, that is the so called first Hirzebruch surface. We recall that Hirzebruch surfaces $\F_k$ are smooth algebraic ruled surfaces birational to $\P^1\times\P^1$. Additionally, they can be interpreted as the compactification of the total space of the line bundle $\O(k)$, obtained by adding a so-called "section at infinity". Hirzebruch surfaces have a very rich birational geometry, the interested reader will find more details in \cite{Beauville_1996}. A geometrical property that we will use in the following is that in each Hirzebruch surface $\F_k$ there is a unique curve of self intersection $-k$: it corresponds to the line $\P(\O)\subseteq\P(\O\oplus\O(k))$, which is exactly the "section at infinity" we mentioned before to compactify the total space of $\O(k)$. \\
For the following, it is necessary to understand how elementary transformations act on the Riccati foliation (see also Section 1.5 of \cite{Heu2019FlatRT}).\\
Consider a point $x_0\in U\subseteq\P^1$ and $\{e_1,e_2\}$ a basis of $E_{|U}\cong U\times \C$ such that the elementary transformations have the coordinate expression as in the formula \ref{elemtransexpr}. If we consider a section $Y=(y_1,y_2)$, then
\[\mathrm{elm}^+_{x_0}(Y)=\begin{pmatrix}x-x_0&0\\0&1\end{pmatrix}\begin{pmatrix}y_1\\y_2\end{pmatrix}=\begin{pmatrix}(x-x_0)y_1\\y_2\end{pmatrix}.\]
In the projectivisation, we have hence that 
\[\mathrm{elm}^+_{x_0}[y_1:y_2]=[(x-x_0)y_1:y_2],\]
and we see that on the fiber above $x_0$, any local section is therefore forced to pass through the point $[0:1]$. In fact, an elementary transformation is nothing more than a flip.
\begin{center}
    \includegraphics[width=12cm]{image28.jpeg}
\end{center}
%Noticing that the projectivisation commutes with any morphism of bundles, we can easily deduce that
%\[\mathrm{elm}^\pm_{x_0}\F_k=\F_{k\pm1}.\]
Let us treat in detail the case of PV type connections. Let $(\nabla, \O(-1)\oplus\O, D)$ be a PV type connection. The same reasoning used in Section 5 of \cite{Loray_2016} for a Painlevé VI type connection, shows that the Riccati foliation induced on $\F_1$ has exactly one tangency point with the section at infinity. Let us call $q$ the projection over $\P^1$ of that point. We can then apply an elementary transformation (flip) based in the tangency point. All the leaves of the Riccati foliation need then to pass through a certain point $(q,[1,p])\in\F_2$, rising a radial singularity.
\begin{center}
    \includegraphics[width=15cm]{image282.png}
    \captionof{figure}{}\label{elmq}
\end{center}
In particular the position of the point $[1:p]$ in the fiber above $q$ depends on the second order of tangency of the foliation with the infinity section. %We introduce a very general and important statement:
%\begin{thm}\label{Riccati}
%    There exists a one to one correspondence between rank 2 connections over $E=\O\oplus\O(1)$ and Riccati foliations over the second Hirzebruch surface $\F_2$.
%\end{thm}
%\begin{proof}[Sketch of proof.]
 %Remark 2.1 \cite{ricfol}, Remark 4.3 \cite{ricfol2}, Section 1.3 \cite{Heu2019FlatRT}, Section 3.1 \cite{loray:hal-01133705}.
%\end{proof}
%In what follows, we use this correspondence only in restriction to PV type connections.





\subsection{Normal Form}\label{secnormform}
The goal of this section is to use the action of meromorphic gauge transformations to find a suitable representative in each isomorphism class of connections of PV type. This enlightens the link with the Riccati foliation and then becomes the cornerstone of the construction of the moduli space. \\In a recent paper, Diarra and Loray \cite{Diarra} proved that we can uniquely choose such a representative by considering a connection in $E=\O\oplus\O(2)\to \P^1$ of the following form:
\begin{align*}
    \nabla_{|0}= & \,d+\Omega_0=\\&d+\begin{pmatrix}0&1\\0&t\end{pmatrix}\frac{dx}{(x-1)^2}+\begin{pmatrix}0&-1\\0&-\kappa_1\end{pmatrix}\frac{dx}{x-1}+\begin{pmatrix}0&1\\0&-\kappa_0\end{pmatrix}\frac{dx}{x}+\begin{pmatrix}0&0\\-\rho^{V}&0\end{pmatrix}xdx+\begin{pmatrix}0&0\\\hat p&-1\end{pmatrix}\frac{dx}{x-q}+\begin{pmatrix}0&0\\\hat K&0\end{pmatrix}dx,
\end{align*}
where 
\begin{itemize}
    \item $\kappa_0, \kappa_1, \kappa_\infty$ are generic residual spectral data, as explained in Section \ref{intro1};
    \item $\rho^V=\frac{(\kappa_0+\kappa_1-1)^2}{4}-\frac{\kappa_\infty^2}{4}$;
    \item $\hat K=\frac{\hat p^{2}}{q \left(q -1\right)^{2}}+\frac{\left(\mathit{\kappa_0} \left(q -1\right)^{2}+(\kappa_1-1)  q \left(q -1\right)-t q \right) \hat p}{q \left(q -1\right)^{2}}+\rho^V q +\frac{\hat p}{q -1}$;
    \item $t, \hat p$ and $q$ are three free parameters uniquely defining the connection.
\end{itemize}
\begin{oss}
        In the literature, as in \cite{Ohyama_2006}, the parameter $\kappa_1$ can be found under the form as $\theta+1$. The advantage of using $\theta$ is its more direct link to the Painlevé fifth equation.
\end{oss}
\begin{oss}
    This normal form can also be obtained from the connection on $\O\oplus\O(-1)$ presented in \cite{Ohyama_2006} by changing the variables $(x,y,z)$ into $(x,q,-\frac{\hat p}{q(q-1)^2})$ and by applying the elementary transformation 
\[\begin{pmatrix}
    1&0\\0&\frac{-1}{x(x-1)^2}
\end{pmatrix},\]
landing on $\O\oplus\O(2)$.
\end{oss}
An apparent singularity in $x=q$ arises, corresponding indeed to the radial singularity of the Riccati foliation in $\F_2$ shown in Figure \ref{elmq}.
\begin{defn}
   A singularity for a meromorphic connection is said \textit{apparent} if it can be removed via a (meromorphic) gauge transformation.
\end{defn}
\begin{oss}
    Apparent singularities have nilpotent or diagonalizable  residual matrix with integer eigenvalues and trivial monodromy.
\end{oss}
We have now more material for "decorating" a little bit more the picture of the Riccati foliation. Let us consider for instance the residual matrix of the logarithmic pole 0. Its diagonal form is
\[\begin{pmatrix}0&1\\0&-\kappa_0\end{pmatrix}=\begin{pmatrix}1&-\frac{1}{\kappa_0}\\0&1\end{pmatrix}\begin{pmatrix}0&0\\0&-\kappa_0\end{pmatrix}\begin{pmatrix}1&\frac{1}{\kappa_0}\\0&1\end{pmatrix},\]
pointing out the two eigenvalues $0$ and $-\kappa_0$, with the respective eigenvectors. We have then two special directions in the fiber over 0, represented by two points in $\F_2$, respectively $(0,[1:0])$ and $(0, [1:-\kappa_0])$, in red in the figure below. Notice that the same reasoning (with much more involved computations) can be applied for the other logarithmic pole at $\infty$. From now on we identify the point $[1:a]\in\P^1$ in the fibers of $\F_2$ with the element $a\in \P^1$. In the following of this section, the local coordinate on the fibers of $U\subseteq\F_2$ are called $y=[1:y]$.
\begin{oss}
    There is a geometrical interpretation of these points, making the link with the Riccati foliation more explicit. Indeed, considering the Riccati differential equation in a small neighbourhood near 0 yields
    \[dy=-\frac{1}{x}y^2+\frac{\kappa_0}{x}y \;\;\;\text{ and for } x=0,\;\;\; 0=-y^2+\kappa_0y,\]
    whose solutions are precisely 0 and $-\kappa_0$. These correspond to the first term of the Taylor expansion at $x=0$ of the invariant curve of the foliation.
\end{oss}
For the double pole, the residual matrix alone is not sufficient: one has to consider the order two matrix as well. 
%we should not only consider the residual matrix, but also the order two matrix. 
It becomes crucial to better understand the fiber above $x=1$. The Riccati differential equation is in this case 
\[dy=\frac{x-2}{(x-1)^2}y^2+\frac{-\kappa_1(x-1) +t }{\left(x -1\right)^{2}}
y\;\;\;\text{ and for } x=1,\;\;\;0=(x-2)y^2+(-\kappa_1 x +t +\kappa_1)y\]
giving as solutions $y=0$ and $y=-\frac{-\kappa_1x+t+\kappa_1}{x-2}$. Their Taylor expansion of order 2 in $x=1$ are $0$ and 
\[t +\left(-\kappa_1+t \right) \left(x -1\right)+\mathrm{O}\! \left(\left(x -1\right)^{2}\right).\]
The special red points in the fiber over $x=1$ are then $(1,0)$ and $(1,t)\in \F_2$ (corresponding indeed to the eigenvectors of the residual matrix at $x=1$), and to get to the other terms of the Taylor expansion we have to blow up both those points getting to the following picture (in which we omitted the Riccati foliation) that perfectly represents the Okamoto divisor relative to the PV type connections, that is composed by the section at infinity, the singular fibers, and the exceptional divisors $E_{1,0}^\pm$, as shown in the picture below.
\begin{center}
    \includegraphics[width=10cm]{image29.png}
     \captionof{figure}{}\label{F2}
\end{center}
\begin{oss}
     We refer to the Okamoto divisor as defined in \cite{saitotak}. We denote by $F_a$ the fiber above $x=a$ and with $E_{1,0}^\pm$ the two exceptional divisors respectively arising from the blowing up of $(1,0)$ and $(1,t)\in \F_2$, then the Okamoto divisor is 
     \[Ok^V:=[s_\infty]+[F_0]+[F_1]+[F_\infty]+[E_{1,0}^+]+[E_{1,0}^-].\]
     This is not precise. Indeed, in \cite{saitotak}, each irreducible components of the Okamoto divisor has self intersection -2. This can be achieved if we blow-up the six points $P_0^\pm$, $P_{1,1}^\pm$ and $P_\infty^\pm$. The natural ambient space of the Okamoto divisor is then more likely the one appearing in Figure \ref{Okam}.
\end{oss}
We finally remark that also the point $(q,\hat p)$ has, as all the other special red points, the interpretation as a point $q\in \P^1$ and an eigenvector $(1,\hat p)$ of the residual matrix at $x=q$. In particular we stress that for each PV type connection we have a picture as the one above depending on the values of $t,q$ and $\hat p$.

\subsection{Confluence of Singularities}\label{Confl}

Let us conclude the present section with a preliminary study about confluence of singularities, that we will use for building the compactification of the moduli space of PV type connections. We wonder what happens when the apparent singularity $q$ converges into a pole. In other words: we wonder under which conditions, for $a=0,1,\infty$, the $\lim_{q\to a}\Omega_0$ does exist and returns a well defined connection matrix. \\
From the expression of the normal form, we see that for the coefficient
\[\hat K:=\frac{\hat p^{2}}{q \left(q -1\right)^{2}}+\frac{\left(\mathit{\kappa_0} \left(q -1\right)^{2}+(\kappa_1-1)  q \left(q -1\right)-t q \right) \hat p}{q \left(q -1\right)^{2}}+\rho^V q +\frac{\hat p}{q -1},\] the $\lim_{q\to a}\hat K$ is not well defined for any value of $a=0,1,\infty$.

%Before stating the main theorem, that establish when such a confluence of singularities produces a well defined connection, we have to understand how the confluence process behaves in the Riccati foliation inside $\F_2$. It will allows us to give the theorem a very geometrical interpretation.\\
%For example, referring to Figure \ref{F2}, the confluence $q\to1$ forces the point $(q,\hat p)\in\F_2$ to enter the singular fiber. In general, for such a position of the point $(q=1, \hat p)$ no connection exists. If the trajectory of the point $(q,\hat p)$, when approaching the singular fiber, follows the Taylor expansion of the invariant curve we found in the last section, then a limit connection is well defined.
\begin{thm}\label{confl}
    Given a PV type connection in normal form, the confluence of the apparent singularity into a pole $x=a$ produces a new well defined connection if, and only if, the point $(q,\hat p)\in\F_2$ follows the trajectory of the invariant curve at $x=a$ of the induced Riccati foliation up to the order $r$ of its Taylor expansion, where $r+1$ is the order of the pole $x=a$. Moreover, the connection depends on the coefficient $r+1$ of the Taylor expansion of the trajectory.
\end{thm}
\begin{proof}
    We should explicitly compute $\lim_{q=a}\Omega_0$, and we already noticed that this limit is not defined only for the coefficient $\hat K$. \\Let us firstly study the case for $a=0$, that, we recall, is a logarithmic singularity. In Section \ref{secnormform} we proved that the Taylor expansion up to the order 0 of the invariant curves of the Riccati foliation at $x=0$ are
    \[\hat p =0 \;\;\;\text{ and }\;\;\; \hat p = -\kappa_0.\]
    We recall that, since $x=0$ is a logarithmic pole, the Taylor expansion of order 0 should suffice. \\We can hence make a substitution into $\hat K$ by replacing $\hat p$ with $\alpha q+ \beta$. Our goal is then to show that the limit $q\to0$ is well defined if, and only if, $\beta=0,-\kappa_0$. It then holds that 
    \[\mathrm{Res}_{q=0}\hat K= \beta(\beta+\kappa_0)\]
    that is indeed zero if, and only if, $\beta=0, -\kappa_0$, as desired. Moreover a straightforward computation shows that the resulting connection depends on $\alpha$.\\
    The computations for $a=\infty$ (that is the other logarithmic pole) are the same. \\For the pole of order two the procedure is similar, but we have to take in account two terms of the Taylor expansion of the invariant curve: 
    \[\hat p = 0 \;\;\;\text{ and }\;\;\;\hat p = (-\kappa_1+t)(q-1)+t\]
    We can hence make a substitution into $\hat K$ by replacing $\hat p$ with $\alpha (q-1)^2+ \beta(q-1)+\gamma$ and as before show that the limit $q\to1$ is well defined if, and only if, $(\beta,\gamma)=(0,0),(-\kappa_1+t,t)$. It then holds that 
    \[\mathrm{Res}_{q=1}\hat K= -\gamma^{2}+\left(2 \beta +\kappa_1\right) \gamma -\beta  t
    \;\;\;\;\;\text{ and }\;\;\;\;\;\mathrm{Res}_{q=1}(q-1)\hat K= \gamma(\gamma-t).\]
    Moreover a straightforward computation shows that the resulting connection depends on $\alpha$, concluding the proof.
\end{proof}
We are now ready to state the following result, which will be useful in Section \ref{seccomp} to understand the whole 3-dimensional moduli space and its compactification.
\begin{cor}\label{okmodsp}
    The moduli space of PV type connection for a fixed value of $t\in \C\setminus\{0\}$ (called the Okamoto moduli space) is given by the complement of the Okamoto divisor inside $\mathrm{Bl}\F_2$, that is the blow-up of $\F_2$ in the eight points $P_i^\pm$ and $P_{1,j}^\pm$ of Figure \ref{F2}.
\end{cor}
\begin{proof}[Sketch of proof]
    The normal form tells us that for $q\neq0,1,\infty$ and $\hat p \neq \infty$, the point $(q,\hat p)$ uniquely define a PV type connection. The last part of Theorem \ref{confl} shows that for $q=0,1,\infty$, there exist PV type connections lying exactly on those exceptional divisors, in green in the following picture. For a complete proof in the Painlevé VI type case see Theorem 2.7.6 of \cite{geompanv}.
\end{proof}
\begin{center}
    \includegraphics[width=8cm]{image1-8.png}
    \captionof{figure}{}\label{Okam}
\end{center}

















\section{Moduli Space of connections of PV type}\label{S1}
The goal of this section is to study $\Conn$: the moduli space of connections of PV type with fixed residual spectral data $\Theta$ up to gauge equivalence. As mentioned in Section \ref{PVconn}, in a recent paper, Diarra and Loray \cite{Diarra} proved that we can uniquely choose a representative by considering a connection in normal form on the vector bundle $E=\O\oplus\O(2)\to \P^1$.\\
The following three free parameters appear:
\begin{itemize}
    \item $q\in \P^1\setminus \{0,1,\infty\}$ is an apparent singularity that arises during the normalization process;
    \item $\begin{pmatrix}1\\\hat p\end{pmatrix}$ is the eigenvector relative to the non zero eigenvalue of the matrix coefficient of $\frac{dx}{(x-q)}$;
    \item $t\in \C\setminus \{0\}$ is the non zero eigenvalue of the matrix coefficient of $\frac{dx}{(x-1)^2}$.
\end{itemize}
Therefore, $t,q,\hat p$ are the good candidates for being the coordinates in a dense open set of the moduli space of connections of PV type. We can not hope that they parametrise the entire moduli space, since we have seen in the previous section that inside $\Conn$ we should allow connections with $q$ equal to 0, 1 or $\infty$. In other words, we can see $\Conn$ as a family of Okamoto moduli spaces as presented in Corollary \ref{okmodsp} indexed by $t\in\C\setminus\{0\}$. The goal is to understand what happens for $t=0,\infty$.\\We already explored the geometric interpretation and the role of $q$ and $\hat p$ in Section \ref{Confl}. The geometric interpretation of the eigenvalue $t$ seems more mysterious, and it can be understood by studying how it behaves under the action of a change of coordinates in the base $\P^1$. 
\begin{lemma}\label{tangent}
    Let $f\in\mathrm{Aut}(\P^1)$ be a Moebius transformation and $(\nabla, E, D)$ a connection of PV type in normal form. Then $f^*\nabla$ is holomorphically gauge equivalent to $(\nabla, E, D)$, with the same residual spectral data and the eigenvalue $t$ is modified by $t\mapsto Df(1)\cdot t$.
\end{lemma}
\begin{proof}
   Let us apply the variable change $X=f(x)$ and denote by $a=f(1)$. A straightforward calculation shows that, for $i\in\{0,\infty,q\}$:
   \[\mathrm{Res}_{X=f(i)}\big(f^*\Omega\big)=\mathrm{Res}_{x=i}\big(\Omega\big) \;\;\;\;\text{ and }\;\;\;\;\mathrm{Res}_{X=a}\big((X-a)\cdot f^*\Omega\big)=Df(1)\cdot\begin{pmatrix}0&1\\0&t\end{pmatrix},\]
   and we are done.
   %Without loss of generality we can always suppose that $f(1)=1$ and let us consider $U$ a small neighbourhood containing 1 and no others singularities of $f^*\nabla$.  Then the 
\end{proof}
In particular we deduce that $t$ transforms like a vector tangent to $1\in\P^1$, that is $t\in T_1\P^1\setminus\{0\}$. To summarise: we consider the rank 2 vector bundle $E=\O\oplus\O(2)\to \P^1$ with a PV type connection in normal form.
\begin{itemize}
    \item $q\in \P^1$, the base of the vector bundle,
    \item $\hat p\in F_1=\P(E_1)$, the projectivisation of the fiber above 1,
    \item $t\in T_1\P^1$, the tangent space to $\P^1$ in 1.
\end{itemize}
In conclusion, we see $t,q$ as elements of the quasi projective variety 
\[\mathcal M:=\Big\{q\in\P^1\setminus\{0,1,\infty\};\;\;t\in T_1\P^1\setminus \{0\}\Big\}\cong\P^1\times\P^1\setminus(Q^0\cup Q^1\cup Q^\infty\cup A_0\cup A_\infty),\]
where $Q^i:=\{q=i\}$ and $A_j=\{t=j\},$ as shown in the picture below.
\begin{center}
    \includegraphics[width= 12cm]{image2.png}
    \captionof{figure}{}\label{pic1}
\end{center}
\subsection{Irregular Curves}
The space $\mathcal M$ can be seen as a natural generalisation of the moduli space $\mathcal{M}_{0,5}$ of configurations of five points in $\P^1$. In our context, instead of the fifth point, we have a tangent vector. This lead us to give the following definition, that has been introduced in \cite{irrcurv} and \cite{boalchirrcurv}.
\begin{defn}
    An irregular curve $(\mathcal C, D)$ is the datum of a complex curve $\mathcal C$, an effective divisor $D=\sum a_i[p_i]\in \mathrm{Div}(\mathcal C)$ and a collection of jets $j^{a_i-1}p_i$. 
\end{defn}
\begin{oss}
    The quasi projective variety $\mathcal M$ is the moduli space of irregular curves $(\P^1, [a]+2[b]+[c]+[d])$, for distinct $a,b,c,d$.\\ In particular we have the following jets: $j^0(a)=a, j^1(b)=(b,t), j^0(c)=c$ and $j^0(d)=d$. 
\end{oss}
\begin{oss}
    An automorphisms $\phi\in\mathrm{Aut}(\P^1)$ acts on $(\P^1,D)$ as $(\phi(\P^1), \phi^*D)$. We recall that, for a point $p\in\P^1$, it holds that $\phi(j^k(p))=\phi((p, p^{(1)}, \dots,p^{(k)}))=(\phi(p), D\phi(p)(p^{(1)}),\dots,D^k\phi(p)(p^{(k)}))$.
\end{oss}
The giving of a gauge equivalence class of connections of type PV is the same as giving an irregular rational curve in $\mathcal M$ and an additional parameter $\hat p\in \C$:
\[[\nabla] \iff \big(\P^1,[0]+2[1]+[\infty]+[q]\big)\;\text{ and }\;\hat p\in \C.\]
We can then prove the following.
\begin{prop}\label{openmod}
    The moduli space of connection of PV type $\Conn$ contains as a dense open set the trivial line bundle $\mathcal M\times \C$, where $\mathcal M$ is the moduli space of configuration of four points and a tangent vector on $\P^1$.
\end{prop}
\begin{proof}
    The normal form appearing in \cite{Diarra} shows that $\Conn\supseteq\mathcal M\times \C$. Moreover, what is missing are such connections in which $q=0,1,\infty$. Finally, thanks to Lemma \ref{tangent}, we can give $\mathcal M$ the interpretation as the moduli space of configuration of four points and a tangent vector on $\P^1$, concluding the proof.
\end{proof}




\section{Compactified Moduli Space of connections of PV type}\label{compsect}
The goal of this section is to build a compactification for $\Conn$, and understand its birational geometry. We proceed by the following steps: 
\begin{itemize}
    \item In the following two sections we use a similar approach as in the Deligne-Mumford article to compactify $\mathcal M$ by understanding which stable nodal curves appear in the boundary components $Q^i$ and $A_j$. 
     \item In the third section we  describe the connections that lie in the boundary components. That will be crucial to establish a suitable set of coordinates in $\overline \Conn$.
    \item In the fourth section we understand how the trivial bundle $\mathcal M\times \C$ extends over the compactified basis $\overline{\mathcal{M}}$.
    \item Finally, we compactify the fibers of this new vector bundle obtaining the desired space $\overline \Conn$, and we study its birational geometry.
   
\end{itemize}

\subsection{Deligne-Mumford Compactification of $\mathcal M$}
To build the compactification $\overline{\mathcal M}$, we take inspiration by the compactification that Deligne and Mumford \cite{DelMun} described for $\mathcal M_{0,5}$, the moduli space of configurations of five points in $\P^1$.

Their main idea is to establish an isomorphism between $\M_{0,5}$ and the moduli space of smooth marked conics in $\P^2$ passing through four given points in general position (with the marking that is outside the four given points). This isomorphism associates to any such marked conic an element in $\M_{0,5}$, that is indeed a $\P^1$ with five punctures. The advantage of this isomorphism is that a compactification of the moduli space of smooth marked conics is much easier to get, and it is then straightforward to deduce the desired compactification $\overline{\M_{0,5}}$.
\begin{oss}
    We can see $\P^1$ with five punctures as an irregular rational curve $(\P^1 ,[a]+[b]+[c]+[d]+[e])$. Note that the divisor has degree five.
\end{oss}
%We adapt the reasoning of Deligne and Mumford for $\mathcal M$, that is the moduli space of $\P^1$ with four marking and a marked tangent vector. We can more properly say that is the moduli space of $\P^1$ with three marked 0-jets and one marked 1-jet. \\
We adapt the reasoning of Deligne and Mumford for $\mathcal M$, that is the moduli space of irregular rational curves of the form $(\P^1, [0]+2[1]+[\infty]+[q])$ up to Moebius transformations. \\
We then consider on the other side smooth marked conics in $\P^2$ passing through 3 given points $A,B,C\in \P^2$ in general position, with a prescribed tangent vector $l\in T_A\P^2\setminus \{0\}$. Since we want all these data to be "in general position", we ask the tangent direction $l$ not to point to $B$ or $C$. Up to automorphisms of $\P^2$ we can suppose these data to be
\[A=[1:0:0], \;\;\;\;B=[0:1:0],\;\;\;\;C=[0:0:1],\;\;\;\; l=\overrightarrow{AD} \;\text{ for }\; D=[1:1:1].\]
From now on, we denote by $\mathfrak{C}:=\P^2\setminus(\overline{AB}\cup\overline{AC}\cup \overline{BC}\cup \overline{AD})$, where $\overline{AB}$ is the unique line in $\P^2$ passing through $A$ and $B$. 
\begin{lemma}
    The quasi projective variety $\mathfrak{C}$ is the moduli space of smooth marked conics passing through three fixed point $A,B,C\in \P^2$ in general position, with prescribed tangent direction $l\in T_A\P^2$ not pointing $B$ or $C$.
\end{lemma}
\begin{proof}
    It is well known that the moduli space of conics in $\P^2$ is isomorphic to $\P^5$ by taking the projectivisation of the space $\C^6$ of the coefficients of the equation. In particular a conic has five degrees of freedom. Imposing then the four conditions requested, the passage through an additional point $P\notin\{A,B,C\}$ uniquely determines a conic. Moreover, the conic is smooth if, and only if, $P$ is chosen in $\mathfrak C$. Finally, let $P$ and $P'$ be two different points defining the same conic $\mathcal C$, then the marked conics $(\mathcal C;A,B,C,P, l)$ and $(\mathcal C;A,B,C,P',l)$ are not isomorphic since $\dim \mathrm{Aut}(\P^2)=8$ and hence the only automorphism of $\P^2$ fixing pointwise $\{A,B,C, l\}$ is the identity.
\end{proof}
In the picture below, on the left, we show a conic $\mathcal C$ represented by a point $P\in\mathfrak C$, while, on the right, it is shown that picking a $P$ inside one of the forbidden lines gives rise to a singular conic, represented in yellow and blue.
\begin{center}
    \includegraphics[width=12cm]{image9.jpg}
    \captionof{figure}{}\label{pic2}
\end{center}
Now, let us construct the explicit isomorphism $\mathcal M\cong \mathfrak C$. Once we have it, we can build the compactification $\overline{\mathfrak C}$ and deduce $\overline{\mathcal M}$. We recall that a point $(t,q)\in\M$ represents a class of irregular rational curves, that we would like to identify with a conic in $\mathfrak C$. To do so, we have to find a suitable way to associate with each conic $\mathcal C\in \mathfrak C$ the parameters $(t,q)$.\\ Let us call $[X_0:X_1:X_2]$ the coordinates on $\P^2$. The conics in $\mathfrak C$ are the vanishing loci of the polynomials $X_1X_2+a(X_0X_1-X_0X_2)=0$ for $a\in \C.$ 
We want to consider these curves with the parametrisation $\P^1\to\P^2$ that identifies $0,1,\infty\in \P^1$ with respectively $B,A,C\in \P^2$.
\[\begin{matrix}
    \P^1 & \dashrightarrow & \P^2\\ [s_0:s_1] & \mapsto &[s_0s_1:as_1(s_0-s_1):as_0(s_0-s_1)]
\end{matrix}.\] 
We already noticed that any choice of a point $P\in \mathfrak C$ uniquely define a such conic. Indeed, if we impose the passage for the point $P=[X_0^0:X_1^0:X_2^0]$, we get
\[a=\frac{X^0_2X_1^0}{X_2^0X_0^0-X_1^0X_0^0}.\]
In fact, the point $P$ plays the role of $q$ in the irregular rational curve. \\Since $\mathfrak C\cap \{X_0\neq0\}=\mathfrak C$, we can 
work in the affine charts $V_1:=\{s_1\neq0\}\subseteq \P^1$ and $U_0:=\{X_0\neq0\}$, with coordinates $x_i=X_i/X_0$. The parametrisation becomes
\[\begin{matrix}\gamma :&V_1 & \to & U_0\\ &\frac{s_0}{s_1}=s & \mapsto & \Big(\frac{a(s-1)}{s},a(s-1)\Big)\end{matrix}.\]
The preimage of $P$ is then 
\[q=\gamma^{-1}(P)=\frac{x_2^0}{x_1^0}.\] 
The role of $t$ is, of course, played by the tangent vector of the conics in $A$. We already know that it has the same direction as $\overline{AD}$. A computation shows that 
\[\gamma'(1)=(a,a)\;\;\;\text{ and then } \;\;\;t=\frac{1}{a}=\frac{x_2^0-x_1^0}{x_2^0x_1^0}\]
We get then the following birational map
\[\begin{matrix}
    \phi\colon&\P^2&\dashrightarrow &\P^1\times\P^1\\
    
    &[X_0:X_1:X_2]&\mapsto&\big([X_2X_0-X_1X_0:X_1X_2],[X_2:X_1]\big)\\&(x_1,x_2)\in U_0&\mapsto&(t,q)=\Big(\frac{x_2-x_1}{x_2x_1},\frac{x_2}{x_1}\Big)\in V_1\times V_1
\end{matrix}\]
that is not defined in $A,B,C$. We prove that
\begin{lemma}\label{lemma1}
    The restricted map $\phi_{|\mathfrak C}\colon \mathfrak C\to \M$ is an isomorphism.
\end{lemma}
\begin{proof}
We refer to Figure \ref{pic1} for the notations, recalling that 
\[\mathfrak{C}\cong\P^2\setminus(\overline{AB}\cup\overline{AC}\cup \overline{BC}\cup \overline{AD})\;\;\;\text{and}\;\;\;\mathcal M\cong\P^1\times\P^1\setminus(Q^0\cup Q^1\cup Q^\infty\cup A_0\cup A_\infty).\]
We have that $\phi_{|\mathfrak C}$ is everywhere well defined, and hence it induce an isomorphism into its image. It suffice to show $\phi(\mathfrak C)=\M$. First of all we remark that $Q^0, Q^1, Q^\infty$ and $A_\infty$ are not in the image of $\phi$. Moreover $A_0=\phi(\overline{BC})$ and $\overline{BC}\not\subset\mathfrak C$. The other lines not appearing in $\mathfrak C$ are such that 
\[\phi(\overline{AB})=([1:0],[0:1])=(\infty,0)=A_\infty\cap Q^0, \]
\[\phi(\overline{AC})=([1:0],[1:0])=(\infty,\infty)=A_\infty\cap Q^\infty,\;\;\;\;\;\;\phi(\overline{AD})=([0:1],[1:1])=(0,1)=A_0\cap Q^1.\]
And then $\phi(\mathfrak C)=\M$ as desired.
\end{proof}
We build $\overline{\mathfrak C}$ by understanding who are the limit conics lying on the boundary components.
\\\textbf{First Step:}\\
We can easily add to $\mathfrak{C}$ the points lying on the segments $\overline{AB}\setminus\{A,B\}, \overline{AC}\setminus\{A,C\}, \overline{BC}\setminus\{B,C\}, \overline{AD}\setminus\{A,D\}$ by taking into account singular conics, as shown in Picture \ref{pic2}.
\\\textbf{Second Step:}\\
Since we are less comfortable working with tangent vectors, we can blow-up the point $A\in \P^2$, and call $L$ the point in the exceptional divisor $E_A$ corresponding to the direction $l$, as shown in the following picture. Notice that, now, the condition of tangency has been translated by the condition of "passing through $L$". 
\begin{center}
    \includegraphics[width=5cm]{image10.png}
\end{center}
\textbf{Third Step:}\\
In order to distinguish conics that pass through $A,B$ and $L$ by the slope of incidence, let us blow up these three points, denoting respectively by $E_B$, $E_C$ and $E_L$ the corresponding exceptional divisors. In the following picture we denote into brackets the self-intersection number of any shown curve.
\begin{center}
    \includegraphics[width=6cm]{image11.png}
    \captionof{figure}{}\label{p2blow}
\end{center}
\begin{lemma}\label{modC}
    Let us denote by $X$ this blow-up of $\P^2$ and by $P_1 ,P_2$ the intersection points between $E_A$ and the blue lines. There is a bijection between $(X\setminus E_A)\cup \{P_1,P_2\}$ and the set of marked conics in $\P^2$ passing through $A,B,C$ and with prescribed tangent $l$, with the exception that $P_1$ and $P_2$ defines the same marked conic. In particular $\overline{\mathfrak C}$ is isomorphic to the (singular) space obtained by the contraction of the $(-2)$-curve $E_A$ inside $X$.
\end{lemma}
\begin{proof}
    The first part follows by construction, recalling that the exceptional divisor $E_A$ represents the incidence slopes into $A$, but we are looking for conics incident in $A$ with the prescribed slope $l$. Finally points $P_1$ and $P_2$ induce the same marked conics, since we can exchange the two branches via an automorphism. As a consequence, we can contract $E_A$ to $P_1$ giving the thesis.
\end{proof}
We postpone to the next section the study of the stable nodal curve corresponding to points lying in the boundary components of $\overline{\mathfrak C}$.

We are now ready to build the compactification $\overline{\mathcal M}$.
\begin{lemma}\label{lemma3}
    The blow-up of $\P^1\times\P^1$ at the points $(0,1),(\infty,0)$ and $(\infty, \infty)$ give rise to a smooth variety $\widetilde{\M}$ that is isomorphic to the variety $X$ appearing in Lemma \ref{modC}. Moreover, the isomorphism $\phi$ of Lemma \ref{lemma1} lifts to an isomorphism $\widetilde \phi\colon \widetilde X\to \widetilde\M$.
\end{lemma}
\begin{proof}
    The following picture shows the intersection numbers of the curves arising in $\widetilde{\M}$.
    \begin{center}
    \includegraphics[width=6cm]{image12.png}
    \captionof{figure}{}\label{p1blow}
    \end{center}
    Both $\widetilde{\M}$ and $X$ are then isomorphic to a weak Del Pezzo surface of degree 5 and since they have the same number of $(-1)$ and $(-2)$-curves, they are then isomorphic, since the moduli space of such weak Del Pezzo surfaces is just a point as shown in \cite{Martin2020WeakDP}. \\
    We have now to show that $\phi$ lifts to an isomorphism between $X$ and $\widetilde \M$ as shown in Figures \ref{p2blow} and \ref{p1blow}. We already noticed in Lemma \ref{lemma1} that $\phi(\overline{BC})=A_0$. Let us call $E_A, E_B$ and $E_C$ the exceptional divisors of the points $A,B$ and $C$. It holds that
    \begin{align*}\phi(E_A)&=\lim_{X_1\to0}\phi([1:X_1:\alpha X_1])=\lim_{X_1\to 0}([\alpha X_1-X_1:\alpha X_1^2],[\alpha X_1:X_1])=\\&=\lim_{X_1\to 0}([\alpha -1:\alpha X_1],[\alpha :1])=([1:0],[\alpha:1])=A_\infty.\end{align*}
    With a similar computation we get that $\phi(E_B)=Q^0$ and $\phi(E_C)=Q^\infty$.\\
    In Lemma \ref{lemma1}, we have also shown that $\phi(\overline{AB})=(\infty,0)$, $\phi(\overline{AC})=(\infty,\infty)$ and $\phi(\overline{AD})=(0,1)$. We now blow up these points in $\P^1\times\P^1$ and we show that the image of the lines in $\P^2$ fit perfectly the exceptional divisors. We have that
    \[\mathrm{Bl}_{(\infty,\infty)}(\P^1\times\P^1)=\{([t_1:t_2],[q_1:q_2],[R_1:R_2])\in(\P^1)^3\;|\;t_2R_1=q_2R_2\}.\]
    Therefore $\phi(\overline{AC})=\phi([1:0:R])=([1:0],[1:0],[R:1])=B^\infty_\infty$. And the same holds for the two other points. This conclude the proof.
\end{proof}
This implies the following result.
\begin{thm}\label{compbasis}
    The compactification of the moduli space of stable nodal curves $\overline{\mathcal M}$ is given by the contraction of the $(-2)$-curve $A_\infty$ inside $\widetilde{\M}$.   
\end{thm}
\begin{proof}
    The strategy is to use the following diagram. We proved in Lemma \ref{lemma1} that $\M\cong \mathfrak C$. Then we deduced in Lemma \ref{modC} the compactification $\overline{\mathfrak C}$. Following the same path for $\M$, we get to the variety $\widetilde \M$ that we showed being isomorphic to $X$ in Lemma \ref{lemma3}, identifying $A_\infty$ with $E_A$, that is the curve we contracted in $X$ to get to $\overline{\mathfrak C}$. %The two vertical arrows consist finally in the contraction of the same $(-2)$-curve, establishing an isomorphism $\overline \M\cong \overline{\mathfrak C}$, concluding the proof.
%\begin{center}
% https://tikzcd.yichuanshen.de/#N4Igdg9gJgpgziAXAbVABwnAlgFyxMJZABgBoBGAXVJADcBDAGwFcYkQAdDgdy1j0awuAWRABfUuky58hFGQBM1Ok1bsuEWjABOjLGBgACEeMkgM2PASJliyhizaJOHURKmXZRBaTs0Has5cALb0OAAWAGba9ADWhgDCph4y1ig+VP6qTiAAGsnm0lZyyD5KWY7qHJo6egbAIWFRMfEJYuLKMFAA5vBEoNEQwUhkIDgQSOQ0dTlQ9HDhXQWDw4ij40g+KpVBHKER2sHAAEKM7dP6s-OLUMvaQ5M0G4gArBWBLgDGBN13D4gAFieE1e7xyXBwMAAHjhgAAZGDBUJiLjaGCRYDBaBtP6rdYgoHbD5cb5gbrGDiQmHwxHI1Ho4CMWn0ADM7XcIBWSEJzxZYKq+3ChxOZxA03oACMYIwAApFLzObRYbrhHBikAzdhzBZLDlcxBbXn83ak8kQ6GwhFI+gojhojFM63kdmUMRAA
%\begin{tikzcd}
%\M \arrow[rr, "\cong \text{Lemma}\ref{lemma1}"]                               %                               &  & \mathfrak C                               %                              \\
%\widetilde\M \arrow[d, dashed] \arrow[u, "\mathrm{Bl}", dashed] \arrow[rr, %"\cong \text{Lemma}\ref{lemma3}"] &  & X \arrow[d, "\text{Lemma}\ref{modC}"] %\arrow[u, "\mathrm{Bl}"', dashed] \\
%\overline \M \arrow[rr, "\cong"]                                               %                              &  & \overline{\mathfrak C}                                                 
%\end{tikzcd}
%\end{center}
\end{proof}
We point out the analogy of the curves in $\widetilde{\M}$ and $X$. In order to make the comparison clear, let us give a name to some of the curves in $\widetilde{\M}$ that are needed later on in this article.
\begin{center}
    \includegraphics[width=10cm]{image13.png}
    \captionof{figure}{}\label{boundfig}
    \end{center}
\[\begin{matrix}
    E_B &\leadsto&Q^\infty&&&&\{\text{Blue singular conic}\}&\leadsto&B_\infty^0\cup B_\infty^\infty\\
    E_L &\leadsto&Q^1&&\text{and}&&E_A &\leadsto&A_\infty\\
    E_C &\leadsto&Q^0&&&&\{\text{Yellow singular conic}\}&\leadsto&A_0\cup B_0^1
\end{matrix}
\]
Where 
\begin{itemize}
    \item $Q^i=\{q=i\}$ for $i=0,1,\infty$,
    \item $A_j=\{t=j\}$ for $j=0,\infty$ and 
    \item $B_j^i$ are the exceptional divisors.
\end{itemize} 
We end this section with the following result that describes the singularities arising in $\overline{\mathcal M}$. 
\begin{prop}\label{Msing}
    The space $\overline{\mathcal M}$ presents strictly canonical singularities.
\end{prop}
\begin{proof}
First of all we recall that $\overline{\mathcal M}$ is given by the contraction of the $(-2)$-curve $A_\infty$ of $\widetilde{\M}$.\\ If we set $(t,q)$ the local coordinates of $\widetilde\M$ centred at $A_0\cap Q^0$, the volume form $\frac{dt}{t^2}\wedge\frac{dq}{q^2}$ gives the canonical divisor
\[K_{\widetilde{\M}}\equiv-2(Q^0+A_0)-B^1_0-B^0_\infty+B^\infty_\infty.\]
In particular it holds that 
\[(K_{\widetilde{\M}} \cdot A_\infty)=0,\]
and the singularity arising in $\overline{\mathcal M}$ is hence canonical. (Maybe a Du Val singularity ?)To show that are strictly canonical, we can compute the discrepancy: let $\pi\colon \widetilde{\M}\to \overline{\mathcal M}$ the contraction. Then
\[K_{\widetilde{\M}}\equiv\pi^*K_{\overline{\mathcal M}}+\alpha A_\infty, \]
and computing the intersection product with $A_\infty$ we get that
\[0=(K_{\widetilde{\M}} \cdot A_\infty)=(\pi^*K_{\overline{\mathcal M}}\;\cdot A_\infty)-2\alpha,\]
and $(\pi^*K_{\overline{\mathcal M}}\;\cdot A_\infty)=0$ since $A_\infty$ has been contracted to a point. This implies that $\alpha=0$, as desired.  
\end{proof}


\subsection{Stable Nodal Curves}\label{secstabnod}
In this section we extend the universal curve we studied in Figure \ref{pic1} for $\M$ to the boundary components of $\overline \M$. In order to do so we have to introduce the notion of stable nodal curves that appear in the Deligne and Mumford compactification for the moduli space $\mathcal M_{g,n}$. This study will be relevant to understand how the connections on the boundary components of $\overline\Conn$ are made, since they are connections over stable nodal curves, as we will see in the next section. 

\begin{defn}
    A complex curve $X$ is called a \emph{nodal curve} if it has only nodes as singularities.
\end{defn}
\begin{defn}
    A nodal curve $X$, with markings $M=\{p_1,\dots,p_n\}\subseteq X\setminus\mathrm{Sing}(X)$, is \emph{stable} if $\mathrm{Aut}(X; M)$ is finite, where we ask each element of $\mathrm{Aut}(X;M)$ to fix $M$ pointwise.
\end{defn}
\begin{fact}
    Let $X$ be a nodal curve with one node and let $X_1, X_2$ be the two irreducible components. Then $\mathrm{Aut}(X)\hookrightarrow\mathrm{Aut}(X_1)\times\mathrm{Aut}(X_2)$, given by those $(\phi_1, \phi_2)$ automorphisms fixing the node. Notice that the action $\mathrm{Aut}(X)\curvearrowright X$ decomposes into the two separated actions $\mathrm{Aut}(X_1)\curvearrowright X_1$ and $\mathrm{Aut}(X_2)\curvearrowright X_2$.
\end{fact}
We recall that, as shown in Figure \ref{boundfig}, the boundary components of $\overline\M$ have an interpretation as confluence of points. For example $Q^0$ is the boundary components that allows curves in which $q$ has converged into 0. Anyway, we do not want the curves upon $Q^0$ just being a $\P^1$ with four markings instead of five: we want to keep the information of the confluence. We recall moreover that we are considering $\overline \M$ as a moduli space of punctured curves up to Moebius transformations. This helps us since confluences are not invariant under the action of $\mathrm{Aut}(\P^1)$.
\begin{es}
    Let us consider a Moebius transformation $f$ that sends $q$ to $\infty$, and that fixes 0 and $t$. We remark that such transformation does exist since $\mathrm{Aut}(\P^1)$ is 3-transitive. If we consider now $f(\P^1)$, some of our markings moved, but we are still considering an equivalent punctured $\P^1$. We should compare the confluence $q\to0$ in both cases. In the original $\P^1$ the point $q$ is just collapsing into 0, but in $f(\P^1)$ the points $f(1)$ and $f(\infty)$ will collapse together into 1. Roughly speaking, the confluence $q\to0$ is Moebius equivalent to the confluence $1\to\infty$, and we want to keep trace of both configurations, as happens for the stable nodal curves.
\end{es} 
In those kind of curves we will always have two irreducible components that will encode the two different behaviours of a confluence, as we will see just below.\\
For a precise and formal description of stable nodal curves, the interested reader can for instance look at \cite{DelMun} or \cite{Arbarello2011}.
\begin{oss}
    We have to take into account that the curves we want to study are quite different from the ones of Deligne and Mumford: instead to have just points as markings, we have also a tangent datum, $t$, as shown in the picture \ref{pic1}. \\A more algebraic setting in which we can study our problem is the following: since $t$ is an element of $T_1\P^1\cong\C$, we can consider our space as $Bl_1(\P^1)$, with markings $\{0,1,\infty,q\}\subseteq \P^1$ and $t\in E\setminus\{0,\infty\}$, where $E$ is the exceptional divisor. Any automorphism $f\in \mathrm{Aut}(\P^1)$ acts in the following way on $Bl_1(\P^1)$: as $f$ outside the exceptional divisor, and as the multiplication by $Df(1)\in\C^*$ on the exceptional divisor. 
\end{oss}



\textbf{Curves upon $A_0$.}
The boundary $A_0$ represents those curves with $t=0$, but it turns out that setting $t=0$ is not the only possible representation. We recall indeed that we are considering curves up to isomorphism and, in particular, we can apply a Moebius transformation $f_{A_0}$ that fixes 1 and such that $Df_{A_0}(1)\cdot t= 1$. We can then study the limit for $t\to0$ and look at what the resulting curve looks like. One such Moebius transformation is 
\[f_{A_0}(x)=\frac{(x+1)t}{(t-2)x+t+2}.\]
It holds that 
\[\lim_{t=0}f_{A_0}(0)=\lim_{t=0}f_{A_0}(q)=\lim_{t=0}f_{A_0}(\infty)=0.\]
\begin{center}
    \includegraphics[width=10cm]{image1-3.png}
    \captionof{figure}{}\label{A0}
\end{center}


\textbf{Curves upon $Q^0$.}
The boundary $Q^0$ represents those curves with $q=0$. We can apply a Moebius transformation that fixes 0 and sends $q\to \infty$. We can then study the limit for $q\to0$ and look at what the resulting curve looks like. One such Moebius transformation is 
\[f_{Q^0}(x)=\frac{x}{x-q}.\]
By a straightforward computation it holds that 
\[\lim_{q\to0}f_{Q^0}(1)=\lim_{q\to0}f_{Q^0}(\infty)=1,\;\;\;\;\text{ and }\;\;\;\;\lim_{q\to0}Df_{Q^0}(1)\cdot t=1.\]
\begin{center}
    \includegraphics[width=10cm]{image1-4.png}
    \captionof{figure}{}\label{Q0}
\end{center}
\textbf{Curves upon $Q^1$.}
The boundary $Q^1$ represents those curves with $q=1$. We can apply a Moebius transformation that fixes 1 and sends $q\to 0$. We can then study the limit for $q\to1$ and look at what the resulting curve looks like. One such Moebius transformation is 
\[f_{Q^1}(x)=\frac{t(x-q)}{(q - 1 + t)x + (-qt - q + 1)}.\]
By a straightforward computation it holds that
\[\lim_{q\to1}f_{Q^1}(0)=\lim_{q\to1}f_{Q^1}(\infty)=1,\;\;\;\;\text{ and }\;\;\;\;\lim_{q\to1}Df_{Q^1}(1)\cdot t=1.\]

\textbf{Curves upon $Q^\infty$.}
The boundary $Q^\infty$ represents those curves with $q=\infty$. We can apply a Moebius transformation that fixes $\infty$ and sends $q\to 0$. We can then study the limit for $q\to\infty$ and look at what the resulting curve looks like. One such Moebius transformation is 
\[f_{Q^\infty}(x)=-\frac{x}{q} + 1.\]
By a straightforward computation it holds that
\[\lim_{q=\infty}f_{Q^\infty}(0)=\lim_{q=\infty}f_{Q^\infty}(1)=1,\;\;\;\;\text{ and }\;\;\;\;\lim_{q=\infty}Df_{Q^\infty}(1)\cdot t=0.\]
The following pictures show how the nodal curves described so far look like. The black and yellow irreducible components represent the limit curves obtained after the application of the Moebius transformation. 
\begin{center}
    \includegraphics[width=10cm]{immage1-5.jpeg}
    \captionof{figure}{}\label{stcurv}
\end{center}
\begin{oss}
    We see that, as expected, each curve shows only one parameter determining its position on the boundary component.
\end{oss}
\textbf{Curves upon $B_0^1$.}
The boundary $B^1_0$ represents those curves with $t=0$ and $q=1$. As we will precise in the next chapter, we denote by $A=\frac{t}{q-1}$ the coordinate on this exceptional divisor. As for the other boundary components, one can either directly consider the curve with $t=0$ and $q=1$, either one can make the substitution $t=A(q-1)$ and apply a Moebius transformation that fixes $1$ and sends $q\to \infty$. We can then study the limit for $q\to1$ and look at what the resulting curve looks like. One such Moebius transformation is 
\[f_{B^1_0}(x)=\frac{\left(1-q \right) x}{x -q}.\]
By a straightforward computation it holds that
\[\lim_{q\to1}f_{B^1_0}(0)=0,\;\;\;\;\text{ and }\;\;\;\;\lim_{q\to1}\Big(Df_{B^1_0}(1)\cdot A(q-1)\Big)=A.\]

\textbf{Curves upon $B_\infty^0$.}
The boundary $B_\infty^0$ represents those curves with $t=\infty$ and $q=0$. We denote by $R=qt$ the coordinate on this exceptional divisor. Unlike the other times, here one can not directly consider the curve with $t=\infty$ and $q=0$, since we do not know what an infinite tangent vector is. We still have two possible options: either, via a Moebius transformation $f_{B_\infty^0}$, we fix 1, we normalise $t=1$ and we fix 0, either, with $g_{B_\infty^0}$, we fix 1, we normalise $t=1$, but we fix $\infty$. These two Moebius transformations are
\[f_{B_\infty^0}(x)=\frac{\left(t +1\right) x}{t  x +1}\;\;\;\text{ and }\;\;\;g_{B_\infty^0}(x)=\frac{x}{t}+\frac{t -1}{t}.
\]
By a straightforward computation, after, of course, substituting $q=\frac{R}{t}$, it holds that
\[\lim_{t=\infty}f_{B_\infty^0}\Big(\frac{R}{t}\Big)=\frac{R}{R+1},\;\;\;\;\text{ and }\;\;\;\;\lim_{t=\infty}Df_{B_\infty^0}(1)\cdot t=1.\]
\[\lim_{t=\infty}g_{B_\infty^0}(0)=\lim_{t=\infty}g_{B_\infty^0}\Big(\frac{R}{t}\Big)=1,\;\;\;\;\text{ and }\;\;\;\;\lim_{t=\infty}Dg_{B_\infty^0}(1)\cdot t=1.\]

\textbf{Curves upon $B_\infty^\infty$.}
The boundary $B_\infty^\infty$ represents those curves with $t=\infty$ and $q=\infty$. We denote by $S=\frac{t}{q}$ the coordinate on this exceptional divisor. As before, we have the same two possible options, and we can use the same functions
\[f_{B_\infty^0}(x)=\frac{\left(t +1\right) x}{t  x +1}\;\;\;\text{ and }\;\;\;g_{B_\infty^0}(x)=\frac{x}{t}+\frac{t -1}{t}.
\]
By a straightforward computation, after, of course, substituting $q=\frac{t}{S}$, it holds that
\[\lim_{t=\infty}f_{B_\infty^0}(\infty)=\lim_{t=\infty}f_{B_\infty^0}\Big(\frac{t}{S}\Big)=1,\;\;\;\;\text{ and }\;\;\;\;\lim_{t=\infty}Df_{B_\infty^0}(1)\cdot t=1.\]
\[\lim_{t=\infty}g_{B_\infty^0}\Big(\frac{t}{S}\Big)=\frac{S+1}{S},\;\;\;\;\text{ and }\;\;\;\;\lim_{t=\infty}Dg_{B_\infty^0}(1)\cdot t=1.\]

\textbf{Curve upon the point $\overline{A_\infty}$.}
The point $\overline A_\infty$ represents those curves with $t=\infty$ and any value of $q$. As in the other cases we have two possible options: we can fix 0, 1 and normalize $t=1$ or fix $\infty$, 1 and normalize $t=1$. We have then the two following Moebius transformations
\[f_{ A_\infty}(x)=\frac{t x}{(t-1)  x +1}\;\;\;\text{ and }\;\;\;g_{ A_\infty}(x)=\frac{x+t-1}{t}.
\]
In both cases
\[\lim_{t\to\infty}f_{ A_\infty}(x)=\lim_{t\to\infty}g_{ A_\infty}(x)=1,\]
and then all the non fixed points converge to 1.

We can resume it with the following picture.
\begin{center}
    \includegraphics[width=12cm]{image1-6.jpeg}
\end{center}




Let us conclude by resuming in the following picture the coordinates we introduced on $\overline\M$, and that we will use in the next section. Note that $r,s,T$ do not form a coordinate system around the singular points. They just represent the coordinate on the divisors.
%\textbf{Curve over the Intersection Points}
%\begin{center}
%    \includegraphics[width=10cm]{image (24).png}
%\end{center}
\begin{center}
    \includegraphics[width=6cm]{image31.png}\hskip 30 pt
    \includegraphics[width=6cm]{image.png}
\end{center}



\subsection{Connections on the Boundary}\label{cononbou}
We already proved in Proposition \ref{openmod} that the moduli space of PV type connection $\Conn$ contains $\M\times\C$ as a dense open set, and we studied in Theorem \ref{compbasis} the compactification of the basis $\M$. In order to build the compactification $\overline\Conn$, that correspond to understand how the trivial bundle extends over $\overline \M$, we have to make a choice about the "vertical" coordinate, the coordinate along the fibers, both inside $\Conn$ as along the new boundary components. \\
To do that, we notice that a connection on a stable nodal curve is a collection of connections on any irreducible component. One of the irreducible components is always rigid, in the sense that it is an irregular rational curve endowed with a divisor of degree three (the node becomes a pole of the connection). Here we get hence a (sometimes confluent) hypergeometric connection, that is rigid and that is always the same for all the family of stable nodal curves.
\begin{es}
    A stable curve lying on $A_0$ is composed of an lower (in yellow in Figure \ref{stcurv}) irreducible component that is rigid since no free parameters appear on it, and an upper component (green) in which the parameter $q$ appears. A connection on this stable nodal curve is the data of two distinct connections: one on the upper component and one on the lower. Our goal is to describe and understand how both connections are made, but we can already say that we expect the lower connection not to depend on $q$, and we expect that both depend on the same extra parameter $p$, or just one of them depends on it and the other does not. 
\end{es}
On $\M\times \C$ we use the coordinate $\hat p$, that appears in the normal form and that we have already seen in $\F_2$, as in Figure \ref{F2}. On each boundary component we choose a suitable coordinate around a generic point that allows us to compute the generic limit connections lying on the boundary.

For the following, we consider a connection $\nabla$ of PV type, in the normal form as described in section \ref{S1}. We call $\Omega_0$ the connection matrix relative to the trivialising open set $U_0=\P^1\setminus\{\infty\}$. 

\medskip
\textbf{Connections on $A_0$.}
We have a family of stable nodal curves with two irreducible components depending on the parameter $q$. We have hence to describe the limit connections on both irreducible components. \\
On the first component, relative to setting $t=0$, in green in Figure \ref{A0}, we can easily define the limit connection by directly computing
\[\Omega_0^{t=0}:=\lim_{t=0}\Omega_0.\]
After applying an elementary transformation based in $x=1$, the double pole disappears. Therefore $\lim_{t=0}\nabla$ corresponds to an hypergeometric equation with polar divisor $[0]+[1]+[\infty]$. \\
We can study what happens on the other irreducible components, that is the one occurring when the Moebius transformation $f_{A_0}$ is applied. In order to do so, we consider the following limit:
\[\Omega_0^{f_{A_0}}:=\lim_{t=0}f_{A_0}^*\Omega_0=\begin{pmatrix}
0 & \frac{1}{\left(X -1\right)^{2}} 
\\
\frac{\rho \left(q -1\right)}{X^{2}}-\frac{\hat p \left(\hat p -\kappa_1+1 \right)}{\left(q -1\right) X^{2}}+\frac{\hat p^{2}}{\left(q -1\right)^{2} X^{2}}+\frac{\hat p \left(\hat p +\mathit{\kappa_0} \right)}{q \,X^{2}} & \frac{1}{\left(X -1\right)^{2}}+\frac{\kappa_1}{X}-\frac{\kappa_1}{X -1} 
\end{pmatrix}.
\]
\begin{oss}
    We make some observation here, and we will not repeat them for the others computations that follow.
    \begin{itemize}
        \item The pole of order two in $X=0$ becomes simple after applying an elementary transformation in $X=0$: 
        \[\begin{pmatrix}
0 & \frac{1}{X \left(X -1\right)^{2}} 
\\
 \frac{\rho \left(q -1\right)}{X}-\frac{\hat p \left(\hat p -\kappa_1+1 \right)}{\left(q -1\right) X}+\frac{\hat p^{2}}{\left(q -1\right)^{2} X}+\frac{\hat p \left(\hat p +\mathit{\kappa_0} \right)}{q \,X} & \frac{1}{\left(X -1\right)^{2}}+\frac{\kappa_1-1}{X}-\frac{\kappa_1}{X -1}
\end{pmatrix}.
\]
 \item The leading coefficient relative to the pole $X=1$ has now 1 as coefficient instead of $t$. 
    \end{itemize}
\end{oss}
Finally, we show that the residue around the nodal singularity agrees on both connections, as shown in \cite{strata}.
\begin{prop}
    Both connections have the same residual spectral data in the node.
\end{prop}
\begin{proof}
   Computing residual spectral data as shown in the end of Section \ref{resspecd} we get for both connections:
   \[4 \rho( q-1) +\kappa_1^2-\frac{4 \hat p \left(\hat p -\kappa_1+1 \right)}{q -1}+\frac{4 \hat p^{2}}{\left(q -1\right)^{2}}+\frac{4 \hat p \left(\hat p +\kappa_0 \right)}{q}.\]
\end{proof}


\textbf{Connections on $Q^0$.}
Here we have a family of stable nodal curves depending on the parameter $t$. \\
In the first component, relative to the direct setting $q=0$, we can try to compute the limit connection
\[\Omega_0^{q=0}:=\lim_{q\to0}\Omega_0(x; t,q,\hat p).\]
%We get a well defined limit connection as explained in Theorem \ref{confl}.
\begin{lemma}\label{Q0}
The limit
\[\lim_{q\to0}\Omega_0(x; t,q,\hat p)\]
exists if, and only if, for $q\to0$, the point $(q,\hat p)\in \F_2$ follows the trajectory $(q, \alpha q+\beta)$, for $\beta=0,-\kappa_0$. Moreover the resultant connection depends on $\alpha$, the asymptotic slope of the convergence.
\end{lemma}
\begin{proof}
The first part of the Lemma follows from Theorem \ref{confl}. To distinguish them, we call $\alpha^0_-$ the slope of the invariant curve for $\beta=0$, and $\alpha^0_+$ the one for $\beta=-\kappa_0$. Computing the limit for $\beta=0$, we get:
    \[\lim_{q\to0}\Omega_0(x; t,q,\alpha^0_- q )=\begin{pmatrix}
0 & \frac{1}{x\left(x -1\right)^{2}} 
\\
 \alpha^0_-  \mathit{\kappa_0} -x \rho  & \frac{-\mathit{\kappa_0} -1}{x}-\frac{\kappa_1}{x -1}+\frac{t}{\left(x -1\right)^{2}} 
\end{pmatrix}
\]
and, for $\beta=-\kappa_0$:
\[\lim_{q\to0}\Omega_0(x; t,q,\alpha^0_+ q -\kappa_0)=\begin{pmatrix}
0 & \frac{1}{x\left(x -1\right)^{2}} 
\\
 2 \mathit{\kappa_0}^{2}+\left(-\alpha^0_+ +t +\kappa_1-\frac{1}{x}\right) \kappa_0 -x \rho & \frac{-\kappa_0 -1}{x}+\frac{-\kappa_1}{x -1}+\frac{t}{\left(x -1\right)^{2}} 
\end{pmatrix}.\]
In both cases the matrices depends, as desired, on the parameter $t$ defining the stable nodal curve and on the parameter $\alpha$ defining the connection.
\end{proof}
\begin{oss}
    We can say that $\alpha$ is the coordinate of the exceptional divisor arising from blowing up the red points in $\F_2$.
\end{oss}
We still have to study what happens on the other irreducible components, with coordinate $X=f_{Q^0}(x)$. In order to do so, we consider the matrix $f_{Q^0}^*\Omega_0$ with the parameter $\hat p$.
\begin{lemma}
    The limit $\Omega_0^{f_{Q^0}}:=\lim_{q\to0}f_{Q^0}^*\Omega_0$ produces a well defined connection.
\end{lemma}
\begin{proof}
    Computing the limit we get: 
    \[\lim_{q\to0}f_{Q^0}^*\Omega_0(X; t,q,\hat p)=\begin{pmatrix}
0 & -\frac{1}{X \left(X -1\right)}
\\
 -\frac{\hat p}{X-1} +\frac{-\hat p \left(\hat p+\kappa_0  \right)}{(X -1)^2} & -\frac{\kappa_0}{X}+\frac{\kappa_0+1}{X -1}  
\end{pmatrix}.
\]
\end{proof}
A further computation shows that we get two different kinds of connections. For a generic value of $\hat p$ we get a connection with a ramified pole in $X=1$, while for $\hat p=0, -\kappa_0$ (that are the values appearing in Lemma \ref{Q0}) we get the hypergeometric equations we were expecting.
%\begin{oss}
%    It holds that $\lim_{q\to0}pq=-\lim_{q\to0}\frac{\hat p}{(q-1)^2}=-\hat p$.
%\end{oss}
\begin{prop}
    Connections with $\hat p = 0, -\kappa_0$ have the same residual spectral data in the node.
\end{prop}
\begin{proof}
   Computing residual spectral data as shown in the end of Section \ref{resspecd} we get $(\kappa_0+1)^2$ and $(\kappa_0-1)^2$ for $\hat p=0$ and $\hat p =-\kappa_0$ respectively.
\end{proof}

\medskip
\textbf{Connections on $Q^\infty$.}
The same procedure as for $Q^0$ can be applied (see Section \ref{symm}). The vertical coordinate that we should use is $P^\infty=-\frac{\hat p}{(\frac{1}{Q}-1)^2}$, where $Q=1/q$.


\medskip
\textbf{Connections on $B_0^1$.}
We can directly compute the limit for $t\to0$ and $q\to 1$. To do that, we introduced in the last section the coordinate $A=\frac{t}{q-1}$.
\begin{lemma}
    The limit $\lim_{q\to1}\Omega_0(x; A,q,P^1_0)$ produces a well defined connection if we set $P^1_0=-\frac{\hat p}{(q-1)}$.
\end{lemma}
\begin{proof}
    Computing the limit we get: 
    \[\lim_{q\to1}\Omega_0(x; A,q, P^1_0)=\begin{pmatrix}
0 & \frac{1}{x \left(x -1\right )^2} 
\\
 (P^1_0)^{2}+P^1_0\left(A -\kappa_1\right) -\rho\left(x -1\right) & -\frac{\kappa_1+1}{x-1}-\frac{\kappa_0}{x} 

\end{pmatrix}
.
\]
\end{proof}
On the other irreducible components we have to apply the Moebius transformation $f_{B_0^1}$.
\begin{lemma}
    The limit $\lim_{q\to1}f_{B^1_0}^*\Omega_0$ produces a well defined connection if we set $P^1_0=-\frac{\hat p}{(q-1)}$.
\end{lemma}
\begin{proof}
    Computing the limit we get: 
    \[\lim_{q\to1}f_{B^1_0}^*\Omega_0(x; A,q,P^1_0)=\begin{pmatrix}
0 & \frac{1}{\left(X -1\right)^{2} } 
\\
 \frac{P^1_0}{X}+\frac{P^1_0 \left(A +P^1_0 -\kappa_1\right)}{X^2} & -\frac{\kappa_1}{X -1}+\frac{\kappa_1+1}{X}+\frac{A}{\left(X -1\right)^{2}} 
\end{pmatrix}
.
\]
\end{proof}
\begin{prop}
    Both connections have the same residual spectral data in the node.
\end{prop}
\begin{proof}
   Computing residual spectral data as shown in the end of Section \ref{resspecd} we get for both connections:
   \[(\kappa_1+1)^2+4P^1_0(A+P^1_0-\kappa_1).\]
\end{proof}
Let us now study connections on those stable curves that present the double pole in the node for both the irreducible components. We will see that here there will not be equality of the residual spectral data.


\medskip
\textbf{Connections on $Q^1$.}
We expect over $Q^1$ a similar behaviour as over $Q^0$, but the fact that the connection shows a double pole at $x=1$, makes everything more tricky. This time we will start by computing the limit for the pull-back connection $f_{Q^1}^*\Omega_0$. We will introduce a new variable for $Q^1$, and we want to compute the direct limit using that variable.
\begin{lemma}\label{tildeP}
    The limit $\lim_{q\to1}f_{Q^1}^*\Omega_0$ produces a well defined connection if we set $P^1=-\frac{\hat p}{qt}$.
\end{lemma}
\begin{proof}
    Computing the limit we get: 
    \[\lim_{q\to1}f_{Q^1}^*\Omega_0(x; t,q, P^1)=\begin{pmatrix}
0 & \frac{1}{\left(X -1\right)^{2}} 
\\
 \frac{ P^1 \left(( P^1 +2) X -1\right)}{\left(X -1\right)^{2} X} & \frac{1}{\left(X -1\right)^{2}}-\frac{1}{X}+\frac{1}{X -1} 
\end{pmatrix}
.
\]
\end{proof}
We remark that $\rho,\kappa_i$ disappear. It is due to the fact that the limit curve $f_{Q^1}(\P^1)$ for $q\to1$ shows the apparent singularity at infinity and the double pole at one. It means that the relative connection is a confluent hypergeometric with trivial monodromy around the to poles, since $M_\infty M_1=Id$ and $M_\infty=Id$. \\We now have to adapt the invariant curves in the new coordinate $ P^1$.
\begin{lemma}\label{invq1P}
    The invariant curves introduced in Theorem \ref{confl} become
    \[P^1=0 \;\;\;\text{ and }\;\;\; P^1=\frac{\kappa_1-2t}{t}(q-1)-1.\]
\end{lemma}
\begin{proof}
    We apply the same procedure of Theorem \ref{confl}, but after the changing of variable $ P^1=-\frac{\hat p}{qt}$.
\end{proof}
We can now prove the following.
\begin{lemma}\label{Q1}
The limit
\[\lim_{q\to1}\Omega_0(x; t,q, P^1)\]
exists if, and only if, during the limit $q\to1$, the point $(q,\tilde P)\in \F_2$ follows the trajectory $(q, \frac{\alpha}{t^2} (q-1)^2+\beta (q-1)+\gamma)$, for $(\beta,\gamma)$ as in Lemma \ref{invq1P}. Moreover the resultant connection depends on $\alpha$.
\end{lemma}
\begin{proof}
    The first part of the Lemma follows from Theorem \ref{confl} and Lemma \ref{invq1P}. Computing the limit for $(\beta,\gamma)=(0,0)$, we get:
    \[\lim_{q\to1}\Omega_0(x; t,q,\frac{\alpha^1_-}{t^2} (q-1)^2 )=\begin{pmatrix}
0 & \frac{1}{x\left(x -1\right)^{2}} 
\\
 -\rho(x-1)+\alpha^1_-  & \frac{-\kappa_1-1}{x -1}-\frac{\kappa_0}{x}+\frac{t}{\left(x -1\right)^{2}} 
\end{pmatrix}.
\]
and, for $(\beta,\gamma)=(\frac{\kappa_1-2t}{t},-1)$:
\[\lim_{q\to0}\Omega_0(x; t,q,\frac{\alpha^1_+}{t^2} (q-1)^2 +\beta(q-1)+\gamma)=\begin{pmatrix}
0 & \frac{1}{x\left(x -1\right)^{2}} 
\\
 -t^{2}+\left(\kappa_0 +2 \kappa_1+\frac{1}{x -1}\right) t -\rho\left(x-1 \right) -\alpha^1_+  & \frac{-\kappa_1-1}{x -1}-\frac{\kappa_0}{x}+\frac{t}{\left(x -1\right)^{2}} 
\end{pmatrix}.\]
In both cases the matrices depends, as expected, on the parameter $t$ defining the stable nodal curve and on the parameter $\alpha$ defining the connection, as desired.
\end{proof}
Therefore, as for $Q^0$, we should consider only those connections in Lemma \ref{tildeP} with $ P^1=0,-1$. 
\begin{oss}
    As mentioned before, we were expecting equality of residual spectral data in the node also in this situation, but it does not happens. For the pull-back connection we get 1, while for the direct limits we get respectively $\kappa_1-1$ and $\kappa_1+1$.
\end{oss}

\medskip


\textbf{Connections on $B_\infty^0$.}
Here, both the irreducible components of the nodal curve are obtained by a Moebius transformation. We recall that we defined $R=q/T$ the coordinate on $B_\infty^0$, where $T=1/t$.
\begin{lemma}
    The limit $\lim_{T\to0}f_{B_\infty^0}^*\Omega_0$ produces a well defined connection if we set $P_\infty^0=-\frac{\hat p}{(RT-1)^2}$.
\end{lemma}
\begin{proof}
    Computing the limit we get: 
    \[\lim_{T\to0}f_{B_\infty^0}^*\Omega_0(x; T,R, P_\infty^0)=\begin{pmatrix}
0 & -\frac{1}{\left(X -1\right) X} 
\\
 -\frac{P_\infty^0 \left(R +1\right)}{R X -R +X}+\frac{P_\infty^0}{X -1}+\frac{P_\infty^0 \left(P_\infty^0 +R -\kappa_0 \right)}{\left(X -1\right)^{2} R} & \frac{-R -1}{R X -R +X}+\frac{\kappa_0 +1}{X -1}+\frac{1}{\left(X -1\right)^{2}}-\frac{\kappa_0}{X} 
\end{pmatrix}
\]
\end{proof}
And the second one.
\begin{lemma}
    The limit $\lim_{T\to0}g_{B_\infty^0}^*\Omega_0$ produces a well defined connection if we set $P_\infty^0=-\frac{\hat p}{(RT-1)^2}$.
\end{lemma}
\begin{proof}
    Computing the limit we get: 
    \[\lim_{T\to0}g_{B_\infty^0}^*\Omega_0(x; T,R, P_\infty^0)=\begin{pmatrix}
0 & \frac{1}{\left(X -1\right)^{3}} 
\\
 -\rho(X-1) +P_\infty^0 +\frac{P_\infty^0 \left(P_\infty^0 -\kappa_0 \right)}{R} & \frac{1}{\left(X -1\right)^{2}}+\frac{-\kappa_0 -\kappa_1-1}{X -1} 
\end{pmatrix}
\]
\end{proof}
\begin{oss}
    As mentioned before, we were expecting equality of residual spectral data in the node also in this situation, but it does not happens. We get respectively
    \[2 P -\kappa_0 -1+\frac{2 P \left(P -\kappa_0 \right)}{R}\;\;\;\text{ and }\;\;\;-2 P +\kappa_0 +\kappa_1+1-\frac{2 P \left(P -\kappa_0 \right)}{R}\]
\end{oss}

\medskip
\textbf{Connections on $B_\infty^\infty$.}
The same procedure as for $B^0_\infty$ can be applied (see Section \ref{symm}). The vertical coordinate that we should use is $P^\infty_\infty=-\frac{\hat p}{(\frac{1}{ST}-1)^2}$, where $S=\frac{Q}{T}$ is the coordinate on $B_\infty^\infty$, $Q=\frac1q$, $T=\frac1t$.

\medskip
\textbf{Connections on $A_\infty$.}
The limit for $t\to\infty$ is more wild than the other limits we studied before. We have indeed two possible approaches to compute it. In Section \ref{seccomp} we give a geometrical interpretation to both of them.\\The first approach is to consider the two Moebius transformations used to derive the stable nodal curve. 
\begin{lemma}
    The limit $\lim_{T\to0}f_{A_\infty}^*\Omega_0$ produces a well defined connection if we set $P_\infty^+=-\frac{\hat p}{(q-1)^2}$ or $P_\infty^-=-\frac{\hat p}{(q-1)^2}-1$.
\end{lemma}
\begin{proof}
    Computing the limit and applying an elementary transformation we get to:, correspectively
    \[\lim_{T\to0}f_{A_\infty}^*\Omega_0(x; T,q, P_\infty^+)=\begin{pmatrix}
0 & -\frac{1}{\left(X -1\right) X} 
\\
 \frac{P_\infty^+}{\left(X -1\right)^{2}} & -\frac{\kappa_0}{X}+\frac{\kappa_0}{X -1}+\frac{1}{\left(X -1\right)^{2}} 
\end{pmatrix}
\]
\[
\lim_{T\to0}f_{A_\infty}^*\Omega_0(x; T,q, P_\infty^-)=\begin{pmatrix}
0 & -\frac{1}{\left(X -1\right) X} 
\\
 \frac{P_\infty^- +1}{\left(X -1\right)^{2}} & -\frac{\kappa_0}{X}+\frac{\kappa_0}{X -1}+\frac{1}{\left(X -1\right)^{2}} 
\end{pmatrix}
\]
\end{proof}
\begin{lemma}\label{Ppm}
    The limit $\lim_{T\to0}g_{A_\infty}^*\Omega_0$ produces a well defined connection if we set $P_\infty^+=-\frac{\hat p}{(q-1)^2}$ or $P_infty^-=-\frac{\hat p}{(q-1)^2}-1$.
\end{lemma}
\begin{proof}
    Computing the limit and applying an elementary transformation we get to, respectively
    \[\lim_{T\to0}g_{A_\infty}^*\Omega_0(x; T,q, P_\infty^+)=\begin{pmatrix}
0 & \frac{1}{\left(X -1\right)^{3}} 
\\
 -\rho(X-1)+P_\infty^+   & \frac{-\kappa_0 -\kappa_1-1}{X -1}+\frac{1}{\left(X -1\right)^{2}} 
\end{pmatrix}
\]
\[
\lim_{T\to0}g_{A_\infty}^*\Omega_0(x; T,q, P_\infty^-)=\begin{pmatrix}
0 & \frac{1}{\left(X -1\right)^{3}} 
\\
 -\rho( X-1) +P_\infty^- +1 & \frac{-\kappa_0 -\kappa-1}{X -1}+\frac{1}{\left(X -1\right)^{2}} 
\end{pmatrix}
\]
\end{proof}
\begin{oss}
    As mentioned before, we were expecting equality of residual spectral data in the node also in this situation, but it does not happens. We get respectively for the "+" curve
    \[2 P_\infty^+ -\kappa_0 \;\;\;\text{ and }\;\;\;-2 P_\infty^+ +\kappa_0 +\kappa_1+1,\]
    While for the "-" curve we get
    \[2 P_\infty^- -\kappa_0 \;\;\;\text{ and }\;\;\;-2 P_\infty^- +\kappa_0 +\kappa_1-1.\]
\end{oss}
We can also have a different approach. Let us consider a Moebius transformation that fixes 0 and set $t=1$. We have a family depending of one parameter of such transformation and we can for instance choose to fix $-1$ (it suffice not to fix 0, $\infty$ or $q$). The curve we get is no more stable, because only the double pole in 1 survived. Anyway it allows us to use the same coordinate we used for $Q^1$, and the limit we get is still interesting.
\begin{prop}\label{Ainfp}
    Let us consider the Moebius transformation 
    \[h_{A_\infty}(x):=\frac{(t+1)x+t-1}{(t-1)x+t+1}.\]
    The limit $\lim_{T\to0}h^*_{A_\infty}\Omega_0(x; T,q,P)$ produces a well defined connection if we use $ P^1=-\frac{\hat p T}{q}$, the same coordinate as in $Q^1$.
\end{prop}
\begin{proof}
    Computing the limit we get to:
    \[
    \Omega_0^{h_{A_\infty}}:=\begin{pmatrix}
0 & -\frac{1}{\left(X -1\right)^{2}} 
\\
 \frac{4 q  P^1 \left( P^1 +1\right)}{\left(q -1\right)^{2} \left(X -1\right)^{2}} & \frac{1}{\left(X -1\right)^{2}} 
    \end{pmatrix}
    \]
\end{proof}
%\begin{oss}
 %   The coordinate $\tilde P=-\frac{\hat p T}{q}$ is the same we used on $Q^1$.
%\end{oss}
\begin{oss}
    The relation between the two coordinates we used is the following
    \[P_\infty^+=-\frac{\hat p}{(q-1)^2}=\frac{ P^1q}{T(q-1)^2}\;\;\;\text{ and }\;\;\;P_\infty^-=-\frac{\hat p}{(q-1)^2}-1=\frac{P^1q}{T(q-1)^2}-1.\]
\end{oss}








\subsection{Extension of the line bundle}
We recall that the final goal is to compactify the space $\Conn=\mathcal M\times \C$. We will do that in several steps. First of all, in this section, we understand how the line bundle structure extends on the boundary components in $\widetilde M$, defining a new line bundle $\O_{\widetilde M}(D)$ for an appropriate divisor $D\in \mathrm{Div}(\widetilde \M)$. Then, we compactify it and we study its birational geometry, that will be useful because we recall that the real base that we want to use is the singular space $\overline \M$, and we will hence have to show that we can contract the surface $\mathcal A_\infty:=\overline{\O_{ A_\infty}(D)}$ producing only canonical singularities. 

The strategy is to deduce the divisor $D$ by studying the poles and zeros of the extension of the global section $\hat p =1$ defined in $\Conn$. In the previous section we have studies the vertical coordinate in the generic point of any boundary component. To find the divisor is more suitable to define a vertical coordinate based on each intersection points of the boundary components. Some computations show that one possible choice is the following:
\begin{center}
    \includegraphics[width=14cm]{image32.jpeg}
\end{center}
Before going into the proof, let us recall that in Proposition \ref{Msing} we have shown that $\overline\M$ is a singular space, due to the fact that the $(-2)$-curve $A_\infty$ has been contracted. We will hence in a first moment extend the line bundle on the smooth surface $\widetilde{\M}$, and then we show that it is possible to contract the surface lying upon $A_\infty$. In what follows we denote by $\mathcal Q^i$ and $\mathcal A_j$ the restrictions of the line bundle over, respectively, $Q^i$ and $A_j$.
\begin{prop}
    The trivial line bundle $\Conn\cong \M\times \C$ extends to $\widetilde \M$ as $\O_{\widetilde{\mathcal M}}(D)$, where $D\in \mathrm{Div}(\widetilde \M)$ is the divisor
    \[D\equiv - B^1_0+2 Q^\infty+2 B^\infty_\infty+ A_\infty \equiv A_0+2Q^1+2B_0^0-B_\infty^0-B_\infty^\infty.\]
\end{prop}
\begin{proof}
    Since it is easier for us to work with a smooth variety, we study in a first moment the extension of the trivial vector bundle $\mathcal M\times \C$ on $\widetilde{\M}$.\\
    By studying the poles and zeros of the section $\hat p=1$ in all the coordinates, we get to the divisor
    \[- B^1_0+2 Q^\infty+2 B^\infty_\infty+ A_\infty, \]
    as desired.\\
    To get the second expression, we can explicitly study the Picard group of $\widetilde{\M}$. \\We now it is the blow-up of $\P^1\times\P^1$ in three points. Therefore, since $\mathrm{Pic}(\P^1\times\P^1)\cong\Z\times\Z$, we have that $\mathrm{Pic}(\widetilde{\M})\cong \Z^5$ and is is for instance generated by
    \[\mathrm{Pic}(\widetilde{\M})\cong [A_0]\cdot\Z+[Q^0]\cdot\Z+[B^1_0]\cdot\Z+[B^0_\infty]\cdot\Z+[B^\infty_\infty]\cdot\Z.\]
    Moreover we know that in $\P^1\times\P^1$ the divisors $A_0$ and $A_\infty$ are equivalent, and the same holds for $Q^0, Q^1$ and $Q^\infty$. Since we know that a holomorphic function $f\colon\P^1\times\P^1\to\C$ vanishing at a point $p\in\P^1\times\P^1$, vanishes also on the exceptional divisor $E_p\subseteq\mathrm{Bl}_p(\P^1\times\P^1)$, if we remark that $A_i=\{t-i=0\}$ and $Q^j=\{q-j=0\}$, we get the following equivalences
    \[\begin{matrix}
        \begin{cases}
        A_0+B^1_0=A_\infty+B_\infty^0+B^\infty_\infty\\
        Q^0+B_\infty^0=Q^1+B^1_0=Q^\infty+B^\infty_\infty
    \end{cases}
    &\implies&
    \begin{cases}
        A_\infty=A_0+B_0^0-B_\infty^0-B^\infty_\infty\\
        Q^0=Q^1+B^1_0-B_\infty^0\\
        Q^\infty=Q^1+B^1_0-B^\infty_\infty
    \end{cases}
    \end{matrix}\]
    And if we replace them in the expression of $D$ we get
    \begin{align*}
        D&=- B^1_0+2 (Q^1+B^1_0-B^\infty_\infty)+2 B^\infty_\infty+ (A_0+B_0^1-B_\infty^0-B^\infty_\infty)=\\&=A_0+2B^1_0+2Q^1-B^\infty_\infty-B^0_\infty
    \end{align*}
    as desired.
\end{proof}
Since the base space we would like to work with is $\overline \M$, that is the contraction of $A_\infty$ inside $\widetilde \M$, we wonder if we can contract the surface $\mathcal  A_\infty:=\overline{\O_{ \widetilde \M}(D)_{|A_\infty}}$. \\ Before doing that, we consider the compactification of the line bundle, in order to work with a compact algebraic 3-manifold. 
\begin{prop}
    The line bundle $\O_{ \widetilde \M}(D)$ is trivial upon $Q^0+ 2Q^1+ Q^\infty+A_\infty$.
\end{prop}    
\begin{proof}
    We recall that for any pair of divisor $D$ and $D'$ it holds that
    \[\deg\O_{D'}(D)=(D\cdot D').\]
    In particular:
    \begin{align*}
         &(D\cdot Q^0)=\Big((- B^1_0+2 Q^\infty+2 B^\infty_\infty+ A_\infty)\cdot Q^0\Big)=0+0+0+0=0,\\
         &(D\cdot 2Q^1)=\Big((- B^1_0+2 Q^\infty+2 B^\infty_\infty+ A_\infty)\cdot 2Q^1\Big)=-2+0+0+2=0,\\
        &(D\cdot Q^\infty)=\Big((- B^1_0+2 Q^\infty+2 B^\infty_\infty+ A_\infty)\cdot Q^\infty\Big)=0-2+2+0=0,\\
         &(D\cdot A_\infty)=\Big((- B^1_0+2 Q^\infty+2 B^\infty_\infty+ A_\infty)\cdot A_\infty\Big)=0+0+2-2=0.
    \end{align*}
    As desired.
\end{proof}  
We already expected this result the for $Q^i$'s: indeed, Lemmas \ref{Q0} and \ref{Q1} give us those such sections. We can hence choose the following vertical coordinates that trivialise the bundle on those boundary components:
\[P^0=-\frac{\hat p}{(q-1)^2}=-\frac{\hat p}{(RT-1)^2},\;\;\;\;\; P^1=-\frac{\hat p}{qt}=-\frac{\hat pT}{q}=-\hat p s=-\hat pr,\;\;\;\;\; P^\infty=-\frac{\hat p}{(\frac{1}{Q}-1)^2}=-\frac{\hat p}{(\frac{1}{ST}-1)^2}.\]
It suffices then to choose a coordinate at the intersection $A_0\cap B^1_0$, that is $P^1_0=-\frac{\hat p}{q-1}$.
\begin{center}
    \includegraphics[width=7cm]{image33.jpeg}
\end{center}
%We denote by $P_{Q^i}$ the global vertical coordinate relative to $Q^i$ and by $P_{A_\infty}$ the global vertical coordinate relative to $A_\infty$ (that is the same as $P_{Q^1}$).\\
With these coordinates, the sections in $\mathcal Q^0$ are given by the equations $P^{0}=0,P^{0}=-\kappa_0$ and in $\mathcal Q^1$ are given by the equations $P^{1}=0,P^{1}=-1$.

Since we are interested in a compactification, we start by defining the compact manifold $X:=\overline{\O_{ \widetilde \M}(D)}=\P(\O\oplus\O(D))$ and we study its birational geometry in order to understand the singularities arising from the contraction of $\mathcal A_\infty$. We recall that $(A_\infty)^2_{\widetilde \M}=-2$.
\begin{prop}
    Let us denote $X:=\overline{\O_{ \widetilde \M}(D)}$. The contraction of $\mathcal A_\infty$ inside $X$ produces canonical singularities.
\end{prop}
\begin{proof}
    First of all let us notice that, since $\O_{ A_\infty}(D)\cong\O$, then $\mathcal A_\infty:=\overline{\O_{ A_\infty}(D)}\cong\P(\O\oplus\O)\cong \P^1\times\P^1$. \\
    We want then to contract the "horizontal" fibers of $\mathcal A_\infty$, and then we have to show that they are $K_X$-trivial, where $K_X$ is the canonical bundle of $X$. Let us call $C_a=\P^1\times\{a\},C_b=\P^1\times\{b\}$ two of those fibers. They are linearly equivalent in $X$ since they are in $\mathcal A_\infty$ and $\dim A_\infty=2$. We identify $A_\infty$ with $C_0$. It is then sufficient to show that $(K_X\cdot A_\infty)=0$. Since $A_\infty\subseteq\widetilde M$, it holds that $(K_X\cdot A_\infty)=(K_X|_{\widetilde \M}\cdot A_\infty)$.\\By the adjunction formula we have that 
    \[K_X|_{\widetilde \M}=K_{\widetilde \M}-\widetilde M,\]
    and therefore
    \[(K_X\cdot A_\infty)=(K_{\widetilde \M}\cdot A_\infty)-(\widetilde M\cdot A_\infty).\]
    We can compute the first product since we know the canonical bundle of $\widetilde  \M$: 
    \[(K_{\widetilde \M}\cdot A_\infty)=((2(Q^0+A_0)-B^1_0-B^\infty_0+B^\infty_\infty)\cdot A_\infty)=0\]
    and the second product is zero since $A_\infty\subseteq \widetilde \M$.\\We have then shown that all the horizontal fibers of $\mathcal A_\infty$ are $K_X$-trivial. We have now to compute their discrepancies in order to conclude the proof. Let $\hat \pi\colon X\to \hat X$ the contraction map. Then,
    \[K_X=\hat\pi^*K_{\hat X}+\alpha\mathcal A_\infty\]
    and the equality must hold also when intersecting with $A_\infty$:
    \[0=(K_X\cdot A_\infty)=(\hat\pi^*K_{\hat X}\cdot A_\infty)+\alpha(\mathcal A_\infty\cdot A_\infty).\]
    The first intersection product is zero, since $A_\infty$ has been contracted, while we can compute the second via the formula $(\mathcal A_\infty\cdot A_\infty)=-2-(A_\infty)^2_{\mathcal A_\infty}=-2+0$, since $\mathcal A_\infty\cong \P^1\times\P^1$ and $A_\infty$ is just a fiber. We conclude hence that $\alpha=0$ and then that the contraction process produces only canonical singularities.
\end{proof}

\subsection{The compactified moduli space}\label{seccomp}
We immediately remark that $X:=\overline{\O_{ \widetilde \M}(D)}$ is not the moduli space we are looking for, and neither is $\hat X$, where $\mathcal A_\infty$ has been contracted. Indeed, there are points inside it that do not represent any PV type connection (as points on the $\mathcal Q^i$'s outside the two special sections, or most of the points in $\mathcal A_\infty$) and there are some points that represent infinite connections (as points in the special sections in the $\mathcal Q^i$'s).\\
As shown in Lemma \ref{Q0}, for $\mathcal Q^0$ (and the same holds for $\mathcal Q^\infty$) the connections appear only on the special sections $P_{Q^0}=0,-\kappa_0$. We recall that the resulting limits depend on the slope of incidence, and therefore each point of these sections encode an one parameter family of connections. We can then blow up these sections and ignore the other part of $\mathcal Q^0$. For $\mathcal Q^1$ the process is similar, but, following the Lemma \ref{Q1}, we need to blow up two times each section to make the connections appear. We can then ignore $\mathcal Q^1$ and the first exceptional divisor of each section. To have a picture in mind, we have exactly a family of Okamoto spaces as in Figure \ref{Okam} for each choice of $t\in \C^*$. In $t=0$ we have the hypersurfaces $\mathcal A_0$ and $\mathcal B^1_0$, while in $t=\infty$ the situation is more complicated.
\begin{oss}
    We get a new sort of Okamoto divisor given by $[\mathcal S_\infty]+[\mathcal Q^0]+[\mathcal Q^1]+[\mathcal Q^\infty]+[\mathcal E^+]+[\mathcal E^-]$ that we can visualise by considering in families the surfaces of Figure \ref{Okam}, with the differences given by the employment of a different vertical coordinate over $\mathcal Q^1$.
\end{oss}
Over $\mathcal A_\infty$ the coordinate $\tilde P$ has been used, which is the one we find in Proposition \ref{Ainfp}. This is a good coordinate to study the geometry of $X$, but the connections defined via this coordinate are not what we are looking for, since they are not connections on the nodal stable curve we found in Section \ref{secstabnod}. In order to make them appear we have to use the coordinates $P^\pm$ introduced in Lemma \ref{Ppm}. since they are related by the formula 
\[ P^1=\frac{ P_\infty^+ T(q-1)^2}{q} \;\;\;\text{ and }\;\;\;  P^1=\frac{ (P_\infty^-+1) T(q-1)^2}{q}\]
it corresponds geometrically to blow-up the sections $ P^1=0,-1$ in $\mathcal A_\infty$.
\begin{center}
    \includegraphics[width=8cm]{CPXGEOM.jpeg}
\end{center}



\subsection{Symmetries}\label{symm}
The construction we made presents some symmetries. The most visible one is the involution $\Phi_{0,\infty}$ exchanging the role of 0 and $\infty$. There are tree others involutions $\Psi_a$, for $a=0,1,\infty,$ exchanging the residual spectral data $\kappa_a^+\leftrightarrow\kappa_a^-$ of each singularity. We can express them in coordinates, on the Zariski open set $\M\times\C\subseteq \Conn$. They are well defined on the whole compactification of the moduli space.
\[\Phi_{0,\infty}=\begin{cases}\kappa_\infty=\kappa_0\\\kappa_0=\kappa_\infty\\q=\frac1q\\\kappa_1=-\kappa_1\\\hat p=\frac q2\frac{-2\hat pq^3+\kappa_0q^2+\kappa_\infty q^2+4\hat pq^2+(\kappa_1-1)q^2-2\kappa_0q-2\kappa_\infty q-2\hat pq-2qt+\kappa_0+\kappa_\infty+2q-\kappa_1-1}{(q-1)^2}\end{cases}\]
and 
\[\Psi_{0}=\begin{cases}t=t\\\kappa_0=-\kappa_0\\\hat p=\hat p-\frac{\kappa_0}{q}\end{cases}\;\;\;\;\Psi_{1}=\begin{cases}t=-t\\\kappa_1=-\kappa_1\\\hat p=\hat p+\frac {t}{(q-1)^2}-\frac{\kappa_1}{q-1}\end{cases}\;\;\;\;\Psi_{\infty}=\begin{cases}t=t\\\kappa_\infty=-\kappa_\infty\\\hat p=\hat p\end{cases}\]
\begin{prop}
    The involutions $\Phi_{0,\infty},\Psi_{0},\Psi_{1}$ and $\Psi_{\infty}$ commute and they generate a transformation group of order 16 acting on $\overline\Conn$.
\end{prop}
\begin{proof}
Explicit computations show that $(\Phi_{0,\infty})^2=(\Psi_0)^2=(\Psi_1)^2=(\Psi_\infty)^2=Id$, and that $\Phi_{0,\infty}\circ\Psi_a=\Psi_a\circ \Phi_{0,\infty}$, for all $a=0,1,\infty$.
\end{proof}
Geometrically, the symmetries $\Psi_i$ correspond to exchange the two exceptional divisors in $\mathcal Q^i$. For instance, in a neighbour of $\mathcal Q^0$, the two invariant curves $\hat p=0$ and $\hat p = \alpha q-\kappa_0$ are switched, and the same happens for the others $\mathcal Q^i$. The symmetry $\Phi_{0,\infty}$ exchanges $\mathcal Q^0$ and $\mathcal Q^\infty$. This is the reason why, in the previous sections, we often skipped the computations for $\mathcal Q^\infty$.




\newpage
\bibliographystyle{plain}
\bibliography{bibliography}







\end{document}
