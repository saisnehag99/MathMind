\documentclass[12pt, 14paper]{amsart}
\vsize=21.1truecm
\hsize=15.2truecm
\vskip.1in
\usepackage{amsmath,amsfonts,amssymb,mathtools}
%\usepackage[T1]{fontenc}
%\usepackage{linespace}

\usepackage{longtable}
\usepackage[mathscr]{eucal}
\usepackage{amsmath, amsthm}
%\usepackage{mathrsfs}
\usepackage{amsbsy}
%\usepackage{dsfont}
%\usepackage{bbm}
\usepackage{wasysym}
%\usepackage{stmaryrd}
\usepackage{url}

%\usepackage[autostyle]{csquotes}
\theoremstyle{plain}
\usepackage{color}



\newenvironment{pf}{\par\noindent{\em Proof}.}{\hfill\framebox(6,6)
\par\medskip}

\newtheorem{theorem}[subsection]{Theorem}
\newtheorem{conjecture}[subsection]{Conjecture}
\newtheorem{proposition}[subsection]{Proposition}
%\newtheorem{equation}[section]{Equation}
\newtheorem{lemma}[subsection]{Lemma}
\newtheorem{sublemma}[subsection]{Sublemma}
\newtheorem{remark}[subsection]{Remark}
\newtheorem{remarks}[subsection]{Remarks}
\newtheorem{definition}[subsection]{Definition}
\newtheorem{corollary}[subsection]{Corollary}
\newtheorem{example}[subsection]{Example}
\newtheorem{examples}[subsection]{examples}
%\newtheorem{equation}[subsection]{Equation}

\numberwithin{equation}{section}


\input xypic
\xyoption{all}
\usepackage[breaklinks]{hyperref}
%\theoremstyle{thmrm}
\newtheorem{exa}{Example}
\newtheorem*{rem}{Remark}
\newcommand \ZZ {{\mathbb Z}}
\newcommand \NN {{\mathbb N}}
\newcommand \RR {{\mathbb R}}
\newcommand \CC {{\mathbb C}}
\newcommand \PR {{\mathbb P}}
\newcommand \AF {{\mathbb A}}
\newcommand \GG {{\mathbb G}}
\newcommand \QQ {{\mathbb Q}}
\newcommand \bcA {{\mathscr A}}
\newcommand \bcC {{\mathscr C}}
\newcommand \bcD {{\mathscr D}}
\newcommand \bcE {{\mathscr E}}
\newcommand \bcF {{\mathscr F}}
\newcommand \bcG {{\mathscr G}}
\newcommand \bcH {{\mathscr H}}
\newcommand \bcM {{\mathscr M}}
%\newcommand \bcI {{\mathscr I}}
\newcommand \bcJ {{\mathscr J}}
\newcommand \bcL {{\mathscr L}}
\newcommand \bcO {{\mathscr O}}
\newcommand \bcP {{\mathscr P}}
\newcommand \bcQ {{\mathscr Q}}
\newcommand \bcR {{\mathscr R}}
\newcommand \bcS {{\mathscr S}}
\newcommand \bcU {{\mathscr U}}
\newcommand \bcV {{\mathscr V}}
\newcommand \bcW {{\mathscr W}}
\newcommand \bcX {{\mathscr X}}
\newcommand \bcY {{\mathscr Y}}
\newcommand \bcZ {{\mathscr Z}}
\newcommand \goa {{\mathfrak a}}
\newcommand \gob {{\mathfrak b}}
\newcommand \goc {{\mathfrak c}}
\newcommand \gom {{\mathfrak m}}
\newcommand \gon {{\mathfrak n}}
\newcommand \gop {{\mathfrak p}}
\newcommand \goq {{\mathfrak q}}
\newcommand \goQ {{\mathfrak Q}}
\newcommand \goP {{\mathfrak P}}
\newcommand \goM {{\mathfrak M}}
\newcommand \goN {{\mathfrak N}}
\newcommand \uno {{\mathbbm 1}}
\newcommand \Le {{\mathbbm L}}
\newcommand \Spec {{\rm {Spec}}}
\newcommand \Gr {{\rm {Gr}}}
\newcommand \Pic {{\rm {Pic}}}
\newcommand \Jac {{{J}}}
\newcommand \Alb {{\rm {Alb}}}
\newcommand \Corr {{Corr}}
\newcommand \Chow {{\mathscr C}}
\newcommand \Sym {{\rm {Sym}}}
\newcommand \Prym {{\rm {Prym}}}
\newcommand \cha {{\rm {char}}}
\newcommand \eff {{\rm {eff}}}
\newcommand \tr {{\rm {tr}}}
\newcommand \Tr {{\rm {Tr}}}
\newcommand \pr {{\rm {pr}}}
\newcommand \ev {{\it {ev}}}
\newcommand \cl {{\rm {cl}}}
\newcommand \interior {{\rm {Int}}}
\newcommand \sep {{\rm {sep}}}
\newcommand \td {{\rm {tdeg}}}
\newcommand \alg {{\rm {alg}}}
\newcommand \im {{\rm im}}
\newcommand \gr {{\rm {gr}}}
\newcommand \op {{\rm op}}
\newcommand \Hom {{\rm Hom}}
\newcommand \Hilb {{\rm Hilb}}
\newcommand \Sch {{\mathscr S\! }{\it ch}}
\newcommand \cHilb {{\mathscr H\! }{\it ilb}}
\newcommand \cHom {{\mathscr H\! }{\it om}}
\newcommand \colim {{{\rm colim}\, }} % colimit
\newcommand \End {{\rm {End}}}
\newcommand \coker {{\rm {coker}}}
\newcommand \id {{\rm {id}}}
\newcommand \van {{\rm {van}}}
\newcommand \spc {{\rm {sp}}}
\newcommand \Ob {{\rm Ob}}
\newcommand \Aut {{\rm Aut}}
%\newcommand \cor {{\rm {cor}}}
\newcommand \Supp {{\rm {supp}}}
\newcommand \Cor {{\it {Corr}}}
\newcommand \res {{\rm {res}}}
\newcommand \red {{\rm{red}}}
\newcommand \Gal {{\rm {Gal}}}
\newcommand \PGL {{\rm {PGL}}}
\newcommand \Bl {{\rm {Bl}}}
\newcommand \Sing {{\rm {Sing}}}
\newcommand \spn {{\rm {span}}}
\newcommand \Nm {{\rm {Nm}}}
\newcommand \inv {{\rm {inv}}}
\newcommand \codim {{\rm {codim}}}
\newcommand \Div{{\rm{Div}}}
\newcommand \sg {{\Sigma }}
\newcommand \DM {{\sf DM}}
\newcommand \Gm {{{\mathbb G}_{\rm m}}}
\newcommand \tame {\rm {tame }}
\newcommand \znak {{\natural }}
\newcommand \lra {\longrightarrow}
\newcommand \hra {\hookrightarrow}
\newcommand \rra {\rightrightarrows}
\newcommand \ord {{\rm {ord}}}
\newcommand \Rat {{\mathscr Rat}}
\newcommand \rd {{\rm {red}}}
\newcommand \bSpec {{\bf {Spec}}}
\newcommand \Proj {{\rm {Proj}}}
\newcommand \pdiv {{\rm {div}}}
\newcommand \CH {{\it {CH}}}
\newcommand \wt {\widetilde }
\newcommand \ac {\acute }
\newcommand \ch {\check }
\newcommand \ol {\overline }
\newcommand \Th {\Theta}
\newcommand \cAb {{\mathscr A\! }{\it b}}
\newcommand\Mycomb[2][^n]{\prescript{#1\mkern-0.5mu}{}C_{#2}}

\usepackage{graphicx}
\begin{document}
\title{Class groups of imaginary biquadratic fields}
\author{Kalyan Banerjee,  Kalyan Chakraborty, Arkabrata Ghosh }
\address{SRM University AP}
 \email{kalyan.ba@srmap.edu.in}
 \email{kalyan.c@srmap.edu.in}
  \email{arkabrata.g@srmap.edu.in}
\keywords{Diophantine equations;  congruent numbers; class groups; elliptic curve; polygonal numbers; rank of elliptic curve; quadratic Fields}
\subjclass[2020] {11D25, 11G05, 11R11, 11R29, 14G05}
\maketitle

\begin{abstract}
We present two distinct families of imaginary biquadratic fields, each of which contains infinitely many members, with each member having large class groups. Construction of the first family involves elliptic curves and their quadratic twists, whereas to find the other family, we use a combination of elliptic and hyperelliptic curves. Two main results are used, one from Soleng \cite{S} and the other from Banerjee and Hoque \cite{BH}.

%More precisely, we have shown that if a twisted elliptic curve has an $n$-torsion where $n \in \mathbb{N}$, then there exist infinitely many bi-quadratic fields with their class groups having an element of order $n$ or $2n$. \\
%In addition, we have exhibited a family of certain biquadratic fields with an element of order $ln$ in the class group, where $l$ is a positive integer. This family arises from hyper-elliptic curves.  


\end{abstract}

\section{Introduction}
The class number of number fields is one of the most intriguing and fundamental questions associated with field extensions. 
%Many researchers study the class groups of the number fields and their structure. 
It is well known that there exist infinitely many quadratic fields, each with a class number divisible by a particular integer. The Cohen-Lenstra heuristic \cite{CL} predicts a certain behavior of the asymptotics of the class number of infinitely many real
quadratic fields.  We refer to \cite{FK06} for additional related work in this direction.

In this paper, we study the class group of imaginary biquadratic fields. Currently, there is only a handful of literature available relating to class groups of biquadratic fields. 
%have been conducted on the real biquadratic fields,
%and anyone interested can look into 
Sime in \cite{Sim} and \cite{Sim1} studied the class group of real biquadratic fields, but very little is known about the class group of imaginary biquadratic fields, and to our knowledge, Athaide et.al.\cite{ACT} is the only work.  

In this direction, we should also mention that the works of Buell \cite{B}, Buell-Call \cite{BC}, Griffin-Ono-Tsai \cite{GOT}, are very interesting and important from the analytic perspective of class groups and its relations with elliptic curves.

This work is motivated towards shedding some light on the class group of imaginary biquadratic fields, and these fields (that we construct) are kind of geometric in the sense that their origin is from elliptic and hyperelliptic curves. The novelty of this work lies that we are extending the Soleng's technique over imaginary quadratic fields and use the result in \cite{BH} to produce element in the class groups of imaginary biquadratic fields.

Let us begin by briefly stating our plan to move from curves to imaginary biquadratic fields. 

%We start with an elliptic curve $E$ defined over $\QQ$ and then twist it over $\QQ(\sqrt{-d})$ where $d$ is a positive square-free integer. We obtain an elliptic curve $E_d$ 
%defined over $\QQ$ 

We start with an elliptic curve $E$ defined by
\begin{equation}
\label{eq:1.1}
    E: y^2 = x^3 + ax +b,
\end{equation}

where $a,b \in \QQ$. Taking a positive square-free integer $d$ and twisting $E$ over $\QQ(\sqrt{-d})$,
we shall get an elliptic curve,
\begin{equation}
\label{eq:1.2}
E_d: y^2 = x^3 +d^2 ax + d^3 b,
\end{equation}
and it is defined over $\QQ(\sqrt{-d})$ for $E_d$.

 We fix the notation of the twisted elliptic curve as $E_d$ over $\QQ(\sqrt{-d})= K$. In the sequel, $cl(K)$ denotes the class group of $K$.

 
 
 Note that if $P = (x,y)$ 
 is a point on $E$, 
%given by \eqref{eq:1.1}, 
then $\bigg((x+s), y\bigg)$ is a point on 
\begin{equation}
\label{eq:1.3}
y^2 = x^3 -3x^2s + (3s^2 +a)x + (-s^3 -as + b)
\end{equation}
for any integer $s$.
A rational point $p=(x,y)=\bigg(A/C^2, B/C^3 \bigg)$ (here $A,B$ and $C$ are integers) gives a proper ideal $ (A, -kB + \sqrt {-s^3 -as + b})$
(where $k \in \ZZ$ and satisfies $kC^3 + s_1 A = 1$ for some $s_1 \in \ZZ$) in 
%a family of 
imaginary quadratic field $\QQ (\sqrt{-s^3 -as + b})$ with discriminant $D= 4( -s^3 -as + b)$ which is not a perfect square.


Thus, taking a torsion point of $E_d$ over $\mathbb{Q}$ and applying Soleng's result (Theorem $4.1$, \cite{S}), we can go from elliptic curves to class groups of imaginary quadratic fields. Then by going up/going down theorem (\cite{DF07}, Chapter 15, \S, 3, Theorem 26), we prove the following result. 
%for imaginary biquadratic fields.
\begin{theorem}
\label{Thm:1.1}
Let $d$ be a square-free integer and let $E_d$ be the quadratic twist of an elliptic curve $E$, defined over $\QQ(\sqrt{-d})$ such that $d^3 b$ is not a square. Let $E_d(\QQ)$ has a primitive element of order $n\leq 16$. Then there are infinitely many biquadratic fields $\QQ(\sqrt{-d},\sqrt{-p^3-d^2pa+d^3b})$ such that the class group of each of them contains an element of order $n$ or $2n$ with $n\neq 2$, provided that $(-p^3-d^2pa+d^3b)$ is not a perfect square.
\end{theorem}
%Then we move on to 

Here primitive torsion point of an elliptic curve is defined in the beginning of \S2.

In the second part, we consider the hyperelliptic curve
\begin{equation}
\label{eq:1.4}
    y^2 =k^2 -x^l
\end{equation}

for large $l ~\text{and}~ k \in \mathbb{Z}$. Then, using the construction described in Theorem $5.1$ (see\cite{BH}), we get an element of order $l$ in the class group of $\QQ(\sqrt{k^2-p^l})$. 
%for infinitely many primes $p$. 
%\textbf{Here and afterwards, I have changed Kalyan da's notion and replace $n$ by $l$ and $m $ by $n$. Its main effect will be seen in the introduction and proof of Theorem $1.2$.}\\
 
 This gives us a family of infinitely many imaginary quadratic fields (there exist infinitely many primes $p$ such that the expression $(k^2-p^l)$ is negative and  is not a square). 
 Thus, by the construction of Soleng (\cite{S}), 
 %we find that 
 $\QQ(\sqrt{-q^3-d^2qa+d^3b})$ has an element of order $n$, provided that the elliptic curve $E$ given by \eqref{eq:1.1} has a $n$-torsion (here $n \leq 16$ according to Mazur's theorem). Next, we consider the compositum of the fields 
$$
\QQ(\sqrt{k^2-p^l}) ~\text{and}~ \QQ(\sqrt{-q^3-d^2qa+d^3b})
$$
to obtain a biquadratic field $L$ that has an element of order $ln$. This is possible only if 
$$
 \QQ(\sqrt{k^2-p^l})\cap \QQ(\sqrt{-q^3-d^2qa+d^3b})=\QQ,
$$
that is, they are linearly disjoint. The second result is the following.
\begin{theorem}
\label{Thm:1.2}
 The imaginary biquadratic field $\QQ(\sqrt{k^2-p^l}, \sqrt{-q^3-d^2qa+d^3b})$ has an element of order $ln$ in the class group, where $n\leq 16$ and $n\neq 2$. 
\end{theorem}

In \S 2, we shall briefly discuss Soleng's technique (\cite{S}), which is the stepping stone of this article. Then in \S 3, we shall prove Theorem \eqref{Thm:1.1} followed by an example. Finally, in \S 4, we shall prove Theorem \eqref{Thm:1.2}


\section{Soleng's technique}
Let $P=(x,y)$ be a rational point on the elliptic curve 
\begin{equation}
\label{eq:2.1}
    Y^2=X^3+a_2X^2+a_4X+a_6 
\end{equation}
defined over $\ZZ$. According to Soleng \cite{S}, $P$ is a primitive element if $x,2y$ and $x^2+a_2x+a_4$ are pairwise coprime. By convention, the point at infinity is a primitive point.

The following Proposition (\cite{S}, Proposition 2.1, Page 3) confirms the group structure of these points on the curve.
\begin{proposition}
\label{prop:1}
The set of all primitive points in $E(\QQ)$ is a group. 
\end{proposition}
Let $P=(A/C^2, B/C^3)$ be a point on \eqref{eq:2.1}, then $(A,B)$ is a point on the elliptic curve 
$$
Y^2=X^3+a_2C^2X^2+a_4C^4X+a_6C^6.
$$
Here $a_6\neq 0$ and it is not a perfect square.


Let us consider the integral ideal given by 
$$
I=[A,-B+C^3\sqrt{a_6}].
$$
 Then $I$ defines a proper integral ideal in the order of $\QQ(\sqrt{a_6})$.
The following homomorphism (\cite{S}, Theorem 2.1) gives the desired connection. Let $a_6$ be not a perfect square. Then we have the following theorem.

\begin{theorem}[Theorem 2.1 \cite{S}]\label{thm2}
The homomorphism 
    $$\Phi: E(\QQ)_{prim}\to cl(\QQ[\sqrt{a_6}])$$
is given by 
$$(A/C^2, B/C^3)\to (A,-kB+\sqrt{a_6})$$
where $k$ is an integer satisfying 
$$
kC^3+sA=1
$$
for some $s\in \ZZ$. Here, $cl(\QQ[\sqrt{a_6}])$ denotes the class group of the field $\QQ(\sqrt{a_6})$.
\end{theorem}
%This $\Phi$ is a homomorphism . 

Therefore, if we start with an element of order $n$ in $E(\QQ)$ which is a primitive element, then the image of this element under the homomorphism $\Phi$ is of order at most $n$. This homomorphism restricted on non-trivial torsion elements of the group $E(\QQ)_{prim}$  is non-trivial.

\section{The connection between the class groups of biquadratic fields and elliptic curves}
Soleng's main theorem \cite{S}, gives the connection between the rational points of the elliptic curves and the class group of certain number fields. 

%Let us state the main Theorem ( \cite{S}, Theorem $4.1$) that we are going to use in this section:
%\begin{theorem}%\label{thm1}
%\label{thm1}Let $P_1=(A_1/C_1^2, B_1/C_1^3), P_2=(A_2/C_2^2, B_2/C_2^3)$ be two rational points on the elliptic curve given by the equation $y^2=x^3+ax+b$ where $a,b$ are integers. Suppose further that on the elliptic curve $P_1\neq \pm P_2$, then these points correspond to two inequivalent quadratic forms with the same discriminant $D=4(-n^3+an-b)$ and hence they correspond to two different ideal classes of the class group of the number field $\mathbb{Q}(\sqrt{-n^3+an-b})$ for large $n$.
%\end{theorem}


%Having this information let us come back to the elliptic curve originated from considering the perfect squares, obtained above $y^2=x^3-x\;.$ Now suppose $P$ be a torsion point on the curve.
%Let us consider $E(\mathbb{Q})_{tors}$ to be the set of rational torsion points on the above elliptic curve. It is isomorphic to $\ZZ_2\times \ZZ_2$. Let $P$ be one such point and suppose $P, Q$ are two different $2$-torsion.  Then neither $P=Q$  nor $P=-Q$. Therefore we can apply Soleng's Theorem \ref{thm1} and we have two in-equivalent quadratic forms corresponding to the points $P, Q$ and so they give two different elements in the class group $\mathbb{Q}(\sqrt{-n^3-n})$ for large values of $n$. Due to a result of Hooley \cite{H76} such numbers $n^3+n$ are square-free for infinitely many $n$ and hence they indeed give rise to a family of imaginary quadratic fields.



%\begin{theorem}
%Let $d$ be a square-free integer and let $E_d$ be the quadratic twist of the elliptic curve $E$ over $\QQ(\sqrt{-d})$. Suppose $E_d(\QQ)$ has an $n$-torsion. Then there are infinitely many fields of the form $\QQ(\sqrt{-d},\sqrt{-p^3-d^2p+d^3b})$ such that the class group of them contains an element of order $n$ or $2n$.
%\end{theorem}

Now using Theorem (\ref{thm2}), we shall prove Theorem \eqref{Thm:1.1}.

First, we prove the following lemma.

\begin{lemma}
\label{lemma1}
Let $E_d$ be the twist of the elliptic curve $E:y^2=x^3+ax+b$ defined over $\QQ(\sqrt{-d})$,  given by 
$$y^2=x^3+d^2ax+d^3b$$
Then the above homomorphism $\Phi$ is non-trivial from $(E_d)_{prim}$ to 
$$\QQ(\sqrt{-p^3-d^2p+d^3b})$$ for infinitely many primes $p$ (i.e. except for finitely many primes $p$) with loss of primitivity.
\end{lemma}
\begin{proof}
We need to verify that the hypothesis of Theorem \ref{thm2} is satisfied for infinitely many primes $p$ (except for finitely many primes with loss of primitivity). The equation of the twisted elliptic curve is 
$$y^2=x^3+d^2ax+d^3b$$
If $(x,y)$ is a point in this curve then 
$$(x+p,y)$$ is a point on the curve given by the equation
$$y^2=x^3-3px^2+(3p^2+d^2a)x+(-p^3-d^2p+d^3b)$$
Let us call this curve as $E_{1p}$. By a result of\cite{H76}, we know that the number 
$$(-p^3-d^2p+d^3b)$$
is not a perfect square for infinitely many primes $p$ / or except for a finite set of primes. This can also be proven by using a celebrated result of Siegel \cite{Si29}, which states that, any algebraic curve with genus bigger than zero has finitely many integral points on it. Thus  the elliptic curve

\begin{equation}
\label{eq:3.1}
    y^2=x^3+d^2x+d^3b 
\end{equation}

has only finitely many integral solutions. So for any prime $p$ for which the curve \eqref{eq:3.1} has an non integral solution, the expression 
$$(-p^3-d^2p+d^3b)$$
will not a perfect square and these happens except finitely many primes $p$. Hence, in this way we can establish the positive density of the set of primes such that the above number is squarefree.

The other thing we need to check is that if $(x,y)$ is a primitive point on $E_d$, then $(x+p,y)$ is a primitive point on $E_{1p}$. As $(x,y)$ is a primitive point,  we know that $x,2y, x^2+d^2a$ are pairwise coprime. Thus, it follows that $x+p,2y,(x+p)^2+d^2a$ are pairwise coprime for infinitely many primes $p$. Hence $(x+p,y)$ defines a primitive point on $E_{1p}$ for each such prime $p$, and then we can follow Soleng's construction of $\Phi$.
\end{proof}

Now using Theorem (\ref{thm2}) and the above lemma, we shall prove Theorem \eqref{Thm:1.1}.


\begin{proof}
Let us take an elliptic curve $E$ defined over $\QQ$ given by the equation \eqref{eq:1.1}
%$$y^2=x^3+ax+b$$
and we consider its twist over the field $\QQ(\sqrt{-d})$, denoted by $E_d$, which is given by the equation \eqref{eq:1.2}.
%$$y^2=x^3+d^2ax+d^3b$$
This quadratic twist is defined over $\QQ(\sqrt{-d})$.
Now, if we consider the elliptic curve $E_d$ and a non-trivial torsion $\alpha$ on $E_d$, then the specialization of the element $\alpha$ at the point $-p$, denoted by $\alpha_p$ is in the class group of $F=\QQ(\sqrt{-p^3-d^2p+d^3b})$ by the main theorem of Soleng \cite{S} and by \ref{lemma1}. Then by the going up/going down theorem for integral extensions (see \cite{DF07}, Chapter 15, \S 3, Theorem 26), we have an ideal $\beta_p$ in $cl(FK)$ such that $N_{FK/K}(\beta_p)=\alpha_p\neq 0$, and then we have 
$$j_{FK/K}N_{FK/K}(\beta_p)=2\beta_p\;.$$
Here $j_{FK/K},N_{FK/K}$ denote the transfer and the Norm homomorphism at the level of class groups of number fields.


\begin{lemma}
Suppose $\alpha_p$ is of order $n\neq 2$, then we have $\beta_p $ is of order $n$ or $2n$.
\end{lemma}

\begin{proof}
This follows from the identity
$$N_{FK/K}(\beta_p)=\alpha_p$$
and 
$$j_{FK/K}N_{FK/K}(\beta_p)=2\beta_p\;.$$
Therefore, if $n\alpha_p=0$ we have 
$$N_{FK/K}(n\beta_p)=0$$
and hence we have 
$$j_{FK/K}N_{FK/K}(n\beta_p)=2n\beta_p=0$$
So the order of $\beta_p$ divides $2n$. Now suppose that the order is less than $2n$. From the above identity it follows that $n\beta_p$ can be zero if it is not zero then $2n\beta_p=0$. So, in the case when $n\beta_p\neq 0$ we have $2n\beta_p=0$. If the order of $\beta_p$ is less than $2n$ in this case, we have the order divides $2n$ but not $n$. Hence the order is $2$. Then $2\alpha_p=0$ but $n$ is the order of $\alpha_p$ which is not equal to $2$. Therefore we achieve a contradiction. 

In the other case, $n\beta_p=0$, so the order of $\beta_p$ is less than $n$, say $m$. Then by the equality:

$$N_{FK/K}(\beta_p)=\alpha_p$$
we have
$$m\alpha_p=0$$
which is not possible as the order of $\alpha_p$ is $n$ and $m<n$. Therefore the lemma is proven.


\end{proof}

\end{proof}



Now we shall give one, example to illustrate Theorem \eqref{Thm:1.1}


%\subsection{Examples}

    
\begin{example}
Let us consider the elliptic curve $E$ given by the equation 
$$E :~ y^2=x^3 + 16$$
over the rational numbers and its twist over 
$$E_d: ~ \QQ(\sqrt{-3})\;.$$
The equation of the twist $E_{-3}$ is given by 
$$y^2=x^3-432\;.$$

%The $x$ coordinate of the torsion points is 
%$3.$
Here, $E_{-3}(\mathbb{Q})_{tors} \cong \mathbb{Z}/3$.

 If we specialize the torsion point in $E_{-3}(\QQ)$ generating $\ZZ/3$, on the class group of $L=\QQ(\sqrt{-3}, \sqrt{-p^3-432})$ for a prime $p$,  we get a non-trivial element in the class group of $L$.

 If we call this element $\alpha$, then by the previous theorem \eqref{Thm:1.1}, the order of $\alpha$ is $3$ or $6$.
 
 %the going up/going down theorem for integral extensions (\cite{DF07}, Chapter $15$, section $3$, Theorem $26$), we have that $\alpha$ is the pull-back of an ideal class $\beta$ in the class group of $L$. Let $j$ be the inclusion of $K\subset L$, where $K=\QQ(\sqrt{-p^3-27})$.
%Then $$N_{L/K}(\beta)=\alpha\neq 0$$
%and hence we have 
%$$j_{L/K}N_{L/K}(\beta)=2\beta$$
%So either $2\beta=0$ or not, and since $2\alpha=0$, we have $4\beta=0$. So in this process, we obtain an element of order 2 or 4 in the class group of the biquadratic field $L$.
%\subsection{A reverse construction}
\end{example}


Such examples of elliptic curves can be found in the LMFDB database \cite{LMFDB}, and for each of them, we can make computations related to Soleng's result and produce elements in the class groups of imaginary quadratic fields.




    
%\end{proof}



\section{Proof of Theorem ~\texorpdfstring{\eqref{Thm:1.2}}{Thm:1.2}}
\begin{proof}
    

In the proof of Theorem \eqref{Thm:1.2}, we shall use the techniques of \cite{BH} to prove that $\QQ(\sqrt{k^2-p^l})$ has an element of order $l$ in the class group. Also, by Soleng's technique, $\QQ(\sqrt{-q^3-qd^2a+d^3b})$ has an element of order $n$ where $n\leq 16$ in the class group. Then the compositum has an element of order $ln$ in the class group, provided the intersection of the individual fields in the compositum is $\QQ$.  This is equivalent to $\QQ(\sqrt{k^2-p^l})$ and $ \QQ(\sqrt{-q^3-qd^2a+d^3b})$ are linearly disjoint field extensions $\QQ$. This amounts to showing :

\begin{lemma}The fundamental discriminants satisfy
$$gcd(k^2-p^l, -q^3-qd^2a+d^3b)=1$$ for infinitely many primes $p,q$, such that both of the above numbers are squarefree and congruent to $1$ modulo $4$.
\end{lemma}

\begin{proof}
The claim mentioned above is true  because if for infinitely many primes $p,q$ the above numbers have a common divisor, then we have 
$$d'|k^2-p^l$$
$$d'|-q^3-qd^2a+d^3b$$
and $d'>1$. 
On the other hand, we have $$ -sp^l + t (-q^3) = 1$$
for some integers $s,t$. So, 
$$d'|s(k^2-p^l)+t(-q^3-qd^2a+d^3b)$$
 hence we have 
$$d'|1+sk^2+td^3b-tqd^2a$$
So we have 
$$1+sk^2+td^3b-tqd^2a=md'$$
So, it follows that $d'$ is not a common factor of  $k^2, d^3b-qd^2a$. Since one of the above numbers is $k^2$ which is fixed, then the number of common factors of the two above mentioned numbers is fixed and finite. Hence, the result follows.


\end{proof}
%\textbf{I have some issues regarding the next statement. Fields that are not equal are not the same statement as their intersection is the base field. For example $F_1 =\QQ(\sqrt{4}) $ is clearly not equal to $F_2 =\QQ(\sqrt{2})$. Here $F_1 \cap F_2 = \QQ(\sqrt{4})$. So we need to give some explanation why they are the same in our scenario}.




%That is, $\QQ(\sqrt{k^2-p^l})\neq \QQ(\sqrt{-q^3-qd^2a+d^3b})$. This amounts to showing that 
%$$k^2-p^l\neq -q^3-qd^2a+d^3b$$
%for infinitely many primes $p,q$.  



%\textbf{The above sentence may not be correct as per my understanding, as the definition of gcd says there exist some integers that satisfy this relation when $p$ and $q$ are relatively prime. I fail to understand why this has to be $l$ and $n$ ( or $n$ or $m$ as per Kalyan da's notation. It must be a notation issue.} 




\end{proof}








\section*{Funding and Conflict of Interests/Competing Interests} The author has no financial or non-financial interests to disclose that are directly or indirectly related to the work. The author has no funding sources to report.

\section*{Data availability statement} No outside data was used to prepare this manuscript.



\section*{Acknowledgement}
The authors thank SRM University-AP for providing a suitable atmosphere and support to carry out this research work. The authors thank the anonymous referee for careful reading and suggestions to improve the manuscript.

%\section*{Data availability statement} No outside data was used to prepare this manuscript.

\begin{thebibliography}{30}

\bibitem{ACT} Elizabeth Athaide, Emma Cardwell and Christina Thompson, \textit{Class Number Formulas for Certain Biquadratic Fields}, Hardy Ramanujan Journal, \textbf{46}~(2023), 63-89.
 
 \bibitem{An05} Antoniewicz, A., \textit{On a family of elliptic curves}, Universitatis Iagellonicae Acta Mathematica, \textbf{1285}~(2005), 21–32.

 \bibitem{BH} K. Banerjee and A. Hoque, \textit{ Chow groups, pull back and class groups}, Monatsh Math (2024).\url{doi: 10.1007/s00605-024-02008-3}.
 
\bibitem{BC} Buell, D., Call, G., \textit{Class pairings and isogenies on elliptic curves}, Journal of Number Theory, \textbf{167}~(2016), 31-73.

\bibitem{B} Buell, D., \textit{Elliptic Curves and Class Groups of Quadratic Fields}, Journal of LMS, \textbf{2}~(1977), no. 15, 19-25.

\bibitem{CL} Cohen, H., Lenstra, H.(Jr), \textit{Heuristics on class groups of number fields}, Number Theory Noordwijkerhout, 1984, Lecture notes in Mathematics, 33-62.

 
 %\bibitem{BP99} Behera, A., Panda, A., \textit{On the square roots of triangular numbers}, Fibonacci Quart., \textbf{37}, 1999, 98-105. 

% \bibitem{CGP18} Chahal, Jasbir S.,  Griffin, M., and Priddis, N., \textit{When are Multiples of Polygonal Numbers again Polygonal Numbers?}, Hardy-Ramanujan Journal, \textbf{41}, 2018, 58-67.

 %\bibitem{D71} Dickson, L., \textit{History of the Theory of Numbers }, Vol. II, reprinted by Chelsea, 1971.

 \bibitem{DF07} {Dummit}, D.S., {Foote}, R., \textit{Abstract Algebra}, 3rd Edition,  Wiley , 2004.


 \bibitem{DM97} Darmon, H., Merel, L., \textit{Winding quotients and some variants of Fermat's Last Theorem}, J. Reine Angew. Math., \textbf{490}~(1997), 81-100.

 %\bibitem{G97} Gyory, K., \textit{On the {D}iophantine equation $ \Mycomb[m]{k} = x^l$}, Acta Arith., \textbf{80}(3), 1997, 289-295.

 %\bibitem{ES} Erdos P., Suranyi, J., \textit{Selected Topics in Number Theory}, 2nd ed., Szeged, 1996.

 \bibitem{H76} Hooley, C., \textit{Application of Sieve Methods to the Theory of of Numbers}, Cambridge Univ. Press, London, New York, 1976.

 %\bibitem{Jo12} Jones, M.A., \textit{Proof without words: The square of a balancing number is a triangular number}, Coll. Math. J., \textbf{43}, 2012, 212.


% \bibitem{KOB84} Koblitz, N., Introduction to Elliptic Curves and Modular Forms, Grad. Texts in Math. \textbf{97}, Springer, New York, 1984.

   % \bibitem{MS23} Majumdrer, D., Sury, B., \textit{Fruit Diophantine equation}, Math. Gaz., \textbf{107}(569), 2023, 302-306.

    %\bibitem{MO88} Mohanty S. P., \textit{Integer points on $y^2=x^3-4x+1$}, J. Number Theory, \textbf{30}, 1988, 86-93.
    
  % \bibitem{Ne16} Nelsen, Roger B., \textit{Multi-Polygonal numbers}, \textbf{89}(3). 2016, 159-164.
   
    %\bibitem{BM02} Brown, E., Myers, B.T., \textit{Elliptic curves from Mordell to Diophantus and back}, The American mathematical monthly, \textbf{109}(7), (2002), 639–649.

   % \bibitem{PK24} Prakash, Om, Chakraborty, K., \textit{Generalized fruit diophantine equation and hyperelliptic curves}, Monatsh. Math., \textbf{203}(3), 2024, 667-676.

   %\bibitem{Sa} SAGE Software, version 9.3, \textit{http://www.sagemath.com}.


  \bibitem{FK06} Fouvry, É., Klüners, J., \textit{Cohen–Lenstra Heuristics of Quadratic Number Fields.} In: Hess, F., Pauli, S., Pohst, M. (eds) Algorithmic Number Theory. ANTS 2006. Lecture Notes in Computer Science, \textbf{4076}, Springer, Berlin, Heidelberg. $https://doi.org/10.1007/11792086_4$.
  
  \bibitem{GOT} Griffin, M., Ono, Ken, Tsai, Wei-Lun, (2021), \textit{Quadratic twists of elliptic curves and class numbers}, Journal of Number Theory, \textbf{227} (2021),  1-29.

  

  \bibitem{LMFDB} https://www.lmfdb.org/EllipticCurve/Q/.

   \bibitem{S} R.Soleng \textit{ Homomorphisms From the Group of Rational Points On Elliptic Curves to Class Groups of Quadratic Number Fields}, Journal of Number Theory, \textbf{46} (1994), no. 2, 214-229. 

   \bibitem{Si29} C. L. Siegel, Uber einige Anwendungen Diophantischer Approximationen, Abh. Preuss. Akad.
Wiss. Phys. Math. Kl. \textbf{1} (1929), 1-70; Ges. Abh., Band \textbf{1}, 209–266.

   \bibitem{ST15} J.H. Silverman and J.T. Tate, \textit{Rational points on Elliptic Curves}, 2nd edition, Springer, 2015.

   \bibitem{S09} J.H. Silverman, \textit{The arithmetic of Elliptic Curves}, 2nd edition, Springer, 2009.
   \bibitem{Sim} P. Sime, \textit{On the ideal class groups of real biquadratic fields}, Transactions of AMS, \textbf{347}~(1995), no. 12, 4855-4876.

   
   \bibitem{Sim1} P. Sime, \textit{Hilbert Class Fields of Real Biquadratic Fields}, Journal of Number Theory, \textbf{50}~(1995), no. 1, 154-166.

   

  % \bibitem{TSP78} \textit{Triangular-square-pentagonal numbers}, Problem E $2618$, Amer. Math. Monthly,  \textbf{85}, 1978, 51–52.

   %\bibitem{VS22} Vaishya, L., Sharma, R., \textit{A class of fruit {D}iophantine equations}, Monatsh. Math., \textbf{199}(4), 899-907, 2022.

   

%\bibitem{Sa} SAGE Software, version 9.3, \textit{http://www.sagemath.com}.
  
  
\end{thebibliography}

   








\end{document}
