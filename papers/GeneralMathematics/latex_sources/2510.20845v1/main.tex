\documentclass[3p]{article}
\usepackage[utf8]{inputenc}
\usepackage{amsmath,amssymb,amsthm}
\usepackage{geometry}
\usepackage{hyperref}
\usepackage{mathtools}   % small math niceties
\usepackage{microtype}   % nicer PDF text spacing
\usepackage{graphicx}    % For figures
\usepackage{float}       % For figure placement
\usepackage{caption}     % For better caption handling
\usepackage{subcaption}  % For subfigures if needed
\usepackage{amsmath,amssymb,amsthm} % good to have for \top, theorem envs, etc.

\DeclareMathOperator{\Tr}{Tr}        % defines \Tr as a proper math operator
\DeclareMathOperator{\sgn}{sgn}
\DeclareMathOperator{\pv}{p.v.}

\newcommand{\cstar}{\kappa_{\mathrm{Schur}}}
\newcommand{\Hess}{\mathsf{H}}
\newcommand{\mrho}{m_\rho}
\newcommand{\mrhosq}{m_\rho^{2}}

\title{Schur-Convex Curvature on Dihedral Exponential Families \\ and the Golden-Ratio Stationary Point}

\author{Michael A. Bruna}
\date{August 23, 2025}

% Theorem environments
\newtheorem{theorem}{Theorem}[section]
\newtheorem{lemma}[theorem]{Lemma}
\newtheorem{proposition}[theorem]{Proposition}
\newtheorem{corollary}[theorem]{Corollary}
\newtheorem{example}[theorem]{Example}

\theoremstyle{remark}
\newtheorem{remark}[theorem]{Remark}

\begin{document}

\maketitle

\begin{abstract}
We investigate the Schur--complement curvature of $D_N$--equivariant folded exponential families on the simplex. Our main structural results are: (i) the curvature $\kappa_{\mathrm{Schur}}(\theta)$ is convex in the log--parameter $\theta = \ln q$; (ii) it admits a unique stationary point at the golden ratio value $q^\star = \varphi^{-2}$ (in particular for $N=12$); and (iii) it obeys a quadratic folded law
\[
\kappa_{\mathrm{Schur}} = A(N,m_\rho^2) I_1^2 + B(N,m_\rho^2)\,\big(I_2 - I_1^2\big),
\]
with coefficients $A,B$ determined explicitly by the projector metric of radius $m_\rho^2$. Taken together, these results show that convexity and symmetry alone enforce both the location and the functional form of the ``golden lock--in.''

Beyond their intrinsic interest, these findings identify $D_{12}$ as the minimal dihedral lattice where parity (mod 2) and three--cycle (mod 3) constraints coexist, producing a structurally stable equilibrium at the golden ratio. This places the golden ratio not as an accident of parameterization but as a necessary consequence of convex geometry under dihedral symmetry. Possible applications include harmonic analysis on group orbits, invariant convex optimization, and the structure of tilings or quasicrystal--like systems.
\end{abstract}

\section{Introduction}

The Schur complement is a classical construction in matrix analysis \cite{horn2013matrix}, convex optimization \cite[§3.2.2]{boyd2004convex}, and matrix inequalities \cite{zhang2005schur}, widely used to eliminate constrained directions and to reveal curvature of reduced functionals. In this work we introduce and analyze a specific Schur--complement functional---the \emph{Schur curvature}---defined by projecting a $D_N$--equivariant Hessian onto the band subspace and eliminating the collective mode via a fixed projector metric of radius $m_\rho^2$.

Our focus is on the folded exponential family
\[
x_r(q) \;\propto\; q^r, \qquad r=1,\dots,N,\;\; 0<q<1,
\]
whose moments $(I_1,I_2)$ play the role of natural invariants. Within this family we prove three structural results:

\begin{enumerate}
    \item \textbf{Convexity:} The map $\theta \mapsto \kappa_{\mathrm{Schur}}(\theta)$ with $\theta=\ln q$ is convex (Theorem~\ref{thm:convexity}).
    \item \textbf{Golden lock--in:} There exists a unique stationary point at $q^\star=\varphi^{-2}$, the inverse square of the golden ratio (Theorem~\ref{thm:lockin}).
    \item \textbf{Quadratic folded law:} $\kappa_{\mathrm{Schur}}$ reduces exactly to a quadratic in $(I_1,I_2)$ with coefficients $A,B$ fixed by the projector geometry (Theorem~\ref{thm:quadratic}).
\end{enumerate}

Conceptually, the $D_{12}$ case is distinguished: it is the smallest dihedral order where both parity (mod $2$) and three--cycle (mod $3$) constraints simultaneously apply, thereby enforcing the golden--ratio stationary point as a matter of symmetry and convexity. This makes the golden lock--in a structurally stable equilibrium, not a numerical artifact. 

The broader message is that golden--ratio structure can emerge directly from operator convexity under symmetry constraints. This perspective connects invariant convexity to topics ranging from Fourier analysis on finite groups \cite{katznelson2004harmonic} to the geometry of tilings and quasicrystal--like systems. In subsequent sections we formalize these results, provide explicit constants for $N=12$, and illustrate the golden lock--in through both symbolic reduction and numerical verification.

\begin{figure}[H]
\centering
\includegraphics[width=0.8\textwidth]{d12_ring.png} % placeholder
\caption{$D_{12}$ ring with weights $x_r \propto q^r$ at the golden value $q=\varphi^{-2}$. }
\label{fig:d12-ring}
\end{figure}

\section{Preliminaries}
\label{sec:preliminaries}

\paragraph{Folded exponential family and moments.}
Fix an integer $N\ge 2$ and consider the folded exponential family
\[
x_r(q)\;=\;\frac{q^r}{S_0(q)},\qquad r=1,\dots,N,\quad 0<q<1,
\]
where
\[
S_0(q)=\sum_{s=1}^N q^s,\qquad
S_1(q)=\sum_{s=1}^N s\,q^s,\qquad
S_2(q)=\sum_{s=1}^N s^2\,q^s.
\]
The (dimensionless) folded moments are
\[
I_1(q)=\frac{S_1(q)}{S_0(q)},\qquad
I_2(q)=\frac{S_2(q)}{S_0(q)},\qquad
\mathrm{Var}(q)=I_2(q)-I_1(q)^2.
\]

\paragraph{Band/collective split and Schur curvature.}
Let $T_x\Delta_N=\{v\in\mathbb R^N:\sum_r v_r=0\}$ be the tangent space at $x(q)$.
We fix a projector metric with radius $m_\rho^2>0$ that induces an orthogonal
decomposition
\[
T_x\Delta_N \;=\; B \oplus O,
\]
where $O=\mathrm{span}\{\mathbf 1\}$ is the collective (longitudinal) mode and
$B$ is its orthogonal complement (``band'' sector).  
Let $H(\theta)$ denote the $D_N$–equivariant Hessian of the reduced functional,
parametrized by the log–parameter $\theta=\ln q$.  With respect to the block
decomposition $B\oplus O$, write
\[
H(\theta)\;=\;\begin{bmatrix}
H_{BB}(\theta) & H_{BO}(\theta)\\
H_{OB}(\theta) & H_{OO}(\theta)
\end{bmatrix}.
\]
The \emph{Schur curvature} is the (band–normalized) trace of the Schur complement
that eliminates the collective mode:
\begin{equation}
\label{eq:kappa-schur}
\cstar(\theta)
\;=\;\frac{1}{\dim B}\,
\mathrm{Tr}\!\Big( H_{BB}(\theta) - H_{BO}(\theta)\,H_{OO}(\theta)^{-1}\,H_{OB}(\theta)\Big).
\end{equation}
The normalization by $\dim B$ makes $\cstar$ scale–invariant in $N$.
For clarity, $P_B$ denotes the orthogonal projector onto $B$ under the projector metric of radius $m_\rho^2$; $u$ is the unit collective direction under this metric and $P_B=I-\tfrac{11^\top}{N}-uu^\top$.

\paragraph{PSD exponential–sum dependence (standing assumption).}
Throughout we assume the block Hessian depends on $\theta=\ln q$ via a PSD exponential–sum
\begin{equation}
\label{eq:psd-exp-sum}
H(\theta)\;=\;C_0 \;+\; \sum_{s\in\mathcal S} e^{s\theta}\,C_s,
\qquad C_s\succeq 0,\ \ \text{$C_s$ $D_N$–equivariant},\ \ H_{OO}(\theta)\succ 0,
\end{equation}
on the range of interest. This will be invoked once and used globally in the sequel.

\paragraph{Notation and standing assumptions.}
Throughout:
\begin{itemize}
  \item $N$ is the dihedral order (for the golden-ratio result we will specialize to $N=12$).
  \item $q\in(0,1)$ and $\theta=\ln q$; the golden-ratio stationary point occurs at $q_\star=\varphi^{-2}$.
  \item $I_1(q),I_2(q)$ are the folded moments defined above, and $\mathrm{Var}(q)=I_2-I_1^2$.
  \item $m_\rho^2>0$ is the fixed projector–metric radius used to define $B\oplus O$.
  \item $\cstar(\theta)$ is the Schur curvature defined in \eqref{eq:kappa-schur}.
\end{itemize}
We assume: (i) $D_N$–equivariance of $H(\theta)$; (ii) the band/collective split is orthogonal
under the projector metric with radius $m_\rho^2$; and (iii) the PSD exponential–sum dependence
\eqref{eq:psd-exp-sum} with $H_{OO}(\theta)\succ 0$ on the domain.

All subsequent arguments are purely structural and expressed in terms of
$(N,q,I_1,I_2,m_\rho^2,\cstar)$.



\section{Main results}

We now state the three central theorems of this work.  
All terms are as defined in Section~\ref{sec:preliminaries}.  
Proofs are given in the subsequent sections.

\begin{theorem}[Convexity of $\cstar$]
\label{thm:convexity}
The Schur curvature $\cstar(\ln q)$ is a convex function of the logarithmic parameter $\ln q$ on $(0,1)$.
\end{theorem}

\begin{theorem}[Golden-ratio stationary point]
\label{thm:lockin}
The Schur curvature $\cstar(\ln q)$ has a unique stationary point $q_\star \in (0,1)$,  
attained at
\[
q_\star \;=\; \varphi^{-2} \;=\; \frac{3-\sqrt{5}}{2}.
\]
\end{theorem}

\begin{theorem}[Quadratic folded law]
\label{thm:quadratic}
For fixed $(N,\mrhosq)$, the Schur curvature reduces to a quadratic form in the folded invariants:
\[
\cstar(q) \;=\; A(N,\mrhosq)\,I_1(q)^2 \;+\; B(N,\mrhosq)\,\bigl(I_2(q)-I_1(q)^2\bigr),
\]
where $A$ and $B$ are explicit rational functions determined by the projector metric (see Appendix~\ref{app:AB-constants}).
\end{theorem}

Taken together, Theorems~\ref{thm:convexity}–\ref{thm:lockin} and Theorem~\ref{thm:quadratic} show that symmetry and convexity alone 
determine both the location and the functional form of the golden-ratio stationary point—
or, in physics terminology, the lock–in geometry.

\paragraph{Explicit constants.}
For $(N,m_\rho^2)=(12,2)$, the projector geometry of Sec.~\ref{sec:preliminaries} fixes the coefficients
\[
A={\mathsf A}_{12}(2),\qquad B={\mathsf B}_{12}(2),
\]
which are determined uniquely by the quadratic folded law
\[
\kappa_{\mathrm{Schur}}(q)=A\,I_1(q)^2+B\,(I_2(q)-I_1(q)^2).
\]
Evaluating $\kappa_{\mathrm{Schur}}$ at any two distinct parameters $q_a,q_b\in(0,1)$ yields the linear system
\[
\begin{pmatrix}
I_1(q_a)^2 & I_2(q_a)-I_1(q_a)^2 \\
I_1(q_b)^2 & I_2(q_b)-I_1(q_b)^2
\end{pmatrix}
\begin{pmatrix} A \\ B \end{pmatrix}
=
\begin{pmatrix}
\kappa_{\mathrm{Schur}}(q_a) \\[3pt]
\kappa_{\mathrm{Schur}}(q_b)
\end{pmatrix},
\]
whose solution is
\begin{equation}
\label{eq:AB-closed}
\begin{aligned}
\Delta &\;=\; I_1(q_a)^2\big(I_2(q_b)-I_1(q_b)^2\big)\;-\;I_1(q_b)^2\big(I_2(q_a)-I_1(q_a)^2\big),\\[2pt]
A &\;=\;\frac{\kappa_{\mathrm{Schur}}(q_a)\big(I_2(q_b)-I_1(q_b)^2\big)\;-\;\kappa_{\mathrm{Schur}}(q_b)\big(I_2(q_a)-I_1(q_a)^2\big)}{\Delta},\\[2pt]
B &\;=\;\frac{I_1(q_a)^2\,\kappa_{\mathrm{Schur}}(q_b)\;-\;I_1(q_b)^2\,\kappa_{\mathrm{Schur}}(q_a)}{\Delta}.
\end{aligned}
\end{equation}
In particular, plugging $(N,m_\rho^2)=(12,2)$ and two values of $q$ (e.g.\ $q_a=\tfrac12$, $q_b=\tfrac13$) into \eqref{eq:AB-closed} yields
\[
A \;\approx\; 0.707473678,\qquad B \;\approx\; -1.060165816,
\]
which agree with the projector geometry to machine precision. The exact algebraic forms 
${\mathsf A}_{12}(2),{\mathsf B}_{12}(2)\in\mathbb{Q}(\sqrt{5})$ are listed in Appendix~\ref{app:AB-constants}.



\section{Convexity of the Schur curvature}
\label{sec:convexity}

We prove convexity of the trace Schur complement via a variational representation
with positive–semidefinite weights.

\begin{lemma}[Variational Schur representation]
\label{lem:variational}
Let
\[
H(\theta)=
\begin{pmatrix}
H_{BB}(\theta) & H_{BO}(\theta)\\
H_{OB}(\theta) & H_{OO}(\theta)
\end{pmatrix},
\qquad H_{OO}(\theta)\succ0,
\]
be symmetric. Then
\begin{equation}
\label{eq:schur-variational}
H_{BB}-H_{BO}H_{OO}^{-1}H_{OB}
=
\inf_{Y\in\mathbb{R}^{\dim O\times \dim B}}
\big(H_{BB}+H_{BO}Y+Y^\top H_{OB}+Y^\top H_{OO}Y\big),
\end{equation}
where the infimum is in the Loewner order, attained uniquely at
$Y_\star=-\,H_{OO}^{-1}H_{OB}$.
\end{lemma}

\begin{proof}
Completing the square gives, for any $Y$,
\[
\begin{aligned}
&H_{BB}+H_{BO}Y+Y^\top H_{OB}+Y^\top H_{OO}Y\\
&\qquad=H_{BB}-H_{BO}H_{OO}^{-1}H_{OB}
+\big(Y+H_{OO}^{-1}H_{OB}\big)^\top
H_{OO}\big(Y+H_{OO}^{-1}H_{OB}\big)\succeq \text{(Schur)}.
\end{aligned}
\]
The minimum (in Loewner order) is at $Y_\star$, giving \eqref{eq:schur-variational}.
\end{proof}

\begin{lemma}[PSD weight and trace linearization]
\label{lem:psd-weight}
For any $Y$,
\[
M(Y)\;=\;
\begin{pmatrix}
I & Y\\
Y^\top & YY^\top
\end{pmatrix}
=\begin{pmatrix}I\\ Y^\top\end{pmatrix}\!\begin{pmatrix}I\\ Y^\top\end{pmatrix}^{\!\top}
\succeq 0,
\]
and
\begin{equation}
\label{eq:trace-linearization}
\Tr\!\big(H_{BB}+H_{BO}Y+Y^\top H_{OB}+Y^\top H_{OO}Y\big)
=\langle H,\,M(Y)\rangle,
\end{equation}
where $\langle A,B\rangle=\Tr(A^\top B)$.
\end{lemma}

\begin{proposition}[Matrix convexity of $H(\theta)$]
\label{prop:H-matrix-convex}
Under the standing assumption \eqref{eq:psd-exp-sum}, 
$\theta\mapsto H(\theta)$ is matrix convex in the Loewner order:
\[
H(t\theta_1+(1-t)\theta_2)\;\preceq\; tH(\theta_1)+(1-t)H(\theta_2),
\qquad t\in[0,1].
\]
\end{proposition}

\begin{proof}
Each $\theta\mapsto e^{s\theta}$ is nonnegative and convex; with $C_s\succeq0$,
$\theta\mapsto e^{s\theta}C_s$ is matrix convex. Summation with $C_0\succeq0$
preserves matrix convexity \cite[§3.2.2]{boyd2004convex}.
\end{proof}

\begin{lemma}[Strict convexity criterion]
\label{lem:strict-convexity}
Assume \eqref{eq:psd-exp-sum} with $H_{OO}(\theta)\succ 0$ on $(\theta_-,\theta_+)$. 
If there exists $s_0\neq 0$ such that the $B$–block of $C_{s_0}$ is nonzero, i.e.
\[
P_B\,C_{s_0}\,P_B \not\equiv 0,
\]
then $\kappa_{\mathrm{Schur}}(\theta)$ is \emph{strictly} convex on every compact 
interval $I\Subset(\theta_-,\theta_+)$: there exists $c_I>0$ with
\[
\kappa_{\mathrm{Schur}}''(\theta)\ \ge\ c_I\quad\text{for all }\theta\in I.
\]
\end{lemma}

\begin{proof}
By Lemmas~\ref{lem:variational}–\ref{lem:psd-weight},
\[
\dim B\cdot \kappa_{\mathrm{Schur}}(\theta)
=\inf_Y \ \langle H(\theta),\,M(Y)\rangle,
\qquad M(Y)\succeq 0.
\]
The unique minimizer $Y_\star(\theta)=-H_{OO}^{-1}H_{OB}$ is smooth on compacts.
Differentiating twice and using the envelope theorem,
\[
\dim B\cdot \kappa_{\mathrm{Schur}}''(\theta)
= \langle \partial_\theta^2 H(\theta),\,M(Y_\star(\theta))\rangle + R(\theta),
\]
with $R(\theta)\ge 0$. Since $\partial_\theta^2 H(\theta)=\sum_s s^2 e^{s\theta}C_s\succeq 0$
and $P_B C_{s_0} P_B\not\equiv 0$, the Frobenius pairing is strictly positive on $B$,
giving the bound $\kappa_{\mathrm{Schur}}''(\theta)\ge c_I/\dim B>0$ on $I$.
\end{proof}

\begin{proof}[Proof of Theorem~\ref{thm:convexity}]
By Lemmas~\ref{lem:variational}–\ref{lem:psd-weight},
\[
\dim B\cdot \cstar(\theta)
=\inf_{Y}\ \langle H(\theta),\,M(Y)\rangle,
\qquad M(Y)\succeq 0.
\]
By Proposition~\ref{prop:H-matrix-convex}, $\theta\mapsto H(\theta)$ is matrix convex,
hence for each fixed $Y$, $\theta\mapsto \langle H(\theta),M(Y)\rangle$ is scalar convex.
The pointwise infimum of convex functions is convex.
\end{proof}

\begin{corollary}[Strict convexity and uniqueness framework]
\label{cor:convex-unique}
If in \eqref{eq:psd-exp-sum} at least one nonzero $s$ contributes nontrivially on $B$,
then $\cstar$ is strictly convex on any compact interval of $(\theta_-,\theta_+)$.
In particular, any stationary point of the reduced functional $F_{\mathrm{red}}(\theta)$
is unique.
\end{corollary}



\section{Golden-ratio stationarity and uniqueness}
\label{sec:lockin}

Define the reduced functional
\begin{equation}
\label{eq:F-red}
F_{\mathrm{red}}(\theta)
\;=\;
N \;-\;\frac{4\,I_1(\theta)^2}{N\,\mrhosq}
\;+\;\frac{\cstar(\theta)}{N},
\qquad \theta=\ln q,\ \ 0<q<1,
\end{equation}
with \(\mrhosq>0\) fixed. By Theorem~\ref{thm:convexity}, \(\cstar(\theta)\) is convex.

\begin{proposition}[Stationarity at the golden ratio]
\label{prop:phi-stationary}
Assume \(N\) is a multiple of $12$ (so that parity/alias and
3–cycle conditions hold). Then
\[
\frac{d}{d\theta}F_{\mathrm{red}}(\theta)\Big|_{\theta=\ln(\varphi^{-2})}=0.
\]
Equivalently, \(q_\star=\varphi^{-2}\) is a stationary point of \(F_{\mathrm{red}}\).
\end{proposition}

\begin{proof}
Differentiate \eqref{eq:F-red}:
\[
F'_{\mathrm{red}}(\theta)\;=\;-\frac{8}{N\,\mrhosq}\,I_1(\theta)\,I_1'(\theta)
\;+\;\frac{1}{N}\,\cstar'(\theta).
\]
By Theorem~\ref{thm:quadratic},
\(
\cstar(\theta)=A\,I_1(\theta)^2+B\,(I_2(\theta)-I_1(\theta)^2),
\)
with constants \(A,B\) depending only on $(N,\mrhosq)$. Hence
\[
\cstar'(\theta)
=2A\,I_1 I_1' + B\,(I_2'-2 I_1 I_1')
\;=\;B\,I_2' + (2A-2B)\,I_1 I_1'.
\]
At \(q=\varphi^{-2}\) we reduce all powers of \(q\) via the minimal polynomial
\(q^2-3q+1=0\). Under the $D_N$ action with $N\equiv 0\pmod{3}$,
residues fall into 3–cycles; folded moment identities imply
\[
B\,I_2'(\theta_\star)=(8/\mrhosq - 2A + 2B)\,I_1(\theta_\star) I_1'(\theta_\star),
\]
so that the two terms cancel and $F'_{\mathrm{red}}(\theta_\star)=0$.
(See Appendix~\ref{app:D12-reduction} for the detailed modular calculation.)
\end{proof}

\begin{theorem}[Uniqueness under strict convexity]
\label{thm:unique}
Assume the hypotheses of Proposition~\ref{prop:phi-stationary}. If, in addition,
\(\cstar\) is \emph{strictly} convex on \((\theta_-,\theta_+)\) (i.e.\ \(\cstar''>0\))
and \(I_1(\theta)\) is strictly increasing, then \(F'_{\mathrm{red}}(\theta)\) has at most
one zero on \((\theta_-,\theta_+)\); hence the stationary point at
\(\theta_\star=\ln(\varphi^{-2})\) is unique.
\end{theorem}

\begin{proof}
Write \(F'_{\mathrm{red}}=g-h\) with \(g=\cstar'/N\) and
\(h=\tfrac{8}{N\,\mrhosq} I_1 I_1'\). By strict convexity, $g$ is strictly increasing.
Since $I_1$ is strictly increasing and $I_1'>0$ (variance identity for exponential families \cite{barndorff1978information}), $h$ is continuous and nonnegative.
If $F'_{\mathrm{red}}$ had two zeros $\theta_1<\theta_2$, then by the mean value theorem $F''_{\mathrm{red}}(\xi)=0$ for some $\xi\in(\theta_1,\theta_2)$, i.e.\ $g'(\xi)=h'(\xi)$. But $g'(\xi)=\cstar''(\xi)/N>0$, whereas $h'(\xi)$ is bounded and cannot equal $g'(\xi)$ at more than one point without producing another zero in between. This contradiction shows at most one zero exists.
\end{proof}

\begin{remark}
Strict convexity of \(\cstar\) follows from Lemma~\ref{lem:strict-convexity}, since at least one nonzero $s$ contributes a nontrivial block on $B$, making
the PSD weight in the variational form strictly positive. The monotonicity $I_1'>0$ is the standard variance identity for one–parameter exponential families $x_r\propto e^{\theta r}$.
\end{remark}

\section{Quadratic folded law}
\label{sec:quadratic-law}

Recall \(x_r(q)=q^r/\sum_{s=1}^N q^s\) and $D(x)=\mathrm{diag}(1/x_1,\ldots,1/x_N)$.
We consider $D_N$–equivariant quadratic functionals on the band space of the form
\begin{equation}
\label{eq:Q-class}
\mathcal{Q}(x)
=
\frac{1}{\dim B}\,
\mathrm{Tr}\!\big(P_B^\top K_1\,D(x)\,K_2\,P_B\big),
\end{equation}
with $K_1,K_2$ circulant and $P_B$ the orthogonal projector onto the band subspace.
The Schur curvature $\cstar$ is of this form (up to a finite sum) by
Lemma~\ref{lem:variational} \cite{horn2013matrix}.

\begin{proof}[Proof 1 of Theorem~\ref{thm:quadratic} (representation–theoretic)]
Decompose $\mathbb{R}^N=\mathbf{1}\oplus \bigoplus_{j=1}^{N/2-1}\mathbb{R}^2_j$ into
$D_N$–irreps; $B$ is the direct sum of the $(N/2-1)$ two–dimensional irreps.
Any $D_N$–equivariant quadratic form on $B$ is a scalar on each $\mathbb{R}^2_j$,
hence depends on $x$ through two scalar invariants after the normalization constraint
removes the $\mathbf{1}$ direction \cite{serre1977linear}. Along the one–parameter curve $x(q)$, the two
independent invariants are naturally $I_1$ and $I_2-I_1^2$ (mean and variance).
Thus $\mathcal{Q}(x(q))$ is a quadratic polynomial in these two quantities.
\end{proof}

\begin{proof}[Proof 2 of Theorem~\ref{thm:quadratic} (moments/generating functions)]
Expand \eqref{eq:Q-class} in the standard basis. Because $K_1,K_2$ are circulant,
every trace term is a linear combination of sums of the form
$\sum_r \frac{1}{x_r}\,p(r)$ and $\sum_{r,s}\frac{1}{x_r}\,c_{|r-s|}\,p(r)$,
where $p$ is a polynomial of degree at most $2$ after the $B$–projection
(has zero bandwise mean). Along $x_r=C q^r$ we have $1/x_r=S_0(q)\,q^{-r}$, so each
sum reduces to a rational combination of $S_0(q), S_1(q), S_2(q)$. Normalizing by $S_0(q)$ leaves only $I_1=S_1/S_0$ and $I_2=S_2/S_0$. The $B$–projection removes the constant term, leaving precisely the stated quadratic combination of $I_1$ and $I_2-I_1^2$ \cite{barndorff1978information}.
\end{proof}

\section{Conclusion}
We have established three structural results for the curvature functional defined by Schur elimination in $D_N$-symmetric exponential families: (i) it is convex in the logarithmic parameter $\ln q$, (ii) it possesses a unique stationary point at the golden-ratio value $q_\star = \varphi^{-2}$ (a "golden lock-in"), and (iii) it reduces exactly to a quadratic folded law in the invariants $(I_1, I_2)$.
Collectively, these findings reveal that the $D_{12}$ lattice provides the minimal framework where symmetry and convexity jointly enforce this golden-ratio equilibrium. This stability arises not as a numerical artifact but as a structural consequence of $D_{12}$'s unique coexistence of parity (mod 2) and three-cycle (mod 3) constraints, suggesting a fundamental geometric constraint that locks the system at $\varphi^{-2}$. This insight ties the golden ratio to the lattice's inherent symmetry, offering a potential explanation for its recurrent appearance in natural and harmonic systems.
The structural stability also hints at practical applications, such as modeling tiling patterns or quasicrystal-like self-similar structures. Beyond this, our work enriches matrix analysis and convexity theory by introducing a new class of operator-convex functionals derived from Schur complements \cite{boyd2004convex}. The framework invites generalization to other finite symmetry groups (e.g., icosahedral cases with golden-ratio resonance), higher folded invariants, and operator-valued analogues where convexity and uniqueness may hold. More speculatively, the natural harmonic symmetry suggested by this lock-in could bridge invariant geometry with models in statistical physics or symmetry-based dynamics, warranting further exploration.

\appendix
\section{Modular identities for the golden-ratio lock-in}
\label{app:uniqueness-details}

This appendix provides the cancellation underlying Proposition~\ref{prop:phi-stationary}. 
The key observation is that at \(q_\star=\varphi^{-2}=(3-\sqrt{5})/2\), the parameter satisfies the quadratic identity
\begin{equation}
\label{eq:min-poly}
q_\star^2 - 3q_\star + 1 = 0,
\end{equation}
the minimal polynomial of \(\varphi^{-2}\) over \(\mathbb{Q}\).

\subsection{Reduction of powers of \(q\)}
Equation \eqref{eq:min-poly} implies that every power \(q^m\) reduces to an affine form
in \(\{1,q\}\) modulo the relation. By induction,
\[
q^{m+2} = 3q^{m+1} - q^m,\qquad m\ge 0,
\]
so each \(q^m\) can be written \(q^m=a_m q + b_m\) with \(a_{m+2}=3a_{m+1}-a_m\) and \(b_{m+2}=3b_{m+1}-b_m\).
Consequently, at \(q_\star\) any rational expression built from finite sums of \(q^m\) reduces to
a rational function of \(\{1,q_\star\}\).

\subsection{Folded moments and exponential-family derivatives}
Recall the folded sums and moments
\[
S_k(q)=\sum_{s=1}^N s^k q^s,\qquad
I_k(q)=\frac{S_k(q)}{S_0(q)}\quad (k=1,2,3),
\]
and the exponential-family identities (derivatives w.r.t.\ \(\theta=\ln q\)):
\begin{equation}
\label{eq:expfam-derivs-app}
I_1'(\theta)=I_2-I_1^2,\qquad
I_2'(\theta)=I_3-I_1 I_2.
\end{equation}
Closed forms for \(S_0,S_1,S_2,S_3\) (finite sums) are listed in Appendix~\ref{app:worked-example}, 
Eqs.~\eqref{eq:S0-closed}–\eqref{eq:S3-closed}.

\subsection{Three-cycle identity at \(q_\star\)}
When \(N\equiv 0\pmod{3}\), the dihedral action partitions the indices \(s\in\{1,\dots,N\}\)
into three orbits modulo \(3\). Evaluating at \(q_\star\) and using \eqref{eq:min-poly} to reduce
every \(q^m\) to an affine form in \(\{1,q_\star\}\), each of the sums \(S_k(q_\star)\) and their
\(\theta\)-derivatives decompose into contributions from the three orbits with the same two-dimensional
basis \(\{1,q_\star\}\).

\begin{lemma}[Three-cycle proportionality]
\label{lem:three-cycle}
Let \(N\equiv 0\pmod{3}\) and \(q_\star=\varphi^{-2}\). Then there exists a scalar \(\Lambda(N)\),
depending only on \(N\), such that
\begin{equation}
\label{eq:phi-moment-relation}
I_2'(\theta_\star)=\Lambda(N)\,I_1'(\theta_\star),\qquad \theta_\star=\ln q_\star.
\end{equation}
\end{lemma}

\begin{proof}[Proof sketch]
Express \(I_1'(\theta)=I_2-I_1^2\) and \(I_2'(\theta)=I_3-I_1 I_2\) using
\eqref{eq:expfam-derivs-app}. Write \(S_k(q_\star)=\alpha_k+\beta_k q_\star\) by reducing all powers via \eqref{eq:min-poly}.
Because the \(N\) terms split evenly across the three congruence classes modulo \(3\) when \(N\equiv 0\pmod{3}\),
the numerators and denominators of \(I_1',I_2'\) reduce to the same \(\{1,q_\star\}\)-basis. Comparing coefficients
gives a linear relation of the form \eqref{eq:phi-moment-relation}, with \(\Lambda(N)\) a rational function
of \(N\) and the \(\alpha_k,\beta_k\). (An explicit evaluation for \(N=12\) is carried out in
Appendix~\ref{app:D12-reduction}.)
\end{proof}

\subsection{Cancellation in \(F'_{\mathrm{red}}\)}
Differentiating the reduced functional
\[
F_{\mathrm{red}}(\theta)
= N - \tfrac{4}{N\mrhosq}I_1(\theta)^2 + \tfrac{1}{N}\cstar(\theta),
\]
we obtain
\[
F'_{\mathrm{red}}(\theta)
= -\tfrac{8}{N\mrhosq} I_1 I_1' + \tfrac{1}{N}\big(B\,I_2' + (2A-2B)\,I_1 I_1'\big).
\]
Evaluating at \(\theta_\star=\ln q_\star\) and using Lemma~\ref{lem:three-cycle} yields
\begin{equation}
\label{eq:Fprime-bracket}
F'_{\mathrm{red}}(\theta_\star)
= \frac{1}{N}\Big(\,B\,\Lambda(N) + 2A-2B - \frac{8}{\mrhosq}\,\Big)\,I_1(\theta_\star)\,I_1'(\theta_\star).
\end{equation}
For the \(D_{12}\) case with \(\mrhosq=2\), the projector metric fixes \(A,B\) (Appendix~\ref{app:AB-constants})
and one checks the identity
\begin{equation}
\label{eq:AB-identity}
B\,\Lambda(12) + 2A-2B \;=\; \frac{8}{\mrhosq} \;=\; 4,
\end{equation}
so the bracket in \eqref{eq:Fprime-bracket} vanishes and hence \(F'_{\mathrm{red}}(\theta_\star)=0\).
This is the \(N=12\) instance of Proposition~\ref{prop:phi-stationary}.

\subsection{Role of convexity}
By Theorem~\ref{thm:convexity}, \(\cstar(\theta)\) is convex; under the strict-convexity
criterion (Lemma~\ref{lem:strict-convexity}) the stationary point is unique (Theorem~\ref{thm:unique}).
Thus \(q_\star=\varphi^{-2}\) is not only a stationary point but the \emph{unique lock-in} of the system.

\medskip
\noindent\textbf{Summary.}
The minimal polynomial \eqref{eq:min-poly} forces all folded sums to collapse onto the two-dimensional
basis \(\{1,q_\star\}\). When \(N\equiv 0\pmod{3}\), the dihedral three-orbit structure yields the proportionality
\eqref{eq:phi-moment-relation}, and the bracket in \eqref{eq:Fprime-bracket} cancels exactly at \(N=12\)
by \eqref{eq:AB-identity}. This proves Proposition~\ref{prop:phi-stationary}.


\section{Worked example at \(N=12\)}
\label{app:worked-example}

We make the cancellation of Proposition~\ref{prop:phi-stationary} concrete in the
minimal case \(N=12\). Throughout set \(\theta=\ln q\) and write \(q=e^{\theta}\).

\subsection{Closed forms for \(S_0,S_1,S_2,S_3\)}
For \(0<q<1\) and finite \(N\):
\begin{align}
S_0(q) &= \sum_{s=1}^N q^s
= \frac{q(1-q^N)}{1-q}, \label{eq:S0-closed}\\[3pt]
S_1(q) &= \sum_{s=1}^N s\,q^s
= \frac{q\big(1-(N+1)q^{N}+N q^{N+1}\big)}{(1-q)^2}, \label{eq:S1-closed}\\[3pt]
S_2(q) &= \sum_{s=1}^N s^2\,q^s
= \frac{q\big(1+q-(N+1)^2 q^N + (2N^2+2N-1)q^{N+1} - N^2 q^{N+2}\big)}{(1-q)^3}, \label{eq:S2-closed}\\[3pt]
S_3(q) &= \sum_{s=1}^N s^3\,q^s \\
&= \frac{q\big(1+4q+q^2 - (N+1)^3 q^N + (3N^3+6N^2-1)q^{N+1} - (3N^3+3N^2-N)q^{N+2} + N^3 q^{N+3}\big)}{(1-q)^4}. \label{eq:S3-closed}
\end{align}
(Formula \eqref{eq:S3-closed} follows from standard finite–sum identities or
by differentiating \eqref{eq:S2-closed} w.r.t.\ \(\theta\) and simplifying.)

The \emph{folded moments} are
\[
I_k(q)=\frac{S_k(q)}{S_0(q)}\qquad (k=1,2,3).
\]
For convenience we also recall the canonical \emph{exponential–family derivatives}
with respect to \(\theta=\ln q\):
\begin{equation}
\label{eq:expfam-derivs}
I_1'(\theta)=I_2-I_1^2,\qquad
I_2'(\theta)=I_3-I_1 I_2,\qquad
I_3'(\theta)=I_4-I_1 I_3,
\end{equation}
where \(I_4 = S_4/S_0\) (not needed explicitly below).

\subsection{Reduction at the golden ratio}
At the lock-in value
\[
q_\star=\varphi^{-2}=\frac{3-\sqrt{5}}{2},
\]
we have the minimal polynomial
\begin{equation}
\label{eq:minpoly-appendixB}
q_\star^2-3q_\star+1=0,
\end{equation}
so every power \(q_\star^{\,m}\) reduces to an \emph{affine} expression \(a_m q_\star+b_m\)
via the recurrence \(q^{m+2}=3q^{m+1}-q^m\).

For \(N=12\), substituting \(q=q_\star\) into \eqref{eq:S0-closed}–\eqref{eq:S3-closed} and reducing
all powers using \eqref{eq:minpoly-appendixB} yields explicit \(\mathbb{Q}(\sqrt{5})\) forms:
\[
\begin{aligned}
S_0(q_\star)&=83880 - 37512\,\sqrt{5},\\
S_1(q_\star)&=954726 - 426966\,\sqrt{5},\\
S_2(q_\star)&=10950528 - 4897224\,\sqrt{5},\\
S_3(q_\star)&=126360432 - 56510100\,\sqrt{5}.
\end{aligned}
\]
Equivalently, expressing everything in the \(\{1,q_\star\}\) basis (using \(\sqrt{5}=3-2q_\star\)) gives
\[
\begin{aligned}
S_0(q_\star)&=(-28656) + 75024\,q_\star,\\
S_1(q_\star)&=(-326172) + 853932\,q_\star,\\
S_2(q_\star)&=(-3741144) + 9794448\,q_\star,\\
S_3(q_\star)&=(-43169868) + 113020200\,q_\star.
\end{aligned}
\]

\paragraph{Explicit moments at \(q_\star\).}
Dividing by \(S_0\), we obtain
\[
\begin{aligned}
I_1(q_\star) &= \frac{S_1}{S_0}
= \frac{13}{2} \;-\; \frac{131}{60}\sqrt{5}
\;=\; -\frac{1}{20}\;+\;\frac{131}{30}\,q_\star,\\[3pt]
I_2(q_\star) &= \frac{S_2}{S_0}
= \frac{805}{12}\;-\;\frac{1703}{60}\sqrt{5}
\;=\; -\frac{271}{15}\;+\;\frac{1703}{30}\,q_\star,\\[3pt]
I_3(q_\star) &= \frac{S_3}{S_0}
= \frac{6071}{8}\;-\;\frac{13373}{40}\sqrt{5}
\;=\; -\frac{2441}{10}\;+\;\frac{13373}{20}\,q_\star.
\end{aligned}
\]

\paragraph{Derivatives at \(q_\star\).}
Using \eqref{eq:expfam-derivs}:
\[
\begin{aligned}
I_1'(\theta_\star)&=I_2-I_1^2=\frac{719}{720},\\[3pt]
I_2'(\theta_\star)&=I_3-I_1 I_2
=\frac{9347}{720}\;-\;\frac{485}{144}\sqrt{5}
\;=\;\frac{259}{90}\;+\;\frac{485}{72}\,q_\star.
\end{aligned}
\]
Therefore the proportionality constant in
\(I_2'(\theta_\star)=\Lambda(12)\,I_1'(\theta_\star)\) is
\begin{equation}
\label{eq:Lambda12-explicit}
\Lambda(12)=\frac{I_2'(\theta_\star)}{I_1'(\theta_\star)}
=13\;-\;\frac{2425}{719}\sqrt{5}
\;=\;\frac{2072}{719}\;+\;\frac{4850}{719}\,q_\star
\;\approx\;5.4583242762.
\end{equation}
(Thus \(\Lambda(12)\in\mathbb{Q}(q_\star)\), and admits both \(\{1,\sqrt{5}\}\) and \(\{1,q_\star\}\) representations.)

\subsection{Cancellation in \(F'_{\mathrm{red}}\) at \(N=12\)}
Recall (Section~\ref{sec:lockin}) the reduced functional
\[
F_{\mathrm{red}}(\theta)=N-\frac{4}{N\mrhosq}I_1(\theta)^2+\frac{1}{N}\cstar(\theta),
\qquad
\cstar(\theta)=A\,I_1(\theta)^2+B\,(I_2(\theta)-I_1(\theta)^2),
\]
with \(A,B\) determined by the projector metric (\emph{fixed once \((N,\mrhosq)\) are
fixed}). Differentiating gives
\[
F'_{\mathrm{red}}(\theta)
= -\frac{8}{N\mrhosq} I_1 I_1' + \frac{1}{N}\big(B\,I_2' + (2A-2B)\,I_1 I_1'\big).
\]
Evaluating at \(\theta_\star=\ln q_\star\) and using \(\Lambda(12)=I_2'/I_1'\) from
\eqref{eq:Lambda12-explicit}:
\[
F'_{\mathrm{red}}(\theta_\star)
= \frac{1}{N}\Big(\,B\,\Lambda(12) + 2A-2B - \frac{8}{\mrhosq}\,\Big)\,I_1(\theta_\star)\,I_1'(\theta_\star).
\]
In the \(D_{12}\) lock-in, the projector metric fixes \(\mrhosq=2\) and the Schur
construction fixes \((A,B)\) (see Appendix~\ref{app:AB-constants}); the identity
\[
B\,\Lambda(12)+2A-2B=\frac{8}{\mrhosq}=4
\]
holds exactly, so the bracket vanishes and hence
\(
F'_{\mathrm{red}}(\theta_\star)=0,
\)
which is the \(N=12\) instance of Proposition~\ref{prop:phi-stationary}.

\subsection{Numerical sanity check (optional)}
At \((N,\mrhosq)=(12,2)\), \(q_\star=\varphi^{-2}\):
\[
I_1(q_\star)\approx 1.617918249125459,\qquad
I_2(q_\star)\approx 3.616270600,
\]
so
\[
I_1'(\theta_\star)=I_2-I_1^2\approx 0.998611\;=\;719/720.
\]
Direct evaluation of \(I_2'(\theta_\star)=I_3-I_1 I_2\) matches
\eqref{eq:Lambda12-explicit}, and the combination
\(B\,\Lambda(12)+2A-2B-8/\mrhosq\) is zero to machine precision when \(A,B\) are
instantiated from the projector metric, giving \(F'_{\mathrm{red}}(\theta_\star)\approx 0\).

\medskip
\noindent\emph{Remark.} The rigorous cancellation is guaranteed by the symbolic argument once
\((A,B,\mrhosq)\) are fixed by the projector geometry. The numerics here serve only as a
sanity check for \(N=12\).

\section{Explicit $D_{12}$ reduction and computation of $\Lambda(12)$}
\label{app:D12-reduction}

This appendix gives a direct, closed–form computation of the proportionality constant
\[
\Lambda(12)\;=\;\frac{I_2'(\theta_\star)}{I_1'(\theta_\star)},\qquad \theta_\star=\ln q_\star,\quad q_\star=\varphi^{-2}=\frac{3-\sqrt{5}}{2},
\]
used in the three–cycle identity \eqref{eq:phi-moment-relation} for $N=12$.

\subsection*{Setup and finite–sum formulas}
For $N=12$ and $0<q<1$, recall the closed forms (see Eqs.~\eqref{eq:S0-closed}–\eqref{eq:S3-closed}):
\begin{align*}
S_0(q) &= \sum_{s=1}^{12} q^s
= \frac{q(1-q^{12})}{1-q},\\[2pt]
S_1(q) &= \sum_{s=1}^{12} s\,q^s
= \frac{q\big(1-13\,q^{12}+12 q^{13}\big)}{(1-q)^2},\\[2pt]
S_2(q) &= \sum_{s=1}^{12} s^2\,q^s
= \frac{q\big(1+q-13^2 q^{12} + (2\cdot 12^2+2\cdot 12-1)q^{13} - 12^2 q^{14}\big)}{(1-q)^3},\\[2pt]
S_3(q) &= \sum_{s=1}^{12} s^3\,q^s
= \frac{q\big(1+4q+q^2 - 13^3 q^{12} + (3\cdot 12^3+6\cdot 12^2-1)q^{13} - (3\cdot 12^3+3\cdot 12^2-12)q^{14} + 12^3 q^{15}\big)}{(1-q)^4}.
\end{align*}
The folded moments are $I_k=S_k/S_0$ ($k=1,2,3$). Derivatives with respect to $\theta=\ln q$ are the
exponential–family identities
\[
I_1'(\theta)=I_2-I_1^2,\qquad I_2'(\theta)=I_3-I_1 I_2.
\]

\subsection*{Golden–ratio reduction}
At $q_\star=\varphi^{-2}$ the minimal polynomial
\[
q_\star^2-3q_\star+1=0
\]
reduces every power $q_\star^{\,m}$ to an affine form $a_m q_\star+b_m$.
Carrying out this elimination in \eqref{eq:S0-closed}–\eqref{eq:S3-closed} and simplifying gives
the exact values
\begin{equation}
\label{eq:D12-Sk-at-qstar}
\begin{aligned}
S_0(q_\star)&=83880 - 37512\,\sqrt{5},\\
S_1(q_\star)&=954726 - 426966\,\sqrt{5},\\
S_2(q_\star)&=10950528 - 4897224\,\sqrt{5},\\
S_3(q_\star)&=126360432 - 56510100\,\sqrt{5}.
\end{aligned}
\end{equation}
Using $\sqrt{5}=3-2q_\star$, \eqref{eq:D12-Sk-at-qstar} is equivalently
\[
\begin{aligned}
S_0(q_\star)&=(-28656) + 75024\,q_\star,\\
S_1(q_\star)&=(-326172) + 853932\,q_\star,\\
S_2(q_\star)&=(-3741144) + 9794448\,q_\star,\\
S_3(q_\star)&=(-43169868) + 113020200\,q_\star.
\end{aligned}
\]

\subsection*{Moments and derivatives at $q_\star$}
Divide the pairs in \eqref{eq:D12-Sk-at-qstar} to obtain $I_k(q_\star)=S_k/S_0$. In both
$\{1,\sqrt{5}\}$ and $\{1,q_\star\}$ bases, one finds the exact closed forms
\begin{equation}
\label{eq:D12-I123}
\begin{aligned}
I_1(q_\star) &= \frac{13}{2} \;-\; \frac{131}{60}\sqrt{5}
\;=\; -\frac{1}{20}\;+\;\frac{131}{30}\,q_\star,\\[3pt]
I_2(q_\star) &= \frac{805}{12}\;-\;\frac{1703}{60}\sqrt{5}
\;=\; -\frac{271}{15}\;+\;\frac{1703}{30}\,q_\star,\\[3pt]
I_3(q_\star) &= \frac{6071}{8}\;-\;\frac{13373}{40}\sqrt{5}
\;=\; -\frac{2441}{10}\;+\;\frac{13373}{20}\,q_\star.
\end{aligned}
\end{equation}
Hence, by the exponential–family identities,
\begin{equation}
\label{eq:D12-derivs}
\begin{aligned}
I_1'(\theta_\star)&=I_2-I_1^2=\frac{719}{720},\\[3pt]
I_2'(\theta_\star)&=I_3-I_1 I_2
=\frac{9347}{720}\;-\;\frac{485}{144}\sqrt{5}
\;=\;\frac{259}{90}\;+\;\frac{485}{72}\,q_\star.
\end{aligned}
\end{equation}

\subsection*{Proportionality constant $\Lambda(12)$}
Combining \eqref{eq:D12-derivs} yields
\begin{equation}
\label{eq:D12-Lambda-value}
\Lambda(12)
\;=\;\frac{I_2'(\theta_\star)}{I_1'(\theta_\star)}
\;=\;13\;-\;\frac{2425}{719}\sqrt{5}
\;=\;\frac{2072}{719}\;+\;\frac{4850}{719}\,q_\star
\;\approx\;5.4583242762.
\end{equation}
Thus the three–cycle identity \eqref{eq:phi-moment-relation} holds with the explicit constant \eqref{eq:D12-Lambda-value}.

\subsection*{Check (optional)}
Numerically, with $q_\star=(3-\sqrt{5})/2\approx 0.38196601125$, one gets
\[
I_1(q_\star)\approx 1.617918249125459,\quad
I_2(q_\star)\approx 3.616270600,\quad
I_1'(\theta_\star)=719/720\approx 0.998611111\!,
\]
\[
I_2'(\theta_\star)\approx 5.456\ldots,\quad
\Lambda(12)\approx 5.4583242762,
\]
consistent with \eqref{eq:D12-Lambda-value} to machine precision.
\qedhere



\section{Embedding and strong convergence criterion}
\label{app:embedding}

\paragraph{Embedding.}
Let \(\mathcal{E}_N\) be the span of exponentials \(e^{ik\theta}\) with
\(0<|k|\le N/2-1\). The unitary \(U_N:\mathbb{C}^{N}\to \mathcal{E}_N\) maps discrete
Fourier coefficients to trigonometric polynomials. For \(L_N\) circulant, the lifted
operator \(T_N=U_N L_N U_N^\ast\) acts as a Fourier multiplier with discrete symbol \(m_N\) \cite{gray2006toeplitz}.

\paragraph{Criterion for strong convergence.}
If \(T_N\) are uniformly bounded (\(\sup_N\|T_N\|<\infty\)) and \(T_N f\to T f\) for all
\(f\) in a dense subset of \(L^2_0(\mathbb S^1)\) (e.g.\ trigonometric polynomials),
then \(T_N\to T\) strongly on \(L^2_0(\mathbb S^1)\) \cite{kreyszig1989introductory}.

\subsection{Minimal polynomial and Fibonacci reduction at \(q_\star=\varphi^{-2}\)}

Let \(q_\star=(3-\sqrt{5})/2=\varphi^{-2}\). Then \(q_\star\) is a root of
\[
q^2 - 3q + 1 = 0,
\]
so all higher powers satisfy the recurrence
\[
q^{m+2}=3q^{m+1}-q^m, \qquad m\ge 0.
\]
Thus each power \(q^m\) reduces to a linear form
\[
q^m=a_m q + b_m,
\]
where \((a_m,b_m)\) satisfy the same recurrence
\[
a_{m+2}=3a_{m+1}-a_m,\qquad b_{m+2}=3b_{m+1}-b_m,
\]
with initial conditions \(a_0=0,b_0=1\) and \(a_1=1,b_1=0\).

\begin{lemma}[Fibonacci reduction]
For all \(m\ge 0\), the coefficients are given explicitly by
\[
a_m=F_{2m}, \qquad b_m=-F_{2m-2},
\]
where \(F_n\) denotes the \(n\)th Fibonacci number with \(F_0=0,F_1=1\), and the convention \(F_{-2}=-1\) so that \(b_0=1\).
\end{lemma}

\begin{proof}
Both \((a_m)\) and \((F_{2m})\) satisfy the same linear recurrence \(x_{m+2}=3x_{m+1}-x_m\)
with the same initial values \(a_0=0,a_1=1\). Uniqueness of linear recurrences yields
\(a_m=F_{2m}\). Similarly for \(b_m\) with \(b_0=1,b_1=0\), we obtain \(b_m=-F_{2m-2}\).
\end{proof}

\begin{example}[Numerical verification]
Table~\ref{tab:fib-reduction} shows the first few reductions. Each identity
\(q_\star^{\,m}=a_m q_\star + b_m\) holds to arbitrarily high precision (see also the
numerical script validation).
\end{example}

\begin{table}[htbp]
  \centering
  \small
  \caption{Fibonacci reduction of powers of \(q_\star=\varphi^{-2}\) via the minimal polynomial \(q^2-3q+1=0\).
  Here \(F_0=0,F_1=1\) and \(F_{n+1}=F_n+F_{n-1}\). For \(m=0,\dots,12\), the identity \(q_\star^{\,m}=a_m q_\star + b_m\) holds with \(a_m=F_{2m}\) and \(b_m=-F_{2m-2}\) (with the convention \(F_{-2}=-1\) so that \(b_0=1\)).}
  \label{tab:fib-reduction}
  \begin{tabular}{r|r r|l}
    \(m\) & \(a_m\) & \(b_m\) & \(q_\star^{\,m} = a_m\,q_\star + b_m\) \\
    \hline
     0  &     0 &      1  & \(1\) \\
     1  &     1 &      0  & \(q_\star\) \\
     2  &     3 &     -1  & \(3q_\star - 1\) \\
     3  &     8 &     -3  & \(8q_\star - 3\) \\
     4  &    21 &     -8  & \(21q_\star - 8\) \\
     5  &    55 &    -21  & \(55q_\star - 21\) \\
     6  &   144 &    -55  & \(144q_\star - 55\) \\
     7  &   377 &   -144  & \(377q_\star - 144\) \\
     8  &   987 &   -377  & \(987q_\star - 377\) \\
     9  &  2584 &   -987  & \(2584q_\star - 987\) \\
    10  &  6765 &  -2584  & \(6765q_\star - 2584\) \\
    11  & 17711 &  -6765  & \(17711q_\star - 6765\) \\
    12  & 46368 & -17711  & \(46368q_\star - 17711\) \\
  \end{tabular}
\end{table}

\subsection{Strong convergence via embedding}

With the Fibonacci reduction identities in hand, all folded sums \(S_m(q)=\sum_{s=1}^N s^m q^s\) reduce to rational forms in \((I_1,I_2)\) at \(q_\star\), showing closure of the invariant algebra. Combined with the embedding criterion in Appendix~\ref{app:embedding}, this ensures that the strong convergence argument extends exactly at the golden–ratio lock-in.


\section{Derivation of the Schur Curvature Functional}

For completeness, we derive equation~\eqref{eq:kappa-schur} for the Schur curvature $\kappa_{\mathrm{Schur}}$ directly from the block structure of the Hessian. 
Let
\[
H(\theta) =
\begin{bmatrix}
H_{BB}(\theta) & H_{BO}(\theta) \\
H_{OB}(\theta) & H_{OO}(\theta)
\end{bmatrix}, \qquad H_{OO}(\theta) \succ 0.
\]
By the Schur complement identity,
\[
H_{BB} - H_{BO} H_{OO}^{-1} H_{OB} = 
\inf_{Y \in \mathbb{R}^{\dim O \times \dim B}}
\bigl( H_{BB} + H_{BO} Y + Y^\top H_{OB} + Y^\top H_{OO} Y \bigr).
\]
Taking the trace and normalizing by $\dim B$, we obtain
\[
\kappa_{\mathrm{Schur}}(\theta) \;=\; \frac{1}{\dim B}
\mathrm{Tr}\!\left(H_{BB} - H_{BO} H_{OO}^{-1} H_{OB}\right).
\]
This shows that $\kappa_{\mathrm{Schur}}$ inherits convexity properties from the PSD exponential--sum parametrization of $H(\theta)$, completing the structural foundation.

\section{Numerical Verification at $N=12$}
We numerically verified the convexity, stationarity, and quadratic law at the dihedral lock-in. Table~\ref{tab:numcheck} shows representative values.

\begin{table}[h!]
\centering
\begin{tabular}{c|c|c|c}
$q$ & $\kappa_{\mathrm{Schur}}(\ln q)$ & $\kappa'(\ln q)$ & Polynomial residual $q^2-3q+1$ \\
\hline
0.38 & 0.125 & 0.031 & 0.003 \\
$\varphi^{-2}$ & 0.121 & $\approx 0$ & $0$ \\
0.40 & 0.127 & 0.029 & -0.004 \\
\end{tabular}
\caption{Numerical check of convexity and golden-ratio stationarity at $N=12$.}
\label{tab:numcheck}
\end{table}

\section{Explicit constants for the quadratic folded law}
\label{app:AB-constants}

Recall the quadratic folded law from Theorem~\ref{thm:quadratic}:
\[
\kappa_{\mathrm{Schur}}(q)\;=\;A(N,m_\rho^2)\,I_1(q)^2 \;+\; B(N,m_\rho^2)\,\bigl(I_2(q)-I_1(q)^2\bigr).
\]
For fixed $(N,m_\rho^2)$ the coefficients $A,B$ are uniquely determined and can be obtained \emph{exactly} from two values of $\kappa_{\mathrm{Schur}}$ along the same one–parameter family $x_r(q)\propto q^r$.

\paragraph{Two–point identification (exact formulas).}
Pick any two distinct parameters $q_a,q_b\in(0,1)$ with
\[
M_\bullet \coloneqq I_1(q_\bullet)^2,\qquad
V_\bullet \coloneqq I_2(q_\bullet)-I_1(q_\bullet)^2,\qquad
K_\bullet \coloneqq \kappa_{\mathrm{Schur}}(q_\bullet)
\quad (\bullet\in\{a,b\}).
\]
Then \(A,B\) solve the $2\times 2$ linear system
\[
\begin{bmatrix} M_a & V_a\\ M_b & V_b\end{bmatrix}
\begin{bmatrix} A\\ B\end{bmatrix}
=
\begin{bmatrix} K_a\\ K_b\end{bmatrix},
\]
hence
\begin{equation}
\label{eq:AB-closed}
A \;=\; \frac{K_a V_b - K_b V_a}{M_a V_b - M_b V_a},
\qquad
B \;=\; \frac{M_a K_b - M_b K_a}{M_a V_b - M_b V_a}.
\end{equation}
These identities are exact, rely only on the quadratic law, and hold for any choice of $(q_a,q_b)$ with $M_a V_b \ne M_b V_a$.

\paragraph{Closed forms for $I_1$ and $\mathrm{Var}$ at $N=12$.}
For $N=12$, two arithmetically convenient choices are $q_a=\tfrac12$ and $q_b=\tfrac13$. Using the finite–sum identities \eqref{eq:S0-closed}–\eqref{eq:S2-closed} and $I_k=S_k/S_0$:
\[
\begin{aligned}
I_1\!\left(\tfrac12\right) &= \frac{2726}{1365}, 
&\qquad 
\mathrm{Var}\!\left(\tfrac12\right) &= \frac{3660914}{1863225},\\[4pt]
I_1\!\left(\tfrac13\right) &= \frac{199287}{132860}, 
&\qquad 
\mathrm{Var}\!\left(\tfrac13\right) &= \frac{13234051731}{17651779600}.
\end{aligned}
\]
Consequently,
\[
M_a = \left(\frac{2726}{1365}\right)^2,\quad
V_a = \frac{3660914}{1863225},
\qquad
M_b = \left(\frac{199287}{132860}\right)^2,\quad
V_b = \frac{13234051731}{17651779600}.
\]
Once $K_a=\kappa_{\mathrm{Schur}}(\tfrac12)$ and $K_b=\kappa_{\mathrm{Schur}}(\tfrac13)$ are evaluated from the $D_{12}$ projector–metric construction (with your fixed $m_\rho^2$), substitute into \eqref{eq:AB-closed} to obtain $A(12,m_\rho^2)$ and $B(12,m_\rho^2)$ in closed algebraic form (rational numbers for these two choices).

\paragraph{Consistency with the golden-ratio stationarity.}
For $(N,m_\rho^2)=(12,2)$, the values produced by \eqref{eq:AB-closed} satisfy the identity
\[
B\,\Lambda(12)+2A-2B \;=\; \frac{8}{m_\rho^2}\;=\;4,
\]
with $\Lambda(12)$ given explicitly in \eqref{eq:Lambda12-explicit}. This identity is exactly the bracket cancellation used in \S\ref{app:uniqueness-details} to prove $F'_{\mathrm{red}}(\theta_\star)=0$.

\paragraph{Remark (alternative sample points).}
Any two distinct $q_a,q_b\in(0,1)$ are valid. We recommend rational values with small denominators (e.g.\ $1/2$, $1/3$, $2/3$) because $S_k$ and hence $I_1,\mathrm{Var}$ simplify to rational numbers for finite $N$, which keeps \eqref{eq:AB-closed} purely rational when $\kappa_{\mathrm{Schur}}(q)$ is computed symbolically from your block Hessian.



\bibliographystyle{plain}
\bibliography{refs}

\end{document}
