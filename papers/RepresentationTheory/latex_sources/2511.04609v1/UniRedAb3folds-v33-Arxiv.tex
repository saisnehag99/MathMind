\documentclass[11pt,letter]{amsart}
\usepackage{amsmath, amssymb,amsthm}
\usepackage{color, hyperref}
\usepackage[all]{xy}
\usepackage{enumitem}

\usepackage{amsmath, amssymb,amsthm}
\usepackage{mathrsfs}


\renewcommand {\theenumi}{\roman{enumi}}

\textheight=600pt
\textwidth=455pt
\oddsidemargin=.21in
\evensidemargin=.21in
\topmargin=0in

\headheight=.1in
\headsep=.5in
\footskip=.75in
\renewcommand{\baselinestretch}{1.25}

\newtheorem{thm}{Theorem}[section]
\newtheorem{prop}[thm]{Proposition}
\newtheorem{lemma}[thm]{Lemma}
\newtheorem{cor}[thm]{Corollary}
\newtheorem*{claim}{Claim}
\newtheorem*{thm*}{Theorem}
\newtheorem{theoremletter}{Theorem}
\renewcommand*{\thetheoremletter}{\Alph{theoremletter}}


\theoremstyle{definition}
\newtheorem{defn}[thm]{Definition}
\newtheorem{rem}[thm]{Remark}



\newcommand{\A}{\mathbf{A}}
\newcommand{\C}{\mathbf{C}}
\newcommand{\F}{\mathbf{F}}
\newcommand{\G}{\mathbf{G}}
\newcommand{\Q}{\mathbf{Q}}
\newcommand{\R}{\mathbf{R}}
\newcommand{\Z}{\mathbf{Z}}


\newcommand{\rN}{\mathrm{N}}


\newcommand{\cF}{\mathcal{F}}

\newcommand{\cL}{\mathcal{L}}
\newcommand{\cM}{\mathcal{M}}
\newcommand{\cO}{\mathcal{O}}
\newcommand{\cP}{\mathcal{P}}
\newcommand{\cS}{\mathcal{S}}




\newcommand{\fg}{\mathfrak{g}}
\newcommand{\fq}{\mathfrak{q}}
\newcommand{\fc}{\mathfrak{c}}
\newcommand{\fh}{\mathfrak{h}}
\renewcommand{\fg}{\mathfrak{g}}
\newcommand{\fP}{\mathfrak{P}}
\newcommand{\fp}{\mathfrak{p}}
\newcommand{\fD}{\mathfrak{D}}
\newcommand{\fN}{\mathfrak{N}}

\newcommand{\bs}{\backslash}

\newcommand{\wK}{\widetilde{K}}
\newcommand{\wG}{\widetilde{G}}
\newcommand{\Cl}{\mathscr{C}\!\ell}



\DeclareMathOperator{\Ad}{Ad}

\DeclareMathOperator{\Alb}{Alb}
\DeclareMathOperator{\Aut}{Aut}
\DeclareMathOperator{\diag}{diag}

\DeclareMathOperator{\GL}{GL}
\DeclareMathOperator{\Gal}{Gal}
\DeclareMathOperator{\End}{End}
\DeclareMathOperator{\Lie}{Lie}
\DeclareMathOperator{\Ind}{Ind}
\DeclareMathOperator{\ind}{ind}
\DeclareMathOperator{\Hom}{Hom}
\DeclareMathOperator{\SL}{SL}
\DeclareMathOperator{\St}{St}
\DeclareMathOperator{\Spec}{Spec}
\DeclareMathOperator{\PGL}{PGL}
\DeclareMathOperator{\PSL}{PSL}
\DeclareMathOperator{\Tr}{Tr}
\DeclareMathOperator{\vol}{vol}
\DeclareMathOperator{\rH}{H}
\DeclareMathOperator{\rL}{L}
\DeclareMathOperator{\rU}{U}
\DeclareMathOperator{\MT}{MT}





\title[On the $\ell$-primary part of Abelian $3$-folds of Picard type]
{Uniform irreducibility of Galois action \\ on the $\ell$-primary part of Abelian $3$-folds of Picard type}
\author{Mladen Dimitrov \and Dinakar Ramakrishnan}

\address{University of Lille, Laboratoire Paul Painlev\'e, CNRS--UMR 8524,  59000 Lille, France}
\email{mladen.dimitrov@univ-lille.fr}


\address{Caltech, Mathematics 253-37,  Pasadena, CA 91125, USA}
\email{dinakar@caltech.edu}




\newcommand{\dedicationpage}{
  \begin{center}
  \usefont{\encodingdefault}{pzc}{m}{n}
Dedicated to the memory of Yuri Manin
  \end{center}
}

\begin{document}

\maketitle

\vspace{-1cm}
\begin{abstract}
Half a century ago Manin showed that given a  number field $k$ and a rational prime $\ell$,  there exists a uniform bound for the order of cyclic $\ell$-power isogenies between two non-CM elliptic curves over $k$. We generalize this to certain $2$-dimensional families of  abelian $3$-folds  with multiplication by an imaginary quadratic field.
\end{abstract}


\dedicationpage

\addtocontents{toc}{\setcounter{tocdepth}{0}}


\section*{Introduction}

Given a prime number  $\ell$ and a number field $k$,  Manin showed  in \cite{manin}  that there exists an integer  $r=r(\ell,k)$ such that for any non-CM elliptic curve $E$ over $k$, $E[\ell^r]\simeq (\Z/\ell^r\Z)^2$ does not contain a $k$-rational line, or equivalently that the image of the reduction modulo $\ell^r$ of its $\ell$-adic Galois representation
\[\Gal_k=\Gal(\bar{k}/k)\longrightarrow \Aut_{\Z/\ell^r\Z}\left(E[\ell^r]\right)\simeq \GL(2,\Z/\ell^r\Z)\]
is not contained in a Borel subgroup. Manin's original proof can be greatly simplified using Faltings' proof of Mordell's conjecture, which came later.
In a series of papers Cadoret and Tamagawa established a definitive result regarding the uniform boundedness of the $\ell$-primary torsion for $1$-dimensional families of abelian varieties. In this paper we prove an analogous statement
for certain $2$-dimensional families of  abelian $3$-folds which we believe to be the first  genuine
result over a  two-dimensional base.

Henceforth we fix an imaginary quadratic field $M$ of odd fundamental discriminant $-D$ and denote by  $\cO_M$  its ring of integers. An abelian $3$-fold of {\it Picard type} over a field $k$ containing $M$ will always  stand for a
principally polarized abelian variety over $k$ of dimension $3$   having multiplication by $\cO_M$ defined over $k$.   Its $\ell$-adic Tate module $T_\ell A$ is free of rank $3$ over $\cO :=\Z_\ell\otimes\cO_M$   endowed with a continuous  $\cO$-linear action of $\Gal_k$. By a line (resp. plane) in $T_\ell A$, we   would  mean a $\cO$-submodule of rank $1$ (resp. $2$) which is a direct factor. More generally, given a positive integer $r$, a line (resp. plane) in $A[\ell^r]$ will always be assumed be the image, under the natural reduction map, of a line (resp. plane) in  $T_\ell A$. Finally, by a full flag we would mean a tuple of a line  sitting as direct factor in a plane.
Lines (resp. planes) will be called $k$-rational if they are stable by $\Gal_k$ (but not necessarily point-wise fixed).

Our first main result addresses  the semi-stable case.

\begin{theoremletter} \label{theoremA}
Given a number field $k$, a prime number $\ell$ inert in $M$ and a finite set $S$  of places of $M$, there exists an integer $r=r(\ell,k,S)$ such that for any non-CM abelian $3$-fold  $A$ over $k$ of  Picard type which is semi-stable outside $S$,    $A[\ell^r]$ does not contain a  full $k$-rational flag.
\end{theoremletter}

As in the case of elliptic curves, the conclusion of Theorem~\ref{theoremA} asks  the image of the attached Galois representation
\[\Gal_k\longrightarrow \Aut_{\cO/\ell^r\cO}\left(A[\ell^r]\right)\simeq \mathrm{GU}(3,\Z/\ell^r\Z)\]
not to be contained in a Borel subgroup. Also, as in the case of elliptic curves, it is necessary to cast aside the  CM abelian varieties, as their  $\ell$-adic representations are potentially  reducible.

We next show how one can relax the semi-stability assumption by adding a tiny bit of level structure at $D$.
 Given a prime $v$ of $M$ above some $p\mid D$, the projective $\Gal_k$-action on the
 $\F_p$-vector space $A[v]$ yields a homomorphism $\widetilde\rho_{A,p}: \Gal_k \to \PGL_2(\F_p)$ (see \eqref{eq:star}).
 Taking quotient by the unique index two subgroup $\PSL_2(\F_p)$ of $\PGL_2(\F_p)$ yields a canonical homomorphism
$\varepsilon_{A,p}:  \Gal_k \to\{ \pm 1 \}$ and we let $\varepsilon_{A,D}=\prod_{p \mid  D} \varepsilon_{A,p}:  \Gal_k \to\{ \pm 1 \}$.



\begin{theoremletter} \label{theoremB}
Given a number field $k$ containing $M$ and a prime number $\ell$  inert in $M$, there exists an integer $r=r(\ell,k)$ such that for any non-CM  abelian $3$-fold  $A$ over $k$ of  Picard type  and such that $\varepsilon_{A,D}$ is trivial,   $A[\ell^r]$ does not contain a full $k$-rational flag.
\end{theoremletter}

Theorem~\ref{theoremB} is the main result of this paper and implies Theorem~\ref{theoremA} as follows.
Let $k'$ be the compositum of the (finitely many) quadratic extensions of $k$ which are unramified outside
$S$ and the primes dividing $D$. Given any abelian $3$-fold  $A$ as in Theorem~\ref{theoremA}, we claim that $\varepsilon_{A,D}(\Gal_{k'})$ is trivial. Indeed, by a theorem of Grothendieck \cite[Prop.~3.5]{grothendieck-monodromie}  the semi-stability of $A$ at $v\notin S$, $v\nmid D$ implies that the inertia subgroup of $\Gal_{k}$ at $v$ acts unipotently on the $D$-adic Tate module of $A$, in particular its image by $\varepsilon_{A,D}$ is pro-$D$ hence trivial (as $D$ is odd).
Therefore the base change of $A$ to $k'$  satisfies the additional assumption in Theorem~\ref{theoremB}, implying that  Theorem~\ref{theoremA}  holds with $r(\ell,k')$ from Theorem~\ref{theoremB}.

For an individual abelian variety $A$, the  conclusion of Theorem~\ref{theoremB} is a consequence of the
Mumford--Tate conjecture which is known for abelian $3$-folds (see \S\ref{MT-groups}), so the important feature of the result is its uniformity.  As  abelian $3$-folds of Picard type are parametrized by Shimura surfaces of Picard type, a natural way to proceed would be to show that the $k$-rational points are not Zariski dense in any of their connected components $Y_\Gamma$.
Let us for the moment consider the simpler situation from  our earlier paper \cite{dimitrov-ramakrishnan} where the congruence subgroups
$\Gamma$   were neat.  Our method there had two principal steps. The first step involved showing the existence of three linearly independent global holomorphic $1$-forms on the toroidal compactification $X_\Gamma$ (see \S\ref{odds-ends} for an amended list of 
$\Gamma$ to which our methods apply). By a theorem of Faltings concerning the associated Albanese variety this 
 implies  that the $k$-rational points on  $X_\Gamma$ are contained in a divisor $Z$, as predicted by a conjecture of Bombieri and Lang as $X_\Gamma$ turns out to be of general type. The second step consisted in applying a result of  Nadel requiring $\Gamma$  to be neat and the  canonical divisor to be big (in his sense) to deduce that any curve $C$ of genus $\leqslant 1$ contained in $X_\Gamma$ is in fact contained in the complement of $Y_\Gamma$. Consequently, every curve in $Z$ meeting the open surface $Y_\Gamma$ must be of genus $\geqslant 2$ thus, by Faltings' proof of Mordell's conjecture for curves, $Y_\Gamma(k)$ is finite for any number field $k$. 

Let us now say a few words about the techniques involved in the  proof of Theorem~\ref{theoremB}.
 As we are  led to consider congruence subgroups of Iwahori type  $\Gamma_0(\ell^r)$, which are {\it never neat} as they have torsion, both steps mentioned above encounter difficulty and we have to resort to new methods. We produce  irregularity 
 by constructing  explicit endoscopic automorphic forms in certain  non-generic representations  $\pi$  on the unitary group in $3$ variables.  It is here  that  the  index $2$ projective Galois image condition at $D$, suggested to us by Gross,  is essential, as otherwise all of the Picard modular surfaces involved would have trivial Albanese and our approach would fail for global reasons. 
 Making this strategy actually work yet requires  to address  some delicate representation theoretic  questions  to which a significant part of the paper is devoted and on which we will elaborate now. 
 

By Rogawski's theory $\pi$ is an element of an endoscopic Arthur packet  parametrized by an anti-cyclotomic (more precisely, conjugate-symplectic) Hecke character $\lambda$ of $M$.  Theorem~\ref{theoremB} imposes conditions so stringent so that 
$\lambda$ must  differ from Gross' minimally ramified `canonical' characters by a finite order  character $\chi$ 
only ramified at $\ell$.  The local  Arthur packet at $\ell$ contains two representations, a supercuspidal $\pi_{c,\ell}$ and a non-tempered $\pi_{n,\ell}$, both non-generic. The difficulty of finding  $\Gamma_0(\ell^r)$-invariants in $\pi_{c,\ell}$  forces us to work with the global  $\pi_{n}$, which is automorphic if, and only if, the root  number $W(\lambda^3)$ is $+1$. 
For $D\equiv 3\pmod{8}$   Gross' canonical characters work and  a computation of  matrix coefficients performed in \S\ref{iwahori-inv} shows that the resulting $\pi_{n,\ell}$  has invariants even by the hyperspecial maximal compact subgroup.  

When   $D\equiv 7 \pmod{8}$  the canonical characters yield the wrong sign, leading us to consider 
$\lambda$'s which are tamely ramified   at $\ell$ to switch the sign. 
It remains however to  show that the non-tempered representations $\pi_{n,\ell}$ attached to such $\lambda$'s
admit  $\Gamma_0(\ell^r)$-invariants for some $r$. For this we use a  more involved argument,  based on Jacquet modules and 
intertwining operators involving some precise averages of exponential sums,   occupying  the entirety  section \S\ref{intertwining}. 
 
  
Once the irregularity of $X_\Gamma$ has been shown to the at least $3$ and the Bombieri--Lang conjecture established, one has to 
deal with the possible curves $C$ of genus $\leqslant 1$ contained in $Y_\Gamma$. 
Using our key Lemma~\ref{lem:reflexions} affirming  that our Picard modular surfaces only admit a finite number of isolated singularities, 
we show that, after removing a finite number of points,  $C$ is  endowed with an abelian family
(see \S\ref{shimura-surfaces}).  This allows us apply the results of Cadoret and Tamagawa regarding the uniform boundedness of the Galois action on the Tate module of  such $1$-dimensional  families.

Finally, each of the finitely many non-CM  $k$-rational points are dealt with using the results 
on the Mumford--Tate conjecture for abelian $3$-folds of Picard type recalled in  \S\ref{MT-groups}. 


As our Picard modular surfaces $X_\Gamma$ have irregularity  $q\geqslant 3$, the Kodaira--Spencer classification implies that they are either ruled of genus $q$, or elliptic, or else they are of general type. In the last case, which according to Holzapfel \cite[\S 5.4]{holzapfel-book} occurs for all odd $D\notin\{3, 7,11,19, 23, 31, 39, 47, 71\}$,  we show that the   Bombieri--Lang Conjecture holds, {\it i.e.}, that  the $k$-rational points are not Zariski dense.  Investigating small values of $D$, as suggested by Mazur,  seems  even more interesting. It is established in {\it loc. cit.} that for all $D\ne 71$ in the above list the level $1$ Picard modular surfaces are rational and it would be natural to investigate the nature of their degree $2$ Gross covers that we consider. A way to shed light on this question  would be to find an explicit $2$-parameter family of abelian $3$-folds of  Picard type to which our theorem  applies.  


It might be worthwhile remarking that we could have also considered the simpler case of the moduli of principally polarized abelian {\it surfaces} $A$ over $k$ with multiplication by $\cO_M$,  which will involve ${\rm U}(1,1)$. However, as ${\rm SU}(1,1)\simeq \SL(2)$
this essentially  reduces  to the  modular curve case. On the other hand, if we consider
 principally polarized abelian  surfaces $A$ with {\it real multiplication}, then the family is parametrized by a Hilbert modular surface which has trivial ${\rm H}^1$, thus our methods, which rely on the Albanese variety, do not lead to an establishment the  Bombieri--Lang Conjecture.


  { \noindent{\it Acknolwedgements:} { \small
We would like to thank our respective institutions, and also the TIFR where part of the work was done. We would like to thank A.~ Cadoret, T.~Graber, M.~ Goreski, B.~Gross, H.~ Hida, A.~Jorza, K.-W.~Lan, B~.Mazur, J.~ Nekov\'a\v{r}, D.~ Prasad, A.~ Raghuram, D.Rohrlich, J.-P.~Serre and  J.~ Tilouine  for helpful discussions.

The first author acknowledges  support from  the I-SITE ULNE grant ANR-16-IDEX-0004. The second author thanks the Simons Foundation for the grant 523557. }



\tableofcontents



\addtocontents{toc}{\setcounter{tocdepth}{2}}



\section{Levels for endoscopic non-tempered representations of $\mathrm{U}(3)$}


It goes back to the work of Casselman that  admissible irreducible  representations having  non-zero Iwahori invariants are exactly those occurring as sub-quotients in  parabolic inductions of unramified characters.  Whereas  the dimension of the invariants by the depth $r$  Iwahori subgroup  in the
 full induced representation grows as  $r$  goes to infinity, this might not always be the case for all its sub-quotients, as shown by  the example of the trivial representation of $\GL_2$, realized as a quotient of a unramified principal series representation.

Another challenging question is to determine which sub-quotient of a parabolically induced unramified character picks up the invariants by a  given maximal open compact subgroup. Whereas MacDonald's formula for zonal spherical functions yields an answer in the case of a maximal hyperspecial subgroup, the general case appears to be an open question.

In this section we fully answer those two natural questions in the case of certain non-tempered endoscopic representations of $\rU_3$ attached  to a quadratic  extension $E/\Q_p$. It will be later applied in a global setting to $E=M_p$, where $M$ is  an imaginary quadratic field in which the prime $p$ does not split.





{\it In this section  of our paper we will adopt local notations. }

Let $E$ be a quadratic field extension of $\Q_p$, $\cO$ be its ring of integers, $\cP$ its maximal ideal and $\varpi$ a uniformizer.
We assume that $E$ is {\it not} a ramified extension of $\Q_2$.
  Denote by $x\mapsto \bar x$  the automorphism of $E$ induced by the non-trivial element of $\Gal(E/\Q_p)$ and fix a generator $\xi$  of the different $\mathcal{D}$ of $E/\Q_p$  such that $\overline\xi=-\xi$.
We fix an additive character $\psi: \Q_p\to \C^\times$ of conductor $0$, i.e. $\ker(\psi)=\Z_p$, and we consider the additive character
$\psi_E$ of $E$ defined as $\psi_E(x)=\psi(\Tr_{E/\Q_p}(x))$.


   Let $G$  be the  unique  quasi-split unitary group  in $3$ variables relative to the extension  $E/\Q_p$. It can be realized as the automorphisms of  $E^3$ preserving  the hermitian pairing
\[\langle x,y\rangle = \bar x_1 y_3 + \xi \bar x_2 y_2 -\bar x_3 y_1.\]


\subsection{The Bruhat-Tits tree of  $\rU_3$}\label{tree}
As $G$ has rank $1$, its Bruhat-Tits building is a tree. We will first describe its standard apartment.
The relative roots of $G$  are obtained by decomposing the adjoint action on the  Lie algebra of the maximal $\Q_p$-split torus $T_0=\left\{\mathrm{diag}(a,1,a^{-1}) |    a\in \Q_p^\times \right\} $ of $G$. The positive elements of the associated root system $\Phi$ are $\{\zeta, 2\zeta\}$. Let $h: \G_m \to T_0  \subset  G$ be the generator of the co-character lattice $X_\ast(T_0) \simeq \Z$ such that $\langle \zeta, h\rangle=1$. Then the co-root sub-lattice is generated by $\zeta^\vee = 2h$, so that we have the standard normalization $\langle\zeta, \zeta^\vee\rangle=2$. According to  \cite[\S 1.15]{tits} the affine roots are
$\{ \pm \zeta+\Z\}\cup \{ \pm 2\zeta+\Z\}$ if $E$ is unramified, and
$\{ \pm \zeta+\tfrac{1}{2}\Z\}\cup \{ \pm 2\zeta+\Z+\tfrac{1}{2}\}$ if $E$ is ramified;
note that $\delta=0$ in {\it loc. cit.} as $E$ is not a ramified extension of $\Q_2$.
The apartment associated to $T_0$ is $\R h$ and its  walls are the vanishing sets of these (affine) roots,
hence they are given by $\tfrac{1}{2}\Z h= \Z h \cup \tfrac{1}{2}\Z h$, resp. $\tfrac{1}{4}\Z h=
\left(\tfrac{1}{2}\Z +\tfrac{1}{4}\right)h\cup \tfrac{1}{2}\Z h$,  if  $E$ is unramified, resp. ramified.
Given an $\cO$-lattice $\cL$ in $E^3$ we define its dual as
\[\cL^\perp=\Hom_{\cO}(\cL, \mathcal{D}^{-1})=  \{x\in E^3 |    \langle x, \cL\rangle \subset  \mathcal{D}^{-1}\}.\]


\begin{lemma} \label{special}
There are two conjugacy classes of maximal compact subgroups in $G$, those which are  stabilizers of
 self-dual lattices, and those which are stabilizers of  almost self-dual   lattices, {\it i.e.},   lattices $\cL$ such that $\cL\subsetneq \cL^\perp \subsetneq\varpi^{-1}\cL$.  They are all special, and the hyperspecial ones are those
 stabilizing a self-dual lattice when $E$ is unramified.
\end{lemma}
\begin{proof}
A conjugacy class of maximal compact subgroups can be represented by a wall in the standard apartment.
By definition, a wall is hyperspecial if for every $\zeta'\in \Phi$ here exists an affine root with gradient
$\zeta'$ vanishing on that wall. Since $\left(\tfrac{1}{2}\Z +\tfrac{1}{4}\right)h\cap \tfrac{1}{2}\Z h=\varnothing $
this only can happen when $E$ is unramified, in which case the hyperspecial walls are
$ \Z h \cap \tfrac{1}{2}\Z h= \Z h$. All walls are special, as elements of $\Phi$ are rational multiple of one another.
\end{proof}

We now give an explicit description of the maximal compact subgroups corresponding to the
walls of a chamber in the standard apartment.

The standard maximal compact subgroup $K^\circ$ of $G$ is defined as the stabilizer of the self-dual lattice
$\cL^\circ=\cO\oplus \xi^{-1}\cO\oplus \xi^{-1}\cO$. It is hyperspecial if and only if $E$ is unramified, and
 $$K^\circ=\begin{pmatrix} \cO & \xi\cO& \xi\cO\\
\xi^{-1}\cO  & \cO &\cO \\ \xi^{-1}\cO  &\cO  & \cO \end{pmatrix}\cap G.$$
The reductive quotient $\overline{G}{}^\circ$ is given by $\rU_3(\F_p)$ if $E$ is unramified, and by $\mathrm{O}_3(\F_p)$ if $E$ is ramified.

The other standard maximal compact subgroup  $K'$ of $G$, defined as the stabilizer of the  almost self-dual  lattice $\cL'=\cO\oplus \xi^{-1}\cO\oplus \xi^{-1}  \cP$, is given by
 $$K'=\begin{pmatrix} \cO & \xi\cO& \xi\cP^{-1}\\
\xi^{-1}\cP & \cO^\times & \cO \\ \xi^{-1}\cP   & \cP  & \cO \end{pmatrix}\cap G.$$
One has
$\cL'^\perp=\cP^{-1} \oplus \xi^{-1}\cO  \oplus \xi^{-1}\cO$  and
 $K'$ acts on $\varpi^{-1}\cL'/\cL'^\perp\simeq \cO/\cP$ via its middle coefficient.
The reductive quotient $\overline{G}{}'$ is isomorphic to   $ \rU_{1,1}(\F_p)\times \rU_1(\F_p)$ if $E$ is unramified, and to
$\pm\SL_2(\F_p)\times\{\pm 1\}$,  if $E$ is ramified.


The standard Iwahori subgroup of $G$ is defined as
$I=K^\circ\cap K'=\begin{pmatrix} \cO^\times & \xi\cO & \xi\cO\\
\xi^{-1}\cP  & \cO^\times &\cO \\ \xi^{-1}\cP  &\cP  & \cO^\times \end{pmatrix}\cap G$.

Finally let
$I_{2,1}=K^\circ  \cap \gamma K^\circ \gamma^{-1}=\begin{pmatrix} \cO^\times & \xi\cO &  \xi\cO\\
\xi^{-1}\cP  & \cO^\times &\cO \\ \xi^{-1}\cP^2  & \cP  & \cO^\times \end{pmatrix}\cap G$, where
$\gamma=\left(\begin{smallmatrix}\varpi^{-1}  & & \\ & 1 & \\ & & \varpi \end{smallmatrix}\right)$.


The standard apartment in the Bruhat-Tits tree of $G$ is as follows:
$$\xymatrix@C=40pt{
\ar@{--}[r]  & \gamma^{-1} K' \gamma
  \ar@{-}[r] & K^\circ \ar@{-}_{I}[r] \ar@/^1pc/@{.}[rr]^{I_{2,1}}& K'
  \ar@{-}[r] & \gamma K^\circ \gamma^{-1} \ar@{--}[r] &}$$
If  $E$ is unramified,  then  $G\supset K'\supset I \supset I_{2,1}\supset I_2$, where $I_r=\left(\begin{smallmatrix}\cO^\times &\cO &\cO\\p^r\cO &\cO^\times &\cO\\ p^r\cO & p^r\cO &\cO^\times\end{smallmatrix} \right)\cap G$.





\subsection{Review  of $L$-parameters and $A$-packets} \label{packets}

For any integer $n\geqslant 1$ there are exactly two (up to isomorphism) $n$-dimensional hermitian spaces over $E$, depending on the image of the discriminant in $\Q_p^\times/\rN_{E/\Q_p}(E^\times)$, and the corresponding unitary groups $\rU(n)$ are isomorphic if and only if $n$ is odd. When $n=2$, by analogy with the Archimedean case, we will denote by $\rU(1,1)$ the quasi-split form and by $\rU(2)$ the compact one.

The $L$-group  (of the  quasi-split form) of $\rU(n)$ is given by $\GL_n(\C)\rtimes W_{\Q_p}$ with the Weil group $W_{\Q_p}$ acting on $\GL_n(\C)$ through its
quotient $\Gal(E/\Q_p)$ whose non-trivial element sends  $g$ to $w_n{}^{t}g^{-1} w_n^{-1}$, where   $w_n$   denotes the  anti-diagonal matrix $(1,-1,1,\dots,(-1)^{n-1})$. By definition, an $L$-parameter for $\rU(n)$  is  a homomorphism
$W_{\Q_p}\times \SL_2(\C) \longrightarrow {}^LG$,  but as one knows (see \cite[\S 3]{GGP1})  it is equivalent to
ask for its restriction
\[\phi: W_E\times \SL_2(\C) \longrightarrow \GL_n(\C), \]
to be conjugate $(-1)^n$-dual, {\it i.e., conjugate-orthogonal} if  $n$ is odd and {\it conjugate-symplectic} if $n$ is even.
Recall that $\phi$  is conjugate-self-dual  if $\bar\phi\simeq \phi^\vee$, or equivalently, if the induced representation $\Ind_{W_E}^{W_{\Q_p}}(\phi)$ is self-dual. Furthermore,  $\phi$ is
conjugate-orthogonal, resp. conjugate-symplectic, if it  preserves a non-degenerate symmetric, resp. skew-symmetric, bilinear form. Note that while Schur's Lemma implies that any irreducible self-dual  (or conjugate-self-dual) parameter has a well defined sign, this need not be always the case for reducible parameters.

For $n=1$, a character of  $E^\times$ is conjugate-orthogonal (resp. conjugate-symplectic) if its
 restriction to $\Q_p^\times$ is   trivial (resp. is the quadratic character attached to $E/\Q_p$). For $n\in \Z_{\geqslant 0}$, the $n$-th symmetric power of the standard $2$-dimensional representation $\St$ of SL$(2,\C)$, with $W_E$ acting trivially, is  conjugate-symplectic if $n$ is odd and conjugate-orthogonal if $n$ is even.

The base change $\nu_E(z)=\nu(z/\overline z)$ of a character $\nu$  of $E^1$ is conjugate-orthogonal and conversely any 
conjugate-orthogonal character of $E^\times$ is  obtained in that way. 
For $\lambda$   a conjugate-symplectic character of $E^\times$, an example of key relevance to us is the  conjugate-orthogonal representation
\[  (\lambda \otimes \St) \oplus  \nu_E  : W_E\times \SL_2(\C) \longrightarrow \GL_3(\C). \]
It yields an  $L$-parameter $\phi_{\lambda, \nu}$ of $G$, coming from an $L$-parameter of the (unique) cuspidal endoscopic subgroup $H= \rU(1,1) \times \rU(1)$ of $G$. The cardinality of the corresponding $L$-packet  $\Pi_L(\phi_{\lambda, \nu})$ is given by  the order of the centralizer in ${}^LG^0$ modulo the center which turns out to be $2$. More precisely, $\Pi_L(\phi_{\lambda, \nu})$ contains two  discrete series representations $\pi_2$ and $\pi_c$ of $\rU(3)$, exactly one of them,  namely $\pi_c$, being supercuspidal (see  \cite[Chap.~12.2]{rogawski-U3} where this $L$-packet is denoted $\Pi_L(\St_H(\xi))$). There is another endoscopic $L$-packet for $G$ consisting of a single 
non-tempered representation $\pi_n$ whose  the $L$-parameter  is given by
\[\lambda |\cdot |_E^{1/2}\oplus \lambda |\cdot|_E^{-1/2}\oplus \nu_E: W_E\times \SL_2(\C) \longrightarrow \GL_3(\C). \]

 Rogawski's theory \cite{rogawski-U3,rogawski-A-packets} describes the automorphic representations contributing to the $\rH^1$ of  Shimura surfaces of Picard type  in terms global Arthur packets (see \cite[\S 3.1]{dimitrov-ramakrishnan} for a summary). The corresponding local Arthur packet  at $p$ has $2$ elements
 $\Pi(\lambda,\nu)=\{\pi_n,\pi_c\}$ (see \cite[\S 12.3.3]{rogawski-U3}, where $\pi_c$ is denoted $\pi^s$), and the restriction to $W_E$ of its  $A$-parameter  is given by
\[
(\lambda\otimes \mathbf{1} \otimes \St) \oplus  \nu_E  : W_E\times \SL_2(\C)\times \SL_2(\C) \longrightarrow \GL_3(\C), \]
while the $A$-parameter of $\pi_2$ is given by $(\lambda  \otimes \St \otimes \mathbf{1}) \oplus  \nu_E$. 


Crucial for us would be the description $\pi_n$ and $\pi_2$ as  the Jordan--H\"older constituents of a 
principal series representation $\pi$. Indeed, by \cite[\S 1]{rogawski-A-packets}, $\pi_n$ is the Langlands quotient 
of the  (unitarily normalized) parabolic   induction  of the character 
\begin{equation}
\mu(\bar\alpha, \beta, \alpha^{-1})= \lambda(\bar\alpha)\nu(\beta)|\alpha|_E^{1/2}, 
\end{equation}
with $\pi_2$  being the unique non-zero irreducible sub-representation. Moreover, the extension 
\begin{equation}\label{induction}
0\to \pi_2 \to \pi = \Ind_B^G(\mu)\xrightarrow{\mathrm{pr}} \pi_n \to 0
\end{equation} 
does not split, and the sub and quotient are switched when $\mu$ is  replaced by 
$\mu^w(\bar\alpha, \beta, \alpha^{-1})= \lambda(\bar\alpha)\nu(\beta)|\alpha|_E^{-1/2}$. 
The Jacquet functor is exact and it sends $\pi_2$, resp. $\pi_n$, to 
$\mu\delta^{1/2}$, resp. $\mu^w\delta^{1/2}$, where $w$ is the non-trivial element of the Weyl group of $G$, and
$\delta(\bar\alpha, \beta, \alpha^{-1})= |\alpha|_E^2$ is the modulus character. 
Fixing a non-degenerate character of the unipotent radical $N$ of $B$, one knows by Rodier  \cite[Thm.~2]{rodier}  that
the image of $\Ind_B^G(\mu)$ by a twisted version of the Jacquet functor, singling out generic representations,  is a line. The later being also  exact, this 
implies that exactly one amongst $\pi_2$  and  $\pi_n$  is generic. Since  $\pi_n$ is non-generic  (see \cite[p.174]{rogawski-U3}), this implies that $\pi_2$ is generic.

As $\pi_n$ is non-tempered, the subspace $\pi_2$ consists of  $f\in\pi $ such that for all $f^\vee\in \pi ^\vee$ the matrix coefficient 
$g\mapsto \langle g\cdot f,  f^\vee \rangle$ belongs to $\rL^2(G)$. Conversely the following lemma holds. 

\begin{lemma} \label{lem:L2-crit} Let $f\in \pi $. 
If  $g\mapsto \langle g\cdot f,  f^\vee \rangle$ belongs to $\rL^2(G)$ for some $0\ne f^\vee\in \pi ^\vee$, then $f\in \pi_2$. 
\end{lemma}
\begin{proof} The dual of \eqref{induction} is given by
\[0\to \pi_n^\vee \to \pi ^\vee=\Ind_B^G(\mu^{-1}) \to \pi_2^\vee \to 0,\]
and the irreducibility of  $\pi_2$ and $\pi_n$ implies that $\pi_n^\vee=\{f^\vee\in \pi ^\vee | \langle \pi_2, f^\vee\rangle=0\}$. As  $f^\vee\ne 0$, its $G$-span contains  $\pi_n^\vee$, implying that the matrix coefficient  
$g\mapsto \langle g\cdot f,  f^\vee \rangle$ belongs to $\rL^2(G)$  for all $f^\vee\in\pi_n^\vee$. One deduces that
\[g\mapsto \langle g\cdot f,  f^\vee \rangle= \langle \mathrm{pr}(g\cdot f),  f^\vee \rangle=\langle g\cdot \mathrm{pr}(f),  f^\vee \rangle\in \rL^2(G)\]
As the irreducible $\pi_n$ is not a discrete series representation, this implies  $\mathrm{pr}(f)=0$, {\it i.e.} $f\in \pi_2$. 
   \end{proof}



We will be mostly interested in the following $A$-packets having trivial central characters:
\begin{equation}\label{A-packet-lambda}
\Pi(\lambda)=\Pi(\lambda,\lambda_{| E^1}^{-1}).
\end{equation}


\subsection{The Gross subgroup $K''$}\label{K-second}
In this  subsection, $E$ is  ramified (hence $p$ is  odd). Then $\cO/\cP=\F_p$ and $\cP=\xi\cdot \cO$.
As $|\PGL_2(\F_p)|=|\SL_2(\F_p)|$ all vertices in the  tree of $G$  have valence $p^3+1$
 The map from $K^\circ$  to its
reductive quotient $\overline{G}{}^\circ$ is obtained by reducing $\left(\begin{smallmatrix} \xi/2   & & \\ & 1 & \\ & & 1 \end{smallmatrix}\right)^{-1}K^\circ \left(\begin{smallmatrix} \xi/2  & & \\ & 1 & \\ & & 1 \end{smallmatrix}\right)$ modulo $\cP$
and a direct computation  shows that $\overline{G}{}^\circ$ is isomorphic to  the orthogonal group  $\mathrm{O}_3(\F_p)$ with respect to the quadratic form
represented by   $\left(\begin{smallmatrix}  & & 1 \\ & 2 & \\ 1 & &  \end{smallmatrix}\right)$.


Note that $\mathrm{O}_3(\F_p)=\pm \mathrm{SO}_3(\F_p)$ and the  adjoint action on matrices $\left(\begin{smallmatrix} y & x \\  z  & -y \end{smallmatrix}\right)$
preserving the determinant $-(y^2+xz)$  allows
us to identify $\PGL_2(\F_p)$ and $\mathrm{SO}_3(\F_p)$   as follows:
\begin{equation}\label{eq:adjoint}
\begin{pmatrix} a & b \\  c  & d \end{pmatrix}\mapsto \frac{1}{ad-bc}
\begin{pmatrix} a^2 & 2ab & -b^2\\ ac  & ad+bc &-bd \\ -c^2  &-2cd  & d^2 \end{pmatrix}.
\end{equation}
It follows from that description that  $\mathrm{SO}_3(\F_p)$ is generated by the set
\[\left\{\left(\begin{smallmatrix} & & -1\\   & -1 &\\ -1 & & \end{smallmatrix}\right),
\left(\begin{smallmatrix} a & &\\  & 1 & \\ & & a^{-1} \end{smallmatrix}\right),
\left(\begin{smallmatrix} 1 & 0 & 0\\ c  & 1 &0 \\ -c^2  &-2c & 1 \end{smallmatrix}\right)\Big{|}   a\in \F_p^\times,
c\in\F_p\right\}.\]



\begin{defn}\label{def-double-prime}
Let $K''$  be the index $2$ subgroup of $K^\circ$ defined as the inverse image of the
subgroup of $\mathrm{O}_3(\F_p)$ generated by  $-\mathbf{1}$  and the image of $\PSL_2(\F_p)$. Let $I''=K''\cap K'\subset I$.
\end{defn}

We recall that  $E/\Q_p$ is a ramified quadratic extension with $p$ odd. A conjugate-symplectic character 
$\lambda$ of $E^\times$ is necessarily ramified and its restriction to $\Z_p^\times$ is given by its unique quadratic character. 
If $\lambda$ is tamely ramified, then  its restriction to $\cO^\times$ is also given by its unique quadratic character, and 
the equation $ \lambda (\xi)^2= \lambda(-\xi\bar\xi)=\lambda(-1)=(-1)^{(p-1)/2}$ shows that there are precisely two such characters.  

Interested in determining a level for an element of the $A$-packet $\Pi(\lambda)$ considered in \eqref{A-packet-lambda}, 
we are indebted to B.~Gross for generously sharing a suggestion that led to the following proposition. 


\begin{prop} \label{ram-inv}  
Let $\lambda$ be a  tamely ramified conjugate-symplectic character of $E^\times$ and let 
$\pi_{n}$ be the non-tempered member of the $A$-packet $\Pi(\lambda)$. 
Then  $\dim\pi_{n}^{K''}=\dim\pi_2^{I''}=1$.
\end{prop}
\begin{proof} As $\mu_{|(T\cap I'')}=\mu^w_{|(T\cap I'')}=\mathbf{1}$, applying the  Jacquet functor to the exact sequence
of admissible  $G$-representations \eqref{induction} allows one to see that both  $\pi_2$ and $\pi_n$ have non-trivial $I''$-invariants.
Moreover, as $\big{|}B\backslash G/ I'' \big{|}=\big{|}(B\cap K'')\backslash K''/ I'' \big{|}=2$,  both $\pi_2^{I''}$ and $\pi_n^{I''}$ must be   $1$-dimensional.
By Iwasawa decomposition, the restriction of $\Ind_B^G(\mu)$  to $K''$ is given by $\Ind_{B\cap K''}^{K''}(\mu)$, hence  the
line $\Ind_B^G(\mu)^{K''}$ admits a basis  $f$ uniquely characterized by $f_{|K''}= \mathbf{1}_{K''}$. It follows that
 $\dim\pi_{n}^{K''}+\dim\pi_2^{K''}=1$ and we will show that $\pi_2^{K''}=\{0\}$.

The  line $(\Ind_B^G(\mu^{-1}))^{K''}$ admits a basis  $f^\vee$ uniquely characterized by
\[\langle v, f^\vee\rangle = \frac{1}{\sqrt{\vol(K'')}} \int_{K''} v(k) dk. \]

By Lemma~\ref{lem:L2-crit} one has   $f\notin \pi_2$ if, and only if,
$(g\mapsto \langle g\cdot f,  f^\vee \rangle)\notin \rL^2(K''\backslash G/K'')$.


Recall    $\gamma =\left(\begin{smallmatrix} -\xi^{-1}& &  \\ & 1 & \\ & &  \xi  \end{smallmatrix} \right)$ and let
$\eta =\left(\begin{smallmatrix} \bar u& &  \\ & 1 & \\   & & u^{-1} \end{smallmatrix} \right)$ where
$u\in\cO^\times$ is a fixed non-square element.

As $K^\circ=K''\coprod \eta K''$,  Cartan decomposition for the special maximal compact $K^\circ$ yields:
\[ G= \coprod_{n\geqslant 0} \left(K''\gamma^n K''\right) \amalg \left(K'' \gamma^n \eta K''\right).\]


Since $\eta\cdot f=-f$ one deduces that  $\langle  \gamma^n \eta \cdot f,  f^\vee \rangle=-\langle \gamma^n \cdot f,  f^\vee \rangle$
and checking  that $f\notin \pi_2$  amounts to proving the divergence of  the numerical sequence with general term
\[\vol(K''\gamma^n K'') \big|\langle \gamma^n \cdot f,  f^\vee \rangle \big|^2=[K'':(K''\cap \gamma^n K''\gamma^{-n})]
 \Big|\int_{K''} f(k \gamma^n) dk \Big|^2.\]



By Iwahori decomposition one has $[I'':(K''\cap \gamma^n K''\gamma^{-n})]=p^{2n-1}$ for all $n\geqslant 1$.
As $\big|(\mu\delta^{1/2})(\gamma)\big|= p^{3/2}$ we are led  to establish the divergence of  the sequence with general term
\[ \Phi_n=p^{5n/2} \cdot \Big|\int_{K''} f(\gamma^{-n} k \gamma^n) dk \Big|.\]

In view of the  inequality $p>\sqrt{p}+1$ for $p\geqslant 3$, this will follow from the next lemma.  \end{proof}


\begin{lemma} For all $n\geqslant 1$ one  has  $\big|\int_{K''\backslash K''_{2n}} f(\gamma^{-n} k \gamma^n) dk \big|\leqslant \vol(I'') \cdot (\sqrt{p}+1)p^{-2n}$ and    $\big|\int_{K''_{2n}} f(\gamma^{-n} k \gamma^n) dk \big|=\vol(I'')\cdot p^{1-2n}$.
\end{lemma}
\begin{proof}
The last row of an element $k\in K''$ is given by  $(0,0,1)\cdot k= (\xi^{-1}\cdot c_1(k),  c_2(k), c_3(k))$ with
$(c_1(k),c_2(k),c_3(k))\in \cO^3  \backslash (\xi\cO)^3$. For $j\geqslant 0$ we let
\[K''_j=\left\{k\in K'' \big{|} c_1(k)\in \xi^j\cO \right\} \text{ and } K^{\prime\prime\times}_j=K''_j\backslash K''_{j+1}.\]
Note that $K''_0=K''$ and $K''_1=I''$. We use the partition $K''\backslash K''_{2n}=K_0^{\prime\prime\times}\coprod K^{\prime\prime\times}_1\coprod \dots \coprod  K^{\prime\prime\times} _{2n-1}$   to compute the first integral and $K''_{2n}=K^{\prime\prime\times}_{2n} \coprod K''_{2n+1}$
for the second.

For $j\geqslant 1$, using Iwahori decomposition,  one finds that $[I'':K''_j]=  [I''\cap N^-: K''_j\cap N^-]=p^{j-1}$.

Given any $k\in K^{\prime\prime\times}_j (0\leqslant j \leqslant 2n)$, using the Iwasawa decomposition $G=\gamma^{\Z} N K^\circ=\gamma^{\Z} N K'$,
one finds that $\gamma^{n-j} k \gamma^n\in N K^\circ$, hence $\vert f(\gamma^{-n} k \gamma^n) \vert\leqslant \vert \mu(\gamma^{j-2n})\vert=
p^{\frac{3}{2}j-3n}$.  Therefore
\[\left|\int_{K''\backslash K''_{2n}} f(\gamma^{-n} k \gamma^n) dk \right|\leqslant \vol(K''\backslash I'')\cdot p^{-3n} + \vol(I'')
\sum_{j=1}^{2n-1} p^{\frac{1}{2}j-3n}(p-1)= \]
\[=\vol(I'') \left(p^{\frac{1}{2}-2n}+p^{-2n}-p^{1-3n}-p^{\frac{1}{2}-3n}+([K'':I'']-1)p^{-3n}   \right),\]
proving the desired inequality, as $[K'':I'']=p+1$ (obtained by  going to the reductive quotient).

Since $\vol(I'')\cdot p^{1-2n}=\vol(K''_{2n})$, in order to complete the proof of the lemma,  it suffices to show that
$f(\gamma^{-n} k \gamma^n)$ in constant on  $k\in K''_{2n}$. This is evident for $k\in K''_{2n+1}$, as then $k$ and
$\gamma^{-n} k \gamma^n$ both belong to $I''$ and share same determinant and lower right coefficient $c_3$, implying that
$f(\gamma^{-n} k \gamma^n)=f(k)$. Miraculously, as one can see from \eqref{eq:adjoint}, this remains true for $k\in K''_{2n}\backslash K''_{2n+1}$ as well, {\it i.e.} even thought $\gamma^{-n} k \gamma^n\in K^\circ \backslash I$, the fact that $c_3(\gamma^{-n} k \gamma^n)=c_3(k)$ still implies that
$\gamma^{-n} k \gamma^n\in K''$.  \end{proof}




\subsection{Higher Iwahori invariants via matrix coefficients}\label{iwahori-inv}
In this  subsection we assume that $E/\Q_p$ is unramified. 
The unique unramified Arthur packet is given by $\Pi(\lambda_0)=\Pi(\lambda_0, \mathbf{1})$, where 
 $\lambda_0$ is the unique quadratic unramified character of $E^\times$. 

  It follows from   Iwasawa decomposition that the corresponding  $\Ind_B^G(\mu_0)$ has one dimensional invariants by
 any given maximal open compact subgroup $K$  of $G$. The following proposition  states, depending on $K$,
 whether the $K$-invariant line belongs to $\pi_2$ or maps non-trivially to $\pi_n$.
We recall that $K^\circ$ and $K'$ are the two standard maximal compact subgroups, $K^\circ$ being the hyperspecial one, and that the standard Iwahori subgroup $I$ equals $K^\circ \cap K'$. 

\begin{prop}\label{prop:inert-inv} One has $\dim\pi_2^{K'}=\dim\pi_n^{K^\circ}=1$, if $\mu$ is unramified, and 
$\pi_2^I=\pi_n^I=\{0\}$, otherwise. 
\end{prop}

\begin{proof}
Applying the  Jacquet functor to the exact sequence \eqref{induction} allows one to see that both $\pi_2$ and $\pi_n$ have non-trivial $I$-invariants if $\mu$ is unramified, and none, otherwise. Assuming henceforth that $\mu=\mu_0$ is unramified, we observe that both $\pi_2^I$ and $\pi_n^I$ must be   $1$-dimensional. Moreover, as by Cartan decomposition $G$ is generated by $K$ and $\gamma$, hence by $K^\circ$ and $K'$, it follows  that necessarily  one amongst $\pi_2^I$ and  $\pi_n^I$  is fixed by $K^\circ$, while the other one is fixed by $K'$. 

By Iwasawa decomposition, the restriction of $\pi=\Ind_B^G(\mu_0)$  to $K$ is given by $\Ind_{B\cap K}^K(\mu_0)$, hence  the
line $\pi ^K$ admits a basis  $f_K$  characterized by asking its restriction to $K$ to  be $\mathbf{1}_K$.
Moreover, the  line $(\pi ^\vee)^K$ admits a basis  $f_K^\vee$  characterized by
\[\langle f, f_K^\vee\rangle = \frac{1}{\sqrt{\vol(K)}}\int_K f(k) dk. \]


The remainder of the proof consists in computing the bi-$K$-invariant function $g\mapsto \langle g\cdot f_K,  f_K^\vee \rangle$ and checking whether it belongs or not to
$\rL^2(K\backslash G/K)$. Using Cartan decomposition $G=\coprod_{n\geqslant 0} K\gamma^n K$ this amounts to checking whether
$\rL^2(\Z_{\geqslant 0})$ contains the numerical sequence
\[\sqrt{\vol(K\gamma^n K)} \langle \gamma^n \cdot f_K,  f_K^\vee \rangle=\sqrt{[K:(K\cap \gamma^{-n}K\gamma^n)]}
\int_K f_K(k \gamma^n) dk.\]
Using Iwahori decomposition for all $n\geqslant 1$ we have  $[K:(K\cap \gamma^{-n}K\gamma^n)]/[K:I]=p^{4n-3}$ (resp. $p^{4n-1}$), where  $K=K^\circ$ (resp. $K'$). Since $(\mu\delta^{1/2})(\gamma)= -p^3$ we have $f_K\in \pi_2$ if, and only if,
\begin{align}\label{Phi-n}
(\Phi^K_n)_n\in \rL^2(\Z_{\geqslant 0}) \text{, where }\Phi^K_n=p^{5n} \cdot \int_K f_K(\gamma^{-n} k \gamma^n) dk.
\end{align}

The proof of Proposition is then completed  by the following  Lemma.
\end{proof}


\begin{lemma} The quantity  $p^{n}\cdot\Phi^{K'}_n$ is independent of $n\geqslant 1$, in particular  $(\Phi^{K'}_n)_n\in \rL^2(\Z_{\geqslant 0})$.
\end{lemma}
\begin{proof}
The last row of an element $k\in K'$ is given by  $(0,0,1)\cdot k= (p\cdot c_1(k), p\cdot c_2(k), c_3(k))$ with $c_2(k)\in \cO$ and $(c_1(k),c_3(k))\in
(\cO\times \cO) \backslash (\cP\times \cP)$. For $j\geqslant 0$ we let
\[K'_j=\left\{k\in K' \Big{|} c_1(k)\in \cP^j\right\} \text{ and } K^{\prime\times}_j=K'_j\backslash K'_{j+1}.\]
To compute the above integral we use the partition $K'=K_0^{\prime\times}\coprod K^{\prime\times}_1\coprod \dots \coprod  K^{\prime\times} _{2n-1}\coprod K'_{2n}$.

First, we compute the volume of $K'_j$, for $j\geqslant 1$. Using Iwahori decomposition one finds that:
\[ [K':K'_j]= \frac{[K':I]}{[K'_j:I\cap K'_j]}[I:I\cap K'_j]=  \frac{[K':I]}{[K' \cap N : I\cap N ]}[I\cap N^-: K'_j\cap N^-]=c_0\cdot p^{j+2\left[\frac{j}{2}\right]}. \]


Next we observe that by Iwasawa decomposition, for all $k\in K'_{2n}$ one has $c_2(k)\in \cP^n$, {\it i.e.} $\gamma^{-n} k \gamma^n\in N\cdot K'$, and
therefore $f_K(\gamma^{-n} k \gamma^n)=1$.

Using again Iwasawa decomposition, one checks that  for  $0\leqslant j \leqslant 2n-1$ and for every $k\in K^{\prime\times}_{j}$ one has
$p^{n-j} c_2(k)\in \cO^\times$, hence  $\gamma^{-n} k \gamma^n\in \gamma^{j-2n} N\cdot K'$ and $f_K(\gamma^{-n} k \gamma^n)=(-p^3)^{j-2n}$.

Therefore $\displaystyle \frac{1}{\vol(K')} \int_{K'} f_K(\gamma^{-n} k \gamma^n) dk= \frac{1}{[K':K'_{2n}]} + p^{-6n} \sum_{j=0}^{2n-1} (-1)^j p^{3j}
\frac{1}{[K':K^{\prime\times}_j]}=$

\[  =c_0 \cdot p^{-6n} \left(  p^{2n} +c_0^{-1} -  \sum_{i=1}^{n}  p^{6i-3}(p^{-4i+3}- p^{-4i})  + \sum_{i=0}^{n-1} p^{6i}(p^{-4i} - p^{-4i-1}) \right)=
 p^{-6n} (1+ c_0).\qedhere \]
\end{proof}

\begin{rem} Alternatively,  one could use MacDonald's formula for zonal spherical functions  to see that $\pi_2$ does not admit  non-zero  vectors fixed by  the  hyperspecial maximal   open compact subgroup $K^\circ$. Indeed,  \cite[\S 5.5]{haines-kottwitz-prasad}
allows to express $\Phi^{K^\circ}_n$ from \eqref{Phi-n},  up to a non-zero constant, as
\[   \Gamma_\mu \cdot \mu(\gamma^{-n}) +\Gamma_{\mu^w}\cdot \mu^w(\gamma^{-n}), \text{ where } \Gamma_\nu  =  \frac{1-p^2\cdot\nu(\gamma^{-2})}{1-\nu(\gamma^{-2})}\cdot
\frac{1-p^2\cdot\nu(\gamma^{-1})}{1-\nu(\gamma^{-1})}, \]
where the two factors in $\Gamma_\nu$  correspond respectively to the positive roots $\zeta$ and  $2\zeta$ of $G$ (see\S\ref{tree}). 
As $\mu(\gamma^{-1})=\mu^w(\gamma)=-p^{-1}$, one has  $\Gamma_\mu=0\ne \Gamma_{\mu^w}$, hence
the sequence $(\Phi^{K^\circ}_n)_{n\geqslant 0}$ is not $\rL^2$. This also shows, in passing, that $\pi_n$ is not square integrable.
\end{rem}


As a consequence  we obtain the following lower bound, in the unramified case.

\begin{cor}  \label{cor:deep-iwahori-inv}  For  $r\geqslant 1$ and for $\pi\in\Pi(\lambda_0)$, one has $\dim\left(\pi^{I_{2r}}\right)\geqslant r+1$.
\end{cor}
\begin{proof}
By Proposition~\ref{prop:inert-inv} for all $r\in\Z$, $\pi_2$ contains a (unique) line fixed by  $\gamma^r K'\gamma^{-r}$, having $\gamma^r\cdot f$ as basis, and moreover, the stabilizer in $G$ of that line is $\gamma^r K'\gamma^{-r}$.
We claim that the vectors  $f, \gamma\cdot f, \dots, \gamma^r\cdot f\in \pi_2$ are linearly independent. Indeed, if $f$ was a linear combination of the remaining $r$ vectors then it  would be fixed by $\cap_{1\leqslant j \leqslant r} \gamma^i K' \gamma^{-i}$ of $G$ which is {\it not} contained in $K'$.  As any of  these $(r+1)$ vectors is  fixed by $I_{2r}$,  the claim follows for $\pi_2$.
 
 Arguing the exact  same way,  using $K^\circ$ instead of $K'$,  proves the statement for $\pi_n$.
\end{proof}

While using the unramified $A$-packet $\Pi(\lambda_0)$ would be sufficient in our global applications when $D\equiv 3\pmod{8}$, the case of  discriminants $D\equiv 7\pmod{8}$ would  require the use of  certain tamely ramified $A$-packets $\Pi(\lambda)$ and providing 
explicit levels for them is the object of the next subsection. 


\subsection{Intertwining and an exponential sum} \label{intertwining}

Our arithmetic applications will require to show  existence of non-zero $I_r$-invariants, for some $r\in \Z_{\geqslant 1}$,  in certain ramified $A$-packets. This is delicate because of the lack of new-vector theory for non-generic representations (see Remark~\ref{new-vector}). Also Casselman's result  asserting that $\pi^{I_r}$ surjects onto $\pi_N^{T\cap I_r}$ is inconclusive here  as the latter vanishes, 
contrarily to \cite{dimitrov-ramakrishnan} where the open compact  is a pro-$p$-Iwahori subgroup of a sufficiently deep level
(see \S\ref{odds-ends} where these results are discussed). We will instead resort to explicit methods to prove in Proposition~\ref{inert-inv-ram} that $\pi_n$  has non-zero invariants  $K_T$, which contains a conjugate of $I_3$. It is relatively straightforward to determine all such vectors $f$ in the full induced representation but it becomes a thorny issue to find a non-square integrable 
matrix coefficient $\langle g \cdot f,  f^\vee \rangle$. By another result of Casselman, matrix coefficients can be expressed in terms of the corresponding ones in the Jacquet module, given here by an explicit  scalar product on $\C\cdot\mu \oplus \C\cdot\mu^w $. 
Making this actually work requires non-vanishing under the Jacquet functor which, once  verified,  leads directly to the result we seek. The computation of the Jacquet functor is first reduced to a precise statement about the intertwining operator at the level of finite reductive groups. It involves showing non-vanishing of some explicit exponential sums, bringing out the arithmetic nature of the problem. Although these sums seem extremely hard 
to be computed  individually, we manage to conclude by evaluating an average corresponding to the trace of  finite intertwining.  


\begin{thm}\label{inert-inv-ram} Assume  that $p$  odd  and  $E/\Q_p$  unramified. Let $\lambda$ be a character of $E^\times$ sending $p$ to $-1$ and whose restriction to $\cO^\times$ equals $\chi_E$, where
$\chi:\cO^1\twoheadrightarrow \F_{p^2}^1\to \C^\times$ is a (non-trivial) tamely ramified character.
Letting $\pi_n$ denote the non-tempered representation of the Arthur packet $\Pi(\lambda)$, one has $\dim\pi_n^{K_T}=1$ where $K_T=\left(\begin{smallmatrix}\cO^\times &p\cO &p\cO\\p\cO &\cO^\times &p\cO\\ p\cO & p\cO &\cO^\times\end{smallmatrix} \right)\cap G$.
\end{thm}


\begin{lemma} \label{KT-inv} One has
\[B\backslash G /K_T= \overline{B}\backslash \overline{G} /\overline{T}= \left\{\mathbf{1}, w,
\overline{[0,1]}=\left(\begin{smallmatrix} 1 & 0 & 0\\ 0  & 1 &0 \\ \xi   & 0 & 1 \end{smallmatrix}\right),
\sigma_y=\overline{[1,y]}=\left(\begin{smallmatrix} 1 & 0 & 0\\ -1  & 1 &0 \\ \xi y-\frac{1}{2}  & 1 & 1 \end{smallmatrix}\right)\Big{|} y\in \F_p\right\}. \]

Moreover, for any non-trivial $\chi$,  the $K_T$-invariants in $\Ind_B^G(\mu)$ are supported by the double cosets of $\{\sigma_y, y\in \F_p\}$ and,  in addition if $\chi$ is quadratic, by the double coset of $\overline{[0,1]}$.
\end{lemma}

\begin{proof} 
Mackey's Theorem and Frobenius Reciprocity yield that the dimension of $K_T$-invariants in $\Ind_B^G(\mu)$ equals the number of cosets
$[\sigma]\in B\backslash G /K_T$ such that $\mu$ has trivial restriction to $B\cap \sigma K_T \sigma^{-1}\supset Z$. As $\mu$ is ramified this is never the case for $\sigma=\mathbf{1}$, nor for $\sigma=w$, while $\sigma=\left(\begin{smallmatrix} 1 & 0 & 0\\ 0  & 1 &0 \\ \xi & 0 & 1 \end{smallmatrix}\right)$ works if and only if $\chi$ is quadratic.

 If $b=\left(\begin{smallmatrix} \bar\alpha & * & *\\ 0  & 1 &* \\ 0 & 0 & \alpha^{-1} \end{smallmatrix}\right)\in \sigma K_T \sigma^{-1}$, 
 for some $y\in \F_p$,   performing a matrix multiplication shows that
\[\sigma_y^{-1} b \sigma_y\equiv \begin{pmatrix} * & * & *\\ (\bar\alpha-1) & * & *\\  ?  & (\alpha^{-1}-1) & * \end{pmatrix}\mod{p}\in K_T,\]
hence  $\alpha \equiv 1 \pmod{p}$, ensuring the triviality of $\mu$ on $B\cap \sigma_y K_T \sigma_y^{-1}$. 

In summary, the dimension of $ \left(\Ind_B^G(\mu)\right)^{K_T}$ is $p+1$ for $\chi$  quadratic, and $p$ otherwise.
\end{proof}


\begin{rem} \label{new-vector}
 Recall that $\pi_n$ is the Langlands quotient of $\Ind_B^G(\mu)$  whose other  Jordan-H\"older constituent is $\pi_2$. 
As taking invariants by an open compact subgroup is an exact functor in the category of admissible representations,  Lemma~\ref{KT-inv} implies that    
\[\dim \pi_n^{K_T}+ \dim \pi_2^{K_T}=p (+1).\] 
To show that $\pi_n^{K_T}\ne \{0\}$ one could  try  computing $\dim \pi_2^{K_T}$ as $\pi_2$ is a generic discrete series. Unfortunately 
Miyauchi's theory \cite{miyauchi-U21, miyauchi-L-factor} of conductors  for $\rU(3)$-representations  with respect to the  paramodular groups
$K_r=\left(\begin{smallmatrix}\cO^\times & \cO &p^{-r}\cO\\p^r\cO &\cO^\times &\cO\\ p^r\cO & p^r\cO &\cO^\times\end{smallmatrix} \right)\cap G$ predicts that the ramified $\pi_n$ and $\pi_c$  have no level, {\it i.e.} they have no invariants by $K_r$ for any $r$, while the level  of $\pi_2$ is given by its conductor. When $\chi$ is the quadratic character,  the  $L$-parameter $(\lambda \otimes \mathrm{St})  \oplus \mathbf{1}$ has conductor $2$, therefore the generic member  $\pi_2$ in this $L$-packet has one dimensional invariants by $K_2$, hence also by  $\gamma^{-1}K_2\gamma= \left(\begin{smallmatrix}\cO^\times & p\cO &\cO\\p\cO &\cO^\times &p\cO\\ \cO & p\cO &\cO^\times\end{smallmatrix} \right)\cap G\supset K_T$. For other ramified, $\chi$'s 
$\pi_2$ has  an invariant line by $K_3$ which  has same volume as $K_T$ but is not conjugated to it, thus non-settling the non-vanishing of $\pi_2^{K_T}$ let alone computing its dimension. 
\end{rem}




Recall  the  Jacquet functor  given by 
\[\Ind_B^G(\mu) \longrightarrow  \C\cdot \mu\oplus  \C\cdot\mu^w, \quad  f\mapsto (f(1),(Mf)(1))\]
where the standard intertwining operator  $M:\Ind_B^G(\mu) \to \Ind_B^G(\mu^w)$ is defined via analytic continuation, as follows. 
For $s\in \C$, letting $\mu_s=\mu\cdot \delta^{s/2}$, the  intertwining operator 
\[M_s:\Ind_B^G(\mu_s) \to \Ind_B^G(\mu_s^w), \quad f\mapsto \int_N f(wn\cdot)dn\]
is absolutely convergent for  $\Re(s)\gg 0$ and  $G$-equivariant. 
 Moreover, for  any section $f_s\in \Ind_B^G(\mu_s)$ such that for all $g\in G$ the function $f_s(g)$ is analytic in $s\in \C$, the function $(M_s(f_s))(g)$, a priori only defined for $\Re(s)\gg 0$, is a rational function in $p^{-s}$, hence extends to a meromorphic function on all of $\C$ with only possibly a finite number of poles independent of $f$ and of $g$. In fact, it continues as an intertwining operator, {\it i.e.} 
\[  (M_s(f_s))(gg')= (M_s(f_s(\cdot g'))(g), \text{ for all } g,g'\in G. \]


We refer to \cite[\S 1]{arthur-intertwining} for more detail and proofs, and we will only use that 
\[M(f_0)=\lim_{s\to 0}  M_s(f_{s})\]
computing explicitly the right hand side as a rational function in $p^{-s}$, simultaneously justifying that $M_s$ does not have a pole at $s=0$. 



The Lemma \ref{KT-inv} implies the existence of a non-zero element  $f_y\in\left(\Ind_B^G(\mu)\right)^{K_T}$ supported on  $B\sigma_y K_T$, which we normalize by $f_y(\sigma_y)=1$. Consider a $K^\circ$-flat section  $f_{y,s}$ passing thru $f_{y,0}=f_{y}$, and 
computing $M_s(f_{y,s})$ for $\Re(s)\gg 0$. 



\begin{lemma} \label{Ms-fs} 
For all $y,y'\in \F_p$, we have: 
\[M_s(f_{y',s})(\sigma_y)=\chi(-1) p^{-1} (1-p^{-1})(1-\mu_s(\gamma))^{-1}.\]
\end{lemma}


Thus, to complete the proof, it suffices to show that $\pi_n^{I_2}\ne \{0\}$ or equivalently find $y\in \F_p^\times$ such that  $f_y\notin \pi_2$.












{\it Although we only solve this question in the case of a unitary group in $3$ variables, we feel that it deserves to be studied in greater generality for its own sake. }


\section{Galois representations for  $3$-folds of Picard type}\label{MT-conj}

From this point onwards, we will use  global notations. The  local results of the previous section can be applied to the completion  $E$ of $M$ at any prime number which does not split in that field.

We denote by $\A_f$  the ring of finite adeles of $\Q$, so that $\A=\A_f\times \R$.



\subsection{Abelian $3$-folds of Picard type and Tate modules} \label{moduli}
Let $k$ be any field containing $M$.
 Consider  an abelian $3$-fold  $A/k$   together with an injection
 $\iota^0: M \hookrightarrow \mathrm{End}^0(A/k)=\mathrm{End}(A/k)\otimes\Q$, or equivalently with an injection $\iota$
 of an  order of $M$  into $\mathrm{End}(A/k)$, the most important for us case being  when  $\iota^0$  comes from $\iota: \cO_M\hookrightarrow
\mathrm{End}(A/k)$.

The action of $M$ splits the $3$-dimensional $k$-vector space $\Lie(A/k)$  in a direct sum of two sub-spaces:  one
on which the actions of $M$ and $k$ agree, and  one on which they differ by the complex conjugation.
We say that  $A$  is of truly of Picard type if the pair of dimensions of these spaces, called the signature, equals $(2,1)$.

A polarization on $A$ is an isogeny $\theta: A\to A^\vee$, where $A^\vee$ denotes the dual abelian variety.
By positivity, since $k$ is a field, the Rosati involution induced by $\theta$ on $\iota(\cO)$ is given by  the complex conjugation (see \cite[\S 21]{mumford}). A polarization is called  principal, if it is an isomorphism, and can  can always be acquired over a finite extension of  $k$.

To define a level structure on $A$ we need to  consider its Tate module.
Given a place $v$ of $k$,  the $v$-adic Tate module $T_v A=\varprojlim\limits_r A[v^r]$ of $A$  is free of rank $3$  over $\cO_v$.
Denote $V_v A=M_v \otimes_{\cO_v}T_v A $.
One also considers the adelic Tate module
 \[V_f A=\Q \otimes_{\Z}  \varprojlim\limits_{n}A[n],\]
 which is free of rank $3$ over  $\A_{M,f}$.
Given a polarization $\theta: A\to A^\vee$, the Weil pairing endows  $V_f A$ with a non-degenerate skew-hermitian form,
{\it i.e.}, a non-degenerate alternating pairing
\[\langle \cdot, \cdot \rangle_A:V_f A\times V_f A\to \A_f\]
such that $\langle a\cdot v, v'  \rangle_A=\langle  v, \bar a\cdot v'  \rangle_A$ for all $a\in M$.
If $\theta$ is principlal, then   $\langle \cdot, \cdot \rangle_A$ is a perfect pairing.




\subsection{Shimura surfaces and families of abelian threefolds of Picard type} \label{shimura-surfaces}
Let $(V,\langle \cdot, \cdot \rangle)$ be a $3$-dimensional (non-degenerate) hermitian space over $M$.
The corresponding unitary similitude group $\wG=\mathrm{GU}(V)$ is a reductive group over $\Q$ such that for any $\Q$-algebra $R$ one has:
\[\wG(R)= \{ g\in \GL(V\otimes_\Q R) \mid \forall v,v'\in V\otimes_\Q  R,  \langle g(v), g(v') \rangle=\nu(g) \langle v, v' \rangle\},\]
where $\nu: \mathrm{GU}(V)\to \G_{m,\Q}$ is a homomorphism whose kernel is the  unitary group $ G= \mathrm{U}(V)$.

Note that any hermitian form in $3$ variables over a non-archimedean local field is isotropic, hence $\wG (\Q_p)$ is unique up to isomorphism, while at infinity $\langle \cdot, \cdot \rangle$ is uniquely determined by its signature,  hence there are only two possibilities for $\wG(\R)$ (as  opposite signatures define isomorphic groups).  Hasse's Principle applied to the semi-simple simply connected derived group $\wG^1=\mathrm{SU}(V)$ implies then   that, up to an isomorphism, there exists a unique quasi-split unitary  group, denoted $\mathrm{GU}_{2,1}$, and a unique definite  unitary  group, denoted $\mathrm{GU}_{3,0}$.

We will now  define the Shimura variety for the unitary similitude group $\wG=\mathrm{GU}_{2,1}$
 represented by the matrix $\left(\begin{smallmatrix} & & 1 \\ & \sqrt{-D} & \\ -1 & & \end{smallmatrix} \right)$.
 The one for  $\mathrm{GU}_{1,2}$ is  its complex conjugate.
The   homomorphism of  $\R$-algebraic   groups  
\[\tilde h: \mathrm{Res}^\C_\R \G_{m,\R} \to \wG_\R \text{ , } z\mapsto \begin{pmatrix} \Re(z) & 0 &  \Im(z) \\
0 & z & 0 \\ -\Im(z)  & 0 &  \Re(z) \end{pmatrix}\]
satisfy the Shimura datum axioms for $\wG$, hence for any open compact subgroups  $\wK$  of  $\wG(\A_f)$ one can consider the Shimura surface
\[ Y_{\wK}(\C) =\wG(\Q)\backslash \left(\mathcal{H} \times G(\A_f)/\wK\right),\]
where  $\mathcal{H}\simeq \wG(\R)/\wK_\infty$ is identified  with the $\wG(\R)$-conjugacy classes of $\tilde h$.
By a  fundamental result of  Shimura $Y_{\wK}$ admits  a canonical model over the reflex field $M$. As we will see, the connected components of  $Y_{\wK}$ are Picard modular surfaces, justifying the terminology.


For $\wG$ anisotropic, can analogously define Shimura  sets which are finite and therefore will not alter the uniformity of our results in  \S\ref{s:uniform}.


The Shimura surfaces of Picard type are coarse moduli spaces of abelian $3$-folds of Picard type. Namely,
$Y_{\wK}(\C)$ is in bijection with isogeny classes of $(A,\iota^0,\theta, \eta \circ\wK )$, where
$(A,\iota^0, \theta)$ is a polarized  abelian variety  of Picard type over $\C$, and $\eta: \A_f\otimes_{\Q}V \xrightarrow{\sim} V_f A  $
is an isomorphism sending $\langle \cdot, \cdot \rangle_A$ to a $\A_f^\times$-multiple of  $\langle \cdot, \cdot \rangle_V$.
Note that the usual $\Q^\times$-multiple condition is automatically satisfied as we are in the type C case  ({\bf provide reference}).
When $\wK^\circ$  is the standard maximal open compact subgroup of $G(\A_f)$,  the points of  $Y_{\wK^\circ}(k)$ correspond to  isomorphism classes of principally polarized abelian $3$-folds over $k$ having multiplication by $\cO_M$.
 
 
Henceforth,  we will only consider  abelian $3$-folds which are principally polarized and admit multiplication by $\cO_M$, and we will refer to them simply as being {\it of Picard type}. 
 

It must be noted that, even though each point of $Y_{\wK}(\C)$ is associated to an abelian $3$-fold of Picard type,  there does not exist such a family over the entire $Y_{\wK}(\C)$ unless there is no point with extra automorphisms, in which case $Y_{\wK}(\C)$ would be a fine moduli space. In our cases of interest  $\wK$ is not neat, and therefore  $[S_{\wK}]$ is not a fine moduli space.

 However, given an open compact subgroup $\wK$, we claim that there is an abelian  family of Picard type $A$ over
 any open subset $U$ of  $Y_{\wK}$ which contains no point with a non-trivial stabilizer. 
 By   \cite[\S 2.3.4]{kwlan-book},  the moduli stack $S_{\wK}$   associated to this problem is an algebraic stack (for the \'etale topology), locally of finite type over the base which we may take to $\Spec(M)$. Moreover, by 
  \cite[\S A.7.5]{kwlan-book}, there is a canonical surjective morphism $\phi$ from $S_{\wK}$ to the associated coarse moduli space $[S_{\wK}]$, 
  which in our notations is  $Y_{\wK}$. 
  By  \cite[\S 7]{kwlan-book}, $[S_{\wK}]$    is an algebraic space and even a quasi-projective scheme. Moreover, by a general property of moduli stacks (see \cite[Chap.~7]{olson-book}), $\phi$ is an isomorphism over the locus $U$ where there is {\it no non-trivial automorphism}, by which we mean it has no infinitesimal automorphism; analytically, this corresponds to  points of $Y_{\wK}(\C)$ having no non-trivial stabilizers. 
Now $U$ is a priori an open subscheme of $[S_{\wK}]$, but since it is where $\phi$ is an isomorphism, we get a canonical open  $j: U \to S_{\wK}$. This  map  tautologically yields the desired family $f: A\to U$ of abelian varieties of Picard type,  
whose existence is essential for our proof of the main results. 



In analogy with   Gross' index $2$ subgroups  of maximal compacts  $K^\circ_p$ at ramifies primes introduced in \S\ref{K-second}, we consider the open compact subgroup
\begin{equation}\label{K-double-prime}
\wK''= \wK''_D \prod_{p\nmid D} \wK_p^\circ \subset \wG(\A_f),
\end{equation}
where $\wK''_D$ is  defined as the kernel of the composed homomorphism
\begin{equation}\label{KD-double-prime}
 \prod_{p\mid D}  \wK_p^\circ  \twoheadrightarrow \prod_{p\mid D}  \wK_p^\circ/\wK''_p= \prod_{p\mid D} \{\pm 1\}
 \xrightarrow{\Pi} \{\pm 1\}.
 \end{equation}




Let $(A,\iota,\theta)$ be a principally polarized  abelian $3$-fold of Picard type over $k$.
For $v$ the prime of $M$ above  $p\mid D$, the action of the absolute Galois group $\Gal_k$  on $A[v]$
 factors through $\mathrm{GO}(3,\F_p)$.
 Using the exceptional isomorphism
 $\mathrm{PGO}(3,\F_p)\xrightarrow{\sim} \mathrm{SO}(3,\F_p)\xrightarrow{\sim}  \PGL_2(\F_p)$, one  defines its projectivization
\begin{equation}\label{eq:star}
\widetilde\rho_{A,p}: \Gal_k \to \PGL_2(\F_p).
\end{equation}

Taking quotient by the unique index two subgroup $\PSL_2(\F_p)$ of $\PGL_2(\F_p)$ yields a canonical homomorphism
$\varepsilon_{A,p}:  \Gal_k \to\{ \pm 1 \}$ and we let $\varepsilon_{A,D}=\prod_{p \mid  D} \varepsilon_{A,p}:  \Gal_k \to\{ \pm 1 \}$.

Note that, for any open compact subgroup $\tilde K$, $A$ defines a $k$-rational point in $Y_{\tilde K}$ if, and only if,  the Galois representation on the adelic Tate module has image in $\tilde K$.
Hence a point in  $Y_{\wK''}(k)$ corresponds precisely to an abelian $3$-fold $A$ over $k$ of Picard type having trivial $\varepsilon_{A,D}$.


\subsection{\'Etale fundamental groups and Mumford--Tate groups}\label{MT-groups}

Let $k$ be a number field  containing $M$ over which the connected component of $Y_{\wK}$ are defined, and fix a
connected component $Y$ of $Y_{\wK}\times_M k$ and a smooth open  $U$ of $Y$ endowed with an abelian scheme $f:A\to U$ of Picard type. Denote by  $\eta$ the generic point of the smooth surface $U$.
Fixing a closed geometric point $\bar x$ of $U$ the \'etale fundamental group  sits in the middle of a short exact sequence
\begin{align}\label{fund-group-es}
1\to \Pi_1(U_{\bar k}, \bar x)\to \Pi_1(U, \bar x) \to \Gal_k= \Pi_1(\{x\}, \bar x)\to 1.
\end{align}
The morphism  $f:A\to U$  being proper and smooth, one can consider the  \'etale sheaf $\mathrm{R}^1f_*\Z_\ell$ on $U$.
As $U$ is geometrically connected we have $\Pi_1(U, \bar x)\simeq  \Pi_1(U, \bar \eta)$ and the latter acts on
\[(\mathrm{R}^1f_*\Z_\ell)_{\bar \eta}= \rH^1(A_{\bar \eta}, \Z_\ell)=(T_\ell A_{\eta})^\vee,\]
yielding a continuous representation
\[\Gal(\bar \eta/\eta)\twoheadrightarrow \Pi_1(U, \bar x)\xrightarrow{\rho_{U,\ell}} \Aut_{\Z_\ell}(T_\ell A_{\eta}).\]
Any closed point $x\in U(k)$ yields a section $s_x:\Gal_k\to  \Pi_1(U, \bar x)$ of \eqref{fund-group-es} allowing one to consider
\[\rho_{x,\ell}=\rho_{U,\ell}\circ s_x: \Gal_k \to \Aut_{\Z_\ell}(T_\ell A_{\eta}).\]
Finally for any closed curve $C\subset U$ defined over $k$, there is a natural map   $\Pi_1(C, \bar x)\to  \Pi_1(U, \bar x)$
whose composition with  $\rho_{U,\ell}$ is denoted $\rho_{C,\ell}$. As  $f:A\to U$ is of Picard type, for any $x\in C(k)$
\[\Gamma_x=\mathrm{im}(\rho_{x,\ell})\subset\Gamma_C=\mathrm{im}(\rho_{C,\ell})\subset \Gamma_U=\mathrm{im}(\rho_{U,\ell})\subset K^\circ_\ell.\]

By a series of results of Cadoret--Tamagawa
(see \cite{cadoret-tamagawa-inv, cadoret-tamagawa-duke1}), the set $C_\rho$ of all   $x\in C(k)$ for which $\Gamma_x$ is not open in $ \Gamma_C$ is finite and for all
 $x\in C(k)\setminus C_\rho$ the index $[\Gamma_C: \Gamma_x]$ is uniformly bounded.



The Mumford--Tate group $\MT(A)$ of a polarized abelian variety $A$ over $\C$  is 
the smallest connected reductive subgroup of $\GL(\rH_1(A, \Q))$ over $\Q$, 
whose $\R$-points contain the associated $\R$-morphism $h: \C^\ast \to \GL(\rH_1(A(\C), \R))$ coming from the Hodge decomposition. 
If we assume further that $A$ is defined over a number field  $k\subset \C$  finitely generated  over $\Q$, then the image $\Gamma_\ell$ of $\Gal_k$ acting on $(T_\ell A)$ is an $\ell$-adic Lie group. By a theorem of Deligne  \cite[Chap. I.2]{deligne-milne-ogus-shih}, we have
${\rm Lie}(\Gamma_\ell)_{\Q_\ell} \subset \Lie(\MT(A)\otimes_{\Z_\ell}\Q_\ell)$ and the Mumford--Tate conjecture, known for
abelian varieties of dimension at most $3$,  asserts that they are equal (see {\it e.g.} \cite{chi}).

As the  Mumford--Tate group of the (generic point of the) universal family $f:A\to U$  is given by $\wG=\mathrm{GU}_{2,1}$, it follows
from the above discussion that   the Mumford--Tate group of any abelian $3$-fold  of Picard type is a reductive subgroup of   $\wG$.
We have the following trichotomy. 


\begin{lemma}\label{prop:tricotomy}
Let $\fg$ be a reductive Lie subalgebra of $\mathfrak{gu}(3,k)$ defined over a characteristic $0$ field~$k$. 
If $\fg'\subset \mathfrak{su}(3,k)$ denotes the semi-simple   part   of $\fg$,   exactly one of the following holds:
\begin{enumerate}
\item $\fg'=\{0\}$, {\it i.e.} $\fg$ is abelian, 
\item  $\fg'$ is a form of $\mathfrak{sl}(2,k)$, 
\item  $\fg'=\mathfrak{su}(3,k)$. 
\end{enumerate}
\end{lemma}

\begin{proof}
If $\fg'_{\C}=\{0\}$, then $\fg'=\{0\}$, whereas if $\fg'_{\C}=\mathfrak{sl}(3,\C)$,  then $\fg'=\mathfrak{su}(3,k)$ for dimension reasons. 
In the remaining cases, using the well known fact that any proper non-zero semi-simple Lie subalgebra of $\mathfrak{sl}(3,\C)$ is isomorphic to $\mathfrak{sl}(2,\C)$, we deduce that $\fg'$ is a form of $\mathfrak{sl}(2,k)$.
\end{proof}



If $A$ is (potentially) of  CM type (resp. admits (potentially) a non-trivial  CM quotient), then its Mumford--Tate group is of the first (resp. second) type. It is natural to ask whether Theorem~\ref{theoremB} could be further refined for abelian $3$-folds  without non-trivial CM quotients. 





\subsection{Galois stable lattices and rationality}\label{lattices-rationality}
Suppose henceforth that $k$ is a number field.  Let  $(A,\iota, \theta)$ be a polarized  abelian $3$-fold of Picard type over $k$,
and let  $\eta:\A_f\otimes_{\Q}V  \xrightarrow{\sim} V_f A$  be  an isomorphism
sending $\langle \cdot, \cdot \rangle_A$ to a $\A_f^\times$-multiple of  $\langle \cdot, \cdot \rangle_V$.
 As  $\MT(A,\iota, \theta)\subset \wG$ by  Deligne \cite[Cor.~6.2]{deligne-milne-ogus-shih}, the action of
$\Gal_k$ on the adelic Tate module $V_f A$, together with the choice of $\eta$,  yields  a continuous homomorphism:
\[ \rho_{A,f}: \Gal_k\longrightarrow \wG(\A_f). \]
Moreover, the point $(A,\iota,\theta, \eta \circ \wK )$ on
$Y_{\wK}(\C)$ is defined over $k$ if, and only if, $\rho_{A,f}(\Gal_k)\subset \wK$.


Given any  prime number $\ell$, the resulting continuous homomorphism:
 \[ \rho_{\ell}=\rho_{A,\ell}: \Gal_k\longrightarrow \wG(\Q_\ell),\]
has compact image, which is thus necessarily contained in some  maximal compact $\wK_\ell$ of $\wG(\Q_\ell)$.

We will denote by $\overline\rho_{\ell}$ the composition of $\rho_{\ell}$ with the natural  surjection of  $\wK_\ell$ onto its reductive quotient $\overline{G}_\ell$. Then $\overline\rho_{\ell}$ acts on $\cL_\ell\otimes_{\Z_\ell} \F_\ell$, where
$\cL_\ell$ is a  $\cO\otimes \Z_\ell$-lattice whose stabilizer in $\wG(\Q_\ell)$ is $\wK_\ell$.
While in general $\overline\rho_{\ell}$ depends on the choice of  $\wK_\ell$, or equivalently on the choice of a
$\Gal_k$-stable lattice $\cL_\ell$, it follows from a  Theorem of Brauer and Nesbitt that its
 semi-simplification is independent of these choices.

 If $\ell$ splits in $M$  then $G(\Q_\ell)=\GL_3(\Q_\ell)$ and
 $\overline\rho_{\ell}: \Gal_k\longrightarrow \widetilde \GL_3(\F_\ell)$ is absolutely irreducible if, and only if, there exists a
 unique, up to homothecy,  $\rho_{\ell}(\Gal_k)$-stable $\Z_\ell$-lattice $\cL_\ell$.

 If $\ell$  does not split in $M$, then $G(\Q_\ell)$ has rank $1$ and  every  edge of the corresponding Bruhat-Tits tree 
 links a  vertex with reductive quotient $\overline{G}{}^\circ_\ell$ to a vertex with reductive quotient  $\overline{G}{}'_\ell$ (see \S\ref{tree}).
 As $\rho_{\ell}(\Gal_k)$ acts on the tree by isometries, if it fixes any two (or more) vertices, then it necessarily fixes an edge, hence
 its image in the reductive quotient of any fixed vertex would be reducible. Conversely, since
 no irreducible subgroup of $\overline{G}{}^\circ_\ell$ or of $\overline{G}{}'_\ell$ does fix an adjacent vertices, one can
characterize the representations  $\rho_{\ell}$ fixing a unique vertex as follows.


Note that $\theta$ yields an integral pairing on $\rH_1(A(\C), \Z)$ inducing a pairing
on $T_\ell A\simeq \rH_1(A(\C), \Z_\ell)$ for each $\ell$. If $\ell$ does not divide the degree of $\theta$, then $T_\ell A$ is self-dual, and one can chose  $\wK_\ell\simeq \wK_\ell^\circ$.

\begin{lemma}  Let  $(A,\iota, \theta)$ be a polarized  abelian $3$-fold of Picard type over a number field $k$.
\begin{enumerate}
\item   $\overline\rho_{\ell}$ is absolutely irreducible if, and only if, the exists a unique, up to homothecy, $\rho_\ell(\Gal_k)$-stable lattice.
The latter is necessarily self-dual and  $\wK_\ell$ is conjugated to $\wK_\ell^\circ$.
\item  Suppose $\ell$ be a prime that does not split in $M$. 
Then $\overline\rho_{\ell}(\Gal_k)$ is an irreducible subgroup of $\overline{G}{}'_\ell$ if, and only if,
the exists a unique, up to homothecy, pair of  $\rho_\ell(\Gal_k)$-stable lattices. The latter are almost self-dual and  $\wK_\ell$ is conjugated to $\wK'_\ell$.
\item If $\theta$ is principal, then one can chose  $\wK\simeq \wK^\circ$, {\it i.e.}, $(A,\iota, \theta)$ defines a $k$-rational point
on $Y_{\wK^\circ}$.
\end{enumerate}
\end{lemma}


\subsection{Lie images of Galois representations}

The Lie algebra $\fh\subset \mathfrak{gu}(3,\Z_\ell)$ of the image of $\rho_{A,\ell}$ is algebraic and
$\fh_{\Q_\ell}=\Q_\ell\otimes_{\Z_\ell}\mathfrak{g}_{\Z_\ell}$  is reductive by Faltings \cite[Thm.~3]{faltings-AG}.

Serre \cite[C.3.7]{serre-kyoto} has defined an integral model of the Mumford--Tate group  and  refined Deligne's theorem to show that
its Lie algebra $\fh_{\Z_\ell}$ contains  $\mathfrak{g}_{\Z_\ell}$ as a subgroup. The integral Mumford--Tate conjecture, known for abelian varieties of dimension $\leqslant 3$, asserts that the image is an open subgroup. 

\begin{thm}\label{prop:MT}
 Let $A$ be an abelian $3$-fold of Picard type defined over a field $k$.   Then $\rho_{A,\ell}$ is potentially reducible if, and only if, $A$ has a non-trivial CM quotient.
\end{thm}

\begin{proof} Suppose that  $\mathfrak{g}_{\Q_\ell}$ is a proper subalgebra of $\mathfrak{gu}(3,\Q_\ell)$. It is, by Proposition 2.1, either abelian or contains $\mathfrak{su}(3,\Q_\ell)$ or its semisimple part is a form of $\mathfrak{sl}(2,\Q_\ell)$. In the second case, as it contains homotheties by Bogomolov \cite{bogomolov-SSSR}, it must contain $\mathfrak{su}(3,\Q_\ell) \oplus \Q_\ell$.
In each of the remaining cases, after extending scalars to $\overline\Q_\ell$,  $\rho_{A,\ell}$ is potentially reducible, {\it i.e.}, after  possibly replacing $k$ by a finite extension, $\rho_{A,\ell}$ contains (as a direct factor by Faltings)
a character  $\chi_\ell: k^\times\backslash \A_k^\times \to \overline\Q_\ell^\times$. As a sub-representation of  $\rho_{A,\ell}$,
$\chi_\ell$ is unramified outside a finite set of places and its restriction decomposition groups  at places above $\ell$
 it  Hodge-Tate with weights belonging to  $\{0,-1\}$.  In addition it is pure of weight $-1$.
By Minkowski's proof of the Dirichlet unit theorem,  $\chi_\ell$ corresponds to an algebraic Hecke
character  $\chi: k^\times\backslash \A_k^\times \to \C^\times$ whose infinity component is necessarily of the form
$\rN_{\Phi'}\circ \rN_{k/L'}$ where $\rN_{\Phi'}$ is the partial norm given by a CM type $\Phi'$ for a CM field
$L'\subset k$. By \cite[Lem.~2]{shimura-CM} replacing $(L',\Phi')$ by its double reflex yields the same infinite component, hence we may and do assume that $(L',\Phi')$ is a primitive, {\it i.e.}, coincides with  the reflex of a CM field $L$
endowed with a CM type $\Phi$. Further replacing $k$ by a finite (abelian) extension one can assume that
$\chi_f$ takes values in  $L^\times$.
 By  Casselman (see \cite[Thm.~6]{shimura-CM}), there exists an  abelian variety $B$
  defined over $k\supset L'$  which is CM of type $(L,\Phi)$ and such that  $\rho_{B,\ell}=\chi_\ell$, hence
\[\Hom_{\Gal_k}(\rho_{A,\ell}, \rho_{B,\ell})\neq \{0\}.\]
By Faltings one deduces that $\Hom_{k}(A,B)\ne  \{0\}$, hence $A$ contains a non-trivial   CM quotient $A'$.  One can assume that  $A'\ne A$, hence there exists an abelian variety $A''$ which is not of CM type and such that
 $A$ is isogenous to $A'\times A''$, {\it i.e.}, $V_{A,\ell}=V_{A',\ell}\oplus V_{A'',\ell}$. Furthermore since $\Hom_{k}(A',A'')=\{0\}$, one can show that the  isogeny is $\cO$-linear. Hence $A''$ admits multiplication by $\cO$ and since it is not of CM type, it has dimension $2$,   from which one  deduces that
  $\fh_{\Z_\ell}=\mathfrak{g}(\mathfrak{u}(2)\times \mathfrak{u}(1),\Z_\ell)$.

 Then either $\rho_{A'',\ell}$ is Lie surjective, in which case $\mathfrak{g}_{\Z_\ell}=\fh_{\Z_\ell}$, or else $\rho_{A'',\ell}$ is reducible  which by repeating the above argument would contradict $A''$ not being of CM type. \end{proof}


\begin{lemma}
Let $\wK$ be an  open compact subgroup of $\wG(\A_f)$ and  let $x\in Y_{\wK}(k)$ be such that $\MT(x)\neq \wG$.
Then $x$ belongs to a special subvariety defined over $k$.
\end{lemma}

\begin{proof} By the classification in \S\ref{MT-groups}, $\widetilde{H}=\MT(x)$  is either  isomorphic to a  form of $\G(\mathrm{U}(2)\times \mathrm{U}(1))$ or a torus. Moreover by  a theorem of Deligne \cite{deligne-milne-ogus-shih},  $x$ belongs to the image of the canonical morphism of Shimura varieties $Y^{\widetilde{H}}_{\widetilde{H}\cap \wK}\to Y_{\wK}$. It remains to see that $Y^{\widetilde{H}}_{\widetilde{H}\cap K}$ is defined over $k$.
 This is clear if $\widetilde{H}$ is a torus, as then $Y^{\widetilde{H}}_{\widetilde{H}\cap K}$ is a finite set of points which are Galois conjugates.
In the remaining case  $Y^{\widetilde{H}}_{\widetilde{H}\cap K}$  is a Shimura curve and for any $\sigma\in \Gal_k$ a theorem of Kazhdan ensures that
$\sigma(Y^{\widetilde{H}}_{\widetilde{H}\cap K})$ is also a Shimura subvariety of $Y_K$ containing $x$. If
$\sigma(Y^{\widetilde{H}}_{\widetilde{H}\cap K})\neq Y^{\widetilde{H}}_{\widetilde{H}\cap K}$ then $x$ would belong a  smaller Shimura subvariety, namely
$\sigma(Y^{\widetilde{H}}_{\widetilde{H}\cap K})\cap  Y^{\widetilde{H}}_{\widetilde{H}\cap K}$, contradicting the fact that $\widetilde{H}=\MT(A)$.
\end{proof}





\section{Irregularity for Picard modular surfaces}\label{irregularity}



We denote by $q(X)$ the irregularity of a projective  algebraic surface
$X$ over $\C$,  given by the dimension of $\rH^1(X, \cO_{X})$.
If $X$ is smooth and projective then $q(X)=\dim \rH^0(X, \Omega^1_{X})$.



\subsection{A  lemma on surfaces with isolated singularities}

Let $X$ be a projective irreducible algebraic surface over $\C$ with isolated singularities, {\it i.e.}, such that there exists a
smooth  open $j:U\hookrightarrow X$ whose complement   $Z=X\backslash U$ consists of finitely many closed points.
There exists a smooth resolution
\[\phi: \widetilde{X}  \to X\]
such that $\phi^{-1}(Z)$ is a divisor with normal crossings with $\phi$ restricting to   an isomorphism from
 $\phi^{-1}(U)$ onto $U$.   Thus we get an injection
$\widetilde j: U  \hookrightarrow  \widetilde{X} $ such that $j= \phi \circ\widetilde j $, and we denote by
\[\widetilde j^\ast: \rH^1(\widetilde{X}, \Q)  \to  \rH^1(U, \Q)\]
the pullback homomorphism on cohomology. By \cite[Thm.3.2.5(iii)]{deligne-hodgeII} we know that  $\widetilde j^\ast$ is a homomorphism of
mixed Hodge structure, with $ \rH^1(\widetilde{X}, \Q)$ being   pure of weight $1$.


\begin{lemma} \label{top} The map $\widetilde j^\ast$ is an isomorphism, in particular  $\rH^1(U, \Q)$ is a pure weight $1$ Hodge structure
and $q(X)=\dim \rH^0(U, \Omega^1_U)$.
\end{lemma}

\begin{proof}
Let $\mathrm{IH}^\bullet(X,\Q)$ denote the middle intersection cohomology of $X$. Since
\[\dim(X)-1>0=\dim(Z)\]
by  \cite[Thm.5.4.12]{dimca}
$j^\ast: \mathrm{IH}^1( X, \Q)  \to  \mathrm{IH}^1(U, \Q)$ is an isomorphism, while $\widetilde j^\ast$
is injective. Moreover by Cor.5.4.11 and Prop.5.4.4 in {\it loc.cit.}
$\phi^\ast: \mathrm{IH}^1(X, \Q)  \to  \mathrm{IH}^1(\widetilde{X}, \Q)=\rH^1(\widetilde{X}, \Q)$  is an embedding, while $\mathrm{IH}^1( U, \Q)=\rH^1( U, \Q)$.
This is summarized in the following commutative diagram:
\[\xymatrix@C=40pt{
\mathrm{IH}^1( X, \Q) \ar@{^{(}->}_{\phi^\ast}[d]   \ar_{\sim}^{j^\ast}[r] &\mathrm{IH}^1( U, \Q) \ar@{=}[d] \\
 \rH^1( \widetilde{X}, \Q) \ar@{^{(}->}^{\widetilde j^\ast}[r]   & \rH^1( U, \Q) }\]
It immediately follows that $\widetilde j^\ast$ is an isomorphism and $q(\widetilde{X})=
\dim \rH^0(\widetilde{X}, \Omega^1_{\widetilde{X}})= \dim \rH^0(U, \Omega^1_U)$. Finally $q(X)=q(\widetilde{X})$ as the
irregularity is a birational invariant. \end{proof}




\subsection{A formula for the irregularity}
Let  $z\mapsto \bar z$ be the non-trivial automorphism of $M/\Q$.
Put  $M^1=\{z\in M^\times\mid z\bar z=1\}$, which we will view as an algebraic torus over $\Q$ and denote by  $\A_M^1$ its adelic points.

For $\wK$ an open compact subgroup  of $\wG(\A_f)$  we recall  the Shimura variety of Picard type  defined as  the adelic quotient
\begin{equation}
Y_{\wK} = \wG(\Q)\backslash \wG(\A)/\wK \wK_\infty.
\end{equation}


Let $G^1=\mathrm{SU}(V)$ be the derived group of $\wG$.  As $G^1$ is simply connected and $G^1_\infty$ is not compact, the  Strong Approximation Theorem (see~\cite[Thm.~7.12]{platonov-rapinchuk}) implies that
$G^1(\Q)$ is dense in $G^1(\A_f)$. It follows that the
determinant map defines an isomorphism between the group of connected components  $\pi_0(Y_{\wK})$  and the  idele class group
$\A_M^\times/M^\times \det(\wK)M_\infty^\times$. Further,  Shimura's theory of canonical models implies that the connected components of
$Y_{\wK}$ are all Galois conjugates, hence share the same irregularity, and the same is true for the Shimura variety
$Y_K= G(\Q)\backslash G(\A)/K K_\infty $ for $G$, where $K=\wK\cap G(\A_f)$. Letting
\begin{equation}
\Gamma= \wG(\Q)\cap  \wK \wG(\R),
\end{equation}
 it follows from $\nu(\Gamma)\subset \Q^\times \cap  \widehat\Z^\times \widehat\R_+^\times=\{ 1\}$  that both
$Y_{\wK}$ and $Y_K$ share the same connected component of identity given by $Y_\Gamma=\Gamma\backslash \mathcal{H}$
(see \cite[(8)]{dimitrov-ramakrishnan}). One should be careful to observe that the natural dominant map $Y_{K^1}\to Y_\Gamma$, where $Y_{K^1}$  is
the  Shimura variety of level $K^1=K\cap G^1(\A_f)$ for  $G^1$ is an isomorphism precisely when, either $\det(\Gamma)=\{1\}$, or $-1\in \Gamma$.



\begin{prop} The irregularity of any connected component of the minimal compactification $Y_{\wK}^\ast$ of $Y_{\wK}$ is given by the formula
\begin{equation}\label{q-formula}
q(Y_\Gamma^\ast)= \sum_{(\lambda,\nu)\in \Xi/\widehat{\pi_0(Y_K)}}
\sum_{ \pi_f \in \Pi_f(\lambda,\nu)} \dim(\pi_f^K) \frac{1+W(\lambda\nu_M)(-1)^{s(\pi_f)}}{2}, \text{ where }
 \end{equation}
\begin{itemize}
\item  $\Xi$  is the set of  pairs $(\lambda,\nu)$ of a   unitary  Hecke character $\lambda$ of $M$ whose restriction to $\Q$ is $\left(\tfrac{\cdot}{D}\right)$,  and of a   unitary  character  $\nu$   of  $\A_M^1/M^1$, such that
\begin{equation*}
\lambda_\infty(z)= \frac{\bar{z}}{|z|}\text { , for all } z\in M_\infty^\times\simeq\C^\times,
\text { and }  \nu_\infty(z)= z,\text{ for all } z\in M_\infty^1,
\end{equation*}
\item $\Pi_f(\lambda,\nu)$ is the finite part of a global Arthur packet for $G$ (see \S\ref{packets}),
\item  $W(\lambda\nu_M)\in \{\pm 1\}$  is the global root number, where  $\nu_M(z)= \nu(\bar z/z)$ for $z\in \A_M^\times$,
\item $s(\pi_f)$  the number of finite  places $v$  at which  $\pi_v\simeq \pi_{c}(\lambda_v,\nu_v)$,  and
\item $\mu \in \widehat{\pi_0(Y_K)}$ acts freely on $\Xi$ by sending $(\lambda,\nu)$ to $(\lambda\mu_M^{-1},\nu\mu)$.
\end{itemize}
\end{prop}
\begin{proof} We  show that $Y_\Gamma^\ast$ admits only isolated singularities and we first observe that the complement of
$Y_\Gamma$ in $Y_\Gamma^\ast$ consists of finitely many points,  the cusps. A singular point of $Y_\Gamma$ which is not an elliptic point, is necessarily a fixed point of a single complex reflexion (an order $2$ element in $\Gamma$ fixing a hyperbolic plane).
Although the universal cover $\mathcal{H}\to Y_\Gamma$ is not \'etale at such a point, this is still a smooth point on the quotient, as locally in the analytic geometry the complex reflexion sends $(\tau, z)$ to $(\tau, -z)$ (see  \cite[\S 4.5]{holzapfel-book} for more details and additional material).
Thus  there exists a smooth open $U_K$ of the normal projective surface $Y_K^\ast$  whose complement  consists of finitely many closed points.
Lemma~\ref{top} applied  component-wise  to $U_K$ yields $q(Y_K^\ast)= \dim\rH^0(U_K, \Omega^1_{U_K})$.
Let   $K'$ be any normal finite index torsion free subgroup of $K$, {\it e.g.}
the intersection with the principal congruence subgroup of level $3$ (see  \cite[Lem.~1.4]{dimitrov-ramakrishnan}). By Koecher's Principle, as
$Y_{K'}\backslash U_{K'}$ has codimension at least  $2$  in $Y_{K'}$, we have
\[\dim\rH^0(U_K, \Omega^1_{U_K})=
\dim\rH^0(U_{K'}, \Omega^1_{U_{K'}})^{K/K'}=\dim\rH^0(Y_{K'}, \Omega^1_{Y_{K'}})^{K/K'},\]
where $U_{K'}$ is the inverse image of $U_{K}$ under the natural projection $Y_{K'}\to Y_{K}$.
Taking invariants by the finite group $K/K'$ in Rogawski's formula \cite[(14)]{dimitrov-ramakrishnan} for $q(Y_{K'}^\ast)=\dim\rH^0(Y_{K'}, \Omega^1_{Y_{K'}})$ yields
\[\dim\rH^0(U_K, \Omega^1_{U_K})= \sum_{(\lambda,\nu)\in \Xi}
\sum_{ \pi_f \in \Pi(\lambda_f,\nu_f)} \dim(\pi_f^K) \frac{1+W(\lambda\nu_M)(-1)^{s(\pi_f)}}{2}.\]
One should note a misprint in {\it loc. cit.} where one should read $(1+W(\lambda\nu_M)(-1)^{s(\pi_f)})$ instead of
$(W(\lambda\nu_M)+(-1)^{s(\pi_f)})$. The presence of this root number translates the fact that for $\pi_f \in \Pi(\lambda_f,\nu_f)$ and
$\pi_\infty$ the unique non-tempered holomorphic  representation in the local Arthur packet $\Pi(\lambda_\infty,\nu_\infty)$,
$\pi=\pi_f \otimes \pi_\infty$ is automorphic if, and only if,  $W(\lambda\nu_M)=(-1)^{s(\pi_f)}$.
Both  $\dim(\pi_f^K)$ and $1+W(\lambda\nu_M)(-1)^{s(\pi_f)}$ being preserved by the action of  $\widehat{\pi_0(Y_K)}$,
one deduces the desired formula for $q(Y_\Gamma^\ast)$ as in \cite[(15)]{dimitrov-ramakrishnan}.
\end{proof}

\subsection{Twists of canonical characters and root numbers}\label{canonical}
Hecke characters  $(\lambda,\nu)\in \Xi$ whose local components at each finite place have `minimal' ramification are intimately related to the canonical characters studied by Gross and Rohrlich. They  play a pivotal role in our production of automorphic forms contributing to the irregularity of the Picard modular surfaces of low level. We will now briefly recall some of their properties under the assumption that  $D> 3$ is odd. Consider the character $\lambda_\infty(z)=\bar{z}\cdot  |z|^{-1}$ of $M_\infty^\times\simeq \C^\times$
and let $\lambda_f:\widehat{\cO}_{M}^\times\to \C^\times$ be  a continuous  character  whose restriction  to $\cO_{M,p}^\times$
is given by the unique quadratic character
\[\cO_{M,p}^\times\to \F_p^\times\xrightarrow{\left(\tfrac{\cdot}{p}\right)}\{\pm 1\},\]
 for all $p$ dividing  $D$, and is trivial otherwise. As $\left(\tfrac{-1}{D}\right)=-1$, it follows that $\lambda_\infty$ and $\lambda_f$ agree on  $\cO_M^\times=\{\pm 1\}$.
The finiteness of the class group $\Cl_M=\A_M^\times/M^\times \widehat{\cO}_{M}^\times M_\infty^\times$ guarantees that the resulting character of  $M^\times \widehat{\cO}_{M}^\times M_\infty^\times$ extends to a character $\lambda$ of $\A_M^\times$ and clearly two such extensions must differ by a class character. As by construction the restriction of $\lambda_f$ to $\widehat{\Z}^\times$ agrees with
the  quadratic Dirichlet character  $\left(\tfrac{\cdot}{D}\right)$ viewed as a character of $\A^\times/\Q^\times \mathrm{Nm}(\A_M^\times)=\Gal(M/\Q)$, and  $\A^\times = \Q^\times  \widehat{\Z}^\times \R_+^\times$,
 it follows that the restriction of $\lambda$ to  $\A^\times$ equals  $\left(\tfrac{\cdot}{D}\right)$ ({\it i.e.} $\lambda$ is conjugate-symplectic).
 Such a character is called a canonical character and we will denote it by $\lambda_c$, remembering that it is only unique up to a multiplication by a  character of  $\Cl_M$.   The root number $W(\lambda_c^3)=-W(\lambda_c)=\left(\tfrac{-2}{D}\right)$ is $1$  if, and only if, $D\equiv 3\pmod{8}$ (see  \cite{rohrlich-canonical}).


Assume henceforth that $\det(K)=\widehat{\cO}_{M}^1$, so that $\pi_0(Y_K)=\Cl_M^1$.
Assuming further that $\widehat{\cO}_{M}^1$ embeds centrally into $K$, the central character $\omega=\nu\cdot \lambda_{|M^1}$ of any
$\pi$ contributing to \eqref{q-formula} has to be everywhere unramified, {\it i.e.},
\begin{equation}\label{q-formula-can}
q(Y_\Gamma^\ast)= \frac{1}{|\Cl_M^1|}
\sum_{\chi\in \Xi^1, \omega\in \widehat{\Cl_M^1}}
\sum_{ \pi_f \in \Pi_f(\lambda_{c}\chi_{M},\lambda_{c|M^1}^{-1} \chi^2\omega)} \dim(\pi_f^K) \frac{1+W(\lambda_c^3\chi_M^3)(-1)^{s(\pi_f)}}{2},
\end{equation}
where $\Xi^1$ denotes the set of finite order characters of $\A_M^1/M^1$ (see \cite[(3)]{rohrlich-canonical} for the fact that multiplication by a class character does not change the root number).  


If $3$ does not divide $|\Cl_M^1|$, then the action of  $\mu \in \widehat{\Cl_M^1}=
\A_M^1/M^1 \widehat{\cO}_{M}^1 M_\infty^1$ sending $(\chi,\omega)$ to
$(\chi\mu^{-1},\omega\mu^3)$ allows to twist out the central character and obtain the simpler formula:
\begin{equation}\label{q-formula-simpler}
q(Y_\Gamma^\ast)= \sum_{\chi\in  \Xi^1}
\sum_{ \pi_f \in \Pi_f(\lambda_{c}\chi_{M})} \dim(\pi_f^K) \frac{1+W(\lambda_c^3\chi_M^3)(-1)^{s(\pi_f)}}{2},
\end{equation}
where $\Pi_f(\lambda)$ is a short-hand for $\Pi_f(\lambda,\lambda_{|M^1}^{-1})$.
Proving this formula in general amounts to showing that $\dim(\pi_f^K)$ remains unchanged when multiplying  $\lambda$ or $\nu$ by  class  characters, which we will later check in all cases of interest.

Successfully applying  \eqref{q-formula-simpler} requires to understand how root numbers behave under twisting.
As we are interested in creating irregularity at  level $\Gamma''_0(\ell^r)$, we focus on characters $\chi$ which are 
only ramified at the fixed  inert  prime $\ell$. 


\begin{lemma}\label{lem:twisting} 
Assuming that $\chi_{M}^3$ has Artin conductor $\ell^a$, we have
\[ W(\lambda_c^3\chi_M^3) =(-1)^{a}  \chi_\ell(-1) W(\lambda_c^3).\]
\end{lemma}

\begin{proof} Using the factorization of root numbers $W = \prod_v W_v$, it suffices to prove that
\begin{align} 
\label{root-at-ell}  W_\ell(\lambda_c^3\chi_M^3) &  = (-1)^{a}  \chi_\ell(-1) W_\ell(\lambda_c^3) \text{, and } \\
\label{root-not-ell} W_v(\lambda_c^3\chi_M^3) & = W_v(\lambda_c^3\chi_M^3), \, \text{ for all  }  v \ne \ell, 
\end{align}
where the local  factors are defined  using  the standard additive character $\psi_M=\psi_\Q\circ \Tr_{M/\Q}$. Applying  \cite[Prop.~3]{rohrlich-galois}  to both $\lambda_{c,\ell}^3$ and $\lambda_{c,\ell}^3\chi_{M,\ell}^3$ yields  \eqref{root-at-ell}.
As $\chi_{M, \infty} =1$, it suffices to check \eqref{root-not-ell} for $v$   finite. Moreover, the characters  $\lambda_{c,v}^3$ and $\chi_{M,v}$ are unramified for  $v\nmid\ell D$, hence  both sides of \eqref{root-not-ell}  are $1$. 
Finally, for  $v$ dividing $D$,  $\chi_{M,v}$ is unramified,   $\lambda_{c,v}^3$ is tamely ramified,  while the additive character $\psi_E$ has conductor $1$, implying by \cite[(5.5.1)]{deligne-epsilon} that
$W_v(\lambda_c^3\chi_M^3)$  and $W_v(\lambda_c^3)$ differ by $\chi_{M,v}^3(\xi^{1+1\cdot 1})=(\chi_v(-1))^2=1$.
\end{proof}




\subsection{The Bombieri--Lang Conjecture for Picard modular surfaces}
In this part we follow the general strategy of \cite{dimitrov-ramakrishnan} by adapting it to the case where the level is not neat, the main point being to show that the irregularity of the Picard modular surfaces under consideration is at least $3$.
This requires different techniques  also providing a new proof to the cases treated in {\it loc. cit.}.


We recall the  index $2$ subgroup $\wK''$ of  the maximal open compact subgroup   $\wK^\circ $ of $\wG(\A_f)$ introduced in \eqref{K-double-prime}. Given  $r\in\Z_{\geqslant 1}$ and a prime $\ell$ inert in $M$,  we let $\wK''_0(\ell^r)$  be the subgroup of $\wK''$  whose component at $\ell$ is the depth $r$ Iwahori subgroup $I_r$  and we let   
  \[K''_0(\ell^r)=\wK''_0(\ell^r)\cap G(\A_f) \text{  and }   \Gamma''_0(\ell^r)= G(\Q) \cap K''_0(\ell^r)\cdot G(\R).\]

Consider an  automorphic representation $\pi\in\Pi(\lambda_c\chi_M)$ having non-zero  $K''_0(\ell^r)$-invariants. By Propositions~\ref{ram-inv} and \ref{prop:inert-inv}, for all finite place $v\ne \ell$, we have 
  $\pi_v=\pi_{n,v}$ with  $\chi_v$ unramified, and conversely  it follows from  \eqref{KD-double-prime} that the line $\bigotimes_{p\mid D} \pi_{n,p}^{K''_p}$ is  fixed by $K''_D$.  By  \eqref{q-formula-simpler} that  $\pi$ contributes to $q(Y_{\Gamma''_0(\ell^r)}^\ast)$ only when 
 $\pi_\ell=\pi_{n,\ell}$ and $W(\lambda_c^3\chi_M^3)=1$, or  $\pi_\ell=\pi_{c,\ell}$ and $W(\lambda_c^3\chi_M^3)=-1$. 
As  the choice of a  character  $\chi_\ell:\Z_{\ell^2}^1 \to \C^\times$ uniquely determines, up to a class character, a global character  
$\chi_M$ unramified outside $\ell$, the irregularity formula becomes: 
 \begin{equation}\label{q-formula-at-ell}
\tfrac{1}{h}\cdot q\left(Y_{\Gamma''_0(\ell^r)}^\ast\right)=  \sum_{W(\lambda_c^3\chi_M^3)=-1} \dim(\pi_{c,\ell}^{I_r}) +
\sum_{W(\lambda_c^3\chi_M^3)=1} \dim(\pi_{n,\ell}^{I_r}),  
\end{equation}
 and it can be rendered even more explicit  using  Lemma~\ref{lem:twisting}. 
  
The study of the conductor of the super-cuspidal non-generic representation $\pi_{c,\ell}$ appears to be very delicate even in the depth $0$ case ({\it i.e.} trivial $\chi$), where some preliminary computations suggest that it does not have $K_T$-invariants. Consequently 
we will use the lower bound on the irregularity corresponding to the contribution of $\pi$ which are the everywhere non-tempered. 



\begin{prop} \label{q3} We recall that  $D>3$ is odd and  let  $h$ be the class number of $M=\Q(\sqrt{-D})$.
\begin{enumerate}
\item If $D\equiv 3\pmod{8}$ then $q\left(Y^\ast_{\Gamma_0''(\ell)}\right)=q(Y^\ast_{\Gamma''})=h$ and
$q\left(Y^\ast_{\Gamma_0''(\ell^{2r})}\right)\geqslant (r+1)h$, for $r\in \Z_{\geqslant 1}$. 

\item If $D\equiv 7\pmod{8}$ then $q\left(Y^\ast_{\Gamma_0''(27)}\right)\geqslant h$ and 
$q\left(Y^\ast_{\Gamma_0''(\ell^3)}\right)\geqslant  3h $,   for $\ell\geqslant 7$.   
\end{enumerate}
\end{prop}

\begin{proof}  The proof is based on the inequality 
$\left(Y_{\Gamma''_0(\ell^r)}^\ast\right)= h\cdot  \sum_{W(\lambda_c^3\chi_M^3)=1} \dim(\pi_{n,\ell}^{I_r})$ 
resulting from  \eqref{q-formula-at-ell} and the results on root numbers in \S\ref{canonical}. 

(i) As $W(\lambda_c^3)=1$ we can take $\chi=\mathbf{1}$. The claim follows from 
Proposition~\ref{prop:inert-inv} and Corollary~\ref{cor:deep-iwahori-inv}. 

(ii) As $W(\lambda_c^3)=-1$ we use here a tamely ramified $\chi_\ell$ to switch the sign. 
By  Lemma~\ref{lem:twisting}, in order to have $W(\lambda_c^3\chi_M^3)=-W(\lambda_c^3)=1$, 
we need $\chi_\ell^3$ to be non-trivial and $\chi_\ell(-1)=1$
For $\ell\geqslant 3$, there are precisely $\frac{\ell-1}{2}$ choices for 
$\chi_\ell$, if $3 \nmid (\ell+1)$,  and $\frac{\ell-5}{2}$ choices,  if $3 \mid (\ell+1)$.  In particular, there are at least $3$ choices for all 
 $\ell\geqslant 7$.  
\end{proof}



The following results  from Faltings \cite{faltings-lang}  (see \cite[\S 3.2]{dimitrov-ramakrishnan}  for detail).

\begin{thm}  Let $\wK$ an open compact subgroup   of $\wG(\A_f)$ and let $\Gamma= \wG(\Q)\cap  \wK \wG(\R)$.
If $q\left(Y^\ast_{\Gamma}\right)\geqslant 3$, then  $Y^\ast_{\wK}$ satisfies the   Bombieri--Lang Conjecture, {\it i.e.},
 any  number field $k$ the  $k$-rational points of $Y^\ast_{\wK}$ are not Zariski dense.
\end{thm}


\begin{cor} \label{bomb-lang}
If $D\equiv 3\pmod{8}$,   then the Bombieri--Lang Conjecture holds for $Y_{\wK''(\ell^4)}$, and even for
$Y^\ast_{\wK''}$, when $h\geqslant 3$.
If $D\equiv 7\pmod{8}$,  then the  Bombieri--Lang Conjecture holds for 
$Y^\ast_{\wK''(\ell^3)}$ for $\ell\geqslant 7$, and for $Y^\ast_{\wK''(3^7)}$. 
\end{cor}

As $2$  splits in $M$ for $D\equiv 7\pmod{8}$, we only exclude $\ell=5$ when $D\equiv 7 \text{ or }  23 \pmod{40}$. 

\section{Uniform irreducibility of Galois images}\label{s:uniform}



\subsection{Complex reflexions, elliptic elements and abelian families}\label{sec:abelian-family}

In this subsection we prove that, after enlarging $k$, there exists a natural family of  abelian $3$-folds of Picard type over
$Y_{\wK''}\times_{M} k$ minus a finite number of $k$-rational elliptic points.
This will be crucially used in the proof of Theorem~\ref{theoremB} in the next subsection.
Fix a  geometrically connected component $Y''$ of $Y_{\wK''}\times_{M} k$. We have $Y''(\C)=\Gamma''\backslash\mathcal{H}$, where $\Gamma''=\wG(\Q)\cap g_f\wK'' g_f^{-1}  \wG(\R)$ for some $g_f\in  G(\A_f)$.
We will first show that Gross' level structure $\wK''_D$ introduced in \eqref{KD-double-prime} prevents $\Gamma$ from
containing complex reflexions. Then we will classify the elliptic elements in $\Gamma''$ and deduce the sought for abelian scheme.


Recall that non-scalar element $\gamma$ of the discrete subgroup $\Gamma\subset \wG(\R)$ ({\it i.e.} a non-trivial element in $\overline{\Gamma}=\Gamma/(\Gamma\cap M^\times)$) has a fixed point in $\mathcal{H}$ if, and only if, $\gamma$ has finite order (this is because the stabilizers in $\wG(\R)$
of  points in $\mathcal{H}$ are maximal compact subgroups). Such a  $\gamma$ is called elliptic if it only fixes an isolated point in $\mathcal{H}$, otherwise it is called a complex reflexions.

\begin{lemma}\label{lem:reflexions}
The group $\Gamma''$ does not contain any complex reflexions.
\end{lemma}

\begin{proof} Consider a  complex reflexion $\gamma\in \wG(\Q)\cap g_f\wK^\circ g_f^{-1}  \wG(\R)$ as an endomorphism of the Hermitian space $M^3$ having signature $(2,1)$.
As eigenspaces for $\gamma$ are mutually orthogonal, at most one such eigenspaces could contain a negative line (corresponding to a point in $\mathcal{H}$).
This if $\gamma$ fixes more that one point of $\mathcal{H}$, it necessarily fixes a hyperbolic line in  $\mathcal{H}$. The corresponding endomorphism of  $M^3$ has
has an eigenplane and an orthogonal eigenline (both $M$-rational),  forcing the eigenvalues  to be in $\cO_M^\times=\{\pm 1\}$ (as $D>4$) and not all equal.   It follows, that for any $p\mid D$,  the image of $g_f^{-1}  \gamma  g_f\in\wK^\circ$  into the projectivization of the reductive quotient of $\wK_p^\circ$ belongs to the image under the adjoint isomorphism $\PGL_2(\F_p)  \xrightarrow{\sim}  \mathrm{PGO}(3,\F_p)$
 of an element represented by a matrix having both eigenvalues $1$ and $-1$. In particular its image in
 $\PGL_2(\F_p)/\PSL_2(\F_p)=\{\pm 1\}$ equals $\left(\tfrac{-1}{p}\right)$. As $\prod_{p\mid D} \left(\tfrac{-1}{p}\right)=  \left(\tfrac{-1}{D}\right)=-1$
 it follows that $g_f^{-1}  \gamma  g_f\notin\wK''$ (see  \eqref{K-double-prime}), {\it i.e.}$\gamma \notin \Gamma'''$.
  \end{proof}


\begin{lemma}\label{lem:elliptic}
The set of fixed points in $Y''(\C)$ is finite and defined over a finite extension of~$M$.
\end{lemma}
\begin{proof} One proceeds by explicitly determining the  elliptic elements of  $\Gamma''$. 
  \end{proof}

After possibly enlarging $k$ we may assume that the set  $Z$ from  Lemma~\ref{lem:elliptic} is defined over $k$ and we denote by $U$ the complementary open in $Y''$. Following the discussion in \S\ref{shimura-surfaces} the open $U$  lifts to an open in the corresponding algebraic moduli stack, as
 such it is naturally endowed with a family $f:A\to U$ of abelian $3$-folds of Picard type.


\subsection{Proof of Main Theorem}
At  different stages of the proof we will remove finite sent and deal with them in the last step.
By  \S\ref{moduli} and \S\ref{lattices-rationality} an abelian variety $A$ as in the Theorem~\ref{theoremB} defines a $k$-rational point on $Y_{\wK''}$, where $\wK''$ is
defined in \eqref{K-double-prime}.
If $A[\ell^r]$ admits a full $k$-rational flag (or equivalently, a $k$-rational isotropic line),  we claim that $A$ defines a $k$-rational point on $Y_{\wK''_0(\ell^r)}$,
where $\wK''_0(\ell^r)\subset\wK''$ is the  subgroup whose component at $\ell$ consists of elements whose reduction  modulo $\ell^r$ belong to  the standard Borel subgroup  $\widetilde{B}(\Z/\ell^r\Z)$ of $\wG(\Z/\ell^r\Z)$. Indeed,  by assumption one  knows that there is
 some Borel subgroup containing  the image of $\Gal_k$ acting on $A[\ell^r]$. However $\widetilde{G}(\Q_\ell)$ acts transitively on isotropic lines
(because isometries between hermitian subspaces always extend), hence all Borel subgroups are conjugated by $\widetilde{G}(\Q_\ell)$, and in fact by  $\wK_\ell^\circ=\widetilde{G}(\Z_\ell)$ (using Iwasawa  decomposition).  As  $\wK''$ is a normal subgroup of $\wK^\circ$ we deduce that  the Galois image is contained in $\wK''\cap \wK_0(\ell^r)=\wK''_0(\ell^r)$, proving the claim.


 By   Corollary~\ref{bomb-lang}, $Y^{\ast}_{\wK''_0(\ell^4)}$  satisfies the Strong Bombieri--Lang Conjecture.
 In particular all  its $k$-rational points lie in a subvariety $Z$ defined over $k$ which is a finite union of points and curves.


Let us now take one of the (finitely many) geometrically connected curve  $C$ in $Z$, and after removing finitely many of its points
(which would not affect the wanted result), we may assume that  $C$ is contained in the smooth open $U$ from \S\ref{sec:abelian-family}. In particular, there exists a family $f:A\to C$ of abelian $3$-folds of Picard type.

Consider image $\Gamma_C\subset \wK_\ell^\circ=\widetilde{G}(\Z_\ell)$ of the \'etale fundamental group acting on the $\ell$-adic Tate module of the generic fiber of the family. By Cartan's theorem (see \cite[LG5.42]{serre-LALG}),  $\Gamma_C$ is an $\ell$-adic Lie group hence admits a  Lie algebra $\fh$. By Bogomolov \cite{bogomolov-SSSR} the Lie algebra $\fh_{\Q_\ell}$ is the Lie algebra of the Zariski closure of
$\Gamma_C$ in $\widetilde{G}(\Q_\ell)$, the latter being   furthermore  reductive over $\Q_\ell$ by   Faltings \cite[Thm.~3]{faltings-AG}.
By the Mumford--Tate Conjecture, which is known for is known for abelian $3$-folds (see {\it e.g.} \cite{chi}), we know that
$\fh_{\Q_\ell}$ is the Lie algebra of the Mumford--Tate group $\MT(A)\otimes \Q_\ell$.
 As $C$  has positive dimension, it has to contain non-CM points, whose Mumford--Tate group is not abelian.
By  Lemma~\ref{prop:tricotomy}, the  Lie subalgebra $\fh_{\Q_\ell}\subset \mathfrak{su}(3,\Q_\ell)$  contains a form of $\mathfrak{sl}(2,\Q_\ell)$. 
By  \cite[Thm.~1.1]{cadoret-tamagawa-duke1}  applied to the abelian family $f:A\to C$  there exist $B>0$ such that for all  $x\in C(k)$ outside a finite set  $C_\rho$ we have
  \begin{equation}\label{eq:uniform}
    [\Gamma_C: \Gamma_x]\leqslant B,
    \end{equation}
where  $\Gamma_x=\rho_{A_x,\ell}(\Gal_k)$  with $A_x$  the abelian $3$-fold of Picard type corresponding to $x$.


\begin{lemma}
There exists $r=r(C)\in \Z$ such that $\Gamma_x\supset \exp\left(\mathfrak{su}(2,\ell^{r}\Z_\ell)\right)$ for all $x\in C(k)\setminus C_\rho$.
\end{lemma}

\begin{proof}
We fix an exponential map on $\mathfrak{su}(2,\Q_\ell)$ so that  $\Gamma_C\supset \exp\left(\mathfrak{su}(2,\Z_\ell)\right)$. 
Using  that a subgroup of index at most $B$ contains a normal subgroup of index at most $B!$,  \eqref{eq:uniform} implies that 
 $\Gamma_x$ contains $B! \cdot \exp\left(\mathfrak{su}(2,\Z_\ell)\right)=\exp\left(\mathfrak{su}(2,\ell^{r}\Z_\ell)\right)$, where $r$ is the $\ell$-adic valuation of $B!$.
\end{proof}

It follows that $r\geqslant r(C)$ and for all $A$ as above  corresponding to a $k$-rational point on $C\setminus C_\rho$,  $A[\ell^r]$ does not admit a  full $k$-rational flag. Finally, a direct application of Theorem~\ref{prop:MT} to the finitely many remaining  $k$-rational points yields  an integer $r$ such that all $k$-rational points in $Y_{\wK''_0(\ell^r)}$ are of CM type, proving the Theorem. 



\subsection{Odd and ends} \label{odds-ends} In a previous paper,  we mistakenly switched $\pi_{c}$  and $\pi_{2}$ (which are in the same $L$-packet)  in the  proof of Proposition 3.8 in \cite{dimitrov-ramakrishnan} which also had consequences for the assertions of Theorem~0.3 in  {\it loc. cit.}. Providing $I_r$-invariants for its supercuspidal member $\pi_c$ seem to be a thorny issue. 
Even in the simplest case of $\pi_c\in \Pi(\lambda_0)$ which can be shown to be of depth $0$, {\it i.e.} coming from representation theory of finite groups, preliminary computation via Deligne--Lusztig theory suggest that it does not have $K_T$-invariants. 
This leaves us with no choice but to use $\pi_n$.  
 
The methods of this paper based on ramified $\pi_n$ allow us to find a new proof of part (ii) of Theorem~0.3 in  {\it loc. cit.},  in fact  yielding an even stronger result.  The case (i) will be discussed elsewhere. 



\addtocontents{toc}{\setcounter{tocdepth}{0}}


\bibliographystyle{siam}
\begin{thebibliography}{10}

\bibitem{arthur-intertwining}
{\sc J.~Arthur}, {\em Intertwining operators and residues. {I}. {W}eighted
  characters}, J. Funct. Anal., 84 (1989), pp.~19--84.

\bibitem{bogomolov-SSSR}
{\sc F.~A. Bogomolov}, {\em Points of finite order on an abelian variety}, Izv.
  Akad. Nauk SSSR Ser. Mat., 44 (1980), pp.~782--804.

\bibitem{cadoret-tamagawa-inv}
{\sc A.~Cadoret and A.~Tamagawa}, {\em Uniform boundedness of {$p$}-primary
  torsion of abelian schemes}, Invent. Math., 188 (2012), pp.~83--125.

\bibitem{cadoret-tamagawa-duke1}
\leavevmode\vrule height 2pt depth -1.6pt width 23pt, {\em A uniform open image
  theorem for {$\ell$}-adic representations, {I}}, Duke Math. J., 161 (2012),
  pp.~2605--2634.

\bibitem{chi}
{\sc W.~C. Chi}, {\em On the {$l$}-adic representations attached to simple
  abelian varieties of type {${\rm IV}$}}, Bull. Austral. Math. Soc., 44
  (1991), pp.~71--78.

\bibitem{deligne-hodgeII}
{\sc P.~Deligne}, {\em Th\'{e}orie de {H}odge. {II}}, Inst. Hautes \'{E}tudes
  Sci. Publ. Math.,  (1971), pp.~5--57.

\bibitem{deligne-epsilon}
{\sc P.~Deligne}, {\em Les constantes des \'{e}quations fonctionnelles des
  fonctions {$L$}}, in Modular functions of one variable, {II} ({P}roc.
  {I}nternat. {S}ummer {S}chool, {U}niv. {A}ntwerp, {A}ntwerp, 1972), Lecture
  Notes in Math., Vol. 349, Springer, Berlin-New York, 1973, pp.~501--597.

\bibitem{deligne-milne-ogus-shih}
{\sc P.~Deligne, J.~S. Milne, A.~Ogus, and K.-Y. Shih}, {\em Hodge cycles,
  motives, and {S}himura varieties}, vol.~900 of Lecture Notes in Mathematics,
  Springer-Verlag, Berlin-New York, 1982.

\bibitem{dimca}
{\sc A.~Dimca}, {\em Sheaves in topology}, Universitext, Springer-Verlag,
  Berlin, 2004.

\bibitem{dimitrov-ramakrishnan}
{\sc M.~Dimitrov and D.~Ramakrishnan}, {\em Arithmetic quotients of the complex
  ball and a conjecture of {L}ang}, Doc. Math., 20 (2015), pp.~1185--1205.

\bibitem{faltings-AG}
{\sc G.~Faltings}, {\em Finiteness theorems for abelian varieties over number
  fields}, in Arithmetic geometry ({S}torrs, {C}onn., 1984), Springer, New
  York, 1986, pp.~9--27.
\newblock Translated from the German original [Invent. Math. {{\bf{7}}3}
  (1983), no. 3, 349--366; MR0718935; ibid. {{\bf{7}}5} (1984), no. 2, 381;
  MR0732554] by Edward Shipz.

\bibitem{faltings-lang}
\leavevmode\vrule height 2pt depth -1.6pt width 23pt, {\em The general case of
  {S}. {L}ang's conjecture}, in Barsotti {S}ymposium in {A}lgebraic {G}eometry
  ({A}bano {T}erme, 1991), vol.~15 of Perspect. Math., Academic Press, San
  Diego, CA, 1994, pp.~175--182.

\bibitem{GGP1}
{\sc W.~T. Gan, B.~H. Gross, and D.~Prasad}, {\em Symplectic local root
  numbers, central critical {$L$} values, and restriction problems in the
  representation theory of classical groups}, Ast\'{e}risque, 346 (2012),
  pp.~1--109.
\newblock Sur les conjectures de Gross et Prasad. I.

\bibitem{grothendieck-monodromie}
{\sc A.~Grothendieck}, {\em Mod\`{e}les de {N}\'{e}ron et monodromie}, vol.~288
  of Lecture Notes in Mathematics, Springer-Verlag, 1972, pp.~313--523.

\bibitem{haines-kottwitz-prasad}
{\sc T.~J. Haines, R.~E. Kottwitz, and A.~Prasad}, {\em Iwahori-{H}ecke
  algebras}, J. Ramanujan Math. Soc., 25 (2010), pp.~113--145.

\bibitem{holzapfel-book}
{\sc R.-P. Holzapfel}, {\em Ball and surface arithmetics}, vol.~E29 of Aspects
  of Mathematics, Friedr. Vieweg \& Sohn, Braunschweig, 1998.

\bibitem{kwlan-book}
{\sc K.-W. Lan}, {\em Arithmetic compactifications of {PEL}-type {S}himura
  varieties}, vol.~36 of London Mathematical Society Monographs Series,
  Princeton University Press, Princeton, NJ, 2013.

\bibitem{manin}
{\sc J.~I. Manin}, {\em The {$p$}-torsion of elliptic curves is uniformly
  bounded}, Izv. Akad. Nauk SSSR Ser. Mat., 33 (1969), pp.~459--465.

\bibitem{miyauchi-U21}
{\sc M.~Miyauchi}, {\em On local newforms for unramified {${\rm U}(2,1)$}},
  Manuscripta Math., 141 (2013), pp.~149--169.

\bibitem{miyauchi-L-factor}
{\sc M.~Miyauchi}, {\em On {$L$}-factors attached to generic representations of
  unramified {$\rm U(2,1)$}}, Math. Z., 289 (2018), pp.~1381--1408.

\bibitem{mumford}
{\sc D.~Mumford}, {\em Abelian varieties}, vol.~5 of Tata Institute of
  Fundamental Research Studies in Mathematics, Published for the Tata Institute
  of Fundamental Research, Bombay; by Hindustan Book Agency, New Delhi, 2008.
\newblock With appendices by C. P. Ramanujam and Yuri Manin, Corrected reprint
  of the second (1974) edition.

\bibitem{olson-book}
{\sc M.~Olsson}, {\em Algebraic spaces and stacks}, vol.~62 of American
  Mathematical Society Colloquium Publications, American Mathematical Society,
  Providence, RI, 2016.

\bibitem{platonov-rapinchuk}
{\sc V.~Platonov and A.~Rapinchuk}, {\em Algebraic groups and number theory},
  vol.~139 of Pure and Applied Mathematics, Academic Press, Inc., Boston, MA,
  1994.
\newblock Translated from the 1991 Russian original by Rachel Rowen.

\bibitem{rodier}
{\sc F.~Rodier}, {\em Whittaker models for admissible representations of
  reductive {$p$}-adic split groups}, in Harmonic analysis on homogeneous
  spaces ({P}roc. {S}ympos. {P}ure {M}ath., {V}ol. {XXVI}, {W}illiams {C}oll.,
  {W}illiamstown, {M}ass., 1972), 1973, pp.~425--430.

\bibitem{rogawski-U3}
{\sc J.~D. Rogawski}, {\em Automorphic representations of unitary groups in
  three variables}, vol.~123 of Annals of Mathematics Studies, Princeton
  University Press, Princeton, NJ, 1990.

\bibitem{rogawski-A-packets}
\leavevmode\vrule height 2pt depth -1.6pt width 23pt, {\em The multiplicity
  formula for {$A$}-packets}, in The zeta functions of {P}icard modular
  surfaces, Univ. Montr\'{e}al, Montreal, QC, 1992, pp.~395--419.

\bibitem{rohrlich-canonical}
{\sc D.~E. Rohrlich}, {\em On the {$L$}-functions of canonical {H}ecke
  characters of imaginary quadratic fields}, Duke Math. J., 47 (1980),
  pp.~547--557.

\bibitem{rohrlich-galois}
\leavevmode\vrule height 2pt depth -1.6pt width 23pt, {\em Galois theory,
  elliptic curves, and root numbers}, Compositio Math., 100 (1996),
  pp.~311--349.

\bibitem{serre-kyoto}
{\sc J.-P. Serre}, {\em Repr\'{e}sentations {$l$}-adiques}, in Algebraic number
  theory ({K}yoto {I}nternat. {S}ympos., {R}es. {I}nst. {M}ath. {S}ci., {U}niv.
  {K}yoto, {K}yoto, 1976), Japan Soc. Promotion Sci., Tokyo, 1977,
  pp.~177--193.

\bibitem{serre-LALG}
\leavevmode\vrule height 2pt depth -1.6pt width 23pt, {\em Lie algebras and
  {L}ie groups}, vol.~1500 of Lecture Notes in Mathematics, Springer-Verlag,
  Berlin, 2006.
\newblock 1964 lectures given at Harvard University, Corrected fifth printing
  of the second (1992) edition.

\bibitem{shimura-CM}
{\sc G.~Shimura}, {\em On the zeta-function of an abelian variety with complex
  multiplication}, Ann. of Math. (2), 94 (1971), pp.~504--533.

\bibitem{tits}
{\sc J.~Tits}, {\em Reductive groups over local fields}, in Automorphic forms,
  representations and {$L$}-functions ({P}roc. {S}ympos. {P}ure {M}ath.,
  {O}regon {S}tate {U}niv., {C}orvallis, {O}re., 1977), {P}art 1, Proc. Sympos.
  Pure Math., XXXIII, Amer. Math. Soc., Providence, R.I., 1979, pp.~29--69.

\end{thebibliography}


\end{document}

