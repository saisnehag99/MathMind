\documentclass[15pt,oneside,reqno]{amsart} 
\usepackage{latexsym}
\usepackage{amscd}
\usepackage{amsmath}
\usepackage{amssymb}
\usepackage[mathscr]{euscript}
\usepackage{graphicx}
\usepackage{geometry}
\usepackage{mathtools}
\usepackage[all]{xy}
\usepackage[dvipsnames]{xcolor}
\usepackage[colorlinks,final,hyperindex]{hyperref}
\usepackage{tikz}
\usepackage{tikz-cd}
\usetikzlibrary{decorations.pathmorphing,decorations.markings,arrows,calc,shapes.geometric,arrows.meta,positioning}
\usepackage{enumerate}
\usepackage{multirow}

\usepackage[bb=dsserif]{mathalpha}
\usepackage{bm}

\newcommand{\QQ}{\ensuremath{\mathbb{Q}}}
\newcommand{\DD}{\ensuremath{\mathbb{D}}}
\newcommand{\CC}{\ensuremath{\mathbb{C}}}
\newcommand{\RR}{\ensuremath{\mathbb{R}}}
\newcommand{\ZZ}{\ensuremath{\mathbb{Z}}}
\newcommand{\NN}{\ensuremath{\mathbb{N}}}

\newcommand{\g}{\ensuremath{\mathfrak{g}}}
\renewcommand{\SS}{\ensuremath{\mathbb{S}}}

\newcommand{\id}{\ensuremath{\mathrm{id}}}

\makeatletter

\newcommand{\bbone}{\mathbb{1}}

\newcommand{\calA}{\mathcal{A}}
\newcommand{\calB}{\mathcal{B}}
\newcommand{\calC}{\mathcal{C}}
\newcommand{\calD}{\mathcal{D}}
\newcommand{\calE}{\mathcal{E}}
\newcommand{\calF}{\mathcal{F}}
\newcommand{\calG}{\mathcal{G}}
\newcommand{\calH}{\mathcal{H}}
\newcommand{\calI}{\mathcal{I}}
\newcommand{\calJ}{\mathcal{J}}
\newcommand{\calK}{\mathcal{K}}
\newcommand{\calL}{\mathcal{L}}
\newcommand{\calM}{\mathcal{M}}
\newcommand{\calN}{\mathcal{N}}
\newcommand{\calO}{\mathcal{O}}
\newcommand{\calP}{\mathcal{P}}
\newcommand{\calQ}{\mathcal{Q}}
\newcommand{\calR}{\mathcal{R}}
\newcommand{\calS}{\mathcal{S}}
\newcommand{\calT}{\mathcal{T}}
\newcommand{\calU}{\mathcal{U}}
\newcommand{\calV}{\mathcal{V}}
\newcommand{\calW}{\mathcal{W}}
\newcommand{\calX}{\mathcal{X}}
\newcommand{\calY}{\mathcal{Y}}
\newcommand{\calZ}{\mathcal{Z}}
\newcommand{\kk}{\Bbbk}

\newcommand{\scrY}{\mathscr{Y}}
\newcommand{\scrV}{\mathscr{V}}
\newcommand{\scrP}{\mathscr{P}}

\newcommand{\Rep}{\mathop{\mathrm{Rep}}\nolimits}
\newcommand{\alg}{\mathop{\mathrm{alg}}\nolimits}
\newcommand{\Span}{\mathop{\mathrm{Span}}\nolimits}
\newcommand{\proj}{\mathop{\mathrm{proj}}\nolimits}
\newcommand{\Hom}{\mathop{\mathrm{Hom}}\nolimits}
\newcommand{\PHom}{\mathop{\mathrm{PHom}}\nolimits}
\newcommand{\Tw}{\mathop{\mathrm{Tw}}\nolimits}
\newcommand{\Dim}{\mathop{\mathrm{Dim}}\nolimits}
\newcommand{\Imm}{\mathop{\mathrm{Im}}\nolimits}
\newcommand{\Aus}{\mathop{\mathrm{Aus}}\nolimits}
\newcommand{\Ann}{\mathop{\mathrm{Ann}}\nolimits}
\newcommand{\APC}{\mathop{\mathrm{APC}}\nolimits}
\newcommand{\Barr}{\mathop{\mathrm{Bar}}\nolimits}
\newcommand{\Cone}{\mathop{\mathrm{Cone}}\nolimits}
\newcommand{\modu}{\mathop{\mathrm{mod}}\nolimits}
\newcommand{\dgMod}{\mathop{\mathrm{dgMod}}\nolimits}
\newcommand{\soc}{\mathop{\mathrm{soc}}\nolimits}
\newcommand{\dg}{\mathop{\mathrm{dg}}\nolimits}
\newcommand{\red}{\mathop{\mathrm{red}}\nolimits}
\newcommand{\Map}{\mathop{\mathrm{Map}}\nolimits}
\newcommand{\inj}{\mathop{\mathrm{inj}}\nolimits}
\newcommand{\op}{\mathop{\mathrm{op}}\nolimits}
\newcommand{\Ker}{\mathop{\mathrm{Ker}}\nolimits}
\newcommand{\Tor}{\mathop{\mathrm{Tor}}\nolimits}
\newcommand{\Tot}{\mathop{\mathrm{Tot}}\nolimits}
\newcommand{\Ext}{\mathop{\mathrm{Ext}}\nolimits}
\newcommand{\sg}{\mathop{\mathrm{sg}}\nolimits}
\newcommand{\Sl}{\mathop{\mathfrak{sl}}\nolimits}
\newcommand{\nc}{\mathop{\mathrm{nc}}\nolimits}
\newcommand{\Tr}{\mathop{\mathrm{Tr}}\nolimits}
\newcommand{\Irr}{\mathop{\mathrm{Irr}}\nolimits}
\newcommand{\HC}{\mathop{\mathrm{HC}}\nolimits}
\newcommand{\HH}{\mathop{\mathrm{HH}}\nolimits}
\newcommand{\THH}{\mathop{\mathrm{TH}}\nolimits}
\newcommand{\Perf}{\mathop{\mathrm{Perf}}\nolimits}
\newcommand{\del}{\partial}
\newcommand{\ev}{\mathop{\mathrm{ev}}\nolimits}
\newcommand{\co}{\mathop{\mathrm{co}}\nolimits}
\newcommand{\pt}{\mathop{\mathrm{pt}}\nolimits}
\newcommand{\wt}{\mathop{\mathrm{wt}}\nolimits}
\newcommand{\pCY}{\mathop{\mathrm{pCY}}\nolimits}
\newcommand{\gr}{\mathop{\mathrm{gr}}\nolimits}
\newcommand{\coker}{\mathop{\mathrm{coker}}\nolimits}


\newcommand{\Top}{\mathop{\mathrm{Top}}\nolimits}
\newcommand{\Set}{\mathop{\mathrm{Set}}\nolimits}
\newcommand{\Vect}{\mathop{\mathrm{Vect}}\nolimits}
\newcommand{\Mod}{\mathop{\mathrm{Mod}}\nolimits}
\newcommand{\Ob}{\mathop{\mathrm{Ob}}\nolimits}
\newcommand{\Ind}{\mathop{\mathrm{Ind}}\nolimits}
\newcommand{\Sym}{\mathop{\mathrm{Sym}}\nolimits}
\newcommand{\Alt}{\mathop{\mathrm{Alt}}\nolimits}
\newcommand{\End}{\mathop{\mathrm{End}}\nolimits}
\newcommand{\Ch}{\mathop{\mathrm{Ch}}\nolimits}


\newcommand{\Ass}{\mathop{\mathrm{Ass}}\nolimits}
\newcommand{\Com}{\mathop{\mathrm{Com}}\nolimits}
\newcommand{\Lie}{\mathop{\mathrm{Lie}}\nolimits}
\newcommand{\colim}{\mathop{\mathrm{colim}}\nolimits}

\newcommand{\Op}{\mathop{\mathsf{Operads}}\nolimits}
\newcommand{\Diop}{\mathop{\mathsf{Dioperads}}\nolimits}
\newcommand{\PDiop}{\mathop{\mathsf{PDioperads}}\nolimits}
\newcommand{\Coop}{\mathop{\mathsf{Cooperads}}\nolimits}
\newcommand{\Codiop}{\mathop{\mathsf{Codioperads}}\nolimits}
\newcommand{\PCodiop}{\mathop{\mathsf{PCodioperads}}\nolimits}

\newcommand{\antish}{\ensuremath{\mbox{!`}}}


\tikzset{->-/.style={decoration={
			markings,
			mark=at position #1 with {\arrow{>}}},postaction={decorate}}}
\tikzset{-w-/.style={decoration={
			markings,
			mark=at position #1 with {\arrow{Stealth[fill=white,scale=1.4]}}},postaction={decorate}}}
\tikzset{->-/.default=0.65}
\tikzset{-w-/.default=0.65}
\tikzstyle{bullet}=[circle,fill=black,inner sep=0.5mm]
\tikzstyle{circ}=[circle,draw=black,fill=white,inner sep=0.5mm]
\tikzstyle{vertex}=[circle,draw=black,thick,inner sep=0.5mm]
\tikzstyle{dot}=[draw,circle,fill=black,minimum size=0.5mm,inner sep = 0mm, outer sep = 0mm]
\tikzset{darrow/.style={double distance = 4pt,>={Implies},->},
	darrowthin/.style={double equal sign distance,>={Implies},->},
	tarrow/.style={-,preaction={draw,darrow}},
	qarrow/.style={preaction={draw,darrow,shorten >=0pt},shorten >=1pt,-,double,double
		distance=0.2pt}}
\newcommand\tikcirc[1][2.5]{\tikz[baseline=-#1]{\draw[thick](0,0)circle[radius=#1mm];}}
\newcommand{\tikzfig}[1]{\begin{tikzpicture}[auto,baseline={([yshift=-.5ex]current bounding box.center)}]#1\end{tikzpicture}}
\usetikzlibrary{decorations.pathreplacing}

\usepackage{amsthm}
\usepackage{thmtools}
\usepackage[noabbrev,capitalize]{cleveref}
\newtheorem{theorem}{Theorem}
\newtheorem{lemma}[theorem]{Lemma}
\newtheorem{proposition}[theorem]{Proposition}
\newtheorem{corollary}[theorem]{Corollary}
\newtheorem{conjecture}[theorem]{Conjecture}
\newtheorem{Claim}[theorem]{Claim}

\newtheorem*{theorem*}{Theorem}

\theoremstyle{definition}
\newtheorem{definition}{Definition}
\newtheorem{example}{Example}
\newtheorem{cexample}{Counterexample}
\newtheorem{exercise}{Exercise}
\newtheorem{question}{Question}
\newtheorem{remark}{Remark}

\addtolength{\textheight}{1.75cm}
\addtolength{\voffset}{-0.5cm}
\addtolength{\footskip}{+0.6cm}
\geometry{margin=1.2in}


\usepackage[backend=biber, maxbibnames=99, bibencoding=utf8, doi=false, isbn=false, url=false, style=alphabetic]{biblatex}
\addbibresource{refs.bib}


\newcommand{\isMainDocument}{} % 各sectionに影響を与えるフラグを設定

\begin{document}
\maketitle

\begin{abstract}
Jarossay introduced adjoint multiple zeta values, and he found $\Q$-algebraic relations among adjoint multiple zeta values, referred to as the \emph{adjoint double shuffle relations}, by using Racinet's dual formulation of the generating series of multiple zeta values.
Jarossay defined the affine scheme $\AdDMR_0$ determined by the adjoint double shuffle relations and posed a question whether $\AdDMR_0$ is isomorphic to Racinet's double shuffle group $\DMR_0$. 
In this paper, we refine Jarossay's question by formulating what we call the adjoint conditions and by addressing its Lie algebraic side. 
Within this framework, we construct the Lie algebra associated with the adjoint double shuffle relations by imposing Hirose's parity results.
%Our main theorem provides the first nontrivial algebraic structure from the $\Q$-linear relations among adjoint multiple zeta values.
%We then give an isomorphism of $\Q$-linear spaces obtained by a bigraded version of the double shuffle relations and that of the adjoint double shuffle relations.
\end{abstract}

\tableofcontents

\ifdefined\isMainDocument
\else
    \documentclass[10pt,oneside,reqno]{amsart}
    \usepackage{xr}
    \externaldocument{../main} % Main の .aux を参照
    \usepackage{latexsym}
\usepackage{amscd}
\usepackage{amsmath}
\usepackage{amssymb}
\usepackage[mathscr]{euscript}
\usepackage{graphicx}
\usepackage{geometry}
\usepackage{mathtools}
\usepackage[all]{xy}
\usepackage[dvipsnames]{xcolor}
\usepackage[colorlinks,final,hyperindex]{hyperref}
\usepackage{tikz}
\usepackage{tikz-cd}
\usetikzlibrary{decorations.pathmorphing,decorations.markings,arrows,calc,shapes.geometric,arrows.meta,positioning}
\usepackage{enumerate}
\usepackage{multirow}

\usepackage[bb=dsserif]{mathalpha}
\usepackage{bm}

\newcommand{\QQ}{\ensuremath{\mathbb{Q}}}
\newcommand{\DD}{\ensuremath{\mathbb{D}}}
\newcommand{\CC}{\ensuremath{\mathbb{C}}}
\newcommand{\RR}{\ensuremath{\mathbb{R}}}
\newcommand{\ZZ}{\ensuremath{\mathbb{Z}}}
\newcommand{\NN}{\ensuremath{\mathbb{N}}}

\newcommand{\g}{\ensuremath{\mathfrak{g}}}
\renewcommand{\SS}{\ensuremath{\mathbb{S}}}

\newcommand{\id}{\ensuremath{\mathrm{id}}}

\makeatletter

\newcommand{\bbone}{\mathbb{1}}

\newcommand{\calA}{\mathcal{A}}
\newcommand{\calB}{\mathcal{B}}
\newcommand{\calC}{\mathcal{C}}
\newcommand{\calD}{\mathcal{D}}
\newcommand{\calE}{\mathcal{E}}
\newcommand{\calF}{\mathcal{F}}
\newcommand{\calG}{\mathcal{G}}
\newcommand{\calH}{\mathcal{H}}
\newcommand{\calI}{\mathcal{I}}
\newcommand{\calJ}{\mathcal{J}}
\newcommand{\calK}{\mathcal{K}}
\newcommand{\calL}{\mathcal{L}}
\newcommand{\calM}{\mathcal{M}}
\newcommand{\calN}{\mathcal{N}}
\newcommand{\calO}{\mathcal{O}}
\newcommand{\calP}{\mathcal{P}}
\newcommand{\calQ}{\mathcal{Q}}
\newcommand{\calR}{\mathcal{R}}
\newcommand{\calS}{\mathcal{S}}
\newcommand{\calT}{\mathcal{T}}
\newcommand{\calU}{\mathcal{U}}
\newcommand{\calV}{\mathcal{V}}
\newcommand{\calW}{\mathcal{W}}
\newcommand{\calX}{\mathcal{X}}
\newcommand{\calY}{\mathcal{Y}}
\newcommand{\calZ}{\mathcal{Z}}
\newcommand{\kk}{\Bbbk}

\newcommand{\scrY}{\mathscr{Y}}
\newcommand{\scrV}{\mathscr{V}}
\newcommand{\scrP}{\mathscr{P}}

\newcommand{\Rep}{\mathop{\mathrm{Rep}}\nolimits}
\newcommand{\alg}{\mathop{\mathrm{alg}}\nolimits}
\newcommand{\Span}{\mathop{\mathrm{Span}}\nolimits}
\newcommand{\proj}{\mathop{\mathrm{proj}}\nolimits}
\newcommand{\Hom}{\mathop{\mathrm{Hom}}\nolimits}
\newcommand{\PHom}{\mathop{\mathrm{PHom}}\nolimits}
\newcommand{\Tw}{\mathop{\mathrm{Tw}}\nolimits}
\newcommand{\Dim}{\mathop{\mathrm{Dim}}\nolimits}
\newcommand{\Imm}{\mathop{\mathrm{Im}}\nolimits}
\newcommand{\Aus}{\mathop{\mathrm{Aus}}\nolimits}
\newcommand{\Ann}{\mathop{\mathrm{Ann}}\nolimits}
\newcommand{\APC}{\mathop{\mathrm{APC}}\nolimits}
\newcommand{\Barr}{\mathop{\mathrm{Bar}}\nolimits}
\newcommand{\Cone}{\mathop{\mathrm{Cone}}\nolimits}
\newcommand{\modu}{\mathop{\mathrm{mod}}\nolimits}
\newcommand{\dgMod}{\mathop{\mathrm{dgMod}}\nolimits}
\newcommand{\soc}{\mathop{\mathrm{soc}}\nolimits}
\newcommand{\dg}{\mathop{\mathrm{dg}}\nolimits}
\newcommand{\red}{\mathop{\mathrm{red}}\nolimits}
\newcommand{\Map}{\mathop{\mathrm{Map}}\nolimits}
\newcommand{\inj}{\mathop{\mathrm{inj}}\nolimits}
\newcommand{\op}{\mathop{\mathrm{op}}\nolimits}
\newcommand{\Ker}{\mathop{\mathrm{Ker}}\nolimits}
\newcommand{\Tor}{\mathop{\mathrm{Tor}}\nolimits}
\newcommand{\Tot}{\mathop{\mathrm{Tot}}\nolimits}
\newcommand{\Ext}{\mathop{\mathrm{Ext}}\nolimits}
\newcommand{\sg}{\mathop{\mathrm{sg}}\nolimits}
\newcommand{\Sl}{\mathop{\mathfrak{sl}}\nolimits}
\newcommand{\nc}{\mathop{\mathrm{nc}}\nolimits}
\newcommand{\Tr}{\mathop{\mathrm{Tr}}\nolimits}
\newcommand{\Irr}{\mathop{\mathrm{Irr}}\nolimits}
\newcommand{\HC}{\mathop{\mathrm{HC}}\nolimits}
\newcommand{\HH}{\mathop{\mathrm{HH}}\nolimits}
\newcommand{\THH}{\mathop{\mathrm{TH}}\nolimits}
\newcommand{\Perf}{\mathop{\mathrm{Perf}}\nolimits}
\newcommand{\del}{\partial}
\newcommand{\ev}{\mathop{\mathrm{ev}}\nolimits}
\newcommand{\co}{\mathop{\mathrm{co}}\nolimits}
\newcommand{\pt}{\mathop{\mathrm{pt}}\nolimits}
\newcommand{\wt}{\mathop{\mathrm{wt}}\nolimits}
\newcommand{\pCY}{\mathop{\mathrm{pCY}}\nolimits}
\newcommand{\gr}{\mathop{\mathrm{gr}}\nolimits}
\newcommand{\coker}{\mathop{\mathrm{coker}}\nolimits}


\newcommand{\Top}{\mathop{\mathrm{Top}}\nolimits}
\newcommand{\Set}{\mathop{\mathrm{Set}}\nolimits}
\newcommand{\Vect}{\mathop{\mathrm{Vect}}\nolimits}
\newcommand{\Mod}{\mathop{\mathrm{Mod}}\nolimits}
\newcommand{\Ob}{\mathop{\mathrm{Ob}}\nolimits}
\newcommand{\Ind}{\mathop{\mathrm{Ind}}\nolimits}
\newcommand{\Sym}{\mathop{\mathrm{Sym}}\nolimits}
\newcommand{\Alt}{\mathop{\mathrm{Alt}}\nolimits}
\newcommand{\End}{\mathop{\mathrm{End}}\nolimits}
\newcommand{\Ch}{\mathop{\mathrm{Ch}}\nolimits}


\newcommand{\Ass}{\mathop{\mathrm{Ass}}\nolimits}
\newcommand{\Com}{\mathop{\mathrm{Com}}\nolimits}
\newcommand{\Lie}{\mathop{\mathrm{Lie}}\nolimits}
\newcommand{\colim}{\mathop{\mathrm{colim}}\nolimits}

\newcommand{\Op}{\mathop{\mathsf{Operads}}\nolimits}
\newcommand{\Diop}{\mathop{\mathsf{Dioperads}}\nolimits}
\newcommand{\PDiop}{\mathop{\mathsf{PDioperads}}\nolimits}
\newcommand{\Coop}{\mathop{\mathsf{Cooperads}}\nolimits}
\newcommand{\Codiop}{\mathop{\mathsf{Codioperads}}\nolimits}
\newcommand{\PCodiop}{\mathop{\mathsf{PCodioperads}}\nolimits}

\newcommand{\antish}{\ensuremath{\mbox{!`}}}


\tikzset{->-/.style={decoration={
			markings,
			mark=at position #1 with {\arrow{>}}},postaction={decorate}}}
\tikzset{-w-/.style={decoration={
			markings,
			mark=at position #1 with {\arrow{Stealth[fill=white,scale=1.4]}}},postaction={decorate}}}
\tikzset{->-/.default=0.65}
\tikzset{-w-/.default=0.65}
\tikzstyle{bullet}=[circle,fill=black,inner sep=0.5mm]
\tikzstyle{circ}=[circle,draw=black,fill=white,inner sep=0.5mm]
\tikzstyle{vertex}=[circle,draw=black,thick,inner sep=0.5mm]
\tikzstyle{dot}=[draw,circle,fill=black,minimum size=0.5mm,inner sep = 0mm, outer sep = 0mm]
\tikzset{darrow/.style={double distance = 4pt,>={Implies},->},
	darrowthin/.style={double equal sign distance,>={Implies},->},
	tarrow/.style={-,preaction={draw,darrow}},
	qarrow/.style={preaction={draw,darrow,shorten >=0pt},shorten >=1pt,-,double,double
		distance=0.2pt}}
\newcommand\tikcirc[1][2.5]{\tikz[baseline=-#1]{\draw[thick](0,0)circle[radius=#1mm];}}
\newcommand{\tikzfig}[1]{\begin{tikzpicture}[auto,baseline={([yshift=-.5ex]current bounding box.center)}]#1\end{tikzpicture}}
\usetikzlibrary{decorations.pathreplacing}

\usepackage{amsthm}
\usepackage{thmtools}
\usepackage[noabbrev,capitalize]{cleveref}
\newtheorem{theorem}{Theorem}
\newtheorem{lemma}[theorem]{Lemma}
\newtheorem{proposition}[theorem]{Proposition}
\newtheorem{corollary}[theorem]{Corollary}
\newtheorem{conjecture}[theorem]{Conjecture}
\newtheorem{Claim}[theorem]{Claim}

\newtheorem*{theorem*}{Theorem}

\theoremstyle{definition}
\newtheorem{definition}{Definition}
\newtheorem{example}{Example}
\newtheorem{cexample}{Counterexample}
\newtheorem{exercise}{Exercise}
\newtheorem{question}{Question}
\newtheorem{remark}{Remark}

\addtolength{\textheight}{1.75cm}
\addtolength{\voffset}{-0.5cm}
\addtolength{\footskip}{+0.6cm}
\geometry{margin=1.2in}


\usepackage[backend=biber, maxbibnames=99, bibencoding=utf8, doi=false, isbn=false, url=false, style=alphabetic]{biblatex}
\addbibresource{refs.bib}

    \begin{document}
\fi



\section{Introduction}\label{section:Introduction}

\nsubsection{Notations}

Throughout this paper, let $X = \{x_0,x_1\}$ be a set of letters, $X^*$ the free monoid generated by $X$ and $\ide{h}:= \mathbb{Q}\langle X\rangle$ be free associative $\mathbb{Q}$-algebra generated by $X$.
We define a weight $\wt w$ on $X^*$ by the number of letters in $w\in X^*$.
The weight is extended to $\ide{h}$. 
Let $\ide{h}=\bigoplus_{m\ge0}\ide{h}^{(m)}$ be a direct sum decomposition with respect to weight, where $\ide{h}^{(m)}$ is the $\mathbb{Q}$-linear subspace of homogeneous elements of weight $m$.
We define $\ide{h}^{\vee}=\varprojlim_{n}\,\ide{h}/\bigl(\bigoplus_{m\ge n}\ide{h}^{(m)}\bigr)$.
It is known that, we can consider $\ide{h}^{\vee}$ as a completed free associative $\mathbb{Q}$-algebra $\mathbb{Q}\langle\langle X \rangle\rangle$ generated by $X$.


%\subsection*{Multiple zeta values and double shuffle relations}
\nsubsection{Multiple zeta values and extended double shuffle relations}


For a tuple of integers $(k_1,\ldots,k_r)\in\mathbb{Z}_{>0}^r$ with $k_r>1$, the \textit{multiple zeta value} (MZV) is a real number defined by
\[
\zeta(k_1,\ldots,k_r) := \sum_{0<m_1<\cdots <m_r} \frac{1}{m_1^{k_1}\cdots m_r^{k_r}}.
\]
We set $\zeta(\emptyset) = 1$.
The condition $k_r>1$ ensures the convergence of the above multiple series. Let $\mathcal{Z}$ be the $\mathbb{Q}$-linear space of MZVs.
When $r=1$, $\zeta(k_1)$ coincides with the special values of the Riemann zeta function.
MZVs were first studied by Euler and Goldbach.
They studied double zeta values (the case $r=2$) and proved the $\Q$-linear relations among MZVs which is known as the sum formula nowadays.
Around 1990, Hoffman \cite{Hoffman-92} and Zagier \cite{Zagier-92} rediscovered MZVs and their applications.
Multiple zeta values can be written as iterated integral expressions \cite{Zagier-94} and these expressions allow us to interpret them as the periods of mixed Tate motives (\cite{Deligne-Goncharov}).
Brown \cite{Brown2012} showed that every period of mixed Tate motives over $\mathbb{Z}$ is a $\Q[(\pi i)^{-1}]$-linear combination of MZVs.

The extended double shuffle relations (Definition \ref{df:eds}) are a family of $\mathbb{Q}$-linear relations among MZVs.
These relations arise from the combination of two kinds of product-to-sum relations, where the first one originates from the iterated integral expression, and the second one originates from the multiple series expressions.
By $\mathcal{Z}^f$, we denote the $\Q$-algebra generated by the formal symbols $\zeta^f(k_1,\ldots,k_r)$ ($r\in\mathbb{Z}_{>0}$, $k_1,\ldots,k_r\in\mathbb{Z}_{>0}$) which satisfy the extended double shuffle relations (the $\Q$-algebra $\mathcal{Z}^f$ was first introduced in \cite{Ihara-Kaneko-Zagier} as $R_{\mathrm{EDS}}$). The $\Q$-algebra $\mathcal{Z}^f$ is called the formal multiple zeta spaces, and its generators $\zeta^f(k_1,\ldots,k_r)$ are called the formal multiple zeta values. 
The $\mathbb{Q}$-algebra $\mathcal{Z}^f$ is equipped with the shuffle product (see Section \ref{section:setup}).
Conjecturally, the map $\mathcal{Z}^f\to\mathcal{Z}$ sending $\zeta^f(k_1,\ldots,k_r)$ to $\zeta^{\shuffle}(k_1,\ldots,k_r)$ (for $r\ge1$, $k_i\in\mathbb{Z}_{>0}$, see Section \ref{section:setup} for $\zeta^{\shuffle}(k_1,\ldots,k_r)$. Note that $\zeta^{\shuffle}(k_1,\ldots,k_r)$ is well defined even$k_r=1$).
) is a $\mathbb{Q}$-algebra isomorphism.
 This conjecture suggests that the extended double shuffle relations generate all $\mathbb{Q}$-algebraic relations among MZVs.

Let $\Set$ denote the category of sets, and $\Qalg$ the category of unital, commutative, and associative $\mathbb{Q}$-algebras.
The extended double shuffle relations are conjectured to generate all $\Q$-linear relations among MZVs.  
From the perspective of the extended double shuffle relations, Racinet \cite{Racinet} introduced the affine scheme $\DMR_0 : \Qalg \rightarrow \Set; (R \mapsto \DMR_0(R))$ whose coordinate ring is $\mathcal{Z}^f/\zeta^f(2)\mathcal{Z}^f$. 
Note that we can regard $\DMR_0(R)$ as the set of elements $\Phi\in R\cotimes \ide{h}^{\vee}$ whose coefficients satisfy the extended double shuffle relations.
For instance, let
\[
\Phi_{\shuffle} := \sum_{w \in X^{*}} \zeta^{\shuffle}(w)w = 1 + \zeta^{\shuffle}(2) x_0x_1  - \zeta^{\shuffle}(2) x_1x_0 + \zeta^{\shuffle}(3)x_0x_0x_1 +  \zeta^{\shuffle}(1,2)x_1x_0x_1+\cdots
\]
and consider $\overline{\Phi}_{\shuffle}$ as the image of $\Phi_{\shuffle}$ in $(\mathcal{Z}/\zeta(2)\mathcal{Z}) \cotimes \ide{h}^{\vee}$. Then $\overline{\Phi}_{\shuffle} \in \DMR_0(\mathcal{Z}/\zeta(2) \mathcal{Z})$.
A remarkable property of $\DMR_0$ is that it possesses a natural structure of an affine group scheme, with a group law given by the Ihara product $\circledast$ (see Section \ref{section:review-on-DMR}).
The group $\DMR_0$ is conjectured to be isomorphic to the prounipotent part of the motivic Galois group $U^{\rm{dR}}$ (\cite[Question~~3.31]{Furusho2}).  
\begin{comment}
Indeed, the prounipotent part $U^{\rm{dR}}$ is also expected to be isomorphic to the Grothendieck-Teichm\"{u}ller group $\mathrm{GRT}$. %Belyi (\cite{Belyi}) 
Combining Brown's theorem \cite{Brown2012} and Ihara's theorem \cite{Ihara}, it shows the existence of the embedding $U^{\rm{dR}} \hookrightarrow \mathrm{GRT}_1$.
Furusho \cite{Furusho} showed $\mathrm{GRT}_1 \hookrightarrow \DMR_0$. Hence $U^{\rm{dR}} \hookrightarrow \DMR_0$.
Moreover, the Kashiwara-Vergne group $\mathrm{KRV}$, which arises from the Kashiwara-Vergne conjecture, is also expected to be isomorphic to $U^{\rm{dR}}$ 
\end{comment}

\nsubsection{Adjoint multiple zeta values}


In this paper, we mainly focus on \textit{adjoint multiple zeta values} (AdMZVs) and the $\Q$-linear relations among them, termed as adjoint double shuffle relations.
These concepts were introduced by Jarossay \cite{Jarossay}.
The purpose of this study is to investigate the adjoint double shuffle relations along with the theory of $\DMR_0$.


For $(k_1,\ldots,k_r) \in \mathbb{Z}_{>0}^r$ and $l \in \mathbb{Z}_{>0}$, we define AdMZVs by
\begin{align*}
\Admzv{l}{k_1,\ldots,k_r} :=& \sum_{i = 0}^r (-1)^{k_{i+1}+\cdots+k_r + l} \zeta^{\shuffle}(k_1,\ldots,k_i)\zeta^{\shuffle}_l(k_r,\ldots,k_{i+1}) \mod \zeta(2),
\end{align*}
where
\[
\zeta_{l}^{\shuffle}(k_r,\ldots,k_{i+1}) = (-1)^l \sum_{\substack{l_{i+1}+\cdots+l_r = l \\ l_{i+1},\ldots,l_r\ge 0}}
\left(\prod_{j = i+1}^r \binom{k_j + l_j+1}{l_j}\right)
\zeta^{\shuffle}(k_r+l_r,\ldots,k_{i+1}+l_{i+1}).
\]
%It is known that $\zeta^{\shuffle}_l$ appears in certain coefficients of $\Phi_{\shuffle}$.
For each $\mathbf{k} \in \mathbb{Z}_{>0}^r$ and $l\in \mathbb{Z}_{>0}$, AdMZVs $\Admzv{l}{\mathbf{k}}$ is a generalizations of the \textit{symmetric multiple zeta values} (SMZVs) \cite[Section 4]{Kan1} $\zeta_{\S}(\mathbf{k})$. Indeed, $\zeta_{\S}(\mathbf{k}) = \Admzv{0}{\mathbf{k}}$ holds.


SMZVs are conjectured to share the same $\Q$-linear relations among finite multiple zeta values $\zeta_{\A}(\mathbf{k})$, referred to as the Kaneko-Zagier conjecture.
Let $P$ be the set of prime numbers, and we define a $\Q$-algebra $\A$ by
\[
\A := \left. \prod_{ p\in P} \mathbb{F}_p\middle/ \bigoplus_{ p\in P}\mathbb{F}_p \right.
\]
Here, $\mathbb{F}_p$ is the $p$-th finite field.
This $\Q$-algebra $\A$ is referred to as the ring of integers modulo infinitely large primes, or ``Poor man's ad\`{e}le ring".
For $(k_1,\ldots,k_r) \in \mathbb{Z}_{>0}^r$, we define a \textit{finite multiple zeta value} (FMZV) $\za(k_1,\ldots,k_r)$ by
\[
\za(k_1,\ldots,k_r) = \left( \sum_{0<m_1<\cdots<m_r < p} \frac{1}{m_1^{k_1}\cdots m_r^{k_r}} \mod p \right)_p \in \A.
\]
Let $\mathcal{Z}^{\A}$ be a $\Q$-algebra generated by $1$ and FMZVs.
Kaneko-Zagier conjecture suggests that there is a well-defined $\Q$-algebra homomorphism.
\begin{align}
\mathcal{Z}^{\A} \rightarrow \mathcal{Z}/\zeta(2)\mathcal{Z} ; \zeta_{\A}(k_1,\ldots,k_r) \mapsto \zeta_{\S}(k_1,\ldots,k_r). \label{eq:Kaneko-Zagier}
\end{align}
Yasuda \cite{Yasuda} showed that SMZVs span the $\Q$-linear space $\mathcal{Z}/\zeta(2)\mathcal{Z}$.
This implies that the above $\Q$-algebra homomorphism \eqref{eq:Kaneko-Zagier} is a surjection under the assumption of its well-definedness.
Recently, the ring $\A$, to which FMZVs belong, has attracted attention.
Rosen first developed an $\A$-analog of periods \cite{Rosen}, and later constructed an $\A$-analog of algebraic numbers \cite{Rosen1}.
Subsequently, Kaneko–Matsusaka–Seki proposed an $\A$-analog of Euler's constant
\cite{Kaneko-Matsusaka-Seki}.


Jarossay studied the element $\overline{\Phi}_{\shuffle}^{-1}x_1\overline{\Phi}_{\shuffle} \in (\mathcal{Z} / \zeta(2)\mathcal{Z}) \cotimes \ide{h}^{\vee}$ and conducted extensive research on $\overline{\Phi}_{\shuffle}^{-1}x_1\overline{\Phi}_{\shuffle}$ in several papers (\cite{Jarossay}, \cite{Jarossay2020}) and found some non-trivial properties.
Indeed, Jarossay obtained the $\Q$-linear relations among AdMZVs referred to as the \textit{adjoint double shuffle relations}, by adopting Racinet's dual formulation of the double shuffle relations to $\overline{\Phi}_{\shuffle}^{-1}x_1\overline{\Phi}_{\shuffle}$.
According to these properties of  $\overline{\Phi}_{\shuffle}^{-1}x_1\overline{\Phi}_{\shuffle}$, Jarossay introduced the \textit{adjoint double shuffle affine scheme} whose defining equation is the adjoint double shuffle relations $\AdDMR_0$.
Related to $\AdDMR_0$, he posed the question, \textit{``Are the two affine schemes, $\DMR_0$ and the adjoint double shuffle scheme, actually isomorphic?"} (\cite[Question 3.2.8]{Jarossay}).
By \cite[Proposition 1.3.6 (i)]{Jarossay2015} and \cite[Proposition 3.2.7]{Jarossay}, there is a closed immersion 
\begin{align}
\Ad(x_1) : \DMR_0 \rightarrow \AdDMR_0 \label{eq:Adjoint-map}
\end{align}
 defined by $\DMR_0(R) \rightarrow \AdDMR_0(R) ; \phi \mapsto \phi^{-1}x_1\phi$ for each $R \in \Qalg$.


\begin{comment}
\textcolor{red}{
Conjecture \ref{conj:Jarossay-1} suggests $U^{\mathrm{dR}} \cong \AdDMR_0\times_{\TM_1} \FAd$ through the conjecture $U^{\mathrm{dR}} \cong \DMR_0$.
This suggests that progress on the relations among AdMZVs, FMZVs, and SMZVs can probably contribute to the isomorphism problem for $U^{\mathrm{dR}}$.
Furthermore, we expect that this will provide a starting point for the Formal Kaneko–Zagier conjecture proposed by Bachmann and Risan \cite{Risan}, which concerns these families of $\Q$-linear relations.
%Furthermore, for a unified understanding of various kinds of multiple zeta values.
}
\end{comment}

\nsubsection{Verification, Question, and our result}

The purpose of this paper is to investigate Jarossay's question mentioned above.
However, computational calculation suggests that the answer is likely to be negative.
To assess this indication, we analyze the tangent space $\dmr$ of $\DMR_0(\Q)$ at $1$ and the tangent space $\addmr$ of $\AdDMR_0(\Q)$ at $x_1$.
Then, the map \eqref{eq:Adjoint-map} induces the following embedding map:
\begin{align}
\ad(x_1) : \dmr\rightarrow \addmr ; \psi \mapsto [x_1,\psi].
\end{align}
We shall consider the direct product decomposition of the weight homogeneous of $\dmr$ and $\addmr$, namely,
\begin{align*}
\dmr =& \prod_{k\ge 0} \dmr^{(k)}\\
\addmr =& \prod_{k\ge 0} \addmr^{(k)}.
\end{align*}
Then we observe that $\dim \dmr^{(k)} \neq \dim \addmr^{(k+1)}$ for $k \in \mathbb{Z}_{\ge 0}$ (see Appendix~\ref{section:computational data}).
(Remineded that the adjoint map increases the weight by $1$).


To refine Jarossay's question, we first define the affine scheme $\FAd:\Qalg\rightarrow\Set$ as follows: for $R\in\Qalg$, set
\[
\FAd(R) := \{ \Phi \in R\cotimes \ide{h}^{\vee} \mid  \text{${}^{\exists} \phi \in \ide{h}^{\vee}$ with $\Delta_{\shuffle}(\phi) = \phi \otimes \phi$ such that  $\Phi = \phi^{-1}x_1 \phi$ }  \}.
\]
Here, $\Delta_{\shuffle} : \ide{h}^{\vee} \rightarrow \ide{h}^{\vee} \cotimes \ide{h}^{\vee}$ is a $\Q$-algebraic homomorphism defined by $x_i \mapsto  x_i \otimes 1 + 1\otimes x_i$ for $i = 0$, $1$. 
For each $R \in \Qalg$, the set $\FAd(R)$ is a group under the operation $\circledast_1$ (see Section \ref{sect:AdDMR}) with a unit $x_1$ and this describes that $\FAd$ is an affine group scheme.
Let $\Fad$ denote the tangent space of $\FAd(\Q)$ at $x_1$.
Then $\Fad$ is a Lie algebra with respect to the bracket $\{\mathchar`-,\mathchar`-\}_1$ (see Section \ref{sect:AdDMR}),
and it can be described explicitly by
\[
\Fad = \{ \phi \in R\cotimes \ide{h}^{\vee} \mid \text{${}^{\exists} \phi \in \ide{h}^{\vee}$ with $\Delta_{\shuffle}(\phi) = \phi \otimes 1 + 1\otimes \phi$ such that  $\Phi = [x_1 ,\phi]$ } \}.
\]

We next consider the fiber product (intersection in the sense of affine schemes) $\AdDMR_0 \times_{\TM_1} \FAd$ (for $\TM_1$, see Section \ref{sect:AdDMR}). We can see that the tangent space of $(\AdDMR_0 \times_{\TM_1} \FAd)(\Q)$ at $x_1$ is equal to $\addmr \cap \Fad$.
Considering the weight homogeneous decomposition of $\dmr = \prod_{k\ge 0} \dmr^{(k)}$ and $\addmr \cap \Fad = \prod_{k\ge 0} \addmr^{(k)} \cap \Fad$, we observe $\dim \dmr^{(k)} = \dim \addmr^{(k+1)} \cap \Fad$ up to $k = 10$ (Appendix~\ref{section:computational data}). 
According to this observation, we expect that the following isomorphism of $\Q$-linear space $\dmr \cong \addmr \cap \Fad$ (Question \ref{qu:Jarossay-Lie})  might be true. If this holds, $ \addmr \cap \Fad$ forms a Lie algebra with respect to $\{\mathchar`-,\mathchar`-\}_1$.
Since $\dmr$ and $\addmr \cap \Fad$ are expected to be pronilpotent Lie algebras, the correspondence between
prounipotent affine group schemes and pronilpotent Lie algebras further suggest that the following isomorphism of affine schemes
$\DMR_0 \cong \AdDMR_0 \times_{\TM_1} \FAd$ would be true (Question \ref{qu:Jarossay}). If this holds, $ \AdDMR_0 \times_{\TM_1} \FAd$ forms an affine group scheme with respect to $\circledast_1$.


According to this observation, we obtain partial positive results toward Question \ref{qu:Jarossay-Lie}.
We utilize the following $\Q$-linear space:
\begin{align*}
V_{\strprty}:=& \left\{\Psi \in \ide{h}^{\vee} \middle|
\begin{matrix}
\text{$\langle \Psi \mid w \rangle = 0$ for $w \in X^*$ with $\wt w\le 1$}\\
 \Psi^{11} + \Psi^{10} + \Psi^{01} = 0
\end{matrix}
\right\},
\end{align*}
where we define
\[
\Psi = x_0\,\Psi^{00}\,x_0+ x_0\,\Psi^{01}\,x_1+ x_1\,\Psi^{10}\,x_0+ x_1\,\Psi^{11}\,x_1 .
\]
for $\Psi \in \ide{h}^{\vee}$ such that the coefficients of $1$, $x_0$, and $x_1$ are $0$.
The defining equations of $V_{\strprty}$ are related to explicit parity formulas \cite{Hirose-parity}.
Further details on $V_{\strprty}$ are given in Section \ref{sect:Main Theorem}.
Below, we describe a sketch of our main theorem.
\begin{mt}[simplified version]\label{mt:Lie-str}
An intersection of $\Q$-linear space $\addmr \cap \Fad \cap V_{\strprty}$ forms a Lie algebra equipped with a certain bracket $\{\mathchar`-,\mathchar`-\}_1$ (see Section \ref{sect:AdDMR}).
\end{mt}

It was previously unknown whether the relations among AdMZVs, FMZVs, or SMZVs possess a natural algebraic structure.
Our main theorem provides the first evidence that one can obtain a nontrivial algebraic structure from the $\Q$-linear relations among AdMZVs.



\begin{comment}
\nsubsection{Adjoint multiple zeta values and adjoint double shuffle relations}

Inspired by the Kaneko-Zagier conjecture, the ring $\mathscr{A}$, in which finite multiple zeta values are defined, is expected to be an analogue of $\mathbb{R}$.
Indeed, Rosen \cite{Rosen1} constructed $\mathscr{A}$-analogues of periods and algebraic numbers.
Kaneko, Matsusaka, and Seki \cite{Kaneko-Matsusaka-Seki} proposed an $\mathscr{A}$-analogue of Euler's constant.
The author \cite{Anzawa-Funakura} has also investigated $\mathscr{A}$ and constructed transcendental elements of $\mathscr{A}$ under the generalized Riemann hypothesis (GRH) by using the $q$-Fibonacci sequence.
Recently, Luca and Zudilin \cite{Luca-Zudilin} refined the main theorem of \cite{Anzawa-Funakura}.
They removed the GRH assumption and constructed another transcendental element of $\mathscr{A}$ from Frobenius traces of elliptic curves (\cite{Luca-Zudilin-2}).
\end{comment}






\begin{comment}
\nsubsection{今後の展望}

Our investigation predicts three works.
First is to solve the formal Kaneko-Zagier conjecture.

Second is to determine the $\Q$-linear relations which


\textcolor{red}{ここから書き直し}

\nsubsection{Interpretation of $\mathcal{Z}/\zeta(2)\mathcal{Z}$ as a coordinate ring of some algebraic groups}


The motivic Galois group $G^{\rm{dr}}$ is an affine group scheme over $\Q$ defined as the Tannakian fundamental group of the category of the mixed Tate motives over $\Z$. 
It is known that, via the embedding $G^{\mathrm{dR}}\hookrightarrow \ide{h}^{\vee}$, the group law on $G^{\mathrm{dR}}$ corresponds to the Ihara product $\circledast$ (see Section \ref{section:review-on-DMR}).
Moreover, there is a canonical semidirect product decomposition
\[
G^{\mathrm{dR}} \cong U^{\mathrm{dR}} \rtimes \mathbb{G}_m.
\]
Here, $\mathbb{G}_m$ denotes the multiplicative group scheme. The closed pro-unipotent subgroup scheme $U^{\mathrm{dR}}\subset G^{\mathrm{dR}}$ is called the pro-unipotent part of $G^{\mathrm{dR}}$.
Deligne and Goncharov constructed a motivic lift of MZVs from mixed Tate motives over $\mathbb{Z}$ \cite{Deligne-Goncharov}.
Let $\mathcal{Z}^{\ide{m}}$ denote the $\mathbb{Q}$-linear space spanned by motivic MZVs. We note that $\mathcal{Z}^{\ide{m}}$ is a $\mathbb{Q}$-algebra equipped with the shuffle product, and there is a natural surjection $\mathcal{Z}^{\ide{m}} \twoheadrightarrow \mathcal{Z}$ (conjecturally, a $\mathbb{Q}$-algebra isomorphism).
Then the coordinate ring of $U^{\mathrm{dR}}$ is $\mathcal{Z}^{\ide{m}}/(\zeta^{\ide{m}}(2))$, where $\zeta^{\ide{m}}(2)\in\mathcal{Z}^{\ide{m}}$ denotes the motivic element sent to $\zeta(2)$ by the natural surjection $\mathcal{Z}^{\ide{m}}\twoheadrightarrow\mathcal{Z}$.
Therefore, $\mathcal{Z}^{\ide{m}}/\zeta^{\ide{m}}(2)\mathcal{Z}^{\ide{m}}$ possesses the structure of Hopf algebra and $\mathcal{Z}^{\ide{m}}$ possesses the structure of the ($\mathcal{Z}^{\ide{m}}/\zeta^{\ide{m}}(2)\mathcal{Z}^{\ide{m}}$)-Hopf comodule.
The coaction on $\mathcal{Z}^{\ide{m}}$, known as the Goncharov--Brown coaction, plays a key role in Brown's proof that every period of mixed Tate motives over $\mathbb{Z}$ is generated, over $\mathbb{Q}[(2\pi i)^{-1}]$, by a specific set of motivic MZVs \cite{Brown2012}, \cite{Goncharov2}.


Let $\Set$ denote the category of sets, and $\Qalg$ the category of unital, commutative, and associative $\mathbb{Q}$-algebras.
The extended double shuffle relations are conjectured to generate all $\Q$-linear relations among MZVs.  
From the perspective of the extended double shuffle relations, Racinet (\cite{Racinet}) introduced the affine scheme $\DMR_0 : \Qalg \rightarrow \Set; (R \mapsto \DMR_0(R))$ whose coordinate ring is $\mathcal{Z}^f/\zeta^f(2)\mathcal{Z}^f$. 
Note that we can regard $\DMR_0(R)$ as the set of elements $\Phi\in R\cotimes \ide{h}^{\vee}$ whose coefficients satisfy the extended double shuffle relations.
For instance, let
\[
\Phi_{\shuffle} := \sum_{w \in X^{*}} \zeta^{\shuffle}(w) = 1 + \zeta^{\shuffle}(2) x_0x_1  - \zeta^{\shuffle}(2) x_1x_0 + \zeta^{\shuffle}(3)x_0x_0x_1 +  \zeta^{\shuffle}(1,2)x_1x_0x_1+\cdots
\]
and consider $\overline{\Phi}_{\shuffle}$ as the image of $\Phi_{\shuffle}$ in $(\mathcal{Z}/\zeta(2)\mathcal{Z}) \cotimes \ide{h}^{\vee}$. Then $\overline{\Phi}_{\shuffle} \in \DMR_0(\mathcal{Z}/\zeta(2) \mathcal{Z})$.
A remarkable property of $\DMR_0$ is that it possesses a natural structure of an affine group scheme, with group law given by the Ihara product $\circledast$ (see Section \ref{section:review-on-DMR}).
The group $\DMR_0$ is conjectured to be isomorphic to the motivic Galois group $U^{\rm{dr}}$.
Indeed, the pro-unipotent part $U^{\rm{dr}}$ is also expected to be isomorphic to the Grothendieck-Teichm\"{u}ller group $\mathrm{GRT}$. %Belyi (\cite{Belyi}) 
Combining Brown's theorem \cite{Brown2012} and Ihara's theorem \cite{Ihara} shows the existence of the embedding $U^{\rm{dr}} \hookrightarrow \mathrm{GRT}_1$.
Furusho (\cite{Furusho}) showed $\mathrm{GRT}_1 \hookrightarrow \DMR_0$. Hence $U^{\rm{dr}} \hookrightarrow \DMR_0$.
Moreover, the Kashiwara-Vergne group $\mathrm{KRV}$, which arises from the Kashiwara-Vergne conjecture, is also predicted to be isomorphic to $U^{\rm{dr}}$ (\cite[Question 3.31]{Furusho2}).  

%\subsection*{$\mathcal{Z}/\zeta(2)\mathcal{Z}$ and Kaneko--Zagier conjecture}

\nsubsection{$\mathcal{Z}/\zeta(2)\mathcal{Z}$ and Kaneko-Zagier conjecture}

In connection with the quotient $\mathbb{Q}$-algebra $\mathcal{Z}/\zeta(2) \mathcal{Z}$, the Kaneko--Zagier conjecture has attracted attention since around 2010. 
Let $t$ be an indeterminate.
For $l\in\mathbb{Z}_{>0}$, the \textbf{(generalized) Kaneko--Zagier conjecture} (\cite{Kan1}) predicts the existence of a well-defined $\mathbb{Q}$-algebra isomorphism such that
\begin{align}
\begin{array}{ccc}
\mathcal{Z}^{\mathscr{A}_l} &\longrightarrow &  \mathcal{Z}/\zeta(2) \mathcal{Z} \otimes \mathbb{Q}[t]/(t^l)  \\
\rotatebox{90}{$\in$}&& \rotatebox{90}{$\in$}\\
\zeta_{\mathscr{A}_l} ( \mathbf{k})  & \longmapsto & \zeta_{\S_l}(\mathbf{k})\\
\mathbf{p}& \longmapsto &t
\end{array}
\label{eq:KZ-conj}
\end{align}
Here, given $(k_1,\ldots,k_r)\in\mathbb{Z}_{>0}^r$ and $l \in \mathbb{Z}_{> 0}$, $l$-\textbf{finite multiple zeta values} ($l$-FMZVs) are defined by
\[
\zeta_{\A_l}(k_1,\ldots,k_r) := \left( \sum_{0<m_1<\cdots<m_r<p} \frac{1}{m_1^{k_1}\cdots m_r^{k_r}} \mod p^l \right)_p \in \A_l,
\]
where
\[
\mathscr{A}_l := \prod_{p\text{ : prime}} \mathbb{Z}/p^l \mathbb{Z} \Bigm/ \bigoplus_{p \text{ : prime}} \mathbb{Z}/p^l\mathbb{Z}.
\]
We define $\mathcal{Z}^{\A_l}$ as the $\Q$-algebra generated by $l$-FMZVs and $\mathbf{p}:= (2,3,5,7,\ldots) \in \A$.

The counterparts of FMZVs, in the context of the Kaneko-Zagier conjecture \eqref{eq:KZ-conj}, are $l$-\textbf{symmetric multiple zeta values} ($l$-SMZVs) given by 
\begin{align*}
\zeta_{\S_l}(k_1,\ldots,k_r) := \sum_{0\le l' < l} \Admzv{l'}{k_1,\ldots,k_r} \otimes t^{l'}
\end{align*}
Here, for $l\in\mathbb{Z}_{>0}$, we define \textbf{adjoint multiple zeta values} (AdMZVs), $\Admzv{l}{k_1,\ldots,k_r}$ by
\begin{align*}
\Admzv{l}{k_1,\ldots,k_r} :=& \sum_{i = 0}^r (-1)^{k_{i+1}+\cdots+k_r + l} \zeta^{\shuffle}(k_1,\ldots,k_i)\zeta^{\shuffle}_l(k_r,\ldots,k_{i+1}) \mod \zeta(2),
\end{align*}
where
\[
\zeta_{l}^{\shuffle}(k_r,\ldots,k_{i+1}) = (-1)^l \sum_{\substack{l_{i+1}+\cdots+l_r = l \\ l_{i+1},\ldots,l_r\ge 0}}
\left(\prod_{j = i+1}^r \binom{k_j + l_j+1}{l_j}\right)
\zeta^{\shuffle}(k_r+l_r,\ldots,k_{i+1}+l_{i+1}).
\]
It is known that $\zeta^{\shuffle}_l$ appears in certain coefficients of $\Phi_{\shuffle}$.
If $l = 1$, we simply call $l$-FMZVs (respectively, $l$-SMZVs) as FMZVs (respectively, SMZVs).

These concepts were first introduced by Kaneko and Zagier (\cite{Kan1}) for $l = 1$ and by Rosen (\cite{Rosen}) for $l \ge 2$.
Thus, according to the Kaneko--Zagier conjecture~\eqref{eq:KZ-conj}, it is expected that studying congruences for the truncation of MZVs can contribute to the research on MZVs.
This motivates the study of FMZVs and SMZVs. 
In particular, FMZVs and SMZVs have been extensively studied by many researchers  (\cite{Hirose}, \cite{Rosen}, \cite{BTT}, etc). 
Indeed, Yasuda (\cite{Yasuda}) proved that SMZVs span $\mathcal{Z}$.
If \eqref{eq:KZ-conj} is an well-defined $\Q$-algebra homomorphism, then Yasuda's theorem implies that \eqref{eq:KZ-conj} is surjective.
Besides, while the definition of SMZVs is defined in terms of MZVs, Akagi--Hirose--Yasuda conjectured that FMZVs admit explicit expressions in terms of $p$-adic MZVs; this was proved by Jarossay \cite{Jarossay2020}. Building on Jarossay’s theorem, Rosen \cite{Rosen} constructed a motivic version of multiple harmonic sums.
The ring $\mathscr{A}_1$, in which finite multiple zeta values are defined, is motivated by the Kaneko--Zagier conjecture and is expected to serve as a finite analogue of $\mathbb{R}$.
Indeed, Rosen constructed $\mathscr{A}_1$-analogues of periods and of algebraic numbers \cite{Rosen1}.
Kaneko, Matsusaka, and Seki proposed an $\mathscr{A}_1$-analogue of Euler's constant \cite{Kaneko-Matsusaka-Seki}.
The author has also investigated the structure of $\mathscr{A}_1$. By using the $q$-Fibonacci sequence, he constructed transcendental elements of $\mathscr{A}_1$ under GRH \cite{Anzawa-Funakura}.
Recently, Luca and Zudilin extended the main theorem of \cite{Anzawa-Funakura}.
They removed the GRH assumption and constructed another transcendental element of $\mathscr{A}_1$ from Frobenius traces of elliptic curves (\cite{Luca-Zudilin},\cite{Luca-Zudilin-2}).

\nsubsection{The adjoint double shuffle relations}


Jarossay studied the occurrence of AdMZVs\footnote{In Jarossay's terminology, he called them adjoint MZVs. However, the name ``SMZVs" is spread and similar to many researchers. So we rename them.} in the coefficients of $\overline{\Phi}_{\shuffle}^{-1}x_1\overline{\Phi}_{\shuffle}$.
He conducted extensive research on AdMZVs in several papers (\cite{Jarossay}, \cite{Jarossay2020}) and found some non-trivial conditions.
One of his results was the proposal of the adjoint double shuffle relations.
The conditions dual to the extended double shuffle relations can be phrased as two coproduct conditions for $\overline{\Phi}._{\shuffle}$
Based on this, he derived two properties of $\overline{\Phi}_{\shuffle}^{-1}x_1\overline{\Phi}_{\shuffle}$ from the dual conditions of the extended double shuffle relations.
From these, he deduced two identities of $\overline{\Phi}_{\shuffle}^{-1}x_1\overline{\Phi}_{\shuffle}$, and the found $\mathbb{Q}$-linear relations among AdMZVs are called the adjoint double shuffle relations.
Jarossay also proposes the \textbf{adjoint double shuffle affine scheme}, which arises from the adjoint double shuffle relations, and he posed the questions, \textit{Are the two affine schemes, $\DMR_0$ and the adjoint double shuffle scheme, actually isomorphic ?}
Computational calculation suggests that this question is likely not true. However, imposing a natural assumption, termed the adjoint condition, we establish the conjecture from Jarossay's conjecture. 
Let $\FAd$ denote the algebraic group defined by the adjoint condition.

\begin{conj}\label{conj:Jarossay-1}
There is a natural isomorphism $\Ad(x_1) : \DMR_0 \Longrightarrow \AdDMR_0 \times_{\TM_1} \FAd$ ; $(\Ad(x_1)^R : \DMR_0(R) \rightarrow \AdDMR_0(R) \cap \Fad(R) ; \phi \mapsto \phi^{-1}x_1\phi$.
\end{conj}



In this paper, we report a partial and positive result about this conjecture.


\nsubsection{Results}

To consider Jarossay's question, we focus on a tangent space $\addmr$ of $\AdDMR_0(\Q)$ at the unit because the tangential space of an affine group scheme equips a structure of Lie algebra.
For our main theorems, we utilize the following $\Q$-linear spaces
\begin{align*}
V_{\strprty}:=& \left\{\Psi \in \ide{h}^{\vee} \middle|
\begin{matrix}
 \wt \Psi \ge 2 \\
 \Psi^{11} + \Psi^{10} + \Psi^{01} = 0
\end{matrix}
\right\}.
\end{align*}
The details of this space are in Section \ref{sect:Main Theorem}.
%Related to these two $\Q$-linear spaces, we construct affine group schemes $\exp(\Fad)$ and $\exp(V_{\strprty})$.
Then, we obtain the following main theorem:
\begin{mt}\label{mt:Lie-str}
Let $\Fad$ be a corresponding Lie algebra of $\FAd$. Then the following $\addmr \cap \Fad \cap V_{\strprty}$ forms a Lie algebra with respect to a specific operation.
\end{mt}

\end{comment}


\ifdefined\isMainDocument
\else
    %\bibliographystyle{amsplain}
    %\bibliography{reference-adjoint}
    \end{document}
\fi

\ifdefined\isMainDocument
\else
    \documentclass[10pt,oneside,reqno]{amsart}
    \usepackage{xr}
    \externaldocument{../main} % Main の .aux を参照
    \usepackage{latexsym}
\usepackage{amscd}
\usepackage{amsmath}
\usepackage{amssymb}
\usepackage[mathscr]{euscript}
\usepackage{graphicx}
\usepackage{geometry}
\usepackage{mathtools}
\usepackage[all]{xy}
\usepackage[dvipsnames]{xcolor}
\usepackage[colorlinks,final,hyperindex]{hyperref}
\usepackage{tikz}
\usepackage{tikz-cd}
\usetikzlibrary{decorations.pathmorphing,decorations.markings,arrows,calc,shapes.geometric,arrows.meta,positioning}
\usepackage{enumerate}
\usepackage{multirow}

\usepackage[bb=dsserif]{mathalpha}
\usepackage{bm}

\newcommand{\QQ}{\ensuremath{\mathbb{Q}}}
\newcommand{\DD}{\ensuremath{\mathbb{D}}}
\newcommand{\CC}{\ensuremath{\mathbb{C}}}
\newcommand{\RR}{\ensuremath{\mathbb{R}}}
\newcommand{\ZZ}{\ensuremath{\mathbb{Z}}}
\newcommand{\NN}{\ensuremath{\mathbb{N}}}

\newcommand{\g}{\ensuremath{\mathfrak{g}}}
\renewcommand{\SS}{\ensuremath{\mathbb{S}}}

\newcommand{\id}{\ensuremath{\mathrm{id}}}

\makeatletter

\newcommand{\bbone}{\mathbb{1}}

\newcommand{\calA}{\mathcal{A}}
\newcommand{\calB}{\mathcal{B}}
\newcommand{\calC}{\mathcal{C}}
\newcommand{\calD}{\mathcal{D}}
\newcommand{\calE}{\mathcal{E}}
\newcommand{\calF}{\mathcal{F}}
\newcommand{\calG}{\mathcal{G}}
\newcommand{\calH}{\mathcal{H}}
\newcommand{\calI}{\mathcal{I}}
\newcommand{\calJ}{\mathcal{J}}
\newcommand{\calK}{\mathcal{K}}
\newcommand{\calL}{\mathcal{L}}
\newcommand{\calM}{\mathcal{M}}
\newcommand{\calN}{\mathcal{N}}
\newcommand{\calO}{\mathcal{O}}
\newcommand{\calP}{\mathcal{P}}
\newcommand{\calQ}{\mathcal{Q}}
\newcommand{\calR}{\mathcal{R}}
\newcommand{\calS}{\mathcal{S}}
\newcommand{\calT}{\mathcal{T}}
\newcommand{\calU}{\mathcal{U}}
\newcommand{\calV}{\mathcal{V}}
\newcommand{\calW}{\mathcal{W}}
\newcommand{\calX}{\mathcal{X}}
\newcommand{\calY}{\mathcal{Y}}
\newcommand{\calZ}{\mathcal{Z}}
\newcommand{\kk}{\Bbbk}

\newcommand{\scrY}{\mathscr{Y}}
\newcommand{\scrV}{\mathscr{V}}
\newcommand{\scrP}{\mathscr{P}}

\newcommand{\Rep}{\mathop{\mathrm{Rep}}\nolimits}
\newcommand{\alg}{\mathop{\mathrm{alg}}\nolimits}
\newcommand{\Span}{\mathop{\mathrm{Span}}\nolimits}
\newcommand{\proj}{\mathop{\mathrm{proj}}\nolimits}
\newcommand{\Hom}{\mathop{\mathrm{Hom}}\nolimits}
\newcommand{\PHom}{\mathop{\mathrm{PHom}}\nolimits}
\newcommand{\Tw}{\mathop{\mathrm{Tw}}\nolimits}
\newcommand{\Dim}{\mathop{\mathrm{Dim}}\nolimits}
\newcommand{\Imm}{\mathop{\mathrm{Im}}\nolimits}
\newcommand{\Aus}{\mathop{\mathrm{Aus}}\nolimits}
\newcommand{\Ann}{\mathop{\mathrm{Ann}}\nolimits}
\newcommand{\APC}{\mathop{\mathrm{APC}}\nolimits}
\newcommand{\Barr}{\mathop{\mathrm{Bar}}\nolimits}
\newcommand{\Cone}{\mathop{\mathrm{Cone}}\nolimits}
\newcommand{\modu}{\mathop{\mathrm{mod}}\nolimits}
\newcommand{\dgMod}{\mathop{\mathrm{dgMod}}\nolimits}
\newcommand{\soc}{\mathop{\mathrm{soc}}\nolimits}
\newcommand{\dg}{\mathop{\mathrm{dg}}\nolimits}
\newcommand{\red}{\mathop{\mathrm{red}}\nolimits}
\newcommand{\Map}{\mathop{\mathrm{Map}}\nolimits}
\newcommand{\inj}{\mathop{\mathrm{inj}}\nolimits}
\newcommand{\op}{\mathop{\mathrm{op}}\nolimits}
\newcommand{\Ker}{\mathop{\mathrm{Ker}}\nolimits}
\newcommand{\Tor}{\mathop{\mathrm{Tor}}\nolimits}
\newcommand{\Tot}{\mathop{\mathrm{Tot}}\nolimits}
\newcommand{\Ext}{\mathop{\mathrm{Ext}}\nolimits}
\newcommand{\sg}{\mathop{\mathrm{sg}}\nolimits}
\newcommand{\Sl}{\mathop{\mathfrak{sl}}\nolimits}
\newcommand{\nc}{\mathop{\mathrm{nc}}\nolimits}
\newcommand{\Tr}{\mathop{\mathrm{Tr}}\nolimits}
\newcommand{\Irr}{\mathop{\mathrm{Irr}}\nolimits}
\newcommand{\HC}{\mathop{\mathrm{HC}}\nolimits}
\newcommand{\HH}{\mathop{\mathrm{HH}}\nolimits}
\newcommand{\THH}{\mathop{\mathrm{TH}}\nolimits}
\newcommand{\Perf}{\mathop{\mathrm{Perf}}\nolimits}
\newcommand{\del}{\partial}
\newcommand{\ev}{\mathop{\mathrm{ev}}\nolimits}
\newcommand{\co}{\mathop{\mathrm{co}}\nolimits}
\newcommand{\pt}{\mathop{\mathrm{pt}}\nolimits}
\newcommand{\wt}{\mathop{\mathrm{wt}}\nolimits}
\newcommand{\pCY}{\mathop{\mathrm{pCY}}\nolimits}
\newcommand{\gr}{\mathop{\mathrm{gr}}\nolimits}
\newcommand{\coker}{\mathop{\mathrm{coker}}\nolimits}


\newcommand{\Top}{\mathop{\mathrm{Top}}\nolimits}
\newcommand{\Set}{\mathop{\mathrm{Set}}\nolimits}
\newcommand{\Vect}{\mathop{\mathrm{Vect}}\nolimits}
\newcommand{\Mod}{\mathop{\mathrm{Mod}}\nolimits}
\newcommand{\Ob}{\mathop{\mathrm{Ob}}\nolimits}
\newcommand{\Ind}{\mathop{\mathrm{Ind}}\nolimits}
\newcommand{\Sym}{\mathop{\mathrm{Sym}}\nolimits}
\newcommand{\Alt}{\mathop{\mathrm{Alt}}\nolimits}
\newcommand{\End}{\mathop{\mathrm{End}}\nolimits}
\newcommand{\Ch}{\mathop{\mathrm{Ch}}\nolimits}


\newcommand{\Ass}{\mathop{\mathrm{Ass}}\nolimits}
\newcommand{\Com}{\mathop{\mathrm{Com}}\nolimits}
\newcommand{\Lie}{\mathop{\mathrm{Lie}}\nolimits}
\newcommand{\colim}{\mathop{\mathrm{colim}}\nolimits}

\newcommand{\Op}{\mathop{\mathsf{Operads}}\nolimits}
\newcommand{\Diop}{\mathop{\mathsf{Dioperads}}\nolimits}
\newcommand{\PDiop}{\mathop{\mathsf{PDioperads}}\nolimits}
\newcommand{\Coop}{\mathop{\mathsf{Cooperads}}\nolimits}
\newcommand{\Codiop}{\mathop{\mathsf{Codioperads}}\nolimits}
\newcommand{\PCodiop}{\mathop{\mathsf{PCodioperads}}\nolimits}

\newcommand{\antish}{\ensuremath{\mbox{!`}}}


\tikzset{->-/.style={decoration={
			markings,
			mark=at position #1 with {\arrow{>}}},postaction={decorate}}}
\tikzset{-w-/.style={decoration={
			markings,
			mark=at position #1 with {\arrow{Stealth[fill=white,scale=1.4]}}},postaction={decorate}}}
\tikzset{->-/.default=0.65}
\tikzset{-w-/.default=0.65}
\tikzstyle{bullet}=[circle,fill=black,inner sep=0.5mm]
\tikzstyle{circ}=[circle,draw=black,fill=white,inner sep=0.5mm]
\tikzstyle{vertex}=[circle,draw=black,thick,inner sep=0.5mm]
\tikzstyle{dot}=[draw,circle,fill=black,minimum size=0.5mm,inner sep = 0mm, outer sep = 0mm]
\tikzset{darrow/.style={double distance = 4pt,>={Implies},->},
	darrowthin/.style={double equal sign distance,>={Implies},->},
	tarrow/.style={-,preaction={draw,darrow}},
	qarrow/.style={preaction={draw,darrow,shorten >=0pt},shorten >=1pt,-,double,double
		distance=0.2pt}}
\newcommand\tikcirc[1][2.5]{\tikz[baseline=-#1]{\draw[thick](0,0)circle[radius=#1mm];}}
\newcommand{\tikzfig}[1]{\begin{tikzpicture}[auto,baseline={([yshift=-.5ex]current bounding box.center)}]#1\end{tikzpicture}}
\usetikzlibrary{decorations.pathreplacing}

\usepackage{amsthm}
\usepackage{thmtools}
\usepackage[noabbrev,capitalize]{cleveref}
\newtheorem{theorem}{Theorem}
\newtheorem{lemma}[theorem]{Lemma}
\newtheorem{proposition}[theorem]{Proposition}
\newtheorem{corollary}[theorem]{Corollary}
\newtheorem{conjecture}[theorem]{Conjecture}
\newtheorem{Claim}[theorem]{Claim}

\newtheorem*{theorem*}{Theorem}

\theoremstyle{definition}
\newtheorem{definition}{Definition}
\newtheorem{example}{Example}
\newtheorem{cexample}{Counterexample}
\newtheorem{exercise}{Exercise}
\newtheorem{question}{Question}
\newtheorem{remark}{Remark}

\addtolength{\textheight}{1.75cm}
\addtolength{\voffset}{-0.5cm}
\addtolength{\footskip}{+0.6cm}
\geometry{margin=1.2in}


\usepackage[backend=biber, maxbibnames=99, bibencoding=utf8, doi=false, isbn=false, url=false, style=alphabetic]{biblatex}
\addbibresource{refs.bib}

    \begin{document}
\fi

\setcounter{section}{1}
\section{Algebraic setup}\label{section:setup}

This section treats algebraic notation and recalls the duality theory for noncommutative formal power series, together with Hoffman's shuffle and harmonic framework.
Let $\cat{Set}$ be the category of sets, $\Qalg$ the category of unital, commutative, and associative $\mathbb{Q}$-algebras,  and $\cat{Grp}$ the category of groups.
Let $\Lalg$ be the category of Lie algebras, which does not fix the coefficient ring.

Throughout this paper, let $R \in \Qalg$.

\nsubsection{Notation of affine group schemes}

In this subsection, let us recall affine schemes. 
We refer to  \cite{Demazure-Gabriel}, \cite{Milne}, and \cite{Waterhouse} for the notions of affine group schemes.
An affine scheme over $\mathbb{Q}$ is a functor $X:\Qalg\to\Set$ that is naturally isomorphic to a representable functor.
An affine group scheme is an affine scheme whose value of $R \in \Qalg$ possesses a group structure.
For an affine scheme $F$, its coordinate ring $\mathcal{O}(F)$ is defined as the $\mathbb{Q}$-algebra representing a functor naturally isomorphic to $F$.


Let $F_1$ and $F_2$ be affine schemes.
A morphism between affine schemes $\tau : F_1 \rightarrow F_2$ is a natural transformation $(\tau^R: F_1(R) \rightarrow F_2(R))_{R\in \Qalg}$.
We say that $F_1$ is a closed affine subscheme of $F_2$ if there exists a surjective $\Q$-algebra morphism $\mathcal{O}(F_2) \twoheadrightarrow \mathcal{O}(F_1)$.
By Yoneda's lemma, this definition is equivalent to the existence of a natural transformation $(\tau^R: F_1(R) \rightarrow F_2(R))_{R\in \Qalg}$ such that $\tau^R$ is injective for each $R\in \Qalg$.
%Let $X$ be an affine scheme that contains $F_1$ and $F_2$ as closed affine subschemes.
%Then we define an intersection $F_1 \cap_{X} F_2$ of $F_1$ and $F_2$ relative to $X$ as the fibre product $F_1 \times_X F_2$.
%When $X$ is clear, we simply write $F_1\cap F_2$ for $F_1\cap _X F_2$,

For a $\Q$-linear space $V$, by $V_{\ide{a}}$, we denotes a functor $\Qalg \rightarrow \Set ; R \mapsto R \otimes V$.
Abusing the notation, if $V$ is the inverse limit $\varprojlim_{n}  V_n$, we define a functor $V_{\ide{a}} := \Qalg \rightarrow \Set ;  R \mapsto  \varprojlim R \otimes V_n (= R\cotimes V)$, where the completed tensor product is taken with respect to the inverse system on $V$.

In general, for an affine group scheme $G$, there uniquely exists the Lie algebra $\ide{g}$ defined by
\[
\ide{g} := \ker (G(\Q[\varepsilon])\rightarrow G(\Q)).
\]

Here, $\varepsilon$ is a parameter satisfying $\varepsilon^2 = 0$ and the above map $\Q[\varepsilon] \rightarrow \Q$ is given by $ a+ \varepsilon b \mapsto a$ for $a$, $b \in \mathbb{Q}$.
%There is no established method for explicitly describing the Lie bracket of $g(R)$. 
%However, in most cases, the Lie bracket is determined by the representation of the Lie algebra induced by the representation of the affine group scheme.
We call $\ide{g}$ a corresponding Lie algebra of the affine group scheme $G$.
%For example, for $\Q$-linear space $V$, a functor $\GL(V) : R \rightarrow R\otimes \GL(R\otimes V)$ forms an affine group schemes and its correspoinding Lie algebra is $\ide{gl}(V) : R \rightarrow R\otimes \mathrm{End}(R\otimes V)$.

\nsubsection{(Pro)unipotent affine group schemes}

Let $G$ be an affine group scheme. 
We say that $G$ is unipotent if there exists a faithful linear representation $\rho : G\rightarrow \GL(V)$ on some $\mathbb{Q}$-linear space $V$ such that the following holds:
\begin{itemize}
\item V contains a finite flag $V = V_0 \supset V_1\supset \cdots \supset V_n = \{0\}$,
\item For $\mathbb{Q}$-algebra $R$, $\rho^R(G(R))(R \otimes V_i) \subset R \otimes V_i$, and
\item For $\mathbb{Q}$-algebra $R$, the action of $G(R)$ on $R \otimes (V_i/V_{i+1})$ is trivial.
\end{itemize}
An affine group scheme $G$ is prounipotent if $G$ is an inverse limit of some unipotent affine group schemes.


\begin{comment}
Let $\ide{g}$ be the corresponding Lie algebra of (pro)unipotent affine group scheme $G$ and $(V,\rho_V)$ be a (finite) dimensional representation of $G$.
This representation defines a representation of the Lie algebra $d\rho_V : \ide{g} \rightarrow \ide{gl}(V)$.

\begin{prop}[{\ref{enumi:unipotent-2} = \cite[IV, Section 2, no 4]{Demazure-Gabriel}}]\label{prop:unipotent}
\begin{enumerate}
\item\label{enumi:unipotent-0}
Let $G$ be an affine group scheme and let $\ide{g}$ be a corresponding Lie algebra. 
Then, for $x \in \ide{g}(R)$, there uniquely exists $\exp(tx) \in G(R[[t]])$ such that $\exp(tx)$ satisfies some properties.
\item\label{enumi:unipotent-1}
For a representation $(V, \rho_V)$ of (pro)unipotent group $G$ on some $\Q$-linear space $V$, there exists a natural isomorphism $\exp^G := (\ide{g}(R) \rightarrow G(R))_{R \in \Qalg}$ such that the following diagram
\[
\xymatrix{
G \ar[r]^{\rho_V} & \GL(V) \\
\ide{g} \ar[u]^{\exp^G} \ar[r]_{d\rho_V} &\ide{gl}(V) \ar[u]_{\exp}
}
\]
commutes. Here $\exp : \ide{gl}(V) \rightarrow \GL(V)$ on the right vertical line above is the usual exponential map.
\item\label{enumi:unipotent-2} If $G$ is a prounipotent group and $\ide{h}$ is a closed Lie subalgebra of the corresponding Lie algebra $\ide{g}$ of $G$, then, $\exp(\ide{h})$ is a closed prounipotent affine group subscheme of $G$.
\end{enumerate}
\end{prop}
\end{comment}
 
\nsubsection{Shuffle algebra and harmonic product algebra}

In \cite{Hoffman}, Hoffman introduced certain commutative $\mathbb{Q}$-algebras for studying MZVs.
In this subsection, we present two types of such $\Q$-algebras.

First, we describe the shuffle product on $\ide{h}$ or on $\Q$-subspaces of $\ide{h}$.
We define two $\mathbb{Q}$-subspaces $\ide{h}^0$ and $\ide{h}^1$ of $\ide{h}$ by
\[
\ide{h}^1:= \mathbb{Q} + \bigoplus_{\substack{w\in x_{1}X^*}} \mathbb{Q} w
\supset
\ide{h}^0 := \mathbb{Q} +  \bigoplus_{\substack{w\in x_{1}X^*\\  w\notin X^*x_1 }} \mathbb{Q} w.
%\ide{h}^0 := \ide{h}^1x_{\mu}\ide{h}^0:= \Span_{\mathbb{Q}}\left\{ w\in X\cap \ide{h}^1 \middle| w\notin X x_1 \right \}.
\]

We define the shuffle product $\shuffle$ on $\ide{h}$ bilinearly and recursively by $1\shuffle w = w \shuffle 1 = w$ and 
\begin{align*}
l_1w_1\shuffle l_2w_2 =& l_1(w_1\shuffle l_2w_2) + l_2(l_1w_1 \shuffle w_2 )
\end{align*}
for $l_1$, $l_2\in X$ and $w$, $w_1$, $w_2\in X^*$.
A pair $(\ide{h}, \shuffle)$ forms the unital, commutative and associative $\mathbb{Q}$-algebra and we denote it by $\ide{h}^{\shuffle}$.
Then, $\ide{h}^1$ and $\ide{h}^0$ are also closed under $\shuffle$ and become $\mathbb{Q}$-subalgebras of $\ide{h}^{\shuffle}$.
We respectively denote them by $\ide{h}^{1,\shuffle}$ and $\ide{h}^{0,\shuffle}$.
\begin{rmk}{}{}
By the weights on $\ide{h}$, $\ide{h}^{\shuffle}$ constitute the graded $\mathbb{Q}$-algebra with respect to the shuffle product.
\end{rmk}


Let $Y := \{y_{n}\}_{\substack{n\ge 1}}$ be a set of letters and $Y^*$ be the free monoid generated by $Y$.
By abuse of notation, we also use $1$ to denote the empty word on $Y$. Let $\Q\langle Y \rangle$ be the free associative $\Q$-algebra generated by $Y$.
We define the $\Q$-subspace $\Q\langle Y \rangle^0$ of $\Q\langle Y \rangle$ by $\Q\langle Y \rangle^0 := \displaystyle \mathbb{Q} + \bigoplus_{\substack{k\in\mathbb{Z}_{>1}\\  w\in Y^*}} \Q wy_{k}$.


We define the $\mathbb{Q}$-bilinear binary operation $*$, called the harmonic product on $\Q\langle Y \rangle$ inductively by $1 * w = w * 1 = w$ and
\begin{align*}
y_{k_1}w_1 * y_{k_2}w_2 :=& y_{k_1}(w_1 * y_{k_2} w_2) + y_{k_2}(y_{k_1}w_1 * w_2) + y_{k_1 + k_2}(w_1*w_2)
\end{align*}
for letters $y_{k_1}$, $y_{k_2} \in Y$ and words $w$, $w_1$ and $w_2\in Y^*$.

The pair $(\Q\langle Y \rangle, *)$ constitutes an unital, commutative, and associative $\mathbb{Q}$-algebra, which we denote by $\Q\langle Y \rangle^{*}$.
We note that $\Q\langle Y \rangle^0$ is also closed under $*$. Therefore, $\Q\langle Y \rangle^0$ constitutes a $\mathbb{Q}$-algebra with respect to $*$ and we denote it by $\Q\langle Y \rangle^{0,*}$.
\begin{rmk}{}{}
We define the weight of $\Q\langle Y \rangle$ by $\wt y_k = k$ for $k\in\mathbb{Z}_{>0}$,
which is preserved under the following natural $\Q$-linear map:
\[
\mathbf{p} : \Q\langle Y \rangle \rightarrow \ide{h} ; y_{k_1}\cdots y_{k_r} \mapsto x_1x_0^{k_1-1}\cdots x_1x_0^{k_r-1}.
\]
Thus, $\Q\langle Y \rangle$ constitutes a graded $\mathbb{Q}$-algebra with respect to the concatenation product and the harmonic product, respectively.
\end{rmk}

Before we finish this subsection, we define the $\mathbb{Q}$-linear map by
\begin{align*}
\mathbf{q} : \ide{h} \rightarrow \Q\langle Y \rangle ;& \mathbf{q}(x_0^{k_0-1}x_1x_0^{k_1-1}x_1\cdots x_1x_0^{k_{r}-1}) = 
\begin{cases}
y_{k_1}\cdots y_{k_r} & k_{0} = 1,\\
0& \text{otherwise.}
\end{cases}
\end{align*}
Then, $\mathbf{q}$ is the left inverse of $\mathbf{p}$.

\nsubsection{Duals of $\ide{h}^{\shuffle}$ and $\Q\langle Y \rangle^*$}
This subsection discusses the dual of $\ide{h}^{\shuffle}$ and $\Q\langle Y \rangle^*$. Recall $\ide{h}^{\vee} = \Q\langle \langle X \rangle\rangle$.
We write $\Phi$ as
\[
\Phi = \sum_{w\in X^*} \langle \Phi \mid w \rangle w = \langle \Phi \mid 1 \rangle + \langle \Phi \mid x_0 \rangle x_0 + \langle \Phi \mid x_1 \rangle x_1 + \cdots \hspace{0.5cm}( \langle \Phi \mid w \rangle \in \mathbb{Q}),
\]
where $\langle \Phi \mid w \rangle$ is the coefficient of $w\in X^*$ in $\Phi$.
This notation induces a $\mathbb{Q}$-bilinear map $\langle \mathchar`- | \mathchar`-\rangle : \ide{h}^{\vee} \cotimes \ide{h} \rightarrow \mathbb{Q} \hspace{0.2cm} ; \hspace{0.2cm}  \Phi \otimes w \mapsto \langle \Phi \mid w \rangle$.
We define the shuffle coproduct $\Delta_{\shuffle} : \ide{h}^{\vee} \rightarrow  \ide{h}^{\vee} \cotimes \ide{h}^{\vee}$ by
\[
\Delta_{\shuffle}(\Phi) := \sum_{u,v\in X} \langle \Phi \mid u\shuffle v \rangle u\otimes v. 
\]
Then, $\Delta_{\shuffle}$ is a continuous algebra homomorphism and satisfies $\Delta_{\shuffle}(x_i) = x_i \otimes 1 + 1\otimes x_i$.
Let $S_X^{\vee} : \ide{h}^{\vee}\rightarrow \ide{h}^{\vee}$ be the anti-automorphism of $\ide{h}^{\vee}$ defined by $x_i \mapsto -x_i$ ($i = 0$, $1$).
Then, the tuple $(\ide{h}^{\vee}, \cdot, \Delta_{\shuffle},S^{\vee}_X)$ is a completed Hopf algebra that is topologically dual to $\ide{h}^{\shuffle}$.
One notes that $\ide{h}^{\vee}$ can be regarded as an affine scheme, i.e., for a $\mathbb{Q}$-algebra $R$, we obtain the natural isomorphism:
\[
\begin{array}{ccc}
R \widehat{\otimes} \ide{h}^{\vee} & \longrightarrow & \Hom_{\Qalg}(\mathbb{Q}[u_{w}]_{w\in X^*}, R)\\
\rotatebox{90}{$\in$}&&\rotatebox{90}{$\in$}\\
\Phi & \longmapsto& (u_w \mapsto \langle \Phi \mid w \rangle),
\end{array} 
\]
where $R \widehat{\otimes} \ide{h}^{\vee}$ is the completion of the graded $R$-algebra $\varprojlim_{n}\, R\otimes \left(\ide{h}/\bigoplus_{m\ge n}\ide{h}^{(m)}\right)$.


In the same way, we shall construct the dual of $\Q\langle Y \rangle^*$.
By $\Q\langle\langle Y \rangle\rangle$, we denote the completed free associative $\Q$-algebra generated by $Y$.


Abusing the notation, for $\Phi \in \Q\langle\langle Y \rangle\rangle$, we write $\Phi$ as
\[
\Phi = \sum_{w\in Y^*} \langle \Phi \mid w \rangle w = \langle \Phi \mid 1 \rangle + \langle \Phi \mid y_1 \rangle y_1 + \langle \Phi \mid y_2 \rangle y_2 + \cdots \hspace{0.5cm}( \langle \Phi \mid w \rangle \in \mathbb{Q}).
\]
This notation induces a $\mathbb{Q}$-bilinear map $\langle \mathchar`- | \mathchar`- \rangle : \Q\langle\langle Y \rangle\rangle \cotimes \Q\langle Y \rangle \rightarrow \mathbb{Q} \hspace{0.2cm} ; \hspace{0.2cm}  \Phi \otimes w \mapsto \langle \Phi \mid w \rangle$.
We define the harmonic coproduct $\Delta_{*} : \Q\langle\langle Y \rangle\rangle \rightarrow  \Q\langle\langle Y \rangle\rangle^{\otimes 2}$ by
\[
\Delta_{*}(\Phi) := \sum_{u,v\in Y^*} \langle \Phi \mid u* v \rangle u\otimes v. 
\]
Then, $\Delta_{*}$ is a continuous algebra homomorphism which satisfies $\Delta_{*}(y_k) = y_k \otimes 1 + 1\otimes y_k + \sum_{\substack{i + j = k \\ i,j>0}} y_i \otimes y_j$ for $k\in\mathbb{Z}_{>0}$.

A tuple $(\Q\langle\langle Y \rangle\rangle, \cdot, \Delta_{*})$ constitutes a completed Hopf algebra which is topologically dual to $\Q\langle  Y \rangle^*$.
Similarly to $\ide{h}^{\vee}$, we can regard $\Q\langle\langle Y \rangle\rangle$ as an affine scheme, that is, for $\mathbb{Q}$-algebra $R$, there exists a natural isomorphism of $\mathbb{Q}$-algebra:
\[
\begin{array}{ccc}
R \widehat{\otimes} \Q\langle\langle Y \rangle\rangle & \longrightarrow & \Hom_{\Qalg}(\mathbb{Q}[u_{w}]_{u\in Y^*}, R)\\
\rotatebox{90}{$\in$}&&\rotatebox{90}{$\in$}\\
\Phi & \longmapsto& (u_w \mapsto \langle \Phi \mid w \rangle).
\end{array} 
\]



Before we finish this subsection, we define a continuous $\Q$-linear map $\mathbf{q}^{\vee} : \ide{h}^{\vee} \rightarrow\Q\langle\langle Y \rangle\rangle$ by
\[
\mathbf{q}^{\vee}(x_0^{k_1-1}x_1\cdots x_0^{k_r-1}x_1x_0^{k_{r+1}-1}) 
=
\begin{cases}
y_{k_1}\cdots y_{k_r}& \text{if } k_{r+1}=1 \text{ and}\\
0& \text{otherwise.}
\end{cases}
\]

\nsubsection{Completed free Lie algebra $\ide{F}_2$}

We define the completed free Lie algebra  generated by $X$ as
\[
\ide{F}_2:= \{\Psi \in \ide{h}^{\vee} \mid \Delta_{\shuffle}(\Psi) = \Psi \otimes 1 + 1 \otimes \Psi\}.
\]
Namely, the subspace of primitive elements with respect to the shuffle coproduct. Equivalently, a series $\Psi\in\ide{h}^{\vee}$ lies in $\ide{F}_2$ if and only if $\langle \Psi\mid u\shuffle v\rangle=0$ for all nonempty words $u,v\in X^*$.
Then the universal enveloping ring of $\ide{F}_2$ is isomorphic to $\ide{h}^{\vee}$ as Hopf algebras.
In particular, for $\Psi \in \ide{F}_2$,
\begin{align}
S_X^{\vee}(\Psi) = - \Psi \label{eq:antipode-X}
\end{align}
holds.
%Let $\exp \ide{F}_2 := \{\Phi \mid \Delta_{\shuffle}(\Phi) = \Phi\otimes \Phi\text{ and }\langle \Phi \mid 1\rangle = 1\}$. Then $\exp \ide{F}_2$ forms a group with respect to the concatenation product, and there exists a natural isomorphism $\exp : \ide{F}_2 \rightarrow \exp \ide{F}_2$ as affine schemes.
For each positive integer $k$, we define
\[
\ide{F}_2^{\ge k} := \{ \Psi \in \ide{F}_2 \mid \text{ $\langle\Psi \mid w \rangle = 0$ for $w \in X^*$ with $\wt w \le k-1$} \}.
\]


\ifdefined\isMainDocument
\else
    %\bibliographystyle{amsplain}
    %\bibliography{reference-adjoint}
    \end{document}
\fi

\ifdefined\isMainDocument
\else
    \documentclass[10pt,oneside,reqno]{amsart}
    \usepackage{xr}
    \externaldocument{../main} % Main の .aux を参照
    \usepackage{latexsym}
\usepackage{amscd}
\usepackage{amsmath}
\usepackage{amssymb}
\usepackage[mathscr]{euscript}
\usepackage{graphicx}
\usepackage{geometry}
\usepackage{mathtools}
\usepackage[all]{xy}
\usepackage[dvipsnames]{xcolor}
\usepackage[colorlinks,final,hyperindex]{hyperref}
\usepackage{tikz}
\usepackage{tikz-cd}
\usetikzlibrary{decorations.pathmorphing,decorations.markings,arrows,calc,shapes.geometric,arrows.meta,positioning}
\usepackage{enumerate}
\usepackage{multirow}

\usepackage[bb=dsserif]{mathalpha}
\usepackage{bm}

\newcommand{\QQ}{\ensuremath{\mathbb{Q}}}
\newcommand{\DD}{\ensuremath{\mathbb{D}}}
\newcommand{\CC}{\ensuremath{\mathbb{C}}}
\newcommand{\RR}{\ensuremath{\mathbb{R}}}
\newcommand{\ZZ}{\ensuremath{\mathbb{Z}}}
\newcommand{\NN}{\ensuremath{\mathbb{N}}}

\newcommand{\g}{\ensuremath{\mathfrak{g}}}
\renewcommand{\SS}{\ensuremath{\mathbb{S}}}

\newcommand{\id}{\ensuremath{\mathrm{id}}}

\makeatletter

\newcommand{\bbone}{\mathbb{1}}

\newcommand{\calA}{\mathcal{A}}
\newcommand{\calB}{\mathcal{B}}
\newcommand{\calC}{\mathcal{C}}
\newcommand{\calD}{\mathcal{D}}
\newcommand{\calE}{\mathcal{E}}
\newcommand{\calF}{\mathcal{F}}
\newcommand{\calG}{\mathcal{G}}
\newcommand{\calH}{\mathcal{H}}
\newcommand{\calI}{\mathcal{I}}
\newcommand{\calJ}{\mathcal{J}}
\newcommand{\calK}{\mathcal{K}}
\newcommand{\calL}{\mathcal{L}}
\newcommand{\calM}{\mathcal{M}}
\newcommand{\calN}{\mathcal{N}}
\newcommand{\calO}{\mathcal{O}}
\newcommand{\calP}{\mathcal{P}}
\newcommand{\calQ}{\mathcal{Q}}
\newcommand{\calR}{\mathcal{R}}
\newcommand{\calS}{\mathcal{S}}
\newcommand{\calT}{\mathcal{T}}
\newcommand{\calU}{\mathcal{U}}
\newcommand{\calV}{\mathcal{V}}
\newcommand{\calW}{\mathcal{W}}
\newcommand{\calX}{\mathcal{X}}
\newcommand{\calY}{\mathcal{Y}}
\newcommand{\calZ}{\mathcal{Z}}
\newcommand{\kk}{\Bbbk}

\newcommand{\scrY}{\mathscr{Y}}
\newcommand{\scrV}{\mathscr{V}}
\newcommand{\scrP}{\mathscr{P}}

\newcommand{\Rep}{\mathop{\mathrm{Rep}}\nolimits}
\newcommand{\alg}{\mathop{\mathrm{alg}}\nolimits}
\newcommand{\Span}{\mathop{\mathrm{Span}}\nolimits}
\newcommand{\proj}{\mathop{\mathrm{proj}}\nolimits}
\newcommand{\Hom}{\mathop{\mathrm{Hom}}\nolimits}
\newcommand{\PHom}{\mathop{\mathrm{PHom}}\nolimits}
\newcommand{\Tw}{\mathop{\mathrm{Tw}}\nolimits}
\newcommand{\Dim}{\mathop{\mathrm{Dim}}\nolimits}
\newcommand{\Imm}{\mathop{\mathrm{Im}}\nolimits}
\newcommand{\Aus}{\mathop{\mathrm{Aus}}\nolimits}
\newcommand{\Ann}{\mathop{\mathrm{Ann}}\nolimits}
\newcommand{\APC}{\mathop{\mathrm{APC}}\nolimits}
\newcommand{\Barr}{\mathop{\mathrm{Bar}}\nolimits}
\newcommand{\Cone}{\mathop{\mathrm{Cone}}\nolimits}
\newcommand{\modu}{\mathop{\mathrm{mod}}\nolimits}
\newcommand{\dgMod}{\mathop{\mathrm{dgMod}}\nolimits}
\newcommand{\soc}{\mathop{\mathrm{soc}}\nolimits}
\newcommand{\dg}{\mathop{\mathrm{dg}}\nolimits}
\newcommand{\red}{\mathop{\mathrm{red}}\nolimits}
\newcommand{\Map}{\mathop{\mathrm{Map}}\nolimits}
\newcommand{\inj}{\mathop{\mathrm{inj}}\nolimits}
\newcommand{\op}{\mathop{\mathrm{op}}\nolimits}
\newcommand{\Ker}{\mathop{\mathrm{Ker}}\nolimits}
\newcommand{\Tor}{\mathop{\mathrm{Tor}}\nolimits}
\newcommand{\Tot}{\mathop{\mathrm{Tot}}\nolimits}
\newcommand{\Ext}{\mathop{\mathrm{Ext}}\nolimits}
\newcommand{\sg}{\mathop{\mathrm{sg}}\nolimits}
\newcommand{\Sl}{\mathop{\mathfrak{sl}}\nolimits}
\newcommand{\nc}{\mathop{\mathrm{nc}}\nolimits}
\newcommand{\Tr}{\mathop{\mathrm{Tr}}\nolimits}
\newcommand{\Irr}{\mathop{\mathrm{Irr}}\nolimits}
\newcommand{\HC}{\mathop{\mathrm{HC}}\nolimits}
\newcommand{\HH}{\mathop{\mathrm{HH}}\nolimits}
\newcommand{\THH}{\mathop{\mathrm{TH}}\nolimits}
\newcommand{\Perf}{\mathop{\mathrm{Perf}}\nolimits}
\newcommand{\del}{\partial}
\newcommand{\ev}{\mathop{\mathrm{ev}}\nolimits}
\newcommand{\co}{\mathop{\mathrm{co}}\nolimits}
\newcommand{\pt}{\mathop{\mathrm{pt}}\nolimits}
\newcommand{\wt}{\mathop{\mathrm{wt}}\nolimits}
\newcommand{\pCY}{\mathop{\mathrm{pCY}}\nolimits}
\newcommand{\gr}{\mathop{\mathrm{gr}}\nolimits}
\newcommand{\coker}{\mathop{\mathrm{coker}}\nolimits}


\newcommand{\Top}{\mathop{\mathrm{Top}}\nolimits}
\newcommand{\Set}{\mathop{\mathrm{Set}}\nolimits}
\newcommand{\Vect}{\mathop{\mathrm{Vect}}\nolimits}
\newcommand{\Mod}{\mathop{\mathrm{Mod}}\nolimits}
\newcommand{\Ob}{\mathop{\mathrm{Ob}}\nolimits}
\newcommand{\Ind}{\mathop{\mathrm{Ind}}\nolimits}
\newcommand{\Sym}{\mathop{\mathrm{Sym}}\nolimits}
\newcommand{\Alt}{\mathop{\mathrm{Alt}}\nolimits}
\newcommand{\End}{\mathop{\mathrm{End}}\nolimits}
\newcommand{\Ch}{\mathop{\mathrm{Ch}}\nolimits}


\newcommand{\Ass}{\mathop{\mathrm{Ass}}\nolimits}
\newcommand{\Com}{\mathop{\mathrm{Com}}\nolimits}
\newcommand{\Lie}{\mathop{\mathrm{Lie}}\nolimits}
\newcommand{\colim}{\mathop{\mathrm{colim}}\nolimits}

\newcommand{\Op}{\mathop{\mathsf{Operads}}\nolimits}
\newcommand{\Diop}{\mathop{\mathsf{Dioperads}}\nolimits}
\newcommand{\PDiop}{\mathop{\mathsf{PDioperads}}\nolimits}
\newcommand{\Coop}{\mathop{\mathsf{Cooperads}}\nolimits}
\newcommand{\Codiop}{\mathop{\mathsf{Codioperads}}\nolimits}
\newcommand{\PCodiop}{\mathop{\mathsf{PCodioperads}}\nolimits}

\newcommand{\antish}{\ensuremath{\mbox{!`}}}


\tikzset{->-/.style={decoration={
			markings,
			mark=at position #1 with {\arrow{>}}},postaction={decorate}}}
\tikzset{-w-/.style={decoration={
			markings,
			mark=at position #1 with {\arrow{Stealth[fill=white,scale=1.4]}}},postaction={decorate}}}
\tikzset{->-/.default=0.65}
\tikzset{-w-/.default=0.65}
\tikzstyle{bullet}=[circle,fill=black,inner sep=0.5mm]
\tikzstyle{circ}=[circle,draw=black,fill=white,inner sep=0.5mm]
\tikzstyle{vertex}=[circle,draw=black,thick,inner sep=0.5mm]
\tikzstyle{dot}=[draw,circle,fill=black,minimum size=0.5mm,inner sep = 0mm, outer sep = 0mm]
\tikzset{darrow/.style={double distance = 4pt,>={Implies},->},
	darrowthin/.style={double equal sign distance,>={Implies},->},
	tarrow/.style={-,preaction={draw,darrow}},
	qarrow/.style={preaction={draw,darrow,shorten >=0pt},shorten >=1pt,-,double,double
		distance=0.2pt}}
\newcommand\tikcirc[1][2.5]{\tikz[baseline=-#1]{\draw[thick](0,0)circle[radius=#1mm];}}
\newcommand{\tikzfig}[1]{\begin{tikzpicture}[auto,baseline={([yshift=-.5ex]current bounding box.center)}]#1\end{tikzpicture}}
\usetikzlibrary{decorations.pathreplacing}

\usepackage{amsthm}
\usepackage{thmtools}
\usepackage[noabbrev,capitalize]{cleveref}
\newtheorem{theorem}{Theorem}
\newtheorem{lemma}[theorem]{Lemma}
\newtheorem{proposition}[theorem]{Proposition}
\newtheorem{corollary}[theorem]{Corollary}
\newtheorem{conjecture}[theorem]{Conjecture}
\newtheorem{Claim}[theorem]{Claim}

\newtheorem*{theorem*}{Theorem}

\theoremstyle{definition}
\newtheorem{definition}{Definition}
\newtheorem{example}{Example}
\newtheorem{cexample}{Counterexample}
\newtheorem{exercise}{Exercise}
\newtheorem{question}{Question}
\newtheorem{remark}{Remark}

\addtolength{\textheight}{1.75cm}
\addtolength{\voffset}{-0.5cm}
\addtolength{\footskip}{+0.6cm}
\geometry{margin=1.2in}


\usepackage[backend=biber, maxbibnames=99, bibencoding=utf8, doi=false, isbn=false, url=false, style=alphabetic]{biblatex}
\addbibresource{refs.bib}

    \begin{document}
\fi
\setcounter{section}{2}

\section{Extended double shuffle relations and $\DMR_0$}\label{section:review-on-DMR}

\nsubsection{Extended double shuffle relations}



MZVs are known to satisfy two types of the product to sum relations among MZVs arising from their integral and series expressions.
Combining these two types of the product to sum relations, one can derive the $\Q$-linear relations among MZVs, termed as double shuffle relations. 
However, 
%due to technical difficulties related to the convergence problem of MZVs, 
the double shuffle relation does not suffice all $\Q$-linear relations among MZVs.
 In \cite{Ihara-Kaneko-Zagier}, the authors introduced the extended double shuffle relation as a refinement of the double shuffle relation, and it is conjectured that the extended double shuffle relations derives all $\Q$-linear relations among MZVs. In this subsection, we briefly discuss this framework in more general settings.


For a $\mathbb{Q}$-algebra homomorphism $Z_R : \ide{h}^{0,\shuffle} \rightarrow R$, we say $Z_R$ satisfies the double shuffle conditions if $Z_R \circ \mathbf{p} : \Q\langle Y \rangle^{0,*} \rightarrow R$ is the $\Q$-algebra homomorphism.

Let $Z_R : \ide{h}^{0,\shuffle} \rightarrow R$ be the $\Q$-algebra homomorphism satisfying the double shuffle conditions.
Since $\ide{h}^{1,\shuffle}=\ide{h}^{0,\shuffle}[x_1]$ and
$\Q\langle Y\rangle^{*}=\Q\langle Y\rangle^{0,*}[y_1]$, there exist two unique $\Q$-algebra homomorphisms which extend $Z_R$.
Namely,
\[
\widetilde{Z}^{\shuffle}_R:\ \ide{h}^{1,\shuffle}\longrightarrow R[T],\qquad
\widetilde{Z}^{\shuffle}_R\!\mid_{\ide{h}^{0,\shuffle}}=Z_R,\ \ \widetilde{Z}^{\shuffle}_R(x_1)=T,
\]
and
\[
\widetilde{Z}^{*}_R:\ \Q\langle Y\rangle^{*}\longrightarrow R[T],\qquad
\widetilde{Z}^{*}_R\!\mid_{\Q\langle Y\rangle^{0,*}}=Z_R\circ\mathbf{p},\ \ \widetilde{Z}^{*}_R(y_1)=T,
\]
where $T$ is an indeterminate.

The main theorem of \cite{Ihara-Kaneko-Zagier} is given as follows.

\begin{prop}[{\cite[Theorem 2]{Ihara-Kaneko-Zagier}}]\label{prop:IKZ-EDS}
Let $T$ be a variable and $(R, Z_R)$ a pair of a $\Q$-algebra $R$ and an element $Z_R$ of $\Hom_{\Qalg}(\ide{h}^{0,\shuffle},R)$ with double shuffle conditions. Then the following is equivalent:
\begin{enumerate}
\item 
The following equality holds in $\Hom_{\text{$\Q$-lin}} (\ide{h}^{1}, R)$:
\[
\widetilde{Z}_R = \rho_R \circ \widetilde{Z}^*_R.
\]
Here, we define the $R$-module map $\rho_R : R[T] \rightarrow R[T]$ by the identity below in $\Q[[u]]$, where $u$ is an indeterminate:
\[
\rho_R (e^{Tu}) = \exp \left(\sum_{n\ge 2} \frac{(-1)^n}{n} Z_R(x_0^{n-1}x_1)u^n \right).
\]
\item For $w_1\in \ide{h}^{1,\shuffle}$ and $w_0 \in \ide{h}^{0,\shuffle}$, $Z_R^{\shuffle}(w_1\shuffle w_0 - \mathbf{p}(\mathbf{q}(w_1) * \mathbf{q}(w_0))) = 0$ holds.
\end{enumerate}
\end{prop}


\begin{df}\label{df:eds}
Let $Z_{R} : \ide{h}^{0,\shuffle} \rightarrow R$ be a $\mathbb{Q}$-algebra homomorphism satisfying the double shuffle conditions and Proposition \ref{prop:IKZ-EDS}. The \textbf{extended double shuffle relations} are the $\Q$-linear relations derived from
\[
(\widetilde{Z}_{R} - \rho_{R} \circ \widetilde{Z}^*_{R})(w) |_{T = 0} = 0
\]
for all $w\in X^*$.
\end{df}


\begin{ex}\label{ex:reg-MZV}
The most fundamental examples of Proposition \ref{prop:IKZ-EDS} are MZVs.
Let 
\[
Z_{\mathbb{R}} : \ide{h}^{0,\shuffle} \rightarrow \mathbb{R} ; x_1x_0^{k_1-1}\cdots x_1x_0^{k_r-1} \mapsto \zeta(k_1,\ldots,k_r) \text{\hspace{1cm}($k_1,\ldots,k_r\in\mathbb{Z}_{>0}$ with $k_r>1$}).
\]
Since MZVs have the iterated integral expression, $Z_{\mathbb{R}}$ is the $\mathbb{Q}$-algebra homomorphism.
Moreover, $Z_{\mathbb{R}}$ satisfies the double shuffle condition due to its multiple series expression. 
Therefore, there exist two $\mathbb{Q}$-algebra homomorphisms $\widetilde{Z}_{\mathbb{R}} : \ide{h}^{1,\shuffle} \rightarrow \mathbb{R}$ and $\widetilde{Z}^*_{\mathbb{R}} : \Q\langle Y \rangle^{*} \rightarrow \mathbb{R}$. 
Then, $\widetilde{Z}_{\mathbb{R}}$ and $\widetilde{Z}^*_{\mathbb{R}}$ satisfy the conditions of Proposition \ref{prop:IKZ-EDS}.
In relation to these two maps, we define certain regularizations of MZVs. For $(k_1,\ldots,k_r)\in\mathbb{Z}_{>0}^r$, we respectively define shuffle regularized MZVs $\zeta^{\shuffle}$ and harmonic regularized MZVs $\zeta^*$ by
\begin{align*}
\zeta^{\shuffle}(k_1,\ldots,k_r) =& \widetilde{Z}_{\mathbb{R}}(x_1x_0^{k_1-1}\cdots x_1x_0^{k_r-1})(0)\\
\zeta^{*}(k_1,\ldots,k_r) =& \widetilde{Z}^*_{\mathbb{R}}(y_{k_1}\cdots y_{k_r})(0).
\end{align*}
\end{ex}

Let $I_{\EDS}$ be a two-sided ideal of $\ide{h}^{\shuffle}$ generated by $x_0$, $x_1$, and $w_1\shuffle w_0 - \mathbf{p}(\mathbf{q}(w_1) * \mathbf{q}(w_0))$ for $w_1\in \ide{h}^{1,\shuffle}$ and $w_0 \in \ide{h}^{0,\shuffle}$, and we define
\[
\mathcal{Z}^f := \ide{h}^{\shuffle}/I_{\EDS}.
\]
For $(k_1,\ldots,k_r)\in \mathbb{Z}_{>0}^r$ with $r\ge1$, we write $\zeta^f(k_1,\ldots,k_r)$ for the image of the word $x_1x_0^{k_1-1}\cdots x_1x_0^{k_r-1}$ in $\mathcal{Z}^f$, and we put $\zeta^f(\emptyset)=1$.
In particular, the classes $\zeta^f(k_1,\ldots,k_r)$ generate $\mathcal{Z}^f$ as a $\Q$-vector space.
In this framework, the authors of \cite{Ihara-Kaneko-Zagier} proposed the following conjecture.

\begin{conj}\label{conj:EDS}
The following $\Q$-algebra homomorphism
\[
\mathcal{Z}^f \rightarrow \mathcal{Z} : \zeta^f(k_1,\ldots,k_r)\mapsto  \zeta(k_1,\ldots,k_r)
\]
is a $\Q$-algebra isomorphism.
Namely, all $\Q$-linear relations among MZVs are obtained from the extended double shuffle relations.
\end{conj}
Define the affine scheme $\EDS : \Qalg\rightarrow \cat{Set}; R\mapsto \Hom_{\Qalg}(\mathcal{Z}^f,R)$ whose coordinate ring is $\mathcal{Z}^f$.

\nsubsection{The double shuffle group $\DMR_0$}

Within Hoffman's framework, the double shuffle relations are equivalent to group-likeness for the coproducts $\Delta_{\shuffle}$ and $\Delta_{*}$.
From this observation, Racinet \cite{Racinet} developed another approach to the regularization theorem (Proposition \ref{prop:IKZ-EDS}) and proposed a specific affine group scheme $\DMR_0$ called the \textbf{double shuffle group}. It is well-known that the coordinate ring $\mathcal{O}(\DMR_0)$ of $\DMR_0$ is $\mathcal{Z}^f/\zeta^f(2)\mathcal{Z}^f$.
The group $\DMR_0$ is expected to satisfy certain specific conditions.
For instance, $\DMR_0$ is conjecturally isomorphic to the motivic Galois group over the $\mathbb{Z}$, Grothendieck-Teichm\"{u}ller group, and the Kashiwara-Vergne group.
\begin{comment}
\textcolor{red}{
In this section, we treat two affine group schemes, $\TM$ and $\DMR_0$.
The group law of $\DMR_0$ is called the Ihara product $\circledast$ (see the sentence before Proposition \ref{prop:TM-Grp}).
To establish the property of affine group schemes in $\DMR_0$, Racinet proposed the affine group scheme $\TM$ called the \textbf{twisted Magnus group}. 
As a result, Racinet proved that $\DMR_0$ is the closed affine group subscheme of $\TM$.
We present the review of $\TM$ and $\DMR_0$. Additionally, we also review its corresponding Lie algebras $\tm$ and $\dmr$.
At the end of this section, we treat a bigraded Lie algebra $\ls$.
The bigraded Lie algebra $\ls$ is related to the main theorems in this paper.
}
\end{comment}





%, and its bigraded Lie algebra $\ls$}



As mentioned in the previous subsection, the authors of \cite{Ihara-Kaneko-Zagier} introduced the $\shuffle$-regularized MZVs and the $*$-regularized MZVs to show the connection between the shuffle product relations and the harmonic product relations among MZVs.
Independently, Racinet established the regularized theorem by using the generating function of MZVs. 

Let
\begin{align*}
\Phi_{\shuffle} :=& \sum_{w\in X} \zeta^{\shuffle}(w) \overset{\leftarrow}{w}, \text{ and }\\
\Phi_{*} :=& \sum_{w\in Y} \zeta^{*}(w) \overset{\leftarrow}{w},
\end{align*}
where, $\overset{\leftarrow}{w}$ is the reversal word of $w$.
Since $\widetilde{Z}_{\mathbb{R}}$ (respectively, $\widetilde{Z}_{\mathbb{R}}^*$) is the $\Q$-algebra homomorphism with respect to $\shuffle$ (respectively, $*$), and since a $\Q$-linear map $\mathbb{R}[T] \rightarrow \mathbb{R} ; aT + b \mapsto b$ for $a$, $b\in \mathbb{R}$ is also a $\Q$-algebra homomorphism, we have 
\[
\langle \Phi_{\shuffle} \mid u \shuffle v \rangle = \Phi(u)\Phi(v)
\]
for $u$, $v \in X$.
\[
 \text{(respectively, }\langle \Phi_* \mid u * v \rangle = \langle \Phi_* \mid u \rangle \langle \Phi_* \mid  v \rangle
\]
for $u$, $v\in Y$).
Consequently, we have 
\begin{align*}
\Delta_{\shuffle}(\Phi_{\shuffle}) =& \Phi_{\shuffle} \otimes \Phi_{\shuffle}\\
\left(\text{respectively, } \right. \Delta_*(\Phi_*) =&\left. \Phi_* \otimes \Phi_* \right).
\end{align*}
Related to these two generating functions, Racinet showed the following:
\begin{thm}[{\cite[Corollary 2.24]{Racinet}}]\label{thm:regularized-Racinet}
Let
\[
\Gamma_{\Phi_{\shuffle}} :=\exp \left( \sum_{n\ge 2} \frac{(-1)^{n-1}}{n} \langle \Phi_{\shuffle} \mid x_0^{n-1}x_1 \rangle y_1^n \right).
\]
Then, we have
\[
\Phi_* = \Gamma_{\Phi_{\shuffle}} \mathbf{q}^{\vee}(\Phi_{\shuffle}).
\]
\end{thm}

From Theorem \ref{thm:regularized-Racinet}, Racinet constructed the certain subset of $R\cotimes \ide{h}^{\vee}$. 
\begin{df}[{\cite[D\'{e}finition 3.2.1]{Racinet}}]\label{def:DMR}
Let $R$ be a $\mathbb{Q}$-algebra.
The double shuffle set $\DMR(R)$ consists of those $\Phi \in R \widehat{\otimes} \ide{h}^{\vee}$:
\begin{enumerate}
\item\label{def:DMR-1} $\langle \Phi \mid 1\rangle = 1$,
\item\label{def:DMR-2} $\langle \Phi \mid x_0\rangle = \langle \Phi \mid x_1\rangle = 0$,
\item\label{def:DMR-3} $\Delta_{\shuffle}(\Phi) = \Phi\otimes \Phi$, and
\item\label{def:DMR-4} $\Delta_{*}(\Phi_{\star}) = \Phi_{\star}\otimes \Phi_{\star}$.
\end{enumerate}
Here, $\Phi_{\star} = \Gamma_{\Phi}\mathbf{q}^{\vee}(\Phi)$ and $\Gamma_{\Phi}$ is defined by
\[
\Gamma_{\Phi} = \exp \left(\sum_{n\ge 2}\frac{(-1)^{n}}{n}\langle \Phi \mid x_0^{k-1}x_1 \rangle y_1^n\right).
\]
Let $\lambda\in R$. By $\DMR_{\lambda}(R)$, we denote the subset of $\DMR(R)$ satisfying the additional condition:
\begin{enumerate}
\setcounter{enumi}{4}
\item\label{def:DMR-5} $\langle \Phi \mid x_0x_1\rangle =- \frac{ \lambda^2 }{24}$.
\end{enumerate} 
\end{df}
We define functors $\DMR : \Qalg\rightarrow \cat{Set}; R\mapsto \DMR(R)$ and $\DMR_0 : \Qalg\rightarrow \cat{Set}; R\mapsto \DMR_0(R)$.

\begin{rmk}
Given $\Phi\in\DMR(R)$, the generating series determines a $\Q$-algebra homomorphism $Z_{R,\Phi} : \mathcal{Z}^f \rightarrow R ; \zeta^f(k_1,\ldots,k_r) \mapsto \langle \Phi \mid x_0^{k_r-1}x_1\cdots x_0^{k_1-1}x_1 \rangle$ for $\Phi \in \DMR(R)$. It can be verified that $Z_{R,\Phi} \in \EDS(R)$ induces a natural isomorphism $\DMR \Longrightarrow \EDS$. Further details can be found in \cite{Bachmann-Yaddaden}.
Their result indicates that the works of Ihara, Kaneko, and Zagier (\cite{Ihara-Kaneko-Zagier}) and Racinet (\cite{Racinet}) are essentially equivalent.
\end{rmk}

For $\phi$ and $\psi\in \ide{h}^{\vee}$ with $\langle \phi \mid 1\rangle = \langle \psi \mid 1\rangle = 1$, we define 
\[
\phi \circledast \psi := \phi \cdot \kappa_{\phi^{-1}x_1\phi}(\psi).
\]
Here, $\kappa_{f}$ is an endomorphism of $\ide{h}^{\vee}$ as a $\mathbb{Q}$-algebra, associated with fixed $f\in \ide{h}^{\vee}$ such that $\langle f \mid 1\rangle = 0$, defined by
\[
x_0\mapsto x_0\hspace{1.0ex},\hspace{1.0ex} x_1\mapsto f.
\]

The operation $\circledast$ is referred to as the \textbf{Ihara product}.

\begin{thm}[{\cite[Th\'{e}or\`{e}m I]{Racinet}}]
Let $\DMR_0 : \Qalg\rightarrow \cat{Set} ; R \mapsto \DMR_0(R)$. Then, $\DMR_0$ is the affine group scheme, i.e. for $\mathbb{Q}$-algebra $R$, $\DMR_0(R)$ is a group with respect to $\circledast$.
\end{thm}


\ifdefined\isMainDocument
\else
    %\bibliographystyle{amsplain}
    %\bibliography{reference-adjoint}
    \end{document}
\fi

\ifdefined\isMainDocument
\else
    \documentclass[10pt,oneside,reqno]{amsart}
    \usepackage{xr}
    \externaldocument{../main} % Main の .aux を参照
    \usepackage{latexsym}
\usepackage{amscd}
\usepackage{amsmath}
\usepackage{amssymb}
\usepackage[mathscr]{euscript}
\usepackage{graphicx}
\usepackage{geometry}
\usepackage{mathtools}
\usepackage[all]{xy}
\usepackage[dvipsnames]{xcolor}
\usepackage[colorlinks,final,hyperindex]{hyperref}
\usepackage{tikz}
\usepackage{tikz-cd}
\usetikzlibrary{decorations.pathmorphing,decorations.markings,arrows,calc,shapes.geometric,arrows.meta,positioning}
\usepackage{enumerate}
\usepackage{multirow}

\usepackage[bb=dsserif]{mathalpha}
\usepackage{bm}

\newcommand{\QQ}{\ensuremath{\mathbb{Q}}}
\newcommand{\DD}{\ensuremath{\mathbb{D}}}
\newcommand{\CC}{\ensuremath{\mathbb{C}}}
\newcommand{\RR}{\ensuremath{\mathbb{R}}}
\newcommand{\ZZ}{\ensuremath{\mathbb{Z}}}
\newcommand{\NN}{\ensuremath{\mathbb{N}}}

\newcommand{\g}{\ensuremath{\mathfrak{g}}}
\renewcommand{\SS}{\ensuremath{\mathbb{S}}}

\newcommand{\id}{\ensuremath{\mathrm{id}}}

\makeatletter

\newcommand{\bbone}{\mathbb{1}}

\newcommand{\calA}{\mathcal{A}}
\newcommand{\calB}{\mathcal{B}}
\newcommand{\calC}{\mathcal{C}}
\newcommand{\calD}{\mathcal{D}}
\newcommand{\calE}{\mathcal{E}}
\newcommand{\calF}{\mathcal{F}}
\newcommand{\calG}{\mathcal{G}}
\newcommand{\calH}{\mathcal{H}}
\newcommand{\calI}{\mathcal{I}}
\newcommand{\calJ}{\mathcal{J}}
\newcommand{\calK}{\mathcal{K}}
\newcommand{\calL}{\mathcal{L}}
\newcommand{\calM}{\mathcal{M}}
\newcommand{\calN}{\mathcal{N}}
\newcommand{\calO}{\mathcal{O}}
\newcommand{\calP}{\mathcal{P}}
\newcommand{\calQ}{\mathcal{Q}}
\newcommand{\calR}{\mathcal{R}}
\newcommand{\calS}{\mathcal{S}}
\newcommand{\calT}{\mathcal{T}}
\newcommand{\calU}{\mathcal{U}}
\newcommand{\calV}{\mathcal{V}}
\newcommand{\calW}{\mathcal{W}}
\newcommand{\calX}{\mathcal{X}}
\newcommand{\calY}{\mathcal{Y}}
\newcommand{\calZ}{\mathcal{Z}}
\newcommand{\kk}{\Bbbk}

\newcommand{\scrY}{\mathscr{Y}}
\newcommand{\scrV}{\mathscr{V}}
\newcommand{\scrP}{\mathscr{P}}

\newcommand{\Rep}{\mathop{\mathrm{Rep}}\nolimits}
\newcommand{\alg}{\mathop{\mathrm{alg}}\nolimits}
\newcommand{\Span}{\mathop{\mathrm{Span}}\nolimits}
\newcommand{\proj}{\mathop{\mathrm{proj}}\nolimits}
\newcommand{\Hom}{\mathop{\mathrm{Hom}}\nolimits}
\newcommand{\PHom}{\mathop{\mathrm{PHom}}\nolimits}
\newcommand{\Tw}{\mathop{\mathrm{Tw}}\nolimits}
\newcommand{\Dim}{\mathop{\mathrm{Dim}}\nolimits}
\newcommand{\Imm}{\mathop{\mathrm{Im}}\nolimits}
\newcommand{\Aus}{\mathop{\mathrm{Aus}}\nolimits}
\newcommand{\Ann}{\mathop{\mathrm{Ann}}\nolimits}
\newcommand{\APC}{\mathop{\mathrm{APC}}\nolimits}
\newcommand{\Barr}{\mathop{\mathrm{Bar}}\nolimits}
\newcommand{\Cone}{\mathop{\mathrm{Cone}}\nolimits}
\newcommand{\modu}{\mathop{\mathrm{mod}}\nolimits}
\newcommand{\dgMod}{\mathop{\mathrm{dgMod}}\nolimits}
\newcommand{\soc}{\mathop{\mathrm{soc}}\nolimits}
\newcommand{\dg}{\mathop{\mathrm{dg}}\nolimits}
\newcommand{\red}{\mathop{\mathrm{red}}\nolimits}
\newcommand{\Map}{\mathop{\mathrm{Map}}\nolimits}
\newcommand{\inj}{\mathop{\mathrm{inj}}\nolimits}
\newcommand{\op}{\mathop{\mathrm{op}}\nolimits}
\newcommand{\Ker}{\mathop{\mathrm{Ker}}\nolimits}
\newcommand{\Tor}{\mathop{\mathrm{Tor}}\nolimits}
\newcommand{\Tot}{\mathop{\mathrm{Tot}}\nolimits}
\newcommand{\Ext}{\mathop{\mathrm{Ext}}\nolimits}
\newcommand{\sg}{\mathop{\mathrm{sg}}\nolimits}
\newcommand{\Sl}{\mathop{\mathfrak{sl}}\nolimits}
\newcommand{\nc}{\mathop{\mathrm{nc}}\nolimits}
\newcommand{\Tr}{\mathop{\mathrm{Tr}}\nolimits}
\newcommand{\Irr}{\mathop{\mathrm{Irr}}\nolimits}
\newcommand{\HC}{\mathop{\mathrm{HC}}\nolimits}
\newcommand{\HH}{\mathop{\mathrm{HH}}\nolimits}
\newcommand{\THH}{\mathop{\mathrm{TH}}\nolimits}
\newcommand{\Perf}{\mathop{\mathrm{Perf}}\nolimits}
\newcommand{\del}{\partial}
\newcommand{\ev}{\mathop{\mathrm{ev}}\nolimits}
\newcommand{\co}{\mathop{\mathrm{co}}\nolimits}
\newcommand{\pt}{\mathop{\mathrm{pt}}\nolimits}
\newcommand{\wt}{\mathop{\mathrm{wt}}\nolimits}
\newcommand{\pCY}{\mathop{\mathrm{pCY}}\nolimits}
\newcommand{\gr}{\mathop{\mathrm{gr}}\nolimits}
\newcommand{\coker}{\mathop{\mathrm{coker}}\nolimits}


\newcommand{\Top}{\mathop{\mathrm{Top}}\nolimits}
\newcommand{\Set}{\mathop{\mathrm{Set}}\nolimits}
\newcommand{\Vect}{\mathop{\mathrm{Vect}}\nolimits}
\newcommand{\Mod}{\mathop{\mathrm{Mod}}\nolimits}
\newcommand{\Ob}{\mathop{\mathrm{Ob}}\nolimits}
\newcommand{\Ind}{\mathop{\mathrm{Ind}}\nolimits}
\newcommand{\Sym}{\mathop{\mathrm{Sym}}\nolimits}
\newcommand{\Alt}{\mathop{\mathrm{Alt}}\nolimits}
\newcommand{\End}{\mathop{\mathrm{End}}\nolimits}
\newcommand{\Ch}{\mathop{\mathrm{Ch}}\nolimits}


\newcommand{\Ass}{\mathop{\mathrm{Ass}}\nolimits}
\newcommand{\Com}{\mathop{\mathrm{Com}}\nolimits}
\newcommand{\Lie}{\mathop{\mathrm{Lie}}\nolimits}
\newcommand{\colim}{\mathop{\mathrm{colim}}\nolimits}

\newcommand{\Op}{\mathop{\mathsf{Operads}}\nolimits}
\newcommand{\Diop}{\mathop{\mathsf{Dioperads}}\nolimits}
\newcommand{\PDiop}{\mathop{\mathsf{PDioperads}}\nolimits}
\newcommand{\Coop}{\mathop{\mathsf{Cooperads}}\nolimits}
\newcommand{\Codiop}{\mathop{\mathsf{Codioperads}}\nolimits}
\newcommand{\PCodiop}{\mathop{\mathsf{PCodioperads}}\nolimits}

\newcommand{\antish}{\ensuremath{\mbox{!`}}}


\tikzset{->-/.style={decoration={
			markings,
			mark=at position #1 with {\arrow{>}}},postaction={decorate}}}
\tikzset{-w-/.style={decoration={
			markings,
			mark=at position #1 with {\arrow{Stealth[fill=white,scale=1.4]}}},postaction={decorate}}}
\tikzset{->-/.default=0.65}
\tikzset{-w-/.default=0.65}
\tikzstyle{bullet}=[circle,fill=black,inner sep=0.5mm]
\tikzstyle{circ}=[circle,draw=black,fill=white,inner sep=0.5mm]
\tikzstyle{vertex}=[circle,draw=black,thick,inner sep=0.5mm]
\tikzstyle{dot}=[draw,circle,fill=black,minimum size=0.5mm,inner sep = 0mm, outer sep = 0mm]
\tikzset{darrow/.style={double distance = 4pt,>={Implies},->},
	darrowthin/.style={double equal sign distance,>={Implies},->},
	tarrow/.style={-,preaction={draw,darrow}},
	qarrow/.style={preaction={draw,darrow,shorten >=0pt},shorten >=1pt,-,double,double
		distance=0.2pt}}
\newcommand\tikcirc[1][2.5]{\tikz[baseline=-#1]{\draw[thick](0,0)circle[radius=#1mm];}}
\newcommand{\tikzfig}[1]{\begin{tikzpicture}[auto,baseline={([yshift=-.5ex]current bounding box.center)}]#1\end{tikzpicture}}
\usetikzlibrary{decorations.pathreplacing}

\usepackage{amsthm}
\usepackage{thmtools}
\usepackage[noabbrev,capitalize]{cleveref}
\newtheorem{theorem}{Theorem}
\newtheorem{lemma}[theorem]{Lemma}
\newtheorem{proposition}[theorem]{Proposition}
\newtheorem{corollary}[theorem]{Corollary}
\newtheorem{conjecture}[theorem]{Conjecture}
\newtheorem{Claim}[theorem]{Claim}

\newtheorem*{theorem*}{Theorem}

\theoremstyle{definition}
\newtheorem{definition}{Definition}
\newtheorem{example}{Example}
\newtheorem{cexample}{Counterexample}
\newtheorem{exercise}{Exercise}
\newtheorem{question}{Question}
\newtheorem{remark}{Remark}

\addtolength{\textheight}{1.75cm}
\addtolength{\voffset}{-0.5cm}
\addtolength{\footskip}{+0.6cm}
\geometry{margin=1.2in}


\usepackage[backend=biber, maxbibnames=99, bibencoding=utf8, doi=false, isbn=false, url=false, style=alphabetic]{biblatex}
\addbibresource{refs.bib}

    \begin{document}
\fi

\section[$\AdDMR_0$, $\addmr$, and $\adls$]{Adjoint double shuffle scheme $\AdDMR_0$} \label{sect:AdDMR}
% and its bigraded version  $\adls$}



The purpose of this paper is to study AdMZVs. To begin with, we define AdMZVs more precisely. Let $\bullet\in \{\shuffle,*\}$. For $k_1,\ldots,k_r\in\mathbb{Z}_{>0}$ and $l\in\mathbb{Z}_{\ge 0}$, $\bullet$-AdMZVs are defined by
\begin{align*}
\bAdmzv{l}{k_1,\ldots,k_r} :=& \sum_{i = 0}^r (-1)^{k_{i+1}+\cdots+k_r + l} \zeta^{\bullet}(k_1,\ldots,k_i)\zeta^{\bullet}_l(k_r,\ldots,k_{i+1}) \in \mathcal{Z}.
\end{align*}
Here, 
\[
\zeta_{l}^{\bullet}(k_r,\ldots,k_{i+1}) = (-1)^l \sum_{\substack{l_{i+1}+\cdots+l_r = l \\ l_{i+1},\ldots,l_r\ge 0}}
\left(\prod_{j = i+1}^r \binom{k_j + l_j+1}{l_j}\right)
\zeta^{\bullet}(k_r+l_r,\ldots,k_{i+1}+l_{i+1}).
\]
It should be noted that the images of these two types of AdMZVs in $\mathcal{Z}/\zeta(2)\mathcal{Z}$ are identical, i.e.,
\begin{align}
\sAdmzv{l}{k_1,\ldots,k_r}. \equiv \hAdmzv{l}{k_1,\ldots,k_r} \mod \zeta(2)\mathcal{Z} \label{eq:gen-AdMZV}
\end{align}
holds. 
We define AdMZVs as the image of $\bullet$-AdMZVs in $\mathcal{Z}/\zeta(2)\mathcal{Z}$ and denote $\sAdmzv{l}{k_1,\ldots,k_r} \mod \zeta(2)\mathcal{Z}$ by simply $\Admzv{l}{k_1,\ldots,k_r}$.
In this section, we review Jarossay's work \cite{Jarossay}.
We consider a generating function for AdMZVs as follows:
\begin{align*}
\Phi_{\mathrm{Ad}_{\shuffle}} :=& \sum_{x_0^lx_1x_0^{k_r-1}x_1\cdots x_0^{k_1-1}x_1 \in X^{*}} \sAdmzv{l}{k_1,\ldots,k_r} x_0^lx_1x_0^{k_r-1}x_1\cdots x_0^{k_1-1}x_1 + \text{(addtional terms).}%\\
%\Phi_{\mathrm{Ad}_*} :=& \sum_{l \ge 0} \sum_{Y_{k_1}\cdots Y_{k_r} \in Y^*} \hAdmzv{l}{k_1,\ldots,k_r} Y_{k_r}\cdots Y_{k_1}.
\end{align*}
The fundamental idea of Jarossay's work is the following equality $\Phi_{\mathrm{Ad}_{\shuffle}} = \Phi_{\shuffle}^{-1}x_1\Phi_{\shuffle}$.
From this perspective, he developed the property of $\Phi_{\shuffle}^{-1}x_1\Phi_{\shuffle}$ and proposed the $\Q$-linear relation among AdMZVs. He named these relations \textbf{adjoint double shuffle relations}.
Additionally, he introduced an affine scheme called the \textit{adjoint double shuffle scheme} and raised the question of whether two affine schemes, $\DMR_0$ and the adjoint double shuffle scheme, are actually isomorphic.
From this observation, we then consider an affine scheme \textit{the adjoint double shuffle scheme} $\AdDMR_0$.
 
\nsubsection{The generating function of AdMZVs and adjoint double shuffle relations}

A key idea in Jarossay's work is the occurrences of the AdMZVs in the coefficient of $\Phi_{\shuffle}^{-1}x_1\Phi_{\shuffle}$.
The following proposition formalizes this observation.
\begin{prop}[{\cite[Proposition 3.2.2]{Jarossay}}]
For $l\in \mathbb{Z}_{\ge 0}$ and $k_1,\ldots,k_r\in\mathbb{Z}_{>0}$, the following holds:
\[
\langle \Phi_{\shuffle}^{-1}x_1\Phi_{\shuffle} \mid x_0^lx_1x_0^{k_r-1}x_1\cdots x_0^{k_1-1}x_1\rangle = \sAdmzv{l}{k_1,\ldots,k_r} .
\]
\end{prop}
\begin{proof}
The claim follows from the identities below.
\begin{align*}
\langle \Phi_{\shuffle}^{-1}x_1\Phi_{\shuffle} \mid x_0^{k_r-1}x_1\cdots x_0^{k_1-1}x_1x_0^l \rangle
=& \sum_{j = 0}^r \langle\Phi_{\shuffle}^{-1} \mid x_0^lx_1\cdots x_0^{k_1-1}x_1\cdots x_0^{k_i-1}\rangle \langle \Phi_{\shuffle}\mid x_0^{k_{i+1}-1}x_1 \cdots x_0^{k_r-1}x_1\rangle \\
=& \sum_{j = 0}^r(-1)^{k_1 + \cdots + k_i + l}\langle \Phi_{\shuffle} \mid x_0^{k_i-1}x_1 \cdots x_0^{k_1}x_1x_0^l \rangle\\
&\times \langle \Phi_{\shuffle}\mid x_0^{k_{i+1}-1}x_1 \cdots x_0^{k_r-1}x_1\rangle \displaybreak[3] \\
=&\sum_{j = 0}^r(-1)^{k_1 + \cdots + k_i + l} \zeta_l^{\shuffle}(k_1,\ldots,k_j)\zeta^{\shuffle}(k_r,\ldots,k_{j+1})\\
=&\sAdmzv{l}{k_r,\ldots,k_1}.
\end{align*}
Here, we use the antipode property of the group-like element, namely, $S_{X}^{\vee}(\Phi_{\shuffle}) = \Phi_{\shuffle}^{-1}$ in  the second equality and
\begin{align*}
\langle \Phi_{\shuffle} \mid x_0^{k_r-1}x_1\cdots x_0^{k_1-1}x_1x_0^l \rangle =  \zeta_{l}^{\shuffle} (k_1,\ldots,k_r) 
\end{align*}
in the third equality.
\end{proof}
Since $\Delta_{\shuffle}(\Phi_{\shuffle}) = \Phi_{\shuffle} \otimes \Phi_{\shuffle}$,
it follows that
\begin{align}
\Delta_{\shuffle}(\Phi_{\shuffle}^{-1}x_1\Phi_{\shuffle} ) = \Phi_{\shuffle}^{-1}x_1\Phi_{\shuffle}  \otimes 1 + 1\otimes \Phi_{\shuffle}^{-1}x_1\Phi_{\shuffle}. \label{eq:ad-shuffle}
\end{align}
Since the condition $\Delta_{\shuffle}(\Phi_{\shuffle}) = \Phi_{\shuffle} \otimes \Phi_{\shuffle}$ corresponds to the shuffle product relations of $\shuffle$-regularized MZVs, the condition \eqref{eq:ad-shuffle} corresponds to some $\Q$-linear relations among $\shuffle$-AdMZVs which are related to the binary operation $\shuffle$.
We call these $\Q$-linear relations in \eqref{eq:ad-shuffle}  \textbf{adjoint shuffle relations}.

We next investigate other $\Q$-linear relations among AdMZVs which correspond to the harmonic product relations.
Let $l\in \mathbb{Z}_{\ge 0}$, $k_1,\ldots,k_r\in\mathbb{Z}_{>0}$, and $T$ be an indeterminate. We define $\mathbf{q}_{\#}^{\vee} :  \ide{h}^{\vee} \rightarrow \mathbb{Q}[[T]]\cotimes \Q\langle\langle Y \rangle\rangle$ as a $\mathbb{Q}$-linear map by
\[
\mathbf{q}_{\#}^{\vee} (x_0^{k_1-1}x_1\cdots x_1x_0^{k_{r+1}-1})
=
\begin{cases}
T^{k_1-1}y_{k_2}\cdots y_{k_{r}} &\text{if }k_{r+1}=1\text{ and}\\
0&\text{otherwise, this case includes $\text{(argument)} =1$}.
\end{cases}
\]
From \eqref{eq:gen-AdMZV}, it follows that
\begin{align*}
\mathbf{q}_{\#}^{\vee}(\Phi_{\shuffle}^{-1}x_1\Phi_{\shuffle}) \equiv \sum_{w = y_{k_1}\cdots y_{k_r} \in Y^*}\left( \sum_{l\ge 0} T^l \zeta_{\mathrm{Ad}}(k_r,\ldots,k_1,l)\right) w \mod \zeta(2)\mathcal{Z} \cotimes \ide{h}^{\vee}.
\end{align*}
On the other hand, we define a $\Q[[T]]$-module morphism $\sft_*^{\vee}$ of $\mathbb{Q}[[T]] \cotimes \Q\langle \langle Y \rangle \rangle$ by
\[
y_k \mapsto \sum_{j=0}^{k-1} \binom{k-1}{j} T^{j} y_{k-j}.
\]
Then, it follows that $\sft_*^{\vee}$ is the coalgebra homomorphism with respect to $\Delta_*$.
Since, for $\Phi \in \ide{h}^{\vee}$, $\sft_*^{\vee}(\Phi)$ is explicitly given by
\begin{align}
\sft_*^{\vee}(\Phi) = \sum_{w = y_{k_1}\cdots y_{k_r} \in Y^*} \sum_{l\ge 0} T^l \sum_{\substack{l_1 + \cdots +l_r = l \\ l_1,\ldots, l_r\ge 0}} \prod_{j=1}^r \binom{k_j + l_j -1}{l_j} \langle\Phi \mid y_{k_1+l_1}\cdots y_{k_r+l_r} \rangle w,\label{eq:sft-exc}
\end{align}
this leads to the following relation between $\mathbf{q}_{\#}^{\vee}(\Phi_{\shuffle}^{-1}x_1\Phi_{\shuffle})$ and $\Phi_*$:
\begin{align*}
\mathbf{q}_{\#}^{\vee}(\Phi_{\shuffle}^{-1}x_1\Phi_{\shuffle}) =& 
\sum_{w = y_{k_1}\cdots y_{k_r} \in Y^*}\left( \sum_{l\ge 0} T^l \zeta_{\mathrm{Ad}}^{\shuffle}(k_r,\ldots,k_1;l)\right) w  \displaybreak[3] \\
\stackrel{\eqref{eq:gen-AdMZV}}{\equiv} &  \sum_{w = y_{k_1}\cdots y_{k_r} \in Y^*}\left( \sum_{l\ge 0} T^l \zeta_{\mathrm{Ad}}^{*}(k_r,\ldots,k_1;l)\right) w \displaybreak[3] \\
\equiv &  \sum_{w = y_{k_1}\cdots y_{k_r} \in Y^*}\left( \sum_{l\ge 0} T^l  \sum_{j = 0}^r (-1)^{k_1+\cdots+k_j + l} \zeta^*(k_r,\ldots,k_{j+1})\zeta^*(k_1,\ldots,k_j ; l)  \right) w \displaybreak[3] \\
\equiv& \sum_{w = y_{k_1}\cdots y_{k_r} \in Y^*} \sum_{ j = 0}^r   \\
& \left(\sum_{l\ge 0} T^l \sum_{\substack{l_1 +\cdots + l_r = l \\l_1,\ldots,l_r\ge 0 }} \left(\prod_{i = 1}^j \binom{k_i + l_i -1}{l_i}\right) \zeta^{*}(k_1 + l_1,\ldots,k_j + l_j) \right) \inv(y_{k_j}\cdots y_{k_1}) \\
&\times  \zeta^*(k_r,\ldots,k_j) y_{k_{j+1}}\cdots y_{k_r} \displaybreak[3] \\
\equiv & \sum_{w = y_{k_1}\cdots y_{k_r} \in Y^*}  \sum_{ j = 0}^r \\
&  \left(\sum_{l\ge 0} T^l \sum_{\substack{l_1 +\cdots + l_r = l \\ l_1,\ldots,l_r\ge 0} } \left(\prod_{i = 1}^j \binom{k_i + l_i -1}{l_i}\right) \langle \Phi_* \mid y_{k_j + l_j}\cdots y_{k_1+l_1}\rangle \right) \inv(y_{k_j}\cdots y_{k_1}) \\
&\times  \langle \Phi_* \mid y_{k_{j+1}}\cdots y_{k_r} \rangle y_{k_{j+1}}\cdots y_{k_r} \displaybreak[3] \\
\stackrel{\eqref{eq:sft-exc}}{\equiv} & \sum_{w = y_{k_1}\cdots y_{k_r} \in Y^*}  \sum_{ j = 0}^r \\
&\left(\langle \sft_*^{\vee}(\Phi_*) \mid y_{k_j}\cdots y_{k_1} \rangle \inv(y_{k_j}\cdots y_{k_1})\right)\times 
\left( \langle \Phi_* \mid y_{k_{j+1}}\cdots y_{k_r} \rangle  y_{k_{j+1}}\cdots y_{k_r}\right) \displaybreak[3] \\ 
\equiv &\inv \circ \sft_*^{\vee}(\Phi_*)\Phi_*
\end{align*}
modulo $\zeta(2)\mathcal{Z} \cotimes \ide{h}^{\vee}$.
Here, $\inv$ is an anti-automorphism on $\Q\langle \langle Y \rangle\rangle$ defined by $y_k \mapsto (-1)^k y_k$ ($k\in\mathbb{Z}_{>0}$).
Since $\sft_*^{\vee}$ is the coalgebra homomorphism with respect to $\Delta_*$, it follows that the following equality holds:
\begin{align}
\Delta_*(\mathbf{q}_{\#}^{\vee}(\Phi_{\shuffle}^{-1}x_1\Phi_{\shuffle})) = \mathbf{q}_{\#}^{\vee}(\Phi_{\shuffle}^{-1}x_1\Phi_{\shuffle})\otimes \mathbf{q}_{\#}^{\vee}(\Phi_{\shuffle}^{-1}x_1\Phi_{\shuffle}). \label{eq:ad-harmonic}
\end{align}
Here, we regard $\Delta_*$ as $\Q[[T]]$-module via the coefficient expansion.
Since the condition $\Delta_{*}(\Phi_{*}) = \Phi_{*} \cotimes \Phi_{*}$ corresponds to the harmonic product relations of $*$-regularized MZVs, the Condition \eqref{eq:ad-harmonic} gives rise to $\Q$-linear relations among $*$-AdMZVs, which correspond to the binary operation $*$.
Jarossay showed this result in a more general setting:
\begin{thm}[{\cite[Proposition 3.2.5]{Jarossay}}]\label{thm:t-adic}
For $\Phi \in \DMR_0(R)$, we have
\begin{enumerate}
\item\label{thm:t-adic-1} $\Delta_{\shuffle}(\Phi^{-1}x_1\Phi) = \Phi^{-1}x_1\Phi \otimes 1 + 1\otimes \Phi^{-1}x_1\Phi$, and
\item\label{thm:t-adic-2} $\Delta_{*}(\mathbf{q}_{\#}^{\vee}(\Phi^{-1}x_1\Phi)) = \mathbf{q}_{\#}^{\vee}(\Phi^{-1}x_1\Phi) \otimes \mathbf{q}_{\#}^{\vee}(\Phi^{-1}x_1\Phi)$. Here, we regard \(\Delta_*\) as a $R[[T]]$-module morphism via the coefficient expansion.
\end{enumerate}
\end{thm}



\begin{rmk}
Let $t$ be an indeterminate. Ono, Seki and Yamamoto (\cite{Ono-Seki-Yamamoto}) proposed the $t$-adic SMZVs, which is the $t$-adic completion of $\sum_{0\le l<l'} \Admzv{l}{k_1,\ldots,k_r} t^l$, i.e. a $t$-adic SMZV is defined by 
\[
\zeta_{\widehat{\S}}(k_1,\ldots,k_r) := \sum_{0\le l} \Admzv{l}{k_1,\ldots,k_r} t^l \in \mathcal{Z}/\zeta(2)\mathcal{Z} \cotimes \mathbb{Q}[[t]].
\]
Independently, Jarossay introduced the $\Lambda$-adjoint multiple zeta values \cite[Definition~2.3.1(ii)]{Jarossay}. 
They are the same completion, obtained by replacing the indeterminate $t$ with $\Lambda$.
Ono, Seki, and Yamamoto also discussed certain analogs of the double shuffle relations among $t$-adic SMZVs, called the double shuffle relations of $t$-adic SMZVs.
It should be noted that when a linear relation among $t$-adic SMZVs is given, comparing the coefficients of the indeterminate $t$ gives a linear relation among AdMZVs.

Based on the discussion up to this point, we derive two families of linear relations among AdMZVs, Jarossay's one and Ono-Seki-Yamamoto's one.
However, the difference between them is only the shuffle relations.
To be more precise, Theorem \ref{thm:t-adic}. (\ref{thm:t-adic-1}) gives Jarossay's shuffle linear relations, while Ono, Seki, and Yamamoto's shuffle linear relation is
\begin{align*}
\langle \Phi \mid x_0^lx_1(u\shuffle vx_i)\rangle = -\sum_{l_1+l_2 = l} \langle \Phi \mid x_0^{l_1}x_i(x_0^{l_2}\shuffle S_X(v))x_1u\rangle
%\label{eq:AppB-1}
\end{align*}
for all $u$, $v\in X$ and $i = 0$, $1$. 
Jarossay proved that these two types of linear shuffle relations are equivalent (see \cite[Proposition 3.4.1]{Jarossay}).
\end{rmk}



Motivated by Theorem \ref{thm:t-adic}, Jarossay introduced the adjoint double shuffle scheme, defined by the adjoint double shuffle relations.
\begin{df}\label{def:AdDMR}
We define $\AdDMR_0(R)$ as the set of $\Phi \in R\cotimes \ide{h}^{\vee}$ satisfying the following properties:
\begin{enumerate}
%\item\label{def:AdDMR-1} There exists $\phi \in R\cotimes \ide{h}^{\vee}$ with $\Delta_{\shuffle}(\phi) = \phi\otimes \phi$, $\langle \phi \mid\mathbf{1}\rangle = 1$ and $\langle \phi \mid x_0\rangle = \langle \phi \mid x_1\rangle = 0$ such that $\Phi = \phi^{-1}x_1\phi$ holds,
\item\label{def:AdDMR-2} $\Phi - x_1 \in \ide{F}_2^{\ge 4}$,
\item\label{def:AdDMR-4} $\Delta_{\shuffle}(\Phi) = \Phi \otimes 1 + 1\otimes \Phi$, and
\item\label{def:AdDMR-3} $\Delta_{*}(\Phi_{\#}) = \Phi_{\#} \otimes  \Phi_{\#} $, where $\Phi_{\#} := \mathbf{q}_{\#}^{\vee}(\Phi)$.
\end{enumerate}
We define the adjoint double shuffle scheme as a functor $\AdDMR_0 : \Qalg\rightarrow \cat{Set}$ ; $R \mapsto \AdDMR_0(R)$.
%We can check that $\AdDMR_0$ is the affine scheme and $\mathcal{O}(\AdDMR_0) = \mathcal{Z}_{\mathrm{Ad}}^f$. 
\end{df}

By Theorem \ref{thm:t-adic}, there is a morphism of an affine scheme 
\[
\Ad(x_1)  : \DMR_0 \rightarrow \AdDMR_0,
\]
that is, for each $\Q$-algebra $R$, we define $\Ad^R(x_1) : \DMR_0(R) \rightarrow \AdDMR_0(R) ; \phi \mapsto \phi^{-1}x_1\phi$.
Jarossay's question asks whether this natural morphism is an isomorphism.
\begin{qu}[{\cite[Quetion 3.2.8]{Jarossay}}]\label{qu:Jarossay}
Is the morphism of affine schemes
\[
\Ad(x_1) : \DMR_0 \longrightarrow \AdDMR_0
\]
a naturally isomorphism?
\end{qu}


To verify Question~\ref{qu:Jarossay}, we focus on the tangent spaces of $\DMR_0$ and $\AdDMR_0$.
Proving that $\DMR_0$ is an affine group scheme, Racinet showed that the tangent space $\dmr$ at $1$ is a Lie algebra and that the image of $\dmr$ under a certain exponential map acts transitively on $\DMR_0$. 
Consequently, studying $\dmr$ is essentially as informative as studying $\DMR_0$, so we focus on $\dmr$.
Since the natural map $\Ad^{\Q}(x_1): \DMR_0(\Q)\rightarrow \AdDMR_0(\Q)$ sends $1$ to $x_1$, we compare $\dmr$ with the tangent space of $\AdDMR_0(\Q)$ at $x_1$, denoted $\addmr$.
By Racinet, \cite[Definitions in Section 3.3.1]{Racinet}, $\dmr$ is written as
\[
\dmr = \{\psi \in \ide{F}_2^{\ge 3} \mid \Delta_*(\psi_{\star}) = \psi_{\star} \otimes 1 + 1\otimes \psi_{\star} \},
\]
where $\psi_{\star} = \mathbf{q}(\psi) + \sum_{n\ge 2} \frac{(-1)^n}{n} \langle \psi \mid x_0^{n-1}x_1\rangle y_1^n$. 
On the adjoint side, we set as follows:
\begin{df}\label{def:addmr}
We define 
\begin{align*}
\addmr := \{\Psi \in  \ide{h}^{\vee} \mid x_1 + \varepsilon \Psi \in \AdDMR_0(\Q[\varepsilon]/(\varepsilon^2))\}.
\end{align*}
Equivalently, $\addmr$ is the $\Q$-linear space of all $\Psi\in \ide{h}^{\vee}$ such that
\begin{enumerate}
%\item\label{def:addmr--1}  $\wt \Psi \ge 4$,
\item\label{def:addmr--2}  $\Psi \in \ide{F}_2^{\ge 4}$, and 
\item\label{def:addmr--3}  $\Delta_{*}(\Psi_{\#}) = \Psi_{\#} \otimes 1 + 1 \otimes \Psi_{\#}$.
\end{enumerate}
\end{df}
Each $\Q$-linear spaces are decomposed by weight as
\[
\dmr = \prod_{k\ge0}\dmr^{(k)},\qquad
\addmr = \prod_{k\ge0}\addmr^{(k)}.
\]
Here, $\dmr^{(k)}$ (resp. $\addmr^{(k)}$) denotes the homogeneous subspace of weight $k$ in $\dmr$ (resp. $\addmr$).
Then, the table below shows $\dim_\Q \dmr^{(k)}$ up to $k=10$, and $\dim_\Q \addmr^{(k)}$ up to $k=11$ respectively.
\begin{center}
\begin{tabular}{c|ccccccccccccc}
$k$ & $0$ & $1$ & $2$ & $3$ & $4$ & $5$ & $6$ & $7$ & $8$ & $9$ & $10$ & $11$ & $\cdots$ \\
\hline
$\dim \dmr^{(k)}$ & $0$ & $0$ & $0$ & $1$ & $0$ & $1$ & $0$ & $1$ & $1$ & $1$ & $1$ &  & $\cdots$ \\
$\dim \addmr^{(k)}$ & $0$ & $0$ & $0$ & $0$ & $2$ & $2$ & $3$ & $3$ & $4$ & $5$ & $6$ & $7$ & $\cdots$ 
\end{tabular}
\end{center}

Since the derivation $\ad(x_1)$ of $\Ad(x_1)$ raises the weight by one (see next subsection), we expect $\dim \dmr^{(k)} = \dim \addmr^{(k+1)}$ for all $k\in\mathbb{Z}_{\ge 0}$.
However, these data above indicate $\dim \dmr^{(k)} \neq \dim \addmr^{(k+1)}$ for $3 \le k \le 10$, thus Question \ref{qu:Jarossay} is negative.

\begin{comment}
\textcolor{red}{このJarossayの問題について確認するために$\DMR_0(\Q)$の$1$における接平面$\dmr$と、$\AdDMR_0(\Q)$の$x_1$における接平面$\addmr$に着目する。ここで$\AdDMR_0(\Q)$の``$x_1$"における接平面について考察する理由は$1$は$\Ad(x_1)^{\Q} : \DMR_0(\Q)\rightarrow \AdDMR_0(\Q)$により$x_1$に移ることにある。
具体的には$\dmr$と$\addmr$はexplicitにこの様に書き下せる。
($\dmr$と$\addmr$のexplicit formula)
また、$\Ad(x_1) : \DMR_0 \rightarrow \AdDMR_0$の微分写像は$\ad(x_1) : \dmr\rightarrow \addmr ; \psi \mapsto [x_1,\psi]$となる(詳しくは次のsubsectionで)
このとき、$\dmr$と$\addmr$はweight direct product decompositionを持つ。
(式を書く)
ここで、$\dmr^{(k)}$のup to $k = 10$までの次元と$\addmr^{(k)}$のup to $k = 11$までの次元は以下のように書き下せる。
(次元の表)
$\ad(x_1)$がweightを一つ上げる写像であることを考慮すると$\dim \dmr^{(k)} = \dim \addmr^{(k+1)} $が期待されるが、この表からは期待する次元にはならないことがうかがえる。
}
To verify Question \ref{qu:Jarossay}, we computed the dimensions of the Lie algebras associated with $\DMR_0$ and $\AdDMR_0$, and this calculation shows that Question \ref{qu:Jarossay} is probably negative (Appendix \ref{section:computational data}). However, by imposing some additional conditions, we obtain a refined formulation of Jarossay's problem.\footnote{Professor Minoru Hirose at Kagoshima University kindly provided the proto type of the source codes related to the adjoint double shuffle relations; we are grateful for his support.}
\end{comment}



\nsubsection{Affine group scheme $\TM_1$ and corresponding Lie algebra}

From now on, we refine Jarossay's question.
To discuss $\AdDMR_0$, we introduce an affine group scheme $\TM_1$. This setup is based on \cite[Section 1.1.1]{Jarossay2015} and \cite[Proposition A.1.1]{Jarossay2020}.
We define $\TM_1(R) := \{ \Phi \in R\widehat{\otimes}\ide{h}^{\vee} \mid \langle \Phi \mid x_0^k \rangle = 0 \text{ for all $k\in\mathbb{Z}_{\ge 0}$},  \langle \Phi \mid x_1 \rangle = 1 \}.$
We consider the binary operation $\circledast_1$ \cite[Definition 1.1.3]{Jarossay2015} on $\TM_1$ defined by
\[
\Phi_1 \circledast_{1} \Phi_2 = \kappa_{\Phi_1}(\Phi_2)
\]
for $\Phi_1$, $\Phi_2 \in \TM_1(R)$.

\begin{prop}\label{prop;TM1}
A pair $(\TM_1(R), \circledast_1)$ forms a group.
Additionally, let $\TM_1$ be a functor $\Qalg\rightarrow \cat{Grp}$ ; $R\mapsto \TM_1(R)$. 
Then, $\TM_1$ is the prounipotent affine group scheme.
\end{prop}

\begin{proof}

Let $\Phi\in \TM_1(R)$ and define $\tau^R : \TM_1(R) \rightarrow \mathrm{End}_{R\text{-alg}}(R\cotimes\ide{h}^{\vee}) : \Phi \mapsto \kappa_{\Phi}$.
For $m\in\mathbb{Z}_{\ge 0}$, we define the $R$-submodules $R\cotimes \ide{h}^{\vee}_{m}$ of $R\cotimes \ide{h}^{\vee}$ as the ideals $(x_0,x_1)^m$. Since $\kappa_{\Phi}$ is the endomorphism of $R\cotimes \ide{h}^{\vee}$ as an $R$-algebra and $\kappa_{\Phi}(x_i) - x_i$ do not have terms with weight $1$, we have
\begin{enumerate}
\item\label{enumi:prouni-1} $\tau^R(\Phi_1 \circledast_1 \Phi_2) = \tau^R(\Phi_1) \circ \tau^R(\Phi_2)$,
\item\label{enumi:prouni-2} $\tau^R(\Phi_1) = \tau^R(\Phi_2)$ if and only if  $\Phi_1 =  \Phi_2$,
\item\label{enumi:prouni-3} $\kappa_{\Phi} (R\otimes \ide{h}^{\vee}_m) \subset R\otimes \ide{h}^{\vee}_m$, and
\item\label{enumi:prouni-4} the following composition of the $R$-module map
\[
\xymatrix{
R\cotimes \ide{h}^{\vee} \ar[r]^{\kappa_{\Phi}|_{R\cotimes \ide{h}^{\vee}_m}} & R\cotimes \ide{h}^{\vee}_m  \ar@{->>}[r]&  R\cotimes \ide{h}^{\vee}_m/  \ide{h}^{\vee}_{m+1}
}
\]
is the identity.
\end{enumerate}
Therefore, if we show that $\TM_1(R)$ equips the structure of a group with respect to $\circledast_1$, then $\tau^R$ is the $R$-module representation of $\TM_1(R)$ and satisfies the conditions of the prounipotency.

Given these preliminary considerations, we show the group structure of $\TM_1(R)$.
By the definition of $\circledast_1$, we can immediately check that $x_1$ is the identity element of $\TM_1(R)$.
The associativity law of $\circledast_1$ is given as follows:
Let $\Phi_1$, $\Phi_2$, and $\Phi_3 \in \TM_1(R)$. Since $\kappa_{\Phi_1 \circledast_1 \Phi_2} = \tau^R (\Phi_1 \circledast_1 \Phi_2) = \tau^R (\Phi_1) \circ \tau^R (\Phi_2) = \kappa_{\Phi_1} \circ \kappa_{\Phi_2}$ holds, we have
\[
(\Phi_1\circledast_1 \Phi_2) \circledast_1 \Phi_3 
= \kappa_{\Phi_1\circledast_1 \Phi_2}(\Phi_3)
=\kappa_{\Phi_1} \circ \kappa_{\Phi_2} (\Phi_3)
=\kappa_{\Phi_1} (\Phi_2 \circledast_1 \Phi_3)
=\Phi_1 \circledast_1(\Phi_2 \circledast \Phi_3).
\]
Lastly, we show the existence of the inverse $\Phi'$ of $\Phi\in \TM_1(R)$.
We shall illustrate it constructively.
By the definition of $\TM_1(R)$, the coefficients of $\Phi'$ in weight $1$  must be $\langle \Phi' \mid x_1 \rangle = 1$ and $\langle \Phi' \mid x_0 \rangle = 0$.
%We put $w\in X^*$ with the weight degree at least $2$ and assume $\Phi'$ is determined up to degree $\deg w$.
Let $w\in X^*$ with $2\le \wt w$. Since $\kappa_{\Phi}$ satisfies the \eqref{enumi:prouni-3} and \eqref{enumi:prouni-4} as mentinoned in this proof, we have
\[
 \langle \kappa_{\Phi}(\Phi') \mid w \rangle = \langle \Phi' \mid w \rangle - \sum_{\wt w' < \wt w} c_{w',w} \langle \Phi' \mid w' \rangle
\]
for $c_{w',w} \in R$. Therefore, by recursively putting $\langle \Phi' \mid w \rangle =  \sum_{\wt w' < \wt w} c_{w',w} \langle \Phi' \mid w' \rangle$, we find out $\Phi'$ to be the inverse of $\Phi$ with respect to $\circledast_1$.


\end{proof}


\begin{rmk}
One may notice that the product $\circledast_1$ on $\TM_1(R)$ is defined so that $\Ad^R(x_1) : (\TM(R) , \circledast) \rightarrow (\TM_1(R),\circledast_1)$ is a group homomorphism \cite[Proposition 1.1.4 (ii)]{Jarossay2015}, where $\TM(R):=\{\Phi\in R\widehat{\otimes}\ide{h}^{\vee}\mid \langle \Phi,1\rangle=1\}$, equipped with the product $\circledast$.
One also notes that $\DMR_0(R)$ is a subgroup of $\TM(R)$.
\end{rmk}





We now define the corresponding Lie algebra of $\TM_1$. We define
\[
\tm_1 := \ker(\TM_1(\Q[\varepsilon]) \xrightarrow{\varepsilon = 0} \TM_1(\Q)).
\]
A direct calculation shows
\[
\tm_1= \{\Psi \in  \ide{h}^{\vee} \mid \langle \Psi \mid x_1 \rangle = 0,  \langle \Psi \mid x_0^l \rangle = 0 \text{ ($l\in\mathbb{Z}_{\ge 0}$)}\}.
\]
For $\Psi_1$, $\Psi_2 \in \tm_1$, we define $\Q$-bilinear binary operation $\{\mathchar`-,\mathchar`-\}_1$ \cite[Appendix A.1.2]{Jarossay2020} on $\tm_1$ by
\[
\{\Psi_1,\Psi_2\}_{1} := d_{\Psi_1}(\Psi_2) - d_{\Psi_2}(\Psi_1).
\]
Here, for $\Psi \in \ide{h}^{\vee}$, we define the derivation $d_{\Psi} : \ide{h}^{\vee} \rightarrow \ide{h}^{\vee}$ by
\[
d_{\Psi}(x_0) = 0 \;\;,\;\;d_{\Psi}(x_1)= \Psi.
\] 
Then, we have the following.

\begin{prop}
A pair $(\tm_1,\{\mathchar`-,\mathchar`-\}_1)$ froms a Lie algebra.
\end{prop}

\begin{proof}
Let $\der( \ide{h}^{\vee})$ be a derivation on $\ide{h}^{\vee}$. Then, $\der (\ide{h}^{\vee})$ is the corresponding Lie algebra of an affine group scheme $\Aut(\ide{h^{\vee}}) : \Qalg \rightarrow \Grp ; R\mapsto \Aut_{R\text{-\textbf{alg}}}(R\cotimes \ide{h^{\vee}})$.

We focus on the corresponding Lie algebra homomorphism $d\tau : \tm_1\rightarrow \der(\ide{h}^{\vee})$ of $\tau^\Q : \TM_1(\Q) \rightarrow \Aut_{\Qalg}(\ide{h}^{\vee})$ defined by $\tau_{1 + \varepsilon \Psi} = \id + \varepsilon d\tau(\Psi) $ ($1 + \varepsilon \Psi \in \ker (\TM_1(\Q[\varepsilon]) \xrightarrow{\varepsilon = 0} \TM_1(\Q))$).
Then by construction, $d\tau$ is a Lie algebra homomorphism and the image of $d\tau$ is a derivation on $ \ide{h}^{\vee}$.
Since 
\begin{align*}
\tau(1 + \varepsilon)(x_0) = &\kappa_{1 + \varepsilon \Psi}(x_0) = x_0 = x_0 + \varepsilon \cdot 0,\\
\tau(1 + \varepsilon)(x_1) = &\kappa_{1 + \varepsilon \Psi}(x_1) = 1 + \varepsilon \Psi,
\end{align*}
we have $d\tau(\Psi)(x_0) = 0$ and $d\tau(\Psi)(x_1) = \Psi$. This implies that $d\tau(\Psi) = d_{\Psi}$ holds.
Let $\langle\mathchar`-, \mathchar`-\rangle$ be the Lie bracket equipped in $\tm_1$.
Then, we have
\begin{align*}
\langle\Psi_1 ,\Psi_2\rangle =  d\tau (\langle \Psi_1, \Psi_2\rangle)(x_1)
 = [d\tau(\Psi_1), d\tau(\Psi_2)](x_1) = d_{\Psi_1}(\Psi_2) - d_{\Psi_2}(\Psi_1) = \{\Psi_1,\Psi_2\}_1.
\end{align*}
The second equality above follows from the fact that $d\tau$ is a Lie algebra homomorphism.
Therefore, $\langle\mathchar`-,\mathchar`-\rangle = \{\mathchar`-,\mathchar`-\}_1$ holds.

\end{proof}

\begin{cor}\label{cor:tm-freeLie}
The intersection of $\Q$-linear spaces $\tm_1 \cap  \ide{F}_2$ is a Lie subalgebra of $\tm_1$.
\end{cor}

\begin{proof}
Let $\Psi_1,\Psi_2\in \tm_1\cap \ide{F}_2$.
Because $d_{\Psi_1}$ is a derivation of the Lie algebra, it follows that $d_{\Psi_1}(\Psi_2)\in \tm_1\cap \ide{F}_2$. Consequently, ${\Psi_1,\Psi_2}\in \tm_1\cap \ide{F}_2$.
\end{proof}

\begin{comment}
By Proposition \ref{prop:unipotent}. (\ref{enumi:unipotent-1}) provides the exponential map $\exp^{\circledast} : \tm_1\rightarrow \TM_1$ which satisfies
\[
\tau\circ \exp^{\circledast_1}) = \exp \circ d\tau.
\]
Substituting $x_1$ into the above, $\tau(f)(\phi) = \phi$ for $\phi \in \TM_1(R)$ implies
\[
\exp^{\circledast} = \tau(\exp^{\circledast_1})(x_1) = \exp (d\tau)(x_1).
\]
\end{comment}

\nsubsection{A Lie algebra $\Fad$ and an affine group scheme $\FAd$}
For $f \in \ide{h}^{\vee}$, we define a $\Q$-linear map $\ad(f) : \ide{h}^{\vee}\rightarrow \ide{h}^{\vee} ; g \mapsto [f,g]$ and a $\Q$-linear space
\[
\ide{F}_2^{\ad_{x_1}} :=  \left\{ \Psi \in \ide{F}^{\ge 3}\middle|
\Psi^{00} = 0
\right\},
\]

which is inspired by the following Schneps's proposition.

\begin{prop}[{\cite[Proposition 2.2]{Schneps}}]\label{prop:Schneps}
Let $\Psi \in  \ide{h}^{\vee}$.
Then, the conditions
\begin{itemize}
\item $\Psi \in  \ide{F}_2^{\ge 3}$ and $\Psi^{00} = 0$ holds, and
\item there exists $\psi \in\ide{F}_2^{\ge 2}$ such that
\[
\Psi = [x_1,\psi],
\]
\end{itemize}
are equivalent.
\end{prop}


This proposition implies
\[
 \Fad  = \left\{ \Psi \in \ide{F}_2^{\ge 3} \middle|
\begin{matrix}
\text{${}^{\exists}\psi \in \ide{F}_2^{\ge 2}$ such that }\Psi = [x_1,\psi]
\end{matrix}
\right\}.
\]

\begin{prop}[{\cite[Proposition A.1.1]{Jarossay2020}}]\label{prop:Fad}
The $\Q$-linear space $\Fad$ is a Lie subalgebra of $\tm_1$.
\end{prop}

\begin{proof}
We focus on $\ide{F}_2^{\ge 2}$.
Then, $\ide{F}_2^{\ge 2}$ forms a Lie algebra with respect to $\{\mathchar`-,\mathchar`-\}$. Here, for $\phi$, $\psi \in \ide{F}_2^{\ge 2}$, we define $\{\phi,\psi\} := d_{[x_1,\phi]}(\psi) - d_{[x_1,\psi]}(\phi) + [\phi,\psi]$ as in \cite[Proposition 3.5, Corollary 3.6, and Proposition 3.7]{Racinet}.
We can easily see that the image of the mapping $\ad(x_1) : \ide{F}_2^{\ge 2} \rightarrow \tm_1 ; \phi \mapsto [x_1,\phi]$ is equal to $\Fad$ by Proposition \ref{prop:Schneps}. Therefore, it suffices to show that $\ad(x_1)$ is a Lie algebra homomorphism.

Let $\psi_1$, $\psi_2 \in \ide{F}_2^{\ge 2}$. Then we have
\begin{align*}
\ad(x_1)(\{\psi_1,\psi_2\}) =& [x_1,\{\psi_1,\psi_2\}]\\
=&[x_1, d_{[x_1,\psi_1]}(\psi_2) - d_{[x_1,\psi_2]}(\psi_1) + [\psi_1,\psi_2]]\\
=&[x_1,d_{[x_1,\psi_1]}(\psi_2)] - [x_1,d_{[x_1,\psi_2]}(\psi_1)] + x_1\psi_1\psi_2 + \psi_2\psi_1 x_1 - x_1\psi_2\psi_1 - \psi_1\psi_2x_1\\
=&[x_1,d_{[x_1,\psi_1]}(\psi_2)]  + [x_1,\psi_1]\psi_2 + \psi_1x_1\psi_2 - \psi_2[x_1 , \psi_1] + \psi_2x_1\psi_1\\
&-[x_1,d_{[x_1,\psi_2]}(\psi_1)] -[x_1,\psi_2]\psi_1 - \psi_2x_1\psi_1 + \psi_1[x_1,\psi_2] - \psi_1x_1\psi_2\\
=&[x_1,d_{[x_1,\psi_1]}(\psi_2)] +  [d_{[x_1,\psi_1]}(x_1),\psi_2] - [x_1,d_{[x_1,\psi_2]}(\psi_1)] - [d_{[x_1,\psi_2]}(x_1),\psi_1]\\
=&d_{[x_1,\psi_1]}(\ad(x_1)(\psi_2)) - d_{[x_1,\psi_2]}(\ad(x_1)(\psi_1)) \\
=&\{\ad(x_1)(\psi_1), \ad(x_1)(\psi_2) \}_1
\end{align*}
as claimed.
\end{proof}

\begin{rmk}
We can check $\ker \ide{ad}(x_1) = \Q\langle\langle x_1 \rangle \rangle $.
Thus, putting $\widetilde{\ide{F}}_2 := \{\Phi \in \ide{h}^{\vee} \mid \langle \Phi \mid x_1^n \rangle = 0\text{ for $n \in \mathbb{Z}_{>0}$}  \}$, we have an isomorphism of Lie algebras $\ide{ad}(x_1) : \widetilde{\ide{F}}_2^{\ge 2} \rightarrow \Fad$.
\end{rmk}
Let us define
\[
\FAd(R) := 
\left\{
\Phi \in R\cotimes \ide{h}^{\vee} 
\middle|
\begin{matrix}
\Phi - x_1 \in \ide{F}_2^{\ge 3}\text{ and}\\
\text{ ${}^{\exists} \phi \in \exp(\ide{F}_2)$ such that }\Phi=\phi^{-1}x_1\phi.
\end{matrix}
\right\}
\]
and a functor $\FAd : \Qalg \rightarrow \Set ; R \mapsto \FAd(R)$.
In a similar way to Proposition \ref{prop:Fad}, we can check that $\FAd(R)$ is a subgroup of $\TM_1(R)$, and this implies that $\FAd$ is a closed affine subscheme of $\TM_1$ as an affine group scheme.
Additionally, a corresponding Lie algebra of $\FAd$ is $\Fad$.

Let $f$ be the inverse map of $\ad(x_1)\mid_{\widetilde{\ide{F}}_2^{\ge 2}}$. 
The defining equations that define $\FAd$ are as follows:
\begin{prop}\label{prop:Adjoint}
Let $\Phi-x_1\in \ide{F}_2^{\ge 3}$ and write
\[
\Phi-x_1=\sum_{k\ge 3}\Phi^{(k)} \qquad (\Phi^{(k)} \text{: homogeneous of weight }k).
\]
Define sequences $\{U_n\}_{n\ge 3}\subset \ide{F}_2$ and 
$\{\psi_n\}_{n\ge 2}\subset \ide{F}_2$ recursively by
\[
U_3:=0,\qquad U_4:=0,
\]
and for $n\ge 5$ set
\[
U_n
:=\sum_{r\ge 2}\frac{(-1)^r}{r!}
\!\!\!\sum_{\substack{m_1,\dots,m_r\ge 2\\ m_1+\cdots+m_r=n-1}}
\!\!\!
\ad(\psi_{m_1}) \circ\cdots \circ\ad(\psi_{m_r})(x_1),
\]
then put, for every $n\ge 3$,
\[
\psi_{n-1}:=f(\Phi^{(n)} - U_n).
\]
Let $\psi:=\sum_{m\ge 2}\psi_m$ and $\phi:=\exp(-\psi)$. If $(U_n - \Phi^{(n)})^{00} = 0$ for $n\ge 3$, then $\Phi = \phi^{-1}x_1\phi$. Thus, $\Phi \in \FAd(\Q)$.
Conversely, if $\Phi \in \FAd(\Q)$, then $(U_n - \Phi^{(n)})^{00} = 0$ for all $n \in \mathbb{Z}_{\ge 3}$.
\end{prop}

\begin{proof}
Recall the adjoint identity:
\[
\exp(-\psi)\,x_1\,\exp(\psi)
= \exp(\ad(-\psi))(x_1)
= x_1+\sum_{r\ge 1}\frac{1}{r!}\,\ad(\psi)^{\,r}(x_1).
\]
Write $\psi=\sum_{m\ge 2}\psi_m$ with $\psi_m$ homogeneous of weight $m$.
The weight $n$ part of $\ad(\psi)^{\,r}(x_1)$ is the sum over
$r$-tuples $(m_1,\ldots,m_r)$ with $m_i\ge 2$ and $m_1+\cdots+m_r=n-1$:
\[
\bigl(\ad(\psi)^{\,r}(x_1)\bigr)^{(n)}
=
\sum_{r\ge 0} \frac{1}{r!}\sum_{\substack{m_1,\dots,m_r\ge 2\\ m_1+\cdots+m_r=n-1}}
\ad(-\psi_{m_1})\circ \cdots \circ\ad(-\psi_{m_r})(x_1).
\]
Separating the $r=1$ gives
\begin{equation}\label{eq:weight-n-expansion}
\bigl(\exp(-\psi)\,x_1\,\exp(\psi)\bigr)^{(n)}
=
\ad(x_1)(\psi_{n-1})+
\sum_{r\ge 2}\frac{(-1)^r}{r!}\sum_{\substack{m_1,\dots,m_r\ge 2\\ m_1+\cdots+m_r=n-1}}\ad(\psi_{m_1})\circ \cdots \circ\ad(\psi_{m_r})(x_1).
\end{equation}
By definition, the second term is exactly $U_n$.

We first show that if $(\Phi^{(n)} - U_n)^{00}=0$ for all $n$, then $\Phi \in \FAd(\Q)$.
To begin with, we inductively construct $\psi_n\in \ide{F}_2$ so that, putting $\psi=\sum_{m\ge2}\psi_m$ and $\phi:=\exp(-\psi)\in \exp(\ide{F}_2)$, we obtain $\Phi=\phi^{-1}x_1\phi$.
For $n=3,4$, since $U_3=U_4=0$, set $\psi_2:=f(\Phi^{(3)})$ and $\psi_3:=f(\Phi^{(4)})$.
Assume $\psi_2,\dots,\psi_{n-2}$ have been constructed. Then $U_{n+1}$ is determined by $\psi_2,\dots,\psi_{n-2}$.
By the assumption $(\Phi^{(n+1)}-U_{n+1})^{00}=0$ and Proposition \ref{prop:Schneps}, there exists
\[
\psi_n:=f\!\bigl(\Phi^{(n+1)}-U_{n+1}\bigr)\in \ide{F}_2
\quad\text{with}\quad
\ad(x_1)(\psi_n)=\Phi^{(n+1)}-U_{n+1}.
\]
This completes the inductive construction of $\psi$ and $\phi=\exp(-\psi)$.

Next, we prove that for $n\ge3$,
\[
(\phi^{-1}x_1\phi\bigr)^{(n)}=\Phi^{(n)}.
\]
By \eqref{eq:weight-n-expansion} and the definition of $U_n$, it follows that
\[
\bigl(\exp(-\psi)\,x_1\,\exp(\psi)\bigr)^{(n)}
=\ad(x_1)(\psi_{\,n-1})+U_n
=\bigl(\Phi^{(n)}-U_n\bigr)+U_n
=\Phi^{(n)}.
\]
So we have $\Phi=\phi^{-1}x_1\phi$.

Conversely, let $\Phi=\exp(-\psi)\,x_1\,\exp(\psi) \in \Fad(\Q)$ with
$\psi=\sum_{m\ge 2}\psi_m$.
Then \eqref{eq:weight-n-expansion} yields
\[
\Phi^{(n)} - U_n=\ad(x_1)(\psi_{n-1}),
\]
so $(\Phi^{(n)} - U_n)^{00}=0$ for all $n\ge 3$.
\end{proof}

\begin{rmk}
Considering coefficient expansion, Proposition \ref{prop:Adjoint} holds over any $\Q$-algebra $R$.
\end{rmk}

We now refine Question~\ref{qu:Jarossay}.

\begin{qu}\label{qu:Jarossay}
Is there a natural isomorphism of affine group schemes $\Ad^R(x_1) : \DMR_0 \rightarrow \AdDMR_0 \times_{\TM_1} \FAd$?
\end{qu}

Here, the fiber product over $\TM_1$ is an intersection in the sense of affine schemes. Note that $\AdDMR_0$ and $\FAd$ is emmbeded in $\TM_1$.
As in the previous subsection, we consider tangent spaces of $\DMR_0$ and $\AdDMR_0 \times_{\TM_1} \FAd$.
We now state the question in terms of tangent spaces.

\begin{qu}\label{qu:Jarossay-Lie}
Is the $\Q$-linear map $\ad(x_1) : \dmr \rightarrow \addmr \cap \Fad ; \psi \mapsto [x_1,\psi]$ isomorphic?
\end{qu}
Comparing dimensions of the homogeneous components of $\dmr$ and $\addmr\cap\FAd$ we obtain
\begin{center}
\begin{tabular}{c|ccccccccccccc}
$k$ & $0$ & $1$ & $2$ & $3$ & $4$ & $5$ & $6$ & $7$ & $8$ & $9$ & $10$ & $11$ & $\cdots$ \\
\hline
$\dim \dmr^{(k)}$ & $0$ & $0$ & $0$ & $1$ & $0$ & $1$ & $0$ & $1$ & $1$ & $1$ & $1$ &  & $\cdots$ \\
$\dim \addmr^{(k)} \cap \Fad$
& $0$ & $0$ & $0$ & $0$ & $1$ & $0$ & $1$ & $0$ & $1$ & $1$ & $1$ & $1$ & $\cdots$ \\
\end{tabular}
\end{center}


\ifdefined\isMainDocument
\else
    %\bibliographystyle{amsplain}
    %\bibliography{reference-adjoint}
    \end{document}
\fi

\ifdefined\isMainDocument
\else
    \documentclass[10pt,oneside,reqno]{amsart}
    \usepackage{xr}
    \externaldocument{../main} % Main の .aux を参照
    \usepackage{latexsym}
\usepackage{amscd}
\usepackage{amsmath}
\usepackage{amssymb}
\usepackage[mathscr]{euscript}
\usepackage{graphicx}
\usepackage{geometry}
\usepackage{mathtools}
\usepackage[all]{xy}
\usepackage[dvipsnames]{xcolor}
\usepackage[colorlinks,final,hyperindex]{hyperref}
\usepackage{tikz}
\usepackage{tikz-cd}
\usetikzlibrary{decorations.pathmorphing,decorations.markings,arrows,calc,shapes.geometric,arrows.meta,positioning}
\usepackage{enumerate}
\usepackage{multirow}

\usepackage[bb=dsserif]{mathalpha}
\usepackage{bm}

\newcommand{\QQ}{\ensuremath{\mathbb{Q}}}
\newcommand{\DD}{\ensuremath{\mathbb{D}}}
\newcommand{\CC}{\ensuremath{\mathbb{C}}}
\newcommand{\RR}{\ensuremath{\mathbb{R}}}
\newcommand{\ZZ}{\ensuremath{\mathbb{Z}}}
\newcommand{\NN}{\ensuremath{\mathbb{N}}}

\newcommand{\g}{\ensuremath{\mathfrak{g}}}
\renewcommand{\SS}{\ensuremath{\mathbb{S}}}

\newcommand{\id}{\ensuremath{\mathrm{id}}}

\makeatletter

\newcommand{\bbone}{\mathbb{1}}

\newcommand{\calA}{\mathcal{A}}
\newcommand{\calB}{\mathcal{B}}
\newcommand{\calC}{\mathcal{C}}
\newcommand{\calD}{\mathcal{D}}
\newcommand{\calE}{\mathcal{E}}
\newcommand{\calF}{\mathcal{F}}
\newcommand{\calG}{\mathcal{G}}
\newcommand{\calH}{\mathcal{H}}
\newcommand{\calI}{\mathcal{I}}
\newcommand{\calJ}{\mathcal{J}}
\newcommand{\calK}{\mathcal{K}}
\newcommand{\calL}{\mathcal{L}}
\newcommand{\calM}{\mathcal{M}}
\newcommand{\calN}{\mathcal{N}}
\newcommand{\calO}{\mathcal{O}}
\newcommand{\calP}{\mathcal{P}}
\newcommand{\calQ}{\mathcal{Q}}
\newcommand{\calR}{\mathcal{R}}
\newcommand{\calS}{\mathcal{S}}
\newcommand{\calT}{\mathcal{T}}
\newcommand{\calU}{\mathcal{U}}
\newcommand{\calV}{\mathcal{V}}
\newcommand{\calW}{\mathcal{W}}
\newcommand{\calX}{\mathcal{X}}
\newcommand{\calY}{\mathcal{Y}}
\newcommand{\calZ}{\mathcal{Z}}
\newcommand{\kk}{\Bbbk}

\newcommand{\scrY}{\mathscr{Y}}
\newcommand{\scrV}{\mathscr{V}}
\newcommand{\scrP}{\mathscr{P}}

\newcommand{\Rep}{\mathop{\mathrm{Rep}}\nolimits}
\newcommand{\alg}{\mathop{\mathrm{alg}}\nolimits}
\newcommand{\Span}{\mathop{\mathrm{Span}}\nolimits}
\newcommand{\proj}{\mathop{\mathrm{proj}}\nolimits}
\newcommand{\Hom}{\mathop{\mathrm{Hom}}\nolimits}
\newcommand{\PHom}{\mathop{\mathrm{PHom}}\nolimits}
\newcommand{\Tw}{\mathop{\mathrm{Tw}}\nolimits}
\newcommand{\Dim}{\mathop{\mathrm{Dim}}\nolimits}
\newcommand{\Imm}{\mathop{\mathrm{Im}}\nolimits}
\newcommand{\Aus}{\mathop{\mathrm{Aus}}\nolimits}
\newcommand{\Ann}{\mathop{\mathrm{Ann}}\nolimits}
\newcommand{\APC}{\mathop{\mathrm{APC}}\nolimits}
\newcommand{\Barr}{\mathop{\mathrm{Bar}}\nolimits}
\newcommand{\Cone}{\mathop{\mathrm{Cone}}\nolimits}
\newcommand{\modu}{\mathop{\mathrm{mod}}\nolimits}
\newcommand{\dgMod}{\mathop{\mathrm{dgMod}}\nolimits}
\newcommand{\soc}{\mathop{\mathrm{soc}}\nolimits}
\newcommand{\dg}{\mathop{\mathrm{dg}}\nolimits}
\newcommand{\red}{\mathop{\mathrm{red}}\nolimits}
\newcommand{\Map}{\mathop{\mathrm{Map}}\nolimits}
\newcommand{\inj}{\mathop{\mathrm{inj}}\nolimits}
\newcommand{\op}{\mathop{\mathrm{op}}\nolimits}
\newcommand{\Ker}{\mathop{\mathrm{Ker}}\nolimits}
\newcommand{\Tor}{\mathop{\mathrm{Tor}}\nolimits}
\newcommand{\Tot}{\mathop{\mathrm{Tot}}\nolimits}
\newcommand{\Ext}{\mathop{\mathrm{Ext}}\nolimits}
\newcommand{\sg}{\mathop{\mathrm{sg}}\nolimits}
\newcommand{\Sl}{\mathop{\mathfrak{sl}}\nolimits}
\newcommand{\nc}{\mathop{\mathrm{nc}}\nolimits}
\newcommand{\Tr}{\mathop{\mathrm{Tr}}\nolimits}
\newcommand{\Irr}{\mathop{\mathrm{Irr}}\nolimits}
\newcommand{\HC}{\mathop{\mathrm{HC}}\nolimits}
\newcommand{\HH}{\mathop{\mathrm{HH}}\nolimits}
\newcommand{\THH}{\mathop{\mathrm{TH}}\nolimits}
\newcommand{\Perf}{\mathop{\mathrm{Perf}}\nolimits}
\newcommand{\del}{\partial}
\newcommand{\ev}{\mathop{\mathrm{ev}}\nolimits}
\newcommand{\co}{\mathop{\mathrm{co}}\nolimits}
\newcommand{\pt}{\mathop{\mathrm{pt}}\nolimits}
\newcommand{\wt}{\mathop{\mathrm{wt}}\nolimits}
\newcommand{\pCY}{\mathop{\mathrm{pCY}}\nolimits}
\newcommand{\gr}{\mathop{\mathrm{gr}}\nolimits}
\newcommand{\coker}{\mathop{\mathrm{coker}}\nolimits}


\newcommand{\Top}{\mathop{\mathrm{Top}}\nolimits}
\newcommand{\Set}{\mathop{\mathrm{Set}}\nolimits}
\newcommand{\Vect}{\mathop{\mathrm{Vect}}\nolimits}
\newcommand{\Mod}{\mathop{\mathrm{Mod}}\nolimits}
\newcommand{\Ob}{\mathop{\mathrm{Ob}}\nolimits}
\newcommand{\Ind}{\mathop{\mathrm{Ind}}\nolimits}
\newcommand{\Sym}{\mathop{\mathrm{Sym}}\nolimits}
\newcommand{\Alt}{\mathop{\mathrm{Alt}}\nolimits}
\newcommand{\End}{\mathop{\mathrm{End}}\nolimits}
\newcommand{\Ch}{\mathop{\mathrm{Ch}}\nolimits}


\newcommand{\Ass}{\mathop{\mathrm{Ass}}\nolimits}
\newcommand{\Com}{\mathop{\mathrm{Com}}\nolimits}
\newcommand{\Lie}{\mathop{\mathrm{Lie}}\nolimits}
\newcommand{\colim}{\mathop{\mathrm{colim}}\nolimits}

\newcommand{\Op}{\mathop{\mathsf{Operads}}\nolimits}
\newcommand{\Diop}{\mathop{\mathsf{Dioperads}}\nolimits}
\newcommand{\PDiop}{\mathop{\mathsf{PDioperads}}\nolimits}
\newcommand{\Coop}{\mathop{\mathsf{Cooperads}}\nolimits}
\newcommand{\Codiop}{\mathop{\mathsf{Codioperads}}\nolimits}
\newcommand{\PCodiop}{\mathop{\mathsf{PCodioperads}}\nolimits}

\newcommand{\antish}{\ensuremath{\mbox{!`}}}


\tikzset{->-/.style={decoration={
			markings,
			mark=at position #1 with {\arrow{>}}},postaction={decorate}}}
\tikzset{-w-/.style={decoration={
			markings,
			mark=at position #1 with {\arrow{Stealth[fill=white,scale=1.4]}}},postaction={decorate}}}
\tikzset{->-/.default=0.65}
\tikzset{-w-/.default=0.65}
\tikzstyle{bullet}=[circle,fill=black,inner sep=0.5mm]
\tikzstyle{circ}=[circle,draw=black,fill=white,inner sep=0.5mm]
\tikzstyle{vertex}=[circle,draw=black,thick,inner sep=0.5mm]
\tikzstyle{dot}=[draw,circle,fill=black,minimum size=0.5mm,inner sep = 0mm, outer sep = 0mm]
\tikzset{darrow/.style={double distance = 4pt,>={Implies},->},
	darrowthin/.style={double equal sign distance,>={Implies},->},
	tarrow/.style={-,preaction={draw,darrow}},
	qarrow/.style={preaction={draw,darrow,shorten >=0pt},shorten >=1pt,-,double,double
		distance=0.2pt}}
\newcommand\tikcirc[1][2.5]{\tikz[baseline=-#1]{\draw[thick](0,0)circle[radius=#1mm];}}
\newcommand{\tikzfig}[1]{\begin{tikzpicture}[auto,baseline={([yshift=-.5ex]current bounding box.center)}]#1\end{tikzpicture}}
\usetikzlibrary{decorations.pathreplacing}

\usepackage{amsthm}
\usepackage{thmtools}
\usepackage[noabbrev,capitalize]{cleveref}
\newtheorem{theorem}{Theorem}
\newtheorem{lemma}[theorem]{Lemma}
\newtheorem{proposition}[theorem]{Proposition}
\newtheorem{corollary}[theorem]{Corollary}
\newtheorem{conjecture}[theorem]{Conjecture}
\newtheorem{Claim}[theorem]{Claim}

\newtheorem*{theorem*}{Theorem}

\theoremstyle{definition}
\newtheorem{definition}{Definition}
\newtheorem{example}{Example}
\newtheorem{cexample}{Counterexample}
\newtheorem{exercise}{Exercise}
\newtheorem{question}{Question}
\newtheorem{remark}{Remark}

\addtolength{\textheight}{1.75cm}
\addtolength{\voffset}{-0.5cm}
\addtolength{\footskip}{+0.6cm}
\geometry{margin=1.2in}


\usepackage[backend=biber, maxbibnames=99, bibencoding=utf8, doi=false, isbn=false, url=false, style=alphabetic]{biblatex}
\addbibresource{refs.bib}

    \begin{document}
\fi

%\setcounter{section}{4}
\section[The main Theorem]{The Main Theorem}\label{sect:Main Theorem}
% and its bigraded version  $\adls$}



\nsubsection{Parity results}

Let us define $\mathcal{Z}_{>0} := \Span_{\mathbb{Q}} \{\zeta(\mathbf{k}) \mid \mathbf{k} \neq \emptyset\}$.
In this subsection, we put  $R = \mathcal{Z} / (\mathcal{Z}_{>0}^2 +\Q \pi^2)$.
In our main theorem, we utilize the parity results, which state that $\zeta(k_1,\ldots,k_r)$ with $k_1+\dots + k_r + r \equiv 1 \mod 2$ lies in $\Q[\pi^2]$-span of MZVs of depth less than $r$ (The depth of MZV means the number of entries in an argument of MZV).
These parity results are first proved analytically in \cite{Tsumura} and later algebraically proved in \cite{Ihara-Kaneko-Zagier}.
In \cite{Hirose-parity}, Hirose provided an explicit formula of the parity results from the point of view of the multitangent functions.
The significant point of his proof is the following formula, which follows from the functional equation of multitangent functions
\begin{thm}[\cite{Hirose-parity}]
For $(k_1,\ldots,k_r)\in\mathbb{Z}_{>0}^r$, we have
\begin{align}
\begin{aligned}
&\sum_{j = 0}^r (-1)^{k_{j+1}+\cdots + k_r} \zeta^*(k_{j},\ldots,k_1)\zeta^*(k_{j+1},\ldots,k_r) \\
=&\delta^{k_1,\ldots,k_r} + \sum_{j = 1}^r \sum_{\substack{a+2m+b = k_j \\ a,b,m\ge 0}} (-1)^{b+k_{j+1}+\cdots +k_r + m + 1}\\
&\times \frac{(2\pi )^{2m}}{(2m)!}B_{2m}\zeta_a^*(k_{j-1},\ldots,k_1)\zeta_b^*(k_{j+1},\ldots,k_d).
\end{aligned}
\label{eq:parity}
\end{align}
\end{thm}
We omit the definition of $\delta^{k_1,\ldots,k_r}$ but only note that $\delta^{k_1,\ldots,k_d} \in \Q[\pi^2]$.
The left-hand side of \eqref{eq:parity} is no longer AdMZVs, but we must note that AdMZVs appear after tanking modulo some sums of a product of MZVs and $\pi^2$.
From this perspective, we determine the $\Q$-linear relations of AdMZVs.

By definition of AdMZVs, we have
\begin{align*}
\Admzv{0}{k_1,\ldots,k_r} &\equiv \zeta(k_1,\ldots,k_r) + (-1)^{k_1+\cdots + k_r}\zeta(k_r,\ldots,k_1) \\
\Admzv{l}{k_1,\ldots,k_r} &\equiv (-1)^{k_1+\cdots + k_r + l} \zeta_l(k_r,\ldots,k_1)
\end{align*}
modulo $\mathcal{Z}_{>0}^2 +\Q \pi^2$ for arbitrary $k_1,\ldots,k_r\in\mathbb{Z}_{>0}$ and $l\in\mathbb{Z}_{ \ge 1 }$.

Considering \eqref{eq:parity} as the image of $R$, we obtain
\begin{align*}
&(-1)^{k_1+\cdots + k_r}(\zeta^*(k_1,\ldots,k_r) + (-1)^{k_1+\cdots + k_r} \zeta^*(k_r,\ldots,k_1))\\
\equiv &- \zeta_{k_r}^*(k_{r-1},\ldots,k_1) - (-1)^{k_1+\cdots+k_r} \zeta_{k_1}(k_2,\ldots,k_r)
\end{align*}
as an equality of $R$.
Hence, the parity result of AdMZVs can be written as
\begin{align}
\zeta_{\mathrm{Ad}}(k_1,\ldots,k_r;0)
\equiv&- \zeta_{\mathrm{Ad}}(k_1,\ldots,k_{r-1};k_r) - (-1)^{k_1+\cdots + k_r} \zeta_{\mathrm{Ad}}(k_r,\ldots,k_{2};k_1)  \label{eq:parity-s}
\end{align}
modulo $\mathcal{Z}_{>0}^2 +\Q \pi^2$.
%This is the precise $\Q$-linear relations among AdMZVs that we seek.

Next, we determine the relation for the commutative generating function derived from \eqref{eq:parity-s}.
From \eqref{eq:parity-s}, we get the following equality in $R\cotimes \ide{h}^{\vee}$:

\begin{prop}
Let $\Phi := \Phi_{\shuffle}^{-1}x_1\Phi_{\shuffle}$. Then we have
\begin{align*}
x_1\Phi^{11} x_1 = -x_1\Phi^{01}x_1 + S_X^{\vee}(x_1\Phi^{01}x_1). 
\end{align*}
Equivalently, 
\[
\Phi^{11} + \Phi^{01} - S_X^{\vee}(\Phi^{01}) = 0
\]
holds.
\end{prop}

\begin{proof}
Since
\[
\langle \Phi \mid x_0^lx_1x_0^{k_r}x_1\cdots x_0^{k_1}x_1 \rangle
= \zeta_{\mathrm{Ad}}(k_1+1.\ldots,k_r+1;l)
\]
holds, \eqref{eq:parity-s} turns out to be
\begin{align*}
& \langle \Phi \mid x_1x_0^{k_r}x_1\cdots x_0^{k_1}x_1 \rangle\\
=&-  \langle \Phi \mid x_0^{k_r+1}x_1x_0^{k_{r-1}}\cdots x_0^{k_1}x_1 \rangle - (-1)^{k_1+\cdots +k_r + r} \langle \Phi \mid x_0^{k_1+1}x_1x_0^{k_{2}}\cdots x_0^{k_r}x_1 \rangle.
\end{align*}

Therefore, we have
\begin{align*}
x_1\Phi x_1 =& \sum_{k_1.\ldots,k_r\ge 0}  \langle \Phi \mid x_1x_0^{k_r}x_1\cdots x_0^{k_1}x_1 \rangle x_1x_0^{k_r}x_1\cdots x_0^{k_1}x_1 \\
&
\begin{aligned}
\hspace{-0.3cm}=&\sum_{k_1.\ldots,k_r\ge 0} (-  \langle \Phi \mid x_0^{k_r+1}x_1x_0^{k_{r-1}}\cdots x_0^{k_1}x_1 \rangle\\
&\qquad - (-1)^{k_1+\cdots +k_r + r} \langle \Phi \mid x_0^{k_1+1}x_1x_0^{k_{2}}\cdots x_0^{k_r}x_1 \rangle ) x_1x_0^{k_r}x_1\cdots x_0^{k_1}x_1
\end{aligned}\\
&
\begin{aligned}
\hspace{-0.3cm}=&-\sum_{k_1.\ldots,k_r\ge 0}   \langle x_0\Phi^{01} x_1 \mid x_0^{k_r+1}x_1x_0^{k_{r-1}}\cdots x_0^{k_1}x_1 \rangle _1x_0^{k_r}x_1\cdots x_0^{k_1}x_1 \\
&\qquad + \sum_{k_1.\ldots,k_r\ge 0} \langle \Phi \mid x_0^{k_1+1}x_1x_0^{k_{2}}\cdots x_0^{k_r}x_1 \rangle S_X^{\vee} (x_1x_0^{k_1}x_1\cdots x_0^{k_r}x_1)\\
\end{aligned}\\
=& - x_1\Phi^{01}x_1 + S_X^{\vee}(x_1\Phi^{01}x_1).
\end{align*}

\end{proof}

From this perspective, we say $\Psi \in \ide{h}^{\vee}$ with $\wt \Psi\ge 2$ satisfies  the \textbf{strong parity result} if $\Psi$ satisfies
\begin{align}
\Psi^{11} + \Psi^{01} - S_{X}^{\vee}(\Psi^{01}) = 0. \label{eq:str-parity-pri}
\end{align}

Assume that $\Psi \in \ide{F}_2$ satisfies Equality \eqref{eq:str-parity-pri}.
Then by Equality \eqref{eq:antipode-X}, we have $S_{X}^{\vee}(\Psi^{10}) = -\Psi^{01}$.
Therefore, the strong parity result turns out to be
\begin{align}
\Psi^{11} + \Psi^{10} + \Psi^{01} = 0. \label{eq:str-prty-3cycle}
\end{align}
 
To conclude, we define two $\Q$-linear subspaces of $\ide{h}^{\vee}$ as follows:
\begin{align*}
V_{\strprty} =& \{\Psi \in  \ide{h}^{\vee} \mid \Psi^{11} + \Psi^{10} + \Psi^{01} = 0\}.
\end{align*}

\begin{prop}\label{prop:deriv-strprty}
We have a Lie subalgebra $V_{\strprty} \cap \Fad \subset \tm_1$.
\end{prop}

This proposition follows from the following lemma.

\begin{lem}
For $\Psi_1$, $\Psi_2 \in  \Fad \cap V_{\strprty}$, we have
\begin{align*}
&d_{\Psi_1}(\Psi_1)^{11} + d_{\Psi_1}(\Psi_2)^{01} + d_{\Psi_1}(\Psi_2)^{10}\\
&\begin{aligned}
=&(\Psi_{1}^{11}x_1\Psi_2^{11} + \Psi_{2}^{11}x_1\Psi_1^{11}) - ( \Psi_1^{01}x_1\Psi_2^{11} + \Psi_2^{01}x_1\Psi_1^{11}) \\
&\quad-(\Psi_1^{11}x_1\Psi_2^{10} + \Psi_2^{11}x_1\Psi_1^{10}) + (\Psi_{1}^{10}x_0\Psi_2^{01}  +  \Psi_{2}^{10}x_0\Psi_1^{01} ).
\end{aligned}
\end{align*}
\end{lem}

\begin{proof}

Since 
\begin{align*}
x_1d_{\Psi_1}(x_1\Psi_2^{11}x_1)^{11}x_1=x_1\Psi_1^{11}x_1 \Psi_2^{11}x_1 + x_1\Psi_1^{10}x_0 \Psi_2^{11}x_1 + x_1\Psi_2^{11}x_1\Psi_1^{11}x_1 + x_1\Psi_2^{11}x_0\Psi_1^{01}x_1 + x_1d_{\Psi_1}(\Psi_2^{11})x_1
\end{align*}
holds, we have
\begin{align*}
d_{\Psi_1}(\Psi_2)^{11} = \Psi_1^{11}x_1\Psi_2^{11} + \Psi_2^{11}x_1\Psi_1^{11} + \Psi_1^{10}x_0\Psi_2^{11} + \Psi_2^{11}x_0\Psi_1^{01} + d_{\Psi_1}(\Psi_2^{11}).
\end{align*}

In a similar manner, we have
\begin{align*}
&d_{\Psi_1}(x_0\Psi_2^{01}x_1 + x_1\Psi_2^{11}x_1)\\
=&x_0\Psi_2^{01}x_1\Psi_1^{11}x_1 + x_0\Psi_{2}^{01}x_0\Psi_1^{01}x_1 + x_0\Psi_{1}^{01}x_1\Psi_2^{11}x_1 + x_0\Psi_{1}^{00}x_0\Psi_2^{11}x_1 + x_0d_{\Psi_1}(\Psi_2^{01})x_1
\end{align*}
and
\begin{align*}
&d_{\Psi_1}(x_1\Psi_2^{10}x_0 + x_1\Psi_2^{11}x_1)\\
=&x_1\Psi_1^{11}x_1\Psi_2^{10}x_0 + x_1\Psi_{1}^{10}x_0\Psi_2^{10}x_0 + x_1\Psi_{2}^{11}x_1\Psi_2^{10}x_0 + x_1\Psi_{2}^{11}x_0\Psi_1^{00}x_0 + x_1d_{\Psi_1}(\Psi_2^{10})x_0.
\end{align*}
Thus we have

\begin{align*}
d_{\Psi_1}(\Psi_2)^{01} =& \Psi_2^{01}x_1\Psi_1^{11} + \Psi_{2}^{01}x_0\Psi_1^{01} + \Psi_{1}^{01}x_1\Psi_2^{11} + \Psi_{1}^{00}x_0\Psi_2^{11} + d_{\Psi_1}(\Psi_2^{01})
\end{align*}

and

\begin{align*}
d_{\Psi_1}(\Psi_2)^{10} =&\Psi_1^{11}x_1\Psi_2^{10} + \Psi_{1}^{10}x_0\Psi_2^{10} + \Psi_{2}^{11}x_1\Psi_2^{10} + \Psi_{2}^{11}x_0\Psi_1^{00} + d_{\Psi_1}(\Psi_2^{10}).
\end{align*}

Therefore, we get

\begin{align*}
&d_{\Psi_1}(\Psi_1)^{11} + d_{\Psi_1}(\Psi_2)^{01} + d_{\Psi_1}(\Psi_2)^{10}\\
&\begin{aligned}
=&(\Psi_{1}^{11}x_1\Psi_2^{11} + \Psi_{2}^{11}x_1\Psi_1^{11}) + ( \Psi_1^{01}x_1\Psi_2^{11} + \Psi_2^{01}x_1\Psi_1^{11}) \\
&\quad+(\Psi_1^{11}x_1\Psi_2^{10} + \Psi_2^{11}x_1\Psi_1^{10}) + (\Psi_1^{00}x_0\Psi_2^{11} + \Psi_2^{11}x_0\Psi_1^{00})\\
&\qquad + (\Psi_{1}^{10}x_0\Psi_2^{11} +  \Psi_{2}^{11}x_0\Psi_1^{01} + \Psi_1^{10}x_0\Psi_2^{10} + \Psi_2^{01}x_0 \Psi_1^{01})\\
&\qquad +(d_{\Psi_1}(\Psi_2)^{11} + d_{\Psi_1}(\Psi_2^{01}) + d_{\Psi_1}(\Psi_2^{10})).
\end{aligned}
\end{align*}

Since $\Psi_2 \in V_{\strprty}$, we have

\begin{align*}
 &\Psi_{1}^{10}x_0\Psi_2^{11} +  \Psi_{2}^{11}x_0\Psi_1^{01} + \Psi_1^{10}x_0\Psi_2^{10} + \Psi_2^{01}x_0 \Psi_1^{01}\\
\stackrel{\eqref{eq:str-prty-3cycle}}{=}&-\Psi_{1}^{10}x_0\Psi_2^{10} - \Psi_{1}^{10}x_0\Psi_2^{01}  -  \Psi_{2}^{10}x_0\Psi_1^{01} -  \Psi_{2}^{01}x_0\Psi_1^{01} + \Psi_1^{10}x_0\Psi_2^{10} + \Psi_2^{01}x_0 \Psi_1^{01}\\
=&-\Psi_{1}^{10}x_0\Psi_2^{01}  -  \Psi_{2}^{10}x_0\Psi_1^{01} 
\end{align*}

and 
\begin{align*}
&d_{\Psi_1}(\Psi_2)^{11} + d_{\Psi_1}(\Psi_2^{01})  d_{\Psi_1}(\Psi_2^{10}) \\
=&d_{\Psi_1}(\Psi_2^{11} + \Psi_2^{10} + \Psi_2^{01})\\
\stackrel{\eqref{eq:str-prty-3cycle}}{=}& 0.
\end{align*}

Since $\Psi \in \Fad$ satisfies $\Psi^{00} = 0$, we have
\[
\Psi_1^{00}x_0\Psi_2^{11} + \Psi_2^{11}x_0\Psi_1^{00} = 0.
\]


Therefore, it follows that
\begin{align*}
&d_{\Psi_1}(\Psi_1)^{11} - d_{\Psi_1}(\Psi_2)^{01} - d_{\Psi_1}(\Psi_2)^{10}\\
&\begin{aligned}
=&(\Psi_{1}^{11}x_1\Psi_2^{11} + \Psi_{2}^{11}x_1\Psi_1^{11}) + ( \Psi_1^{01}x_1\Psi_2^{11} + \Psi_2^{01}x_1\Psi_1^{11}) \\
&\quad +(\Psi_1^{11}x_1\Psi_2^{10} + \Psi_2^{11}x_1\Psi_1^{10}) - (\Psi_{1}^{10}x_0\Psi_2^{01}  +  \Psi_{2}^{10}x_0\Psi_1^{01} ).
\end{aligned}
\end{align*}

\end{proof}






\nsubsection{Main Theorems}

In this paper, we provide the following:

\begin{mt}\label{mt:addmr-Lie}
The $\Q$-linear space $\addmr \cap \Fad \cap V_{\strprty}$ is a Lie subalgebra of $\adtm_1$.
Equivalently, the fiber product
\[
\addmr_{\ide{a}} \times_{\tm_{1,\ide{a}}}
\ide{F}_{2,\ide{a}}^{\ad(x_1)} \times_{\tm_{1,\ide{a}}}
V_{\strprty,\ide{a}}
\]
is a closed affine subscheme of $\tm_{1,\ide{a}}$ whose codomain is $\Lalg$.
\end{mt}

\begin{comment}
As a corollary, we get the following as our main theorem.

\begin{mt}\label{mt:AdDMR-Grp}
For $\Q$-algebra $R$, $R$-module $\AdDMR(R) \cap \FAd(R) \cap V_{\strprty}(R)$ is the subgroup of $\AdTM_1(R)$.
Equivalently, we define a functor $\AdDMR \cap \FAd \cap V_{\strprty} : \Qalg \rightarrow \Grp$ by $R \mapsto  \AdDMR(R) \cap \FAd(R) \cap V_{\strprty}(R)$. Then, $\AdDMR \cap \FAd \cap V_{\strprty} $ is the closed affine group subscheme of $\TM_1$.
\end{mt}
\end{comment}




\ifdefined\isMainDocument
\else
    %\bibliographystyle{amsplain}
    %\bibliography{reference-adjoint}
    \end{document}
\fi

\ifdefined\isMainDocument
\else
    \documentclass[10pt,oneside,reqno]{amsart}
    \usepackage{xr}
    \externaldocument{../main} % Main の .aux を参照
    \usepackage{latexsym}
\usepackage{amscd}
\usepackage{amsmath}
\usepackage{amssymb}
\usepackage[mathscr]{euscript}
\usepackage{graphicx}
\usepackage{geometry}
\usepackage{mathtools}
\usepackage[all]{xy}
\usepackage[dvipsnames]{xcolor}
\usepackage[colorlinks,final,hyperindex]{hyperref}
\usepackage{tikz}
\usepackage{tikz-cd}
\usetikzlibrary{decorations.pathmorphing,decorations.markings,arrows,calc,shapes.geometric,arrows.meta,positioning}
\usepackage{enumerate}
\usepackage{multirow}

\usepackage[bb=dsserif]{mathalpha}
\usepackage{bm}

\newcommand{\QQ}{\ensuremath{\mathbb{Q}}}
\newcommand{\DD}{\ensuremath{\mathbb{D}}}
\newcommand{\CC}{\ensuremath{\mathbb{C}}}
\newcommand{\RR}{\ensuremath{\mathbb{R}}}
\newcommand{\ZZ}{\ensuremath{\mathbb{Z}}}
\newcommand{\NN}{\ensuremath{\mathbb{N}}}

\newcommand{\g}{\ensuremath{\mathfrak{g}}}
\renewcommand{\SS}{\ensuremath{\mathbb{S}}}

\newcommand{\id}{\ensuremath{\mathrm{id}}}

\makeatletter

\newcommand{\bbone}{\mathbb{1}}

\newcommand{\calA}{\mathcal{A}}
\newcommand{\calB}{\mathcal{B}}
\newcommand{\calC}{\mathcal{C}}
\newcommand{\calD}{\mathcal{D}}
\newcommand{\calE}{\mathcal{E}}
\newcommand{\calF}{\mathcal{F}}
\newcommand{\calG}{\mathcal{G}}
\newcommand{\calH}{\mathcal{H}}
\newcommand{\calI}{\mathcal{I}}
\newcommand{\calJ}{\mathcal{J}}
\newcommand{\calK}{\mathcal{K}}
\newcommand{\calL}{\mathcal{L}}
\newcommand{\calM}{\mathcal{M}}
\newcommand{\calN}{\mathcal{N}}
\newcommand{\calO}{\mathcal{O}}
\newcommand{\calP}{\mathcal{P}}
\newcommand{\calQ}{\mathcal{Q}}
\newcommand{\calR}{\mathcal{R}}
\newcommand{\calS}{\mathcal{S}}
\newcommand{\calT}{\mathcal{T}}
\newcommand{\calU}{\mathcal{U}}
\newcommand{\calV}{\mathcal{V}}
\newcommand{\calW}{\mathcal{W}}
\newcommand{\calX}{\mathcal{X}}
\newcommand{\calY}{\mathcal{Y}}
\newcommand{\calZ}{\mathcal{Z}}
\newcommand{\kk}{\Bbbk}

\newcommand{\scrY}{\mathscr{Y}}
\newcommand{\scrV}{\mathscr{V}}
\newcommand{\scrP}{\mathscr{P}}

\newcommand{\Rep}{\mathop{\mathrm{Rep}}\nolimits}
\newcommand{\alg}{\mathop{\mathrm{alg}}\nolimits}
\newcommand{\Span}{\mathop{\mathrm{Span}}\nolimits}
\newcommand{\proj}{\mathop{\mathrm{proj}}\nolimits}
\newcommand{\Hom}{\mathop{\mathrm{Hom}}\nolimits}
\newcommand{\PHom}{\mathop{\mathrm{PHom}}\nolimits}
\newcommand{\Tw}{\mathop{\mathrm{Tw}}\nolimits}
\newcommand{\Dim}{\mathop{\mathrm{Dim}}\nolimits}
\newcommand{\Imm}{\mathop{\mathrm{Im}}\nolimits}
\newcommand{\Aus}{\mathop{\mathrm{Aus}}\nolimits}
\newcommand{\Ann}{\mathop{\mathrm{Ann}}\nolimits}
\newcommand{\APC}{\mathop{\mathrm{APC}}\nolimits}
\newcommand{\Barr}{\mathop{\mathrm{Bar}}\nolimits}
\newcommand{\Cone}{\mathop{\mathrm{Cone}}\nolimits}
\newcommand{\modu}{\mathop{\mathrm{mod}}\nolimits}
\newcommand{\dgMod}{\mathop{\mathrm{dgMod}}\nolimits}
\newcommand{\soc}{\mathop{\mathrm{soc}}\nolimits}
\newcommand{\dg}{\mathop{\mathrm{dg}}\nolimits}
\newcommand{\red}{\mathop{\mathrm{red}}\nolimits}
\newcommand{\Map}{\mathop{\mathrm{Map}}\nolimits}
\newcommand{\inj}{\mathop{\mathrm{inj}}\nolimits}
\newcommand{\op}{\mathop{\mathrm{op}}\nolimits}
\newcommand{\Ker}{\mathop{\mathrm{Ker}}\nolimits}
\newcommand{\Tor}{\mathop{\mathrm{Tor}}\nolimits}
\newcommand{\Tot}{\mathop{\mathrm{Tot}}\nolimits}
\newcommand{\Ext}{\mathop{\mathrm{Ext}}\nolimits}
\newcommand{\sg}{\mathop{\mathrm{sg}}\nolimits}
\newcommand{\Sl}{\mathop{\mathfrak{sl}}\nolimits}
\newcommand{\nc}{\mathop{\mathrm{nc}}\nolimits}
\newcommand{\Tr}{\mathop{\mathrm{Tr}}\nolimits}
\newcommand{\Irr}{\mathop{\mathrm{Irr}}\nolimits}
\newcommand{\HC}{\mathop{\mathrm{HC}}\nolimits}
\newcommand{\HH}{\mathop{\mathrm{HH}}\nolimits}
\newcommand{\THH}{\mathop{\mathrm{TH}}\nolimits}
\newcommand{\Perf}{\mathop{\mathrm{Perf}}\nolimits}
\newcommand{\del}{\partial}
\newcommand{\ev}{\mathop{\mathrm{ev}}\nolimits}
\newcommand{\co}{\mathop{\mathrm{co}}\nolimits}
\newcommand{\pt}{\mathop{\mathrm{pt}}\nolimits}
\newcommand{\wt}{\mathop{\mathrm{wt}}\nolimits}
\newcommand{\pCY}{\mathop{\mathrm{pCY}}\nolimits}
\newcommand{\gr}{\mathop{\mathrm{gr}}\nolimits}
\newcommand{\coker}{\mathop{\mathrm{coker}}\nolimits}


\newcommand{\Top}{\mathop{\mathrm{Top}}\nolimits}
\newcommand{\Set}{\mathop{\mathrm{Set}}\nolimits}
\newcommand{\Vect}{\mathop{\mathrm{Vect}}\nolimits}
\newcommand{\Mod}{\mathop{\mathrm{Mod}}\nolimits}
\newcommand{\Ob}{\mathop{\mathrm{Ob}}\nolimits}
\newcommand{\Ind}{\mathop{\mathrm{Ind}}\nolimits}
\newcommand{\Sym}{\mathop{\mathrm{Sym}}\nolimits}
\newcommand{\Alt}{\mathop{\mathrm{Alt}}\nolimits}
\newcommand{\End}{\mathop{\mathrm{End}}\nolimits}
\newcommand{\Ch}{\mathop{\mathrm{Ch}}\nolimits}


\newcommand{\Ass}{\mathop{\mathrm{Ass}}\nolimits}
\newcommand{\Com}{\mathop{\mathrm{Com}}\nolimits}
\newcommand{\Lie}{\mathop{\mathrm{Lie}}\nolimits}
\newcommand{\colim}{\mathop{\mathrm{colim}}\nolimits}

\newcommand{\Op}{\mathop{\mathsf{Operads}}\nolimits}
\newcommand{\Diop}{\mathop{\mathsf{Dioperads}}\nolimits}
\newcommand{\PDiop}{\mathop{\mathsf{PDioperads}}\nolimits}
\newcommand{\Coop}{\mathop{\mathsf{Cooperads}}\nolimits}
\newcommand{\Codiop}{\mathop{\mathsf{Codioperads}}\nolimits}
\newcommand{\PCodiop}{\mathop{\mathsf{PCodioperads}}\nolimits}

\newcommand{\antish}{\ensuremath{\mbox{!`}}}


\tikzset{->-/.style={decoration={
			markings,
			mark=at position #1 with {\arrow{>}}},postaction={decorate}}}
\tikzset{-w-/.style={decoration={
			markings,
			mark=at position #1 with {\arrow{Stealth[fill=white,scale=1.4]}}},postaction={decorate}}}
\tikzset{->-/.default=0.65}
\tikzset{-w-/.default=0.65}
\tikzstyle{bullet}=[circle,fill=black,inner sep=0.5mm]
\tikzstyle{circ}=[circle,draw=black,fill=white,inner sep=0.5mm]
\tikzstyle{vertex}=[circle,draw=black,thick,inner sep=0.5mm]
\tikzstyle{dot}=[draw,circle,fill=black,minimum size=0.5mm,inner sep = 0mm, outer sep = 0mm]
\tikzset{darrow/.style={double distance = 4pt,>={Implies},->},
	darrowthin/.style={double equal sign distance,>={Implies},->},
	tarrow/.style={-,preaction={draw,darrow}},
	qarrow/.style={preaction={draw,darrow,shorten >=0pt},shorten >=1pt,-,double,double
		distance=0.2pt}}
\newcommand\tikcirc[1][2.5]{\tikz[baseline=-#1]{\draw[thick](0,0)circle[radius=#1mm];}}
\newcommand{\tikzfig}[1]{\begin{tikzpicture}[auto,baseline={([yshift=-.5ex]current bounding box.center)}]#1\end{tikzpicture}}
\usetikzlibrary{decorations.pathreplacing}

\usepackage{amsthm}
\usepackage{thmtools}
\usepackage[noabbrev,capitalize]{cleveref}
\newtheorem{theorem}{Theorem}
\newtheorem{lemma}[theorem]{Lemma}
\newtheorem{proposition}[theorem]{Proposition}
\newtheorem{corollary}[theorem]{Corollary}
\newtheorem{conjecture}[theorem]{Conjecture}
\newtheorem{Claim}[theorem]{Claim}

\newtheorem*{theorem*}{Theorem}

\theoremstyle{definition}
\newtheorem{definition}{Definition}
\newtheorem{example}{Example}
\newtheorem{cexample}{Counterexample}
\newtheorem{exercise}{Exercise}
\newtheorem{question}{Question}
\newtheorem{remark}{Remark}

\addtolength{\textheight}{1.75cm}
\addtolength{\voffset}{-0.5cm}
\addtolength{\footskip}{+0.6cm}
\geometry{margin=1.2in}


\usepackage[backend=biber, maxbibnames=99, bibencoding=utf8, doi=false, isbn=false, url=false, style=alphabetic]{biblatex}
\addbibresource{refs.bib}

    \begin{document}
\fi

%\setcounter{section}{4}
\section{Proof of the main theorem}

In this section, we prove the Main Theorem.

\nsubsection{A lemma}
The following lemma plays an essential role in the proof of our main theorem and its proof is given in Appendix \ref{section:proof of lemma}.

\begin{lem}\label{lem:essential}
For $\Psi_1$, $\Psi_2 \in \addmr \cap \Fad \cap V_{\strprty}$, we have
\begin{align}
\begin{aligned}
\Delta_*(d_{\Psi_1}(\Psi_2)_{\#}) =&
d_{\Psi_1}(\Psi_2)_{\#} \otimes 1 + 1\otimes d_{\Psi_1}(\Psi_2)_{\#} + \Psi_{1,\#} \otimes  \Psi_{2,\#}  + \Psi_{2,\#} \otimes  \Psi_{1,\#}.  
\end{aligned}  \label{eq:essential}
\end{align}
\end{lem}

\begin{comment}
\begin{align}
\begin{aligned}
\Delta_*(d_{\Psi_1}^n(\Psi_2)) =&
d_{\Psi_1}^n(\Psi_2) \otimes 1 + 1\otimes d_{\Psi_1}^n(\Psi_2)\\
&\quad+ \sum_{n_1 + n_2 = n-1} \binom{n-1}{n_1} d_{\Psi_1}^{n_1}(\Psi_1) \otimes d_{\Psi_1}^{n_2}(\Psi_2) + \binom{n-1}{n_1} d_{\Psi_1}^{n_1}(\Psi_2) \otimes d_{\Psi_1}^{n_2}(\Psi_1)
\end{aligned}  \label{eq:essential}
\end{align}


\begin{cor}\label{cor:essential}
For a positive integer $n$, and $\Psi_1$ and $\Psi_2 \in \addmr(R)$, 
\begin{align*}
\Delta_*(d_{\Psi_1}^n(x_1)) =& \sum_{n_1 + n_2 = n} \binom{n}{n_1} d_{\Psi_1}^{n_1}(x_1) \otimes d_{\Psi_1}^{n_2}(x_1). 
\end{align*}
\end{cor}

\begin{proof}

\end{proof}

\end{comment}


\nsubsection{Proof of the Main Theorem}

\begin{proof}[Proof of Proposition \ref{mt:addmr-Lie}]

Since, by Corollary \ref{cor:tm-freeLie} and Proposition \ref{prop:deriv-strprty}, $ (\ide{F}_2^{\ge 4}\cap \Fad \cap  V_{\strprty}) \subset \tm_1$ as a Lie algebra, it suffices to show that $\Delta_*(\{\Phi,\Psi\}_{1,\#}) = \{\Phi,\Psi\}_{1,\#} \otimes 1 + 1\otimes \{\Phi,\Psi\}_{1,\#}$.
By \eqref{eq:essential}, we have
\[
\Delta_*(d_{\Psi_1}(\Psi_2)_{\#})  =d_{\Psi_1}(\Psi_2) \otimes 1 + 1\otimes d_{\Psi_1}(\Psi_2) +  \Psi_{1,\#} \otimes \Psi_{2,\#} + \Psi_{2,\#} \otimes \Psi_{1,\#}.
\]
Therefore, we have
\begin{align*}
\Delta_*(\{\Psi_1,\Psi_2\}_{1,\#}) &= \Delta_*(d_{\Psi_1}(\Psi_2)_{\#} - d_{\Psi_2}(\Psi_1)_{\#} ) \\
&\begin{aligned}
&=d_{\Psi_1}(\Psi_2) \otimes 1 + 1\otimes d_{\Psi_1}(\Psi_2) +  \Psi_{1,\#} \otimes \Psi_{2,\#} + \Psi_{2,\#} \otimes \Psi_{1,\#}\\
&\qquad-d_{\Psi_1}(\Psi_2) \otimes 1 - 1\otimes d_{\Psi_1}(\Psi_2)  -  \Psi_{2,\#} \otimes \Psi_{1,\#} + \Psi_{1,\#} \otimes \Psi_{2,\#}\\
\end{aligned}\\
&= d_{\Psi_1}(\Psi_2) \otimes 1 + 1\otimes d_{\Psi_1}(\Psi_2) - d_{\Psi_1}(\Psi_2) \otimes 1 - 1\otimes d_{\Psi_1}(\Psi_2)\\
&=\{\Psi_1,\Psi_2\}_1 \otimes1 + 1\otimes \{\Psi_1,\Psi_2\}_1,
\end{align*}
as claimed.
\end{proof}

\begin{comment}

\nsubsection{A proof of Main Theorem \ref{mt:AdDMR-Grp}}


To prove Main Theorem \ref{mt:AdDMR-Grp}, we consider the affine group scheme $\exp^{\circledast_1}(\addmr(R) \cap \Fad(R) \cap V_{\strprty}(R)))$. Since $\addmr \cap \Fad \cap V_{\strprty}(R) \subset \tm_1(R)$,  $\exp^{\circledast_1}(\addmr(R) \cap \Fad(R) \cap V_{\strprty}(R))$ is a subgroup of $\TM_1(R)$ by Proposition \ref{prop:unipotent}. (\ref{enumi:unipotent-2}).

\begin{lem}
The group $\exp^{\circledast_1}(\addmr(R) \cap \Fad(R) \cap V_{\strprty}(R)$ acts on $\AdDMR_0(R)$.
\end{lem}
\begin{proof}

\end{proof}

Let $\ide{h}^{(n)} := \Q + \Q\cdot X + \cdots + \Q\cdot X^{n-1}$ and we define 
\[
\AdDMR_0^{(n)}(R) := \left\{\Phi \in \ide{h}^{(n)} \middle| 
\begin{matrix}
\wt \Phi - x_1 \ge 4\\
\Delta_{\shuffle}(\Phi) = \Phi \otimes 1 + 1\otimes \Phi\\
\Delta_*(\Phi_{\#}) = \Phi_{\#} \otimes\Phi_{\#}.
\end{matrix}
\right\}.
\]
For example, $\AdDMR_0^{(n)} = 0$ for $n = 0$, $\AdDMR_0^{(n)}(R) = \{x_1\}$ for $n = 2$, $3$.
For $n_1$, $n_2 \in \mathbb{Z}_{\ge 0}$ with $n_1 > n_2$, let us define $ \pi_{n_1,n_2} : \AdDMR_0^{(n_1)}(R) \rightarrow \AdDMR_0^{(n_2)}$ as modded $\Q\cdot X^{n_2}$.
Then, a pair of classes of sets and maps $\left(\{\AdDMR_0^{(n)}\}_{n \in \mathbb{Z}_{\ge 0}},\{\pi_{n_1,n_2}\}_{\substack{n_1> n_2  \\n_1,n_2\in\mathbb{Z}_{\ge 0}}}\right)$ is the projective system and it is easily seen that a projective limit $\displaystyle \lim_{\longleftarrow } \AdDMR_0^{(n)}(R)$ is equal to $\AdDMR_0$.
Let $\pi_n : \AdDMR_0(R) \rightarrow \AdDMR_0^{(n)}(R)$ be a natural projection.

\begin{lem}\label{lem:difference}
For each $n \in \mathbb{Z}_{\ge 0}$, and $\Phi_1$, $\Phi_2 \in \AdDMR_0^{(n+1)}(R)$ with $\pi_{n+1,n}(\Phi_1) = \pi_{n+1,n}(\Phi_2)$,
we have $\Phi_1 - \Phi_2 \in \addmr(R)$ with weight homogeneous degree $n$.
\end{lem}

\begin{proof}
Let us consider the following decomposition $\Phi_i := x_1 + \Phi_i^{(2)} + \cdots + \Phi_i^{(n)}$ for $i = 0$. $1$.
Then, for each $2\le j \le n$, $x_1 + \Phi_i^{(2)} + \cdots \Phi_i^{(j)} \in \AdDMR_0^{(j+1)}(R)$. 
Therefore, 
\[
\langle \Phi_1 - \Phi_2 \mid x_1 \rangle = 0,
\]
\begin{align*}
&\Delta_{\shuffle}(\Phi_1 - \Phi_2)\\
=&\Phi_1 \otimes 1 + 1 \otimes \Phi_1 - \Phi_2 \otimes 1 + 1 \otimes \Phi_2\\
=&\left(\sum_{1 \le j \le n-1} \Phi_1^{j}\right) \otimes 1 + 1\otimes \left(\sum_{1 \le j \le n-1} \Phi_1^{j}\right)  +\Phi_1^{n} \otimes 1 + 1\otimes  \Phi_1^{n} \\
&-\left(\sum_{1 \le j \le n-1} \Phi_1^{j} \right) \otimes 1 - 1\otimes \left( \sum_{1 \le j \le n-1} \Phi_1^{j}\right) - \Phi_2^{n} \otimes 1 - 1\otimes  \Phi_2^{n}\\
=&(\Phi_1^n - \Phi_2^n) \otimes 1 - 1 \otimes (\Phi_1 - \Phi_2),
\end{align*}

and

\begin{align*}
&\Delta_{*}(\Phi_{1,\#} - \Phi_{2,\#})\\
=&\sum_{k + l = n}\left( \Phi_{1,\#}^{(k)} \otimes \Phi_{1,\#}^{(l)} \right)+ \Phi_{1,\#}^{(n)} \otimes 1 + 1\otimes  \Phi_{1,\#}^{(n)}\\
&-\sum_{k + l = n}\left( \Phi_{2,\#}^{(k)} \otimes \Phi_{2,\#}^{(l)} \right)+ \Phi_{2,\#}^{(n)} \otimes 1 + 1\otimes  \Phi_{2,\#}^{(n)}\\
=&(\Phi^{(n)}_{1,\#} -  \Phi^{(n)}_{2,\#})\otimes 1 + 1\otimes  (\Phi^{(n)}_{1,\#} - \Phi^{(n)}_{2,\#}).
\end{align*}
holds.
\end{proof}

\begin{lem}\label{lem:transitivity}
For each $n \in \mathbb{Z}_{\ge 0}$, the action of $\exp(\addmr(R) \cap \Fad(R) \cap V_{\strprty}(R))$ to $\AdDMR_0^{(n)}(R)$ and $\AdDMR_0(R)$ is transitive.
\end{lem}

\begin{proof}
We prove by  induction on $n \in \mathbb{Z}_{\ge 0}$.
The case where $n = 1$, $2$, and $3$ is trivial because $\AdDMR_0^{(n)}(R) = \{x_1\}$.
Assume that the action of $\exp(\addmr(R) \cap \Fad(R) \cap V_{\strprty}(R))$ to $\AdDMR_0^{(n)}(R)$ is transitive.
Then, for $\Phi_1$, $\Phi_2 \in \AdDMR_0^{n+1}(R)$, there exists $\psi \in \addmr(R)$ such that
\[
\pi_{n+1,n}(\exp(d_{\psi_n})(\Phi_1)) = \pi_{n+1,n}(\Phi_2).
\]
Thus by Lemma \ref{lem:difference}, 
\[
\psi':=\pi_{n+1}(\exp(d_{\psi_n})(\Phi_1)) - \Phi_2 \in \addmr(R).
\]
and $\wt \psi = n$. 

Let us consider $\pi_{n}\circ \exp(d_{\psi'})$. Since, for any $x \in \ide{h}^{(n)}$ with $\langle x \mid x_1 \rangle = 1$, we have
\[
\langle \exp(d_{\psi'})(x) = x + \psi' \mod \Q \cdot X^{n+1},
\]
Therefore, 
\begin{align*}
&\pi_{n+1}\circ \exp(d_{\psi'}) \circ \exp(d_{\psi_n})(\Phi_1)\\
=&\pi_{n+1}(\psi' + \exp(d_{\psi_n})(\Phi_1))\\
=&\Phi_2.
\end{align*}
Put $\psi_{n+1} := \BCH(\psi_n,\psi')$.
Here $\BCH$ is the BaKer-Cambell-Hausdorff series.
Since $\psi_n$, $\psi' \in \addmr(R)\cap \Fad(R) \cap V_{\strprty}(R)$, $\BCH(\psi_n.\psi')  \in \addmr(R)\cap \Fad(R) \cap V_{\strprty}(R)$.
Addtionally, since $\addmr(R) \cap \Fad(R) \cap V_{\strprty}(R)\ni \psi \mapsto d_{\psi} \in \ide{gl}(\ide{h}^{\vee})$ is a Lie algebra homomorphism, $d_{\BCH(\psi_n,\psi')} = \BCH(d_{\psi_n}, d_{\psi'})$ holds.
Therefore, 
\begin{align*}
\pi_{n+1}(\exp(d_{\psi_{n+1}})(\Phi_1)) =& \pi_{n+1}(\exp(d_{\BCH(\psi_n,\psi')})(\Phi_1))\\
=&\pi_{n+1}(\exp(\BCH(d_{\psi_n},d_{\psi'}))(\Phi_1))\\
=&\pi_{n+1}(\exp(d_{\psi_n}) \circ \exp(d_{\psi'})(\Phi_1))\\
=&\Phi_2.
\end{align*}
In the second equality above, we use the property of Backer-Cambell-Hausdorff series. 
The transitivity of the action of $\exp(\addmr(R) \cap \Fad(R) \cap V_{\strprty}(R))$ to $\AdDMR_0(R)$ is follows from taking projective limit of the action $\exp(\addmr(R) \cap \Fad(R) \cap V_{\strprty}(R))$ to $\AdDMR_0^{(n)}(R)$.
\end{proof}

\begin{proof}[A proof of Main Theorem \ref{mt:AdDMR-Grp}]\verb| |

By lemma \ref{lem:transitivity}, 

\[
\exp^{\circledast_1}(\addmr(R) \cap \Fad(R) \cap V_{\strprty}(R) ) \circledast_1 \Phi = \AdDMR_0(R).
\]
for any $\Phi \in \AdDMR_0(R)$. Especially, if $\Phi = x_1$, we have
\[
\exp^{\circledast_1}(\addmr(R) \cap \Fad(R) \cap V_{\strprty}(R) ) = \exp^{\circledast_1}(\addmr(R) \cap \Fad(R) \cap V_{\strprty}(R) ) \circledast_1 x_1 = \AdDMR_0(R).
\]

\end{proof}
\end{comment}


\begin{comment}
\begin{proof}[A proof of Main Theorem \ref{mt:AdDMR-Grp}]\verb| |

Since $\addmr(R) \cap \Fad(R) \cap V_{\strprty}(R)$ is a Lie subalgebra of $\tm_1(R)$, $\exp^{\circledast_1}(\addmr(R) \cap \Fad(R) \cap V_{\strprty}(R)$ is a subgroup of $\TM_1(R)$.
We now show that $\AdDMR(R)\cap \FAd(R) \cap \exp(V_{\strprty}(R)) = \exp^{\circledast_1}(\addmr(R) \cap \Fad(R) \cap V_{\strprty}(R))$.

First we show $\AdDMR(R)\cap \FAd(R) \cap \exp(V_{\strprty}(R)) \supset \exp^{\circledast_1}(\addmr(R) \cap \Fad(R) \cap V_{\strprty}(R))$. 
Let $\psi \in  \exp^{\circledast_1}(\addmr(R) \cap \Fad(R) \cap V_{\strprty}(R))$. 
It suffices to show $\Delta_*(\exp^{\circledast_1}(\psi)) = \exp^{\circledast_1}(\psi) \otimes \exp^{\circledast_1}(\psi)$.
By Corollary \ref{cor:essential} to the third equality below, we have
\begin{align*}
\Delta_*(\exp^{\circledast_1}(\psi)) =& \Delta_*\left(\left(\sum_{n\ge 0} \frac{d_{\psi}^n}{n!}(x_1)\right)_{\#}\right) \displaybreak[3]\\
=&\sum_{n\ge 0} \frac{(\Delta_* \circ d_{\psi}^n(x_1))_{\#}}{n!} \displaybreak[3]\\
=&\sum_{n\ge 0} \sum_{n_1 + n_2 = n} \binom{n}{n_1} \frac{1}{n!} (d_{\psi}^{n_1}(x_1) \otimes d_{\psi}^{n_2}(x_1)) \displaybreak[3]\\
=&\sum_{n\ge 0} \sum_{n_1 + n_2 = n} \left(\frac{1}{n_1!} d_{\psi}^{n_1}(x_1)\right) \otimes \left( \frac{1}{n_2!} d_{\psi}^{n_2}(x_1)\right) \displaybreak[3]\\
=&\left(\sum_{n_1 \ge 0}\frac{d_{\psi}^{n_1}(x_1)}{n_1!}\right) \otimes \left(\sum_{n_2 \ge 0}\frac{d_{\psi}^{n_2}(x_1)}{n_2!}\right) \\
=&\exp^{\circledast_1}(\psi) \otimes \exp^{\circledast_1}(\psi)
\end{align*}
as claimed.

Next we show the opposite inclusion.
\end{proof}
\end{comment}


\ifdefined\isMainDocument
\else
    %\bibliographystyle{amsplain}
    %\bibliography{reference-adjoint}
    \end{document}
\fi

\ifdefined\isMainDocument
\else
    \documentclass[10pt,oneside,reqno]{amsart}
    \usepackage{xr}
    \externaldocument{../main} % Main の .aux を参照
    \usepackage{latexsym}
\usepackage{amscd}
\usepackage{amsmath}
\usepackage{amssymb}
\usepackage[mathscr]{euscript}
\usepackage{graphicx}
\usepackage{geometry}
\usepackage{mathtools}
\usepackage[all]{xy}
\usepackage[dvipsnames]{xcolor}
\usepackage[colorlinks,final,hyperindex]{hyperref}
\usepackage{tikz}
\usepackage{tikz-cd}
\usetikzlibrary{decorations.pathmorphing,decorations.markings,arrows,calc,shapes.geometric,arrows.meta,positioning}
\usepackage{enumerate}
\usepackage{multirow}

\usepackage[bb=dsserif]{mathalpha}
\usepackage{bm}

\newcommand{\QQ}{\ensuremath{\mathbb{Q}}}
\newcommand{\DD}{\ensuremath{\mathbb{D}}}
\newcommand{\CC}{\ensuremath{\mathbb{C}}}
\newcommand{\RR}{\ensuremath{\mathbb{R}}}
\newcommand{\ZZ}{\ensuremath{\mathbb{Z}}}
\newcommand{\NN}{\ensuremath{\mathbb{N}}}

\newcommand{\g}{\ensuremath{\mathfrak{g}}}
\renewcommand{\SS}{\ensuremath{\mathbb{S}}}

\newcommand{\id}{\ensuremath{\mathrm{id}}}

\makeatletter

\newcommand{\bbone}{\mathbb{1}}

\newcommand{\calA}{\mathcal{A}}
\newcommand{\calB}{\mathcal{B}}
\newcommand{\calC}{\mathcal{C}}
\newcommand{\calD}{\mathcal{D}}
\newcommand{\calE}{\mathcal{E}}
\newcommand{\calF}{\mathcal{F}}
\newcommand{\calG}{\mathcal{G}}
\newcommand{\calH}{\mathcal{H}}
\newcommand{\calI}{\mathcal{I}}
\newcommand{\calJ}{\mathcal{J}}
\newcommand{\calK}{\mathcal{K}}
\newcommand{\calL}{\mathcal{L}}
\newcommand{\calM}{\mathcal{M}}
\newcommand{\calN}{\mathcal{N}}
\newcommand{\calO}{\mathcal{O}}
\newcommand{\calP}{\mathcal{P}}
\newcommand{\calQ}{\mathcal{Q}}
\newcommand{\calR}{\mathcal{R}}
\newcommand{\calS}{\mathcal{S}}
\newcommand{\calT}{\mathcal{T}}
\newcommand{\calU}{\mathcal{U}}
\newcommand{\calV}{\mathcal{V}}
\newcommand{\calW}{\mathcal{W}}
\newcommand{\calX}{\mathcal{X}}
\newcommand{\calY}{\mathcal{Y}}
\newcommand{\calZ}{\mathcal{Z}}
\newcommand{\kk}{\Bbbk}

\newcommand{\scrY}{\mathscr{Y}}
\newcommand{\scrV}{\mathscr{V}}
\newcommand{\scrP}{\mathscr{P}}

\newcommand{\Rep}{\mathop{\mathrm{Rep}}\nolimits}
\newcommand{\alg}{\mathop{\mathrm{alg}}\nolimits}
\newcommand{\Span}{\mathop{\mathrm{Span}}\nolimits}
\newcommand{\proj}{\mathop{\mathrm{proj}}\nolimits}
\newcommand{\Hom}{\mathop{\mathrm{Hom}}\nolimits}
\newcommand{\PHom}{\mathop{\mathrm{PHom}}\nolimits}
\newcommand{\Tw}{\mathop{\mathrm{Tw}}\nolimits}
\newcommand{\Dim}{\mathop{\mathrm{Dim}}\nolimits}
\newcommand{\Imm}{\mathop{\mathrm{Im}}\nolimits}
\newcommand{\Aus}{\mathop{\mathrm{Aus}}\nolimits}
\newcommand{\Ann}{\mathop{\mathrm{Ann}}\nolimits}
\newcommand{\APC}{\mathop{\mathrm{APC}}\nolimits}
\newcommand{\Barr}{\mathop{\mathrm{Bar}}\nolimits}
\newcommand{\Cone}{\mathop{\mathrm{Cone}}\nolimits}
\newcommand{\modu}{\mathop{\mathrm{mod}}\nolimits}
\newcommand{\dgMod}{\mathop{\mathrm{dgMod}}\nolimits}
\newcommand{\soc}{\mathop{\mathrm{soc}}\nolimits}
\newcommand{\dg}{\mathop{\mathrm{dg}}\nolimits}
\newcommand{\red}{\mathop{\mathrm{red}}\nolimits}
\newcommand{\Map}{\mathop{\mathrm{Map}}\nolimits}
\newcommand{\inj}{\mathop{\mathrm{inj}}\nolimits}
\newcommand{\op}{\mathop{\mathrm{op}}\nolimits}
\newcommand{\Ker}{\mathop{\mathrm{Ker}}\nolimits}
\newcommand{\Tor}{\mathop{\mathrm{Tor}}\nolimits}
\newcommand{\Tot}{\mathop{\mathrm{Tot}}\nolimits}
\newcommand{\Ext}{\mathop{\mathrm{Ext}}\nolimits}
\newcommand{\sg}{\mathop{\mathrm{sg}}\nolimits}
\newcommand{\Sl}{\mathop{\mathfrak{sl}}\nolimits}
\newcommand{\nc}{\mathop{\mathrm{nc}}\nolimits}
\newcommand{\Tr}{\mathop{\mathrm{Tr}}\nolimits}
\newcommand{\Irr}{\mathop{\mathrm{Irr}}\nolimits}
\newcommand{\HC}{\mathop{\mathrm{HC}}\nolimits}
\newcommand{\HH}{\mathop{\mathrm{HH}}\nolimits}
\newcommand{\THH}{\mathop{\mathrm{TH}}\nolimits}
\newcommand{\Perf}{\mathop{\mathrm{Perf}}\nolimits}
\newcommand{\del}{\partial}
\newcommand{\ev}{\mathop{\mathrm{ev}}\nolimits}
\newcommand{\co}{\mathop{\mathrm{co}}\nolimits}
\newcommand{\pt}{\mathop{\mathrm{pt}}\nolimits}
\newcommand{\wt}{\mathop{\mathrm{wt}}\nolimits}
\newcommand{\pCY}{\mathop{\mathrm{pCY}}\nolimits}
\newcommand{\gr}{\mathop{\mathrm{gr}}\nolimits}
\newcommand{\coker}{\mathop{\mathrm{coker}}\nolimits}


\newcommand{\Top}{\mathop{\mathrm{Top}}\nolimits}
\newcommand{\Set}{\mathop{\mathrm{Set}}\nolimits}
\newcommand{\Vect}{\mathop{\mathrm{Vect}}\nolimits}
\newcommand{\Mod}{\mathop{\mathrm{Mod}}\nolimits}
\newcommand{\Ob}{\mathop{\mathrm{Ob}}\nolimits}
\newcommand{\Ind}{\mathop{\mathrm{Ind}}\nolimits}
\newcommand{\Sym}{\mathop{\mathrm{Sym}}\nolimits}
\newcommand{\Alt}{\mathop{\mathrm{Alt}}\nolimits}
\newcommand{\End}{\mathop{\mathrm{End}}\nolimits}
\newcommand{\Ch}{\mathop{\mathrm{Ch}}\nolimits}


\newcommand{\Ass}{\mathop{\mathrm{Ass}}\nolimits}
\newcommand{\Com}{\mathop{\mathrm{Com}}\nolimits}
\newcommand{\Lie}{\mathop{\mathrm{Lie}}\nolimits}
\newcommand{\colim}{\mathop{\mathrm{colim}}\nolimits}

\newcommand{\Op}{\mathop{\mathsf{Operads}}\nolimits}
\newcommand{\Diop}{\mathop{\mathsf{Dioperads}}\nolimits}
\newcommand{\PDiop}{\mathop{\mathsf{PDioperads}}\nolimits}
\newcommand{\Coop}{\mathop{\mathsf{Cooperads}}\nolimits}
\newcommand{\Codiop}{\mathop{\mathsf{Codioperads}}\nolimits}
\newcommand{\PCodiop}{\mathop{\mathsf{PCodioperads}}\nolimits}

\newcommand{\antish}{\ensuremath{\mbox{!`}}}


\tikzset{->-/.style={decoration={
			markings,
			mark=at position #1 with {\arrow{>}}},postaction={decorate}}}
\tikzset{-w-/.style={decoration={
			markings,
			mark=at position #1 with {\arrow{Stealth[fill=white,scale=1.4]}}},postaction={decorate}}}
\tikzset{->-/.default=0.65}
\tikzset{-w-/.default=0.65}
\tikzstyle{bullet}=[circle,fill=black,inner sep=0.5mm]
\tikzstyle{circ}=[circle,draw=black,fill=white,inner sep=0.5mm]
\tikzstyle{vertex}=[circle,draw=black,thick,inner sep=0.5mm]
\tikzstyle{dot}=[draw,circle,fill=black,minimum size=0.5mm,inner sep = 0mm, outer sep = 0mm]
\tikzset{darrow/.style={double distance = 4pt,>={Implies},->},
	darrowthin/.style={double equal sign distance,>={Implies},->},
	tarrow/.style={-,preaction={draw,darrow}},
	qarrow/.style={preaction={draw,darrow,shorten >=0pt},shorten >=1pt,-,double,double
		distance=0.2pt}}
\newcommand\tikcirc[1][2.5]{\tikz[baseline=-#1]{\draw[thick](0,0)circle[radius=#1mm];}}
\newcommand{\tikzfig}[1]{\begin{tikzpicture}[auto,baseline={([yshift=-.5ex]current bounding box.center)}]#1\end{tikzpicture}}
\usetikzlibrary{decorations.pathreplacing}

\usepackage{amsthm}
\usepackage{thmtools}
\usepackage[noabbrev,capitalize]{cleveref}
\newtheorem{theorem}{Theorem}
\newtheorem{lemma}[theorem]{Lemma}
\newtheorem{proposition}[theorem]{Proposition}
\newtheorem{corollary}[theorem]{Corollary}
\newtheorem{conjecture}[theorem]{Conjecture}
\newtheorem{Claim}[theorem]{Claim}

\newtheorem*{theorem*}{Theorem}

\theoremstyle{definition}
\newtheorem{definition}{Definition}
\newtheorem{example}{Example}
\newtheorem{cexample}{Counterexample}
\newtheorem{exercise}{Exercise}
\newtheorem{question}{Question}
\newtheorem{remark}{Remark}

\addtolength{\textheight}{1.75cm}
\addtolength{\voffset}{-0.5cm}
\addtolength{\footskip}{+0.6cm}
\geometry{margin=1.2in}


\usepackage[backend=biber, maxbibnames=99, bibencoding=utf8, doi=false, isbn=false, url=false, style=alphabetic]{biblatex}
\addbibresource{refs.bib}

    \begin{document}
\fi

\section{Concluding remarks}% Section名

%\setcounter{section}{6}

\nsubsection{Questions related to $\AdDMR_0$}

In this subsection, we discuss remaining questions concerning $\AdDMR_0$.
Our first question asks whether $\addmr\cap \Fad$ is contained in $V_{\strprty}$.

\begin{qu}
The affine scheme $\addmr\cap \Fad$ is a closed affine subscheme of $V_{\strprty}$. In other words, $\Psi^{11} + \Psi^{10} + \Psi^{01} = 0$ follows from the defining relation of $\addmr$ for every $\Psi \in \addmr\cap \Fad$.
\end{qu}

Indeed, imposing $V_{\strprty}$ does not change the dimensions:

\begin{center}
\begin{tabular}{c|ccccccccccccc}
$k$ & $0$ & $1$ & $2$ & $3$ & $4$ & $5$ & $6$ & $7$ & $8$ & $9$ & $10$ & $11$ & $\cdots$ \\
\hline
$\dim \addmr^{(k)} \cap \Fad$
& $0$ & $0$ & $0$ & $0$ & $1$ & $0$ & $1$ & $0$ & $1$ & $1$ & $1$ & $1$ & $\cdots$ \\
$\dim \addmr^{(k)} \cap \Fad \cap V_{\strprty}$
& $0$ & $0$ & $0$ & $0$ & $1$ & $0$ & $1$ & $0$ & $1$ & $1$ & $1$ & $1$ & $\cdots$ \\
\end{tabular}
\end{center}


Lastly, let us consider the exponential map of $\TM_1$.
Let $\tau : \TM_1 \rightarrow \Aut(\ide{h}^{\vee}) $ be a collection $(\tau^R : \TM_1(R) \rightarrow \Aut_{R\text{-\textbf{alg}}}(R\cotimes\ide{h}^{\vee}))_{R \in \Qalg}$.
One notes that a differential map $d\tau : \ide{tm}_1 \rightarrow \ide{der}(\ide{h}^{\vee}) $ of $\tau^{\Q}$ induces a natural transformation $\tm_{1,\ide{a}} \rightarrow \ide{der}(\ide{h}^{\vee})$. We denote this natural transformation by $d\tau_{\ide{a}}$.
Since $d\tau_{\ide{a}}(R)$ is a pronilpotent derivation map on $R\cotimes \ide{h}^{\vee}$, then, there exist naturally isomorphisms of affine schemes  
\[
\exp^{\circledast_1} : \tm_{1,\ide{a}} \rightarrow \TM_1 \quad \text{and} \quad \exp : \ide{der}(\ide{h}^{\vee}) \rightarrow \Aut(\ide{h}^{\vee})
\]
that make the following diagram commute:
\[
\xymatrix{
\TM_1 \ar[r]^-{\tau} & \Aut(\ide{h}^{\vee}) \\
\tm_{1,\ide{a}} \ar[u]^{\exp^{\circledast_1}}\ar[r]_-{d\tau_{\ide{a}}} &\ide{der}(\ide{h}^{\vee})_{\ide{a}} \ar[u]_{\exp}
}
\]
That is, for every \(\Q\)-algebra \(R\) and any \(\psi \in \tm_{1,\ide{a}}(R)\), the following equality holds:
\[
\kappa_{\exp^{\circledast_1,R}(\psi)} = \tau^R \circ \exp^{\circledast_1,R}(\psi) = \exp^R \circ d\tau_{\ide{a}}^R(\psi) = \exp^R(d_{\psi})
\]
Substituting $x_1$ into the above, we obtain the explicit formula for the exponential map:
\[
\exp^{\circledast_1,R}(\psi) = \exp^R(d_{\psi})(x_1).
\]


We then expect the following to hold:

\begin{qu}\label{conj:AdDMR-exp}
Is the following exponential map
\[
\exp^{\circledast_1} : \addmr_{\ide{a}} \times_{\tm_{1,\ide{a}}} \ide{F}_{2,\ide{a}}^{\ad(x_1)} \rightarrow \AdDMR_0\times_{\TM_1} \FAd
\]
isomorphic? 
\end{qu}
One may note that the tangent space of $\AdDMR_0 \times_{\TM_1} \FAd$ at $x_1$ is $\addmr \cap \Fad$ and the following $(\addmr \cap \Fad)_{\ide{a}} \cong \addmr_{\ide{a}} \times_{\tm_{1,\ide{a}}} \ide{F}_{2,\ide{a}}^{\ad(x_1)}$ holds.


\nsubsection{Toward formal Kaneko-Zagier conjecture}

The formal Kaneko-Zagier conjecture, stated by Kaneko and Zagier \cite{Kan1} and recently rearranged by Bachmann and Risan \cite{Risan}, is one of the lifts of the Kaneko-Zagier conjecture.
To begin with, we recall a study on the FMZVs and SMZVs.
\begin{df}
\begin{enumerate}
\item For $(k_1,\ldots,k_r)$, we define finite multiple zeta values as elements of $\mathscr{A}$ (see Section \ref{section:Introduction}) by
\[
\zeta_{\A}(k_1,\ldots,k_r) := \left( \sum_{0<m_1<\cdots < m_r<p} \frac{1}{m_1^{k_1}\cdots m_r^{k_r}} \right)_p \in \A.
\]
We set $\zeta_{\A}(\emptyset) = 1$.
\item For $(k_1,\ldots,k_r)$, we define SMZVs as elements of $\mathcal{Z}/\zeta(2)\mathcal{Z}$ by
\[
\zeta_{\S}(k_1,\ldots,k_r) := \zeta_{\mathrm{Ad}}(k_1,\ldots,k_r;0).
\]
We set $\zeta_{\S}(\emptyset) = 1$.
\end{enumerate}
\end{df}
Let $\mathcal{Z}_{\A}$ (respectively, $\mathcal{Z}_{\S}$) be the $\Q$-linear space generated by finite multiple zeta values (respectively, SMZVs).
By Yasuda's theorem (\cite[Theorem 6.1]{Yasuda}), it follows that $\mathcal{Z}_{\S} = \mathcal{Z}/ \zeta(2)\mathcal{Z}$.
Put $\bullet \in \{\A,\S\}$. We define two $\Q$-linear maps 
\begin{align*}
Z_{\bullet}^{\shuffle} : &\Q + x_1 \ide{h} \rightarrow \mathcal{Z}_{\bullet} ; x_1x_0^{k_1-1}\cdots x_1x_0^{k_r-1} \mapsto \zeta_{\bullet}(k_1,\ldots,k_r)\\
Z_{\bullet}^{*} : & \Q\langle Y \rangle \rightarrow \mathcal{Z}_{\bullet} ; y_{k_1}\cdots y_{k_r} \mapsto \zeta_{\bullet}(k_1,\ldots,k_r)
\end{align*}
for $k_1,\ldots,k_r \in \mathbb{Z}_{>0}$.

Both analogs of multiple zeta values satisfy the following $\Q$-linear relations

\begin{thm}\label{thm:ds-AS}
\begin{enumerate}
\item Let $x_1u$, $x_1v \in  x_1\ide{h}$. Then, it follows that
\[
Z_{\bullet}^{\shuffle}(x_1u \shuffle x_1v) = (-1)^{\wt u + 1} Z_{\bullet}^{\shuffle}(x_1( \overset{\leftarrow}{v} x_1u )).
\]
\item Let $u$, $v \in \Q\langle Y \rangle$. Then, it follows that
\[
Z_{\bullet}^* (u * v) = Z_{\bullet}^*(u) Z_{\bullet}^*(v).
\]
\end{enumerate}
\end{thm}
From the point of view of Theorem \ref{thm:ds-AS}, the conjecture, stated by Kaneko and Zagier\cite{Kan1}, can be seen as a lift of the Kaneko-Zagier conjecture.

\begin{df}\label{df:ffmzs}
We define a formal finite multiple zeta space $\mathcal{Z}_{\A}^f$ as the $\Q$-algebra generated by formal symbols $\zeta_{\A}^f (u)$ ($u\in X^*$) satisifying the following:
\begin{enumerate}
\item\label{df:ffmzs-1} $\zeta_{\A}^f(\emptyset) = 1$,
\item\label{df:ffmzs-2} $\zeta_{\A}^f(x_0) = \zeta_{\A}^f(x_1) = 0$,
\item\label{df:ffmzs-3} $\zeta_{\A}(x_1 u \shuffle x_1 v) =  (-1)^{\wt u + 1} \zeta_{\A}^f(x_1( \overset{\leftarrow}{v} x_1u ))$ for  $x_1u$, $x_1v \in  x_1\ide{h}$, and
\item\label{df:ffmzs-4} $\zeta_{\A}^f (\mathbf{p}(u * v)) = \zeta_{\A}^f(\mathbf{p}(u)) \zeta_{\A}^f(\mathbf{p}(v))$ for $u$, $v \in \Q\langle Y \rangle$.
\end{enumerate}
\end{df}
Then by Definition \ref{df:ffmzs}. (\ref{df:ffmzs-1}), and (\ref{df:ffmzs-4}), $\mathcal{Z}_{\A}^f$ forms an unital, commutative, and associative $\Q$-algebra.

\begin{conj}[{\cite{Kan1}}]\label{conj:formal-KZ}
We denote $\zeta_{\A}^f (x_1x_0^{k_1-1}\cdots x_1x_0^{k_r-1})$ by $\zeta_{\A}^f (k_1,\ldots,k_r)$.
Then,  the following map is the isomorphism as $\Q$-algebra:
\[
\mathcal{Z}_{\A}^f \rightarrow \mathcal{Z}^f/\zeta^f(2)\mathcal{Z}^f ; \zeta_{\A}^f(k_1,\ldots,k_r) \mapsto \zeta_{\S}^f(k_1,\ldots,k_r).
\]
Here, $\zeta_{\S}^f(k_1,\ldots,k_r) : = \sum_{j = 0}^r (-1)^{k_{j+1}+ \cdots +k_r} \zeta^f(k_1,\ldots,k_j)\zeta^f(k_r,\ldots,k_{j+1})$.
\end{conj}



We define the affine scheme
\[
  \DMR_0^{\A}:=\Hom_{\Qalg}(\mathcal{Z}_{\A}^f,-):\Qalg\rightarrow\Set ,
\]
the functor represented by the $\Q$-algebra $\mathcal{Z}_{\A}^f$.
Our target is the conjectural identification
\[
 \DMR_0^{\A} \;\cong\; \AdDMR_0 \times_{\TM_1} \FAd .
\]
This identification is closely connected with Rosen's lifting conjecture \cite[Conjecture A]{Rosen-asymptotic}.
Namely, it asks whether every adjoint multiple zeta value in the regularized range $l>0$ can be expressed as a $\Q$-linear combination of those with $l = 0$ (i.e. SMZVs).
Indeed, AdMZVs admit iterated integral expressions \cite{Hirose}. When an integral diverges, its value is defined by a canonical regularization. For $l=0$ (SMZVs) no regularization is needed, while for $l>0$ regularization is required. The lifting problem can therefore be restated as follows: are AdMZVs with $l>0$ expressible as $\Q$-linear combinations of those with $l=0$?
One may note that Yasuda proved that SMZVs (the case $l=0$) span $\mathcal{Z}$ \cite{Yasuda}.
Thus, AdMZVs with $l>0$ can be written as $\Q$-linear combinations of SMZVs by Yasuda's theorem. However, Rosen's lifting requires an explanation within the adjoint/iterated integral framework, and this remains an open question.




\ifdefined\isMainDocument
\else
    %\bibliographystyle{amsplain}
    %\bibliography{reference-adjoint}
    \end{document}
\fi

\ifdefined\isMainDocument
\else
    \documentclass[10pt,oneside,reqno]{amsart}
    \usepackage{xr}
    \externaldocument{../main} % Main の .aux を参照
    \usepackage{latexsym}
\usepackage{amscd}
\usepackage{amsmath}
\usepackage{amssymb}
\usepackage[mathscr]{euscript}
\usepackage{graphicx}
\usepackage{geometry}
\usepackage{mathtools}
\usepackage[all]{xy}
\usepackage[dvipsnames]{xcolor}
\usepackage[colorlinks,final,hyperindex]{hyperref}
\usepackage{tikz}
\usepackage{tikz-cd}
\usetikzlibrary{decorations.pathmorphing,decorations.markings,arrows,calc,shapes.geometric,arrows.meta,positioning}
\usepackage{enumerate}
\usepackage{multirow}

\usepackage[bb=dsserif]{mathalpha}
\usepackage{bm}

\newcommand{\QQ}{\ensuremath{\mathbb{Q}}}
\newcommand{\DD}{\ensuremath{\mathbb{D}}}
\newcommand{\CC}{\ensuremath{\mathbb{C}}}
\newcommand{\RR}{\ensuremath{\mathbb{R}}}
\newcommand{\ZZ}{\ensuremath{\mathbb{Z}}}
\newcommand{\NN}{\ensuremath{\mathbb{N}}}

\newcommand{\g}{\ensuremath{\mathfrak{g}}}
\renewcommand{\SS}{\ensuremath{\mathbb{S}}}

\newcommand{\id}{\ensuremath{\mathrm{id}}}

\makeatletter

\newcommand{\bbone}{\mathbb{1}}

\newcommand{\calA}{\mathcal{A}}
\newcommand{\calB}{\mathcal{B}}
\newcommand{\calC}{\mathcal{C}}
\newcommand{\calD}{\mathcal{D}}
\newcommand{\calE}{\mathcal{E}}
\newcommand{\calF}{\mathcal{F}}
\newcommand{\calG}{\mathcal{G}}
\newcommand{\calH}{\mathcal{H}}
\newcommand{\calI}{\mathcal{I}}
\newcommand{\calJ}{\mathcal{J}}
\newcommand{\calK}{\mathcal{K}}
\newcommand{\calL}{\mathcal{L}}
\newcommand{\calM}{\mathcal{M}}
\newcommand{\calN}{\mathcal{N}}
\newcommand{\calO}{\mathcal{O}}
\newcommand{\calP}{\mathcal{P}}
\newcommand{\calQ}{\mathcal{Q}}
\newcommand{\calR}{\mathcal{R}}
\newcommand{\calS}{\mathcal{S}}
\newcommand{\calT}{\mathcal{T}}
\newcommand{\calU}{\mathcal{U}}
\newcommand{\calV}{\mathcal{V}}
\newcommand{\calW}{\mathcal{W}}
\newcommand{\calX}{\mathcal{X}}
\newcommand{\calY}{\mathcal{Y}}
\newcommand{\calZ}{\mathcal{Z}}
\newcommand{\kk}{\Bbbk}

\newcommand{\scrY}{\mathscr{Y}}
\newcommand{\scrV}{\mathscr{V}}
\newcommand{\scrP}{\mathscr{P}}

\newcommand{\Rep}{\mathop{\mathrm{Rep}}\nolimits}
\newcommand{\alg}{\mathop{\mathrm{alg}}\nolimits}
\newcommand{\Span}{\mathop{\mathrm{Span}}\nolimits}
\newcommand{\proj}{\mathop{\mathrm{proj}}\nolimits}
\newcommand{\Hom}{\mathop{\mathrm{Hom}}\nolimits}
\newcommand{\PHom}{\mathop{\mathrm{PHom}}\nolimits}
\newcommand{\Tw}{\mathop{\mathrm{Tw}}\nolimits}
\newcommand{\Dim}{\mathop{\mathrm{Dim}}\nolimits}
\newcommand{\Imm}{\mathop{\mathrm{Im}}\nolimits}
\newcommand{\Aus}{\mathop{\mathrm{Aus}}\nolimits}
\newcommand{\Ann}{\mathop{\mathrm{Ann}}\nolimits}
\newcommand{\APC}{\mathop{\mathrm{APC}}\nolimits}
\newcommand{\Barr}{\mathop{\mathrm{Bar}}\nolimits}
\newcommand{\Cone}{\mathop{\mathrm{Cone}}\nolimits}
\newcommand{\modu}{\mathop{\mathrm{mod}}\nolimits}
\newcommand{\dgMod}{\mathop{\mathrm{dgMod}}\nolimits}
\newcommand{\soc}{\mathop{\mathrm{soc}}\nolimits}
\newcommand{\dg}{\mathop{\mathrm{dg}}\nolimits}
\newcommand{\red}{\mathop{\mathrm{red}}\nolimits}
\newcommand{\Map}{\mathop{\mathrm{Map}}\nolimits}
\newcommand{\inj}{\mathop{\mathrm{inj}}\nolimits}
\newcommand{\op}{\mathop{\mathrm{op}}\nolimits}
\newcommand{\Ker}{\mathop{\mathrm{Ker}}\nolimits}
\newcommand{\Tor}{\mathop{\mathrm{Tor}}\nolimits}
\newcommand{\Tot}{\mathop{\mathrm{Tot}}\nolimits}
\newcommand{\Ext}{\mathop{\mathrm{Ext}}\nolimits}
\newcommand{\sg}{\mathop{\mathrm{sg}}\nolimits}
\newcommand{\Sl}{\mathop{\mathfrak{sl}}\nolimits}
\newcommand{\nc}{\mathop{\mathrm{nc}}\nolimits}
\newcommand{\Tr}{\mathop{\mathrm{Tr}}\nolimits}
\newcommand{\Irr}{\mathop{\mathrm{Irr}}\nolimits}
\newcommand{\HC}{\mathop{\mathrm{HC}}\nolimits}
\newcommand{\HH}{\mathop{\mathrm{HH}}\nolimits}
\newcommand{\THH}{\mathop{\mathrm{TH}}\nolimits}
\newcommand{\Perf}{\mathop{\mathrm{Perf}}\nolimits}
\newcommand{\del}{\partial}
\newcommand{\ev}{\mathop{\mathrm{ev}}\nolimits}
\newcommand{\co}{\mathop{\mathrm{co}}\nolimits}
\newcommand{\pt}{\mathop{\mathrm{pt}}\nolimits}
\newcommand{\wt}{\mathop{\mathrm{wt}}\nolimits}
\newcommand{\pCY}{\mathop{\mathrm{pCY}}\nolimits}
\newcommand{\gr}{\mathop{\mathrm{gr}}\nolimits}
\newcommand{\coker}{\mathop{\mathrm{coker}}\nolimits}


\newcommand{\Top}{\mathop{\mathrm{Top}}\nolimits}
\newcommand{\Set}{\mathop{\mathrm{Set}}\nolimits}
\newcommand{\Vect}{\mathop{\mathrm{Vect}}\nolimits}
\newcommand{\Mod}{\mathop{\mathrm{Mod}}\nolimits}
\newcommand{\Ob}{\mathop{\mathrm{Ob}}\nolimits}
\newcommand{\Ind}{\mathop{\mathrm{Ind}}\nolimits}
\newcommand{\Sym}{\mathop{\mathrm{Sym}}\nolimits}
\newcommand{\Alt}{\mathop{\mathrm{Alt}}\nolimits}
\newcommand{\End}{\mathop{\mathrm{End}}\nolimits}
\newcommand{\Ch}{\mathop{\mathrm{Ch}}\nolimits}


\newcommand{\Ass}{\mathop{\mathrm{Ass}}\nolimits}
\newcommand{\Com}{\mathop{\mathrm{Com}}\nolimits}
\newcommand{\Lie}{\mathop{\mathrm{Lie}}\nolimits}
\newcommand{\colim}{\mathop{\mathrm{colim}}\nolimits}

\newcommand{\Op}{\mathop{\mathsf{Operads}}\nolimits}
\newcommand{\Diop}{\mathop{\mathsf{Dioperads}}\nolimits}
\newcommand{\PDiop}{\mathop{\mathsf{PDioperads}}\nolimits}
\newcommand{\Coop}{\mathop{\mathsf{Cooperads}}\nolimits}
\newcommand{\Codiop}{\mathop{\mathsf{Codioperads}}\nolimits}
\newcommand{\PCodiop}{\mathop{\mathsf{PCodioperads}}\nolimits}

\newcommand{\antish}{\ensuremath{\mbox{!`}}}


\tikzset{->-/.style={decoration={
			markings,
			mark=at position #1 with {\arrow{>}}},postaction={decorate}}}
\tikzset{-w-/.style={decoration={
			markings,
			mark=at position #1 with {\arrow{Stealth[fill=white,scale=1.4]}}},postaction={decorate}}}
\tikzset{->-/.default=0.65}
\tikzset{-w-/.default=0.65}
\tikzstyle{bullet}=[circle,fill=black,inner sep=0.5mm]
\tikzstyle{circ}=[circle,draw=black,fill=white,inner sep=0.5mm]
\tikzstyle{vertex}=[circle,draw=black,thick,inner sep=0.5mm]
\tikzstyle{dot}=[draw,circle,fill=black,minimum size=0.5mm,inner sep = 0mm, outer sep = 0mm]
\tikzset{darrow/.style={double distance = 4pt,>={Implies},->},
	darrowthin/.style={double equal sign distance,>={Implies},->},
	tarrow/.style={-,preaction={draw,darrow}},
	qarrow/.style={preaction={draw,darrow,shorten >=0pt},shorten >=1pt,-,double,double
		distance=0.2pt}}
\newcommand\tikcirc[1][2.5]{\tikz[baseline=-#1]{\draw[thick](0,0)circle[radius=#1mm];}}
\newcommand{\tikzfig}[1]{\begin{tikzpicture}[auto,baseline={([yshift=-.5ex]current bounding box.center)}]#1\end{tikzpicture}}
\usetikzlibrary{decorations.pathreplacing}

\usepackage{amsthm}
\usepackage{thmtools}
\usepackage[noabbrev,capitalize]{cleveref}
\newtheorem{theorem}{Theorem}
\newtheorem{lemma}[theorem]{Lemma}
\newtheorem{proposition}[theorem]{Proposition}
\newtheorem{corollary}[theorem]{Corollary}
\newtheorem{conjecture}[theorem]{Conjecture}
\newtheorem{Claim}[theorem]{Claim}

\newtheorem*{theorem*}{Theorem}

\theoremstyle{definition}
\newtheorem{definition}{Definition}
\newtheorem{example}{Example}
\newtheorem{cexample}{Counterexample}
\newtheorem{exercise}{Exercise}
\newtheorem{question}{Question}
\newtheorem{remark}{Remark}

\addtolength{\textheight}{1.75cm}
\addtolength{\voffset}{-0.5cm}
\addtolength{\footskip}{+0.6cm}
\geometry{margin=1.2in}


\usepackage[backend=biber, maxbibnames=99, bibencoding=utf8, doi=false, isbn=false, url=false, style=alphabetic]{biblatex}
\addbibresource{refs.bib}

    \begin{document}
\fi

\appendix

%\setcounter{section}{6}
\section{Proof of Lemma \ref{lem:essential}}\label{section:proof of lemma}

In this section, we prove Lemma \ref{lem:essential} to complete our main theorem.
We use the commutative power series $\vimo_{\Psi}$ which corresponds to $\Psi \in \ide{h}^{\vee}$.
Notations and operations are due to Ecalle's mould theory, but we will not discuss further here.


To begin with, let $U: = \{\u_i, \mid i\ge 0\}$ be a set of indeterminates and $\mathcal{G} := \Q + \bigcup_{k\ge 0}\Q[[\u_0,\ldots,\u_k]]$.

\nsubsection{Commutative powes series corresponding to $\phi \in \ide{h}^{\vee}$}

For $\phi \in \ide{h}^{\vee}$, which suppose that the number of occurrences of $x_1$ is equal to $r\in\mathbb{Z}_{\ge 0}$, we define $\vimo_{\phi}(\u_0,\ldots,\u_r)\in \mathcal{G}$ by
\[
\vimo_{\phi}(\u_0,\u_1,\ldots,\u_r) := \sum_{k_0,\ldots,k_r \ge 0} \langle \phi\mid x_0^{k_r}x_1\cdots x_0^{k_1}x_1x_0^{k_0} \rangle \u_0^{k_0}\cdots \u_r^{k_r}.
\]
Furthermore, we define $\mi_{\phi}$, $\ma_{\phi}$ by
\begin{align*}
\ma_{\phi}(\u_1,\ldots,\u_r) :=& \vimo_{\phi}^{r+1}(0,\u_1,\u_1+\u_2,\ldots,\u_1+\cdots + \u_r)\\
\mi_{\phi}(\u_1,\ldots,\u_r) :=& \vimo_{\phi}^{r+1}(0,\u_1,\ldots,\u_r).
\end{align*}

Let $\mathcal{U}_{\Z}$ be the $\Z$-module generated by $U$, and $\mathcal{U}_{\Z}^{\bullet}$ be the free monoid generated by $\mathcal{U}_{\Z}$ with the empty word $\emptyset$.
 For any word $u \in \mathcal{U}_{\Z}^{\bullet}$, we define $\deg u$ as length of $u$. Let $A_{\mathcal{U}} := \Q \langle \mathcal{U}_{\Z}\rangle$.

Let $f \in \mathcal{G}$.
By the following $\Q$-linear map, we consider an element of $A_{\mathcal{U}}$ as the components of some commutative power series:
\[
A_{\mathcal{U}}\ni (\v_1)\cdots (\v_r) \mapsto f(\v_1,\ldots,\v_r) \in A_{\mathcal{U}}
\]
for $(\v_1)\cdots (\v_r) \in \mathcal{U}_{\Z}^{\bullet}$.
Additionaly, put $\mathcal{K}:= \Q(\u_i \mid i\ge 0)$ and $A_{\mathcal{U}}^{\text{rat}} := \mathcal{K}\langle \mathcal{U}_{\Z}\rangle$.
If necessary, we consider the coefficient expansion of $\mathcal{K}$, that is,
\[
A_{\mathcal{U}}^{\text{rat}}\ni h\cdot (\v_1)\cdots (\v_r) \mapsto h\cdot f(\v_1,\ldots,\v_r) \in A_{\mathcal{U}}^{\text{rat}}
\]
for $h \in \mathcal{K}$, $(\v_1)\cdots (\v_r) \in \mathcal{U}_{\Z}^{\bullet}$.




\nsubsection{Operations around commutative power series}

\begin{comment}

\vspace{0.5cm}
\hspace{-0.5cm}\textbf{neg, anti, mantar and pus.}
\vspace{0.5cm}

Let $f(\u_0,\ldots,\u_{r-1}) \in \mathcal{G}$ and $\x_1\cdots\x_r \in \mathcal{U}_{\Z}^{\bullet}$.

\begin{df}
We define
\begin{align*}
\neg (f)(\x_1,\ldots,\x_r) =& f(-\x_1,\ldots,-\x_r),\\
\anti (f)(\x_1,\ldots,\x_r) =& f(\x_r,\ldots,\x_1),\\
\mantar (f)(\x_1,\ldots,\x_r) =& (-1)^{r-1} f(\x_r,\ldots,\x_1)\text{ and }\\
\pus (f)(\x_1,\ldots,\x_r) =&  f(\x_r,\x_1,\ldots,\x_{r-1}).
\end{align*}
\end{df}

\vspace{0.5cm}
\hspace{-0.5cm}\textbf{Parallel translation $\utilde{t}$.}
\vspace{0.5cm}



The parallel translation $\utilde{t}$ is defined by
\[
\utilde{t}(f)(\x_1,\ldots,\x_r) := 
f(\x_2-\x_1,\ldots,\x_r-\x_1).
\]

The dimould various of the parallel translation $\utilde{t} : \mathcal{M}_2(\mathcal{H}) \rightarrow\mathcal{M}_2^{(1,0)}(\mathcal{H})$ is defined by
\[
\utilde{t}(M)^{r,s}(\x_1,\ldots,\x_r;\y_1,\ldots,\y_l) := 
M^{r-1,s}(\x_2-\x_1,\ldots,\x_r-\x_1;\y_1-\x_1,\ldots,\y_l-\x_1).
\]
By abuse of notation, we use the same symbol $\utilde{t}$ for both cases.
\end{comment}

\vspace{0.5cm}
\hspace{-0.5cm}\textbf{Shuffle product}
\vspace{0.5cm}

We define a product $\shuffle$ on $A_{\mathcal{U}}$ as a $\Q$-bilinear binary operation given by $\emptyset \shuffle \w := \w\shuffle \emptyset = \w$ and
\[
a\w\shuffle b\v := a(\w\shuffle b\v) + b(a\w\shuffle \v)
\]
for $a$, $b\in \mathcal{U}_{\Z}$ and $\w$, $\v\in \mathcal{U}_{\Z}^{\bullet}$.
Note that a pair $(A_{\mathcal{U}},\shuffle)$ forms a unital, commutative, and associative $\mathbb{Q}$-algebra.




The following properties related to $\ma_{\phi}$ and $\mi_{\phi}$ are well-known.
\begin{prop}[{(\cite[(\ref{sh-1-1}) = Lemma 45 (4), (\ref{sh-1-2}) = Proposition 27]{Furusho-Hirose-Komiyama}}]\label{prop:sh-1}
Let $\phi \in \ide{h}^{\vee}$.
\begin{enumerate}
\item \label{sh-1-1}  If $\langle \phi \mid x_1 \shuffle u \rangle = 0$ for $u\in X$ except for $1$, then for $r\in\mathbb{Z}_{>0}$, we have
\begin{align}
\vimo_{\phi}^{r+1}(\x_0,\x_1,\ldots,\x_r) = \vimo_{\phi}^r(0,\x_1-\x_0,\ldots,\x_r-\x_0). \label{eq:shift}
\end{align}
\item\label{sh-1-2} If $\langle \phi \mid u\shuffle v\rangle = 0$ for $u$ and $v\in X^{*}\setminus\{1\}$, then 
\[
\ma_{\phi} (\mathbb{X} \shuffle \mathbb{Y}) = 0
\]
holds for $\mathbb{X}$ and $\mathbb{Y} \in \mathcal{U}_{\Z}^{\bullet} \setminus\{\emptyset\}$.
\end{enumerate}
\end{prop}



\vspace{0.5cm}
\hspace{-0.5cm}\textbf{Harmonic product}
\vspace{0.5cm}

We consider another product $*$ on $A_{\mathcal{U}}^{\mathrm{rat}}$.
The $\Q$-bilinear binary operation $*$ on $A_{\mathcal{U}}^{\mathrm{rat}}$ is inductively defined by
\[
w * \emptyset = \emptyset * w = w
\]
and
\[
a\w * b\v := 
\begin{cases}
0 & a= b\\
a(\w * b\v) + b(a\w * b\v) + \frac{1}{a-b}(a(\w*\v) - b(\w * \v)) & a\neq b
\end{cases}
\]
for $a$, $b\in\mathcal{U}_{\Z}$ and $\w$, $\v\in \mathcal{U}_{\Z}^{\bullet}$.

\begin{rmk}
The harmonic product $*$ on $\Q\langle Y\rangle$ corresponds to the harmonic productn $*$ on $A_{\mathcal{U}}^{\mathrm{rat}}$.
Namely, for $\Psi \in \ide{h}^{\vee}$, it follows that
\begin{align*}
&\vimo_{\Psi}(0,(\x_1,\ldots,\x_r) * (\x_{r+1},\ldots,\x_{r+l}))\\
=&\sum_{k_1,\ldots,k_{r+l}\ge 0} \langle \Pi_Y(\Psi) \mid (y_{k_{r+l}}\cdots y_{r+1}) * (y_{r} \cdots y_1)\rangle \x_1^{k_1}\cdots \x_{r+l}^{k_{r+l}}.
\end{align*} 
\end{rmk}


Put $z_k := x_0^{k-1}x_1$ for $k\in\mathbb{Z}_{\ge 1}$.
In terms of commutative power series, the interpretation of the harmonic product relations modulo products is as follows:

\begin{prop}[{\cite[Remark 31 (i) and Lemma 127]{Furusho-Hirose-Komiyama}}]\label{har-com}
Let $\Psi \in \ide{h}^{\vee}$. Then the condition $\langle \Pi_Y(\Psi) \mid u * v \rangle = 0$ for $u$, $v \in Y^*\setminus\{1\}$ is equivalent to the condition $\mi_{\Psi} (\mathbb{X} * \mathbb{Y}) = 0$ for $\mathbb{X}$, $\mathbb{Y} \in \mathcal{U}_{\Z}^{\bullet}\setminus\{\emptyset\}$.
\end{prop}










\nsubsection{ $\addmr$ via commutative power series}



In this section, we consider $\addmr$ via commutative power series. %$\ls$, and $\adls$.
Based on Theorem \ref{prop:sh-1} and Theorem \ref{har-com}, $\addmr$ can be reformulated as follows:

\begin{prop}\label{prop:addmr-mould}
Let $\Psi \in  \ide{h}^{\vee}$. Then $\Psi \in \addmr(\Q)$ if and only if $\Psi$ satisfies the following properties:
\begin{enumerate}
\item\label{enumi:ad--1} $\mi_{\Psi}(\v,0) = 0$ for $\v \in \mathcal{U}_{\Z}$,
\item\label{enumi:ad--2} $\ma_{\Psi}(\mathbb{X} \shuffle \mathbb{Y}) = 0$ for $\mathbb{X}$, $\mathbb{Y} \in \mathcal{U}_{\Z}^{\bullet}\setminus\{\emptyset\}$, and
\item\label{enumi:ad--3} It follows that 
\begin{align}
\vimo_{\Psi}(\z_1, \mathbb{X} * \mathbb{Y} , \z_2) = 0 \label{eq:adjoint-harmonic}
\end{align}
for $\mathbb{X}$, $\mathbb{Y} \in \mathcal{U}_{\Z}^{\bullet}\setminus\{\emptyset\}$ and $\z_1$ and $\z_2 \in \mathcal{U}_{\Z}\setminus\{\emptyset\}$.
\end{enumerate}
\end{prop}

\begin{proof}
Let $\Psi \in \addmr$.
The equivalence between $\langle \Psi \mid x_1x_0^{k}x_1\rangle = 0 $ for $k\in\mathbb{Z}_{\ge 0}$ and the Condition (\ref{enumi:ad--1}) is clear.

The condition $\Delta_{\shuffle}(\Psi) = \Psi\otimes 1 + 1\otimes \Psi$ corresponds to the Condition (\ref{enumi:ad--2}), as shown in  Proposition \ref{prop:sh-1} (\ref{sh-1-2}).



Let us denote $\mathbb{X} = (\x_1,\ldots,\x_r)$, $\mathbb{Y} = (\y_1,\ldots,\y_l)$.
The Condition (\ref{enumi:ad--3}) is equivalent to Definition \ref{def:addmr} (\ref{def:addmr--3}) and Proposition \ref{prop:sh-1}. (\ref{sh-1-1}). In fact, Definition \ref{def:addmr} (\ref{def:addmr--3}) is equivalent to
\begin{align*}
& \langle \mathbf{q}_{\#}^{\vee} (\Psi) \mid y_{k_1}\cdots y_{k_r} * y_{s_1}\cdots y_{s_t} \rangle = 0\\
\Longleftrightarrow&\sum_{l\ge 1} T^l \langle \mathbf{q}^{\vee}(\Psi) \mid y_l(y_{k_1}\cdots y_{k_r} * y_{s_1}\cdots y_{s_t})\rangle = 0\\
\Longleftrightarrow&  \langle \mathbf{q}^{\vee}(\Psi) \mid y_l(y_{k_1}\cdots y_{k_r} * y_{s_1}\cdots y_{s_t})\rangle = 0\;\;(l\in\mathbb{Z}_{\ge 1})
\end{align*}
for $k_1,\ldots,k_r, s_1,\ldots,s_t \in\mathbb{Z}_{>0}$, which implies
\begin{align*}
&\vimo_{\Psi}(0,\mathbb{X} * \mathbb{Y},\z)\\
=&\sum_{\substack{l\ge 1 \\ k_1,\ldots,k_r\ge 1 \\ s_1,\ldots,s_t\ge 1}}\langle \mathbf{q}^{\vee}(\Psi) \mid y_l(y_{k_1}\cdots y_{k_r} * y_{s_1}\cdots y_{s_t})\rangle \y_{t}^{s_t-1}\cdots \y_{1}^{s_1-1}\x_r^{k_r-1}\cdots \x_1^{k_1-1}\z^{l-1} \\
=&0.
\end{align*}
Because of Proposition \ref{prop:sh-1}. (\ref{sh-1-1}), 
\[
\vimo_{\Psi}(0,(\y_{s_t},\cdots,\y_{s_1}) * (\x_r,\cdots,\x_1),\z) = 0
\]
 turns out to be 
\[
\vimo_{\Psi}(\z_1,(\y_{s_t},\cdots,\y_{s_1}) * (\x_r,\cdots,\x_1),\z_2) = 0.
\]
This completes the proof.



\end{proof}

\begin{prop}\label{prop:adjoint-mould}
Let $\Psi \in \ide{h}^{\vee}$ with $\langle \Psi \mid w \rangle = 0$ for $w\in X^*$ with $\wt w \le 1$.
Then, $\Psi^{00} = 0$  if and only if 
\begin{align}
\vimo_{\Psi}(\x_0,\ldots,\x_r) =& \vimo_{\Psi}(\x_0,\ldots,\x_{r-1},0) + \vimo_{\Psi}(0,\x_1,\ldots,\x_r) - \vimo_{\Psi}(0,\x_1,\ldots,\x_{r-1},0) \label{eq:adjoint-mould}
\end{align}
 for $(\x_0,\ldots,\x_r) \in \mathcal{U}_{\Z}^{\bullet}$.
\end{prop}

\begin{proof}
Since $\Psi^{00} = 0$ is equivalent to $\langle \Psi \mid x_0^{k_0}x_1\cdots x_1x_0^{k_r} \rangle = 0$ for $(k_1,\ldots,k_r) \in \mathbb{Z}_{\ge 0}^r$ with $k_0,k_r > 0$. Therefore,
\begin{align*}
\vimo_{\Psi}(\x_0,\ldots,\x_r) :=& \sum_{k_0,\ldots,k_r\ge 0} \langle \Psi \mid x_0^{k_r}x_1x_0^{r-1}\cdots x_0^{k_1}x_1x_0^{k_0}\rangle\\
=&\left(\sum_{k_1,\ldots,k_r\ge 0} + \sum_{k_0,\ldots,k_{r-1}\ge 0} - \sum_{k_1,\ldots,k_{r-1}\ge 0} \right) \langle \Psi \mid x_0^{k_r}x_1x_0^{k_{r-1}}\cdots x_0^{k_1}x_1x_0^{k_0}\rangle\\
=&\vimo_{\Psi}(\x_0,\ldots,\x_{r-1},0) + \vimo_{\Psi}(0,\x_1,\ldots,\x_r) - \vimo_{\Psi}(0,\x_1,\ldots,\x_{r-1},0)
\end{align*}
as claimed.
\end{proof}

\begin{cor}\label{prop:adjoint-mould}
If $\Psi \in \Fad$, then we have \eqref{eq:adjoint-mould} to hold.
\end{cor}
\begin{proof}
Immediately, this condition follows from the definition of $\Fad$.
\end{proof}

%\textcolor{red}{ここから修正}

\begin{prop}\label{prop:strong-parity}
Let $\Psi \in \ide{h}^{\vee}$. Then $\Psi \in V_{\strprty}$ is equivalent to
\begin{align}
\begin{aligned}
&\vimo_{\Psi}(0,\x_1,\ldots,\x_r,0)\\
=& -\frac{\vimo_{\Psi }(0,\x_1,\ldots,\x_r) - \vimo_{\Psi }(0,\x_1,\ldots,\x_{r-1},0)}{\x_r} \\
&- \frac{\vimo_{\Psi }(\x_1,\ldots,\x_r,0) - \vimo_{\Psi }(0,\x_2,\ldots,\x_{r},0)}{\x_1}
\end{aligned}
 \label{eq:strong-parity}
\end{align}
holds for $r\in\mathbb{Z}_{>0}$.
\end{prop}


\begin{proof}

Since $\Psi \in V_{\strprty}$ satisfies
\begin{align*}
& \langle \Psi \mid x_1x_0^{k_r}x_1\cdots x_0^{k_1}x_1 \rangle\\
=&-  \langle \Psi \mid x_0^{k_r+1}x_1x_0^{k_{r-1}}\cdots x_0^{k_1}x_1 \rangle - \langle \Psi \mid x_1x_0^{k_r}\cdots x_0^{k_2}x_1x_0^{k_1+1}\rangle,
\end{align*}
for $k_1,\ldots,k_r \in \mathbb{Z}_{\ge 0}$, we have

\begin{align*}
&\vimo_{\Psi}(0,\x_1,\ldots,\x_r,0)\\
=&\sum_{k_1,\ldots,k_r\ge 0} \x_1^{k_1}\cdots \x_r^{k_r} \langle \Psi \mid  x_1x_0^{k_r}x_1\cdots x_0^{k_1}x_1 \rangle \\
=&- \sum_{k_1,\ldots,k_r\ge 0}  \x_1^{k_1}\cdots \x_r^{k_r} \left( \langle \Psi \mid x_0^{k_r+1}x_1x_0^{k_{r-1}}\cdots x_0^{k_1}x_1 \rangle + \langle \Psi \mid x_1x_0^{k_r}\cdots x_0^{k_2}x_1x_0^{k_1+1}\rangle  \right)\\
=& -\frac{\vimo_{\Psi }(0,\x_1,\ldots,\x_r) - \vimo_{\Psi}(0,\x_1,\ldots,\x_{r-1},0)}{\x_r} \\
&- \frac{\vimo_{\Psi }(\x_1,\ldots,\x_r,0) - \vimo_{\Psi }(0,\x_2,\ldots,\x_{r},0)}{\x_1}
\end{align*}
as claimed.

\end{proof}

\begin{cor}
If $\Psi \in \ide{F}_2 \cap V_{\strprty}$, we have
\begin{align}
\begin{aligned}
&\vimo_{\Psi}(\y,\x_1,\ldots,\x_r,\y)\\
=& -\frac{\vimo_{\Psi }(0,\x_1,\ldots,\x_r) - \vimo_{\Psi }(0,\x_1,\ldots,\x_{r-1},\y)}{\x_r - \y} - \frac{\vimo_{\Psi }(\x_1,\ldots,\x_r,0) - \vimo_{\Psi }(\y,\x_2,\ldots,\x_{r},0)}{\x_1 - \y}.
\end{aligned} \label{eq:strong-parity-revised}
\end{align}
\end{cor}

\begin{proof}
It follows that
%{\footnotesize
\begin{align*}
&\vimo_{\Psi}(\y,\x_1,\ldots,\x_r,\y) \\
\stackrel{\eqref{eq:shift}}{=} & \vimo_{\Psi}(0,\x_1-\y,\ldots,\x_r-\y,0)\\
\stackrel{\eqref{eq:strong-parity}}{=}& -\frac{\vimo_{\Psi }(0,\x_1-\y,\ldots,\x_r-\y) - \vimo_{\Psi }(0,\x_1-\y,\ldots,\x_{r-1}-\y,0)}{\x_r-\y} \\
&- \frac{\vimo_{\Psi }(\x_1-\y,\ldots,\x_r-\y,0) - \vimo_{\Psi }(0,\x_2-\y,\ldots,\x_{r}-\y,0)}{\x_1-\y}\\
\stackrel{\eqref{eq:shift}}{=} & -\frac{\vimo_{\Psi }(\y,\x_1,\ldots,\x_r) - \vimo_{\Psi }(\y,\x_1,\ldots,\x_{r-1},\y)}{\x_r-\y} \\
&- \frac{\vimo_{\Psi }(\x_1,\ldots,\x_r,\y) - \vimo_{\Psi }(\y,\x_2,\ldots,\x_{r},\y)}{\x_1-\y}\\
\stackrel{\eqref{eq:adjoint-mould}}{=} &  -\frac{\vimo_{\Psi }(0,\x_1,\ldots,\x_r) + \vimo_{\Psi }(\y,\x_1,\ldots,\x_{r-1},0) - \vimo_{\Psi }(0,\x_1,\ldots,\x_{r-1},0)}{\x_r-\y}\\
& + \frac{\vimo_{\Psi }(0,\x_1,\ldots,\x_{r-1},\y) + \vimo_{\Psi }(\y,\x_1,\ldots,\x_{r-1},0) - \vimo_{\Psi }(0,\x_1,\ldots,\x_{r-1},0)}{\x_r-\y}\\
&- \frac{\vimo_{\Psi }(0,\x_2,\ldots,\x_r,\y) + \vimo_{\Psi }(\x_1,\ldots,\x_r,0) - \vimo_{\Psi }(0,\x_2,\ldots,\x_r,0)}{\x_1-\y}\\
&+ \frac{\vimo_{\Psi }(0,\x_2,\ldots,\x_{r},\y) + \vimo_{\Psi }(\y,\x_2,\ldots,\x_{r},0) - \vimo_{\Psi }(0,\x_2,\ldots,\x_{r},0) }{\x_1-\y}\\
=& -\frac{\vimo_{\Psi }(0,\x_1,\ldots,\x_r) - \vimo_{\Psi }(0,\x_1,\ldots,\x_{r-1},\y)}{\x_r - \y} \\
&- \frac{\vimo_{\Psi }(\x_1,\ldots,\x_r,0) - \vimo_{\Psi }(\y,\x_2,\ldots,\x_{r},0)}{\x_1 - \y}
\end{align*}
%}
as claimed.
\end{proof}












%%%%%%%%%%%%%%%%%%%%%%%%%%%%%%%%%%%%%%%%%%%%%%%%%%%%%%%%%%%%

\nsubsection{Preliminaries}

In this subsection, we use special notations for some variables.
Hereafter, unless otherwise stated, capital bold letters denote elements of $\mathcal{U}_{\Z}^{\bullet}$, lowercase script letters denote elements of $\mathcal{U}_{\Z} \setminus\{\emptyset\}$, and lowercase Fraktur letters denote elements of $\mathcal{U}_{\Z}$.

First, we consider $d_{\Psi_1}(\Psi_2)$ via commutative power series for $\Psi_1$, $\Psi_2 \in \tm_1$.
Let $\mathbb{X}$, $\mathbb{Y} \in \mathcal{U}_{\Z}^{\bullet}$.

\begin{prop}\label{prop:deriv-explicit}
For $\Psi_1$, $\Psi_2\in  \ide{h}^{\vee}$, it follows that
\[
\vimo_{d_{\Psi_1}(\Psi_2)}(\x_0,\ldots,\x_r) = \sum_{0\le a < b \le r} \vimo_{\Psi_1}(\x_a,\ldots,\x_b) \vimo_{\Psi_2}(\x_0,\ldots,\x_a,\x_b,\ldots,\x_r).
\]
In other words, we have
\begin{align}
\vimo_{d_{\Psi_1}(\Psi_2)}(\x_0,\ldots,\x_r) =  \sum_{\mathbb{V}_L(\v)\mathbb{V}_M(\v')\mathbb{V}_R = (\z)\cdot \mathbb{X} \cdot (\z') } f_1(\v,\mathbb{V}_M,\v')f_2(\delta_{\z}(\v),(\mathbb{V}_L\cdot (\v)\cdot(\v') \cdot \mathbb{V}_R),\delta_{\z'}(\v')).  \label{eq:deriv-explicit}
\end{align}
Here, for any letters $a$, $b \in \mathcal{U}_{\Z}$, define
\[
\delta_{a}(b):=
\begin{cases}
a & a \neq b\\
\emptyset & a = b.
\end{cases}
\]
\end{prop}



\begin{proof}
Let $k\in\mathbb{Z}_{>0}$, $r\in\mathbb{Z}_{\ge k}$, and $s_0,\ldots,s_k\in\mathbb{Z}_{\ge 0}$. Since the following
{\small
\begin{align*}
&\sum_{n_0,n_1,\ldots,n_r\ge 0} \langle d_{\Psi_1}(x_0^{s_k}x_1\cdots x_1x_0^{s_0}) \mid x_0^{n_r}x_1\cdots x_0^{n_1}x_1x_0^{n_0} \rangle \x_0^{n_0}\cdots \x_r^{n_r}\\
\displaybreak[3]
=&\sum_{n_0,n_1,\ldots,n_r\ge 0}\sum_{j = 1}^k \langle  x_0^{s_k}x_1\cdots x_0^{s_j}\Psi x_0^{s_{j-1}} \cdots x_0^{s_1}x_1x_0^{s_0} \mid x_0^{n_r}x_1\cdots x_0^{n_1}x_1x_0^{n_0} \rangle \x_0^{n_0}\cdots \x_r^{n_r} \\
\displaybreak[3]
=&\sum_{j = 1}^r \x_0^{s_0}\cdots \x_{j-2}^{s_{j-2}}\x_{r + j -k +1}^{s_{j+1}} \cdots \x_r^{s_k} \\
&\times \sum_{\substack{n_j,\ldots,n_{r+j-k-1}\ge 0 \\ n_{j-1}\ge s_{j-1}, n_{r+j-k}\ge s_j}} \langle \Psi \mid x_0^{n_{r + j -k }-s_{j}}x_1x_0^{n_{r+j-k-1}} \cdots x_0^{n_j}x_1x_0^{n_{j-1} - s_{j-1}}\rangle
\x_{j-1}^{n_{j-1} - s_{j-1}} \x_j^{n_j} \cdots \x_{r + j -k -1}^{n_{r+j-k-1}} \x_{r + j -k}^{n_{r + j -k }-s_{j}}  \\
\displaybreak[3]
=&\sum_{j = 1}^r \x_0^{s_0}\cdots \x_{j-2}^{s_{j-2}}\x_{j-1}^{s_{j-1}} \cdot \x_{r+j-k}^{s_j}\x_{r + j -k +1}^{s_{j+1}} \cdots \x_r^{s_k} \\
&\times \sum_{n_{j-1},\ldots,n_{r+j-k}\ge 0 } \langle \Psi \mid x_0^{n_{r + j -k }}x_1x_0^{n_{r+j-k-1}} \cdots x_0^{n_j}x_1x_0^{n_{j-1} }\rangle \x_{j-1}^{n_{j-1} } \x_j^{n_j} \cdots \x_{r + j -k -1}^{n_{r+j-k-1}} \x_{r + j -k}^{n_{r + j -k }}  \\
\displaybreak[3]
=& \sum_{j = 1}^r \x_0^{s_0}\cdots \x_{j-2}^{s_{j-2}}\x_{j-1}^{s_{j-1}} \cdot \x_{r+j-k}^{s_j}\x_{r + j -k +1}^{s_{j+1}} \cdots \x_r^{s_k} \times\vimo_{\Psi_1}(\x_{j-1},\x_j,\ldots,\x_{r+j-k-1},\x_{r+j-k})
\end{align*}
}
holds, we have
{\small
\begin{align*}
&\vimo_{d_{\Psi_1}(\Psi_2)}(\x_0,\ldots,\x_r) \displaybreak[3] \\
=&\sum_{n_0,\ldots,n_r\ge 0} \sum_{k = 1}^r \sum_{s_0,\ldots,s_j\ge 0} \langle \Psi_2\mid x_0^{s_k}x_1\cdots x_1x_0^{s_0} \rangle \langle d_{\Psi_1}(x_0^{s_k}x_1\cdots x_0^{s_1}x_1x_0^{s_0})\mid x_0^{n_r}x_1\cdots x_0^{n_1}x_1x_0^{n_0}\rangle \x_0^{n_0}\cdots \x_r^{n_r} \displaybreak[3] \\
=& \sum_{k = 1}^r \sum_{s_0,\ldots,s_j\ge 0} \langle \Psi_2\mid x_0^{s_k}x_1\cdots x_1x_0^{s_0} \rangle \\
&\times  \sum_{j = 1}^r \x_0^{s_0}\cdots \x_{j-2}^{s_{j-2}}\x_{j-1}^{s_{j-1}} \cdot \x_{r+j-k}^{s_j}\x_{r + j -k +1}^{s_{j+1}} \cdots \x_r^{s_k} \times \vimo_{\Psi_1}(\x_{j-1},\x_j,\ldots,\x_{r+j-k-1},\x_{r+j-k})\\
=&\sum_{j = 1}^r\sum_{k = 1}^r \vimo_{\Psi_1}(\x_{j-1},\x_j,\ldots,\x_{r+j-k-1},\x_{r+j-k}) \vimo_{\Psi_2}(\x_0,\ldots,\x_{j-1},\x_{r+j-k},\ldots,\x_r) \displaybreak[3] \\
=&\sum_{0\le a < b \le r} \vimo_{\Psi_1}(\x_{a},\x_{a+1},\ldots,\x_{b-1},\x_{b}) \vimo_{\Psi_2}(\x_0,\ldots,\x_{a},\x_{b},\ldots,\x_r).
\end{align*}
}
\end{proof}

%Next, we investigate the properties of harmonic products via Mould theory.
\begin{comment}
For $i$, $j\in\mathbb{Z}_{>0}$, we define $\X{i}{j} \in \XZ^{\bullet}$ (respectively, $\Y{i}{j} \in \XZ^{\bullet}$) by
\begin{align*}
\X{i}{j} := 
\begin{cases}
(\x_i)\cdots(\x_j) & \text{if $i\le j$, and }\\
\emptyset \text{ (emptyword of $\XZ^{\bullet}$)} & \text{otherwise.}
\end{cases}\\
\left(\text{respectively, } \Y{i}{j} :=
\begin{cases}
(\y_i,\ldots,\y_j) & \text{if $i\le j$, and }\\
\emptyset  & \text{otherwise.}
\end{cases} \right)
\end{align*}
\end{comment}

For $\x \in \mathcal{U}_{\Z}\setminus\{\emptyset\}$ and $\ide{y} \in \mathcal{U}_{\mathbb{Z}}$, define
\[
c(\x, \ide{y}) := 
\begin{cases}
1 & \ide{y} = \emptyset \\
0& x = \ide{y}\\
\frac{1}{x - \ide{\eta}} & \text{otherwise.}
\end{cases}
\]


\begin{lem}\label{lem:har-inductive}
Let $\mathbb{X} = \mathbb{X}_L (\x) \mathbb{X}_R$, $\mathbb{Y} \in \mathcal{U}_{\mathbb{Z}}^{\bullet}$ with distinct letters.
Then we have
\begin{align}
\begin{aligned}
& \mathbb{X}_L (\x) \mathbb{X}_R * \mathbb{Y}\\
=& \sum_{\mathbb{Y}_L\cdot (\ide{y})\cdot \mathbb{Y}_R =  \mathbb{Y}} c(\x,\ide{y}) (\mathbb{X}_L * \mathbb{Y}_L) \cdot((\x) - (\ide{y}))\cdot (\mathbb{X}_R * \mathbb{Y}_R).
\end{aligned} \label{eq:harmonic-composition}
\end{align}
\end{lem}

\begin{proof}
Let $\mathbb{Y} := (\y_1)\cdots (\y_l)$.
We prove our claim by induction on $\deg\mathbb{X}_L$.
If $\deg \mathbb{X}_L = 0$, that is, $\mathbb{X}_L = \emptyset$, the definition of harmonic product implies
\begin{align*}
(\x) \mathbb{X}_R * \mathbb{Y} =& (\x) (\mathbb{X}_R * \mathbb{Y}) + \frac{1}{\x - \y_1} ((\x) - (\y_1)) (\mathbb{X}_R * ((\y_2)\cdots (\y_r)) ) \\
&+(\y_1) ((\x) (\mathbb{X}_R) * ((\y_2)\cdots (\y_r))) \displaybreak[3] \\
=&\\
\vdots&\\
=&\sum_{j =0}^l (\y_1)\cdots(\y_j)(\x) (\mathbb{X}_R *( (\y_{j+1})\cdots(\y_r))) \\
&+ \sum_{j = 1}^r  \frac{1}{\x - \y_j} (\y_1)\cdots(\y_{j-1})((\x) - (\y_j)) (\mathbb{X}_R *( (\y_{j+1})\cdots(\y_r))) \displaybreak[3] \\
=& \sum_{\mathbb{Y}_L\cdot (\ide{y})\cdot \mathbb{Y}_R =  \mathbb{Y}} c(\x,\ide{y}) \mathbb{Y}_L \cdot((\x) - (\ide{y}))\cdot (\mathbb{X}_R * \mathbb{Y}_R). \displaybreak[3]
\end{align*}
Assume that $\deg \mathbb{X}_L>0$ and our claim holds for all positive integers less than $\deg \mathbb{X}_L$.
Put $\mathbb{X}_L = (\x_1)\mathbb{X}_{M}$. By the induction hypothesis, we have
\begin{align*}
&((\x_1)\mathbb{X}_{M}(\x)\mathbb{X}_R) * \mathbb{Y}\\
\stackrel{\eqref{eq:harmonic-composition}}{=}&\sum_{\mathbb{Y}_L(\ide{y}') \mathbb{Y}_R = \mathbb{Y}} c(\x_1,\ide{y}) \mathbb{Y}_L((\x_1) - (\ide{y}')) ((\mathbb{X}_{M}(\x)\mathbb{X}_R) * \mathbb{Y}_R).
\end{align*}
Since $\deg \mathbb{X}_M < \deg \mathbb{X}_L$, applying the induction hypothesis implies
\begin{align*}
&(\mathbb{X}_{M}(\x)\mathbb{X}_R) * \mathbb{Y}_R\\
=&\sum_{\mathbb{Y}_M (\ide{y})\mathbb{Y}_{RR} = \mathbb{Y}_R} c(\x,\ide{y})(\mathbb{Y}_M * \mathbb{X}_{M})((\x) - (\ide{y}')) (\mathbb{X}_{R} * \mathbb{Y}_{RR}).
\end{align*}
Therefore, we have
\begin{align*}
&((\x_1)\mathbb{X}_{M}(\x)\mathbb{X}_R) * \mathbb{Y} \displaybreak[3] \\
=&\sum_{\substack {\mathbb{Y}_L(\ide{y}') \mathbb{Y}_R = \mathbb{Y} \\ \mathbb{Y}_M (\ide{y})\mathbb{Y}_{RR} = \mathbb{Y}_R}} c(\x_1,\ide{y}') c(\x,\ide{y}) \mathbb{Y}_L((\x_1) - (\ide{y}))(\mathbb{Y}_M * \mathbb{X}_{M})((\x) - (\ide{y}')) (\mathbb{X}_{R} * \mathbb{Y}_{RR}) \displaybreak[3] \\
=&\sum_{\mathbb{Y}_L(\ide{y})\mathbb{Y}_R = \mathbb{Y}}c(\x,\ide{y})\left\{ \sum_{\mathbb{Y}_{LL}(\ide{y}')\mathbb{Y}_M = \mathbb{Y}_L}c(\x_1,\ide{y}') \mathbb{Y}_{LL} ((\x_1) - (\ide{y})') (\mathbb{X}_M * \mathbb{Y}_M) \right\} ((\x) - (\ide{y}')) (\mathbb{X}_{R} * \mathbb{Y}_{R}) \\
\stackrel{\eqref{eq:harmonic-composition}}{=}&\sum_{\mathbb{Y}_L(\ide{y})\mathbb{Y}_R = \mathbb{Y}} c(\x,\ide{y}) \mathbb{Y}_L * ((\x_1) \mathbb{X}_M) ((\x) - (\ide{y}))  (\mathbb{X}_{R} * \mathbb{Y}_{R}) \\
=&\sum_{\mathbb{Y}_L(\ide{y})\mathbb{Y}_R = \mathbb{Y}} c(\x,\ide{y}) (\mathbb{Y}_L *  \mathbb{X}_L) ((\x) - (\ide{y}))  (\mathbb{X}_{R} * \mathbb{Y}_{R})
\end{align*}
as claimed.
\end{proof}


\begin{lem}\label{lem:-har-expl}
For $\Psi_1$, $\Psi_2 \in  \ide{h}^{\vee}$, we have
\[
\vimo_{d_{\Psi_1}}(\Psi_2)(\z,\,\mathbb{X} * \mathbb{Y},\,\z')
= \sum_{(\V,\V')\in S} A_{\V\V'} .
\]
Here $S:=\{(\z,\z'),(\z,\X'),(\z,\Y'),(\X,\z'),(\Y,\z'),(\X,\X'),(\X,\Y'),(\Y,\X'),(\Y,\Y')\}$, and we define

\begin{align*}
A_{\z\z'} =&  f_1(\z, \mathbb{X} * \mathbb{Y},\z')f_2(\z,\z')\\
A_{\z X'} := &\sum_{\substack{\mathbb{X}_L (\x) \mathbb{X}_R = \mathbb{X} \\  \mathbb{Y}_L(\ide{y}) \mathbb{Y}_R = \mathbb{Y}}}
c(\x,\ide{y}) f_1(\z,\mathbb{X}_L * \mathbb{Y}_L, \x)f_2(\z,\x,\mathbb{X}_R * \mathbb{Y}_R,\z') \displaybreak[3] \\
A_{\z Y'} := &\sum_{\substack{\mathbb{X}_L (\ide{x}) \mathbb{X}_R = \mathbb{X} \\  \mathbb{Y}_L (\y) \mathbb{Y}_R = \mathbb{Y}}}
c(\y, \ide{x}) f_1(\z,\mathbb{X}_L * \mathbb{Y}_L, \y)f_2(\z,\y,\mathbb{X}_R * \mathbb{Y}_R,\z') \displaybreak[3] \\
A_{X\z'} :=&\sum_{\substack{\mathbb{X}_L(x)\mathbb{X}_R = \mathbb{X} \\ \mathbb{Y}_L(\ide{y})\mathbb{Y}_R = \mathbb{Y}}}c(\x,\ide{y}) f_1(\x,\mathbb{X}_R * \mathbb{Y}_R, \z')f_2(\z,\mathbb{X}_L * \mathbb{Y}_L, \x ,\z') \displaybreak[3] \\
A_{Y\z'} :=&\sum_{\substack{\mathbb{X}_L(\ide{x})\mathbb{X}_R = \mathbb{X} \\ \mathbb{Y}_L(\y)\mathbb{Y}_R = \mathbb{Y}}}c(\y,\ide{x}) f_1(\y,\mathbb{X}_R * \mathbb{Y}_R, \z')f_2(\z,\mathbb{X}_L * \mathbb{Y}_L, \y,\z') \displaybreak[3] \\
A_{XX'} :=& \sum_{\substack{\mathbb{X}_L (\x) \mathbb{X}_M(\x')\mathbb{X}_R = \mathbb{X} \\ \mathbb{Y}_L (\ide{y}) \mathbb{Y}_M(\ide{y}')\mathbb{Y}_R = \mathbb{Y} }} c(\x,\ide{y})c(\x',\ide{y}')f_1(\x,\mathbb{X}_M * \mathbb{Y}_M, \x')f_2(\z, \mathbb{X}_L * \mathbb{Y}_L , \x, \x', \mathbb{X}_R * \mathbb{Y}_R , \z') \displaybreak[3] \\
A_{XY'} :=& \sum_{\substack{\mathbb{X}_L (\x) \mathbb{X}_M(\ide{x}')\mathbb{X}_R = \mathbb{X} \\ \mathbb{Y}_L (\ide{y}) \mathbb{Y}_M(\y')\mathbb{Y}_R = \mathbb{Y} }} c(\x,\ide{y})c(\y',\ide{x}')f_1(\x,\mathbb{X}_M * \mathbb{Y}_M, \y')f_2(\z, \mathbb{X}_L * \mathbb{Y}_L , \x, \y', \mathbb{X}_R * \mathbb{Y}_R , \z') \displaybreak[3] \\
A_{YX'} :=& \sum_{\substack{\mathbb{X}_L (\ide{x}) \mathbb{X}_M(\x')\mathbb{X}_R = \mathbb{X} \\ \mathbb{Y}_L (\y) \mathbb{Y}_M(\ide{y}')\mathbb{Y}_R = \mathbb{Y} }} c(\y,\ide{x})c(\x',\ide{y}')f_1(\y,\mathbb{X}_M * \mathbb{Y}_M, \x')f_2(\z, \mathbb{X}_L * \mathbb{Y}_L , \y, \x', \mathbb{X}_R * \mathbb{Y}_R , \z') \displaybreak[3] \\
A_{YY'} :=& \sum_{\substack{\mathbb{X}_L (\ide{x}) \mathbb{X}_M(\ide{x}')\mathbb{X}_R = \mathbb{X} \\ \mathbb{Y}_L (\y) \mathbb{Y}_M(\y')\mathbb{Y}_R = \mathbb{Y} }} c(\y,\ide{x})c(\y',\ide{x}')f_1(\y,\mathbb{X}_M * \mathbb{Y}_M, \y')f_2(\z, \mathbb{X}_L * \mathbb{Y}_L , \y, \y', \mathbb{X}_R * \mathbb{Y}_R , \z').
\end{align*}




\end{lem}






\begin{proof}

By Proposition \ref{prop:deriv-explicit}, $\vimo_{d_{\Psi_1}(\Psi_2)}(\z,\mathbb{X} * \mathbb{Y},\z')$ is decomposed as
\begin{align*}
&\vimo_{d_{\Psi_1}(\Psi_2)}(\z,\mathbb{X} * \mathbb{Y},\z')\\
=& \sum_{\substack{\mathbb{V}_L (\v) \mathbb{V}_M (\v') \mathbb{V}_R \\ = (\z) (\mathbb{X} * \mathbb{Y})(\z')}}
f_1(\v,\mathbb{V}_M, \v')f_2(\delta_{\z}(\v), \mathbb{V}_L, (\v), (\v'), \mathbb{V}_R,\delta_{\z'}(\v')).
\end{align*}

We divide it into cases on the values of $\v$ and $\v'$. 
First, we can easily see
\[
A_{\z\z'} =  f_1(\z, \mathbb{X} * \mathbb{Y},\z')f_2(\z,\z').
\]
For the remaining cases, we shall show the case where $\v = \z$ and $\v'$ appears in $\mathbb{X}$ is equal to $A_{\z\X'}$ and the case where $\v = \z$ and $\v'$ appear in $\mathbb{X}$ is equal to $A_{\X\X'}$.
It suffices to demonstrate these two cases because of the discussion of the symmetrality. Namely,
\[
\xymatrix{
A_{\z\X'} \ar[d]_{\text{$\mathbb{X}\leftrightarrow \mathbb{Y}$}} \ar[r]^{\begin{matrix} \text{counterwise} \\ \text{discussion} \end{matrix}}   & \ar[d]^{\text{$\mathbb{X}\leftrightarrow \mathbb{Y}$}}A_{\X\z'} \ar[l] \\
A_{\z\Y'} \ar[u] \ar[r]_{\begin{matrix} \text{counterwise} \\ \text{discussion} \end{matrix}}  & A_{\Y\z'} \ar[u] \ar[l]
}\hspace{1.0cm}
\xymatrix{
A_{\X\X'} \ar[r]^{\text{$\mathbb{X}\leftrightarrow \mathbb{Y}$}} \ar[d]_{\footnotesize{\begin{matrix}\mathbb{X}\leftrightarrow \mathbb{Y}\\ \text{for $\x'$ and $\y'$}\end{matrix}}} & A_{\Y\Y'} \ar[l] \ar[d]^{\footnotesize{\begin{matrix}\mathbb{X}\leftrightarrow \mathbb{Y}\\ \text{for $\x'$ and $\y'$}\end{matrix}}}   \\
A_{\X\Y'} \ar[u] \ar[r]_{\begin{matrix} \text{counterwise} \\ \text{discussion}\end{matrix}}& A_{\Y\X'} \ar[u] \ar[l].  
}
\]


\begin{itemize}
\item Consider the case where $\v = \z$ and $\v'$ appears in $\mathbb{X}$. Let $\mathbb{X} = \mathbb{X}_L(\x')\mathbb{X}_R$. Lemma \ref{lem:har-inductive} and Proposition \ref{prop:deriv-explicit} imply
\begin{align*}
&\vimo_{d_{\Psi_1}(\Psi_2)}(\z, \mathbb{X} * \mathbb{Y},\z') \displaybreak[3]  \\
\stackrel{\eqref{eq:harmonic-composition}}{=}
&\sum_{\mathbb{Y}_L(\ide{y}) \mathbb{Y}_R = \mathbb{Y}}c(\x',\ide{y})\vimo_{d_{\Psi_1}(\Psi_2)}((\z)(\mathbb{X}_L * \mathbb{Y}_L)((\x') - \ide{y}) (\mathbb{X}_R * \mathbb{Y}_R)(\z')) \displaybreak[3]  \\
\stackrel{\eqref{eq:deriv-explicit}}{=}& \sum_{\mathbb{Y}_L(\ide{y}) \mathbb{Y}_R = \mathbb{Y}} c(\x',\ide{y})
\sum_{\substack{\mathbb{V}_L (\v) \mathbb{V}_M (\v') \mathbb{V}_R \\ = (\z)(\mathbb{X}_L * \mathbb{Y}_L)((\x') - \ide{y}) (\mathbb{X}_R * \mathbb{Y}_R)(\z')  }} f_1(\v,\mathbb{V}_M, \v')f_2(\delta_{\z}(\v), \mathbb{V}_L, (\v), (\v'), \mathbb{V}_R,\delta_{\z'}(\v')) \displaybreak[3]  \\
=&\sum_{\mathbb{Y}_L(\ide{y}) \mathbb{Y}_R = \mathbb{Y}} c(\x',\ide{y}) f_1(\z,\mathbb{X}_L * \mathbb{Y}_L, \x')f_2(\z,\x',\mathbb{X}_R * \mathbb{Y}_R,\z')\\
&+ (\text{the cases where $(\v,\v')\neq (\z,\x')$}).
\end{align*}
Therefore, by summing over all the choices of $\x'$ in the decomposition $\mathbb{X} = \mathbb{X}_L(\x')\mathbb{X}_R$, we have
\begin{align*}
&\vimo_{d_{\Psi_1}(\Psi_2)}(\z, \mathbb{X} * \mathbb{Y},\z')\\
& = \sum_{\substack{\mathbb{X}_L (\x') \mathbb{Y}_R = \mathbb{X} \\  \mathbb{Y}_L(\ide{y}) \mathbb{Y}_R = \mathbb{Y}}}
c(\x',\ide{y}) f_1(\z,\mathbb{X}_L * \mathbb{Y}_L, \x')f_2(\z,\x',\mathbb{X}_R * \mathbb{Y}_R,\z')\\
& + (\text{the cases where $(\v,\v')\neq (\z,\x')$ for all $\x'$ which appear in $\mathbb{X}$}).
\end{align*}
The first term of the right-hand side above is nothing but $A_{\z\X'}$.

\item Consider the case where $\v$ and $\v'$ appear in $\mathbb{X}$. Let $\mathbb{X} = \mathbb{X}_L(\x)\mathbb{X}_M(\x')\mathbb{X}_R$.
Lemma \ref{lem:har-inductive} and Proposition \ref{prop:deriv-explicit} imply
{\footnotesize
\begin{align*}
&\vimo_{d_{\Psi_1}(\Psi_2)}(\z, \mathbb{X} * \mathbb{Y},\z') \\
\stackrel{\eqref{eq:harmonic-composition}}{=}&
\sum_{\mathbb{Y}_L(\ide{y}) \mathbb{Y}_R = \mathbb{Y}}c(\x,\ide{y})\vimo_{d_{\Psi_1}(\Psi_2)}((\z)(\mathbb{X}_L * \mathbb{Y}_L)((\x) - (\ide{y})) (\mathbb{X}_R * \mathbb{Y}_R)(\z')) \displaybreak[3]\\
\stackrel{\eqref{eq:harmonic-composition}}{=}&\sum_{\mathbb{Y}_L(\ide{y}) \mathbb{Y}_M(\ide{y}')\mathbb{Y}_R = \mathbb{Y}}c(\x,\ide{y})c(\x',\ide{y}')\vimo_{d_{\Psi_1}(\Psi_2)}((\z)(\mathbb{X}_L * \mathbb{Y}_L)((\x) - (\ide{y})) (\mathbb{X}_M * \mathbb{Y}_M)((\x') - (\ide{y}')) (\mathbb{X}_R * \mathbb{Y}_R)(\z')) \displaybreak[3] \\
\stackrel{\eqref{eq:deriv-explicit}}{=}& \sum_{\mathbb{Y}_L(\ide{y}) \mathbb{Y}_M(\ide{y}')\mathbb{Y}_R = \mathbb{Y}}c(\x,\ide{y})c(\x',\ide{y}') \\
&\qquad \times \sum_{\substack{\mathbb{V}_L(\v) \mathbb{V}_M \mathbb{V}_R \\ = (\z)(\mathbb{X}_L * \mathbb{Y}_L)((\x) - (\ide{y})) (\mathbb{X}_M * \mathbb{Y}_M)((\x') - (\ide{y}')) (\mathbb{X}_R * \mathbb{Y}_R)(\z')}} f_1(\v,\mathbb{V}_M, \v')f_2(\delta_{\z}(\v), \mathbb{V}_L, (\v), (\v'), \mathbb{V}_R,\delta_{\z'}(\v'))   \displaybreak[3]\\
=& \sum_{\mathbb{Y}_L(\ide{y}) \mathbb{Y}_M(\ide{y}')\mathbb{Y}_R = \mathbb{Y}}c(\x,\ide{y})c(\x',\ide{y}') f_1(\x,\mathbb{X}_M * \mathbb{Y}_M, \x')f_2(\z, \mathbb{X}_L * \mathbb{Y}_L , \x, \x', \mathbb{X}_R * \mathbb{Y}_R , \z')\\
&+  (\text{the cases where $(\v,\v') \neq (\x,\x')$}).
\end{align*}
}
Therefore, by summing over all the choices of $\x$ and $\x'$ in the decomposition $\mathbb{X} = \mathbb{X}_L(\x)\mathbb{X}_M(\x')\mathbb{X}_R$, we have
\begin{align*}
&\vimo_{d_{\Psi_1}(\Psi_2)}(\z, \mathbb{X} * \mathbb{Y},\z')\\
=& \sum_{\substack{\mathbb{X}_L (\x) \mathbb{X}_M(\x')\mathbb{X}_R = \mathbb{X} \\ \mathbb{Y}_L (\ide{y}) \mathbb{Y}_M(\ide{y}')\mathbb{Y}_R = \mathbb{Y} }} c(\x,\ide{y})c(\x',\ide{y}')f_1(\x,\mathbb{X}_M * \mathbb{Y}_M, \x')f_2(\z, \mathbb{X}_L * \mathbb{Y}_L , \x, \x', \mathbb{X}_R * \mathbb{Y}_R , \z') \\
&+ (\text{the cases where $(\v,\v')\neq (\x,\x')$ for all $\x$ and $\x'$ which appear in $\mathbb{X}$}).
\end{align*}
The first term of the right-hand side above is nothing but $A_{\X\X'}$.
\end{itemize}
\end{proof}

\begin{comment}
\textcolor{red}{
These $A$'s are explicitly obtained as follows.
If $(\V,\V') = (\z,\z')$, then we have
\begin{align*}
A_{\z\z'} =& f_1(\z, \mathbb{X} * \mathbb{Y},\z')f_2(\z,\z').
\end{align*}
If $\V = \z$ and $\V'\in \{\X',\Y'\}$, we have
\begin{align*}
A_{\z\V'} =& \sum_{\substack{\mathbb{X}_L \beta_x(\V') \mathbb{X}_R = \mathbb{X} \\  \mathbb{Y}_L\beta_y(\V') \mathbb{Y}_R = \mathbb{Y}}} c(\v'(V'), \gamma_2(V')) f_1(\z,\mathbb{X}_L * \mathbb{Y}_L, \v'(\V'))f_2(\z,\v'(\V'),\mathbb{X}_R * \mathbb{Y}_R,\z').
\end{align*}
If $\V \in \{\X,\Y\}$ and $\V' = \z'$, we have
\begin{align*}
A_{\V\z'} =& \sum_{\substack{\mathbb{X}_L \alpha_x(\V) \mathbb{X}_R = \mathbb{X} \\  \mathbb{Y}_L\alpha_y(\V) \mathbb{Y}_R = \mathbb{Y}}}c(\v(V), \gamma_1(V)) f_1(\v(\V),\mathbb{X}_R * \mathbb{Y}_R, \z')f_2(\z,\mathbb{X}_L * \mathbb{Y}_L,\v(\V),\z').
\end{align*}
If $\V \in \{\X,\Y\}$ and $\V' \in \{\X',\Y'\}$, we have,
{\footnotesize
\begin{align*}
\begin{aligned}
A_{\V\V'} =& \sum_{\substack{\mathbb{X}_L \alpha_x(\V) \mathbb{X}_M \beta_x(\V')\mathbb{X}_R = \mathbb{X} \\  \mathbb{Y}_L\alpha_y(\V) \mathbb{Y}_M (\y')\mathbb{Y}_R = \mathbb{Y}}}c(\v(V), \gamma_1(V))c(\v'(V'), \gamma_2(V')) \\
&\qquad\times f_1(\v(\V),\mathbb{X}_M * \mathbb{Y}_M, \v'(\V'))f_2(\z,\mathbb{X}_L * \mathbb{Y}_L,\v(\V),\v'(\V'),\mathbb{X}_R * \mathbb{Y}_R,\z').
\end{aligned}
\end{align*}
}
Here, we define the following notations for variables:
\[
\begin{array}{ll}
\alpha_x(\V):=
\begin{cases}
\x & \V = \X\\
\ide{x} & \V = \Y,
\end{cases} &
\alpha_y(\V):=
\begin{cases}
\ide{y} & \V = \X\\
\y & \V = \Y,
\end{cases} \\
\beta_x(\V'):=
\begin{cases}
\x' & \V' = \X'\\
\ide{x}' & \V' = \Y',
\end{cases} &
\beta_y(\V'):=
\begin{cases}
\ide{y}' & \V' = \X'\\
\y' & \V' = \Y',
\end{cases} \\
\v(\V) :=
\begin{cases}
\alpha_x(\V) & \V = \X \\
\alpha_y(\V) & \V = \Y, \\
\end{cases}&
\v'(\V') :=
\begin{cases}
\beta_x(\V') & \V' = \X' \\
\beta_y(\V') & \V' = \Y', \\
\end{cases}\\
\gamma_1(\V) :=
\begin{cases}
\alpha_y(V) & \V = \X \\
\alpha_x(V) & \V = \Y, \\
\end{cases}\text{ and } &
\gamma_2(\V') :=
\begin{cases}
\beta_y(V) & \V' = \X' \\
\beta_x(V) & \V' = \Y'. \\
\end{cases}\\
\end{array}
\]
}


Let us define
\begin{align*}
\ConvL(f,g;\mathbb{X},\mathbb{Y}):=& \sum_{\mathbb{X}_{L}\mathbb{X}_{R} = \mathbb{X}} F_{\z0}(\mathbb{X}_{L})G_{\z\z'}(\mathbb{X}_{R} * \mathbb{Y}) +  \sum_{\mathbb{Y}_{L}\mathbb{Y}_{R} = \mathbb{Y}} F_{\z0}(\mathbb{Y}_{L})G_{\z\z'}(\mathbb{X} * \mathbb{Y}_{R}) \displaybreak[3] \\
\ConvR(f,g;\mathbb{X},\mathbb{Y}):=& \sum_{\mathbb{X}_{L}\mathbb{X}_{R} = \mathbb{X}} F_{0\z'}(\mathbb{X}_{R})G_{\z\z'}(\mathbb{X}_{L} * \mathbb{Y})  +  \sum_{\mathbb{Y}_{L}\mathbb{Y}_{R} = \mathbb{Y}} F_{0\z'}(\mathbb{Y}_{R})G_{\z\z'}(\mathbb{X} * \mathbb{Y}_{L}) \displaybreak[3]\\
\ConvLM^{(x)}(\V;f,g;\mathbb{X},\mathbb{Y}):=& \sum_{\substack{\mathbb{X}_L\alpha_x(\V)\mathbb{X}_M\mathbb{X}_R = \mathbb{X} \\ \mathbb{Y}_L\alpha_y(\V)\mathbb{Y}_R = \mathbb{Y} }} c(\v(V), \gamma_1(\V))F_{\v(\V)0}(\mathbb{X}_M)G_{\z\z'} (\mathbb{X}_L * \mathbb{Y}_L, \v(\V), \mathbb{X}_R * \mathbb{Y}_R) \displaybreak[3] \\
\ConvLM^{(y)}(\V;f,g;\mathbb{X},\mathbb{Y}):=& \sum_{\substack{\mathbb{X}_L\alpha_x(V)\mathbb{X}_R = \mathbb{X} \\ \mathbb{Y}_L\alpha_y(V)\mathbb{Y}_M\mathbb{Y}_R = \mathbb{Y} }} c(\v(V), \gamma_1(V))F_{\v(\V)0}(\mathbb{Y}_M)G_{\z\z'}( \mathbb{X}_L * \mathbb{Y}_L, \v(V), \mathbb{X}_R * \mathbb{Y}_R)\\
\ConvMR^{(x)}(\V';f,g;\mathbb{X},\mathbb{Y}):=& \sum_{\substack{\mathbb{X}_L\mathbb{X}_M\beta_x(\V')\mathbb{X}_R = \mathbb{X} \\ \mathbb{Y}_L\beta_y(\V')\mathbb{Y}_R = \mathbb{Y} }} c(\v'(V'), \gamma_2(\V')) F_{0\v'(\V')}( \mathbb{X}_M)G_{\z\z'} (\mathbb{X}_L * \mathbb{Y}_L, \v'(\V'), \mathbb{X}_R * \mathbb{Y}_R) \displaybreak[3] \\
\ConvMR^{(y)}(\V';f,g;\mathbb{X},\mathbb{Y}):=& \sum_{\substack{\mathbb{X}_L\beta_x(V)\mathbb{X}_R = \mathbb{X} \\ \mathbb{Y}_L\mathbb{Y}_M\beta_y(\V')\mathbb{Y}_R = \mathbb{Y} }} c(\v'(\V'), \gamma_2(\V')) F_{0\v'(\V')}( \mathbb{Y}_M)G_{\z\z'} (\mathbb{X}_L * \mathbb{Y}_L, \v'(\V'), \mathbb{X}_R * \mathbb{Y}_R).
\end{align*}
Here, $f$, $g$ lie in $\mathcal{K} \otimes \mathcal{G}$, $\V \in \{\X,\Y\}$, and $\V'\in \{\X',\Y'\}$. We define $F_{\v\v'}$ (respectively, $G_{\v\v'}$) $ : A_{\mathcal{U}}^{\mathrm{rat}} \rightarrow \mathcal{K} \otimes \mathcal{G}$ as $F_{\v\v'}(U) = f(\v,U,\v')$ (respectively, $G_{\z\z'}(U) = g(\z,U,\z')$)




\textcolor{red}{
Let us define 
\begin{align*} \ConvL(F,G;\mathbb{X},\mathbb{Y}):=& \sum_{\mathbb{X}_{L}\mathbb{X}_{R} = \mathbb{X}} F(\mathbb{X}_{L})G(\mathbb{X}_{R} * \mathbb{Y}) + \sum_{\mathbb{Y}_{L}\mathbb{Y}_{R} = \mathbb{Y}} F(\mathbb{Y}_{L})G(\mathbb{X} * \mathbb{Y}_{R}) \displaybreak[3] \\
 \ConvR(F,G;\mathbb{X},\mathbb{Y}):=& \sum_{\mathbb{X}_{L}\mathbb{X}_{R} = \mathbb{X}} F(\mathbb{X}_{R})G(\mathbb{X}_{L} * \mathbb{Y}) + \sum_{\mathbb{Y}_{L}\mathbb{Y}_{R} = \mathbb{Y}} F(\mathbb{Y}_{R})G(\mathbb{X} * \mathbb{Y}_{L}) \displaybreak[3]\\
 \ConvLM^{(x)}(\V;F,G;\mathbb{X},\mathbb{Y}):=& \sum_{\substack{\mathbb{X}_L\alpha_x(\V)\mathbb{X}_M\mathbb{X}_R = \mathbb{X} \\ \mathbb{Y}_L\alpha_y(\V)\mathbb{Y}_R = \mathbb{Y} }} c(\v(V), \gamma_1(\V))F(\mathbb{X}_M)G (\mathbb{X}_L * \mathbb{Y}_L, \v(\V), \mathbb{X}_R * \mathbb{Y}_R) \displaybreak[3] \\ 
\ConvLM^{(y)}(\V;F,G;\mathbb{X},\mathbb{Y}):=& \sum_{\substack{\mathbb{X}_L\alpha_x(V)\mathbb{X}_R = \mathbb{X} \\ \mathbb{Y}_L\alpha_y(V)\mathbb{Y}_M\mathbb{Y}_R = \mathbb{Y} }} c(\v(V), \gamma_1(V))F(\mathbb{Y}_M)G( \mathbb{X}_L * \mathbb{Y}_L, \v(V), \mathbb{X}_R * \mathbb{Y}_R)\\
 \ConvMR^{(x)}(\V';F,G;\mathbb{X},\mathbb{Y}):=& \sum_{\substack{\mathbb{X}_L\mathbb{X}_M\beta_x(\V')\mathbb{X}_R = \mathbb{X} \\ \mathbb{Y}_L\beta_y(\V')\mathbb{Y}_R = \mathbb{Y} }} c(\v'(V'), \gamma_2(\V')) F( \mathbb{X}_M)G (\mathbb{X}_L * \mathbb{Y}_L, \v'(\V'), \mathbb{X}_R * \mathbb{Y}_R) \displaybreak[3] \\
 \ConvMR^{(y)}(\V';F,G;\mathbb{X},\mathbb{Y}):=& \sum_{\substack{\mathbb{X}_L\beta_x(V')\mathbb{X}_R = \mathbb{X} \\ \mathbb{Y}_L\mathbb{Y}_M\beta_y(\V')\mathbb{Y}_R = \mathbb{Y} }} c(\v'(\V'), \gamma_2(\V')) F( \mathbb{Y}_M)G (\mathbb{X}_L * \mathbb{Y}_L, \v'(\V'), \mathbb{X}_R * \mathbb{Y}_R). 
\end{align*} 
Here, $F$, $G$ is functions from $A_{\mathcal{U}}^{\mathrm{rat}}$ to $\mathcal{K} \otimes \mathcal{G}$, $\V \in \{\X,\Y\}$, and $\V'\in \{\X',\Y'\}$. 
}


\end{comment}



\begin{lem}\label{lem:vimo-B}
If $\Psi_1 \in \addmr\cap \Fad$ and $\Psi_2 \in \ide{h}^{\vee}$, we have
\begin{align*}
&\vimo_{d_{\Psi_1(\Psi_2)}}(\z, \mathbb{X} * \mathbb{Y}, \z')\\
=&B_{\z0} + B_{0\z'}  - B_{00}^z + B_{\x0} + B_{0\x} + B_{\y0} + B_{0\y} - B_{00}.
\end{align*}
where, 
\begin{align*}
B_{\z0} =&\sum_{\mathbb{X}_L\mathbb{X}_R = \mathbb{X}} f_1(\z,\mathbb{X}_L,0)f_2(\z,\mathbb{X}_R * \mathbb{Y},\z')
+\sum_{\mathbb{Y}_L\mathbb{Y}_R = \mathbb{Y}} f_1(\z,\mathbb{Y}_L,0)f_2(\z,\mathbb{X} * \mathbb{Y}_R,\z') \displaybreak[3] \\
B_{0\z'} =&\sum_{\mathbb{X}_L\mathbb{X}_R = \mathbb{X}} f_1(0,\mathbb{X}_R,\z')f_2(\z,\mathbb{X}_L * \mathbb{Y},\z')
+\sum_{\mathbb{Y}_L\mathbb{Y}_R = \mathbb{Y}} f_1(0,\mathbb{Y}_R,\z')f_2(\z,\mathbb{X} * \mathbb{Y}_L,\z') \displaybreak[3] \\
B_{00}^{\z} =&\sum_{\mathbb{X}_L\mathbb{X}_R = \mathbb{X}} f_1(0,\mathbb{X}_L,0)f_2(\z,\mathbb{X}_R * \mathbb{Y},\z')
+\sum_{\mathbb{Y}_L\mathbb{Y}_R = \mathbb{Y}} f_1(0,\mathbb{Y}_L,0)f_2(\z,\mathbb{X} * \mathbb{Y}_R,\z') \displaybreak[3] \\
B_{\X0} =& \sum_{\substack{\mathbb{X}_L(\x)\mathbb{X}_M\mathbb{X}_R = \mathbb{X} \\ \mathbb{Y}_L(\ide{y}) \mathbb{Y}_R = \mathbb{Y}}} c(\x , \ide{y}) f_1(\x,\mathbb{X}_M, 0)f_2(\z,\mathbb{X}_L * \mathbb{Y}_L , \x, \mathbb{X}_R * \mathbb{Y}_R, \z') \\
&\hspace{0.2cm} + \sum_{\substack{\mathbb{X}_L(\x)\mathbb{X}_R = \mathbb{X} \\ \mathbb{Y}_L(\ide{y}) \mathbb{Y}_M \mathbb{Y}_R = \mathbb{Y}}} c(\x , \ide{y}) f_1(\x,\mathbb{Y}_M, 0)f_2(\z,\mathbb{X}_L * \mathbb{Y}_L , \x, \mathbb{X}_R * \mathbb{Y}_R, \z') \displaybreak[3] \\ 
B_{0\X'} =& \sum_{\substack{\mathbb{X}_L\mathbb{X}_M(\x)\mathbb{X}_R = \mathbb{X} \\ \mathbb{Y}_L(\ide{y}) \mathbb{Y}_R = \mathbb{Y}}} c(\x , \ide{y}) f_1(0,\mathbb{X}_M, \x)f_2(\z,\mathbb{X}_L * \mathbb{Y}_L , \x, \mathbb{X}_R * \mathbb{Y}_R, \z')  \\
&\hspace{0.2cm} + \sum_{\substack{\mathbb{X}_L(\x)\mathbb{X}_R = \mathbb{X} \\ \mathbb{Y}_L \mathbb{Y}_M (\ide{y})\mathbb{Y}_R = \mathbb{Y}}} c(\x , \ide{y}) f_1(\x,\mathbb{Y}_M, 0)f_2(\z,\mathbb{X}_L * \mathbb{Y}_L , \x, \mathbb{X}_R * \mathbb{Y}_R, \z') \displaybreak[3] \\
B_{\Y0} =&  \sum_{\substack{\mathbb{X}_L(\ide{x})\mathbb{X}_M\mathbb{X}_R = \mathbb{X} \\ \mathbb{Y}_L(\y) \mathbb{Y}_R = \mathbb{Y}}} c(\y , \ide{x}) f_1(\y,\mathbb{X}_M, 0)f_2(\z,\mathbb{X}_L * \mathbb{Y}_L , \y, \mathbb{X}_R * \mathbb{Y}_R, \z') \\
&\hspace{0.2cm} + \sum_{\substack{\mathbb{X}_L(\ide{x})\mathbb{X}_R = \mathbb{X} \\ \mathbb{Y}_L(\y) \mathbb{Y}_M \mathbb{Y}_R = \mathbb{Y}}} c(\y,\ide{x}) f_1(\y,\mathbb{Y}_M, 0)f_2(\z,\mathbb{X}_L * \mathbb{Y}_L , \y, \mathbb{X}_R * \mathbb{Y}_R, \z') \displaybreak[3] \\
B_{0\Y'} =& \sum_{\substack{\mathbb{X}_L\mathbb{X}_M(\ide{x})\mathbb{X}_R = \mathbb{X} \\ \mathbb{Y}_L(\y) \mathbb{Y}_R = \mathbb{Y}}} c(\y , \ide{x}) f_1(0,\mathbb{X}_M, \y) f_2(\z,\mathbb{X}_L * \mathbb{Y}_L , \y, \mathbb{X}_R * \mathbb{Y}_R, \z') \\
&\hspace{0.2cm} + \sum_{\substack{\mathbb{X}_L(\ide{x})\mathbb{X}_R = \mathbb{X} \\ \mathbb{Y}_L \mathbb{Y}_M (\y)\mathbb{Y}_R = \mathbb{Y}}} c(\y , \ide{x}) f_1(\x,\mathbb{Y}_M, 0)f_2(\z,\mathbb{X}_L * \mathbb{Y}_L , \y, \mathbb{X}_R * \mathbb{Y}_R, \z') \displaybreak[3] \\
B_{00} =&\sum_{\substack{\mathbb{X}_L(\x)\mathbb{X}_M\mathbb{X}_R = \mathbb{X} \\ \mathbb{Y}_L(\ide{y}) \mathbb{Y}_R = \mathbb{Y}}} c(\x , \ide{y}) f_1(0,\mathbb{X}_M, 0)f_2(\z,\mathbb{X}_L * \mathbb{Y}_L , \x, \mathbb{X}_R * \mathbb{Y}_R, \z')  \\
&\hspace{0.2cm} + \sum_{\substack{\mathbb{X}_L(\x)\mathbb{X}_R = \mathbb{X} \\ \mathbb{Y}_L(\ide{y}) \mathbb{Y}_M \mathbb{Y}_R = \mathbb{Y}}} c(\x , \ide{y}) f_1(0,\mathbb{Y}_M, 0)f_2(\z,\mathbb{X}_L * \mathbb{Y}_L , \x, \mathbb{X}_R * \mathbb{Y}_R, \z') \\ 
&+\sum_{\substack{\mathbb{X}_L(\ide{x})\mathbb{X}_M\mathbb{X}_R = \mathbb{X} \\ \mathbb{Y}_L(\y) \mathbb{Y}_R = \mathbb{Y}}} c(\y , \ide{x}) f_1(0,\mathbb{X}_M, 0)f_2(\z,\mathbb{X}_L * \mathbb{Y}_L , \y, \mathbb{X}_R * \mathbb{Y}_R, \z') \\
&\hspace{0.2cm} + \sum_{\substack{\mathbb{X}_L(\ide{x})\mathbb{X}_R = \mathbb{X} \\ \mathbb{Y}_L(\y) \mathbb{Y}_M \mathbb{Y}_R = \mathbb{Y}}} c(\y,\ide{x}) f_1(0,\mathbb{Y}_M, 0)f_2(\z,\mathbb{X}_L * \mathbb{Y}_L , \y, \mathbb{X}_R * \mathbb{Y}_R, \z').
\end{align*}




\begin{comment}
\begin{align*} B_{\z0} =& \ConvL(f_1(\z,-,0), f_2(\z,-,\z'); \mathbb{X},\mathbb{Y})\displaybreak[3]\\ 
B_{0\z'} =& \ConvR(f_1(0,-,\z'), f_2(\z,-,\z'); \mathbb{X},\mathbb{Y})\displaybreak[3]\\
 B_{00}^{\z} =& \ConvL(f_1(0,-,0), f_2(\z,-,\z'); \mathbb{X},\mathbb{Y})\displaybreak[3]\\
 B_{\V0}:= &\ConvLM^{(x)}(\V;f_1(\v(\V),-,0), f_2(\z,-,\z');\mathbb{X},\mathbb{Y}) + \ConvLM^{(y)}(\V;f_1(\v(\V),-,0),  f_2(\z,-,\z');\mathbb{X},\mathbb{Y}) \displaybreak[3] \\ 
B_{0\V'}:= &\ConvMR^{(x)}(\V';f_1(0,-,\v'(\V')),  f_2(\z,-,\z');\mathbb{X},\mathbb{Y}) + \ConvMR^{(y)}(\V';f_1(0,-,\v'(\V')),  f_2(\z,-,\z');\mathbb{X},\mathbb{Y})\\
B_{00}:=& \sum_{\V \in \{\X,\Y\}} \ConvLM^{(x)}(\V;f_1(0,-,0),  f_2(\z,-,\z');\mathbb{X},\mathbb{Y}) + \ConvLM^{(y)}(\V;f_1(0,-,0),  f_2(\z,-,\z');\mathbb{X},\mathbb{Y}). 
\end{align*}
In other words, we have
\begin{align*}
&\vimo_{d_{\Psi_1}(\Psi_2)}(\z, \mathbb{X} * \mathbb{Y},\z') \\
=& \ConvL(f_1(\z,-,0), f_2(\z,-,\z'); \mathbb{X},\mathbb{Y}) + \ConvR(f_1(0,-,\z'), f_2(\z,-,\z'); \mathbb{X},\mathbb{Y}) - \ConvL(f_1(0,-,0), f_2(\z,-,\z'); \mathbb{X},\mathbb{Y})\\
&+\sum_{\V \in \{\X,\Y\}} \left\{ \ConvLM^{(x)}(\V;f_1(\v(\V),-,0), f_2(\z,-,\z');\mathbb{X},\mathbb{Y}) + \ConvLM^{(y)}(\V;f_1(\v(\V),-,0),  f_2(\z,-,\z');\mathbb{X},\mathbb{Y}) \right\}\\
&+\sum_{\V' \in \{\X',\Y'\}} \left\{  \ConvMR^{(x)}(\V';f_1(0,-,\v'(\V')),  f_2(\z,-,\z');\mathbb{X},\mathbb{Y}) + \ConvMR^{(y)}(\V';f_1(0,-,\v'(\V')),  f_2(\z,-,\z');\mathbb{X},\mathbb{Y}) \right\}\\
&-\sum_{\V \in \{\X,\Y\}}  \left\{\ConvLM^{(x)}(\V;f_1(0,-,0),  f_2(\z,-,\z');\mathbb{X},\mathbb{Y}) + \ConvLM^{(y)}(\V;f_1(0,-,0),  f_2(\z,-,\z');\mathbb{X},\mathbb{Y}) \right\}.
\end{align*}

\begin{align*}
B_{\z0} =& \ConvL(f_1, f_2; \mathbb{X},\mathbb{Y})\displaybreak[3]\\
B_{0\z'} =& \ConvR(f_1, f_2; \mathbb{X},\mathbb{Y})\displaybreak[3]\\
B_{00}^{\z} =& \ConvL(f_1, f_2; \mathbb{X},\mathbb{Y})\displaybreak[3]\\
B_{\V0}:= &\ConvLM^{(x)}(\V;f_1, f_2;\mathbb{X},\mathbb{Y}) + \ConvLM^{(y)}(\V;f_1, f_2;\mathbb{X},\mathbb{Y}) \displaybreak[3] \\
B_{0\V'}:= &\ConvMR^{(x)}(\V';f_1, f_2;\mathbb{X},\mathbb{Y}) + \ConvMR^{(y)}(\V';f_1, f_2;\mathbb{X},\mathbb{Y})\\
B_{00}:=& \sum_{\V \in \{\X,\Y\}} \ConvLM^{(x)}(\V;f_1, f_2;\mathbb{X},\mathbb{Y}) + \ConvLM^{(y)}(\V;f_1, f_2;\mathbb{X},\mathbb{Y}).
\end{align*}

\end{comment}
\end{lem}

\begin{proof}
Since $\Psi_1 \in \addmr$, $A_{\z\z'} =  f_1(\z, \mathbb{X} * \mathbb{Y},\z')f_2(\z,\z') \stackrel{\eqref{eq:adjoint-harmonic}}{=} 0$.
Let us consider $A_{\z\X'}$.
By Proposition \ref{prop:adjoint-mould}, we have

{\footnotesize
\begin{align*}
A_{\z\X'} =& \sum_{\substack{\mathbb{X}_L (\x) \mathbb{X}_R = \mathbb{X} \\  \mathbb{Y}_L\ide{y} \mathbb{Y}_R = \mathbb{Y}}}
c(\x,\ide{y}) f_1(\z,\mathbb{X}_L * \mathbb{X}_R, \x)f_2(\z,\x,\mathbb{X}_R * \mathbb{Y}_R,\z')\\
\stackrel{\eqref{eq:adjoint-mould}}{=}& \sum_{\substack{\mathbb{X}_L (\x) \mathbb{X}_R = \mathbb{X} \\  \mathbb{Y}_L\ide{y} \mathbb{Y}_R = \mathbb{Y}}}
c(\x,\ide{y})f_2(\z,\x,\mathbb{X}_R * \mathbb{Y}_R,\z')\{
 f_1(\z,\mathbb{X}_L * \mathbb{X}_R, 0) +  f_1(0,\mathbb{X}_L * \mathbb{X}_R, \x) -  f_1(0,\mathbb{X}_L * \mathbb{X}_R, 0)  \}.
\end{align*}
}
By $A_{\z\X'\rightarrow \z0}$ (respectively, $A_{\z\X'\rightarrow 0\X'}$, $A_{\z\X'\rightarrow 00}$ ), we denote the first (respectively, second, $(-1) \times $ third) term of the right-hand side above. This means transforming either end component (or both ends) of $f_1$ to $0$.

In a similar manner,  same convention applies to the cases $(\V,\V') \in S$, i.e., put
\[
A_{\V\V'} \;:=\; A_{\V\V'\to \V 0}\;+\;A_{\V\V'\to 0\V'}\;-\;A_{\V\V'\to 00}.
\]



\begin{comment}
Let us consider $A_{\z\V'}$ ($\V' \in \{\X',\Y'\}$).
By Proposition \ref{prop:adjoint-mould}, we have
{\footnotesize
\begin{align*}
A_{\z\V'} =& \sum_{\substack{\mathbb{X}_L \beta_x(\V') \mathbb{X}_R = \mathbb{X} \\  \mathbb{Y}_L\beta_y(\V') \mathbb{Y}_R = \mathbb{Y}}} c(\v'(V'), \gamma_2(V')) f_1(\z,\mathbb{X}_L * \mathbb{Y}_L, \v'(\V'))f_2(\z,\v'(\V'),\mathbb{X}_R * \mathbb{Y}_R,\z') \displaybreak[3] \\
\stackrel{\eqref{eq:adjoint-mould}}{=}& \sum_{\substack{\mathbb{X}_L \beta_x(\V') \mathbb{X}_R = \mathbb{X} \\  \mathbb{Y}_L\beta_y(\V') \mathbb{Y}_R = \mathbb{Y}}} c(\v'(V'), \gamma_2(V')) f_1(\z,\mathbb{X}_L * \mathbb{Y}_L, 0) f_2(\z,\v'(\V'),\mathbb{X}_R * \mathbb{Y}_R,\z')\\
&+\sum_{\substack{\mathbb{X}_L \beta_x(\V') \mathbb{X}_R = \mathbb{X} \\  \mathbb{Y}_L\beta_y(\V') \mathbb{Y}_R = \mathbb{Y}}} c(\v'(V'), \gamma_2(V'))   f_1(0,\mathbb{X}_L * \mathbb{Y}_L, \v'(\V')) f_2(\z,\v'(\V'),\mathbb{X}_R * \mathbb{Y}_R,\z') \\
&-\sum_{\substack{\mathbb{X}_L \beta_x(\V') \mathbb{X}_R = \mathbb{X} \\  \mathbb{Y}_L\beta_y(\V') \mathbb{Y}_R = \mathbb{Y}}} c(\v'(V'), \gamma_2(V'))   f_1(0,\mathbb{X}_L * \mathbb{Y}_L, 0) f_2(\z,\v'(\V'),\mathbb{X}_R * \mathbb{Y}_R,\z')
\end{align*}
}
By $A_{\z\V'\rightarrow \z0}$ (respectively, $A_{\z\V'\rightarrow 0\V'}$, $A_{\z\V'\rightarrow \z0}$ ), we denote the first (respectively, second, $(-1) \times $ third) term of the right-hand side above. This means transforming either end component (or both end) of $f_1$ to $0$.
\end{comment}




It suffices to prove the following:
\begin{align*}
B_{\V0} = \sum_{\V' \in \{\z',\X',\Y'\}}A_{\V\V'\rightarrow \V0}
\end{align*}
 for $\V \in \{\z, \X,\Y\}$,
\begin{align*}
B_{0\V'}=\sum_{\V \in \{\z,\X,\Y\}}A_{\V\V'\rightarrow 0\V'}
\end{align*}
 for $\V' \in \{\z', \X',\Y'\}$,
\begin{align*}
B_{00}^{\z} =  \sum_{\V' \in \{\z',\X',\Y'\}}A_{\z\V'\rightarrow \V0},
\end{align*}
and
\begin{align*}
B_{00} = \sum_{\V \in \{\z,\X,\Y\}}A_{\X\V'\rightarrow 0\V'} +  \sum_{\V' \in \{\z',\X',\Y'\}}A_{\Y\V'\rightarrow 0\V'}.
\end{align*}

Essentially, it is enough to show
\begin{align}
B_{\z0}=&A_{\z\x \rightarrow \z0} + A_{\z\y \rightarrow \z0}  \label{eq:Bz0} \\
B_{\V0}=&A_{\x\z' \rightarrow \x0} + A_{\x\x \rightarrow \x0} +  A_{\x\y \rightarrow \x0} \label{eq:Bx0} 
\end{align}
for $V \in \{\X,\Y\}$
because of the symmetry. Namely,
\[
\xymatrix{
B_{\X0} \ar[r]^{\mathbb{X}\leftrightarrow \mathbb{Y}} & B_{\Y0} \ar[l]
},
\xymatrix{
B_{\V0} \ar[r]^{\begin{matrix} \text{counterwise} \\ \text{discussion} \end{matrix}} & B_{0\V'} \ar[l]
}
,\text{ and}
\xymatrix{
B_{\z0} \ar[r]^{\begin{matrix} \text{counterwise} \\ \text{discussion} \end{matrix}} & B_{0\z'}  \ar[l]
}
\]
and
\[
b_0(B_{\x0} + B_{\y0}) =  B_{00}, b_0'(B_{\z0}) = B_{00}^z
\]
($b_0 (A_{? s\rightarrow ? 0}) = A_{? s \rightarrow 00}$, $b_0' (A_{s? \rightarrow 0?}) = A_{s ? \rightarrow 00}$).

First, let us prove \eqref{eq:Bz0}. 
By direct calculation, we have
\begin{align}
 &\sum_{\V' \in \{\z',\X',\Y'\}}A_{\z\V'\rightarrow \z0} \displaybreak[3] \notag\\
=&\sum_{\substack{\mathbb{X}_L (\x) \mathbb{X}_R = \mathbb{X} \\  \mathbb{Y}_L\ide{y} \mathbb{Y}_R = \mathbb{Y}}}
c(\x,\ide{y}) f_1(\z,\mathbb{X}_L * \mathbb{Y}_L, 0)f_2(\z,\x,\mathbb{X}_R * \mathbb{Y}_R,\z') \notag \\
&+\sum_{\substack{\mathbb{X}_L (\ide{x}) \mathbb{X}_R = \mathbb{X} \\  \mathbb{Y}_L (\y) \mathbb{Y}_R = \mathbb{Y}}}
c(\y, \ide{x}) f_1(\z,\mathbb{X}_L * \mathbb{Y}_L, 0)f_2(\z,\y,\mathbb{X}_R * \mathbb{Y}_R,\z') \notag \displaybreak[3] \\
=& \sum_{\substack{\mathbb{X}_L \mathbb{X}_R = \mathbb{X} \\ \mathbb{Y}_L \mathbb{Y}_R = \mathbb{Y}}}
f_1(\z,\mathbb{X}_L * \mathbb{Y}_L, 0) \notag \\
&\times\left\{ \sum_{\substack{(\x)\mathbb{X}_{RR} = \mathbb{X}_R \\ (\ide{y})\mathbb{Y}_{RR} = \mathbb{Y}_R}} c(\x,\ide{y})f_2(\z,\x,\mathbb{X}_{RR} * \mathbb{Y}_{RR},\z')
+ \sum_{\substack{(\ide{x})\mathbb{X}_{RR} = \mathbb{X}_R \\ (\y)\mathbb{Y}_{RR} = \mathbb{Y}_R}} 
c(\y, \ide{x}) f_2(\z,\y,\mathbb{X}_{RR} * \mathbb{Y}_{RR},\z') \right\} \notag \displaybreak[3] \\
=&\sum_{\substack{\mathbb{X}_L \mathbb{X}_R = \mathbb{X} \\ \mathbb{Y}_L \mathbb{Y}_R = \mathbb{Y}}}
f_1(\z,\mathbb{X}_L * \mathbb{Y}_L, 0) f_2(\z , \mathbb{X}_R * \mathbb{Y}_R , \z').  \displaybreak[3] \label{eq:Bz0-1}
\end{align}

\begin{comment}
\begin{align}
 &\sum_{\V' \in \{\z',\X',\Y'\}}A_{\z\V'\rightarrow \z0} \displaybreak[3] \notag\\
=&\sum_{\V' \in \{\X,\Y\}} \sum_{\substack{\mathbb{X}_L \beta_x(\V')\mathbb{X}_R = \mathbb{X} \\ \mathbb{Y}_L \beta_y(\V')\mathbb{Y}_R = \mathbb{Y} }} c(\v'(\V'), \gamma_2(\V')) f_1(\z , \mathbb{X}_L * \mathbb{Y}_L,0)f_2(\z, \v'(\V'), \mathbb{X}_R * \mathbb{Y}_R,\z')\displaybreak[3] \notag\\
=&\sum_{\substack{\mathbb{X}_L \mathbb{X}_R = \mathbb{X} \\ \mathbb{Y}_L \mathbb{Y}_R = \mathbb{Y}}}f_1(\z,\mathbb{X}_L * \mathbb{Y}_L,0)\left\{\sum_{\substack{\beta_x(\V')\mathbb{X}_{RR} = \mathbb{X}_R \\ \beta_y(\V')\mathbb{Y}_{RR} = \mathbb{Y}_R }} c(\v'(\V'), \gamma_2(\V'))f_2(\z,\v'(\V'), \mathbb{X}_{RR} * \mathbb{Y}_{RR},\z') \right\}. 
\end{align}

Since 
\begin{align*}
&\sum_{\substack{\beta_x(\V')\mathbb{X}_{RR} = \mathbb{X}_R \\ \beta_y(\V')\mathbb{Y}_{RR} = \mathbb{Y}_R }} 
= \sum_{\substack{(\x)\mathbb{X}_{RR} = \mathbb{X}_R \\ (\ide{y})\mathbb{Y}_{RR} = \mathbb{Y}_R}}
+\sum_{\substack{(\ide{x})\mathbb{X}_{RR} = \mathbb{X}_R \\ (\y)\mathbb{Y}_{RR} = \mathbb{Y}_R}}
\end{align*}
holds, it follows that
\begin{align}
&\sum_{\substack{\beta_x(\V')\mathbb{X}_{RR} = \mathbb{X}_R \\ \beta_y(\V')\mathbb{Y}_{RR} = \mathbb{Y}_R }} c(\v'(\V'), \gamma_2(\V'))f_2(\z,\v'(\V'), \mathbb{X}_{RR} * \mathbb{Y}_{RR},\z') \displaybreak[3] \notag \\
=&\sum_{\substack{(\x)\mathbb{X}_{RR} = \mathbb{X}_R \\ (\ide{y})\mathbb{Y}_{RR} = \mathbb{Y}_R}}
c(\x, \ide{y})f_2(\z,\x, \mathbb{X}_{RR} * \mathbb{Y}_{RR},\z') +\sum_{\substack{(\ide{x})\mathbb{X}_{RR} = \mathbb{X}_R \\ (\y)\mathbb{Y}_{RR} = \mathbb{Y}_R}}
c(\y, \ide{x})f_2(\z,\y, \mathbb{X}_{RR} * \mathbb{Y}_{RR},\z') \displaybreak[3] \notag \\
=&f_2(\z, ((\x) \mathbb{X}_{RR}) * ((\y)\mathbb{Y}_{RR}), \z') \displaybreak[3] \notag\\
=&f_2(\z,  \mathbb{X}_{R} * \mathbb{Y}_{R}, \z'). \label{eq:Bz0-2}
\end{align}
\end{comment}

Thus, due to $\Psi_1 \in \addmr$, we conclude

\begin{align*}
&\sum_{\V' \in \{\z',\X',\Y'\}}A_{\V\V'\rightarrow \z0} \displaybreak[3]\\
\stackrel{\eqref{eq:Bz0-1}}{=}& \sum_{\substack{\mathbb{X}_L \mathbb{X}_R = \mathbb{X} \\ \mathbb{Y}_L \mathbb{Y}_R = \mathbb{Y}}}f_1(\z,\mathbb{X}_L * \mathbb{Y}_L,0)f_2(\z,  \mathbb{X}_{R} * \mathbb{Y}_{R}, \z') \displaybreak[3]\\
\stackrel{\eqref{eq:adjoint-harmonic}}{=}&
\sum_{\substack{\mathbb{X}_L \mathbb{X}_R = \mathbb{X} }}f_1(\z,\mathbb{X}_L ,0)f_2(\z,  \mathbb{X}_{R} * \mathbb{Y}, \z')+\sum_{\substack{\mathbb{Y}_L \mathbb{Y}_R = \mathbb{Y} }}f_1(\z,\mathbb{Y}_L ,0)f_2(\z,  \mathbb{X} * \mathbb{Y}_R, \z') \displaybreak[3] \\
=&B_{\z0}.
\end{align*}

Next, let us prove \eqref{eq:Bx0}. Direct calculations show
{\footnotesize
\begin{align}
& A_{\X\z' \rightarrow \X0} + A_{\X\X' \rightarrow \X0} +  A_{\X\Y' \rightarrow \X0} \notag \\
=&\sum_{\substack{\mathbb{X}_L(x)\mathbb{X}_R = \mathbb{X} \\ \mathbb{Y}_L(\ide{y})\mathbb{Y}_R = \mathbb{Y}}}c(\x,\ide{y}) f_1(\x,\mathbb{X}_R * \mathbb{Y}_R, 0)f_2(\z,\mathbb{X}_L * \mathbb{Y}_L, \x ,\z') \notag \\
&+ \sum_{\substack{\mathbb{X}_L (\x) \mathbb{X}_M(\x')\mathbb{X}_R = \mathbb{X} \\ \mathbb{Y}_L (\ide{y}) \mathbb{Y}_M(\ide{y}')\mathbb{Y}_R = \mathbb{Y} }} c(\x,\ide{y})c(\x',\ide{y}')f_1(\x,\mathbb{X}_M * \mathbb{Y}_M, 0)f_2(\z, \mathbb{X}_L * \mathbb{Y}_L , \x, \x', \mathbb{X}_R * \mathbb{Y}_R , \z') \notag \\
&+\sum_{\substack{\mathbb{X}_L (\x) \mathbb{X}_M(\ide{x}')\mathbb{X}_R = \mathbb{X} \\ \mathbb{Y}_L (\ide{y}) \mathbb{Y}_M(\y')\mathbb{Y}_R = \mathbb{Y} }} c(\x,\ide{y})c(\y',\ide{x}')f_1(\x,\mathbb{X}_M * \mathbb{Y}_M, 0)f_2(\z, \mathbb{X}_L * \mathbb{Y}_L , \x, \y', \mathbb{X}_R * \mathbb{Y}_R , \z') \displaybreak[3]  \notag \\
=&\sum_{\substack{\mathbb{X}_L(x)\mathbb{X}_R = \mathbb{X} \\ \mathbb{Y}_L(\ide{y})\mathbb{Y}_R = \mathbb{Y}}}c(\x,\ide{y}) f_1(\x,\mathbb{X}_R * \mathbb{Y}_R, 0)f_2(\z,\mathbb{X}_L * \mathbb{Y}_L, \x ,\z') \notag \\
&+\sum_{\substack{ \mathbb{X}_L (\x) \mathbb{X}_M \mathbb{X}_R = \mathbb{X} \\\mathbb{Y}_L (\ide{y}) \mathbb{Y}_M \mathbb{Y}_R = \mathbb{Y}  }} c(\x,\ide{y}) f_1(\x, \mathbb{X}_M  * \mathbb{Y}_M ,0) \notag \\
& \times \left\{  \sum_{\substack{(\x')\mathbb{X}_{RR} = \mathbb{X}_R \\ (\ide{y}')\mathbb{Y}_{RR} = \mathbb{Y}_R }} c(\x',\ide{y}')f_2(\z, \mathbb{X}_L * \mathbb{Y}_L , \x, \x', \mathbb{X}_{RR} * \mathbb{Y}_{RR} , \z') + \sum_{\substack{(\ide{x}')\mathbb{X}_{RR} = \mathbb{X}_R \\ (\y')\mathbb{Y}_{RR} = \mathbb{Y}_R }}  c(\y',\ide{x}')f_2(\z, \mathbb{X}_L * \mathbb{Y}_L , \x, \y', \mathbb{X}_{RR} * \mathbb{Y}_{RR} , \z')  \right\} \notag \displaybreak[3] \\
=& \sum_{\substack{\mathbb{X}_L(x)\mathbb{X}_R = \mathbb{X} \\ \mathbb{Y}_L(\ide{y})\mathbb{Y}_R = \mathbb{Y}}}c(\x,\ide{y}) f_1(\x,\mathbb{X}_R * \mathbb{Y}_R, 0)f_2(\z,\mathbb{X}_L * \mathbb{Y}_L, \x ,\z') +\sum_{\substack{ \mathbb{X}_L (\x) \mathbb{X}_M \mathbb{X}_R = \mathbb{X} \\\mathbb{Y}_L (\ide{y}) \mathbb{Y}_M \mathbb{Y}_R = \mathbb{Y} \\ (\mathbb{X}_R, \mathbb{Y}_R) \neq (\emptyset, \emptyset) }} c(\x,\ide{y}) f_1(\x, \mathbb{X}_M,0)f_2(\z, \mathbb{X}_L * \mathbb{Y}_L , \x, \mathbb{X}_{R} * \mathbb{Y}_{R} , \z') \notag \\
=&\sum_{\substack{ \mathbb{X}_L (\x) \mathbb{X}_M \mathbb{X}_R = \mathbb{X} \\\mathbb{Y}_L (\ide{y}) \mathbb{Y}_M \mathbb{Y}_R = \mathbb{Y}}} c(\x,\ide{y}) f_1(\x, \mathbb{X}_M * \mathbb{Y}_M,0)f_2(\z, \mathbb{X}_L * \mathbb{Y}_L , \x, \mathbb{X}_{R} * \mathbb{Y}_{R} , \z')  \label{eq:BV0-1}
\end{align}
}


Therefore, due to $\Psi_1 \in \addmr$, we have
\begin{align*}
&A_{\X\z'\rightarrow \X0} + A_{\X\X'\rightarrow \X0} + A_{\X\X'\rightarrow \X0} \\
\stackrel{\eqref{eq:BV0-1}}{=}& \sum_{\substack{ \mathbb{X}_L (\x) \mathbb{X}_M \mathbb{X}_R = \mathbb{X} \\\mathbb{Y}_L (\ide{y}) \mathbb{Y}_M \mathbb{Y}_R = \mathbb{Y}}} c(\x,\ide{y}) f_1(\x, \mathbb{X}_M  * \mathbb{Y}_M ,0)f_2(\z, \mathbb{X}_L * \mathbb{Y}_L , \x, \mathbb{X}_{R} * \mathbb{Y}_{R} , \z') \displaybreak[3] \\
\stackrel{\eqref{eq:adjoint-harmonic}}{=}&
 \sum_{\substack{ \mathbb{X}_L (\x) \mathbb{X}_M \mathbb{X}_R = \mathbb{X} \\\mathbb{Y}_L (\ide{y}) \mathbb{Y}_R = \mathbb{Y}}} c(\x,\ide{y}) f_1(\x, \mathbb{X}_M  ,0)f_2(\z, \mathbb{X}_L * \mathbb{Y}_L , \x, \mathbb{X}_{R} * \mathbb{Y}_{R} , \z') \\
&+ \sum_{\substack{ \mathbb{X}_L (\x) \mathbb{X}_R = \mathbb{X} \\\mathbb{Y}_L (\ide{y}) \mathbb{Y}_M \mathbb{Y}_R = \mathbb{Y}}} c(\x,\ide{y}) f_1(\x,  \mathbb{Y}_M ,0)f_2(\z, \mathbb{X}_L * \mathbb{Y}_L , \x, \mathbb{X}_{R} * \mathbb{Y}_{R} , \z') 
\displaybreak[3]\\
=&B_{\X0}
\end{align*}
as claimed.


\begin{comment}
Next, let us prove \eqref{eq:Bx0}. Direct calculations show
{\footnotesize
\begin{align}
&A_{\V\z'\rightarrow \V0} + A_{\X\z'\rightarrow \X0} + A_{\Y\z'\rightarrow \Y0} \notag \\
=& \sum_{\substack{\mathbb{X}_L \alpha_x(\V) \mathbb{X}_R = \mathbb{X} \\  \mathbb{Y}_L\alpha_y(\V) \mathbb{Y}_R = \mathbb{Y}}}c(\v(V), \gamma_1(V)) f_1(\v(\V),\mathbb{X}_R * \mathbb{Y}_R, 0)f_2(\z,\mathbb{X}_L * \mathbb{Y}_L,\v(\V),\z') \notag \\
&+\sum_{\V \in \{\X.\Y\}}  \sum_{\substack{\mathbb{X}_L \alpha_x(\V) \mathbb{X}_M \beta_x(\V')\mathbb{X}_R = \mathbb{X} \\  \mathbb{Y}_L\alpha_y(\V) \mathbb{Y}_M (\y')\mathbb{Y}_R = \mathbb{Y}}}c(\v(V), \gamma_1(V))c(\v'(V'), \gamma_2(V')) \notag \\
&\qquad\times f_1(\v(\V),\mathbb{X}_M * \mathbb{Y}_M, 0)f_2(\z,\mathbb{X}_L * \mathbb{Y}_L,\v(\V),\v'(\V'),\mathbb{X}_R * \mathbb{Y}_R,\z') \displaybreak[3] \notag\\
=&
\sum_{\substack{\mathbb{X}_L \alpha_x(\V) \mathbb{X}_R = \mathbb{X} \\  \mathbb{Y}_L\alpha_y(\V) \mathbb{Y}_R = \mathbb{Y}}}c(\v(V), \gamma_1(V)) f_1(\v(\V),\mathbb{X}_R * \mathbb{Y}_R, 0)f_2(\z,\mathbb{X}_L * \mathbb{Y}_L,\v(\V),\z') \notag \\
&+ \sum_{\substack{\mathbb{X}_L\alpha_x(\V)\mathbb{X}_M\mathbb{X}_R  = \mathbb{X} \\ \mathbb{Y}_L\alpha_y(\V)\mathbb{Y}_M\mathbb{Y}_R  = \mathbb{Y} \\ (\mathbb{X}_R, \mathbb{Y}_R) \neq (\emptyset, \emptyset) }} c(\v(V), \gamma_1(V))f_1(\v(\V),\mathbb{X}_M * \mathbb{Y}_M, 0) \notag \\
&\times \left\{\sum_{\V' \in \{\X', \Y'\}} \sum_{\substack{ \beta_x(\V')\mathbb{X}_{RR} = \mathbb{X}_{R} \\   \beta_y(\V')\mathbb{Y}_{RR} = \mathbb{Y}_{R} }} c(\v'(V'), \gamma_2(V'))  f_2(\z,\mathbb{X}_L * \mathbb{Y}_L,\v(\V),\v'(\V'),\mathbb{X}_{RR} * \mathbb{Y}_{RR},\z')  \right\}. 
\end{align}
}
In a similar manner of \eqref{eq:Bz0-2}, it follows that

\begin{align}
& \sum_{\V' \in \{\X', \Y'\}} \sum_{\substack{ \beta_x(\V')\mathbb{X}_{RR} = \mathbb{X}_{R} \\   \beta_y(\V')\mathbb{Y}_{RR} = \mathbb{Y}_{R} }} c(\v'(V'), \gamma_2(V'))  f_2(\z,\mathbb{X}_L * \mathbb{Y}_L,\v(\V),\v'(\V'),\mathbb{X}_{RR} * \mathbb{Y}_{RR},\z')  \displaybreak[3] \notag \\
=& f_2(\z,\mathbb{X}_L * \mathbb{Y}_L,\v(\V),\mathbb{X}_{R} * \mathbb{Y}_{R},\z') \label{eq:BV0-2}
\end{align}

\end{comment}

\end{proof}

\begin{lem}\label{eq:Conv-harmonic-z}
For $\Psi_1$ $\in \addmr \cap \Fad $ and $\Psi_2\in \addmr$,
\begin{align*}
B_{\z0} + B_{0\z'} - B_{00}^{\z}=&f_1(\z,\mathbb{X},\z')f_2(\z,\mathbb{Y},\z') +  f_1(\z,\mathbb{Y},\z')f_2(\z,\mathbb{X},\z').
\end{align*}
\end{lem}

\begin{proof}
Since $\Psi_2 \in \addmr(\Q)$, we have
\begin{align*}
B_{\z0} =& f_1(0,\mathbb{X},\z')f_2(\z,\mathbb{Y},\z') +  f_1(0,\mathbb{Y},\z')f_2(\z,\mathbb{X},\z')\\ 
B_{0\z'} =& f_1(\z,\mathbb{X},0)f_2(\z,\mathbb{Y},\z') +  f_1(\z,\mathbb{Y},0)f_2(\z,\mathbb{X},\z')\\ 
B_{00}^{\z} =& f_1(0,\mathbb{X},0)f_2(\z,\mathbb{Y},\z') +  f_1(0,\mathbb{Y},0)f_2(\z,\mathbb{X},\z').
\end{align*}
Since $\Psi_1 \in \Fad$, we have
{\footnotesize
\begin{align*}
&B_{\z0} + B_{0\z'} - B_{00}^z \\
=&\left\{ f_1(0,\mathbb{X},\z') + f_1(\z,\mathbb{X},0) - f_1(0,\mathbb{X},0) \right\}f_2(\z,\mathbb{Y},\z') + \left\{ f_1(0,\mathbb{Y},\z') + f_1(\z,\mathbb{Y},0) - f_1(0,\mathbb{Y},0) \right\}f_2(\z,\mathbb{X},\z')\\
\stackrel{\eqref{eq:adjoint-mould}}{=} &f_1(\z,\mathbb{X},\z')f_2(\z,\mathbb{Y},\z') +  f_1(\z,\mathbb{Y},\z')f_2(\z,\mathbb{X},\z')
\end{align*}
}
as claimed.
\end{proof}

For $(\V,\V') \in \{(\X,0),(\Y,0),(0,\X'),(0,\Y'),(0,0)\}$, decompose $B_{\V\V'}$ as
\[
B_{\V\V'} \;=\; B_{\V\V'}^{\X} \;+\; B_{\V\V'}^{\Y},
\]
where $B_{\V\V'}^{\X}$ (respectively, $B_{\V\V'}^{\Y}$) is the sum of those terms of $B_{\V\V'}$ for which the middle part of $f_{1}$ (i.e., $f_{1}$ with its first and last letters removed) is a word in $\mathbb{X}$ (respectively, over $\mathbb{Y}$).





\begin{lem}\label{lem:Conv-harmonic}
For $\Psi_1$ $\in \addmr \cap \Fad$ and $\Psi_2\in \addmr$, we have
{\footnotesize
\begin{align}
&B_{\X0}^{\X} =  \sum_{\substack{\mathbb{X}_L(\x)\mathbb{X}_M\mathbb{X}_R = \mathbb{X} \\ \mathbb{Y}_L(\y) \mathbb{Y}_R = \mathbb{Y}}} c(\x,\y ) f_1(\x,\mathbb{X}_M, 0)f_2(\z,\mathbb{X}_L * \mathbb{Y}_L , \y, \mathbb{X}_R * \mathbb{Y}_R, \z'), \label{eq:Conv-harmonic-1} \\
&B_{0\X'}^{\X} =  \sum_{\substack{\mathbb{X}_L\mathbb{X}_M(\x)\mathbb{X}_R = \mathbb{X} \\ \mathbb{Y}_L(\y) \mathbb{Y}_R = \mathbb{Y}}} c(\x,\y ) f_1(0,\mathbb{X}_M, \x)f_2(\z,\mathbb{X}_L * \mathbb{Y}_L , \y, \mathbb{X}_R * \mathbb{Y}_R, \z'), \text{ and } \label{eq:Conv-harmonic-2}\\
 &\sum_{\substack{\mathbb{X}_L(\x)\mathbb{X}_M\mathbb{X}_R = \mathbb{X} \\ \mathbb{Y}_L(\ide{y}) \mathbb{Y}_R = \mathbb{Y}}} c(\x , \ide{y}) f_1(0,\mathbb{X}_M, 0)f_2(\z,\mathbb{X}_L * \mathbb{Y}_L , \x, \mathbb{X}_R * \mathbb{Y}_R, \z') =  \sum_{\substack{\mathbb{X}_L(\x)\mathbb{X}_M\mathbb{X}_R = \mathbb{X} \\ \mathbb{Y}_L(\y) \mathbb{Y}_R = \mathbb{Y}}} c(\x,\y ) f_1(0,\mathbb{X}_M, 0)f_2(\z,\mathbb{X}_L * \mathbb{Y}_L , \y, \mathbb{X}_R * \mathbb{Y}_R, \z'). \label{eq:Conv-harmonic-3} 
\end{align}
}


\begin{comment}
{\footnotesize
\begin{align}
\ConvLM^{(x)}(\X; f_1(\v(\X),-,0), f_2(\z,-, \z'); \mathbb{X}, \mathbb{Y}) =&  \sum_{\substack{\mathbb{X}_L(\x)\mathbb{X}_M\mathbb{X}_R = \mathbb{X} \\ \mathbb{Y}_L(\y) \mathbb{Y}_R = \mathbb{Y}}} c(\x,\y ) f_1(\x,\mathbb{X}_M, 0)f_2(\z,\mathbb{X}_L * \mathbb{Y}_L , \y, \mathbb{X}_R * \mathbb{Y}_R, \z') \label{eq:Conv-harmonic-1} \\
\ConvMR^{(x)}(\X; f_1(0,-,\v'(\V')), f_2(\z,-, \z'); \mathbb{X}, \mathbb{Y}) =&  \sum_{\substack{\mathbb{X}_L\mathbb{X}_M(\x)\mathbb{X}_R = \mathbb{X} \\ \mathbb{Y}_L(\y) \mathbb{Y}_R = \mathbb{Y}}} c(\x,\y ) f_1(0,\mathbb{X}_M, \x)f_2(\z,\mathbb{X}_L * \mathbb{Y}_L , \y, \mathbb{X}_R * \mathbb{Y}_R, \z') \label{eq:Conv-harmonic-2}\\
\ConvLM^{(x)}(\X; f_1(0,-,0), f_2(\z,-, \z'); \mathbb{X}, \mathbb{Y}) =&  \sum_{\substack{\mathbb{X}_L(\x)\mathbb{X}_M\mathbb{X}_R = \mathbb{X} \\ \mathbb{Y}_L(\y) \mathbb{Y}_R = \mathbb{Y}}} c(\x,\y ) f_1(0,\mathbb{X}_M, 0)f_2(\z,\mathbb{X}_L * \mathbb{Y}_L , \y, \mathbb{X}_R * \mathbb{Y}_R, \z'). \label{eq:Conv-harmonic-3} 
\end{align}
}
\end{comment}

\end{lem}

\begin{proof}

We prove only \eqref{eq:Conv-harmonic-1}. The others can be shown in a similar manner.

Consider the following decomposition: $\mathbb{X}_L (\x)\mathbb{X}_M\mathbb{X}_R = \mathbb{X}$.
Then, applying Lemma \ref{lem:har-inductive} to $(\mathbb{X}_L(\x)\mathbb{X}_R) * \mathbb{Y}$, we have
\begin{align}
(\mathbb{X}_L(\x)\mathbb{X}_R) * \mathbb{Y}
\stackrel{\eqref{eq:harmonic-composition}}{=}\sum_{\mathbb{Y}_L(\ide{y})\mathbb{Y}_R = \mathbb{Y}} c (\x,\ide{y})(\mathbb{X}_L * \mathbb{Y}_L)((\x) - (\ide{y}))\mathbb{X}_R * \mathbb{Y}_R. \label{eq:Conv-harmonic-10}
\end{align}
Since $\Psi_2 \in \addmr$, it follows that
{\footnotesize
\begin{align}
0=& f_2(\z , (\mathbb{X}_L\cdot (\x)\cdot \mathbb{X}_R) * \mathbb{Y},\z') \notag \\
\stackrel{\eqref{eq:Conv-harmonic-10}}{=}&\sum_{\mathbb{Y}_L(\ide{y})\mathbb{Y}_R = \mathbb{Y}} c(\x,\ide{y}) f_2(\z,( \mathbb{X}_L * \mathbb{Y}_L), ((\x) - (\ide{y})), (\mathbb{X}_R * \mathbb{Y}_R),\z') \notag \\
\Longleftrightarrow& \notag \\
&\sum_{\mathbb{Y}_L(\ide{y})\mathbb{Y}_R = \mathbb{Y}} c(\x,\ide{y}) f_2(\z,( \mathbb{X}_L * \mathbb{Y}_L), (\x), (\mathbb{X}_R * \mathbb{Y}_R),\z') = \sum_{\mathbb{Y}_L(\y)\mathbb{Y}_R = \mathbb{Y}} c(\x,\y) f_2(\z,( \mathbb{X}_L * \mathbb{Y}_L), (\y), (\mathbb{X}_R * \mathbb{Y}_R),\z'). \label{eq:Conv-harmonic-11}
\end{align}
}
Thus we have
\begin{align*}
&B_{\X0}^{\X}\\
=&\sum_{\substack{\mathbb{X}_L(\x)\mathbb{X}_M \mathbb{X}_R \\ \mathbb{Y}_L(\ide{y})\mathbb{Y}_R = \mathbb{Y}}} c(\x,\ide{y}) f_1(\x,\mathbb{X}_M,0)f_2(\z,(\mathbb{X}_L * \mathbb{Y}_L),\x,(\mathbb{X}_R * \mathbb{Y}_R),\z')\\
\stackrel{\eqref{eq:Conv-harmonic-11}}{=}&
\sum_{\substack{\mathbb{X}_L(\x)\mathbb{X}_M \mathbb{X}_R \\ \mathbb{Y}_L(\y)\mathbb{Y}_R = \mathbb{Y}}} c(\x,\y) f_1(\x,\mathbb{X}_M,0)f_2(\z,(\mathbb{X}_L * \mathbb{Y}_L),\y,(\mathbb{X}_R * \mathbb{Y}_R),\z')
\end{align*}
as claimed.
\end{proof}

\begin{lem}\label{lem:Conv-harmonic-00}
For $\Psi_1$ $\in \addmr \cap \Fad$ and $\Psi_2\in \addmr$, we have
\begin{align}
B_{00}^{\X}
=\sum_{\substack{\mathbb{X}_L\mathbb{X}_M\mathbb{X}_R = \mathbb{X} \\ \mathbb{Y}_L(\y) \mathbb{Y}_R = \mathbb{Y}}}  f_1(0,\mathbb{X}_M, 0)f_2(\z,\mathbb{X}_L * \mathbb{Y}_L , \y, \mathbb{X}_R * \mathbb{Y}_R, \z'). \label{eq:Conv-harmonic-00}
\end{align}
\end{lem}

\begin{proof}

By Lemma \ref{lem:Conv-harmonic}, we have

{\footnotesize
\begin{align*}
&B_{00}^{\X}\\
=&\sum_{\substack{\mathbb{X}_L(\x)\mathbb{X}_M\mathbb{X}_R = \mathbb{X} \\ \mathbb{Y}_L(\ide{y}) \mathbb{Y}_R = \mathbb{Y}}} c(\x , \ide{y}) f_1(0,\mathbb{X}_M, 0)f_2(\z,\mathbb{X}_L * \mathbb{Y}_L , \x, \mathbb{X}_R * \mathbb{Y}_R, \z')  + \sum_{\substack{\mathbb{X}_L(\x)\mathbb{X}_R = \mathbb{X} \\ \mathbb{Y}_L(\ide{y}) \mathbb{Y}_M \mathbb{Y}_R = \mathbb{Y}}} c(\x , \ide{y}) f_1(0,\mathbb{Y}_M, 0)f_2(\z,\mathbb{X}_L * \mathbb{Y}_L , \x, \mathbb{X}_R * \mathbb{Y}_R, \z') \\ 
\stackrel{\eqref{eq:Conv-harmonic-3}}{=}& \sum_{\substack{\mathbb{X}_L(\x)\mathbb{X}_M\mathbb{X}_R = \mathbb{X} \\ \mathbb{Y}_L(\y) \mathbb{Y}_R = \mathbb{Y}}} c(\x,\y ) f_1(0,\mathbb{X}_M, 0)f_2(\z,\mathbb{X}_L * \mathbb{Y}_L , \y, \mathbb{X}_R * \mathbb{Y}_R, \z') +\sum_{\substack{\mathbb{X}_L(\ide{x})\mathbb{X}_M\mathbb{X}_R = \mathbb{X} \\ \mathbb{Y}_L(\y) \mathbb{Y}_R = \mathbb{Y}}} c(\y , \ide{x}) f_1(0,\mathbb{X}_M, 0)f_2(\z,\mathbb{X}_L * \mathbb{Y}_L , \y, \mathbb{X}_R * \mathbb{Y}_R, \z') \displaybreak[3] \\
=& \sum_{\substack{\mathbb{X}_L(\x)\mathbb{X}_M\mathbb{X}_R = \mathbb{X} \\ \mathbb{Y}_L(\y) \mathbb{Y}_R = \mathbb{Y}}} c(\x,\y ) f_1(0,\mathbb{X}_M, 0)f_2(\z,\mathbb{X}_L * \mathbb{Y}_L , \y, \mathbb{X}_R * \mathbb{Y}_R, \z') +\sum_{\substack{\mathbb{X}_L(\x)\mathbb{X}_M\mathbb{X}_R = \mathbb{X} \\ \mathbb{Y}_L(\y) \mathbb{Y}_R = \mathbb{Y}}} c(\y , \x) f_1(0,\mathbb{X}_M, 0)f_2(\z,\mathbb{X}_L * \mathbb{Y}_L , \y, \mathbb{X}_R * \mathbb{Y}_R, \z') \\
&+\sum_{\substack{\mathbb{X}_L\mathbb{X}_M\mathbb{X}_R = \mathbb{X} \\ \mathbb{Y}_L(\y) \mathbb{Y}_R = \mathbb{Y}}}  f_1(0,\mathbb{X}_M, 0)f_2(\z,\mathbb{X}_L * \mathbb{Y}_L , \y, \mathbb{X}_R * \mathbb{Y}_R, \z') \displaybreak[3] \\
=&\sum_{\substack{\mathbb{X}_L\mathbb{X}_M\mathbb{X}_R = \mathbb{X} \\ \mathbb{Y}_L(\y) \mathbb{Y}_R = \mathbb{Y}}}  f_1(0,\mathbb{X}_M, 0)f_2(\z,\mathbb{X}_L * \mathbb{Y}_L , \y, \mathbb{X}_R * \mathbb{Y}_R, \z')
\end{align*}
}
as claimed.
\end{proof}

\begin{lem}\label{lem:Conv-harmonic-3}
For $\Psi_1$ $\in \addmr \cap \Fad \cap \V_{\strprty}$ and $\Psi_2\in \addmr$, we have
{\footnotesize
\begin{align}
&B_{\Y0}^{\X} + B_{0\Y'}^{\X} - B_{00}^{\X} \notag \displaybreak[3] \\
=&\sum_{\substack{\mathbb{X}_L\mathbb{X}_M \mathbb{X}_R = \mathbb{X} \\ \mathbb{Y}_L(\y)\mathbb{Y}_R = \mathbb{Y}}} f_1(\y,\mathbb{X}_M,\y)f_2(\z,(\mathbb{X}_L * \mathbb{Y}_L),\y,(\mathbb{X}_R * \mathbb{Y}_R),\z') \notag \\
&+\sum_{\substack{\mathbb{X}_L(\x)\mathbb{X}_M \mathbb{X}_R = \mathbb{X} \\ \mathbb{Y}_L(\y)\mathbb{Y}_R = \mathbb{Y}}} c(\y,\x) f_1(\y,\mathbb{X}_M,0)f_2(\z,(\mathbb{X}_L * \mathbb{Y}_L),\y,(\mathbb{X}_R * \mathbb{Y}_R),\z')
+\sum_{\substack{\mathbb{X}_L\mathbb{X}_M(\x) \mathbb{X}_R \\ \mathbb{Y}_L(\y)\mathbb{Y}_R = \mathbb{Y}}} c(\y,\x) f_1(0,\mathbb{X}_M,\y)f_2(\z,(\mathbb{X}_L * \mathbb{Y}_L),\y,(\mathbb{X}_R * \mathbb{Y}_R),\z'). \label{eq:Conv-harmonic-4}
\end{align}
}
\end{lem}
\begin{proof}
By direct calculation, we have
{\footnotesize
\begin{align*}
&B_{\Y0}^{\X} + B_{0\Y'}^{\X} - B_{00}^{\X} \displaybreak[3] \\
=&\sum_{\substack{\mathbb{X}_L(\ide{x})\mathbb{X}_M \mathbb{X}_R = \mathbb{X} \\ \mathbb{Y}_L(\y)\mathbb{Y}_R = \mathbb{Y}}} c(\y,\ide{x}) f_1(\y,\mathbb{X}_M,0)f_2(\z,(\mathbb{X}_L * \mathbb{Y}_L),\y,(\mathbb{X}_R * \mathbb{Y}_R),\z')+\sum_{\substack{\mathbb{X}_L\mathbb{X}_M(\ide{x}) \mathbb{X}_R = \mathbb{X} \\ \mathbb{Y}_L(\y)\mathbb{Y}_R = \mathbb{Y}}} c(\y,\ide{x}) f_1(0,\mathbb{X}_M,\y)f_2(\z,(\mathbb{X}_L * \mathbb{Y}_L),\y,(\mathbb{X}_R * \mathbb{Y}_R),\z')\\
&-B_{00}^{\X} \displaybreak[3] \\
%
\stackrel{\eqref{eq:Conv-harmonic-00}}{=}&\sum_{\substack{\mathbb{X}_L(\ide{x})\mathbb{X}_M \mathbb{X}_R = \mathbb{X} \\ \mathbb{Y}_L(\y)\mathbb{Y}_R = \mathbb{Y}}} c(\y,\ide{x}) f_1(\y,\mathbb{X}_M,0)f_2(\z,(\mathbb{X}_L * \mathbb{Y}_L),\y,(\mathbb{X}_R * \mathbb{Y}_R),\z')+\sum_{\substack{\mathbb{X}_L\mathbb{X}_M(\ide{x}) \mathbb{X}_R = \mathbb{X} \\ \mathbb{Y}_L(\y)\mathbb{Y}_R = \mathbb{Y}}} c(\y,\ide{x}) f_1(0,\mathbb{X}_M,\y)f_2(\z,(\mathbb{X}_L * \mathbb{Y}_L),\y,(\mathbb{X}_R * \mathbb{Y}_R),\z')\\
&\qquad -\sum_{\substack{\mathbb{X}_L\mathbb{X}_M\mathbb{X}_R = \mathbb{X} \\ \mathbb{Y}_L(\y) \mathbb{Y}_R = \mathbb{Y}}}  f_1(0,\mathbb{X}_M, 0)f_2(\z,\mathbb{X}_L * \mathbb{Y}_L , \y, \mathbb{X}_R * \mathbb{Y}_R, \z') \displaybreak[3] \\
%
=&\sum_{\substack{\mathbb{X}_L\mathbb{X}_M \mathbb{X}_R = \mathbb{X} \\ \mathbb{Y}_L(\y)\mathbb{Y}_R = \mathbb{Y}}}  f_1(\y,\mathbb{X}_M,0)f_2(\z,(\mathbb{X}_L * \mathbb{Y}_L),\y,(\mathbb{X}_R * \mathbb{Y}_R),\z')+\sum_{\substack{\mathbb{X}_L(\x)\mathbb{X}_M \mathbb{X}_R = \mathbb{X} \\ \mathbb{Y}_L(\y)\mathbb{Y}_R = \mathbb{Y}}} c(\y,\x) f_1(\y,\mathbb{X}_M,0)f_2(\z,(\mathbb{X}_L * \mathbb{Y}_L),\y,(\mathbb{X}_R * \mathbb{Y}_R),\z')\\
&+\sum_{\substack{\mathbb{X}_L\mathbb{X}_M\mathbb{X}_R = \mathbb{X} \\ \mathbb{Y}_L(\y)\mathbb{Y}_R = \mathbb{Y}}} f_1(0,\mathbb{X}_M,\y)f_2(\z,(\mathbb{X}_L * \mathbb{Y}_L),\y,(\mathbb{X}_R * \mathbb{Y}_R),\z')+\sum_{\substack{\mathbb{X}_L\mathbb{X}_M(\x) \mathbb{X}_R = \mathbb{X} \\ \mathbb{Y}_L(\y)\mathbb{Y}_R = \mathbb{Y}}} c(\y,\x) f_1(0,\mathbb{X}_M,\y)f_2(\z,(\mathbb{X}_L * \mathbb{Y}_L),\y,(\mathbb{X}_R * \mathbb{Y}_R),\z') \\
&\qquad-\sum_{\substack{\mathbb{X}_L\mathbb{X}_M\mathbb{X}_R = \mathbb{X} \\ \mathbb{Y}_L(\y) \mathbb{Y}_R = \mathbb{Y}}}  f_1(0,\mathbb{X}_M, 0)f_2(\z,\mathbb{X}_L * \mathbb{Y}_L , \y, \mathbb{X}_R * \mathbb{Y}_R, \z') \displaybreak[3] \\
%
=&\sum_{\substack{\mathbb{X}_L\mathbb{X}_M \mathbb{X}_R = \mathbb{X} \\ \mathbb{Y}_L(\y)\mathbb{Y}_R = \mathbb{Y}}}\left\{  f_1(\y,\mathbb{X}_M,0) +  f_1(0,\mathbb{X}_M,\y) -  f_1(0,\mathbb{X}_M,0)\right\}f_2(\z,(\mathbb{X}_L * \mathbb{Y}_L),\y,(\mathbb{X}_R * \mathbb{Y}_R),\z')\\
&+\sum_{\substack{\mathbb{X}_L(\x)\mathbb{X}_M \mathbb{X}_R = \mathbb{X} \\ \mathbb{Y}_L(\y)\mathbb{Y}_R = \mathbb{Y}}} c(\y,\x) f_1(\y,\mathbb{X}_M,0)f_2(\z,(\mathbb{X}_L * \mathbb{Y}_L),\y,(\mathbb{X}_R * \mathbb{Y}_R),\z')
+\sum_{\substack{\mathbb{X}_L\mathbb{X}_M(\x) \mathbb{X}_R \\ \mathbb{Y}_L(\y)\mathbb{Y}_R = \mathbb{Y}}} c(\y,\x) f_1(0,\mathbb{X}_M,\y)f_2(\z,(\mathbb{X}_L * \mathbb{Y}_L),\y,(\mathbb{X}_R * \mathbb{Y}_R),\z')\\ \displaybreak[3]
%
\stackrel{\eqref{eq:adjoint-harmonic}}{=}&\sum_{\substack{\mathbb{X}_L\mathbb{X}_M \mathbb{X}_R = \mathbb{X} \\ \mathbb{Y}_L(\y)\mathbb{Y}_R = \mathbb{Y}}} f_1(\y,\mathbb{X}_M,\y)f_2(\z,(\mathbb{X}_L * \mathbb{Y}_L),\y,(\mathbb{X}_R * \mathbb{Y}_R),\z')\\
&+\sum_{\substack{\mathbb{X}_L(\x)\mathbb{X}_M \mathbb{X}_R = \mathbb{X} \\ \mathbb{Y}_L(\y)\mathbb{Y}_R = \mathbb{Y}}} c(\y,\x) f_1(\y,\mathbb{X}_M,0)f_2(\z,(\mathbb{X}_L * \mathbb{Y}_L),\y,(\mathbb{X}_R * \mathbb{Y}_R),\z')
+\sum_{\substack{\mathbb{X}_L\mathbb{X}_M(\x) \mathbb{X}_R \\ \mathbb{Y}_L(\y)\mathbb{Y}_R = \mathbb{Y}}} c(\y,\x) f_1(0,\mathbb{X}_M,\y)f_2(\z,(\mathbb{X}_L * \mathbb{Y}_L),\y,(\mathbb{X}_R * \mathbb{Y}_R),\z') \displaybreak[3]
\end{align*}
}
as claimed.
\end{proof}

\begin{lem}\label{eq:Conv-harmonic-parity}
For $\Psi_1$ $\in \addmr \cap \Fad \cap \V_{\strprty}$ and $\Psi_2\in \addmr$, we have
\begin{align*}
B_{\X0}^{\X} + B_{0\X'}^{\X} + B_{\Y0}^{\X} + B_{0\Y'}^{\X} - B_{00}^{\X} =0.
\end{align*}
\end{lem}

\begin{proof}
By Lemma \ref{lem:Conv-harmonic} and Lemma \ref{lem:Conv-harmonic-3}, we have
{\footnotesize
\begin{align*}
&B_{\X0}^{\X} + B_{0\X'}^{\X} + B_{\Y0}^{\X} + B_{0\Y'}^{\X} - B_{00}^{\X}  \displaybreak[3] \\
%
\stackrel{\substack{\eqref{eq:Conv-harmonic-1} \\ \eqref{eq:Conv-harmonic-2} }}{=} & 
 \sum_{\substack{\mathbb{X}_L(\x)\mathbb{X}_M\mathbb{X}_R = \mathbb{X} \\ \mathbb{Y}_L(\y) \mathbb{Y}_R = \mathbb{Y}}} c(\x,\y ) f_1(\x,\mathbb{X}_M, 0)f_2(\z,\mathbb{X}_L * \mathbb{Y}_L , \y, \mathbb{X}_R * \mathbb{Y}_R, \z') 
+ \sum_{\substack{\mathbb{X}_L\mathbb{X}_M(\x)\mathbb{X}_R = \mathbb{X} \\ \mathbb{Y}_L(\y) \mathbb{Y}_R = \mathbb{Y}}} c(\x,\y ) f_1(0,\mathbb{X}_M, \x)f_2(\z,\mathbb{X}_L * \mathbb{Y}_L , \y, \mathbb{X}_R * \mathbb{Y}_R, \z') \\
&+B_{\Y0}^{\X} + B_{0\Y'}^{\X} - B_{00}^{\X} \displaybreak[3] \\
%
\stackrel{\eqref{eq:Conv-harmonic-4}}{=}&
 \sum_{\substack{\mathbb{X}_L(\x)\mathbb{X}_M\mathbb{X}_R = \mathbb{X} \\ \mathbb{Y}_L(\y) \mathbb{Y}_R = \mathbb{Y}}} c(\x,\y ) f_1(\x,\mathbb{X}_M, 0)f_2(\z,\mathbb{X}_L * \mathbb{Y}_L , \y, \mathbb{X}_R * \mathbb{Y}_R, \z') 
+ \sum_{\substack{\mathbb{X}_L\mathbb{X}_M(\x)\mathbb{X}_R = \mathbb{X} \\ \mathbb{Y}_L(\y) \mathbb{Y}_R = \mathbb{Y}}} c(\x,\y ) f_1(0,\mathbb{X}_M, \x)f_2(\z,\mathbb{X}_L * \mathbb{Y}_L , \y, \mathbb{X}_R * \mathbb{Y}_R, \z') \\
&+\sum_{\substack{\mathbb{X}_L\mathbb{X}_M \mathbb{X}_R = \mathbb{X} \\ \mathbb{Y}_L(\y)\mathbb{Y}_R = \mathbb{Y}}} f_1(\y,\mathbb{X}_M,\y)f_2(\z,(\mathbb{X}_L * \mathbb{Y}_L),\y,(\mathbb{X}_R * \mathbb{Y}_R),\z') \notag \\
&+\sum_{\substack{\mathbb{X}_L(\x)\mathbb{X}_M \mathbb{X}_R = \mathbb{X} \\ \mathbb{Y}_L(\y)\mathbb{Y}_R = \mathbb{Y}}} c(\y,\x) f_1(\y,\mathbb{X}_M,0)f_2(\z,(\mathbb{X}_L * \mathbb{Y}_L),\y,(\mathbb{X}_R * \mathbb{Y}_R),\z')
+\sum_{\substack{\mathbb{X}_L\mathbb{X}_M(\x) \mathbb{X}_R \\ \mathbb{Y}_L(\y)\mathbb{Y}_R = \mathbb{Y}}} c(\y,\x) f_1(0,\mathbb{X}_M,\y)f_2(\z,(\mathbb{X}_L * \mathbb{Y}_L),\y,(\mathbb{X}_R * \mathbb{Y}_R),\z') \displaybreak[3] \\
=&\sum_{\substack{\mathbb{Y}_L(\y)\mathbb{Y}_R = \mathbb{Y} \\ \mathbb{X}_L(\x)\mathbb{X}_M(\x')\mathbb{X}_R = \mathbb{X}}}f_2(\z,\mathbb{X}_L * \mathbb{Y}_L , \y, \mathbb{X}_R * \mathbb{Y}_R, \z')\\
&\hspace{0.2cm}
\begin{aligned}
&\qquad\times\left\{ f_1(\y,\x,\mathbb{X}_M,\x', \y)  + c(\x,\y ) \left(f_1(\x,\mathbb{X}_M,\x', 0) - f_1(\y,\mathbb{X}_M,\x',0)\right) + c(\x',\y)\left( f_1(0,\x,\mathbb{X}_M, \x') - f_1(0,\x,\mathbb{X}_M,\y)\right)  \right\} 
\end{aligned}\\
\stackrel{\eqref{eq:strong-parity-revised}}{=}0
\end{align*}
}
as claimed.
\end{proof}

\begin{lem}\label{eq:Conv-harmonic-parity-y}
For $\Psi_1$ $\in \addmr \cap \Fad \cap \V_{\strprty}$ and $\Psi_2\in \addmr$, we have
\begin{align*}
B_{\X0}^{\Y} + B_{0\X'}^{\Y} + B_{\Y0}^{\Y} + B_{0\Y'}^{\Y} - B_{00}^{\Y}=0.
\end{align*}
\end{lem}
\begin{proof}
By symmetry, it suffices to replace $\mathbb{X}$ as $\mathbb{Y}$ in the above lemmas.
\end{proof}

\begin{cor}\label{cor:conclusion}
For $\Psi_1$ $\in \addmr \cap \Fad \cap \V_{\strprty}$ and $\Psi_2\in \addmr$, we have
\begin{align*}
\vimo_{d_{\Psi_1(\Psi_2)}}(\z, \mathbb{X} * \mathbb{Y}, \z')
=f_1(\z,\mathbb{X},\z')f_2(\z,\mathbb{Y},\z') + f_1(\z,\mathbb{Y},\z')f_2(\z,\mathbb{X},\z')
\end{align*}

\begin{proof}
This corollary immediately follows from Lemma \ref{lem:vimo-B}, Lemma \ref{eq:Conv-harmonic-z}, Lemma \ref{eq:Conv-harmonic-parity}, and Lemma \ref{eq:Conv-harmonic-parity-y}.
\end{proof}

\end{cor}

\begin{prop}[ = Lemma \ref{lem:essential}]
For $\Psi_1$ $\in \addmr \cap \Fad \cap \V_{\strprty}$ and $\Psi_2\in \addmr$, we have
\[
\Delta_*(d_{\Psi_1}(\Psi_2)_{\#}) = d_{\Psi_1}(\Psi_2)_{\#} \otimes 1 + 1\otimes d_{\Psi_1}(\Psi_2)_{\#} + \Psi_{1.\#}\otimes \Psi_{2,\#} + \Psi_{2,\#}\otimes \Psi_{1,\#}.
\]
\end{prop}

\begin{proof}
Since both $\Delta_*$ and $*$ are dual maps relatively, we have
\begin{align*}
\langle \Delta_*(d_{\Psi_1}(\Psi_2)) \mid w \otimes 1\rangle = \langle d_{\Psi_1}(\Psi_2) \mid w * 1\rangle 
=\langle d_{\Psi_1}(\Psi_2) \otimes 1 \mid w \otimes 1\rangle
\end{align*}
and
\begin{align*}
&\langle \Delta_*(d_{\Psi_1}(\Psi_2)) \mid 1 \otimes w\rangle
=\langle d_{\Psi_1}(\Psi_2) \mid 1 * w\rangle 
=\langle 1\otimes d_{\Psi_1}(\Psi_2)   \mid 1 \otimes w\rangle
\end{align*}
for any $w \in Y^*$.

Next, we investigate $\langle \Delta_*(d_{\Psi_1}(\Psi_2)) \mid w_1 \otimes w_2\rangle$ for any non-empty words $w_1$, $w_2 \in Y^*\setminus\{1\}$.
By Corollary \ref{cor:conclusion}, we have
\begin{align*}
&\langle d_{\Psi_1}(\Psi_2) \mid y_{l+1}((y_{k_{r+s}}\cdots y_{k_{r+1}}) * (y_{k_{r}}\cdots y_{k_{1}})) \rangle \\
=& \sum_{l_1 + l_2 =l} \langle \Psi_1\mid y_{l_1+1}y_{k_{r+s}}\cdots y_{k_{r+1}}\rangle \langle \Psi_2\mid y_{l_2+1}y_{k_{r}}\cdots y_{k_{1}}\rangle + \langle \Psi_2\mid y_{l_1+1}y_{k_{r+s}}\cdots y_{k_{r+1}}\rangle \langle \Psi_1\mid y_{l_2+1}y_{k_{r}}\cdots y_{k_{1}}\rangle
\end{align*}
for $l \in \mathbb{Z}_{\ge 0}$, $r$, $l\in\mathbb{Z}_{>0}$, $k_1,\ldots,k_{r+s} \in \mathbb{Z}_{>0}$.
Thus we have
\begin{align*}
&\langle d_{\Psi_1}(\Psi_2)_{\#} \mid ((y_{k_{r+s}}\cdots y_{k_{r+1}}) * (y_{k_{r}}\cdots y_{k_{1}})) \rangle \\
=&\langle \Psi_{1,\#}\mid y_{k_{r+s}}\cdots y_{k_{r+1}}\rangle \langle \Psi_{2,\#}\mid y_{k_{r}}\cdots y_{k_{1}}\rangle+ \langle \Psi_{2,\#}\mid y_{k_{r+s}}\cdots y_{k_{r+1}}\rangle \langle \Psi_{1,\#}\mid y_{k_{r}}\cdots y_{k_{1}}\rangle.
\end{align*}
Since both $\Delta_*$ and $*$ are dual maps relatively, we have
\begin{align*}
&\langle \Delta_*(d_{\Psi_1}(\Psi_2)_{\#}) \mid y_{k_{r+s}}\cdots y_{k_{r+1}}\otimes y_{k_r}\cdots y_{k_1}\rangle \\
=&\langle d_{\Psi_1}(\Psi_2)_{\#} \mid y_{k_{r+s}}\cdots y_{k_{r+1}}* y_{k_r}\cdots y_{k_1}\rangle \displaybreak[3] \\
=&\langle \Psi_{1,\#}\mid y_{k_{r+s}}\cdots y_{k_{r+1}}\rangle \langle \Psi_{2,\#}\mid y_{k_{r}}\cdots y_{k_{1}}\rangle+ \langle \Psi_{2,\#}\mid y_{k_{r+s}}\cdots y_{k_{r+1}}\rangle \langle \Psi_{1,\#}\mid y_{k_{r}}\cdots y_{k_{1}}\rangle \displaybreak[3]\\
=&\langle \Psi_{1,\#}\otimes \Psi_{2,\#}\mid y_{k_{r+s}}\cdots y_{k_{r+1}} \otimes y_{k_{r}}\cdots y_{k_{1}}\rangle
+ \langle \Psi_{2,\#}\otimes \Psi_{1,\#}\mid y_{k_{r+s}}\cdots y_{k_{r+1}} \otimes y_{k_{r}}\cdots y_{k_{1}}\rangle.
\end{align*}

Hence, it follows that
\begin{align*}
\begin{aligned}
\Delta_*(d_{\Psi_1}(\Psi_2)_{\#}) = d_{\Psi_1}(\Psi_2)_{\#} \otimes 1 + 1\otimes d_{\Psi_1}(\Psi_2)_{\#} + \Psi_{1.\#}\otimes \Psi_{2,\#} + \Psi_{2,\#}\otimes \Psi_{1,\#}
\end{aligned} % \label{eq:essential}
\end{align*}
as claimed.
\end{proof}



\begin{comment}
\begin{lem}\label{lem:sum-B}
For $\Psi_1$ $\in \addmr(\Q) \cap  V_{\strprty}$ and $\Psi_2\in \ide{h}^{\vee}$, then we have
\begin{footnotesize}
\begin{align*}
&B_{\x0} + B_{0\x} + B_{\y0} + B_{0\y} - B_{00} \\
=&\sum_{\substack{\mathbb{X}_L(\x)\mathbb{X}_M(\x')\mathbb{X}_R = \mathbb{X} }} f_1(\x,\mathbb{X}_M, \x')f_2(\z, (\mathbb{X}_L\cdot(\x)\cdot(\x')\cdot \mathbb{X}_R) * \mathbb{Y},\z') + \sum_{\substack{\mathbb{X}_L(\x)\mathbb{X}_R = \mathbb{X} }} f_1(\x,\mathbb{X}_R, 0)f_2(\z, (\mathbb{X}_L\cdot(\x)) * \mathbb{Y},\z') \\
&+ \sum_{\substack{\mathbb{X}_L(\x)\mathbb{X}_R = \mathbb{X} }} f_1(0,\mathbb{X}_L, \x)f_2(\z, ((\x)\cdot \mathbb{X}_R) * \mathbb{Y},\z') - \sum_{\substack{\mathbb{X}_L(\x)\mathbb{X}_R = \mathbb{X} }} f_1(0,\mathbb{X}_R, 0)f_2(\z, (\mathbb{X}_L\cdot(\x)) * \mathbb{Y},\z')\\
&+\sum_{\substack{\mathbb{Y}_L(\y)\mathbb{Y}_M(\y')\mathbb{Y}_R = \mathbb{X} }} f_1(\y,\mathbb{Y}_M, \y')f_2(\z, \mathbb{X} * (\mathbb{Y}_L\cdot(\y)\cdot(\y')\cdot \mathbb{Y}_R),\z') + \sum_{\substack{\mathbb{Y}_L(\y)\mathbb{Y}_R = \mathbb{Y} }} f_1(\y,\mathbb{Y}_R, 0)f_2(\z, (\mathbb{Y}_L\cdot(\y)) ,\z') \\
&+ \sum_{\substack{\mathbb{Y}_L(\y)\mathbb{Y}_R = \mathbb{Y} }} f_1(0,\mathbb{Y}_L, \y)f_2(\z, \mathbb{X} *  ((\y)\cdot \mathbb{Y}_R) ,\z') - \sum_{\substack{\mathbb{Y}_L(\y)\mathbb{Y}_R = \mathbb{Y} }} f_1(0,\mathbb{Y}_R, 0)f_2(\z, \mathbb{X} *  (\mathbb{Y}_L\cdot(\y)) ,\z')
\end{align*}
\end{footnotesize}
\end{lem}

\begin{proof}

\begin{footnotesize}
Let us consider
\begin{align}
\begin{aligned}
&B_{\x0} + B_{0\x} + B_{\y0} + B_{0\y} - B_{00} \\
=&\sum_{\substack{\mathbb{X}_L(\x)\mathbb{X}_M\mathbb{X}_R = \mathbb{X} \\ \mathbb{Y}_L(\ide{y}) \mathbb{Y}_R = \mathbb{Y}}} c(\x , \ide{y}) f_1(\x,\mathbb{X}_M, 0)f_2(\z,\mathbb{X}_L * \mathbb{Y}_L , \x, \mathbb{X}_R * \mathbb{Y}_R, \z') + \sum_{\substack{\mathbb{X}_L(\x)\mathbb{X}_R = \mathbb{X} \\ \mathbb{Y}_L(\ide{y}) \mathbb{Y}_M \mathbb{Y}_R = \mathbb{Y}}} c(\x , \ide{y}) f_1(\x,\mathbb{Y}_M, 0)f_2(\z,\mathbb{X}_L * \mathbb{Y}_L , \x, \mathbb{X}_R * \mathbb{Y}_R, \z') \\
&+ \sum_{\substack{\mathbb{X}_L\mathbb{X}_M(\x)\mathbb{X}_R = \mathbb{X} \\ \mathbb{Y}_L(\ide{y}) \mathbb{Y}_R = \mathbb{Y}}} c(\x , \ide{y}) f_1(0,\mathbb{X}_M, \x)f_2(\z,\mathbb{X}_L * \mathbb{Y}_L , \x, \mathbb{X}_R * \mathbb{Y}_R, \z') + \sum_{\substack{\mathbb{X}_L(\x)\mathbb{X}_R = \mathbb{X} \\ \mathbb{Y}_L \mathbb{Y}_M (\ide{y})\mathbb{Y}_R = \mathbb{Y}}} c(\x , \ide{y}) f_1(\x,\mathbb{Y}_M, 0)f_2(\z,\mathbb{X}_L * \mathbb{Y}_L , \x, \mathbb{X}_R * \mathbb{Y}_R, \z')  \\
&+  \sum_{\substack{\mathbb{X}_L(\ide{x})\mathbb{X}_M\mathbb{X}_R = \mathbb{X} \\ \mathbb{Y}_L(\y) \mathbb{Y}_R = \mathbb{Y}}} c(\y , \ide{x}) f_1(\y,\mathbb{X}_M, 0)f_2(\z,\mathbb{X}_L * \mathbb{Y}_L , \y, \mathbb{X}_R * \mathbb{Y}_R, \z')  + \sum_{\substack{\mathbb{X}_L(\ide{x})\mathbb{X}_R = \mathbb{X} \\ \mathbb{Y}_L(\y) \mathbb{Y}_M \mathbb{Y}_R = \mathbb{Y}}} c(\y,\ide{x}) f_1(\y,\mathbb{Y}_M, 0)f_2(\z,\mathbb{X}_L * \mathbb{Y}_L , \y, \mathbb{X}_R * \mathbb{Y}_R, \z')  \\
&+ \sum_{\substack{\mathbb{X}_L\mathbb{X}_M(\ide{x})\mathbb{X}_R = \mathbb{X} \\ \mathbb{Y}_L(\y) \mathbb{Y}_R = \mathbb{Y}}} c(\y , \ide{x}) f_1(0,\mathbb{X}_M, \y) f_2(\z,\mathbb{X}_L * \mathbb{Y}_L , \y, \mathbb{X}_R * \mathbb{Y}_R, \z')  + \sum_{\substack{\mathbb{X}_L(\ide{x})\mathbb{X}_R = \mathbb{X} \\ \mathbb{Y}_L \mathbb{Y}_M (\y)\mathbb{Y}_R = \mathbb{Y}}} c(\y , \ide{x}) f_1(\x,\mathbb{Y}_M, 0)f_2(\z,\mathbb{X}_L * \mathbb{Y}_L , \y, \mathbb{X}_R * \mathbb{Y}_R, \z')  \\
&-\sum_{\substack{\mathbb{X}_L(\x)\mathbb{X}_M\mathbb{X}_R = \mathbb{X} \\ \mathbb{Y}_L(\ide{y}) \mathbb{Y}_R = \mathbb{Y}}} c(\x , \ide{y}) f_1(0,\mathbb{X}_M, 0)f_2(\z,\mathbb{X}_L * \mathbb{Y}_L , \x, \mathbb{X}_R * \mathbb{Y}_R, \z')   - \sum_{\substack{\mathbb{X}_L(\x)\mathbb{X}_R = \mathbb{X} \\ \mathbb{Y}_L(\ide{y}) \mathbb{Y}_M \mathbb{Y}_R = \mathbb{Y}}} c(\x , \ide{y}) f_1(0,\mathbb{Y}_M, 0)f_2(\z,\mathbb{X}_L * \mathbb{Y}_L , \x, \mathbb{X}_R * \mathbb{Y}_R, \z') \\ 
&-\sum_{\substack{\mathbb{X}_L(\ide{x})\mathbb{X}_M\mathbb{X}_R = \mathbb{X} \\ \mathbb{Y}_L(\y) \mathbb{Y}_R = \mathbb{Y}}} c(\y , \ide{x}) f_1(0,\mathbb{X}_M, 0)f_2(\z,\mathbb{X}_L * \mathbb{Y}_L , \y, \mathbb{X}_R * \mathbb{Y}_R, \z')  - \sum_{\substack{\mathbb{X}_L(\ide{x})\mathbb{X}_R = \mathbb{X} \\ \mathbb{Y}_L(\y) \mathbb{Y}_M \mathbb{Y}_R = \mathbb{Y}}} c(\y,\ide{x}) f_1(0,\mathbb{Y}_M, 0)f_2(\z,\mathbb{X}_L * \mathbb{Y}_L , \y, \mathbb{X}_R * \mathbb{Y}_R, \z')
\end{aligned} \label{eq:sumB}
\end{align}

\end{footnotesize}

Applying Lemma \ref{lem:har-inductive} to the first term above, we have

\begin{footnotesize}
\begin{align*}
&\sum_{\substack{\mathbb{X}_L(\x)\mathbb{X}_M\mathbb{X}_R = \mathbb{X} \\ \mathbb{Y}_L(\ide{y}) \mathbb{Y}_R = \mathbb{Y}}} c(\x , \ide{y}) f_1(\x,\mathbb{X}_M, 0)f_2(\z,\mathbb{X}_L * \mathbb{Y}_L , \x, \mathbb{X}_R * \mathbb{Y}_R, \z')\\
=&\sum_{\substack{\mathbb{X}_L(\x)\mathbb{X}_M\mathbb{X}_R = \mathbb{X} }} f_1(\x,\mathbb{X}_M, 0)f_2(\z, (\mathbb{X}_L\cdot(\x)\cdot\mathbb{X}_R) * \mathbb{Y},\z')+\sum_{\substack{\mathbb{X}_L(\x)\mathbb{X}_M\mathbb{X}_R = \mathbb{X} \\ \mathbb{Y}_L(\y) \mathbb{Y}_R = \mathbb{Y}}} c(\x,\y ) f_1(\x,\mathbb{X}_M, 0)f_2(\z,\mathbb{X}_L * \mathbb{Y}_L , \y, \mathbb{X}_R * \mathbb{Y}_R, \z'),
\end{align*}
\end{footnotesize}

In a similar way, applying Lemma \ref{lem:har-inductive} to the third, and nineth terms above, we have

\begin{footnotesize}
\begin{align*}
&\sum_{\substack{\mathbb{X}_L\mathbb{X}_M(\x)\mathbb{X}_R = \mathbb{X} \\ \mathbb{Y}_L(\ide{y}) \mathbb{Y}_R = \mathbb{Y}}} c(\x , \ide{y}) f_1(0,\mathbb{X}_M, \x)f_2(\z,\mathbb{X}_L * \mathbb{Y}_L , \x, \mathbb{X}_R * \mathbb{Y}_R, \z')\\
=&\sum_{\substack{\mathbb{X}_L\mathbb{X}_M(\x)\mathbb{X}_R = \mathbb{X} }}  f_1(0,\mathbb{X}_M, \x)f_2(\z,(\mathbb{X}_L \cdot (\x) \cdot \mathbb{X}_R)* \mathbb{Y}, \z')+\sum_{\substack{\mathbb{X}_L\mathbb{X}_M(\x)\mathbb{X}_R = \mathbb{X} \\ \mathbb{Y}_L(\y) \mathbb{Y}_R = \mathbb{Y}}} c(\x,\y) f_1(0,\mathbb{X}_M, \x)f_2(\z,\mathbb{X}_L * \mathbb{Y}_L , \y, \mathbb{X}_R * \mathbb{Y}_R, \z'),
\end{align*}
\end{footnotesize}
and
\begin{footnotesize}
\begin{align}
\begin{aligned}
&-\sum_{\substack{\mathbb{X}_L(\x)\mathbb{X}_M\mathbb{X}_R = \mathbb{X} \\ \mathbb{Y}_L(\ide{y}) \mathbb{Y}_R = \mathbb{Y}}} c(\x , \ide{y}) f_1(0,\mathbb{X}_M, 0)f_2(\z,\mathbb{X}_L * \mathbb{Y}_L , \x, \mathbb{X}_R * \mathbb{Y}_R, \z')\\
=&-\sum_{\substack{\mathbb{X}_L(\x)\mathbb{X}_M\mathbb{X}_R = \mathbb{X} }} f_1(0,\mathbb{X}_M, 0)f_2(\z, (\mathbb{X}_L\cdot(\x)\cdot\mathbb{X}_R) * \mathbb{Y},\z')-\sum_{\substack{\mathbb{X}_L(\x)\mathbb{X}_M\mathbb{X}_R = \mathbb{X} \\ \mathbb{Y}_L(\y) \mathbb{Y}_R = \mathbb{Y}}} c(\x,\y ) f_1(0,\mathbb{X}_M, 0)f_2(\z,\mathbb{X}_L * \mathbb{Y}_L , \y, \mathbb{X}_R * \mathbb{Y}_R, \z'),
\end{aligned} \label{eq:00composition}
\end{align}
\end{footnotesize}


Additionally, by considering 11th term of \eqref{eq:sumB} and \eqref{eq:00composition}, we have
\begin{footnotesize}
\begin{align*}
&-\sum_{\substack{\mathbb{X}_L(\x)\mathbb{X}_M\mathbb{X}_R = \mathbb{X} \\ \mathbb{Y}_L(\ide{y}) \mathbb{Y}_R = \mathbb{Y}}} c(\x , \ide{y}) f_1(0,\mathbb{X}_M, 0)f_2(\z,\mathbb{X}_L * \mathbb{Y}_L , \x, \mathbb{X}_R * \mathbb{Y}_R, \z')-\sum_{\substack{\mathbb{X}_L(\ide{x})\mathbb{X}_M\mathbb{X}_R = \mathbb{X} \\ \mathbb{Y}_L(\y) \mathbb{Y}_R = \mathbb{Y}}} c(\y , \ide{x}) f_1(0,\mathbb{X}_M, 0)f_2(\z,\mathbb{X}_L * \mathbb{Y}_L , \y, \mathbb{X}_R * \mathbb{Y}_R, \z') \displaybreak[3] \\
=&-\sum_{\substack{\mathbb{X}_L(\x)\mathbb{X}_M\mathbb{X}_R = \mathbb{X} }} f_1(0,\mathbb{X}_M, 0)f_2(\z, (\mathbb{X}_L\cdot(\x)\cdot\mathbb{X}_R) * \mathbb{Y},\z') -\sum_{\substack{\mathbb{X}_L(\x)\mathbb{X}_M\mathbb{X}_R = \mathbb{X} \\ \mathbb{Y}_L(\y) \mathbb{Y}_R = \mathbb{Y}}} c(\x , \y) f_1(0,\mathbb{X}_M, 0)f_2(\z,\mathbb{X}_L * \mathbb{Y}_L , \y, \mathbb{X}_R * \mathbb{Y}_R, \z')\\
&-\sum_{\substack{\mathbb{X}_L(\ide{x})\mathbb{X}_M\mathbb{X}_R = \mathbb{X} \\ \mathbb{Y}_L(\y) \mathbb{Y}_R = \mathbb{Y}}} c(\y , \ide{x}) f_1(0,\mathbb{X}_M, 0)f_2(\z,\mathbb{X}_L * \mathbb{Y}_L , \y, \mathbb{X}_R * \mathbb{Y}_R, \z') \displaybreak[3] \\
=&-\sum_{\substack{\mathbb{X}_L(\x)\mathbb{X}_M\mathbb{X}_R = \mathbb{X} }} f_1(0,\mathbb{X}_M, 0)f_2(\z, (\mathbb{X}_L\cdot(\x)\cdot\mathbb{X}_R) * \mathbb{Y},\z') -\sum_{\substack{\mathbb{X}_L\mathbb{X}_M\mathbb{X}_R = \mathbb{X} \\ \mathbb{Y}_L(\y) \mathbb{Y}_R = \mathbb{Y}}}  f_1(0,\mathbb{X}_M, 0)f_2(\z,\mathbb{X}_L * \mathbb{Y}_L , \y, \mathbb{X}_R * \mathbb{Y}_R, \z').
\end{align*}
\end{footnotesize}

Therefore, the sum of 1st, 3rd, 5th, 7th, 9th, and 11th of \eqref{eq:sumB} turns to

\begin{footnotesize}
\begin{align*}
&\text{(Sum of 1st, 3rd, 5th, 7th, 9th, and 11th of \eqref{eq:sumB})}\\
=&\sum_{\substack{\mathbb{X}_L(\x)\mathbb{X}_M\mathbb{X}_R = \mathbb{X} \\ \mathbb{Y}_L(\ide{y}) \mathbb{Y}_R = \mathbb{Y}}} c(\x , \ide{y}) f_1(\x,\mathbb{X}_M, 0)f_2(\z,\mathbb{X}_L * \mathbb{Y}_L , \x, \mathbb{X}_R * \mathbb{Y}_R, \z')  + \sum_{\substack{\mathbb{X}_L\mathbb{X}_M(\x)\mathbb{X}_R = \mathbb{X} \\ \mathbb{Y}_L(\ide{y}) \mathbb{Y}_R = \mathbb{Y}}} c(\x , \ide{y}) f_1(0,\mathbb{X}_M, \x)f_2(\z,\mathbb{X}_L * \mathbb{Y}_L , \x, \mathbb{X}_R * \mathbb{Y}_R, \z')  \\
&+  \sum_{\substack{\mathbb{X}_L(\ide{x})\mathbb{X}_M\mathbb{X}_R = \mathbb{X} \\ \mathbb{Y}_L(\y) \mathbb{Y}_R = \mathbb{Y}}} c(\y , \ide{x}) f_1(\y,\mathbb{X}_M, 0)f_2(\z,\mathbb{X}_L * \mathbb{Y}_L , \y, \mathbb{X}_R * \mathbb{Y}_R, \z')  + \sum_{\substack{\mathbb{X}_L\mathbb{X}_M(\ide{x})\mathbb{X}_R = \mathbb{X} \\ \mathbb{Y}_L(\y) \mathbb{Y}_R = \mathbb{Y}}} c(\y , \ide{x}) f_1(0,\mathbb{X}_M, \y) f_2(\z,\mathbb{X}_L * \mathbb{Y}_L , \y, \mathbb{X}_R * \mathbb{Y}_R, \z') \\
&-\sum_{\substack{\mathbb{X}_L(\x)\mathbb{X}_M\mathbb{X}_R = \mathbb{X} \\ \mathbb{Y}_L(\ide{y}) \mathbb{Y}_R = \mathbb{Y}}} c(\x , \ide{y}) f_1(0,\mathbb{X}_M, 0)f_2(\z,\mathbb{X}_L * \mathbb{Y}_L , \x, \mathbb{X}_R * \mathbb{Y}_R, \z')  -\sum_{\substack{\mathbb{X}_L(\ide{x})\mathbb{X}_M\mathbb{X}_R = \mathbb{X} \\ \mathbb{Y}_L(\y) \mathbb{Y}_R = \mathbb{Y}}} c(\y , \ide{x}) f_1(0,\mathbb{X}_M, 0)f_2(\z,\mathbb{X}_L * \mathbb{Y}_L , \y, \mathbb{X}_R * \mathbb{Y}_R, \z') \\
=&\sum_{\substack{\mathbb{X}_L(\x)\mathbb{X}_M\mathbb{X}_R = \mathbb{X} }} f_1(\x,\mathbb{X}_M, 0)f_2(\z, (\mathbb{X}_L\cdot(\x)\cdot\mathbb{X}_R) * \mathbb{Y},\z') + \sum_{\substack{\mathbb{X}_L\mathbb{X}_M(\x)\mathbb{X}_R = \mathbb{X} }}  f_1(0,\mathbb{X}_M, \x)f_2(\z,(\mathbb{X}_L \cdot (\x) \cdot \mathbb{X}_R)* \mathbb{Y}, \z')\\
&-\sum_{\substack{\mathbb{X}_L(\x)\mathbb{X}_M\mathbb{X}_R = \mathbb{X} }} f_1(0,\mathbb{X}_M, 0)f_2(\z, (\mathbb{X}_L\cdot(\x)\cdot\mathbb{X}_R) * \mathbb{Y},\z')\\
&+\sum_{\substack{\mathbb{X}_L(\x)\mathbb{X}_M\mathbb{X}_R = \mathbb{X} \\ \mathbb{Y}_L(\y) \mathbb{Y}_R = \mathbb{Y}}} c(\x,\y ) f_1(\x,\mathbb{X}_M, 0)f_2(\z,\mathbb{X}_L * \mathbb{Y}_L , \y, \mathbb{X}_R * \mathbb{Y}_R, \z') +\sum_{\substack{\mathbb{X}_L\mathbb{X}_M(\x)\mathbb{X}_R = \mathbb{X} \\ \mathbb{Y}_L(\y) \mathbb{Y}_R = \mathbb{Y}}} c(\x,\y) f_1(0,\mathbb{X}_M, \x)f_2(\z,\mathbb{X}_L * \mathbb{Y}_L , \y, \mathbb{X}_R * \mathbb{Y}_R, \z')\\
&+  \sum_{\substack{\mathbb{X}_L(\ide{x})\mathbb{X}_M\mathbb{X}_R = \mathbb{X} \\ \mathbb{Y}_L(\y) \mathbb{Y}_R = \mathbb{Y}}} c(\y , \ide{x}) f_1(\y,\mathbb{X}_M, 0)f_2(\z,\mathbb{X}_L * \mathbb{Y}_L , \y, \mathbb{X}_R * \mathbb{Y}_R, \z') 
+ \sum_{\substack{\mathbb{X}_L\mathbb{X}_M(\ide{x})\mathbb{X}_R = \mathbb{X} \\ \mathbb{Y}_L(\y) \mathbb{Y}_R = \mathbb{Y}}} c(\y , \ide{x}) f_1(0,\mathbb{X}_M, \y) f_2(\z,\mathbb{X}_L * \mathbb{Y}_L , \y, \mathbb{X}_R * \mathbb{Y}_R, \z') \\
&-\sum_{\substack{\mathbb{X}_L(\x)\mathbb{X}_M\mathbb{X}_R = \mathbb{X} \\ \mathbb{Y}_L(\y) \mathbb{Y}_R = \mathbb{Y}}} c(\x,\y ) f_1(0,\mathbb{X}_M, 0)f_2(\z,\mathbb{X}_L * \mathbb{Y}_L , \y, \mathbb{X}_R * \mathbb{Y}_R, \z') -\sum_{\substack{\mathbb{X}_L(\ide{x})\mathbb{X}_M\mathbb{X}_R = \mathbb{X} \\ \mathbb{Y}_L(\y) \mathbb{Y}_R = \mathbb{Y}}} c(\y , \ide{x}) f_1(0,\mathbb{X}_M, 0)f_2(\z,\mathbb{X}_L * \mathbb{Y}_L , \y, \mathbb{X}_R * \mathbb{Y}_R, \z') \\
=&\sum_{\substack{\mathbb{X}_L(\x)\mathbb{X}_M\mathbb{X}_R = \mathbb{X} }} f_1(\x,\mathbb{X}_M, 0)f_2(\z, (\mathbb{X}_L\cdot(\x)\cdot\mathbb{X}_R) * \mathbb{Y},\z') + \sum_{\substack{\mathbb{X}_L\mathbb{X}_M(\x)\mathbb{X}_R = \mathbb{X} }}  f_1(0,\mathbb{X}_M, \x)f_2(\z,(\mathbb{X}_L \cdot (\x) \cdot \mathbb{X}_R)* \mathbb{Y}, \z')\\
&-\sum_{\substack{\mathbb{X}_L(\x)\mathbb{X}_M\mathbb{X}_R = \mathbb{X} }} f_1(0,\mathbb{X}_M, 0)f_2(\z, (\mathbb{X}_L\cdot(\x)\cdot\mathbb{X}_R) * \mathbb{Y},\z')\\
&+\sum_{\substack{\mathbb{X}_L\mathbb{X}_M\mathbb{X}_R = \mathbb{X} \\ \mathbb{Y}_L(\y) \mathbb{Y}_R = \mathbb{Y}}} f_1(\y,\mathbb{X}_M, \y)f_2(\z,\mathbb{X}_L * \mathbb{Y}_L , \y, \mathbb{X}_R * \mathbb{Y}_R, \z')\\
&+\sum_{\substack{\mathbb{X}_L(\x)\mathbb{X}_M\mathbb{X}_R = \mathbb{X} \\ \mathbb{Y}_L(\y) \mathbb{Y}_R = \mathbb{Y}}} c(\x,\y ) \left\{f_1(\x,\mathbb{X}_M, 0) - f_1(\y,\mathbb{X}_M,0) \right\}f_2(\z,\mathbb{X}_L * \mathbb{Y}_L , \y, \mathbb{X}_R * \mathbb{Y}_R, \z')\\
&+\sum_{\substack{\mathbb{X}_L\mathbb{X}_M(\x)\mathbb{X}_R = \mathbb{X} \\ \mathbb{Y}_L(\y) \mathbb{Y}_R = \mathbb{Y}}} c(\x,\y)\left\{ f_1(0,\mathbb{X}_M, \x) - f_1(0,\mathbb{X}_M,\y)\right\}f_2(\z,\mathbb{X}_L * \mathbb{Y}_L , \y, \mathbb{X}_R * \mathbb{Y}_R, \z') \displaybreak[3] \\
=
&\sum_{\substack{\mathbb{X}_L(\x)\mathbb{X}_M\mathbb{X}_R = \mathbb{X} }} f_1(\x,\mathbb{X}_M, 0)f_2(\z, (\mathbb{X}_L\cdot(\x)\cdot\mathbb{X}_R) * \mathbb{Y},\z') + \sum_{\substack{\mathbb{X}_L\mathbb{X}_M(\x)\mathbb{X}_R = \mathbb{X} }}  f_1(0,\mathbb{X}_M, \x)f_2(\z,(\mathbb{X}_L \cdot (\x) \cdot \mathbb{X}_R)* \mathbb{Y}, \z')\\
&-\sum_{\substack{\mathbb{X}_L(\x)\mathbb{X}_M\mathbb{X}_R = \mathbb{X} }} f_1(0,\mathbb{X}_M, 0)f_2(\z, (\mathbb{X}_L\cdot(\x)\cdot\mathbb{X}_R) * \mathbb{Y},\z') \\
&\sum_{\substack{\mathbb{Y}_L(\y)\mathbb{Y}_R = \mathbb{Y} \\ \mathbb{X}_L(\x)\mathbb{X}_M(\x')\mathbb{X}_R = \mathbb{X}}}f_2(\z,\mathbb{X}_L * \mathbb{Y}_L , \y, \mathbb{X}_R * \mathbb{Y}_R, \z')\\
&\hspace{0.2cm}
\begin{aligned}
&\qquad\times\left\{ f_1(\y,\x,\mathbb{X}_M,\x', \y) \right.\\
&\qquad \hspace{0.5cm}\left. + c(\x,\y ) \left(f_1(\x,\mathbb{X}_M,\x', 0) - f_1(\y,\mathbb{X}_M,\x',0)\right) + c(\x',\y)\left( f_1(0,\x,\mathbb{X}_M, \x') - f_1(0,\x,\mathbb{X}_M,\y)\right)  \right\}
\end{aligned}\\
=&\sum_{\substack{\mathbb{X}_L(\x)\mathbb{X}_M\mathbb{X}_R = \mathbb{X} }} f_1(\x,\mathbb{X}_M, 0)f_2(\z, (\mathbb{X}_L\cdot(\x)\cdot\mathbb{X}_R) * \mathbb{Y},\z') + \sum_{\substack{\mathbb{X}_L\mathbb{X}_M(\x)\mathbb{X}_R = \mathbb{X} }}  f_1(0,\mathbb{X}_M, \x)f_2(\z,(\mathbb{X}_L \cdot (\x) \cdot \mathbb{X}_R)* \mathbb{Y}, \z')\\
&-\sum_{\substack{\mathbb{X}_L(\x)\mathbb{X}_M\mathbb{X}_R = \mathbb{X} }} f_1(0,\mathbb{X}_M, 0)f_2(\z, (\mathbb{X}_L\cdot(\x)\cdot\mathbb{X}_R) * \mathbb{Y},\z') \\
=&\sum_{\substack{\mathbb{X}_L(\x)\mathbb{X}_M(\x')\mathbb{X}_R = \mathbb{X} }} f_1(\x,\mathbb{X}_M, 0)f_2(\z, (\mathbb{X}_L\cdot(\x)\cdot(\x')\cdot \mathbb{X}_R) * \mathbb{Y},\z') + \sum_{\substack{\mathbb{X}_L(\x)\mathbb{X}_R = \mathbb{X} }} f_1(\x,\mathbb{X}_R, 0)f_2(\z, (\mathbb{X}_L\cdot(\x)) * \mathbb{Y},\z')\\
&+ \sum_{\substack{\mathbb{X}_L(\x)\mathbb{X}_M(\x')\mathbb{X}_R = \mathbb{X} }}  f_1(0,\mathbb{X}_M, \x')f_2(\z,(\mathbb{X}_L \cdot(\x)\cdot (\x') \cdot \mathbb{X}_R)* \mathbb{Y}, \z') + \sum_{\substack{\mathbb{X}_L(\x)\mathbb{X}_R = \mathbb{X} }} f_1(0,\mathbb{X}_R, \x)f_2(\z, ((\x)\cdot \mathbb{X}_R) * \mathbb{Y},\z') \\
&-\sum_{\substack{\mathbb{X}_L(\x)\mathbb{X}_M(\x')\mathbb{X}_R = \mathbb{X} }} f_1(0,\mathbb{X}_M, 0)f_2(\z, (\mathbb{X}_L\cdot(\x)\cdot(\x')\cdot \mathbb{X}_R) * \mathbb{Y},\z') - \sum_{\substack{\mathbb{X}_L(\x)\mathbb{X}_R = \mathbb{X} }} f_1(0,\mathbb{X}_R, 0)f_2(\z, (\mathbb{X}_L\cdot(\x)) * \mathbb{Y},\z') \displaybreak[3] \\
=&\sum_{\substack{\mathbb{X}_L(\x)\mathbb{X}_M(\x')\mathbb{X}_R = \mathbb{X} }} 
\left( f_1(\x,\mathbb{X}_M, 0) +  f_1(0,\mathbb{X}_M, \x') - f_1(0,\mathbb{X}_M, 0) \right) f_2(\z, (\mathbb{X}_L\cdot(\x)\cdot(\x')\cdot \mathbb{X}_R) * \mathbb{Y},\z') \\
& + \sum_{\substack{\mathbb{X}_L(\x)\mathbb{X}_R = \mathbb{X} }} f_1(\x,\mathbb{X}_R, 0)f_2(\z, (\mathbb{X}_L\cdot(\x)) * \mathbb{Y},\z') + \sum_{\substack{\mathbb{X}_L(\x)\mathbb{X}_R = \mathbb{X} }} f_1(0,\mathbb{X}_R, \x)f_2(\z, ((\x)\cdot \mathbb{X}_R) * \mathbb{Y},\z')\\
& - \sum_{\substack{\mathbb{X}_L(\x)\mathbb{X}_R = \mathbb{X} }} f_1(0,\mathbb{X}_R, 0)f_2(\z, (\mathbb{X}_L\cdot(\x)) * \mathbb{Y},\z')\\
=&\sum_{\substack{\mathbb{X}_L(\x)\mathbb{X}_M(\x')\mathbb{X}_R = \mathbb{X} }} f_1(\x,\mathbb{X}_M, \x')f_2(\z, (\mathbb{X}_L\cdot(\x)\cdot(\x')\cdot \mathbb{X}_R) * \mathbb{Y},\z') + \sum_{\substack{\mathbb{X}_L(\x)\mathbb{X}_R = \mathbb{X} }} f_1(\x,\mathbb{X}_R, 0)f_2(\z, (\mathbb{X}_L\cdot(\x)) * \mathbb{Y},\z') \\
&+ \sum_{\substack{\mathbb{X}_L(\x)\mathbb{X}_R = \mathbb{X} }} f_1(0,\mathbb{X}_L, \x)f_2(\z, ((\x)\cdot \mathbb{X}_R) * \mathbb{Y},\z') - \sum_{\substack{\mathbb{X}_L(\x)\mathbb{X}_R = \mathbb{X} }} f_1(0,\mathbb{X}_R, 0)f_2(\z, (\mathbb{X}_L\cdot(\x)) * \mathbb{Y},\z').
\end{align*}

\end{footnotesize}
In the forth-to-last equality, we use Proposition \ref{prop:strong-parity}.
In the last equality, we use the adjoint conidition Proposition \ref{prop:addmr-mould}. (\ref{enumi:ad-2}).

In a similar way, we have

\begin{footnotesize}
\begin{align*}
&\text{(Sum of 2nd, 4th, 6th, 8th, 10th, and 12th of \eqref{eq:sumB})}\\
=&\sum_{\substack{\mathbb{Y}_L(\y)\mathbb{Y}_M(\y')\mathbb{Y}_R = \mathbb{X} }} f_1(\y,\mathbb{Y}_M, \y')f_2(\z, \mathbb{X} * (\mathbb{Y}_L\cdot(\y)\cdot(\y')\cdot \mathbb{Y}_R),\z') + \sum_{\substack{\mathbb{Y}_L(\y)\mathbb{Y}_R = \mathbb{Y} }} f_1(\y,\mathbb{Y}_R, 0)f_2(\z, (\mathbb{Y}_L\cdot(\y)) ,\z') \\
&+ \sum_{\substack{\mathbb{Y}_L(\y)\mathbb{Y}_R = \mathbb{Y} }} f_1(0,\mathbb{Y}_L, \x)f_2(\z, \mathbb{X} *  ((\y)\cdot \mathbb{Y}_R) ,\z') - \sum_{\substack{\mathbb{Y}_L(\y)\mathbb{Y}_R = \mathbb{Y} }} f_1(0,\mathbb{Y}_R, 0)f_2(\z, \mathbb{X} *  (\mathbb{Y}_L\cdot(\y)) ,\z')
\end{align*}
\end{footnotesize}

Thus, it follows that

\begin{footnotesize}

\begin{align*}
&B_{\x0} + B_{0\x} + B_{\y0} + B_{0\y} - B_{00} \\
=&\sum_{\substack{\mathbb{X}_L(\x)\mathbb{X}_M(\x')\mathbb{X}_R = \mathbb{X} }} f_1(\x,\mathbb{X}_M, \x')f_2(\z, (\mathbb{X}_L\cdot(\x)\cdot(\x')\cdot \mathbb{X}_R) * \mathbb{Y},\z') + \sum_{\substack{\mathbb{X}_L(\x)\mathbb{X}_R = \mathbb{X} }} f_1(\x,\mathbb{X}_R, 0)f_2(\z, (\mathbb{X}_L\cdot(\x)) * \mathbb{Y},\z') \\
&+ \sum_{\substack{\mathbb{X}_L(\x)\mathbb{X}_R = \mathbb{X} }} f_1(0,\mathbb{X}_L, \x)f_2(\z, ((\x)\cdot \mathbb{X}_R) * \mathbb{Y},\z') - \sum_{\substack{\mathbb{X}_L(\x)\mathbb{X}_R = \mathbb{X} }} f_1(0,\mathbb{X}_R, 0)f_2(\z, (\mathbb{X}_L\cdot(\x)) * \mathbb{Y},\z')\\
&+\sum_{\substack{\mathbb{Y}_L(\y)\mathbb{Y}_M(\y')\mathbb{Y}_R = \mathbb{X} }} f_1(\y,\mathbb{Y}_M, \y')f_2(\z, \mathbb{X} * (\mathbb{Y}_L\cdot(\y)\cdot(\y')\cdot \mathbb{Y}_R),\z') + \sum_{\substack{\mathbb{Y}_L(\y)\mathbb{Y}_R = \mathbb{Y} }} f_1(\y,\mathbb{Y}_R, 0)f_2(\z, (\mathbb{Y}_L\cdot(\y)) ,\z') \\
&+ \sum_{\substack{\mathbb{Y}_L(\y)\mathbb{Y}_R = \mathbb{Y} }} f_1(0,\mathbb{Y}_L, \y)f_2(\z, \mathbb{X} *  ((\y)\cdot \mathbb{Y}_R) ,\z') - \sum_{\substack{\mathbb{Y}_L(\y)\mathbb{Y}_R = \mathbb{Y} }} f_1(0,\mathbb{Y}_R, 0)f_2(\z, \mathbb{X} *  (\mathbb{Y}_L\cdot(\y)) ,\z'),
\end{align*}
as claimed
\end{footnotesize}

\end{proof}





\begin{lem}\label{lem:sum-Bz}
For $\Psi_1\in \addmr(R) \cap R\cotimes V_{\strprty}$ and $\Psi_2 \in \ide{h}^{\vee}$, we have
\begin{align}
\begin{aligned}
&B_{\z0} + B_{0\z'} - B_{00}^z  \\
=&f_1(\z, \mathbb{Y} ,\z')f_2(\z,  \mathbb{X} ,\z') + f_1(\z, \mathbb{X} ,\z')f_2(\z,  \mathbb{Y} ,\z') \\
&+\sum_{\mathbb{X}_L\mathbb{X}_R = \mathbb{X}} f_1(\z,\mathbb{X}_L,0)f_2(\z,\mathbb{X}_R * \mathbb{Y},\z')
+\sum_{\mathbb{Y}_L\mathbb{Y}_R = \mathbb{Y}} f_1(\z,\mathbb{Y}_L,0)f_2(\z,\mathbb{X} * \mathbb{Y}_R,\z') \\
&+\sum_{\mathbb{X}_L\mathbb{X}_R = \mathbb{X}} f_1(0,\mathbb{X}_R,\z')f_2(\z,\mathbb{X}_L * \mathbb{Y},\z')
+\sum_{\mathbb{Y}_L\mathbb{Y}_R = \mathbb{Y}} f_1(0,\mathbb{Y}_R,\z')f_2(\z,\mathbb{X} * \mathbb{Y}_L,\z')\\
&-\sum_{\mathbb{X}_L\mathbb{X}_R = \mathbb{X}} f_1(0,\mathbb{X}_L,0)f_2(\z,\mathbb{X}_R * \mathbb{Y},\z')
-\sum_{\mathbb{Y}_L\mathbb{Y}_R = \mathbb{Y}} f_1(0,\mathbb{Y}_L,0)f_2(\z,\mathbb{X} * \mathbb{Y}_R,\z')
\end{aligned}
\end{align}
\end{lem}




\begin{proof}
By adjoint conidition Proposition \ref{prop:addmr-mould}. (\ref{enumi:ad-2}), we have
\begin{align*}
&B_{\z0} + B_{0\z'} - B_{00}^z\\
=&f_1(\z, \mathbb{Y} ,0)f_2(\z,  \mathbb{X} ,\z') + f_1(\z, \mathbb{X} ,0)f_2(\z,  \mathbb{Y} ,\z') \\
&+\sum_{\mathbb{X}_L\mathbb{X}_R = \mathbb{X}} f_1(\z,\mathbb{X}_L,0)f_2(\z,\mathbb{X}_R * \mathbb{Y},\z')
+\sum_{\mathbb{Y}_L\mathbb{Y}_R = \mathbb{Y}} f_1(\z,\mathbb{Y}_L,0)f_2(\z,\mathbb{X} * \mathbb{Y}_R,\z') \\
&+f_1(0, \mathbb{Y} ,\z)f_2(\z,  \mathbb{X} ,\z') + f_1(0, \mathbb{X} ,\z)f_2(\z,  \mathbb{Y} ,\z') \\
&+\sum_{\mathbb{X}_L\mathbb{X}_R = \mathbb{X}} f_1(0,\mathbb{X}_R,\z')f_2(\z,\mathbb{X}_L * \mathbb{Y},\z')
+\sum_{\mathbb{Y}_L\mathbb{Y}_R = \mathbb{Y}} f_1(0,\mathbb{Y}_R,\z')f_2(\z,\mathbb{X} * \mathbb{Y}_L,\z')\\
&-f_1(0, \mathbb{Y} ,0)f_2(\z,  \mathbb{X} ,\z') - f_1(0, \mathbb{X} ,0)f_2(\z,  \mathbb{Y} ,\z') \\
&-\sum_{\mathbb{X}_L\mathbb{X}_R = \mathbb{X}} f_1(0,\mathbb{X}_L,0)f_2(\z,\mathbb{X}_R * \mathbb{Y},\z')
-\sum_{\mathbb{Y}_L\mathbb{Y}_R = \mathbb{Y}} f_1(0,\mathbb{Y}_L,0)f_2(\z,\mathbb{X} * \mathbb{Y}_R,\z') \displaybreak[3]\\
=&f_1(\z, \mathbb{Y} ,\z')f_2(\z,  \mathbb{X} ,\z') + f_1(\z, \mathbb{X} ,\z')f_2(\z,  \mathbb{Y} ,\z') \\
&+\sum_{\mathbb{X}_L(\x)\mathbb{X}_R = \mathbb{X}} f_1(\z,\mathbb{X}_L,0)f_2(\z,((\x)\mathbb{X}_R) * \mathbb{Y},\z')
+\sum_{\mathbb{Y}_L(\y)\mathbb{Y}_R = \mathbb{Y}} f_1(\z,\mathbb{Y}_L,0)f_2(\z,\mathbb{X} * ((\y)\mathbb{Y}_R),\z')\\
&+\sum_{\mathbb{X}_L(\x)\mathbb{X}_R = \mathbb{X}} f_1(0,\mathbb{X}_R,\z')f_2(\z,(\mathbb{X}_L(\x)) * \mathbb{Y},\z')
+\sum_{\mathbb{Y}_L(\y)\mathbb{Y}_R = \mathbb{Y}} f_1(0,\mathbb{Y}_R,\z')f_2(\z,\mathbb{X} * (\mathbb{Y}_L(\y)),\z')\\
&-\sum_{\mathbb{X}_L(\x)\mathbb{X}_R = \mathbb{X}} f_1(0,\mathbb{X}_L,0)f_2(\z,((\x)\mathbb{X}_R) * \mathbb{Y},\z')
-\sum_{\mathbb{Y}_L(\y)\mathbb{Y}_R = \mathbb{Y}} f_1(0,\mathbb{Y}_L,0)f_2(\z,\mathbb{X} * ((\y)\mathbb{Y}_R),\z'),
\end{align*}
as claimed.
\end{proof}


\begin{lem}\label{lem:allsum}

We have

\begin{footnotesize}
\begin{align*}
&\vimo_{d_{\Psi_1}(\Psi_2)}(\z,\mathbb{X} * \mathbb{Y},\z') \\
=&f_1(\z, \mathbb{Y} ,\z')f_2(\z,  \mathbb{X} ,\z') + f_1(\z, \mathbb{X} ,\z')f_2(\z,  \mathbb{Y} ,\z')\\
&+\sum_{\mathbb{V}_L(\v)\mathbb{V}_M(\v')\mathbb{V}_R = (\z)\cdot \mathbb{X} \cdot (\z') } f_1(\v,\mathbb{V}_M,\v')f_2(\z,(\mathbb{V}_L\cdot (\v)\cdot(\v') \cdot \mathbb{V}_R) * \mathbb{Y},\z')\\
&+\sum_{\mathbb{V}_L(\v)\mathbb{V}_M(\v')\mathbb{V}_R = (\z)\cdot \mathbb{Y} \cdot (\z') } f_1(\v,\mathbb{V}_M,\v')f_2(\z,\mathbb{X} * (\mathbb{V}_L\cdot (\v)\cdot(\v') \cdot \mathbb{V}_R) ,\z')
\end{align*}
\end{footnotesize}
If $\v = \z$ (respectively, $\v = \z'$) in the running index of the above sums, we regard $\v$ as the first (respectively, last) component of $f_2$. 

\end{lem}


\begin{proof}

Combining Lemma \ref{lem:sum-B} and Lemma \ref{lem:sum-Bz}, we have
\begin{footnotesize}
\begin{align*}
&B_{\x0} + B_{0\x} + B_{\y0} + B_{0\y} - B_{00}  + B_{\z0} + B_{0\z'} - B_{00}^z\\
=&\sum_{\substack{\mathbb{X}_L(\x)\mathbb{X}_M(\x')\mathbb{X}_R = \mathbb{X} }} f_1(\x,\mathbb{X}_M, \x')f_2(\z, (\mathbb{X}_L\cdot(\x)\cdot(\x')\cdot \mathbb{X}_R) * \mathbb{Y},\z') + \sum_{\substack{\mathbb{X}_L(\x)\mathbb{X}_R = \mathbb{X} }} f_1(\x,\mathbb{X}_R, 0)f_2(\z, (\mathbb{X}_L\cdot(\x)) * \mathbb{Y},\z') \\
&+ \sum_{\substack{\mathbb{X}_L(\x)\mathbb{X}_R = \mathbb{X} }} f_1(0,\mathbb{X}_L, \x)f_2(\z, ((\x)\cdot \mathbb{X}_R) * \mathbb{Y},\z') - \sum_{\substack{\mathbb{X}_L(\x)\mathbb{X}_R = \mathbb{X} }} f_1(0,\mathbb{X}_R, 0)f_2(\z, (\mathbb{X}_L\cdot(\x)) * \mathbb{Y},\z')\\
&+\sum_{\substack{\mathbb{Y}_L(\y)\mathbb{Y}_M(\y')\mathbb{Y}_R = \mathbb{X} }} f_1(\y,\mathbb{Y}_M, \y')f_2(\z, \mathbb{X} * (\mathbb{Y}_L\cdot(\y)\cdot(\y')\cdot \mathbb{Y}_R),\z') + \sum_{\substack{\mathbb{Y}_L(\y)\mathbb{Y}_R = \mathbb{Y} }} f_1(\y,\mathbb{Y}_R, 0)f_2(\z, (\mathbb{Y}_L\cdot(\y)) ,\z') \\
&+ \sum_{\substack{\mathbb{Y}_L(\y)\mathbb{Y}_R = \mathbb{Y} }} f_1(0,\mathbb{Y}_L, \y)f_2(\z, \mathbb{X} *  ((\y)\cdot \mathbb{Y}_R) ,\z') - \sum_{\substack{\mathbb{Y}_L(\y)\mathbb{Y}_R = \mathbb{Y} }} f_1(0,\mathbb{Y}_R, 0)f_2(\z, \mathbb{X} *  (\mathbb{Y}_L\cdot(\y)) ,\z')\\
&+f_1(\z, \mathbb{Y} ,\z')f_2(\z,  \mathbb{X} ,\z') + f_1(\z, \mathbb{X} ,\z')f_2(\z,  \mathbb{Y} ,\z') \\
&+\sum_{\mathbb{X}_L(\x)\mathbb{X}_R = \mathbb{X}} f_1(\z,\mathbb{X}_L,0)f_2(\z,((\x)\mathbb{X}_R) * \mathbb{Y},\z')
+\sum_{\mathbb{Y}_L(\y)\mathbb{Y}_R = \mathbb{Y}} f_1(\z,\mathbb{Y}_L,0)f_2(\z,\mathbb{X} * ((\y)\mathbb{Y}_R),\z')\\
&+\sum_{\mathbb{X}_L(\x)\mathbb{X}_R = \mathbb{X}} f_1(0,\mathbb{X}_R,\z')f_2(\z,(\mathbb{X}_L(\x)) * \mathbb{Y},\z')
+\sum_{\mathbb{Y}_L(\y)\mathbb{Y}_R = \mathbb{Y}} f_1(0,\mathbb{Y}_R,\z')f_2(\z,\mathbb{X} * (\mathbb{Y}_L(\y)),\z')\\
&-\sum_{\mathbb{X}_L(\x)\mathbb{X}_R = \mathbb{X}} f_1(0,\mathbb{X}_L,0)f_2(\z,((\x)\mathbb{X}_R) * \mathbb{Y},\z')
-\sum_{\mathbb{Y}_L(\y)\mathbb{Y}_R = \mathbb{Y}} f_1(0,\mathbb{Y}_L,0)f_2(\z,\mathbb{X} * ((\y)\mathbb{Y}_R),\z') \displaybreak[3]\\
=
&f_1(\z, \mathbb{Y} ,\z')f_2(\z,  \mathbb{X} ,\z') + f_1(\z, \mathbb{X} ,\z')f_2(\z,  \mathbb{Y} ,\z')\\
&+\sum_{\substack{\mathbb{X}_L(\x)\mathbb{X}_M(\x')\mathbb{X}_R = \mathbb{X} }} f_1(\x,\mathbb{X}_M, \x')f_2(\z, (\mathbb{X}_L\cdot(\x)\cdot(\x')\cdot \mathbb{X}_R) * \mathbb{Y},\z') \\
&+ \sum_{\substack{\mathbb{X}_L(\x)\mathbb{X}_R = \mathbb{X} }} (f_1(\x,\mathbb{X}_R, 0) +f_1(0,\mathbb{X}_R,\z') - f_1(0,\mathbb{X}_R, 0) )f_2(\z, (\mathbb{X}_L\cdot(\x)) * \mathbb{Y},\z') \\
&+ \sum_{\substack{\mathbb{X}_L(\x)\mathbb{X}_R = \mathbb{X} }} (f_1(0,\mathbb{X}_L, \x) + f_1(\z,\mathbb{X}_L,0) - f_1(0,\mathbb{X}_L,0))f_2(\z, ((\x)\cdot \mathbb{X}_R) * \mathbb{Y},\z') \\
& +\sum_{\substack{\mathbb{Y}_L(\y)\mathbb{Y}_M(\y')\mathbb{Y}_R = \mathbb{X} }} f_1(\y,\mathbb{Y}_M, \y')f_2(\z, \mathbb{X} * (\mathbb{Y}_L\cdot(\y)\cdot(\y')\cdot \mathbb{Y}_R),\z') \\
&+ \sum_{\substack{\mathbb{Y}_L(\y)\mathbb{Y}_R = \mathbb{Y} }} (f_1(\y,\mathbb{Y}_R, 0) +f_1(0,\mathbb{Y}_R,\z') -  f_1(0,\mathbb{Y}_R, 0) )f_2(\z, (\mathbb{Y}_L\cdot(\y)) ,\z')    \\
&+ \sum_{\substack{\mathbb{Y}_L(\y)\mathbb{Y}_R = \mathbb{Y} }} (f_1(0,\mathbb{Y}_L, \y) + f_1(0,\mathbb{Y}_L, \y) - f_1(0,\mathbb{Y}_L, 0) )f_2(\z, \mathbb{X} *  ((\y)\cdot \mathbb{Y}_R) ,\z') \displaybreak[3]\\
=
&f_1(\z, \mathbb{Y} ,\z')f_2(\z,  \mathbb{X} ,\z') + f_1(\z, \mathbb{X} ,\z')f_2(\z,  \mathbb{Y} ,\z')\\
&+\sum_{\substack{\mathbb{X}_L(\x)\mathbb{X}_M(\x')\mathbb{X}_R = \mathbb{X} }} f_1(\x,\mathbb{X}_M, \x')f_2(\z, (\mathbb{X}_L\cdot(\x)\cdot(\x')\cdot \mathbb{X}_R) * \mathbb{Y},\z') \\
&+ \sum_{\substack{\mathbb{X}_L(\x)\mathbb{X}_R = \mathbb{X} }} f_1(\x,\mathbb{X}_R, \z')f_2(\z, (\mathbb{X}_L\cdot(\x)) * \mathbb{Y},\z') + \sum_{\substack{\mathbb{X}_L(\x)\mathbb{X}_R = \mathbb{X} }} f_1(\z,\mathbb{X}_L, \x) f_2(\z, ((\x)\cdot \mathbb{X}_R) * \mathbb{Y},\z') \\
& +\sum_{\substack{\mathbb{Y}_L(\y)\mathbb{Y}_M(\y')\mathbb{Y}_R = \mathbb{X} }} f_1(\y,\mathbb{Y}_M, \y')f_2(\z, \mathbb{X} * (\mathbb{Y}_L\cdot(\y)\cdot(\y')\cdot \mathbb{Y}_R),\z') \\
&+ \sum_{\substack{\mathbb{Y}_L(\y)\mathbb{Y}_R = \mathbb{Y} }} f_1(\y,\mathbb{Y}_R, \z')f_2(\z,\mathbb{X} * (\mathbb{Y}_L\cdot(\y)) ,\z')  + \sum_{\substack{\mathbb{Y}_L(\y)\mathbb{Y}_R = \mathbb{Y} }} f_1(\z,\mathbb{Y}_L, \y)f_2(\z, \mathbb{X} *  ((\y)\cdot \mathbb{Y}_R) ,\z')\\
=&f_1(\z, \mathbb{Y} ,\z')f_2(\z,  \mathbb{X} ,\z') + f_1(\z, \mathbb{X} ,\z')f_2(\z,  \mathbb{Y} ,\z')\\
&+\sum_{\mathbb{V}_L(\v)\mathbb{V}_M(\v')\mathbb{V}_R = (\z)\cdot \mathbb{X} \cdot (\z') } f_1(\v,\mathbb{V}_M,\v')f_2(\z,(\mathbb{V}_L\cdot (\v)\cdot(\v') \cdot \mathbb{V}_R) * \mathbb{Y},\z')\\
&+\sum_{\mathbb{V}_L(\v)\mathbb{V}_M(\v')\mathbb{V}_R = (\z)\cdot \mathbb{Y} \cdot (\z') } f_1(\v,\mathbb{V}_M,\v')f_2(\z,\mathbb{X} * (\mathbb{V}_L\cdot (\v)\cdot(\v') \cdot \mathbb{V}_R) ,\z'),
\end{align*}
\end{footnotesize}
as claimed. In the two-to-last equality above follows from the adjoint condition Proposition \ref{prop:addmr-mould}. (\ref{enumi:ad-2}).
\end{proof}


\begin{prop}\label{prop:nth-derivation}
For $ n \in \mathbb{Z}_{>0}$, it follows that
\begin{footnotesize}
\begin{align}
&\vimo_{d_{\Psi_1}^n(\Psi_2)} (\z, \mathbb{X} * \mathbb{Y}, \z') \notag\\
=&\sum_{n_1 + n_2 = n-1}\binom{n-1}{n_1} \vimo_{d_{\Psi_1}^{n_1}(\Psi_1)}(\z,\mathbb{X},\z')\vimo_{d_{\Psi_1}^{n_2}(\Psi_2)}(\z,\mathbb{Y},\z')+\sum_{n_1 + n_2 = n-1}\binom{n-1}{n_1} \vimo_{d_{\Psi_1}^{n_1}(\Psi_2)}(\z,\mathbb{X},\z')\vimo_{d_{\Psi_1}^{n_2}(\Psi_1)}(\z,\mathbb{Y},\z'). \label{eq:conclusion-vimo}
\end{align}
\end{footnotesize}
\end{prop}






\begin{proof}
We prove our claim by induction on $n$.
Put $n = 1$. Since Proposition \ref{prop:addmr-mould} (\ref{enumi:ad-3}) holds for $f_2$, Lemma \ref{lem:allsum} tuens to
\begin{align*}
\vimo_{d_{\Psi_1}(\Psi_2)}(0,\X{1}{r}* ,\Y{1}{l},z) = f_1(\z,\mathbb{X},\z') f_2(\z,\mathbb{Y},\z') +  f_1(\z,\mathbb{Y},\z') f_2(\z,\mathbb{X},\z'),
\end{align*}
and we complete the first claim.

We assume \eqref{eq:conclusion-vimo} holds for some positive integer $n = k \in \mathbb{Z}_{>0}$. Now we shall prove \eqref{eq:conclusion-vimo} for $n = k+1$.

By Lemma \ref{lem:allsum}, we have
\begin{footnotesize}
\begin{align*}
&\vimo_{d_{\Psi_1}^{k+1}(\Psi_2)} (\z, \mathbb{X} * \mathbb{Y}, \z') \displaybreak[3]\\
=&\vimo_{d_{\Psi_1}(d_{\Psi_1}^{k}(\Psi_2))} (\z, \mathbb{X} * \mathbb{Y}, \z') \displaybreak[3] \\
=&\vimo_{\Psi_1}(\z, \mathbb{Y} ,\z')\vimo_{d_{\Psi_1}^k(\Psi_2)}(\z,  \mathbb{X} ,\z') + \vimo_{\Psi_1}(\z, \mathbb{X} ,\z')\vimo_{d_{\Psi_1}^{k}(\Psi_2)}(\z,  \mathbb{Y} ,\z') \displaybreak[3] \\
&+\sum_{\mathbb{V}_L(\v)\mathbb{V}_M(\v')\mathbb{V}_R = (\z)\cdot \mathbb{X} \cdot (\z') } \vimo_{\Psi_1}(\v,\mathbb{V}_M,\v')\\
&\times\left\{  \sum_{n_1 + n_2 = k-1}\binom{k-1}{n_1} \vimo_{d_{\Psi_1}^{n_1}(\Psi_1)}(\z,\mathbb{V}_L\cdot (\v)\cdot (\v')\mathbb{V}_R, \z')\vimo_{d_{\Psi_1}^{n_2}(\Psi_2)}(\z,\mathbb{Y},\z') \right.\\
&\qquad\left. +\binom{k-1}{n_1} \vimo_{d_{\Psi_1}^{n_1}(\Psi_2)}(\z,\mathbb{V}_L\cdot (\v)\cdot (\v')\cdot \mathbb{V}_R,\z')\vimo_{d_{\Psi_1}^{n_2}(\Psi_1)}(\z,\mathbb{Y},\z')\right\}\\
&+\sum_{\mathbb{V}_L(\v)\mathbb{V}_M(\v')\mathbb{V}_R = (\z)\cdot \mathbb{Y} \cdot (\z') } \vimo_{\Psi_1}(\v,\mathbb{V}_M,\v')\\
&\times\left\{  \sum_{n_1 + n_2 = k-1}\binom{k-1}{n_1} \vimo_{d_{\Psi_1}^{n_1}(\Psi_1)}(\z,\mathbb{V}_L\cdot (\v)\cdot (\v')\mathbb{V}_R, \z')\vimo_{d_{\Psi_1}^{n_2}(\Psi_2)}(\z,\mathbb{X},\z') \right.\\
&\qquad\left. +\binom{k-1}{n_1} \vimo_{d_{\Psi_1}^{n_1}(\Psi_2)}(\z,\mathbb{V}_L\cdot (\v)\cdot (\v')\cdot \mathbb{V}_R,\z')\vimo_{d_{\Psi_1}^{n_2}(\Psi_1)}(\z,\mathbb{X},\z')\right\}\displaybreak[3]\\
=&\vimo_{\Psi_1}(\z, \mathbb{Y} ,\z')\vimo_{d_{\Psi_1}^k(\Psi_2)}(\z,  \mathbb{X} ,\z') + \vimo_{\Psi_1}(\z, \mathbb{X} ,\z')\vimo_{d_{\Psi_1}^{k}(\Psi_2)}(\z,  \mathbb{Y} ,\z')\\
&+\sum_{n_1 + n_2 = k-1} \binom{k-1}{n_1} \vimo_{d_{\Psi_1}^{n_1 + 1}(\Psi_1)} (\z,\mathbb{X},\z')\vimo_{d_{\Psi_1}^{n_2}(\Psi_2)} (\z,\mathbb{Y},\z')\\
&+\sum_{n_1 + n_2 = k-1} \binom{k-1}{n_1} \vimo_{d_{\Psi_1}^{n_1 + 1}(\Psi_2)} (\z,\mathbb{X},\z')\vimo_{d_{\Psi_1}^{n_2}(\Psi_1)} (\z,\mathbb{Y},\z')\\
&+\sum_{n_1 + n_2 = k-1} \binom{k-1}{n_1} \vimo_{d_{\Psi_1}^{n_1 }(\Psi_1)} (\z,\mathbb{X},\z')\vimo_{d_{\Psi_1}^{n_2+1}(\Psi_2)} (\z,\mathbb{Y},\z')\\
&+\sum_{n_1 + n_2 = k-1} \binom{k-1}{n_1} \vimo_{d_{\Psi_1}^{n_1}(\Psi_2)} (\z,\mathbb{X},\z')\vimo_{d_{\Psi_1}^{n_2+1}(\Psi_1)} (\z,\mathbb{Y},\z')\displaybreak[3]\\
=&\vimo_{\Psi_1}(\z, \mathbb{Y} ,\z')\vimo_{d_{\Psi_1}^k(\Psi_2)}(\z,  \mathbb{X} ,\z') + \vimo_{\Psi_1}(\z, \mathbb{X} ,\z')\vimo_{d_{\Psi_1}^{k}(\Psi_2)}(\z,  \mathbb{Y} ,\z')\\
&+\sum_{n_1 + n_2 = k} \binom{k-1}{n_1-1} \vimo_{d_{\Psi_1}^{n_1}(\Psi_1)} (\z,\mathbb{X},\z')\vimo_{d_{\Psi_1}^{n_2}(\Psi_2)} (\z,\mathbb{Y},\z')\\
&+\sum_{n_1 + n_2 = k} \binom{k-1}{n_1-1} \vimo_{d_{\Psi_1}^{n_1}(\Psi_2)} (\z,\mathbb{X},\z')\vimo_{d_{\Psi_1}^{n_2}(\Psi_1)} (\z,\mathbb{Y},\z')\\
&+\sum_{n_1 + n_2 = k} \binom{k-1}{n_1} \vimo_{d_{\Psi_1}^{n_1 }(\Psi_1)} (\z,\mathbb{X},\z')\vimo_{d_{\Psi_1}^{n_2}(\Psi_2)} (\z,\mathbb{Y},\z')\\
&+\sum_{n_1 + n_2 = k} \binom{k-1}{n_1} \vimo_{d_{\Psi_1}^{n_1}(\Psi_2)} (\z,\mathbb{X},\z')\vimo_{d_{\Psi_1}^{n_2}(\Psi_1)} (\z,\mathbb{Y},\z')\displaybreak[3]\\
=&\vimo_{\Psi_1}(\z, \mathbb{Y} ,\z')\vimo_{d_{\Psi_1}^k(\Psi_2)}(\z,  \mathbb{X} ,\z') + \vimo_{\Psi_1}(\z, \mathbb{X} ,\z')\vimo_{d_{\Psi_1}^{k}(\Psi_2)}(\z,  \mathbb{Y} ,\z')\\
&+\sum_{n_1 + n_2 = k}\left( \binom{k-1}{n_1-1} + \binom{k-1}{n_1} \right) \vimo_{d_{\Psi_1}^{n_1}(\Psi_1)} (\z,\mathbb{X},\z') \vimo_{d_{\Psi_1}^{n_2}(\Psi_2)} (\z,\mathbb{Y},\z')\\
&+\sum_{n_1 + n_2 = k} \left(\binom{k-1}{n_1-1} + \binom{k-1}{n_1}\right) \vimo_{d_{\Psi_1}^{n_1}(\Psi_2)} (\z,\mathbb{X},\z')\vimo_{d_{\Psi_1}^{n_2}(\Psi_1)} (\z,\mathbb{Y},\z')\\
=&\vimo_{\Psi_1}(\z, \mathbb{Y} ,\z')\vimo_{d_{\Psi_1}^k(\Psi_2)}(\z,  \mathbb{X} ,\z') + \vimo_{\Psi_1}(\z, \mathbb{X} ,\z')\vimo_{d_{\Psi_1}^{k}(\Psi_2)}(\z,  \mathbb{Y} ,\z')\\
&+\sum_{\substack{n_1 + n_2 = k \\ n_1>0}}\binom{k-1}{n_1} \vimo_{d_{\Psi_1}^{n_1}(\Psi_1)} (\z,\mathbb{X},\z') \vimo_{d_{\Psi_1}^{n_2}(\Psi_2)} (\z,\mathbb{Y},\z')\\
&+\sum_{\substack{n_1 + n_2 = k \\ n_1>0}} \binom{k-1}{n_1} \vimo_{d_{\Psi_1}^{n_1}(\Psi_2)} (\z,\mathbb{X},\z')\vimo_{d_{\Psi_1}^{n_2}(\Psi_1)} (\z,\mathbb{Y},\z')\\
=&\sum_{\substack{n_1 + n_2 = k }}\binom{k-1}{n_1} \vimo_{d_{\Psi_1}^{n_1}(\Psi_1)} (\z,\mathbb{X},\z') \vimo_{d_{\Psi_1}^{n_2}(\Psi_2)} (\z,\mathbb{Y},\z')\\
&+\sum_{\substack{n_1 + n_2 = k }} \binom{k-1}{n_1} \vimo_{d_{\Psi_1}^{n_1}(\Psi_2)} (\z,\mathbb{X},\z')\vimo_{d_{\Psi_1}^{n_2}(\Psi_1)} (\z,\mathbb{Y},\z'),
\end{align*}
\end{footnotesize}
as claimed. In the third equality, we use adjoint conidition Proposition \ref{prop:addmr-mould}. (\ref{enumi:ad-2}).

\end{proof}

\begin{prop}\label{prop:nth-derivation-2}
For $\Psi_1 \in \addmr(R)\cap \Fad(R) \cap V_{\strprty}$ and $\Psi_2 \in \AdDMR_0(R)$, we have

\begin{footnotesize}
\textcolor{red}{書き直し}
\begin{align}
&\vimo_{d_{\Psi_1}^n(\Psi_2)} (\z, \mathbb{X} * \mathbb{Y}, \z') \notag\\
=&\sum_{n_1 + n_2 = n}\binom{n}{n_1} \vimo_{d_{\Psi_1}^{n_1}(\Psi_1)}(\z,\mathbb{X},\z')\vimo_{d_{\Psi_1}^{n_2}(\Psi_2)}(\z,\mathbb{Y},\z')+\sum_{n_1 + n_2 = n}\binom{n}{n_1} \vimo_{d_{\Psi_1}^{n_1}(\Psi_2)}(\z,\mathbb{X},\z')\vimo_{d_{\Psi_1}^{n_2}(\Psi_1)}(\z,\mathbb{Y},\z'). \label{eq:conclusion-vimo-2}
\end{align}
\end{footnotesize}
\end{prop}

\begin{proof}
If $n = 1$, we have
\begin{align*}
\vimo_{d_{\Psi_1}(\Psi_2)}(\z,\mathbb{X} * \mathbb{Y},\z') =& \vimo_{\Psi_1}(\z,\mathbb{X},\z')\vimo_{\Psi_2}(\z,\mathbb{Y},\z') +  \vimo_{\Psi_2}(\z,\mathbb{X},\z')\vimo_{\Psi_1}(\z,\mathbb{Y},\z').\\
&+\vimo_{d_{\Psi_1}(\Psi_2)}(\z,\mathbb{X},\z')\vimo_{\Psi_2}(\z,\mathbb{Y},\z') +  \vimo_{\Psi_2}(\z,\mathbb{X},\z')\vimo_{d_{\Psi_1}(\Psi_2)}(\z,\mathbb{Y},\z').
\end{align*}
by Proposition \ref{prop:}
\end{proof}



\begin{prop}[= Lemma \ref{lem:essential}]
For a positive integer $n$, and $\Psi_1$ and $\Psi_2 \in \addmr(R)$, 
\begin{align*}
\begin{aligned}
\Delta_*(d_{\Psi_1}^n(\Psi_2)) =&
d_{\Psi_1}^n(\Psi_2) \otimes 1 + 1\otimes d_{\Psi_1}^n(\Psi_2)\\
&\quad+ \sum_{n_1 + n_2 = n-1} \binom{n-1}{n_1} d_{\Psi_1}^{n_1}(\Psi_1) \otimes d_{\Psi_1}^{n_2}(\Psi_2) + \binom{n-1}{n_1} d_{\Psi_1}^{n_1}(\Psi_2) \otimes d_{\Psi_1}^{n_2}(\Psi_1)
\end{aligned} % \label{eq:essential}
\end{align*}
\end{prop}
\end{comment}
















\ifdefined\isMainDocument
\else
    %\bibliographystyle{amsplain}
    %\bibliography{reference-adjoint}
    \end{document}
\fi

\ifdefined\isMainDocument
\else
    \documentclass[10pt,oneside,reqno]{amsart}
    \usepackage{xr}
    \externaldocument{../main} % Main の .aux を参照
    \usepackage{latexsym}
\usepackage{amscd}
\usepackage{amsmath}
\usepackage{amssymb}
\usepackage[mathscr]{euscript}
\usepackage{graphicx}
\usepackage{geometry}
\usepackage{mathtools}
\usepackage[all]{xy}
\usepackage[dvipsnames]{xcolor}
\usepackage[colorlinks,final,hyperindex]{hyperref}
\usepackage{tikz}
\usepackage{tikz-cd}
\usetikzlibrary{decorations.pathmorphing,decorations.markings,arrows,calc,shapes.geometric,arrows.meta,positioning}
\usepackage{enumerate}
\usepackage{multirow}

\usepackage[bb=dsserif]{mathalpha}
\usepackage{bm}

\newcommand{\QQ}{\ensuremath{\mathbb{Q}}}
\newcommand{\DD}{\ensuremath{\mathbb{D}}}
\newcommand{\CC}{\ensuremath{\mathbb{C}}}
\newcommand{\RR}{\ensuremath{\mathbb{R}}}
\newcommand{\ZZ}{\ensuremath{\mathbb{Z}}}
\newcommand{\NN}{\ensuremath{\mathbb{N}}}

\newcommand{\g}{\ensuremath{\mathfrak{g}}}
\renewcommand{\SS}{\ensuremath{\mathbb{S}}}

\newcommand{\id}{\ensuremath{\mathrm{id}}}

\makeatletter

\newcommand{\bbone}{\mathbb{1}}

\newcommand{\calA}{\mathcal{A}}
\newcommand{\calB}{\mathcal{B}}
\newcommand{\calC}{\mathcal{C}}
\newcommand{\calD}{\mathcal{D}}
\newcommand{\calE}{\mathcal{E}}
\newcommand{\calF}{\mathcal{F}}
\newcommand{\calG}{\mathcal{G}}
\newcommand{\calH}{\mathcal{H}}
\newcommand{\calI}{\mathcal{I}}
\newcommand{\calJ}{\mathcal{J}}
\newcommand{\calK}{\mathcal{K}}
\newcommand{\calL}{\mathcal{L}}
\newcommand{\calM}{\mathcal{M}}
\newcommand{\calN}{\mathcal{N}}
\newcommand{\calO}{\mathcal{O}}
\newcommand{\calP}{\mathcal{P}}
\newcommand{\calQ}{\mathcal{Q}}
\newcommand{\calR}{\mathcal{R}}
\newcommand{\calS}{\mathcal{S}}
\newcommand{\calT}{\mathcal{T}}
\newcommand{\calU}{\mathcal{U}}
\newcommand{\calV}{\mathcal{V}}
\newcommand{\calW}{\mathcal{W}}
\newcommand{\calX}{\mathcal{X}}
\newcommand{\calY}{\mathcal{Y}}
\newcommand{\calZ}{\mathcal{Z}}
\newcommand{\kk}{\Bbbk}

\newcommand{\scrY}{\mathscr{Y}}
\newcommand{\scrV}{\mathscr{V}}
\newcommand{\scrP}{\mathscr{P}}

\newcommand{\Rep}{\mathop{\mathrm{Rep}}\nolimits}
\newcommand{\alg}{\mathop{\mathrm{alg}}\nolimits}
\newcommand{\Span}{\mathop{\mathrm{Span}}\nolimits}
\newcommand{\proj}{\mathop{\mathrm{proj}}\nolimits}
\newcommand{\Hom}{\mathop{\mathrm{Hom}}\nolimits}
\newcommand{\PHom}{\mathop{\mathrm{PHom}}\nolimits}
\newcommand{\Tw}{\mathop{\mathrm{Tw}}\nolimits}
\newcommand{\Dim}{\mathop{\mathrm{Dim}}\nolimits}
\newcommand{\Imm}{\mathop{\mathrm{Im}}\nolimits}
\newcommand{\Aus}{\mathop{\mathrm{Aus}}\nolimits}
\newcommand{\Ann}{\mathop{\mathrm{Ann}}\nolimits}
\newcommand{\APC}{\mathop{\mathrm{APC}}\nolimits}
\newcommand{\Barr}{\mathop{\mathrm{Bar}}\nolimits}
\newcommand{\Cone}{\mathop{\mathrm{Cone}}\nolimits}
\newcommand{\modu}{\mathop{\mathrm{mod}}\nolimits}
\newcommand{\dgMod}{\mathop{\mathrm{dgMod}}\nolimits}
\newcommand{\soc}{\mathop{\mathrm{soc}}\nolimits}
\newcommand{\dg}{\mathop{\mathrm{dg}}\nolimits}
\newcommand{\red}{\mathop{\mathrm{red}}\nolimits}
\newcommand{\Map}{\mathop{\mathrm{Map}}\nolimits}
\newcommand{\inj}{\mathop{\mathrm{inj}}\nolimits}
\newcommand{\op}{\mathop{\mathrm{op}}\nolimits}
\newcommand{\Ker}{\mathop{\mathrm{Ker}}\nolimits}
\newcommand{\Tor}{\mathop{\mathrm{Tor}}\nolimits}
\newcommand{\Tot}{\mathop{\mathrm{Tot}}\nolimits}
\newcommand{\Ext}{\mathop{\mathrm{Ext}}\nolimits}
\newcommand{\sg}{\mathop{\mathrm{sg}}\nolimits}
\newcommand{\Sl}{\mathop{\mathfrak{sl}}\nolimits}
\newcommand{\nc}{\mathop{\mathrm{nc}}\nolimits}
\newcommand{\Tr}{\mathop{\mathrm{Tr}}\nolimits}
\newcommand{\Irr}{\mathop{\mathrm{Irr}}\nolimits}
\newcommand{\HC}{\mathop{\mathrm{HC}}\nolimits}
\newcommand{\HH}{\mathop{\mathrm{HH}}\nolimits}
\newcommand{\THH}{\mathop{\mathrm{TH}}\nolimits}
\newcommand{\Perf}{\mathop{\mathrm{Perf}}\nolimits}
\newcommand{\del}{\partial}
\newcommand{\ev}{\mathop{\mathrm{ev}}\nolimits}
\newcommand{\co}{\mathop{\mathrm{co}}\nolimits}
\newcommand{\pt}{\mathop{\mathrm{pt}}\nolimits}
\newcommand{\wt}{\mathop{\mathrm{wt}}\nolimits}
\newcommand{\pCY}{\mathop{\mathrm{pCY}}\nolimits}
\newcommand{\gr}{\mathop{\mathrm{gr}}\nolimits}
\newcommand{\coker}{\mathop{\mathrm{coker}}\nolimits}


\newcommand{\Top}{\mathop{\mathrm{Top}}\nolimits}
\newcommand{\Set}{\mathop{\mathrm{Set}}\nolimits}
\newcommand{\Vect}{\mathop{\mathrm{Vect}}\nolimits}
\newcommand{\Mod}{\mathop{\mathrm{Mod}}\nolimits}
\newcommand{\Ob}{\mathop{\mathrm{Ob}}\nolimits}
\newcommand{\Ind}{\mathop{\mathrm{Ind}}\nolimits}
\newcommand{\Sym}{\mathop{\mathrm{Sym}}\nolimits}
\newcommand{\Alt}{\mathop{\mathrm{Alt}}\nolimits}
\newcommand{\End}{\mathop{\mathrm{End}}\nolimits}
\newcommand{\Ch}{\mathop{\mathrm{Ch}}\nolimits}


\newcommand{\Ass}{\mathop{\mathrm{Ass}}\nolimits}
\newcommand{\Com}{\mathop{\mathrm{Com}}\nolimits}
\newcommand{\Lie}{\mathop{\mathrm{Lie}}\nolimits}
\newcommand{\colim}{\mathop{\mathrm{colim}}\nolimits}

\newcommand{\Op}{\mathop{\mathsf{Operads}}\nolimits}
\newcommand{\Diop}{\mathop{\mathsf{Dioperads}}\nolimits}
\newcommand{\PDiop}{\mathop{\mathsf{PDioperads}}\nolimits}
\newcommand{\Coop}{\mathop{\mathsf{Cooperads}}\nolimits}
\newcommand{\Codiop}{\mathop{\mathsf{Codioperads}}\nolimits}
\newcommand{\PCodiop}{\mathop{\mathsf{PCodioperads}}\nolimits}

\newcommand{\antish}{\ensuremath{\mbox{!`}}}


\tikzset{->-/.style={decoration={
			markings,
			mark=at position #1 with {\arrow{>}}},postaction={decorate}}}
\tikzset{-w-/.style={decoration={
			markings,
			mark=at position #1 with {\arrow{Stealth[fill=white,scale=1.4]}}},postaction={decorate}}}
\tikzset{->-/.default=0.65}
\tikzset{-w-/.default=0.65}
\tikzstyle{bullet}=[circle,fill=black,inner sep=0.5mm]
\tikzstyle{circ}=[circle,draw=black,fill=white,inner sep=0.5mm]
\tikzstyle{vertex}=[circle,draw=black,thick,inner sep=0.5mm]
\tikzstyle{dot}=[draw,circle,fill=black,minimum size=0.5mm,inner sep = 0mm, outer sep = 0mm]
\tikzset{darrow/.style={double distance = 4pt,>={Implies},->},
	darrowthin/.style={double equal sign distance,>={Implies},->},
	tarrow/.style={-,preaction={draw,darrow}},
	qarrow/.style={preaction={draw,darrow,shorten >=0pt},shorten >=1pt,-,double,double
		distance=0.2pt}}
\newcommand\tikcirc[1][2.5]{\tikz[baseline=-#1]{\draw[thick](0,0)circle[radius=#1mm];}}
\newcommand{\tikzfig}[1]{\begin{tikzpicture}[auto,baseline={([yshift=-.5ex]current bounding box.center)}]#1\end{tikzpicture}}
\usetikzlibrary{decorations.pathreplacing}

\usepackage{amsthm}
\usepackage{thmtools}
\usepackage[noabbrev,capitalize]{cleveref}
\newtheorem{theorem}{Theorem}
\newtheorem{lemma}[theorem]{Lemma}
\newtheorem{proposition}[theorem]{Proposition}
\newtheorem{corollary}[theorem]{Corollary}
\newtheorem{conjecture}[theorem]{Conjecture}
\newtheorem{Claim}[theorem]{Claim}

\newtheorem*{theorem*}{Theorem}

\theoremstyle{definition}
\newtheorem{definition}{Definition}
\newtheorem{example}{Example}
\newtheorem{cexample}{Counterexample}
\newtheorem{exercise}{Exercise}
\newtheorem{question}{Question}
\newtheorem{remark}{Remark}

\addtolength{\textheight}{1.75cm}
\addtolength{\voffset}{-0.5cm}
\addtolength{\footskip}{+0.6cm}
\geometry{margin=1.2in}


\usepackage[backend=biber, maxbibnames=99, bibencoding=utf8, doi=false, isbn=false, url=false, style=alphabetic]{biblatex}
\addbibresource{refs.bib}

    \begin{document}
\fi

\section{Computational data for $\dmr$, $\addmr$, $\addmr\cap\Fad$, and $\addmr \cap \Fad \cap V_{\strprty}$}\label{section:computational data}% Section名

In this appendix, we provide the lists of dimensions used in the main text.

\begin{center}
\begin{tabular}{c|ccccccccccccc}
\hline
$k$ & $0$ & $1$ & $2$ & $3$ & $4$ & $5$ & $6$ & $7$ & $8$ & $9$ & $10$ & $11$ & $\cdots$ \\
\hline
$\dim \dmr^{(k)}$ & $0$ & $0$ & $0$ & $1$ & $0$ & $1$ & $0$ & $1$ & $1$ & $1$ & $1$ &  & $\cdots$ \\
$\dim \addmr^{(k)}$ & $0$ & $0$ & $0$ & $0$ & $2$ & $2$ & $3$ & $3$ & $4$ & $5$ & $6$ & $7$ & $\cdots$ \\
$\dim \addmr^{(k)} \cap \Fad$
& $0$ & $0$ & $0$ & $0$ & $1$ & $0$ & $1$ & $0$ & $1$ & $1$ & $1$ & $1$ & $\cdots$ \\
$\dim \addmr^{(k)} \cap \Fad \cap V_{\strprty}$
& $0$ & $0$ & $0$ & $0$ & $1$ & $0$ & $1$ & $0$ & $1$ & $1$ & $1$ & $1$ & $\cdots$ \\
\end{tabular}
\end{center}



\ifdefined\isMainDocument
\else
    %\bibliographystyle{amsplain}
    %\bibliography{reference-adjoint}
    \end{document}
\fi




\begin{center}
\textbf{Acknowledgments}
\end{center}

The author would first like to express his deepest gratitude to Professor Minoru Hirose. Without his kindness and encouragement, the author would not have been able to establish the main theorem in this paper.  
He is also grateful to Dr. Nao Komiyama for giving him valuable advice on the theory of commutative power series.  
He would like to thank Dr. Annika Burmester, Prof. Henrik Bachmann, Prof. Hidekazu Furusho, Prof. David Jarossay, Dr. Iu-Iong, Ng, Prof. Koji Tasaka, and Dr. Khalef Yaddaden for their helpful comments on his paper.
The author thanks Takeshi Shinohara, Kosei Watanabe, and Ku-Yu Fan for reviewing the Main Theorem.
%\textcolor{white}{The author thanks Hidekazu Furusho for his insightful comment. Namely, “It is just an embedding. This work has no effect.”}


\bibliographystyle{amsplain}
\bibliography{reference-adjoint}


\end{document}


