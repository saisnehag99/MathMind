\newpage

%------------------------------------------------------------------
\section{Notation and preliminaries}\label{sec:notation}
%------------------------------------------------------------------

In this section we fix the global parameters and conventions
used throughout the paper.  
Local definitions are given in the sections
where they are first needed.

%------------------------------------------------------------------
\subsection{Paraboloid and normals}\label{subsec:paraboloid}
%------------------------------------------------------------------
We consider the three-dimensional paraboloid
\[
   \Sigma \;=\;
   \bigl\{\,(\xi,\tau)\in\mathbb R^{3}\times\mathbb R
           : \tau = |\xi|^{2}\bigr\}.
\]
Its outward unit normal
\begin{equation}\label{eq:norm}
   n(\xi)
   \;=\;
   \frac{(-2\xi,\,1)}{\sqrt{1+4|\xi|^{2}}},
   \qquad
   \xi\in\mathbb R^{3},
\end{equation}
will play a key role in the geometric analysis.

\begin{lemma}[angular bilipschitzness]\label{lem:bilip}
There exists an absolute constant \(c_{*}\in(0,1)\) such that
for any \(\xi,\eta\) with \(|\xi|,|\eta|\sim\lambda\) one has
\[
   c_{*}\,\angle(\xi,\eta)
   \;\le\;
   \angle\!\bigl(n(\xi),n(\eta)\bigr)
   \;\le\;
   c_{*}^{-1}\,\angle(\xi,\eta).
\]
In particular, for sufficiently large \(\lambda\) one may take
\(c_{*}=1/2\).
\end{lemma}

\begin{proof}
From \eqref{eq:norm},
\(
   n(\xi)=\dfrac{(-\xi,\tfrac12)}{|\xi|}
          \Bigl(1+O(\lambda^{-2})\Bigr)
\)
uniformly for \(|\xi|\sim\lambda\).
Hence
\(
   |n(\xi)-n(\eta)|
   = \dfrac{|\xi-\eta|}{|\xi|}\bigl(1+O(\lambda^{-2})\bigr)
   = \angle(\xi,\eta)\bigl(1+O(\lambda^{-2})\bigr).
\)
Since \(\angle(u,v)=|u-v|\,(1+O(\angle^{2}))\) for unit vectors,
we obtain the desired two-sided comparison with error
\(O(\lambda^{-2})\), which is absorbed by choosing \(c_{*}=1/2\).
\end{proof}

%------------------------------------------------------------------
\subsection{The scale $\lambda$ and associated radii}
\label{subsec:scales}
%------------------------------------------------------------------

Throughout the text we fix the frequency parameter
\begin{equation}\label{eq:lambda-def}
   \lambda \;\ge\; 2,
   \qquad
   D      \;=\; \lambda^{1/12}.
\end{equation}
We use three geometric quantities
\begin{equation}\label{eq:radii}
   r      \;=\; \lambda^{-2/3},\quad % angular radius of a cap
   \rho   \;=\; \lambda^{-1/2},\quad % tube cross-section
   \alpha \;=\; c_{0}\,r\,D^{1/2}
           = c_{0}\,\lambda^{-5/8},
   \qquad
   c_{0}=10^{-3}.
\end{equation}

\smallskip
\noindent
$r$ — angular radius of a cap,  
$\rho$ — cross-section of a “wave” tube,  
$\alpha$ — the basic small angle appearing in
Lemma~\ref{lem:A4}.

%------------------------------------------------------------------
\subsection{Decomposition into caps}\label{subsec:caps}
%------------------------------------------------------------------
In this subsection all angles are measured in the spherical (angular) metric. For a center \(\xi_{0}\in\mathbb{R}^{3}\)
and angular radius \(r>0\) set
\[
   \Theta(\xi_{0},r)
   \;=\;
   \bigl\{\xi:\ |\xi|\sim\lambda,\ \angle(\xi,\xi_{0})\le r\bigr\}.
\]
Below we use the scale \(r=\lambda^{-2/3}\).
Since the area of an angular cap on the sphere \(\{|\xi|=\lambda\}\) scales as
\(\mathrm{area}(\Theta)\simeq \pi(\lambda r)^2\), the total number of caps is of order
\begin{equation}\label{eq:num-caps}
  \#\Theta
  \;\asymp\;
  \frac{\mathrm{area}(S^{2}_\lambda)}{\mathrm{area}(\Theta)}
  \;=\;
  \frac{4\pi\lambda^{2}}{\pi(\lambda r)^{2}}
  \;\simeq\; \frac{4}{r^{2}}
  \;\sim\; \lambda^{4/3},
\end{equation}
which matches the standard angular discretization on the sphere \(\{|\xi|=\lambda\}\).

Let \(F(t,x)\) be a smooth function. For its spatial Fourier transform
\[
   \widehat{F}(t,\xi)=\int_{\mathbb{R}^{3}} F(t,x)\,e^{-2\pi i x\cdot\xi}\,dx
\]
we introduce the cap decomposition
\begin{equation}\label{eq:cap-decomposition}
   F
   \;=\;
   \sum_{\Theta} F_{\Theta},
   \qquad
   \widehat{F_{\Theta}}
   :=\widehat{F}\,\chi_{\Theta},
\end{equation}
where \(\chi_{\Theta}\) is a smooth cutoff of the cap \(\Theta\).



%------------------------------------------------------------------
\subsection{Working cylinder}\label{subsec:cylinder}
%------------------------------------------------------------------

All $L^{p}$ norms are taken on the set
\begin{equation}\label{eq:def-cylinder}
   Q_{\lambda}
   \;=\;
   \bigl\{(t,x)\in\mathbb R\times\mathbb R^{3}\bigm|
          |t|\le\tfrac12\lambda^{-3/2},\;
          |x|\le\tfrac12\lambda^{-1/2}\bigr\}.
\end{equation}

%------------------------------------------------------------------
\subsection{Conventions for estimates}\label{subsec:symbols}
%------------------------------------------------------------------

\begin{itemize}
\item
$A\lesssim B$ means $A\le C\,B$ with an absolute constant $C$ independent of $\lambda$ and $D$.
\item
$A\lesssim_{\delta} B$ — the constant $C$ may depend only on the fixed parameter $\delta>0$.
\item
$C_{\varepsilon}$ — the constant depends only on the chosen $0<\varepsilon\le10^{-2}$ (see~\S\ref{subsec:epsilon})
and is independent of $\lambda,D$; when necessary we write $C_{k,\varepsilon}$ to emphasize
dependence on a fixed integer $k$.
\end{itemize}

%------------------------------------------------------------------
\subsection{Logarithmic budget}\label{subsec:log-budget}
%------------------------------------------------------------------

Everywhere below the logarithmic factors of the form $\log^k\lambda$ and $\log^k D$
arise only from coverings with fixed multiplicity or from a finite number of
dyadic decompositions in scales/angles. The depth of such coverings and the overlap
multiplicity are bounded by an absolute constant independent of $\lambda$ and $D$. Therefore
all such factors can be absorbed into a universal constant $C$,
without affecting the powers of~$\lambda$ and~$D$ in the summary table~\S\ref{subsec:balance-table-final}.

In particular, this means that in the final statement of
Theorem~\ref{thm:main} (§\ref{subsec:main-theorem}) the exponents $\sigma_\lambda$
and $\sigma_D$ are written without $+\varepsilon$ add-ons: the standard factors
$\lambda^\varepsilon$ and $D^\varepsilon$ are indeed absent.

%------------------------------------------------------------------
\subsection{The parameter $\varepsilon$}\label{subsec:epsilon}
%------------------------------------------------------------------

Throughout the paper we \emph{fix} a single value
\[
   0<\varepsilon\le10^{-2}.
\]
All small exponents that appear are denoted by the same letter $\varepsilon$
(after a possible retuning within this budget). Their cumulative contribution
is accounted for in \S\ref{sec:balance}; the dependence of the final constant $C_\varepsilon$
is only on the chosen $\varepsilon$ (and fixed discrete parameters such as $k$),
but not on $\lambda$ or $D$.
