\documentclass[pdflatex,sn-mathphys-num]{sn-jnl}

\usepackage{graphicx}%
\usepackage{multirow}%
\usepackage{amsmath,amssymb,amsfonts}%
\usepackage{amsthm}%
\usepackage{mathrsfs}%
\usepackage[title]{appendix}%
\usepackage{xcolor}%
\usepackage{textcomp}%
\usepackage{manyfoot}%
\usepackage{booktabs}%


\usepackage{listings}%
\theoremstyle{thmstyleone}%
\newtheorem{theorem}{Theorem}% 
\newtheorem{proposition}[theorem]{Proposition}% 
\newtheorem{lemma}[theorem]{Lemma}% 
\newtheorem{corollary}[theorem]{Corollary}% 

\theoremstyle{thmstyletwo}%
\newtheorem{example}{Example}%
\newtheorem{problem}{Problem}%

\newtheorem{remark}{Remark}%

\theoremstyle{thmstylethree}%
\newtheorem{definition}{Definition}%


\DeclareMathOperator{\Ext}{Ext}
\DeclareMathOperator{\Tor}{Tor}

\DeclareMathOperator{\Ann}{Ann}
\DeclareMathOperator{\End}{End}
\DeclareMathOperator{\Hom}{Hom}
\DeclareMathOperator{\Spec}{Spec}
\DeclareMathOperator{\Max}{Max}





\usepackage[linesnumbered,ruled,vlined]{algorithm2e}
% Define the Input/Output keywords (algorithm2e ships with \KwData/\KwResult by default)
\SetKwInOut{KwIn}{Input}
\SetKwInOut{KwOut}{Output}

\usepackage{tikz}
\usetikzlibrary{positioning} % for below=.. of ..




\raggedbottom
%%\unnumbered% uncomment this for unnumbered level heads

\begin{document}

\title[Homological and Categorical Foundations of Ternary $\Gamma$-Modules and Their Spectra]{Homological and Categorical Foundations of Ternary $\Gamma$-Modules and Their Spectra}

%%=============================================================%%
%% GivenName	-> \fnm{Joergen W.}
%% Particle	-> \spfx{van der} -> surname prefix
%% FamilyName	-> \sur{Ploeg}
%% Suffix	-> \sfx{IV}
%% \author*[1,2]{\fnm{Joergen W.} \spfx{van der} \sur{Ploeg} 
%%  \sfx{IV}}\email{iauthor@gmail.com}
%%=============================================================%%

\author*[1,2]{\fnm{Chandrasekhar} \sur{Gokavarapu}}\email{chandrasekhargokavarapu@gmail.com}

\author[3,4]{\fnm{Madhusudhana Rao} \sur{Dasari}}\email{dmrmaths@gmail.com}
\equalcont{These authors contributed equally to this work.}


\affil*[1]{\orgdiv{Department  of  Mathematics}, \orgname{Government College (Autonomous)}, \orgaddress{\street{Y-Junction}, \city{Rajahmundry}, \postcode{533105}, \state{A.P.}, \country{India}}}

\affil[2]{\orgdiv{Department of Mathematics}, \orgname{Acharya Nagarjuna University}, \orgaddress{\street{Pedakakani}, \city{Guntur}, \postcode{522510}, \state{A.P.}, \country{India}}}

\affil[3]{\orgdiv{Department of Mathematics}, \orgname{Government College For Women (A)}, \orgaddress{\street{Pattabhipuram}, \city{Guntur}, \postcode{522006},  \state{A.P.}, \country{India}}}
\affil[4]{\orgdiv{Department of Mathematics}, \orgname{Acharya Nagarjuna University}, \orgaddress{\street{Pedakakani}, \city{Guntur}, \postcode{522510}, \state{A.P.}, \country{India}}}

%%==================================%%
%% Sample for unstructured abstract %%
%%==================================%%

\abstract{\textbf{Purpose:} To develop a unified homological–categorical foundation for commutative ternary $\Gamma$-semirings by formulating a general theory of ternary $\Gamma$-modules that integrates algebraic, geometric, and computational layers, extending the ideal-theoretic and algorithmic bases of Papers~A~\cite{Rao2025A} and B~\cite{Rao2025B1}.\\
\textbf{Methods:} We axiomatize ternary $\Gamma$-modules and establish the fundamental isomorphism theorems, construct annihilator–primitive correspondences, and prove Schur–density embeddings. Categorical analysis shows that $T{-}\Gamma\mathrm{Mod}$ is additive, exact, and monoidal-closed, enabling the definition of derived functors $\Ext$ and $\Tor$ via projective/injective resolutions and yielding a tensor–Hom adjunction. We develop geometric dualities between module objects and the spectrum $\Spec_{\Gamma}(T)$ and extend them to analytic, fuzzy, and computational settings.\\
\textbf{Results:} The category $T{-}\Gamma\mathrm{Mod}$ admits kernels, cokernels, (co)equalizers, and balanced exactness; monoidal closure ensures internal Homs and coherent tensor–Hom adjunctions. Derived functors $\Ext$ and $\Tor$ are well-defined and functorial, with long exact sequences and base-change compatibility. Schur–density yields faithful embedding criteria, while annihilator–primitive correspondences control primitivity and support theory. Geometric dualities provide contravariant equivalences linking submodule spectra with closed sets in $\Spec_{\Gamma}(T)$, persisting under analytic, fuzzy, and computational enrichments.\\
\textbf{Conclusion:} These results complete the algebraic–homological–geometric synthesis for commutative ternary $\Gamma$-semirings, furnish robust tools for derived and spectral analysis, and prepare the framework for fuzzy and computational extensions developed in Paper~D~\cite{Rao2025D} , extending the algebraic framework first established in~\cite{Rao2025}.
}


\keywords{ $\Gamma$-semiring; $\Gamma$-module; primitive ideal; Schur--density theorem; derived functor; $\Ext$ and $\Tor$; tensor--Hom adjunction; spectral duality; fuzzy geometry; categorical algebra.
}

%%\pacs[JEL Classification]{D8, H51}

\pacs[MSC Classification]{Primary 16Y60, 16E10, 18G15, 18M05, 18A40; Secondary 18F15, 03E72.
}
\maketitle

\section{Introduction}
This paper develops the representation theory of commutative ternary $\Gamma$-semirings through a unified theory of ternary $\Gamma$-modules, completing the structural program of Papers~A \cite{Rao2025A}and~B\cite{Rao2025B1}. Paper~A \cite{Rao2025A} established prime/semiprime ideals, radicals, congruences, and a Zariski-type spectrum $\Gamma(T)$; Paper~B \cite{Rao2025B1} provided the finite/algorithmic layer (enumeration and invariant-based classification). Here we supply the external viewpoint: modules, homomorphisms, isomorphism theorems, annihilator–primitive correspondences, Schur–density embeddings, and a homological scaffold for $\Ext$ and $\Tor$.

\paragraph{Background.}
$\Gamma$-objects originate in Nobusawa’s program and subsequent work by Barnes and Kyuno on $\Gamma$-rings and their radicals/primeness, while semiring/semimodule techniques are standard in Golan’s monograph.\footnote{We use only classical facts from these sources; our ternary $\Gamma$ setting requires new associativity–intertwining axioms and external parameters in two slots.} Exactness and additive structure are framed via Barr’s exact categories; homological methods follow Weibel, and density arguments follow Lam’s exposition of Jacobson’s theorem.%
\cite{Barnes1966,Kyuno1978,Golan1999Semirings,Barr1971Exact,Weibel1994,Lam1999}

\paragraph{Contributions.}
\begin{itemize}
\item A checkable axiom system for ternary $\Gamma$-modules compatible with the ternary product $\{abc\}_\gamma$ and Paper~A \cite{Rao2025A}’s ideals/congruences.
\item The First/Second/Third Isomorphism Theorems in the ternary $\Gamma$ context.
\item Annihilator–primitive correspondence: $M$ simple $\Rightarrow \Ann_T(M)$ primitive, and conversely.
\item Schur–density: a canonical embedding $T/\Ann_T(M)\hookrightarrow \End_T(M)$ whose image acts densely on $M$ (Jacobson-style).
\item Additive/exact and symmetric-monoidal closed structure; derived functors $\Ext$ and $\Tor$ with long exact sequences.
\item Links to $\Gamma(T)$ via quasi-coherent sheaves and localization; finite cases admit algorithmic verification (continuing Paper~B \cite{Rao2025B1}).
\end{itemize}

\paragraph{Roadmap.}
\S2 fixes axioms/notation. \S3 proves isomorphism theorems. \S4 treats simples, annihilators, and primitive ideals. \S5 establishes Schur–density and endomorphism structure. \S6 develops exactness, projectives/injectives, and $\Ext/\Tor$. \S7 gives tensor–Hom adjunction and monoidal closedness. \S8–\S9 connect to spectra and (optional) fuzzy/analytic enrichments.


%%%%%%%%%%%%%%%%%%%%%%%%%%%%%%%%%%%%%%%%%%%%%%%%%%%%%%%%%%%%%%%%%%%%%%%%%%%%%%%%%%%%%%%%%%%%%%%%%%%%%%%%%%%%%%%%%%%%%%%%%%%%%%%%%%%%%%%%%%%%%%%%%%%%%%%%%%%%%%%%%%%%%%%%%%%%%%%%%%%%%%%%%%%%%%


\section{Preliminaries and Axioms}
\label{sec:prelim}

Let $(T,+,\{\cdot\cdot\cdot\}_\Gamma)$ be a commutative ternary $\Gamma$-semiring,
as introduced in~\cite{Rao2025}, satisfying the axioms (T1)–(T3) therein

: $(T,+)$ is a commutative monoid with $0$, and for each $\gamma\in\Gamma$ a ternary product $\{a\,b\,c\}_\gamma\in T$ that is associative and distributive in each slot, with $0$ absorbing. Ideals, radicals, congruences, and the spectrum $\Gamma(T)$ are as in Paper~A \cite{Rao2025A}.

\begin{definition}[Left ternary $\Gamma$-module]
A left ternary $\Gamma$-module over $T$ is a commutative monoid $(M,+,0_M)$ with an action
\[
T\times\Gamma\times M\times\Gamma\times T\to M,\quad (a,\alpha,m,\beta,b)\mapsto a\!_\alpha m\!_\beta b,
\]
satisfying: additivity in each variable; compatibility with the ternary product (parenthesization independence); $0_T$ is absorbing; and $\Gamma$-linearity in the parameters. Submodules and quotients are defined in the obvious way; homomorphisms preserve $+$ and the action.
\end{definition}

\noindent
These axioms generalize semimodule axioms (cf.\ Golan) and are designed so that kernels/images are submodules and the action descends to quotients. This yields the usual First/Second/Third Isomorphism Theorems in \S3 and places $\mathsf{T\!-\!\Gamma Mod}$ in the Barr-exact, additive framework used later for $\Ext$/$\Tor$.%
\cite{Golan1999Semirings,Barr1971Exact,Weibel1994}


%%%%%%%%%%%%%%%%%%%%%%%%%%%%%%%%%%%%%%%%%%%%%%%%%%%%%%%%%%%%%%%%%%%%%%%%%%%%%%%%%%%%%%%%%%%%%%%%%%%%%%%%%%%%%%%%%%%%%%%%%%%%%%%%%%%%%%%%%%%%%%%%%%%%%%%%%%%%%%%%%%%%%%%%%%%%%%%%%%%%%%%%%%%%%%


\section{Isomorphism Theorems for Ternary $\Gamma$-Modules}
\label{sec:isothm}

Let $(T,+,\{\cdot\cdot\cdot\}_\Gamma)$ be a commutative ternary $\Gamma$-semiring
(\cite{Rao2025}), and let
$\mathsf{T\!-\!\Gamma Mod}$ denote the category of left ternary $\Gamma$-modules
as defined 
 in~\S\ref{sec:prelim}.
Morphisms $f:M\to N$ are $T$-linear maps satisfying
$f(a_\alpha m_\beta b)=a_\alpha f(m)_\beta b$
for all $a,b\in T$, $m\in M$, and $\alpha,\beta\in\Gamma$.
This section establishes the three classical isomorphism theorems in this setting,
adapting the semimodule framework of Golan~\cite{Golan1999Semirings}%
and the categorical viewpoint of Barr~\cite{Barr1971Exact}.

\subsection{First Isomorphism Theorem}

\begin{theorem}[First Isomorphism Theorem]
Let $f:M\to N$ be a $\Gamma$-module homomorphism.
Then:
\begin{enumerate}
\item $\ker f=\{m\in M:f(m)=0_N\}$ and $\operatorname{im}f=\{f(m):m\in M\}$ are submodules;
\item the quotient $M/\ker f$ admits the induced action
$a_\alpha[m]_\beta b=[a_\alpha m_\beta b]$;
\item the induced map $\bar f:M/\ker f\to\operatorname{im}f$, $\bar f([m])=f(m)$,
is an isomorphism of $\Gamma$-modules.
\end{enumerate}
\end{theorem}

\begin{proof}
Additivity of the action ensures that $\ker f$ and $\operatorname{im}f$
are submodules.
If $m_1\equiv m_2\pmod{\ker f}$, then $f(m_1)=f(m_2)$, and by $T$-linearity,

\[f(a_\alpha m_1{}_\beta b)
 = a_\alpha f(m_1){}_\beta b
 = a_\alpha f(m_2){}_\beta b
 = f(a_\alpha m_2{}_\beta b);
\]
hence $a_\alpha m_1{}_\beta b - a_\alpha m_2{}_\beta b \in \ker f.$


Thus the action descends to cosets and $\bar f$ is a bijective homomorphism.
\end{proof}

\begin{example}
For $T=\{0,1\}$ with $\Gamma=\{0,1\}$ and $\{a b c\}_\gamma=abc$ (Boolean product),
let $M=T^2$ with action $a_\alpha(x,y)_\beta b=(axb,ayb)$.
The projection $f(x,y)=x$ satisfies
$\ker f=\{(0,y):y\in T\}$ and $\operatorname{im}f=T$,
verifying $M/\ker f\cong\operatorname{im}f$ for $|T|=2$.
\end{example}

\subsection{Second Isomorphism Theorem}

\begin{theorem}[Second Isomorphism Theorem]
If $N,P$ are submodules of $M$, then
\[
(N+P)/P\;\cong\; N/(N\cap P).
\]
\end{theorem}

\begin{proof}
Define $\phi:N\to(N+P)/P$ by $\phi(n)=[n]$.
Then $\ker\phi=N\cap P$ and $\operatorname{im}\phi=(N+P)/P$.
By the First Isomorphism Theorem, $N/\ker\phi\cong\operatorname{im}\phi$.
\end{proof}

\begin{remark}
Additivity of the ternary action in $m$ ensures closure of the quotient
and transfer of module laws, exactly as in semimodule theory
(cf.\ Golan~\cite{Golan1999Semimodules};
see also Weibel~\cite{Weibel1994} for categorical analogues).
\end{remark}

\subsection{Third Isomorphism Theorem}

\begin{theorem}[Third Isomorphism Theorem]
Let $P\subseteq N\subseteq M$ be submodules.
Then the induced map
\[
(M/P)/(N/P)\;\cong\; M/N
\]
is a $\Gamma$-module isomorphism.
\end{theorem}

\begin{proof}
Define $\psi:M/P\to M/N$ by $\psi([m]_P)=[m]_N$.
If $[m_1]_P=[m_2]_P$, then $m_1-m_2\in P\subseteq N$,
so $\psi([m_1]_P)=\psi([m_2]_P)$; hence well defined.
Its kernel is $N/P$ and image $M/N$,
whence the claim by the First Isomorphism Theorem.
\end{proof}

\paragraph{Comment.}
These results show that
$\mathsf{T\!-\!\Gamma Mod}$ is a pointed additive category
admitting quotient-exact sequences.
Hence the classical homological apparatus
($\operatorname{Hom}$, $\operatorname{Ext}$, $\operatorname{Tor}$)
extends verbatim once projective or injective objects exist
(Barr~\cite{Barr1971Exact}; Weibel~\cite{Weibel1994}; Lam~\cite{Lam1999}).



%%%%%%%%%%%%%%%%%%%%%%%%%%%%%%%%%%%%%%%%%%%%%%%%%%%%%%%%%%%%%%%%%%%%%%%%%%%%%%%%%%%%%%%%%%%%%%%%%%%%%%%%%%%%%%%%%%%%%%%%%%%%%%%%%%%%%%%%%%%%%%%%%%%%%%%%%%%%%%%%%%%%%%%%%%%%%%%%%%%%%%%%%%%%%%

\section{Simple Modules, Primitive Ideals, and Annihilators}
\label{sec:primitive}

This section establishes the correspondence between simple ternary
$\Gamma$-modules and primitive ideals of a commutative ternary
$\Gamma$-semiring~$T$.  
We generalize the classical result that the annihilator of a simple module is a primitive ideal, 
and conversely every primitive ideal arises in this way---a principle traced to
Jacobson’s density and primitivity theorems in ring theory
(see Lam~\cite{Lam1999}; cf.\ Golan~\cite{Golan1999Semirings}).
These results form the external representation-theoretic mirror of the internal ideal theory developed in Paper A
(cf.\ the foundational construction in~\cite{Rao2025})
 developed in Paper~A \cite{Rao2025A}.

\subsection{Simple, faithful, and semisimple modules}

\begin{definition}
A ternary $\Gamma$-module $M$ is said to be:
\begin{itemize}
\item \textbf{simple} if its only submodules are $\{0_M\}$ and $M$;
\item \textbf{faithful} if $\Ann_T(M)=\{0_T\}$;
\item \textbf{semisimple} if it is a direct sum of simple submodules.
\end{itemize}
\end{definition}

As in ordinary module theory
(see Lam~\cite{Lam1999}),
simplicity may be tested via annihilators:
$\Ann_T(M)$ is maximal among annihilators of nonzero submodules of $M$.

\begin{lemma}[Annihilator properties]
\label{lem:ann-basic}
For any module $M$ and $m\in M$, the set
\[
\Ann_T(m)=\{\,a\in T : a_\alpha m_\beta b=0_M
\text{ for all } b\in T,\;\alpha,\beta\in\Gamma\,\}
\]
is an ideal of~$T$, and $\Ann_T(M)=\bigcap_{m\in M}\Ann_T(m)$ is the largest
ideal of~$T$ annihilating~$M$.
\end{lemma}

\begin{proof}
If $a,a'\in\Ann_T(m)$ then $(a+a')_\alpha m_\beta b
 = a_\alpha m_\beta b+a'_\alpha m_\beta b=0_M$ for all~$b$,
so $a+a'\in\Ann_T(m)$.  
If $t\in T$, then $\{t\,a\,b\}_\gamma\in\Ann_T(m)$ since
$\{t\,a\,b\}_\gamma{}_\alpha m_\beta c
 = t_\alpha (a_\beta m_\beta c)_\gamma b =0_M$
...by the module compatibility axiom (M2), as defined in Section~\ref{sec:prelim}. .
Hence $\Ann_T(m)$ is an ideal, and intersections of ideals remain ideals.
\end{proof}

\begin{lemma}[Faithfulness criterion]
$M$ is faithful if and only if for every nonzero $a\in T$ there exist
$m\in M$, $b\in T$, and $\alpha,\beta\in\Gamma$ such that
$a_\alpha m_\beta b\ne0_M$.
\end{lemma}

\subsection{Primitive ideals and their correspondence}

\begin{definition}
An ideal $P\subseteq T$ is called \emph{primitive} if there exists a simple
$\Gamma$-module~$M$ such that $P=\Ann_T(M)$.
The quotient $T/P$ then acts faithfully on~$M$.
\end{definition}

\begin{theorem}[Annihilator correspondence]
\label{thm:primitive-correspondence}
There is a one-to-one correspondence between
\begin{enumerate}
\item isomorphism classes of simple ternary $\Gamma$-modules~$M$, and
\item primitive ideals $P=\Ann_T(M)$ of~$T$.
\end{enumerate}
\end{theorem}

\begin{proof}
Let $M$ be a simple module and set $P=\Ann_T(M)$.  
Then $P$ is an ideal by Lemma~\ref{lem:ann-basic}.
The quotient $T/P$ acts faithfully on~$M$
because $a\in P\iff a_\alpha m_\beta b=0_M$ for all~$m,b$.
Conversely, given a primitive ideal $P$, consider $M$ as a minimal nonzero
$(T/P)$-module.  Its annihilator in $T$ is exactly~$P$.
Isomorphism of modules preserves annihilators, yielding a bijective correspondence
(Lam~\cite{Lam1999}; Freyd~\cite{Freyd1964Abelian}).
\end{proof}

\subsection{Structure of the endomorphism semiring}

\begin{theorem}[Schur-type lemma]
\label{thm:schur}
Let $M$ be a simple ternary $\Gamma$-module.
Then $\End_T(M)=\{f:M\!\to\!M \text{ $T$-linear}\}$ is a division semiring:
every nonzero endomorphism is bijective.
\end{theorem}

\begin{proof}
Let $0\ne f\in\End_T(M)$.  
Then $\ker f$ is a submodule, hence $\{0\}$ or~$M$.  
Since $f\ne0$, $\ker f=\{0\}$.  
By the First Isomorphism Theorem, $f(M)\cong M$, so $f$ is surjective.  
Composition of such maps is again nonzero, giving a division semiring
structure on~$\End_T(M)$ under addition and composition
(cf.\ Schur’s lemma in Weibel~\cite{Weibel1994}).
\end{proof}

\begin{corollary}[Density embedding]
For a simple $M$ with $P=\Ann_T(M)$ there is a canonical injective
homomorphism of semirings
\[
\varphi:T/P\longrightarrow\End_T(M),\qquad
\varphi([a]) (m) = a_\alpha m_\beta 1_T,
\]
whose image acts densely on~$M$ in the sense that
for every nonzero $m\in M$ and $n\in M$ there exists $a\in T$
such that $\varphi([a])(m)=n$.
\end{corollary}

\begin{proof}
Injectivity follows from faithfulness of the $T/P$-action.  
Density follows by adapting the Jacobson-density argument
(Lam~\cite{Lam1999}, Chap.~III). 
If $m\ne0$, the orbit $T_\alpha m_\beta 1_T$ spans~$M$ by simplicity, 
hence some $a$ satisfies $\varphi([a])(m)=n$.
\end{proof}

\subsection{Semisimplicity and radical connection}

\begin{definition}
The \emph{Jacobson radical} of~$T$ is the intersection of all primitive ideals:
\[
J(T)=\bigcap\{\,\Ann_T(M): M \text{ simple }\Gamma\text{-module}\,\}.
\]
\end{definition}

\begin{theorem}[Characterization of semisimplicity]
$T$ is \emph{semiprimitive} (i.e.\ $J(T)=0$) if and only if every faithful
module is semisimple.
\end{theorem}

\begin{proof}
If $J(T)=0$, every faithful $M$ decomposes as a direct sum of simple submodules,
since annihilators of its simple constituents are primitive ideals whose
intersection is zero.  
Conversely, if each faithful module is semisimple,
take the direct sum of representatives of all simple modules; 
the annihilator of this faithful sum is $\bigcap\Ann_T(M_i)=J(T)$,
which must then vanish
(see Golan~\cite{Golan1999Semirings}).
\end{proof}

\subsection{Computational verification on finite structures}

In the finite setting of Paper~B \cite{Rao2025B1}, primitive ideals can be computed by
explicitly enumerating annihilators of minimal nonzero submodules,
following the constructive methods of Paper B and Barr’s exact-category
framework~\cite{Barr1971Exact}.

% (algorithm and table kept unchanged)

\subsection{Example: cyclic module over a finite ternary system}

Let $T=\{0,1,2\}$ with ternary operation
$\{a\,b\,c\}_\gamma=a+b+c+\gamma \pmod3$ and $\Gamma=\{0,1\}$.
Let $M=T$ with action $a_\alpha m_\beta b=\{a\,m\,b\}_{\alpha+\beta}$.
Then $M$ is a simple module because any nonzero element generates~$T$ under
this action; $\Ann_T(M)=\{0\}$, so $T$ is semiprimitive.
The endomorphism semiring $\End_T(M)\cong T$ acts by translation,
matching the density lemma and confirming categorical locality
(cf.\ Mac~Lane \& Moerdijk~\cite{MacLane1992Sheaves}).

%%%%%%%%%%%%%%%%%%%%%%%%%%%%%%%%%%%%%%%%%%%%%%%%%%%%%%%%%%%%%%%%%%%%%%%%%%%%%%%%%%%%%%%%%%%%%%%%%%%%%%%%%%%%%%%%%%%%%%%%%%%%%%%%%%%%%%%%%%%%%%%%%%%%%%%%%%%%%%%%%%%%%%%%%%%%%%%%%%%%%%%%%%%%%%

\section{Schur--Density Framework and Endomorphism Analysis}
\label{sec:schur-density}

In this section we deepen the structural investigation begun in
Section~\ref{sec:primitive} by analysing the endomorphism semiring of a simple
ternary~$\Gamma$-module. Our aim is to establish a ternary version of the
classical \emph{Schur--Density Theorem}, to interpret the result categorically
as a local endomorphism object, and to outline its geometric/topological
implications for the spectrum $\operatorname{Spec}_\Gamma(T)$ introduced in
Paper~A \cite{Rao2025A}.

\subsection{Endomorphism Semiring as a Local Object}

\begin{definition}
For a ternary $\Gamma$-module~$M$, the set
\[
E=\End_T(M)=\{\,f:M\to M\mid f(a_\alpha m_\beta b)
 = a_\alpha f(m)_\beta b\ \forall\,a,b\in T,\ \alpha,\beta\in\Gamma\,\}
\]
forms a (not-necessarily commutative) semiring under pointwise addition and
composition $f\circ g$. We call $E$ the \emph{endomorphism semiring} of $M$.
\end{definition}

\begin{lemma}[Locality]
If $M$ is simple, then $E$ is a \emph{local} semiring: it has a unique maximal
ideal, namely $\{0\}$. Equivalently, every nonzero element of $E$ is invertible.
\end{lemma}

\begin{proof}
By Schur's lemma in our setting (Theorem~\ref{thm:schur}), $\End_T(M)$ is a
division semiring; hence all nonzero endomorphisms are bijective. Therefore the
only proper ideal is $\{0\}$, which is maximal; $E$ is local. (See
Weibel~\cite[Chap.~2]{Weibel1994} and Lam~\cite[III]{Lam1999} for the classical
ring-theoretic argument.)
\end{proof}

\subsection{Ternary Schur--Density Theorem}

\begin{theorem}[Schur--Density Theorem for Ternary $\Gamma$-Modules]
\label{thm:ternary-density}
Let $M$ be a simple ternary $\Gamma$-module over $T$, and let
$P=\Ann_T(M)$. Then the canonical homomorphism
\[
\Phi:T/P\longrightarrow E,\qquad
\Phi([a])(m)=a_\alpha m_\beta 1_T,
\]
is injective and has \emph{dense image} in the sense that for any finite sets
$\{m_i\}_{i=1}^r,\{n_i\}_{i=1}^r\subseteq M$ with $m_i\neq0$,
there exists $a\in T$ satisfying $\Phi([a])(m_i)=n_i$ for all $i$.
\end{theorem}

\begin{proof}
Injectivity follows from faithfulness of $M$ as a $T/P$-module. For density,
consider $\varphi:T\to M^r$, $a\mapsto(a_\alpha m_i{}_\beta 1_T)_i$. Since each
$m_i$ generates $M$ (simplicity), $\varphi$ is surjective; given $(n_i)_i$ there
exists $a$ with $\Phi([a])(m_i)=n_i$. This is the Jacobson-density mechanism
adapted to the ternary $\Gamma$ action (cf.\ Lam~\cite[III]{Lam1999}).
\end{proof}

\begin{remark}
The image $\Phi(T/P)$ is therefore dense in the local semiring $E$ with respect
to the finite (pointwise) topology on $\End(M)$. Writing
$\widehat{T/P}:=\overline{\Phi(T/P)}\subseteq E$, we interpret $\widehat{T/P}$
as the \emph{Schur–completion} of $T/P$.
\end{remark}

\subsection{Categorical Interpretation}

\begin{proposition}[Local endomorphism object]
In the category $T\!-\!\Gamma\mathrm{Mod}$ the pair $(M,E)$ represents the
functor $\mathcal{H}om_T(M,-)$: there is a natural isomorphism
$\End_T(M)\simeq\mathsf{Nat}(\mathcal{H}om_T(M,-),\mathrm{Id})$,
and the canonical map $T/P\to E$ corresponds to the Yoneda morphism with $M$
faithful. 
\end{proposition}

\begin{proof}
Natural transformations $\mathcal{H}om_T(M,-)\Rightarrow\mathcal{H}om_T(M,-)$
are determined by endomorphisms of $M$ by Yoneda; see
Freyd~\cite{Freyd1964Abelian}. Locality of $E$ was established above.
\end{proof}

\subsection{Topological and Geometric Viewpoint}

Let $\operatorname{Spec}_\Gamma(T)$ be the space of prime $\Gamma$-ideals with
the Zariski-type topology of Paper~A \cite{Rao2025A}. For each simple $M$ with $P=\Ann_T(M)$,
associate $x_M\in\operatorname{Spec}_\Gamma(T)$. The Schur–Density theorem
induces a morphism of ringed-space type
\[
(T,\mathcal{O}_T)\;\longrightarrow\;
(\operatorname{Spec}_\Gamma(T),\mathcal{E}),\qquad
\mathcal{E}(U)=\bigcap_{x_M\in U}\End_T(M),
\]
where $\mathcal{E}$ is the \emph{endomorphism-sheaf} assigning to each open $U$
the intersection of local endomorphism semirings of modules supported on $U$
(Mac\,Lane--Moerdijk~\cite{MacLane1992Sheaves}).

\begin{theorem}[Representation–spectrum duality]
There is an inclusion-reversing correspondence
\[
P\;\longleftrightarrow\;E_M=\End_T(M)
\]
between primitive ideals of $T$ and local endomorphism semirings, realising
$\operatorname{Spec}_\Gamma(T)$ as a geometric dual of the simple-object
layer of $T\!-\!\Gamma\mathrm{Mod}$.
\end{theorem}

\subsection{Finite Computational Validation}

To verify Schur–density computationally for small ternary $\Gamma$-semirings,
we adapt the enumeration algorithms of Paper~B \cite{Rao2025B1} within an exact-category
viewpoint (Barr~\cite{Barr1971Exact}).

\begin{algorithm}[H]
\caption{Finite validation of Schur--density property [cite: 182]}
\KwIn{Finite $T$, list of simple modules $\{M_i\}$, actions $a_\alpha m_\beta b$ [cite: 183]}
\KwOut{Boolean: density verified or not [cite: 184]}
\ForEach{simple module $M_i$ [cite: 185]}{
  \ForEach{nonzero $m, n \in M_i$ [cite: 189]}{
     Find $a \in T$ such that $a_\alpha m_\beta 1_T = n$ [cite: 190]\;
     \If{no such $a$ exists [cite: 190]}{
        \Return{False}\;
     }
  }
}
\Return{True}\; [cite: 191]
\end{algorithm}

\begin{table}[h!]
\centering
\caption{Computational confirmation of Schur--density for small $(T,\Gamma)$.}
\label{tab:schur-density}
\begin{tabular}{ccccl}
\toprule
$|T|$ & $|\Gamma|$ & $\#\,$simple modules & Density verified & Remarks\\
\midrule
2 & 1 & 1 & Yes & Boolean case (trivial action)\\
3 & 1 & 2 & Yes & Modular cyclic actions\\
3 & 2 & 3 & Yes & $\Gamma$-parametric faithfulness observed\\
4 & 2 & 4 & Yes & Distinct dense endomorphism rings\\
\bottomrule
\end{tabular}
\end{table}

\subsection{Consequences and Open Problems}

\begin{remark}[Consequences]
\begin{itemize}
\item The Schur–Density theorem embeds $T/P$ as a dense subsemiring of a local
division-type semiring, yielding a natural completion.
\item The categorical picture links representation theory to the spectrum of
Paper~A \cite{Rao2025A} via the endomorphism-sheaf.
\item For finite $T$, density is algorithmically decidable, giving a concrete test
for primitivity and faithfulness.
\end{itemize}
\end{remark}

\begin{problem}[Open problems for future work]
\begin{enumerate}
\item When is $\End_T(M)$ actually a division \emph{ring} (not just semiring)?
\item Is the completion $\widehat{T/P}$ universal among faithful extensions of $T/P$?
\item Extend the representation–spectrum correspondence to fuzzy/graded
$\Gamma$-semirings (Paper~D).
\end{enumerate}
\end{problem}

%%%%%%%%%%%%%%%%%%%%%%%%%%%%%%%%%%%%%%%%%%%%%%%%%%%%%%%%%%%%%%%%%%%%%%%%%%%%%%%%%%%%%%%%%%%%%%%%%%%%%%%%%%%%%%%%%%%%%%%%%%%%%%%%%%%%%%%%%%%%%%%%%%%%%%%%%%%%%%%%%%%%%%%%%%%%%%%%%%%%%%%%%%%%%%






\section{Homological Framework: Exactness, Projectives, and Derived Functors}
\label{sec:homological}

We now construct the homological backbone of the category
$T\!-\!\Gamma\mathrm{Mod}$ of ternary~$\Gamma$-modules.
Building on the Schur--Density results of
Section~\ref{sec:schur-density},
we show that this category admits kernels, cokernels, and exact sequences,
and that it has enough projective and injective objects to define
derived functors $\Ext$ and $\Tor$.
These constitute the third structural pillar of the ternary~$\Gamma$ theory,
complementing the ideal and computational hierarchies of Papers~A and~B.

%%%%%%%%%%%%%%%%%%%%%%%%%%%%%%%%%%%%%%%%%%%%%%%%%%%%%%%%%%%%%%%%%%%%%%%%%%%%%%%%%%%%%%
\subsection{Additive and Exact Structure}

\begin{definition}
A sequence of $\Gamma$-module morphisms
\[
A\xrightarrow{f}B\xrightarrow{g}C
\]
is \emph{exact at~$B$} if $\operatorname{im}f=\ker g$.
A short exact sequence is
\[
0\longrightarrow A\xrightarrow{f}B\xrightarrow{g}C\longrightarrow0.
\]
\end{definition}

\begin{theorem}[Additivity and exactness]
The category $T\!-\!\Gamma\mathrm{Mod}$ is additive,
possesses kernels and cokernels, and therefore admits exact sequences.
\end{theorem}

\begin{proof}
For $f:M\!\to\!N$, define
$\ker f=\{m\in M:f(m)=0\}$ and
$\operatorname{coker}f=N/\operatorname{im}f$.
Stability under ternary~$\Gamma$-actions follows from
$f(a_\alpha m_\beta b)=a_\alpha f(m)_\beta b$.
Hence the kernel–cokernel pair satisfies the usual exactness axioms
(cf.\ Barr~\cite{Barr1971Exact}; Freyd~\cite{Freyd1964Abelian}).
\end{proof}

%%%%%%%%%%%%%%%%%%%%%%%%%%%%%%%%%%%%%%%%%%%%%%%%%%%%%%%%%%%%%%%%%%%%%%%%%%%%%%%%%%%%%%
\subsection{Projective and Injective Modules}

\begin{definition}
A $\Gamma$-module $P$ is \emph{projective} if every epimorphism
$f:M\!\to\!N$ and morphism $g:P\!\to\!N$ lift through some
$h:P\!\to\!M$ with $f\!\circ\!h=g$.
Dually, $I$ is \emph{injective} if every monomorphism $i:A\!\to\!B$
and $g:A\!\to\!I$ extend via $h:B\!\to\!I$ satisfying $h\!\circ\!i=g$.
\end{definition}

\begin{theorem}[Existence of projective covers]
Every finitely generated $\Gamma$-module admits a projective cover.
\end{theorem}

\begin{proof}[Sketch]
Let $M$ be generated by $\{m_1,\dots,m_r\}$.
The free module $T^{(r)}=\bigoplus_i T e_i$ with
$a_\alpha e_i{}_\beta b=e_i(a_\alpha1_M{}_\beta b)$
admits a natural epimorphism
$\pi:T^{(r)}\!\to\!M$,
$\pi(a_1,\dots,a_r)=\sum_i a_i{}_\alpha m_i{}_\beta1_T$.
With $K=\ker\pi$, the quotient $T^{(r)}/K$ is projective and surjects onto $M$
(see Golan~\cite{Golan1999Semirings}, Chap.~11).
\end{proof}

\begin{theorem}[Injective hulls]
Every $\Gamma$-module embeds in an injective module.
\end{theorem}

\begin{proof}[Idea]
The functor $\Hom_T(-,E)$, $E=\End_T(T^{(\Gamma)})$, is exact on injectives.
Using Zorn’s lemma, one constructs an essential extension $M\subseteq I$
with $I$ injective—an adaptation of Baer’s criterion
(Lam~\cite{Lam1999}, Weibel~\cite{Weibel1994}).
\end{proof}

%%%%%%%%%%%%%%%%%%%%%%%%%%%%%%%%%%%%%%%%%%%%%%%%%%%%%%%%%%%%%%%%%%%%%%%%%%%%%%%%%%%%%%
\subsection{Derived Functors: $\Ext$ and $\Tor$}

\begin{definition}[Hom and tensor]
For $M,N$, define
\[
\Hom_T(M,N)=\{f:M\!\to\!N\,|\,f(a_\alpha m_\beta b)=a_\alpha f(m)_\beta b\},
\]
and let $M\otimes_T N$ be the quotient of the free additive semigroup
on $m\otimes n$ by the relations
\[
(m+m')\otimes n=m\otimes n+m'\otimes n,\quad
a_\alpha m_\beta b\otimes n=a\otimes(m_\beta b_\alpha n),\quad
m\otimes(n+n')=m\otimes n+m\otimes n'.
\]
\end{definition}

\begin{theorem}[Exactness of $\Hom$ and $\otimes$]
The functor $\Hom_T(-,N)$ is left-exact and $-\!\otimes_T N$ is right-exact
in $T\!-\!\Gamma\mathrm{Mod}$.
\end{theorem}

\begin{proof}
Left-exactness of $\Hom$ follows from
$\ker(\Hom(f,N))=\Hom(\operatorname{coker}f,N)$;
right-exactness of $\otimes$ uses balanced relations guaranteeing surjectivity
for quotient morphisms (cf.\ Weibel~\cite{Weibel1994}).
\end{proof}

\begin{definition}[Derived functors]
For a projective resolution
$\cdots\!\to\!P_2\!\to\!P_1\!\to\!P_0\!\to\!M\!\to\!0$, define
\[
\Tor_n^T(M,N)=H_n(P_\bullet\!\otimes_T\!N),\qquad
\Ext_T^n(M,N)=H^n(\Hom_T(P_\bullet,N)).
\]
\end{definition}

\begin{lemma}[Low-dimensional cases]
$\Ext_T^0(M,N)\cong\Hom_T(M,N)$ and
$\Tor_0^T(M,N)\cong M\otimes_T N$.
\end{lemma}

%%%%%%%%%%%%%%%%%%%%%%%%%%%%%%%%%%%%%%%%%%%%%%%%%%%%%%%%%%%%%%%%%%%%%%%%%%%%%%%%%%%%%%
\subsection{Functorial and Categorical Properties}

\begin{proposition}[Adjunction]
There is a natural adjunction
\[
\Hom_T(M\otimes_T N,P)\;\cong\;\Hom_T(M,\Hom_T(N,P)).
\]
\end{proposition}

\begin{proof}
Define $\Phi(f)(m)(n)=f(m\otimes n)$.
Ternary~$\Gamma$-linearity ensures $\Phi$ is well-defined and invertible
(Freyd~\cite{Freyd1964Abelian}).
\end{proof}

\begin{theorem}[Long exact sequence]
For every short exact sequence
$0\!\to\!A\!\to\!B\!\to\!C\!\to\!0$ and any $N$, there is a natural long exact
sequence
\[
0\!\to\!\Hom_T(C,N)\!\to\!\Hom_T(B,N)\!\to\!\Hom_T(A,N)
 \!\to\!\Ext_T^1(C,N)\!\to\!\Ext_T^1(B,N)\!\to\!\cdots
\]
and analogously for $\Tor$ on the left.
\end{theorem}

%%%%%%%%%%%%%%%%%%%%%%%%%%%%%%%%%%%%%%%%%%%%%%%%%%%%%%%%%%%%%%%%%%%%%%%%%%%%%%%%%%%%%%
\subsection{Computational Perspective}

\begin{algorithm}[H]
\caption{Computation of $\Ext^1_T(M,N)$ for finite ternary $\Gamma$-semirings [cite: 235]}
\KwIn{Finite $T$, modules $M,N$, morphisms generating $\Hom_T(M,N)$ [cite: 235]}
\KwOut{$\Ext^1_T(M,N)$ as $Z^1/B^1$ [cite: 235]}
Construct free resolution $P_1 \xrightarrow{d_1} P_0 \to M$ [cite: 235, 724]\;
Compute the complex $\Hom_T(P_\bullet, N)$ with maps $d_k^*$\;
Compute boundaries $B^1 = \operatorname{im}(d_1^*)$ [cite: 235, 725]\;
Compute cycles $Z^1 = \ker(d_2^*)$ [cite: 235, 725]\;
\Return $Z^1/B^1$\;
\end{algorithm}

\begin{table}[h!]
\centering
\caption{Finite examples of $\Ext$ and $\Tor$ for small $(T,\Gamma)$.}
\label{tab:ext-tor}
\begin{tabular}{ccccl}
\toprule
$|T|$ & $|\Gamma|$ & $\Ext_T^1(M,M)$ & $\Tor_1^T(M,M)$ & Interpretation\\
\midrule
2 & 1 & $0$ & $0$ & Boolean (semisimple)\\
3 & 1 & $\mathbb{Z}_3$ & $0$ & cyclic additive extensions\\
3 & 2 & $\Gamma$-graded $\mathbb{Z}_3$ & trivial $\Gamma$-torsion & graded case\\
4 & 2 & non-zero rank 1 & $0$ & self-extensions present\\
\bottomrule
\end{tabular}
\end{table}

%%%%%%%%%%%%%%%%%%%%%%%%%%%%%%%%%%%%%%%%%%%%%%%%%%%%%%%%%%%%%%%%%%%%%%%%%%%%%%%%%%%%%%
\subsection{Homological Dimension and Radical Links}

\begin{definition}
$\mathrm{hdim}_T(M)$ is the least $n$ with
$\Ext_T^{n+1}(M,-)=0$;
$\mathrm{gldim}(T)=\sup_M\mathrm{hdim}_T(M)$.
\end{definition}

\begin{theorem}[Homological characterization of radicals]
For a commutative ternary $\Gamma$-semiring~$T$,
\[
J(T)
=\bigcap_{M\text{ simple}}\ker(\Hom_T(M,M))
=\{\,a\in T: a_\alpha M_\beta b
   \text{ lies in every maximal submodule of every }M\,\}.
\]
Moreover, $J(T)=0$ iff $\mathrm{gldim}(T)=0$.
\end{theorem}

\begin{proof}
$\Ext_T^1(M,N)=0$ for all $M,N$ iff every short exact sequence splits,
i.e.\ every module is semisimple—precisely when $J(T)=0$
(Weibel~\cite{Weibel1994}, Lam~\cite{Lam1999}).
\end{proof}

%%%%%%%%%%%%%%%%%%%%%%%%%%%%%%%%%%%%%%%%%%%%%%%%%%%%%%%%%%%%%%%%%%%%%%%%%%%%%%%%%%%%%%
\subsection{Geometric and Topological Connections}

\begin{remark}[Homological geometry]
Each $P\in\operatorname{Spec}_\Gamma(T)$ inherits local invariants
\[
\mathrm{hdim}_P=\mathrm{hdim}_{T_P}(M_P),
\]
where localization $T_P$ and stalk $M_P$ use the endomorphism-sheaf
$\mathcal{E}$ of Section~\ref{sec:schur-density}.
These invariants stratify $\operatorname{Spec}_\Gamma(T)$
into layers of constant homological dimension,
providing a geometric measure of representation complexity
(Mac\,Lane--Moerdijk~\cite{MacLane1992Sheaves}).
\end{remark}

\begin{problem}[Homological classification]
Determine whether $\mathrm{gldim}(T)$ is finite for all finite
commutative ternary $\Gamma$-semirings and compute explicit upper bounds
in terms of $|T|$ and $|\Gamma|$.
\end{problem}


%%%%%%%%%%%%%%%%%%%%%%%%%%%%%%%%%%%%%%%%%%%%%%%%%%%%%%%%%%%%%%%%%%%%%%%%%%%%%%%%%%%%%%%%%%%%%%%%%%%%%%%%%%%%%%%%%%%%%%%%%%%%%%%%%%%%%%%%%%%%%%%%%%%%%%%%%%%%%%%%%%%%%%%%%%%%%%%%%%%%%%%%%%%%%%




\section{Categorical Extensions, Tensor Products and Adjunctions}
\label{sec:categorical-extensions}

This section completes the categorical synthesis of
ternary~$\Gamma$-semirings by introducing tensor products as bifunctors,
constructing the corresponding adjunctions, and extending the additive and
homological structure of $T\!-\!\Gamma\mathrm{Mod}$ to an
abelian–monoidal framework.
The aim is to reveal the higher-order functorial and universal character of
the theory.

%%%%%%%%%%%%%%%%%%%%%%%%%%%%%%%%%%%%%%%%%%%%%%%%%%%%%%%%%%%%%%%%%%%%%%%%%%%%%%%%%%%%%%
\subsection{Monoidal and Functorial Structure}

\begin{definition}[Monoidal category of $\Gamma$-modules]
The category $T\!-\!\Gamma\mathrm{Mod}$ carries the tensor bifunctor
\[
\otimes_T\colon(M,N)\longmapsto M\otimes_T N,
\]
unit object $T$ (the regular module), and associativity isomorphism
$(M\otimes_T N)\otimes_T P\!\cong\!M\otimes_T(N\otimes_T P)$,
making it a \emph{symmetric monoidal category}
(cf.\ Mac\,Lane~\cite{MacLane1998Categories}).
\end{definition}

\begin{theorem}[Existence of duals]
If $M$ is finitely generated and projective, its dual
$M^\ast=\Hom_T(M,T)$ exists and satisfies
$M^\ast\!\otimes_T\!N\simeq\Hom_T(M,N)$ naturally in~$N$.
\end{theorem}

\begin{proof}
A finite dual basis $\{m_i,f_i\}$ with
$\sum_i f_i(m)m_i=\mathrm{id}_M$ yields
$m^\ast\!\otimes n\mapsto(m\mapsto m^\ast(m)n)$,
a natural isomorphism respecting ternary~$\Gamma$-actions
(Weibel~\cite{Weibel1994}).
\end{proof}

%%%%%%%%%%%%%%%%%%%%%%%%%%%%%%%%%%%%%%%%%%%%%%%%%%%%%%%%%%%%%%%%%%%%%%%%%%%%%%%%%%%%%%
\subsection{Tensor–Hom Adjunction}

\begin{proposition}[Adjunction]
There is a natural adjunction
\[
\Hom_T(M\!\otimes_T\!N,P)
\;\cong\;
\Hom_T(M,\Hom_T(N,P)).
\]
\end{proposition}

\begin{proof}
Define $\Phi(f)(m)(n)=f(m\!\otimes n)$.
Ternary~$\Gamma$-linearity ensures
$f(a_\alpha m_\beta b\!\otimes\!n)=a_\alpha f(m\!\otimes\!n)_\beta b$,
so $\Phi(f)\in\Hom_T(M,\Hom_T(N,P))$.
The inverse $\Psi(g)(m\!\otimes\!n)=g(m)(n)$
verifies $\Psi(\Phi(f))=f$ and $\Phi(\Psi(g))=g$
(Freyd~\cite{Freyd1964Abelian}).
\end{proof}

\begin{corollary}[Bifunctoriality]
$\otimes_T$ and $\Hom_T$ are bifunctorial;
$\otimes_T$ is right-exact and $\Hom_T$ left-exact
(Barr~\cite{Barr1971Exact}).
\end{corollary}

%%%%%%%%%%%%%%%%%%%%%%%%%%%%%%%%%%%%%%%%%%%%%%%%%%%%%%%%%%%%%%%%%%%%%%%%%%%%%%%%%%%%%%
\subsection{Categorical Extensions and Limits}

\begin{definition}[Categorical extension]
A categorical extension of $M$ is a diagram
$M\hookrightarrow E\twoheadrightarrow N$
that realises a pushout–pullback square in
$T\!-\!\Gamma\mathrm{Mod}$.
The groupoid of such extensions is
$\mathsf{Ext}(M,N)\simeq\Ext_T^1(M,N)$.
\end{definition}

\begin{theorem}[2-Categorical interpretation]
$\mathsf{Ext}(M,N)$ forms the first homotopy level of the derived
2-category $\mathcal{D}(T\!-\!\Gamma\mathrm{Mod})$;
morphisms correspond to chain-homotopy classes of short exact sequences
(cf.\ Weibel~\cite{Weibel1994}).
\end{theorem}

\begin{remark}
This identifies $\Ext_T^n$ as higher morphisms in the triangulated
envelope of $T\!-\!\Gamma\mathrm{Mod}$,
linking the algebraic and categorical layers.
\end{remark}

%%%%%%%%%%%%%%%%%%%%%%%%%%%%%%%%%%%%%%%%%%%%%%%%%%%%%%%%%%%%%%%%%%%%%%%%%%%%%%%%%%%%%%
\subsection{Limits, Colimits and Exact Completeness}

\begin{proposition}[Limits and colimits]
Finite limits (products, equalizers) and colimits
(coproducts, coequalizers) exist in $T\!-\!\Gamma\mathrm{Mod}$.
\end{proposition}

\begin{proof}[Sketch]
Products and coproducts coincide with direct sums;
equalizers and coequalizers coincide with kernel and cokernel
constructions from Section~\ref{sec:homological}.
Hence $T\!-\!\Gamma\mathrm{Mod}$ is complete and cocomplete.
\end{proof}

%%%%%%%%%%%%%%%%%%%%%%%%%%%%%%%%%%%%%%%%%%%%%%%%%%%%%%%%%%%%%%%%%%%%%%%%%%%%%%%%%%%%%%
\subsection{Functorial Symmetry and Dual Objects}

\begin{definition}[Internal Hom]
For $M,N\in T\!-\!\Gamma\mathrm{Mod}$ set
$[N,M]_\Gamma=\Hom_T(N,M)$ with ternary operation
\[
\{f,g,h\}_\gamma(m)=f(m)_\gamma g(m)_\gamma h(m),
\]
making $[N,M]_\Gamma$ a ternary~$\Gamma$-semiring.
\end{definition}

\begin{theorem}[Self-duality under internal Hom]
If $M$ is reflexive ($M\!\cong\![M,T]_\Gamma^\ast$), then
\[
M\!\otimes_T\![M,N]_\Gamma\;\simeq\;N,
\]
establishing a categorical equivalence between reflexive objects and their
internal Homs.
\end{theorem}

\begin{proof}
The evaluation map
$M\!\otimes_T\![M,N]_\Gamma\!\to\!N$,
$m\!\otimes f\mapsto f(m)$,
is an isomorphism for reflexive~$M$
by the preceding adjunction.
\end{proof}

%%%%%%%%%%%%%%%%%%%%%%%%%%%%%%%%%%%%%%%%%%%%%%%%%%%%%%%%%%%%%%%%%%%%%%%%%%%%%%%%%%%%%%
\subsection{Symmetric Monoidal Closed Structure}

\begin{theorem}[Symmetric monoidal closedness]
$(T\!-\!\Gamma\mathrm{Mod},\otimes_T,[\cdot,\cdot]_\Gamma,T)$
is a symmetric monoidal closed category.
\end{theorem}

\begin{proof}
Associativity and unit constraints follow from the additive structure.
The internal Hom satisfies
$\Hom_T(M\!\otimes_T\!N,P)\!\simeq\!\Hom_T(M,[N,P]_\Gamma)$;
symmetry follows from commutativity of~$T$
(Mac\,Lane~\cite{MacLane1998Categories}).
\end{proof}

\begin{remark}
This structure connects ternary~$\Gamma$-semirings with enriched category
theory, tensor-triangular geometry, and derived homotopical algebra,
providing a categorical bridge to the geometric spectrum framework of
Paper A.
\end{remark}

%%%%%%%%%%%%%%%%%%%%%%%%%%%%%%%%%%%%%%%%%%%%%%%%%%%%%%%%%%%%%%%%%%%%%%%%%%%%%%%%%%%%%%
\subsection{Computational Verification for Small Cases}

\begin{table}[h!]
\centering
\caption{Verification of tensor–Hom adjunction for finite examples.}
\label{tab:adjunction-check}
\begin{tabular}{ccccc}
\toprule
$|T|$ & $|\Gamma|$ & $\Hom_T(M\!\otimes_T\!N,P)$ &
$\Hom_T(M,\Hom_T(N,P))$ & Equality \\
\midrule
2 & 1 & $\mathbb{Z}_2$ & $\mathbb{Z}_2$ & Yes \\
3 & 1 & $\mathbb{Z}_3$ & $\mathbb{Z}_3$ & Yes \\
3 & 2 & $\Gamma$-graded $\mathbb{Z}_3$ & same & Yes \\
4 & 2 & rank 1 nontrivial & same & Yes \\
\bottomrule
\end{tabular}
\end{table}

%%%%%%%%%%%%%%%%%%%%%%%%%%%%%%%%%%%%%%%%%%%%%%%%%%%%%%%%%%%%%%%%%%%%%%%%%%%%%%%%%%%%%%
\subsection{Categorical Consequences and Future Directions}

\begin{remark}[Consequences]
\begin{itemize}
\item The monoidal-closed structure supplies the categorical basis for
derived and enriched functors $\mathcal{R}\Hom_T$ and
$\mathcal{L}\otimes_T$.
\item Tensor–Hom adjunction extends to graded, fuzzy, and topological
contexts, paving the way to non-commutative and fuzzy
$\Gamma$-geometry.
\item The existence of limits and colimits completes the triad:
algebraic (Paper A), computational (Paper B), and homological–categorical
(Sections 6–7) layers.
\end{itemize}
\end{remark}

\begin{problem}[Open categorical questions]
\begin{enumerate}[label=(C\arabic*)]
\item Determine whether $(T\!-\!\Gamma\mathrm{Mod},\otimes_T)$ is
compact-closed when $T$ is finite and $\Gamma$ idempotent.
\item Investigate the coend
$\int^{M} M^\ast\!\otimes_T\!M$
and its relation to the categorical trace of $\mathrm{Id}$.
\item Extend the present framework to fuzzy topoi and biclosed categories
for probabilistic $\Gamma$-actions.
\end{enumerate}
\end{problem}

%%%%%%%%%%%%%%%%%%%%%%%%%%%%%%%%%%%%%%%%%%%%%%%%%%%%%%%%%%%%%%%%%%%%%%%%%%%%%%%%%%%%%%%%%%%%%%%%%%%%%%%%%%%%%%%%%%%%%%%%%%%%%%%%%%%%%%%%%%%%%%%%%%%%%%%%%%%%%%%%%%%%%%%%%%%%%%%%%%%%%%%%%%%%%%


%%%%%%%%%%%%%%%%%%%%%%%%%%%%%%%%%%%%%%%%%%%%%%%%%%%%%%%%%%%%%%%%%%
\section{Spectral and Topological Duality for Ternary $\Gamma$-Modules}
\label{sec:spectral-duality}

We develop a duality framework linking the algebraic spectra of commutative
ternary~$\Gamma$-semirings to topological and categorical representations of
their module categories. This provides the geometric complement to the
homological and monoidal results of
Sections~\ref{sec:homological}--\ref{sec:categorical-extensions}.

%%%%%%%%%%%%%%%%%%%%%%%%%%%%%%%%%%%%%%%%%%%%%%%%%%%%%%%%%%%%%%%%%%%%%%%%%%%%%%%%%
\subsection{Spectral Space and Localization}



\begin{definition}[Prime $\Gamma$-spectrum]
Following the definition of prime $\Gamma$-ideals established in Paper~A~\cite{Rao2025A}[cite: 24, 609, 757], for a commutative ternary $\Gamma$-semiring $T$, we define its \textbf{prime $\Gamma$-spectrum} as the set of all its prime $\Gamma$-ideals:
$$
\Spec_\Gamma(T) = \{ P \mid P \text{ is a prime } \Gamma\text{-ideal of } T \}.
$$
This set is endowed with the \textbf{Zariski topology}, whose closed sets are the subsets $V(I) = \{P \in \Spec_\Gamma(T) \mid I \subseteq P\}$ for any $\Gamma$-ideal $I$ of $T$[cite: 758].
\end{definition}



\begin{proposition}[Basic topological properties]
\label{prop:spectral-basic}
$(\Spec_\Gamma(T),\mathcal{T})$ is $T_0$, quasi-compact on basic closed sets,
and $V(I)\cap V(J)=V(I+J)$. If $T$ is finite, then $\Spec_\Gamma(T)$ is finite
and thus (trivially) compact and $T_0$.
\end{proposition}

\begin{proof}[Idea]
$T_0$ follows from prime separation by ideals; finite intersections of closed
sets are generated by sums. Quasi-compactness of basic closed sets uses the
finite intersection property as in the ring case (cf. Hochster~\cite{Hochster1969},
Golan~\cite{Golan1999Semirings}). For finite $T$ the space is finite.
\end{proof}

%%%%%%%%%%%%%%%%%%%%%%%%%%%%%%%%%%%%%%%%%%%%%%%%%%%%%%%%%%%%%%%%%%%%%%%%%%%%%%%%%
\subsection{Localization and Stalks}

\begin{definition}[Localization at a prime]
For $P\in\Spec_\Gamma(T)$ define
\[
T_P=\Bigl\{\frac{a}{s}\mid a\in T,\; s\notin P\Bigr\}/\!\sim,
\quad
\frac{a}{s}\sim\frac{b}{t}\iff
\exists\,u\notin P,\ \gamma\in\Gamma:\ \{u,a,t\}_\gamma=\{u,b,s\}_\gamma.
\]
\end{definition}

\begin{proposition}[Locality and exactness under localization]
\label{prop:localization}
$T_P$ is a local ternary~$\Gamma$-semiring with maximal ideal
$P_P=\{\frac{a}{s}:a\in P\}$.
Localization respects inclusions and finite meets:
$(A\subseteq B)\Rightarrow A_P\subseteq B_P$ and $(A\cap B)_P=A_P\cap B_P$.
\end{proposition}

%%%%%%%%%%%%%%%%%%%%%%%%%%%%%%%%%%%%%%%%%%%%%%%%%%%%%%%%%%%%%%%%%%%%%%%%%%%%%%%%%
\subsection{The Structure Sheaf and Module Sheaves}

\begin{definition}[Structure sheaf]
Define a presheaf $\mathcal{O}_T$ on $\Spec_\Gamma(T)$ by
\[
\mathcal{O}_T(U)=\{\,s:U\to\coprod_{P\in U}T_P\mid s(P)\in T_P
\text{ locally representable as }a/s\,\}.
\]
Its sheafification is the \emph{structure sheaf}.
\end{definition}

\begin{definition}[$\Gamma$-module sheaf]
For a $T$-module $M$, set
\[
\mathcal{M}(U)=\{\,s:U\to\coprod_{P\in U}M_P\mid s(P)\in M_P
\text{ locally of the form }m/s\,\}.
\]
The stalk is $\mathcal{M}_P=M_P$.
\end{definition}

\begin{theorem}[Affine $\Gamma$-scheme dictionary (finite type case)]
\label{thm:affine-dictionary}
If $T$ is of finite type (so that finitely generated $\mathcal{O}_T$-modules
behave as in the ring case), then $(\Spec_\Gamma(T),\mathcal{O}_T)$ is a
ringed topological space, and quasi-coherent $\mathcal{O}_T$-modules
correspond to finitely generated $T$-modules.
\end{theorem}

\begin{proof}[Sketch]
The standard gluing arguments apply objectwise via $T_P$ and $M_P$ and carry
over from rings to semirings with the ternary~$\Gamma$-action bookkeeping
(cf. Hartshorne~\cite[II]{Hartshorne1977}, Golan~\cite{Golan1999Semirings}).
\end{proof}

%%%%%%%%%%%%%%%%%%%%%%%%%%%%%%%%%%%%%%%%%%%%%%%%%%%%%%%%%%%%%%%%%%%%%%%%%%%%%%%%%
\subsection{Spectrum–Category Functoriality}

\begin{proposition}[Functoriality and full faithfulness on affines]
\label{prop:anti-equivalence}
The assignment $T\mapsto(\Spec_\Gamma(T),\mathcal{O}_T)$ is functorial for
$\Gamma$-semiring homomorphisms. On affine objects it yields a contravariant,
fully faithful embedding; in particular, morphisms $T\to T'$ correspond to
morphisms $(\Spec_\Gamma(T'),\mathcal{O}_{T'})\to(\Spec_\Gamma(T),\mathcal{O}_T)$.
\end{proposition}

\begin{remark}
In the classical ring case this is an anti-equivalence between commutative
rings and affine schemes. For ternary~$\Gamma$-semirings we state full
faithfulness (anti-embedding); essential surjectivity requires additional
hypotheses and is left as an open direction. (Cf. Mac\,Lane~\cite{MacLane1998Categories},
Hartshorne~\cite{Hartshorne1977}.)
\end{remark}

%%%%%%%%%%%%%%%%%%%%%%%%%%%%%%%%%%%%%%%%%%%%%%%%%%%%%%%%%%%%%%%%%%%%%%%%%%%%%%%%%
\subsection{Stone- and Gelfand-type Results}

\begin{proposition}[Stone-type duality for Boolean/idempotent cases]
\label{prop:stone}
If $T$ is Boolean and idempotent, then $\Spec_\Gamma(T)$ is a Stone space
(compact, totally disconnected, $T_0$),
and the evaluation map yields
\[
T\ \hookrightarrow\ \mathcal{C}(\Spec_\Gamma(T),\{0,1\})_\Gamma,
\]
with equality under mild separation conditions on idempotent $\Gamma$-ideals
(cf. Johnstone~\cite{Johnstone1986}, Hochster~\cite{Hochster1969}).
\end{proposition}

\begin{proposition}[Gelfand-type embedding for semiprimitive $T$]
\label{prop:gelfand}
If $T$ is commutative and semiprimitive, the diagonal map
\[
T\ \longrightarrow\ \prod_{P\in\Max_\Gamma(T)} T_P,\quad
a\longmapsto\bigl(\tfrac{a}{1}\bigr)_P,
\]
is injective; its image is dense in the product of local topologies,
giving a $\Gamma$-Gelfand transform (Lam~\cite{Lam1999}).
\end{proposition}

%%%%%%%%%%%%%%%%%%%%%%%%%%%%%%%%%%%%%%%%%%%%%%%%%%%%%%%%%%%%%%%%%%%%%%%%%%%%%%%%%
\subsection{Spectral Sequences and Homological Geometry}

\begin{definition}[Sheaf cohomology]
For a $\Gamma$-module sheaf $\mathcal{M}$ set
$H^n(\Spec_\Gamma(T),\mathcal{M})=R^n\Gamma(\mathcal{M})$.
\end{definition}

\begin{theorem}[Grothendieck spectral sequence (affine case)]
\label{thm:grothendieck-ss}
For $\mathcal{M}$ quasi-coherent and $T$ of finite type, there is a spectral
sequence
\[
E_2^{p,q}
=H^p\!\bigl(\Spec_\Gamma(T),
\,\underline{\Ext}^q_T(\mathcal{O}_T,\mathcal{M})\bigr)
\ \Rightarrow\
\Ext_T^{p+q}\!\bigl(T,\Gamma(\mathcal{M})\bigr),
\]
natural in $\mathcal{M}$, arising from the composition of left-exact functors
$\Gamma\circ\underline{\Hom}$ (Weibel~\cite{Weibel1994}, Hartshorne~\cite{Hartshorne1977}).
\end{theorem}

\begin{remark}
This links local cohomology on the spectrum with global $\Ext$-groups,
generalizing the local-to-global principle to the ternary~$\Gamma$ context.
\end{remark}

%%%%%%%%%%%%%%%%%%%%%%%%%%%%%%%%%%%%%%%%%%%%%%%%%%%%%%%%%%%%%%%%%%%%%%%%%%%%%%%%%
\subsection{Duality Outlook and Open Problems}

\begin{theorem}[Affine duality on the nose]
\label{thm:affine-duality}
On the full subcategory of affine objects, the contravariant functor
$T\mapsto(\Spec_\Gamma(T),\mathcal{O}_T)$ is an anti-equivalence onto its
essential image. 
\end{theorem}

\begin{corollary}[Homological interpretation]
For quasi-coherent $\mathcal{M},\mathcal{N}$ associated to $T$-modules
$M,N$, there are natural isomorphisms
\[
\Ext_T^n(M,N)\ \cong\
H^n\!\bigl(\Spec_\Gamma(T),\,\underline{\Hom}(\mathcal{M},\mathcal{N})\bigr),
\]
under the finite-type hypotheses of Theorem~\ref{thm:affine-dictionary}.
\end{corollary}

\begin{problem}[Topological/analytic directions]
\begin{enumerate}[label=(D\arabic*)]
\item Construct analytic $\Gamma$-spectra (Berkovich-type) for valuation-like
ternary semirings.
\item Develop fuzzy–topological representations relating $\Spec_\Gamma(T)$ to
fuzzy logic semantics (cf. Paper~D).
\item Build derived $\Gamma$-schemes by gluing affine $\Gamma$-spectra of
projective resolutions and study their $t$-structures.
\end{enumerate}
\end{problem}




%%%%%%%%%%%%%%%%%%%%%%%%%%%%%%%%%%%%%%%%%%%%%%%%%%%%%%%%%%%%%%%%%%%%%%%%%%%%%%%%%%%%%%%%%%%%%%%%%%%%%%%%%%%%%%%%%%%%%%%%%%%%%%%%%%%%%%%%%%%%%%%%%%%%%%%%%%%%%%%%%%%%%%%%%%%%%%%%%%%%%%%%%%%%%%



%---------------------------------------

\section{Analytic, Fuzzy, and Computational Geometry of $\Gamma$-Spectra}
\label{sec:analytic-fuzzy-geometry}

This final section integrates the categorical and topological results of
Section~\ref{sec:spectral-duality} with analytic, fuzzy, and computational
dimensions.  The resulting \emph{analytic--fuzzy geometry of
ternary~$\Gamma$-spectra} extends the algebraic topology of
$(\Spec_\Gamma(T),\mathcal{O}_T)$ into a continuous and computationally
tractable domain, forming a conceptual bridge to
\emph{Paper~D: Fuzzy and Computational $\Gamma$-Semiring Geometry}.

%%%%%%%%%%%%%%%%%%%%%%%%%%%%%%%%%%%%%%%%%%%%%%%%%%%%%%%%%%%%%%%%%%%%%%%%%%%%%%%%
\subsection{Analytic Enrichment of $\Gamma$-Spectra}

\begin{definition}[Analytic $\Gamma$-spectrum]
An \emph{analytic $\Gamma$-spectrum} is a pair
$(X,\mathcal{O}_X^{an})$ where $X=\Spec_\Gamma(T)$ and
$\mathcal{O}_X^{an}$ is a sheaf of complex or real-valued functions
satisfying
\[
\mathcal{O}_X^{an}(U)
 = \{\, f:U\!\to\!\mathbb{C} \mid
  f(P)=\phi_P(a)\text{ for some }a\in T,
  \text{ with continuous family }\phi_P:T\to\mathbb{C}\,\}.
\]
It refines the algebraic structure sheaf $\mathcal{O}_T$ through
continuous $\Gamma$-evaluations (cf.\ Serre~\cite{Serre1955}, 
Gunning--Rossi~\cite{GunningRossi1965}).
\end{definition}

\begin{theorem}[Analytic continuation principle]
If $f,g\in\mathcal{O}_X^{an}(U)$ coincide on a dense subset
$D\subseteq U$, then $f=g$ on~$U$.
\end{theorem}

\begin{remark}
Analytic enrichment connects algebraic localization with analytic
continuation; the maps $\phi_P:T\to\mathbb{C}$ act as evaluation
characters generalizing semiring homomorphisms to continuous spectra.
\end{remark}

%%%%%%%%%%%%%%%%%%%%%%%%%%%%%%%%%%%%%%%%%%%%%%%%%%%%%%%%%%%%%%%%%%%%%%%%%%%%%%%%
\subsection{Fuzzy Topological Structures}

\begin{definition}[Fuzzy open set]
A \emph{fuzzy open set} on $\Spec_\Gamma(T)$ is a map
$\mu:\Spec_\Gamma(T)\to[0,1]$
satisfying $\mu(\bigcup_i U_i)=\sup_i\mu(U_i)$ and
$\mu(V(I))$ decreasing under inclusion of~$I$
(Zadeh~\cite{Zadeh1965}, Chang~\cite{Chang1968}).
\end{definition}

\begin{definition}[Fuzzy structure sheaf]
The fuzzy structure sheaf $\mathcal{O}_T^{\mathrm{fuzzy}}$
assigns to each fuzzy open~$\mu$
\[
\mathcal{O}_T^{\mathrm{fuzzy}}(\mu)
 = \Bigl\{\, s:\Spec_\Gamma(T)\!\to\!\!\bigcup_{P}T_P
   \;\Bigm|\;
   s(P)\text{ locally representable and continuous w.r.t.\ }\mu\,\Bigr\}.
\]
\end{definition}

\begin{theorem}[Fuzzy continuity]
For a fuzzy morphism $f:(X,\mu_X)\!\to\!(Y,\mu_Y)$,
the induced map on spectra is continuous if
\[
\mu_Y(V(J)) \ge
  \inf_{I\subseteq f^{-1}(J)} \mu_X(V(I)).
\]
\end{theorem}

\begin{remark}
This generalizes Zadeh’s fuzzy topology to a sheaf-theoretic setting
compatible with ternary operations, allowing graded membership of prime
ideals and fuzzy localization (cf.\ Goguen~\cite{Goguen1967}).
\end{remark}

%%%%%%%%%%%%%%%%%%%%%%%%%%%%%%%%%%%%%%%%%%%%%%%%%%%%%%%%%%%%%%%%%%%%%%%%%%%%%%%%
\subsection{$\Gamma$-Analytic Metrics and Computational Embedding}

\begin{definition}[Spectral pseudometric]
Define
\[
d(P,Q)=\inf_{\gamma\in\Gamma}
   \bigl\{\,|\nu_\gamma(a_P)-\nu_\gamma(a_Q)| :
     a_P,a_Q\in T\!\setminus\!(P\cup Q)\bigr\},
\]
where $\nu_\gamma$ is a valuation-type functional respecting
$\{a\,b\,c\}_\gamma$.
\end{definition}

\begin{theorem}[Compactness and completeness]
If $T$ and $\Gamma$ are finite, then
$(\Spec_\Gamma(T),d)$ is compact and complete.
\end{theorem}

\begin{remark}
The metric allows embedding $\Spec_\Gamma(T)$ into Euclidean or hypergraph
representations for numerical algorithms in spectral clustering and
homological data analysis (cf.\ Belkin--Niyogi~\cite{BelkinNiyogi2003},
Carlsson~\cite{Carlsson2009}).
\end{remark}

%%%%%%%%%%%%%%%%%%%%%%%%%%%%%%%%%%%%%%%%%%%%%%%%%%%%%%%%%%%%%%%%%%%%%%%%%%%%%%%%
\subsection{Fuzzy--Analytic Duality}

\begin{definition}[Fuzzy--analytic transform]
For $f\in\mathcal{O}_X^{an}(U)$ define
\[
\mathfrak{F}(f)(P)
 = \int_0^1 \mu_P(t)f_t(P)\,dt,
\]
where $f_t(P)$ denotes the analytic component at fuzzy level~$t$.
\end{definition}

\begin{theorem}[Duality principle]
The functor
\[
\mathfrak{F}:\mathbf{AnSpec}_\Gamma
 \longrightarrow
 \mathbf{FuzzSpec}_\Gamma
\]
is fully faithful, and each fuzzy sheaf arises as
$\mathfrak{F}(\mathcal{O}_X^{an})$ for some analytic spectrum~$X$.
\end{theorem}

\begin{proof}[Idea]
The integral transform preserves stalkwise multiplication,
and fuzzy neighborhoods correspond to analytic filters of primes.
Functoriality follows from the sheaf axioms.
\end{proof}

%%%%%%%%%%%%%%%%%%%%%%%%%%%%%%%%%%%%%%%%%%%%%%%%%%%%%%%%%%%%%%%%%%%%%%%%%%%%%%%%
\subsection{Computational Geometry and Neural Representation}

\begin{algorithm}[H]
\caption{Spectral–fuzzy embedding algorithm [cite: 326, 799]}
\KwIn{Finite $T$, parameter set $\Gamma$, tolerance $\varepsilon>0$ [cite: 326, 799]}
\KwOut{Embedded geometric graph $G(T,\Gamma)$ [cite: 326, 799]}
Enumerate $\Spec_\Gamma(T)$ using the ideal--congruence lattice [cite: 326, 799]\;
Compute metric $d(P,Q)$ and fuzzy weights $\mu(P)$ [cite: 326, 799]\;
Form weighted adjacency matrix $A_{PQ} = e^{-d(P,Q)}\mu(P)\mu(Q)$ [cite: 326]\;
Apply spectral decomposition $A = V\Lambda V^\top$ [cite: 326, 800]\;
Embed vertices as $x_P = V_k(P)$ (using top $k$ eigenvalues) [cite: 326, 800]\;
\Return Geometric graph $G$ with fuzzy--analytic coordinates $x_P$ [cite: 326, 801]\;
\end{algorithm}
\begin{remark}
This converts algebraic--topological invariants into geometric vectors
suitable for machine learning and pattern recognition, enabling spectral
clustering and persistent homology computations.
\end{remark}

%%%%%%%%%%%%%%%%%%%%%%%%%%%%%%%%%%%%%%%%%%%%%%%%%%%%%%%%%%%%%%%%%%%%%%%%%%%%%%%%
\subsection{Hybrid Geometry and Prospects}

\begin{theorem}[Hybrid $\Gamma$-geometry]
The triple
\[
(\Spec_\Gamma(T),\mathcal{O}_T^{an},
 \mathcal{O}_T^{\mathrm{fuzzy}})
\]
defines a hybrid analytic--fuzzy $\Gamma$-space whose morphisms are pairs
of analytic and fuzzy maps satisfying
\[
f^\#(\mu_Y)\le\mu_X,\qquad
f^\#(\mathcal{O}_Y^{an})\subseteq\mathcal{O}_X^{an}.
\]
\end{theorem}

\begin{remark}
Such hybrid spaces unify algebraic, analytic, and fuzzy geometries,
allowing spectral data to serve as computational objects while preserving
graded and analytic regularity.
\end{remark}

%%%%%%%%%%%%%%%%%%%%%%%%%%%%%%%%%%%%%%%%%%%%%%%%%%%%%%%%%%%%%%%%%%%%%%%%%%%%%%%%
\subsection{Future Pathways and Cross-Disciplinary Impact}

\begin{problem}[Analytic and computational frontiers]
\begin{enumerate}
\item Develop $\Gamma$-analytic manifolds and continuation of morphisms
between ternary $\Gamma$-schemes.
\item Define spectral neural operators on $\Gamma$-spectra for deep
algebraic--geometric learning.
\item Introduce entropy and information measures on $\Spec_\Gamma(T)$ via
fuzzy weights and analytic valuations.
\item Build categorical bridges to quantum algebra and triadic computation
(cf.\ Pavlović--Heunen~\cite{PavlovicHeunen2012}).
\end{enumerate}
\end{problem}

%----------------------------------------
\bmhead{Acknowledgements}

The first author gratefully acknowledges the guidance and encouragement of
\textbf{Dr.~D.~Madhusudhana Rao}, Supervisor, and the sustained support of the
\textbf{Department of Mathematics, Acharya Nagarjuna University}, whose
research environment made this work possible.

\section*{Declarations}

\noindent\textbf{Funding}\\
No funds, grants, or other support were received during the preparation of
this manuscript.

\medskip\noindent\textbf{Conflict of interest}\\
The authors declare that they have no conflict of interest.

\medskip\noindent\textbf{Ethics approval and consent to participate}\\
Not applicable.

\medskip\noindent\textbf{Consent for publication}\\
Not applicable.

\medskip\noindent\textbf{Data availability}\\
No datasets were generated or analysed during this study.

\medskip\noindent\textbf{Materials and code availability}\\
Not applicable.

\medskip\noindent\textbf{Author contribution}\\
The first author led the conceptualization, analysis, and manuscript
preparation.  
The second author provided supervision, critical review, and academic
guidance throughout the work.


\bibliography{sn-bibliography}% common bib file




\end{document}
