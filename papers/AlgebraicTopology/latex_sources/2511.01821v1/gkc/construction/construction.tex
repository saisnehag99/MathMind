\subsection{Kuranishi charts for buildings in symplectizations}\label{subsec:construction}
We restate our main theorems more precisely here and prove them in the following subsections. Let $\Gamma^+,\Gamma^-$ be finite collections of Reeb orbits of action $\leq  L$, and let $\beta\in H_2(Y,\Gamma^+\sqcup \Gamma^-)$ be a relative homology class. We define $T$ to be the decorated corolla as in Definition \ref{de:decorated-tree} with positive/negative exterior edges labeled by $\Gamma^+$ and $\Gamma^-$, respectively, and with degree $\beta$. We write $\Mbar^{\,J}(T) = \Mbar^{\,J}(\Gamma^+,\Gamma^-;\beta)$.

\begin{definition}\label{de:pre-perturbation} A \emph{pre-perturbation datum} $\fD = (\wt\lambda,\conn,p)$ for $\Mbar^J(T)$ consists of 
	\begin{itemize}[leftmargin=20pt]
		\item a $\cP(T)$-integral approximation $\wt\lambda$ of $\lambda$ as in Definition~\ref{de:approximation-in-symplectisation};
		\item a translation-invariant $J$-linear connection $\conn$ on $T\wh Y$;
		\item an integer $p\gg 1$.
        \end{itemize}
\end{definition}       

\noindent To such a pre-perturbation datum, we can associate the following spaces. Set
\begin{equation}\label{eq:auxiliary-degree}
    d' \coloneqq p\lbr{\s{\gamma\in \Gamma^+}\cA_{\wt\lambda}(\gamma)-\s{\gamma\in \Gamma^-}\cA_{\wt\lambda}(\gamma)}
\end{equation}
and $d \coloneqq d'-2$. Let $\cBR \coloneqq \cBR_{\Gamma^+,\Gamma^-}(d)$ be the smooth manifold with corners defined in \textsection\ref{subsec:framings} and define the groups
\begin{equation}\label{eq:covering-groups}
    G  \coloneqq \PU(d+1) \qquad \qquad \cG \coloneqq \PGL_{d+1}(\bC). 
\end{equation}
We let $\cZ =\cZ_{\wt\lambda}(T)$ be the family of curves over $\cBR$ defined in Definition~\ref{de:family-of-tree}. As before, let $\cC \to \cZ$ be the pullback of the universal family of $\cBR$.

\begin{definition}\label{de:auxiliary-datum} A \emph{perturbation datum} $\alpha = (\fD,\cU,\zeta,E,\mu)$ extending $\fD$ is the data of
        \begin{itemize}[leftmargin=20pt]
        \item a good covering $\cU = \{(U_i,\sigma_i,\chi_i)\}_{i\in I}$ on a subset of $\cZ$ and a $\cG$-equivariant map 
        \[\zeta \cl \cB^{\text{st}}_{3d'}(d)/S_{3d'}\to \cG/G; \]
		\item a finite-dimensional $G$-representation $E$ equipped with an equivariant linear map $\mu \cl E\to \cV$, where
        \begin{equation}\label{de:full-perturbation-space}
            \cV \coloneqq \set{\eta\in C^\infty(\cC^{\circ}\times \wh Y,\cc{\Hom}_\bC(\pr_1^*T_{\cC^{\circ}/\cBR},\pr_2^*T\wh Y))^\bR\mid \text{ supp}(\eta)/\bR \text{ is compact}},
        \end{equation} 
        $\cC^{\inn}$ denoting the complement of the special points of the fibers, so that for any $(\varphi,u)\in \cZ_{\delbar}$ with $\zeta_\cU(\varphi,u) = 0$, the Cauchy--Riemann operator 
    \[D\delbar_{J}(u) +\mu(\cdot)|_{\graph(\varphi,u)} \cl C^\infty_c(\dot{C},u^*T\wh Y)^\bR\oplus E_k \to \Omega^{0,1}_c(\dot{C},u^*T\wh Y)\]
    is surjective, where $\zeta_\cU$ is the map given by Lemma~\ref{lem:map-to-lie-algebra}.\end{itemize}
\end{definition}

\begin{theorem}\label{thm:pardon-gkc} Let $T$ be a decorated tree as at the beginning of the subsection.
	\begin{enumerate}[\normalfont 1),leftmargin=15pt,ref=\arabic*]
		\item\label{gkc-unobstructed-aux} Any pre-perturbation datum $\fD$ can be completed to a perturbation datum $\alpha = (\fD,\cU,\zeta,E.\mu)$ for $\Mbar^{\,J}(T)$ as in Definition~\ref{de:auxiliary-datum}.
		\item\label{gkc-rel-smooth} If $\alpha$ is a perturbation datum, then Construction~\ref{con:thickening} and Definition~\ref{con:obstruction} yield a rel--$C^1$ global Kuranishi chart with corners for $\Mbar^J(T)$.
		\item\label{gkc-orientation} If $\Gamma^\pm$ consists of good Reeb orbits, there exists a canonical isomorphism 
        \begin{equation}
\fo_{\cK_\alpha}\,\cong\,\fo(\bR)\dul\otimes\bigotimes\limits_{\gamma\in \Gamma^+}\fo_{\gamma}\otimes\bigotimes\limits_{\gamma\in \Gamma^-}\fo\dul_{\gamma}
        \end{equation} 
        of orientation lines, where $\fo_\gamma$ is the orientation line of the Reeb orbit, defined in \textsection\ref{subsec:orientation}.
		% \item\label{gkc-equivalent} If $\alpha_1,\alpha_2\in \cA(T,J)^{\reg}$, then $\cK_{\alpha_1}$ and $\cK_{\alpha_2}$ are equivalent global Kuranishi charts and there exists a canonical isomorphism $\fo_{\cK_{\alpha_1}}\cong \fo_{\cK_{\alpha_1}}$ which intertwines the isomorphisms of \eqref{gkc-orientation}
	\end{enumerate}
\end{theorem}

Before giving the construction of the thickening, Construction~\ref{con:thickening}, we have to briefly discuss Cauchy-Riemann operators on punctured Riemann surfaces. More details can be found in \textsection\ref{sec:gluing} or \cite[\textsection7]{Wen16}. Fix $k \geq 4$ and $0 < \delta < 1$ so that 
\begin{equation}\label{eq:small-delta}
    \delta< \mini{\gamma\in \cP(T)}{\inf|\sigma(A_\gamma)|}
\end{equation}
This is well-defined since $\cP(T)$ is finite and $\lambda$ is nondegenerate. For our purposes here, it suffices to consider the linearized Cauchy--Riemann equation without variation of the domain. Fix thus a possibly nodal Riemann surface $(C,\fj)$ with underlying graph contracting to $T$. Choose also a complex linear translation-invariant connection $\conn$ on $T\wh Y$ and a Riemannian metric on $\wh Y$. Then, we can associate to any smooth map $u \cl \dot{C}\to \wh Y$, which is $J$-holomorphic and asymptotic to a trivial cylinder near the punctures, the operator 
\begin{equation}
    D^{\conn}_u \cl W^{k,2,\delta}(C,u^*T\wh Y)\to W^{k-1,2,\delta}(\wt C,\Omega^{0,1}_{\wt C}\otimes_\bC u^*T\wh Y)
\end{equation}
given by the derivative of 
\begin{equation}
    \scF_u(\xi) = \Phi_{\exp_u(\xi) \to u}(d\exp_u(\xi)^{0,1}_{J,\fj}).
\end{equation}
where $\Phi$ is the parallel transport along sufficiently short geodesics. The operator $D_u$ is independent of the choice of connection and metric if $u$ is $J$-holomorphic. Since the exponential weight $\delta$ satisfies \eqref{eq:small-delta}, we have that $D_u$ is Fredholm by \cite[Lemma~7.10]{Wen16}
with index 
\begin{equation}
    \indo(D_u^\conn) = n\chi(C) -n(|\Gamma^+|+|\Gamma^-|) + 2c_1^\tau(u^*T\wh Y) + \s{\gamma\in \Gamma^+}{\mu^\tau_{\text{CZ}}(\gamma)}- \s{\gamma\in \Gamma^-}{\mu^\tau_{\text{CZ}}(\gamma)}.
\end{equation}


\begin{construction}[Thickening]\label{con:thickening} The thickening $\cT = \cT_\alpha$ consists of tuples $(\varphi,u,w)\in \cZ\times E_k$ where
\begin{enumerate}[label = \roman*),leftmargin=20pt,ref= \roman*]
    \item the image $\wch\varphi$ of $\varphi$ under the blow-down map $\cBR\to \cB$ satisfies 
		\begin{equation}
		    d_v \coloneqq\deg(\wch\varphi_v) =|D_v|-2+ p\,\lbr{\s{e' \in E^{\normalfont\text{int},+}_v}{\int\gamma_{v,e'}^*\wt\lambda}-\s{e' \in E^{\normalfont\text{int},+}_v}{\int\gamma_{v,e'}^*\wt\lambda}};,
		\end{equation} 
	for each $v\in V(T_\varphi)$, where $D_v$ is the divisor of special points on $C_v$ and $\gamma_{v,e}$ is the Reeb orbit to which $u_v$ converges at the puncture $z_{v,e}$;
    \item the matching isomorphism $m_e$ associated to $\varphi$ at the edge $e = (v,v')$ intertwines $u_v$ and $u_{v'}$ in the sense that $\wh u_{v'}\g m_{(v,v')} = \wh u_v$;
    \item the perturbation $w$ satisfies
    \begin{equation}
			\hpd_J\, u_v + \mu_{k}(w) |_{\graph(\varphi_v,u_v)}  = 0
		\end{equation}
		for each $v \in V(T_\varphi)$\footnote{By the assumption that elements of $\cV$ are invariant under translation, this equation is well-defined, i.e., does not depend on the choice of representative $u_v$.}, and the map
		\begin{equation}\label{eq:perturbations-suffice}
		     E \to \coker(D_u^\conn): w \mapsto [\mu_{k}(w) |_{\graph(\varphi,u)}]
		\end{equation}
		is surjective.
\end{enumerate}  
\end{construction}

We take the covering group to be 
\begin{equation}
    \wh G := G\times \p{\gamma\in \Gamma^+\sqcup\Gamma^-}S^1,
\end{equation}
where the torus acts by rotating the respective asymptotic marker. The thickening $\cT$ admits a continuous $\wh G$-equivariant map $\Pi\cl\cT\to \cBR$. In general, $\cT$ is not compact and $\Pi$ is not surjective. 

\begin{definition}[Obstruction bundle and section]\label{con:obstruction} We define the obstruction bundle $\cE = \cE_\alpha$ to be the trivial bundle 
	\begin{equation}
		\cE\coloneqq E\, \oplus\, \fp\fu 
	\end{equation}
	and define the (pre-)obstruction section $\fs_{pre} \cl \cT \to \cE$ by 
	\begin{equation}\label{} \
		\fs_{pre}(\varphi,u,w) = (w,\zeta_\cU(\varphi,u)). 
	\end{equation}
\end{definition}

\begin{remark}\label{} The projection of $\fs_{pre}$ to $\fp\fu$ is only continuous. We will replace this part of the obstruction section by an equivariant section with the same zero locus that is of class rel--$C^1$, see Lemma~\ref{lem:better-obstruction-section}.\end{remark}


\begin{proof}[Proof of Theorem \ref{thm:pardon-gkc}\hspace{0.2pt}\eqref{gkc-unobstructed-aux}]
Given the work done in \textsection\ref{sec:prelim-aux}, it remains to find a perturbation space $(E,\mu)$. Let us summarize why. By Lemma~\ref{lem:contact-approximation}, we can find a $\cP(T)$-integral approximation $\wt\lambda$, while the existence of a translation-invariant $J$-linear connection $\wh \conn$ on $T\wh Y$ is immediate. Since our curves have genus zero, any integer $p \gg 3$ is sufficiently large for $\fL_u := \eqref{eq:linebundle-construction}$ to be very ample for any $(\varphi,u)\in\cZ$ close to $\cZ_{\delbar = 0}$ and $H^1(C,\fL_u) = 0$. Good coverings, as in Definition~\ref{de:good-covering}, were constructed in \textsection\ref{subsec:domain-metrics}. To obtain the existence of $(E,\mu)$, we recall from \cite[Definition~4.1]{AMS23} that a \emph{finite-dimensional approximation scheme} of a smooth $G$-vector bundle $V\to B$ over a smooth $G$-manifold is a sequence $(E_k,\iota_k)$ of finite-dimensional $G$-representations with $G$-equivariant linear maps $\iota_k \cl E_k \to C^\infty_c(B,V)$ so that
 \begin{itemize}[leftmargin=20pt]
    \setlength\itemsep{2pt}
     \item $E_k \sub E_{k+1}$ is a sub-representation with $\iota_{k+1}|_{E_k} = \iota_k$,
     \item $\union{k \geq 1}{\im(\iota_k)}$ is dense in $C^\infty_c(B,V)$ in the $C^\infty_{\text{loc}}$-topology.
 \end{itemize}

\begin{lemma}\label{lem:invariant-approximations-exist}
    Finite-dimensional approximation schemes of $\cV$ exist.
\end{lemma}

\begin{proof}
    Apply \cite[Lemma~4.2]{AMS23} to the manifold $B \coloneqq \cC\inn\times \wh Y/\bR$ and the vector bundle $V \coloneqq \cc{\Hom}_\bC(\pr_1^*T_{\cC^{\circ}/\cBR},\bR\oplus\pr_2^*T(\wh Y/\bR))$.
\end{proof}


\begin{lemma}\label{lem:transversality-pointwise} Let $y = (\varphi,u) \in \cZ$ be arbitrary. Then there exists $k_y\geq 0$ so that 
	\begin{equation}
		E_{k_y} \to \coker(D_u) : w\mapsto [\mu_{k}(w) |_{\graph(\varphi,u)}]
	\end{equation}
	is surjective.
\end{lemma}


\begin{proof} This follows from the facts that $D_u$ is Fredholm and that an element in the cokernel is identically zero if it vanishes on an open subset of $C$. See also \cite[Proposition~3.26]{Par19}.
\end{proof}	


\begin{lemma}[Openness of transversality]\label{lem:regular-locus-open} Given $y\in \cZ_\alpha$ and $k_y$ as in Lemma \ref{lem:transversality-pointwise}, there exists a neighborhood $W_y \sub \cZ$ of $y$ so that for any $y' = (\varphi',u',w')\in W_y$ the map 
	\begin{equation}
		E_{k_y} \to \coker(D_{u'}) : w\mapsto [\mu_{k}(w') |_{\graph(\varphi',u')}]
	\end{equation}
	is surjective.
\end{lemma}

\begin{proof} Over a given stratum, this follows from the fact that regularity is an open condition. To see that it is also an open condition under gluing, refer to \cite[Lemma~5.8]{Par19} and the discussion loc. cit.
\end{proof}

\noindent Define the function $\wt k \cl \cZ \to \bN$ by 
	\[\wt k(\varphi,u) = \inf\{\ell \mid E_\ell \to \coker(D_u)\text{ is surjective}\}.\]
By Lemma \ref{lem:transversality-pointwise}, the function $\wt k$ is well-defined and, by Lemma \ref{lem:regular-locus-open}, it is upper semi-continuous. Thus it achieves a maximum $k$ on the compact set $\set{(\varphi,u)\in\cZ_{\delbar}\mid \zeta_\cU(\varphi,u) = 0}$. Setting $E = E_k$, we obtain the desired perturbation space.\end{proof}
