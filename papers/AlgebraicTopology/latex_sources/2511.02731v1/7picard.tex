\section{Relation to Picard groups} \label{sec:picard}

We employ the group actions constructed in the previous section to contribute to the computation of Picard groups of quivers over ring spectra ---in particular, of spectral Picard groups. Our main strategy consists of reducing these computations to the case of coefficients in a field, where we can leverage the results of \cite{MiyYek01}.

\begin{defi}
Let $(\mathcal{V},\otimes, \mathbb{1})$ be a monoidal $\infty$-category. An object $x \in \mathcal{V}$ is called \emph{$\otimes$-invertible} if there is $y \in \mathcal{V}$ such that $x \otimes y \simeq \mathbb{1} \simeq y \otimes x$. The \emph{Picard group} of $\mathcal{V}$ is
$$\mathsf{Pic}(\mathcal{V}) = \{x \in \mathcal{V} \mid x \text{ is $\otimes$-invertible}\}/\simeq,$$
i.e. the group of iso-classes of $\otimes$-invertible objects with multiplication $\otimes$ and unit $\mathbb{1}$.

For $A \in \mathsf{Alg}(\mathcal{C})$, we write $\mathsf{Pic}_\mathcal{C}(A) = \mathsf{Pic}({}_A\mathsf{BMod}_A(\mathcal{C}))$. For an $\mathbb{E}_\infty$-ring $R$, we also shorten $\mathsf{Pic}_R(A) = \mathsf{Pic}_{\Mod_R}(A)$ and $\mathsf{Pic}_R(Q) = \mathsf{Pic}_R(RQ)$
\end{defi}

\begin{rema}
If $\mathcal{V}$ is a monoidal $\infty$-category, then the homotopy category $h\mathcal{V}$ inherits the structure of a monoidal (1-)category and $\mathsf{Pic}(h\mathcal{V}) = \mathsf{Pic}(\mathcal{V})$.
\end{rema}

\begin{examples}
\begin{enumerate}
    \item It is a classical result \cite{HopMahSad94} that $\mathsf{Pic}(\Sp) \cong \mathbb{Z}$, generated by the suspension of the sphere spectrum $\Sigma \mathbb{S}$.
    \item For a commutative ring $R$, the computation in \cite{Fau03} gives $\mathsf{Pic}(\D{R}) \cong \mathsf{Pic}(R) \times \mathsf{Cont}(\mathrm{Spec}(R),\mathbb{Z})$, the first factor being the ordinary Picard group of $R$ and the second the additive group of continuous functions $\mathrm{Spec}(R) \to \mathbb{Z}$. In particular, it follows that $\mathsf{Pic}(\D{\mathbb{Z}}) \cong \mathbb{Z}$ generated by $\Sigma\mathbb{Z}$.
\end{enumerate}
\end{examples}

\begin{rema} \label{rema:eilenberg-watts} 
Let $A$ be an $R$-algebra spectrum. By a higher Eilenberg-Watts theorem (\Cref{example:Eilenberg-Watts}), $A$-bimodules over $R$ correspond to $R$-linear autofunctors of $\Mod_A$ via $M \mapsto - \otimes_A M$. Thus, denoting $\mathsf{Aut}_R(-) \subset \mathsf{Fun}^\mathsf{L}_R(-,-)$ the full subcategory of $R$-linear autoequivalences, one obtains an isomorphism $$\mathsf{Pic}_R(A) \cong \pi_0(\mathsf{Aut}_R(\Mod_A)^{\simeq}),$$ i.e. the Picard group is the group of $R$-linear autoequivalences.
\end{rema}

We say that an \emph{$\infty$-action} $\eta:\mathsf{B}G \to \infCAT$ of a group $G$ on an $\mathcal{E}$-linear $\infty$-category $\mathcal{C}$ is \emph{$\mathcal{E}$-linear} if, for each $g \in G$, the autoequivalence $\eta_g: \mathcal{C} \isoarrow \mathcal{C}$ is $\mathcal{E}$-linear, that is, if $\eta:\mathsf{B}G \to \infCAT$ factors through $\mathsf{Cat}_\mathcal{E} \hookrightarrow \infCAT$.

\begin{lemm} \label{lemm:linear-action}
Let $\mathcal{E}$ be a presentably monoidal stable $\infty$-category, and let $Q$ be a finite acyclic quiver. Then the action $\mathrm{Aut}_\mathsf{tr}(\ZQ) \ \rotatebox[origin=c]{-90}{$\circlearrowright$}\ \mathcal{E}^Q$ of \Cref{theo:actionZQ} is $\mathcal{E}$-linear.
\end{lemm}
\begin{proof}
We use \Cref{prop:closure-linear} repeatedly. First, $\mathcal{E}^Q$ is $\mathcal{E}$-linear as it is any functor $\infty$-category to $\mathcal{E}$ and the restriction functors between them. Since all $\infty$-categories and functors involved are $\mathcal{E}$-linear, the pullback \eqref{eq:ZQ-mesh} can be taken in $\mathsf{Cat}_\mathcal{E}$. This implies $\mathcal{E}^\ZQmesh$ and its forgetful functor $\mathcal{E}^\ZQmesh \to \mathcal{E}^\ZQ$ are $\mathcal{E}$-linear. Moreover, for $f \in \mathrm{Aut}_\mathsf{tr}(\ZQ)$, the isomorphism $\tilde{f^*}: \mathcal{E}^\ZQmesh \cong \mathcal{E}^\ZQmesh$ is induced by $\mathcal{E}$-linear functors, hence it is $\mathcal{E}$-linear too. Finally, the restriction $\mathcal{E}^\ZQmesh \to \mathcal{E}^Q$ is also $\mathcal{E}$-linear as it is a composition the $\mathcal{E}$-linears $\mathcal{E}^\ZQmesh \to \mathcal{E}^\ZQ \to \mathcal{E}^Q$. Then the proof is complete since the equivalences produced by the action are a composition of the isomorphisms $\tilde{f^*}: \mathcal{E}^\ZQmesh \cong \mathcal{E}^\ZQmesh$ with the restriction $\mathcal{E}^\ZQmesh \to \mathcal{E}^Q$ (or its inverse).
\end{proof}


\begin{rema}
Let $R$ be an $\mathbb{E}_\infty$-ring. Since suspension is always $R$-linear, we get combining \Cref{rema:eilenberg-watts} and \Cref{lemm:linear-action} that the action map \eqref{eq:action} defines a group homomorphism
$\mathrm{Aut}_{\mathsf{tr},\sigma}(\Gamma_{Q}^\mathrm{irr}) \xrightarrow{ \ \ } \mathsf{Pic}_{R}(Q)$.
\end{rema}

\begin{example} \label{example:pic-field} 
Let $k$ be a field and let $Q$ be a finite acyclic quiver. The main result of \cite[Theorem 3.8]{MiyYek01} asserts that the natural map 
$$q: \mathsf{Pic}_{\D{k}}(Q) \xrightarrow{ \ \ } \mathrm{Aut}^0_{\mathsf{tr},\sigma}(\Gamma_{Q}^\mathrm{irr}),$$ sending a complex of $kQ$-bimodules $T$ to its action on iso-classes of indecomposables $[M] \mapsto [M\otimes^\mathbb{L}_{kQ}T]$, is a split epimorphism of groups. Moreover, if $Q$ is a tree, then $q$ is an isomorphism (by \cite[Proposition 1.7(2)]{MiyYek01}), and $\mathrm{Aut}^0_{\mathsf{tr},\sigma}(\Gamma_{Q}^\mathrm{irr}) \cong \mathrm{Aut}_{\mathsf{tr},\sigma}(\Gamma_{Q}^\mathrm{irr})$. 

We observe now that the homomorphism provided by the action \eqref{eq:action-0},
$$\varphi: \mathrm{Aut}^0_{\mathsf{tr},\sigma}(\Gamma_{Q}^\mathrm{irr}) \xrightarrow{ \ \ } \mathsf{Pic}_{\D{k}}(Q), \quad f \xmapsto{ \ \ } T_f,$$
gives a section of $q$. Indeed, by \Cref{lemm:auts-field}, the action of $-\otimes^\mathbb{L}_{kQ} T_f$ on $\Gamma_{Q}^\mathrm{irr}$ is precisely that of $f$, and so $q\varphi(f)$ recovers exactly $f$. \qed
\end{example}

\begin{cons}[Base change for Picard groups]
Let $u:\mathcal{C} \to \mathcal{D}$ be a colimit preserving monoidal functor between presentably symmetric monoidal $\infty$-categories, and let $A \in \mathsf{Alg}(\mathcal{C})$ and $B = u(A) \in \mathsf{Alg}(\mathcal{D})$. By \cite[sec. 4.8.3-4.8.5]{Lur17}, one gets an induced functor $\overline{u}: \Mod_A(\mathcal{C}) \to \Mod_B(\mathcal{D})$ such that the diagram
\begin{equation*}
\begin{tikzcd}
\Mod_A(\mathcal{C}) \ar[r,"\overline{u}"] \ar[d,"\mathrm{forget}"'] & \Mod_B(\mathcal{D}) \ar[d,"\mathrm{forget}"] \\
\mathcal{C} \ar[r,"u"] & \mathcal{D}
\end{tikzcd}   
\end{equation*}
commutes. Similarly, there is an induced functor and a commutative diagram
\begin{equation*}
\begin{tikzcd}
{}_A\BMod_A(\mathcal{C}) \ar[r,"\overline{u}"] \ar[d,"\mathrm{forget}"'] & {}_B\BMod_B(\mathcal{D}) \ar[d,"\mathrm{forget}"] \\
\mathcal{C} \ar[r,"u"] & \mathcal{D}.
\end{tikzcd}   
\end{equation*}
We observe that $\overline{u}: {}_A\BMod_A(\mathcal{C}) \to {}_B\BMod_B(\mathcal{D})$ is a monoidal functor. Indeed, the definition of the two-sided Bar construction \cite[Construction 4.4.2.7]{Lur17} is clearly compatible with monoidal functors, that is, $u(\mathrm{Bar}_A(M,N)) = \mathrm{Bar}_B(u(M),u(N))$, and since $u$ preserves geometric realizations, it follows that $u$ preserves the relative tensor product ($u(M\otimes_A N) \simeq u(M)\otimes_B u(N)$, canonically). Because forgetful functors are conservative, $\overline{u}$ also preserves the relative tensor product. Consequently, we obtain an induced homomorphism of Picard groups $$\overline{u}: \mathsf{Pic}_\mathcal{C}(A) \to \mathsf{Pic}_\mathcal{D}(B),$$
whose action coincides with that of $u$ on the underlying $\infty$-categories $\mathcal{C}$ and $\mathcal{D}$.

In particular, if $R$ is an $\mathbb{E}_\infty$-ring, $S$ is a commutative $R$-algebra (equivalently, a morphism of $\mathbb{E}_\infty$-rings $R\to S$), $A$ is an $R$-algebra and $B = S \otimes_RA$, then there is an induced homomorphism of Picard groups
$$S\otimes_R-: \mathsf{Pic}_R(A) \to \mathsf{Pic}_S(B).$$
\end{cons}

\begin{lemm}
Let $K$ be an $\infty$-category with finitely many objects, and let $R$ be an $\mathbb{E}_\infty$-ring and $S$ a commutative $R$-algebra. Then $S\otimes_RRK \simeq SK$ as $S$-algebras.
\end{lemm}
\begin{proof}
By \cite[Lemma 2.7]{AntGep14}, the canonical comparison map
$$S\otimes_RRK = S \otimes_R \mathsf{End}_{RQ}(\textstyle\bigoplus\limits_{k\in K}k_!R) \xrightarrow{ \ \ } \mathsf{End}_{SQ}(S\otimes_R\textstyle\bigoplus\limits_{k\in K}k_!R) \simeq \mathsf{End}_{SQ}(\textstyle\bigoplus\limits_{k\in K}k_!S) = SK$$
is an equivalence.
\end{proof}

\begin{prop} \label{prop:comm-pic}
Let $Q$ be a finite acyclic quiver, and let $R$ be an $\mathbb{E}_\infty$-ring and $S$ a commutative $R$-algebra. There is a commutative diagram of groups
\begin{equation} \label{eq:diag-pic}
\begin{tikzcd}
\mathrm{Aut}_{\mathsf{tr},\sigma}(\Gamma_{Q}^\mathrm{irr}) \ar[r,"\varphi_R"] \ar[rd,"\varphi_S"'] & \mathsf{Pic}_R(Q) \ar[d,"S\otimes_R-"] \\
 & \mathsf{Pic}_S(Q)
\end{tikzcd}   
\end{equation}
where the horizontal and diagonal homomorphisms are given by the action \eqref{eq:action}.
\end{prop}
\begin{proof}
To simplify the notation, let us denote $A = RQ$, $B = SQ$, $u = S\otimes_R-$ and ${}_A\BMod_A^R = {}_A\BMod_A(\Mod_R)$.

Under the identifications $\Mod_A \simeq \Mod_R^Q$ and $\mathsf{Fun}^\mathsf{L}_R(\Mod_A,\Mod_A) \simeq {}_A\BMod_A^R$, the naturality of the action with respect to exact functors (\Cref{rema:naturality-action}) provides, for each $g \in \mathrm{Aut}_{\mathsf{tr},\sigma}(\Gamma_{Q}^\mathrm{irr})$, a commutative diagram
$$\begin{tikzcd}
\Mod_{A} \ar[r,"\overline{u}"] \ar[d,"-\otimes_{A}\varphi_R(g)"'] & \Mod_{B} \ar[d,"-\otimes_{B}\varphi_S(g)"] \\
\Mod_{A} \ar[r,"\overline{u}"] & \Mod_{B}.
\end{tikzcd}$$
Taking functor $\infty$-categories $(-)^{Q^\mathrm{op}}$ and using the identification $\mathsf{Mod}_A^{Q^\mathrm{op}} \simeq {}_{A}\mathsf{BMod}_{A}^R$ (\Cref{coro:bimodules-reps}), we get a commutative diagram on bimodules
$$\begin{tikzcd}
{}_{A}\BMod_{A}^R \ar[r,"\overline{u}"] \ar[d,"-\otimes_{A}\varphi_R(g)"'] & {}_{B}\BMod_{B}^S \ar[d,"-\otimes_{B}\varphi_S(g)"] \\
{}_{A}\BMod_{A}^R \ar[r,"\overline{u}"] & {}_{B}\BMod_{B}^S.
\end{tikzcd}$$
Evaluating the above homotopy identity on $A$, we get 
$$\overline{u}(\varphi_R(g)) \simeq \overline{u}(A \otimes_{A} \varphi_R(g)) \simeq \overline{u}(A) \otimes_{B} \varphi_S(g) \simeq B \otimes_{B} \varphi_S(g) \simeq \varphi_{S}(g)$$
which is precisely the desired commutativity of \eqref{eq:diag-pic}.
\end{proof}

Let $R$ be an $\mathbb{E}_\infty$-ring. We call a \emph{residue field} of $R$ any field $k$ for which there exists a morphism $R \to Hk$. Observe that if $R\not\simeq0$ is connective, then $\pi_0(R)$ contains a maximal ideal $\mathfrak{m}$ and the composition of the canonical morphism $R \to \pi_0(R)$ with the projection to the quotient shows $k = \pi_0(R)/\mathfrak{m}$ as a residue field for $R$.   

\begin{coro} \label{coro:split-mono}
Let $Q$ be a finite acyclic quiver, and let $R$ be any $\mathbb{E}_\infty$-ring with a residue field. Then the action homomorphism
$$\varphi_R: \mathrm{Aut}^0_{\mathsf{tr},\sigma}(\Gamma_{Q}^\mathrm{irr}) \xrightarrow{\ \ } \mathsf{Pic}_{R}(Q)$$
is a split monomorphism.
\end{coro}
\begin{proof}
Let $S = Hk$ for a field $k$ and $u = Hk\otimes_R-:\Mod_R \to \D{k}$ in \Cref{prop:comm-pic}. Then $\varphi_{Hk}$ has a retraction $q$ by \Cref{example:pic-field}, and so $(q\overline{u})\varphi_R = q\varphi_{Hk} = \mathrm{id}$.
\end{proof}

\begin{coro} \label{coro:split-epi}
Let $Q$ be a finite tree, and let $R$ be an $\mathbb{E}_\infty$-ring with residue field $k$. Then the induced homomorphism
$$Hk \otimes_R - : \mathsf{Pic}_{R}(Q) \xrightarrow{\ \ } \mathsf{Pic}_{\D{k}}(Q)$$
is a split epimorphism.
\end{coro}
\begin{proof}
As above, let $S = Hk$ and $u = Hk\otimes_R-:\Mod_R \to \D{k}$ in \Cref{prop:comm-pic}. In this case $\varphi_{Hk}$ is an isomorphism (\Cref{example:pic-field}), and so $\overline{u}(\varphi_R\varphi_{Hk}^{-1}) = \mathrm{id}$
\end{proof}

\begin{examples} \label{example:picard}
\begin{enumerate}
    \item If $R$ is a non-zero connective $\mathbb{E}_\infty$-ring (e.g. the sphere spectrum $\mathbb{S}$, or the Eilenberg-Maclane spectrum of a nonzero commutative ring), then $\mathsf{Pic}_{R}(Q)$ contains $\mathrm{Aut}^0_{\mathsf{tr},\sigma}(\Gamma_{Q}^\mathrm{irr})$ as a semidirect factor.
    \item Let $k$ be any field. The characteristic homomorphism $\mathbb{Z} \to k$ and the unit morphism from the sphere spectrum $\mathbb{S} \to H\mathbb{Z}$ induce homomorphisms
    $$\mathsf{Pic}_{\Sp}(Q) \xrightarrow{H\mathbb{Z}\otimes_\mathbb{S} - } \mathsf{Pic}_{\D{\mathbb{Z}}}(Q) \xrightarrow{k \otimes_\mathbb{Z} -} \mathsf{Pic}_{\D{k}}(Q).$$
    If $Q$ is a tree, then by \Cref{coro:split-epi}, this shows both the integral and spectral Picard groups containing $\mathsf{Pic}_{\D{k}}(Q)$ as a semidirect factor.
\end{enumerate}
\end{examples}

Experience suggests that, due to its universality, the two split epimorphisms above have no kernel. Jointly with Moritz Rahn and Jan Stovicek we plan to prove:

\begin{conj}
If $Q$ is a tree, then the group homomorphisms 
$$\mathsf{Pic}_{\Sp}(Q) \xrightarrow{H\mathbb{Z}\otimes_\mathbb{S} - } \mathsf{Pic}_{\D{\mathbb{Z}}}(Q) \xrightarrow{k \otimes_\mathbb{Z} -} \mathsf{Pic}_{\D{k}}(Q)$$
are both isomorphisms.
\end{conj}