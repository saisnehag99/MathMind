\documentclass[12pt]{article}
\usepackage{authblk}
\usepackage{fixltx2e}
\usepackage[margin=1.3in]{geometry}
\renewcommand{\baselinestretch}{1.12}
\renewcommand\Affilfont{\itshape\small}
\usepackage{amsmath}

\usepackage{amssymb}
\usepackage{color}

\usepackage{amsthm}
%\usepackage{shellesc}
\usepackage{setspace}
\usepackage{hyperref}

%\usepackage[utf8]{inputenc}
%\usepackage{pst-node}
%\usepackage[english]{babel}
\usepackage{amsfonts}
\usepackage{amssymb}
\usepackage{amsmath, amsfonts, bm}
\usepackage{graphicx} % Required for inserting images
\usepackage{amssymb}
\usepackage{xcolor}
\usepackage{cleveref}
\usepackage{xfrac}
\newcommand{\HH}{{\mathbb H}}
\newcommand{\LL}{{\mathbb L}}
\newcommand{\mT}{\mathbb{T}}
\newcommand{\T}{\overline{T}}
\newcommand{\spt}{{\rm Sub}^{p}_{\mathbb{T}^{n}}}
\let\ol=\overline
\let\ap=\alpha
\let\t=\tilde
\let\mi=\setminus
\newcommand{\Aut}{{\rm Aut}}
\newcommand{\Homeo}{{\rm Homeo}}
\newcommand{\Sub}{{\rm Sub}}
\newcommand{\mbb}{\mathbb}
\newcommand{\N}{{\mathbb N}}
\newcommand{\Z}{{\mathbb Z}}
\newcommand{\R}{{\mathbb R}}
\newcommand{\C}{{\mathbb C}}
\newcommand{\Q}{{\mathbb Q}}
\newcommand{\Lc}{{\mathcal L}}
\newcommand{\Sc}{{\mathbb S}}
\newcommand{\mfS}{{\mathfrak{S}}}
\newcommand{\mfH}{{\mathfrak{H}}}
\newcommand{\gf}{{\mathfrak g}}
\newcommand{\hf}{{\mathfrak h}}
\newcommand{\lf}{{\mathfrak l}}
\newcommand{\A}{{\mathcal A}}
\newcommand{\Hc}{{\mathcal H}}
\newcommand{\M}{{\mathcal M}}
\newcommand{\K}{{\mathcal K}}
\newcommand{\D}{{\mathcal D}}
\newcommand{\U}{{\mathcal U}}
\newcommand{\cS}{{\mathcal S}}
\newcommand{\Bc}{{\mathcal B}}
\newcommand{\RP}{{\mathbb{RP}}}
\newcommand{\ea}{\overline{\exp(tX)}}
\newcommand{\spg}{{\rm Sub}^p_G}
\newcommand{\spp}{{\rm Sub}^{p}}
\newcommand{\du}{{\rm d\,}}
\newcommand{\Id}{\,{\rm Id}}
\newcommand{\Inn}{{\rm Inn}}
\newcommand{\inn}{{\rm inn}}
\newcommand{\GL}{{\rm GL}}
\newcommand{\PGL}{{\rm PGL}}
\newcommand{\PSL}{{\rm PSL}}
\newcommand{\SL}{{\rm SL}}
%% \theoremstyle{plain} %% This is the default
\newtheorem{thm}{Theorem}[section]
\newtheorem{cor}[thm]{Corollary}
\newtheorem{lem}[thm]{Lemma}
\newtheorem{prop}[thm]{Proposition}
\newtheorem{ax}{Axiom}
\theoremstyle{definition}
\newtheorem{defn}{Definition}[section]
\newtheorem{rem}[thm]{Remark}
\newtheorem*{notation}{Notation}
\newtheorem{observationn}[thm]{Observation}
\newtheorem{example}[thm]{Example}
\theoremstyle{remark}
\newtheorem{step}{Step}

\title{Dynamics of actions of automorphisms on the space of one-parameter subgroups of a torus and applications}
\author{Debamita Chatterjee, Himanshu Lekharu and Riddhi Shah\\ 
School of Physical Sciences\\
Jawaharlal Nehru University\\
New Delhi 110067, India}
\date{04 November, 2025}

\begin{document}

\maketitle
\begin{abstract}
For a connected Lie group $G$, we study the dynamics of actions of automorphisms of $G$ on certain compact invariant subspaces 
of closed subgroups of $G$ in terms of distality and expansivity. We show that only the finite order automorphisms of $G$ act distally 
on Sub$^p_G$, the smallest compact space containing all closed one-parameter subgroups of $G$, when $G$ is any $n$-torus, $n\in\N$. 
This enables us to relate distality of the $T$-action on Sub$^p_G$ with that of the $T$-action on $G$ and characterise 
the same in terms of compactness of closed subgroups generate by $T$ in the group $\Aut(G)$, in case $G$ is not a vector group. 
We also extend these results to the action of subgroups of automorphisms. We show that any $n$-torus $G$, $n\geq 2$, more 
generally, any connected Lie group $G$ whose central torus has dimension at least 2, does not admit any automorphism 
which acts expansively on Sub$^p_G$. Our results generalise some results on distal actions by Shah and Yadav, and by  
Chatterjee and Shah, and some results on expansive actions by Prajapati and Shah.
\end{abstract}

\smallskip
\noindent{\it Keywords and phrases}: distal and expansive actions of automorphisms, 
 Lie groups, $n$-torus, one-parameter subgroups,
Chabauty topology. 

\smallskip
\noindent 2020 Mathematics Subject Classification:  37B05 (Primary), 22E15, 22D45 (Secondary) 


\tableofcontents

\section{Introduction}
Distal and expansive actions are two significant areas of study in topological dynamics. The notion 
of distality was introduced by David Hilbert to study non-ergodic actions on compact spaces. Distal actions have been studied 
by Ellis \cite{E} and Furstenberg \cite{Fu} on compact spaces, and Abels \cite{Ab1}, Moore \cite{Mo}, Raja-Shah \cite{RSh1, RSh2}
and Shah \cite{Sh1} on Lie groups, see the references cited in \cite{RSh2}. 
The notion of expansivity was introduced by Utz to study chaotic orbits. Since then it has been widely studied by many in different contexts, see 
 Gl\"ockner-Raja \cite{GR}, Shah \cite{Sh2} and Chodhury-Raja \cite{ChoR} and the references cited therein.

For a Hausdorff topological space $X$, a homeomorphism $T$ of $X$  
is said to be {\it distal} (equivalently, $T$ acts {\it distally} on $X$) if for any pair of distinct 
elements $x,y \in X$, the closure of the double orbit 
$\{(T^n(x), T^n(y)) \mid n\in \Z\}$ in $X \times X$ does not intersect 
the diagonal, i.e.\ for $x,y \in X$ with $x \neq y $, 
$\ol{\{(T^n(x), T^n(y))\mid n \in \Z\}} \cap \{(d,d)\mid d \in X\}=\emptyset$.
 If $X$ is a compact metric space with the metric $d$, then $T$ is distal if and only if for every pair 
  $x,y \in X$ with $x \neq y$, $\mathrm{inf}\{ d(T^n(x), T^n(y)) \mid n \in \Z\} > 0.$ If $X$ is a topological
   group and $T$ is an automorphism then $T$ is distal if and only if for every $x \in G$ with $x \neq e$, 
   $e \notin \ol{\{T^n(x)\} \mid n \in \Z\} }$. 

Let $X$ be a metrizable topological space with a metric $d$, a homeomorphism $T$ of $X$ is said to be {\it expansive} 
if there exists $\epsilon> 0$ such that the following hold: If $x,y \in X$ with $ x \neq y$, then
  $\sup\{d(T^n(x), T^n(y))\mid n\in\Z\}> \epsilon$. Here, $\epsilon$ is called an expansive constant for $T$. It is known that 
  the expansivity of a homeomorphism on any compact metric space is independent of the metric. For a locally compact 
  topological group $G$, an automorphism $T$ is expansive if $\cap_{n \in \Z}T^n(U)= \{e\}$ for some neighbourhood 
   $U$ of $e$. In particular, if $T$ is expansive on $G$, then $G$ is first countable and admits a left invariant metric 
   and this definition agrees with the one given above for the metric space. It is also well known that for any infinite compact 
   metric space, the class of distal homeomorphisms and the class of expansive homeomorphisms are mutually disjoint 
   (cf.\ \cite{Bry}, Theorem 2). 

For a locally compact Hausdorff topological group $G$, let $\Sub_G$ denote the set of all closed
 subgroups of $G$ equipped with Chabauty topology. Note that $\Sub_G$ is compact and Hausdorff, and 
 if $G$ is second countable then it is metrizable. Let $\Aut(G)$ denote the group of all automorphisms of $G$
   (i.e.\ homomorphisms of $G$ which are also homeomorphisms). There is a natural action of $\Aut(G)$ on $\Sub_G$; 
   namely,  $(T,H) \mapsto T(H), T \in \Aut(G), H \in \Sub_G$. This action gives rise to a homomorphism from $\Aut(G)$
   to $\mathrm{Homeo}(\Sub_G)$. Motivated by the fact that the image of $\Aut(G)$ yields a large class of 
   homeomorphisms of $\Sub_G$, we aim to explore the dynamics of this subclass, 
   particularly in the context of distality and expansivity. 

Let $G$ be a connected Lie group and $T \in \Aut(G) $. We say that $T$ acts distally (resp.\ expansively) on 
$\Sub_G$ if the homeomorphism of $\Sub_G$ corresponding to $T$ is distal (resp.\ expansive). 
Shah and Yadav \cite{ShY3} first studied such distal actions for connected Lie group $G$ and they showed that for a large
  class of Lie group $G$, which does not have non-trivial compact connected central subgroup, an automorphism 
  $T$ acts distally on $\Sub^a_G$, the space of closed abelian subgroups of $G$, if and only if $T$ generates
  a compact subgroup in $\Aut(G).$ Moreover they showed that if $G$ does not have any nontrivial compact connected 
  central subgroup, then $T$ acts distally on 
 $\Sub^a_G$ implies that $T$ is distal (on $G$); see Corollary 3.7 in \cite{ShY3}. Shah and Prajapati first studied such expansive actions 
 for locally compact second countable groups. They showed that if $T$ acts expansively on $\Sub_G$, then $T$ is expansive on $G$,
 moreover a (nontrivial) connected Lie group $G$ does not admit any automorphism which acts expansively on $\Sub^a_G$.  Shah, 
 together with Palit \cite{PaSh}, and with Palit and Prajapati \cite{PaPrSh}, investigated the distal and expansive action of 
 automorphisms on $\Sub^a_G$ for discrete groups $G$, where $G$ is either polycyclic or a lattice in a connected Lie group.
Note that $\Sub^a_G$ is very large for many groups $G$, in particular if $G$ is abelian, then it is the same as $\Sub_G$. 

A natural question that arose from the above investigations was whether distal (resp.\ expansive) actions of automorphisms of $G$ can be
 characterised (resp.\ exist) on other smaller invariant subspace of $\Sub^a_G$. A recent work by Chatterjee and Shah \cite{CSh} 
 considers one such subspace: the class of (nontrivial) smallest closed connected abelian subgroups, namely closed
 one-parameter subgroup of $G$.  Let $\spg$ denote the smallest closed subset of $\Sub_G$ containing 
 all closed one-parameter subgroups of $G$. Note that $\spg$ is compact and it is invariant under the
 action of $\Aut(G)$. In Theorem 1.1 of \cite{CSh}, the class of distal actions of automorphisms on the 
 space $\spg$ is characterised when $G$ is a Lie group without central torus as follows; if $G$ is abelian, i.e.\ 
 $G$ is isomorphic to $\R^n$ for some $n\in\N$, then $T$ acts distally on $\spg$ if and only if $T \in \K\D$, where $\K$ is a 
 compact subgroup of $\GL(n,\R)$  and $\D$ is the center of $\GL(n,\R)$, and if $G$ is not abelian, then $T$ acts 
 distally on $\spg$ if and only if $T \in \K$, a compact subgroup of $\Aut(G)$. This was also generalised to characterise 
 $T$ which act distally on the maximal central torus, see Theorem 1.4 of \cite{CSh}.
 
 We know that the maximal compact connected central subgroup of a connected Lie group (maximal central torus) is either 
 trivial or isomorphic to $\mT^n$, the $n$-torus (also known as the central torus of $G$), for some $n\in\N$. The work of Chatterjee 
 and Shah \cite{CSh} does not cover the case when $G$ is a torus, or the more general case of connected Lie groups $G$  
 without any condition on the $T$-action on the central torus. Moreover, the behaviour of distal 
 and expansive actions on $\spp_{\mT^n}$ has not yet been explored. In this paper, we investigate these 
 dynamical properties when $G = \mT^n$. We get the following which characterises distal actions on $\spt$. 

\medskip
\noindent {\bf \Cref{distal-t}.}
\emph{Let $G$ be an $n$-torus for some $n\in\N$, and let $T$ be an automorphism on $G$. Then $T$ acts distally on 
$\spg$ if and only if $T^m=\Id$, the identity map, for some $m\in\N$.}
\medskip

The following relates the distal action of automorphism of $G$ on $\spg$ and distlity of the automorphism and it generalises Theorem 1.2 of \cite{CSh}. 

\medskip
\noindent {\bf \Cref{distal-g}.}
\emph{Let $G$ be a connected Lie group which is not a vector group and let $T \in \Aut(G)$. If $T$ acts distally on $\spg$, then $T$ acts distally on $G$.}
\medskip

For a Lie group $G$, let $\Aut(G)$ be endowed with the compact-open topology. It is a Lie group which is identified with a closed subgroup of the 
Lie algebra automorphisms of the Lie algebra of $G$.  Let $\Sub^c_G$ denote the set of all discrete cyclic subgroups of $G$. 
Using \Cref{distal-t} on the distal actions of automorphisms on $\Sub^p_{\mT^n}$, we obtain 
a characterisation of distal actions of automorphisms on the space $\spg$ for any connected Lie group $G$ in the following.

\medskip
\noindent {\bf \Cref{gen-Lie}.}
\emph{Let $G$ be a connected Lie group and let $T \in \Aut(G)$. Consider the following statements.
\begin{enumerate}
\item $T$ acts distally on $\spg$.
\item $T$ acts distally on $\ol{\Sub^c_G}$. 
\item $T$ acts distally on $\Sub^a_G$. 
\item $T$ acts distally on $\Sub_G$. 
\item $T$ is contained in a compact subgroup of $\Aut(G)$. 
\end{enumerate}
Then $(2-5)$ are equivalent. If $G$ is not a vector group, then $(1-5)$ are equivalent.}
 \medskip

Theorem \ref{distal-t} generalises several results from the work of Chatterjee and Shah \cite{CSh}, notably Theorems 1.1\,(2), 1.2 and 1.4 of \cite{CSh}. 
Note that when $G$ is a vector group, Theorem 1.1\,(1) of \cite{CSh} has characterised the distal action of automorphisms 
on $\spg$. We also generalise Theorem 1.5 of \cite{CSh} about the action of a subgroup of $\Aut(G)$ on $\spg$, see Theorem \ref{subgp-action}.

For expansive actions, it is known that $\mT^n$ does not admit any automorphism which acts expansively on $\Sub_{\mT^n}$ (cf.\ \cite{PrSh}, Theorem 3.2). 
The question arises whether any automorphism of $\mT^n$ acts expansively on a smaller invariant subspace of $\Sub_{\mT^n}$. The following shows that 
in case of $\spt$ the answer is negative. 

\medskip
\noindent {\bf \Cref{expansive-t}.}
\emph{Let $G=\mT^n$, the $n$-torus, for any $n\geq 2$. Then $G$ does not admit any automorphism that acts expansively on $\spg$.}
 \medskip

As a consequence of the above theorem, we get the following for a larger class of connected Lie groups. 

\medskip
\noindent {\bf \Cref{Lie-t}.}
\emph{Let $G$ be a connected Lie group such that it contains a central torus of dimension at least $2$. Then $G$ does not admit any automorphism that acts 
expansively on $\spg$.}
\medskip

 \Cref{Lie-t} generalises Theorem 3.1 of \cite{PrSh}, in case $G$ is a connected Lie group as above.  

For many groups $G$, compact spaces $\Sub_G$, $\Sub^a_G$ and $\Sub^c_G$ are identified (see Baik and Clavier \cite{BC1, BC2}, 
Bridson et al.\  \cite{BrHK}, Harmrouni and Kadri \cite{HaKa}, and also Pourezza and Hubbard \cite{PHu}); one can also identify $\Sub^p_G$ 
for some groups $G$, e.g.\ $\Sub^p(\R^n)$ homeomorphic to $\R\mathbb{P}^{n-1}$, the real projective space of dimension $n-1$, $n\geq 2$. 
The study of the action of $\Aut(G)$ on $\Sub_G$ and on its closed (compact) invariant subspaces leads to a better understanding of dynamics 
on these spaces.

For a subgroup $H$ of $G$, let $H^0$ denote that connected component of the identity $e$ in $G$ and $\ol{H}$ is the closure of $H$ in $G$; 
both $H^0$ and $\ol{H}$ are subgroups of $G$. A $k$-torus is a compact connected abelian Lie group of dimension $k$, $k\in\N$. Any compact 
abelian subgroup of divisible if and only if it is connected (i.e.\ it is a torus). A maximal central torus (a maximal compact connected central subgroup) 
of $G$ is characteristic in $G$. Any one-parameter subgroup in $G$ is a a continuous homomorphism from $\R$ to $G$. It is either closed 
and isomorphic to $\R$, $\{e\}$ or a $1$-torus $\mT^1$, or its closure is a $k$-torus for some $k\geq 2$ (cf.\ \cite{CSh}, Lemma 2.3). We may denote 
a one-parameter subgroup $\{x_t\}_{t\in\R}$ by just $\{x_t\}$. 

In \S\,2 we discuss the topology of $\Sub_G$ and the action of $\Aut(G)$ on $\Sub_G$. We also derive some properties of $\spt$ and prove some 
useful lemmas about its structure. In \S\,3 we prove results on distal actions of automorphisms $T$ on $\spt$ and also that of automorphism 
groups of $\mT^n$ on $\spt$. In \S\,4 we prove a result on the orbits of subspaces of $\R^n$ and prove the main result about expansive actions. 

\section{Structure and properties of $\Sub_G$ and $\spt$} 

For any locally compact (Hausdorff) group $G$, the space $\Sub_G$, of all closed subgroups of $G$ is endowed with the Chabauty topology, 
which is generated by a sub-basis 
$$
\{\U_1(K)\mid K\subset G \mbox{ is compact}\}\cup\{\U_2(U)\mid U\subset G \mbox{ is open}\},$$  
where $\U_1(K)=\{H\in\Sub_G\mid H\cap K=\emptyset\}$ and $\U_2(U)=\{H\in\Sub_G\mid H\cap U\neq\emptyset\}$. Note that 
$\Sub_G$ is compact and Hausdorff, and it is metrizable if $G$ is second countable (cf \cite{BP}, Lemma E.1.1). 
Since we deal only with closed subgroups of a connected Lie group $G$, we have that $\Sub_G$ is metrizable. 
The following criteria of convergence in $\Sub_G$ is well-known (see e.g.\ Proposition E.1.2 in \cite{BP}). 

\begin{lem}\label{conv}
Let $G$ be a connected Lie group. A sequence 
$\{H_n\}\subset \Sub_G$ converges to $H\in \Sub_G$ 
if and only if the following hold:
\begin{enumerate}
\item[${\rm(I)}$] For $g\in G$, if there exists a subsequence $\{H_{n_k}\}$ of 
$\{H_n\}$ with $h_k\in H_{n_k}$, $k\in\N$, such that $h_k\to g$ in $G$, then $g\in H$. 
\item[${\rm(II)}$] For every $h\in H$, there exists a sequence $\{h_n\}_{n\in\N}$ such that
 $h_n\in H_n$, $n\in\N$, and $h_n\to h$.
\end{enumerate}
\end{lem}

 Recall that for a connected Lie group $G$, there is a natural action of the space $\Aut(G)$ of (bi-continuous) automorphisms of $G$ on $\Sub_G$; 
 namely, the map $H\mapsto T(H)$, $H\in\Sub_G$, $T\in\Aut(G)$. This is a continuous group action on $\Sub^p_G$ by homeomorphisms and 
 it keeps the following subspaces invariant: $\Sub^a_G$ consisting of closed abelian subgroups, $\Sub^c_G$ consisting of discrete cyclic subgroups 
 and its closure, the space of closed one-parameter subgroups and its closure $\Sub^p_G$. For some basic structural properties of $\Sub_G$ and the 
 action of $\Aut(G)$ on it, we refer the readers to \cite{ShY3}. 

Note that $\Aut(\mT^n)= \GL(n, \Z),$ where $\GL(n, \Z)$ denotes the group of invertible $n \times n$ matrices of determinant $\pm 1$ with integer entries. 
The compact open topology on $\GL(n, \Z)$ is discrete. Therefore, any compact subgroup of $\GL(n, \Z)$ is finite, in particular, an element of $\Aut(\mT^n)$  
generates a relatively compact subgroup in $\Aut(\mT^n)$ if and only if it has finite order.

Any one-parameter subgroup in a torus $\mT^n$ is either closed and isomorphic to the trivial subgroup or to $\mT^1$, or its closure 
is isomorphic to an $k$-dimensional torus for $1<m\leq n$ when $n\geq 2$. Moreover, $\spt$ consists of all $k$-tori, $1\leq k\leq n$ and 
the trivial subgroup (cf.\ \cite{CSh}, Lemma 2.3). If $n\ne 1$, $\spt$ is infinite; in fact, the set of closed one-parameter 
subgroups is infinite as the the set of roots of unity is dense in $\mT^n$, and $\mT^n$ is exponential. Note that $\spt$ is countable as 
any $k$-torus in $\mT^n$ corresponds to a vector subspace $V$ of dimension $k$ in the covering group $\R^n$ (of $\mT^n$) such that 
$V$ is generated by $k$ (linearly independent) elements of $\Z^n$. 


For the sake of convenience, we state these facts as a lemma about the structure of $\spt$ which is essentially known.

\begin{lem} \label{subptn} For an $n$-torus $\mT^n$, the following holds. 
\begin{enumerate}
\item $\spt=\{\mT^k\mid 1\leq k\leq n\}\cup\{\{e\}\}$.
\item $\spt$ is countable.
\item The trivial subgroup $\{e\}$ is isolated in $\spt$.
\end{enumerate}
\end{lem}

Lemma \ref{subptn}\,(3) is known (see e.g.\ Lemma 2.3 in \cite{CSh}).

\smallskip
The set $\Sub^p_\mT$ has only two elements, the trivial subgroup $\{e\}$ and $\mT$. Moreover $\Aut(\mT)$ consists of only two elements; namely, 
the identity map and the map $x\mapsto x^{-1}$, and the action of any of this maps on $\Sub^p_\mT$ is distal as well as expansive. If $n\geq 2$, 
$\mT^n$ contains infinitely many subgroups which are isomorphic to $\mT^k$ for each $k$, $0<k<n$, as observed above. Since $\mT^n$ is 
compact, the Chabauty topology on $\Sub_{\mT^n}$ is induced by the Hausdorff metric (cf.\ \cite{BP}, Proposition E.1.3), and it allows us to use 
results from the theory of compact abelian metric groups related to the Hausdorff metric. 
We now state a useful lemma by Berend (cf.\ \cite{Be}, Lemma 4.7). 

\begin{lem} \rm{(Berend \cite{Be})} \label{Berend1}
Let $G$ be a compact abelian metric group and let $\Gamma$ be the dual group. A sequence $\{G_m\}_{m=1}^\infty$ of closed subgroups 
of $G$ satisfies $G_m \to G$ (in the Hausdorff metric) if and only if for every nonzero $ \gamma \in \Gamma$ we have 
$\gamma \notin \mathrm{Ann}(G_m)$ for sufficiently large $m$ (where $\mathrm{Ann}(H)$ denotes the annihilator in $\Gamma$ of a 
closed subgroup $H$ of $G$). 
\end{lem}

The following is a direct consequence of \Cref{Berend1} (\cite{Be}, Lemma 4.7).

\begin{lem}\label{Berend2}
Let $\mT^n$ be the $n$-torus and let $\{G_m\}$ be a sequence of closed subgroups in $\mT^n$. Then $\{G_m\}$ does not converges to 
$\mT^n$ in $\spt$ if and only if there exists a character $\gamma$ of $\mT^n$ such that $\gamma \in \mathrm{Ann}(G_m)$ for infinitely many $m$.
\end{lem}

We know that $\{e\}$ is isolated in $\spt$. We now show that any proper subtorus is isolated 
in the set of subtori of same dimension, or more generally, in the space of subtori of same or higher dimension. For $1\leq k<n$, let $\mfS_k$ 
be the space of all subtori of dimension $k$ in $\mT^n$ and let $\mfH_k=\cup_{m=k}^{n-1}\mfS_m\cup \{\mT^n\}$. Then 
$\mfH_k\subset\mfH_{k-1}$ for $2\leq k< n$ and 
$$
\spt=\mfH_1\cup \{\{e\}\}=\cup_{k=1}^{n-1}\mfS_k\cup\{\mT^n\}\cup \{e\}.$$ 
It is easy to see that all $\mfS_k$ and $\mfH_k$ are $T$ invariant for every $T\in\GL(n,\Z)$,
for $1\leq k\leq n-1$. As the set of $k$-dimensional subspaces in $\R^n$ is closed in $\Sub_{\R^n}$, we get 
that each $\mfH_k$ is closed in $\spt$. 


\begin{lem} \label{isol} 
Let the notation be as above. For $n\geq 2$, $\mfH_k$ is closed  (compact) in $\spt$. 
Moreover, every $H\in \mfS_k$ is isolated in $\mfH_k$.
 \end{lem}
 
\begin{proof} Note that since any subtorus is divisible and $\mT^n$ is compact, the limit of a sequence of subtori is a subtorus or 
$\mT^n$ (see also \Cref{subptn}). Let $\pi:\R^n\to\mT^n$ be the natural projection with $\ker\pi=\Z^n$.  Then every 
$H\in \mathfrak{S}_k$ corresponds a $k$-dimensional vector subspace $V$ of $\R^n$ such that $\pi(V)=H$. Since the set of all 
$k$-dimensional subspace of $\R^n$ is closed in $\Sub_{\R^n}$, we get that if $H_m\to H$ in $\spt$ for $\{H_m\}\subset \mathfrak{S}_k$, 
then $\dim(H)\geq k$. Hence $H\in \mfH_k$. In particular,  $\mfH_k=\cup_{l=k}^{n-1}\mfS_l\cup\{\mT^n\}$ is 
closed. Thus the first assertion holds.
 
 \medskip
\noindent{\bf Step 1:}  Now we prove the second assertion. We first consider $k=n-1$. Let $H\in \mfS_{n-1}$. We want to show that 
 $H$ is isolated in  $\mfH_{n-1}$. Suppose $H_m\in \mfH_{n-1}$, $m\in\N$, is such that $H_m\to H$, then we may assume that 
 $H_m\in \mfS_{n-1}$ for all $m\in\N$. If possible, suppose $H_m\neq H$ for infinitely many $m$. Passing to a subsequence if necessary, 
 we may assume that $H_m\neq H$ for all $m$. By Lemma 4.7 of \cite{Be} (see also Lemmas \ref{Berend1} 
 or \ref{Berend2}), there exists nonzero character $\psi$ on $\mT^n$ such that $\psi(H_m)=0$, i.e.\ 
 $H_m\subseteq (\ker\psi)^0$, for infinitely many $m$. Since $\dim(H_m)=n-1$, we get that $H_m=(\ker\psi)^0$ for infinitely many $m$, 
 and hence $H=(\ker\psi)^0$. Therefore, $H_m=H$ for infinitely many $m$, which leads to a contradiction. Thus $H_m=H$ for all large $m$. 
 Therefore, each $H\in \mathfrak{S}_{n-1}$ is isolated in $\mfS_{n-1}$, and hence in $\mfH_{n-1}$. In particular, the second assertion holds for $n=2$. 
  
  \medskip
  \noindent{\bf Step 2:} Now suppose $n\geq 3$. Suppose the second assertion holds for every torus with dimension $n-1$. Now we prove the 
  assertion for $\mT^n$. Suppose $1\leq k \leq n-1$. If $k=n-1$, then the assertion follows from Step 1. Now suppose $1\leq k\leq n-2$. Let $H\in\mfS_k$. 
  Suppose $H_m\to H$ for some $\{H_m\}\subset \mfH_k$. 
   If possible, suppose $H_m\ne H$ for infinitely many $m$. Passing to a subsequence if necessary, we may 
   assume that that $H_m\ne H$ for all $m$. Since $H\ne\mT^n$, arguing as in Step 1 using Lemma 4.7 of \cite{Be}, we get that 
  there exists a character $\gamma$ of $\mT^n$ such that $\gamma(H_m)=0$, i.e.\ $H_m\subset (\ker\gamma)^0=K$ for infinitely 
  many $m$, and hence it follows that $H\subset K$. Since $\dim(K)=n-1$, by induction, $H$ is isolated in the set (say) $\mfH_k(K)$ of all 
  subtori of $K$ with dimension greater than or equal to $k$, and hence $H_m=H$ for infinitely many $m$, which leads to a contradiction. 
  Therefore, $H$ is isolated in $\mfH_k$, and the assertion holds by induction for all $n$. 
  \end{proof}

\section{Distal actions of automorphisms of connected Lie groups $G$ on Sub$^p_G$} 

In this section, we first prove \Cref{distal-t} which extends Theorem 1.1\,(2) of \cite{CSh} to compact 
connected abelian Lie groups. Using Theorem \ref{distal-t} we generalise Theorems 1.1\,(2), 1.2, 1.4 and 1.5 of \cite{CSh}.

\begin{thm}\label{distal-t} Let $G$ be an $n$-torus for some $n\in\N$, and let $T$ be an automorphism on $G$. Then $T$ acts distally on 
$\spg$ if and only if $T^m=\Id$, the identity map, for some $m\in\N$.
\end{thm}

\begin{proof}
One way assertion is obvious. We prove the converse by the induction on the dimension $n$ of the torus; i.e.\ we assume that $T$ acts distally 
on $\spg$ and 
show that $T^m=\Id$ for some $m\in\N$. If $G=\mT^1$, the one-dimensional torus, then $\Aut(G)$ has only two elements, hence $T^2 =\Id$. 
Thus the assertion holds trivially for $n=1$. Now suppose the assertion holds for any torus $G$ with $\dim G\leq n-1$. where $n\ge 2$. 
Let $G=\mT^n$. Let $H\in\mfS_{n-1}$. Then $H\in\spg$ by \Cref{subptn} (see also Lemma 2.3 in \cite{CSh}). As $\spg$ is 
compact, there exists a strictly increasing sequence $\{m_k\}$ in $\N$, such that $T^{m_k}(H)\to L$ for some $L\in\spg$.  As $H\in\mfS_{n-1}\subset\mfH_{n-1}$, 
and $\mfH_{n-1}$ is compact and $T$-invariant, we have that $L\in\mfH_{n-1}$. Since $T$ acts distally on $\spg$ 
and $T(G)=G$ and $H\ne G$, it follows that  $L\ne G$. Therefore, $L\in\mfS_{n-1}$. By \Cref{isol}, $L$ is isolated in $\mfH_{n-1}$, and hence  
$T^{m_k}(H)=L$ for all large $k$. Thus for some $k\in\N$, $T^{m_k}(H)=T^{m_{k+i}}(H)$ for all $i\in\N$. For $l=m_{k+1}-m_k$, we have that $H$ is $T^l$-invariant. 
Since $T^l$ acts distally on $\spg$, and also on $\spp_H$, and $\dim H=n-1$, by the induction hypothesis we get that $(T|_H)^{lr}=\Id$ for 
some $r\in\N$. Moreover, since $G/H$ is isomorphic to  $\mT^1$, we have that $\T^2=\Id$, where $\T$ is the automorphisms of $G/H$ induced by $T$.
 It follows that for $m=2lr$, all the eigenvalues of $T^m$ are equal to 1, and hence $T^m$ is unipotent. Since $T^m$ acts distally on $\spg$, 
by Theorem 1.3 of \cite{CSh}, $T^m = \Id$. Thus the assertion holds for all $n$ by induction.
\end{proof}


We say that a subgroup $\Hc$ of homeomorphisms of $X$ acts distally on a topological space $X$ if for any $x,y\in X$ such that $x\ne y$, 
$\ol{\{(T(x),T(y))\mid T\in \Hc\}}\cap\{(d,d)\mid d\in X\}=\emptyset$. Note that if $\Hc$ acts distally on $X$, then every element of $\Hc$
acts distally on $X$. The following corollary will be useful in the proof of \Cref{subgp-action}.

\begin{cor} \label{subgp-t} Let $\Hc$ be a subgroup of $\Aut(\mT^n)$ for some $n\in\N$. Then the following are equivalent: 
\begin{enumerate} 
\item Every element of $\Hc$ acts distally on $\spt$. 
\item $\Hc$ acts distally on $\spt$. 
\item $\Hc$ is a finite group.
\end{enumerate}
\end{cor}

\begin{proof} 
The statements $(3)\implies (2)\implies (1)$ are obvious. Now we show that $(1)\implies (3)$. This hold for $n=1$ as
$\Aut(\mT)$ is a group of order $2$. Now suppose $n\geq 2$. Suppose (1) holds. By Theorem \ref{distal-t}, every element of 
$\Hc$ has finite order. Note that $\Aut(\mT^n)=\GL(n,\Z)$ is a discrete subgroup of $\GL(n,\R)$ as well of $\GL(n,\C)$. Then 
$\Hc$ is closed in $\GL(n,\C)$ and by Theorem 1.1 of \cite{FP}, $\Hc$ is compact. As $\Hc\subset\GL(n,\Z)$ is discrete, we 
get that $\Hc$ is finite. Thus (3) holds, and hence $(1-3)$ are equivalent. 
\end{proof} 
 
 In \cite{CSh}, for $T\in\Aut(G)$ it is shown that $T$ acts distally on $\spg$ implies that it acts distally on $G$, where $G$ is 
a connected non-abelian Lie group without any (nontrivial) central torus; more generally, if $G$ is not a vector group and $T$ acts distally 
on the maximal central torus. The following corollary generalises Theorem 1.2 of \cite{CSh}. 
 
\begin{cor}\label{distal-g}
Let $G$ be a connected Lie group which is not a vector group and let $T \in \Aut(G)$. If $T$ acts distally on $\spg$, then $T$ acts distally on $G$.
\end{cor}

\begin{proof} Suppose $T$ acts distally on $\spg$ for a connected Lie group $G$ which is not a vector group. 
Let $M$ be the largest compact connected central subgroup of $G$. If $M$ is trivial, then the assertion follows from Theorem 1.1\,(2) of \cite{CSh}. 
Now suppose $M$ is nontrivial. Then $M \cong \mT^n$ for some $n\in\N$. Note that $M$ is invariant under $T$ and $T|_M$ acts distally on 
$\spp_M$. By Theorem \ref{distal-t} we get that some power of $T|_M$ is the identity map. In particular, $T$ acts distally on $M$. As $T$ acts distally
on $\spg$, by Theorem 1.2 of \cite{CSh},  $T$ acts distally on $G$. 
\end{proof}

The following theorem generalises Theorems 1.1\,(2) and 1.4 of \cite{CSh}, and also a part of Theorem 4.1 of \cite{ShY3}.

\begin{thm} \label{gen-Lie}
Let $G$ be a connected Lie group and let $T \in \Aut(G)$. Consider the following statements.
\begin{enumerate}
\item $T$ acts distally on $\spg$.
\item $T$ acts distally on $\ol{\Sub^c_G}$. 
\item $T$ acts distally on $\Sub^a_G$. 
\item $T$ acts distally on $\Sub_G$. 
\item $T$ is contained in a compact subgroup of $\Aut(G)$. 
\end{enumerate}
Then $(2-5)$ are equivalent. If $G$ is not a vector group, then $(1-5)$ are equivalent. 
\end{thm}

\begin{proof}
Statements $(5) \implies (4) \implies (3) \implies (2) \implies (1)$ are trivial. Suppose $G$ is not a vector group. We need to prove that $(1) \implies (5)$. 
Suppose $(1)$ holds, i.e.\ $T$ acts distally on $\spg$. Let $M$ be he largest compact connected central subgroup of $G$. Then $M$ is characteristic 
in $G$, i.e.\ $T(M)=M$. Then (1) implies that $T$ acts distally on $\spp_M$. By Theorem \ref{distal-t}, for some $m\in\N$, $T^m$ acts trivially on $M$, 
in particular, $T$ acts distally on $M$. Now by Theorem 1.4 of \cite{CSh}, $T$ is contained in a compact subgroup of $\Aut(G)$. 
Thus $(5)$ holds, and $(1-5)$ are equivalent if $G$ is not a vector group. 

Now suppose $G$ is a vector group and suppose (2) holds. Then (2) implies that $T\in{\rm (NC)}$, i.e.\ the trivial subgroup $\{e\}$ is not a limit point 
of $\{T^n(C)\}_{n\in\Z}$ for any discrete closed cyclic subgroup $C$ of $G$. Then by Theorem 4.1 of \cite{ShY3}, $T$ is contained in a compact subgroup 
of $\Aut(G)$, i.e.\ (5) holds, and $(2-5)$ are equivalent if $G$ is a vector group. 
\end{proof}

 The following generalises a part of Theorem 4.1 of \cite{ShY3} and Theorem 1.5 of \cite{CSh} when $G$ is not a vector group. 
 Note that in case $G$ is a vector group, the equivalence of $(1)$ and $(2)$, as well as that of $(3-6)$ is shown in Theorem 1.5 of \cite{CSh}.
Note also that it is not possible to generalise Theorem 4.1 of \cite{ShY3} for all connected Lie groups $G$, for if $G$ is an $n$-torus, $n\geq 2$, then 
$\Aut(G)\cong\GL(n,\Z)$ and every $T\in\Aut(G)$ belongs to (NC). 

\begin{thm} \label{subgp-action} Let $G$ be a connected Lie group. Let $\Hc$ be a subgroup of $\Aut(G)$. Consider the following statements:
\begin{enumerate}
\item Every element of $\ol{\Hc}$ acts distally on $\spg$. 
\item $\Hc$ acts distally on $\spg$.
\item $\Hc$ acts distally on $\ol{\Sub^c_G}$. 
\item $\Hc$ acts distally on $\Sub^a_G$. 
\item $\Hc$ acts distally on $\Sub_G$. 
\item $\ol{\Hc}$ is a compact group.
\end{enumerate}
\noindent Then $(1)$ and $(2)$ are equivalent, and $(3-6)$ are equivalent. If $G$ is not a vector group, then $(1-6)$ are equivalent. 
\end{thm}

\begin{proof} It is obvious that $(6)\implies (5) \implies (4)\implies (3) \implies(2)$. Also if $H$ acts distally on $\spg$, then so does $\ol{\Hc}$ 
since $\spg$ is compact (cf.\ \cite{E}, Theorem 1). Therefore $(2)\implies (1)$. Now suppose (1) holds. Let $M$ be the largest compact connected 
central subgroup of $G$. Then $M$ is characteristic in $G$ and in particular, it is invariant under the action of $\ol{\Hc}$. If every element of $\ol{\Hc}$ acts 
distally on $\spp_M$, then by \Cref{subgp-t}, $\{T|_M\mid T\in \ol{\Hc}\}$ is a finite group. Therefore $\ol{\Hc}$, and hence, $\Hc$ acts distally 
on $M$. If $G$ is not a vector group, by Theorem 1.5 of \cite{CSh}, $\ol{\Hc}$ is compact. Thus (6) holds, 
and hence $(1-6)$ are equivalent. 

If $G= \R^n$, a vector group, then the assertions $(1)\implies (2)$ and $(3)\implies (6)$ are proven in Theorem 1.5 in \cite{CSh}. 
\end{proof}


\section{Expansivity of actions of automorphisms of $\mT^n$ on $\spt$}

In this section we prove that $\mT^n$ does not have any automorphism that acts expansively on $\spt$ for any $n\geq 2$ (see \Cref{expansive-t}), 
and prove \Cref{Lie-t}. We also prove that every element of $\GL(n,\Z)$ has infinitely many orbits consisting of $(n-1)$-dimensional rational subspaces 
(see \Cref{rational}). 

We first state some well-known properties about expansive maps, as listed in Lemma 2.1 of \cite{PrSh}.  


\begin{lem}[\cite{W}, Corollary 5.22 \& Theorem 5.26] \label{expa-prop} 
	Let $(X,d)$ be a compact metric space. Then the following hold for homeomorphisms of $X$:
\begin{itemize}
\item[(1)] Expansivity is a topological conjugacy invariant.
\item[(2)] Expansivity of a homeomorphism is independent of the metric chosen as long as the metric induces the topology of $X$. 
    However, expansivity constant may change.
    
\!\!\!\!\!\!\!\!\!\!\!\! Moreover, the following hold for any homeomorphism $\phi$ of $X$:

\item[(3)] $\phi^n$ is expansive for some $n\in \Z\setminus \{0\}$ if and only if $\phi^n$ is expansive for all $n\in \mathbb{Z}\setminus \{0\} $.
\item[(4)] For any $n\in \Z\setminus \{0\}$, if $\phi$ is expansive then $\phi^n$ has only finitely many fixed points.
\item[(5)] If $\phi$ is expansive and $Y$ is a closed $\phi$-invariant subset of $X$, the $\phi|_Y$ is also expansive.
\end{itemize}
\end{lem}

Theorem 3.1 of \cite{PrSh} shows that a nontrivial connected Lie group does not admit any automorphism that acts expansively on 
$\Sub^a_G$. As $\spg\subset\Sub^a_G$, the following generalise Theorem 3.1 of \cite{PrSh} for the case when $G$ is an $n$-torus, 
$n\ge 2$. (For $n=1$, $\spg$ has only two elements, and hence it holds trivially that every $T\in\Aut(G)$ acts expansively on $\spg$.) 
 
 \begin{thm}\label{expansive-t}
    Let $G=\mT^n$, the $n$-torus, for any $n\geq 2$. Then $G$ does not admit any automorphism that acts expansively on $\spg$.
 \end{thm}
 
 
 Before proving the theorem, we define a notion of rational subspaces in $\R^n$ and discuss their orbits under the action of 
 $\GL(n,\Z)$. A subspace $W$ of $\R^n$ is said to be a {\it rational subspace}, if $W$ is generated by  $W\cap\Z^n$; 
 equivalently, if $W\cap\Z^n$ is isomorphic to $\Z^k$, where $k=\dim(W)$. Note that $W$ is a rational subspace if and only if 
 the group $W+\Z^n$ is closed in $\R^n$. Each $k$-dimensional rational subspace $W$ corresponds to a $k$-subtorus ($k$-dimensional 
 compact connected subgroup) in $\mT^n$; namely, the image of $W$ in $\mT^n=\R^n/\Z^n$. Any subspace generated by some 
 integer points of $\R^n$ is a rational subspace. Note  that if $W$ is a rational subspace of $\R^n$, then $T(W)$ is also a rational subspace 
 (of $\R^n$) with the same dimension as that of $W$, for all $T\in\GL(n,\Z)$. There are countably infinitely many $k$-dimensional rational 
 subspaces in $\R^n$ for each $k$ with $0< k< n$. Let $\mathbb{H}_k$ (resp.\ $R_k$) denote the space of all $k$-dimensional subspaces 
 (resp.\ rational subspaces) of $\R^n$. Then $\mathbb{H}_k$ is a closed (compact) proper subspace of $\Sub_{\R^n}$, and $\mathbb{H}_k$ 
 (resp.\ $R_k$) is invariant under the action of $\GL(n,\R)$ (resp.\ $\GL(n,\Z)$). 
 It is easy to see that  $R_k$ is dense in $\mathbb{H}_k$, $0\leq k\leq n$. 
 
A linear map $T\in\GL(n,\R)$ is said to be {\it proximal} if it has a unique eigenvalue of maximum absolute value; such an eigenvalue is real. 
It is well-known that if $T$ is proximal, then for any $L\in\Sub^p_{\R^n}$, $T^n(L)\to L_\alpha$, where $L_\alpha$ in $\spp_{\R^n}$ is the 
one-dimensional eigenspace corresponding to the real eigenvalue $\alpha$ with maximum absolute value. 
 
 A linear map $T\in\GL(n,\R)$ is distal (i.e.\ it acts distally on $\R^n$) if and only if all its eigenvalues have 
 absolute value $1$; this is well-known and easy to prove (see e.g.\ \cite{Mo}, \cite{CoGu} or \cite{Ab1}). Moreover, if $T\in\GL(n,\Z)$ is distal, then all its 
 eigenvalues are roots of unity and $T^m$ is unipotent for some $m\in\N$. 
 
 The following proposition will be useful for the proof of \Cref{expansive-t}. Note that the condition below that $T$ does not keep 
 any nontrivial  proper rational subspace of $\R^n$ invariant implies that $T$ does not keep any proper subtorus of $\mT^n$ invariant, 
 and hence that either $T^m=\Id$ for some $m\in\N$ or that $T$ is ergodic on $\mT^n$, $n\geq 2$; this follows from Theorems 2.3 or 
Theorem 3.15 of \cite{KiSch}, (see also Proposition 2.1 of \cite{R} or that of \cite{RSh2}), as there is a connected $T$-invariant subgroup 
$H$ such that the $T$-action on $H$ is ergodic and the corresponding automorphism $\T$ of $G/H$ is distal. Now the condition implies 
that $G=H$ and $T$ is ergodic or $G=\{e\}$  and $T$ is distal, in the later case $T^m$ is unipotent for some $m\in\N$, and hence $T^m=\Id$, 
otherwise $T$ would keep a proper rational subspace invariant.  We will later see that \Cref{rational} holds for all $T\in\GL(n,\Z)$,  
 $n\geq 2$, without this extra condition on $T$ (see \Cref{rem}). 
 
 \begin{prop} \label{rational} Let $T\in\GL(n,\Z)$, $n\ge 2$.  Suppose $T$ does not keep any nonzero proper rational subspace of $\R^n$ 
 invariant. Then there are infinitely many $(n-1)$-dimensional rational subspaces with disjoint $T$-orbits in $\Sub_{\R^n}$. 
  \end{prop}
  
 \begin{proof} For $n\geq 2$, let $T\in\GL(n,\Z)$ be such that $T$ does not keep any nonzero proper rational subspace of $\R^n$ invariant. 
 This is equivalent to the condition that  any nonzero proper $T$-invariant subspace is not contained in any proper rational subspace of $\R^n$. 
 If $V$ is a $T$-invariant subspace contained in a proper rational subspace (say) $W$ of $\R^n$, then $V'=(\ol{V+\Z^n})^0$ is a $T$-invariant rational 
 subspace of $\R^n$ and it is contained in $W$. Therefore, $V'$ is proper and the condition on $T$ as in the hypothesis 
 implies that $V'=\{0\}$, and hence that $V=\{0\}$. 

 
 Now we prove the assertion that there are infinitely many $(n-1)$-dimensional rational subspaces with disjoint $T$-orbits. It is easy to see that 
 the assertion holds for $T$ if and only if it holds for $T^m$ for any $m\in\Z\mi\{0\}$. If $T^m=\Id$ 
 for some $m\in\N$, then all the rational subspaces are $T^m$-invariant, i.e.\ the assertion holds for $T^m$, and hence, for $T$. Now suppose 
 $T^m\ne\Id$ for any $m\in\N$. Note that $T$ is not distal. For if $T$ is distal, then for some $m\in\N$, $T^m$ is unipotent, i.e.\ it has an 
 eigenvalue $1$, and its eigenspace $V_1$ is $T$-invariant, and $V_1$ is a nonzero proper rational subspace, which leads to a contradiction. 
 Thus $T$ is not distal, i.e.\ it has an eigenvalue with absolute value not equal to $1$. Since $\det(T)=\pm\,1$, $T$ has at least one eigenvalue with 
 absolute value greater than $1$, and one with absolute value less than $1$. Thus one can choose two distinct eigenvalues of $T$ with 
 different absolute values, say, $\alpha$ and $\beta$ with $|\alpha|>|\beta|$. 
 
\medskip
 \noindent{\bf Case I}: Suppose $\alpha$ and $\beta$ as above are real, with $|\alpha|>|\beta|$. Let $V_{\alpha}$ (resp.\ $V_{\beta})$ be the 
  the eigenspace corresponding to $\alpha$ (resp.\ $\beta$).  For any proper rational subspace $W$, $W\cap V_\alpha$ is a 
  $T$-invariant subspace, and hence we have that $W\cap V_\alpha=\{0\}$. Since this holds in particular for any rational  
  space $W\in R_{n-1}$,  we have that $\dim(V_\alpha)=1$. Similarly, $V_\beta\cap W=\{0\}$ for every $W\in R_{n-1}$ and 
  $\dim(V_\beta)=1$. 
  
 Let $V_{\alpha\beta}=V_{\alpha}+V_{\beta}$. Then $\dim(V_{\alpha\beta})=2$ and $T(V_{\alpha\beta})=V_{\alpha\beta}$.  
Moreover, for any rational space $W\in R_{n-1}$, 
$\dim(W\cap V_{\alpha\beta})=1$ as $V_{\alpha\beta}\not\subset W$. Let $W\cap V_{\alpha\beta}=L_w$. Note that as 
$|\alpha|>|\beta|$, both $T|_{V_{\alpha\beta}}$ and $T^{-1}|_{V_{\alpha\beta}}$ are proximal. Then 
 $$
 T^m(L_w) \to V_\alpha\ \mbox{ and }\  T^{-m}(L_w) \to V_\beta\ \mbox{ as }\ 
 m\to \infty.$$
 Now suppose $T^{m_k}(W)\to H$ for some unbounded sequence $\{m_k\}\subset\Z$. Then 
 $V_\alpha\subset H$ (resp.\ $V_\beta\subset H)$ if $m_k\to \infty$ (resp.\ if $m_k\to -\infty$). 
 Let 
 $$\mathcal{A}=\{H\in\mathbb{H}_{n-1}\mid V_{\alpha}\subset H \ \mbox{ or } \ V_{\beta}\subset H\}.$$ Then $\mathcal{A}$ is a  
 closed subset of $\mathbb{H}_{n-1}$, the space of all $(n-1)$-dimensional subspaces of $\R^n$, and 
 $\mathbb{H}_{n-1}$ is a proper closed (compact) subset of $\Sub_{\R^n}$.  Any rational subspace $W\in R_{n-1}$  does not belong to 
 $\A$ as it contains neither $V_{\alpha}$ nor $V_{\beta}$. We can show that $\mathcal{A}$ is closed using Lemma 2.2 of \cite{ShY3}. 
 Since $\mathbb{H}_{n-1}$ is a (compact) metric space and $\mathcal{A}$ is a proper closed subset of it, we can choose a neighbourhood 
 $U$ of $\mathcal{A}$ such that $\mathbb{H}_{n-1}\mi U$ has nonempty interior, and hence $\mathbb{H}_{n-1}\mi U$ contains 
 infinitely many rational subspaces from $R_{n-1}$.  Now $U$ contains all but finitely many elements of the orbit $\{T^m(W)\}_{m\in \Z}$ 
 of any rational subspace $W\in R_{n-1}$. Choose $W_1\in R_{n-1}$ such that $W_1\not\in U$ and $T^m(W_1)\in U$  
 for all $m\in\Z$ with $|m|> l_1$ for some $l_1\in\N$. Since $R_{n-1}\mi U$ is infinite, we can choose $W_2\in R_{n-1}$ such that 
 $W_2\not\in U$ and $W_2\neq T^m(W_1)$ if $|m| \leq l_1$. Now $T^m(W_2)\in U$ for all $m\in\Z$ with $|m|> l_2$, for some $l_2\in\N$. 
 Then the $T$-orbit of $W_2$ is disjoint from that of $W_1$. For if $T^m(W_1)=W_2$ for some $m\in\Z$, then $|m|> l_1$, 
 but then $T^m(W_1)\in U$ while $W_2\not\in U$, which leads to a contradiction. 
 
For $k\geq 2$ and $1\leq i\leq k$, suppose there are 
 $W_i\in R_{n-1}$ such that $W_i\not\in U$  with $T^m(W_i)\in U$ for all $m\in\Z$ with $|m|> l_i$ and $W_i\neq T^m(W_j)$ for any 
 $1\leq j< i$ and $|m|\leq l_j$. Since $R_{n-1}\mi U$ is infinite, we can choose a rational subspace $W_{k+1}$  as follows:
 $$
 W_{k+1}\in R_{n-1},\ W_{k+1}\not\in U\ \mbox{ and }\ W_{k+1}\notin\cup_{i=1}^k\{T^m(W_i)\mid  |m|\leq l_i\}.$$ 
Hence the $T$-orbit of $W_{k+1}$ is disjoint from those of $W_j$, $1\leq j\leq k$. Moreover, $T^m(W_{k+1})\in U$ for all $m\in\Z$ with 
$|m|\geq l_{k+1}$ for some $l_{k+1}\in\N$. It is easy to see that the $T$-orbit of $W_{k+1}$ is disjoint from those of $W_1,\ldots, W_k$. 
Thus by induction, there exist infinitely many $(n-1)$-dimensional rational subspaces $W_k$, $k\in\N$, whose $T$-orbits are disjoint. 

   \medskip 
   \noindent{\bf Case II}: Suppose one of the eigenvalues $\alpha$ and $\beta$ of $T$ (as above) is real, and the other is complex. Replacing 
   $T$ by $T^{-1}$ if necessary, we may assume that $\alpha$ is real with $|\alpha|>1$, and $\beta$ is complex with $|\beta|<1$.  
   Let $V_{\alpha}$ be the eigenspace for $\alpha$, and $V_{\beta}$ be the  $2$-dimensional vector subspace of $\R^n$ 
   such that  $T(V_\beta)=V_\beta$, and the eigenvalues of $T|_{V_\beta}$ are $\beta$, $\bar\beta$. Let  
    $V_{\alpha \beta}=V_{\alpha}+V_{\beta}$. Then $\dim(V_{\alpha \beta})=3$, and $T$ keeps $V_{\alpha\beta}$ invariant.  
    There exist $\phi$ and $S$ in $\GL(V_{\alpha\beta})$
  such that $T|_{V_{\alpha\beta}}=\phi S=S\phi$, $S|_{V_\alpha}=T|_{V_\alpha}=\alpha\Id$, $S|_{V_\beta}=|\beta|\Id$, 
  $\phi|_{V_\alpha}=\Id$, $\phi$ keeps $V_\beta$ invariant and $\phi|_{V_\beta}$ is contained in a compact subgroup of $\GL(V_\beta)$. 
  In particular, $\phi$ is contained in a compact subgroup of $\GL(V_{\alpha\beta})$. Since $V_\alpha$ and 
  $V_\beta$ are $T$-invariant, as noted above, neither $V_\alpha$ nor $V_\beta$ is contained in any proper rational subspace. 

For a rational vector subspace $W\in R_{n-1}$, let $L_w=W\cap V_{\beta}$. 
Then $V_\beta=L_w + L'_w$ for some one dimensional subspace $L'_w$ of $V_\beta$. 
Here, $L'_w\not\subset W$ as $V_\beta\not\subset W$. Let $V_1:=V_{\alpha}+L'_w$. Then $\dim(V_1)= 2$ and 
$V_1$ is $S$-invariant. Now $\dim(W\cap V_1)=1$ as neither $V_{\alpha}$ nor $L'_w$ is contained 
in $W$.  Let  $S_1:=S|_{V_1}$. Then $S_1$ has two eigenvalues $\alpha$ and $|\beta|$ and both $S_ 1$ and $S_1^{-1}$ 
are proximal as $|\alpha|> |\beta|$. Let $L_1= W\cap V_1$. Then $S^m(L_1)=S_1^m(L_1)\to V_\alpha$ and 
$S^{-m}(L_1)=S_1^{-m}(L_1)\to L'_w$ in $\Sub^p_{\R^n}$ as $m\to \infty$. Now for $m\in\Z$, 
$$
T^m(W\cap V_{\alpha\beta})=\phi^m S^m(L_w+L_1)=\phi^m(S^m(L_w)+S^m(L_1))=\phi^m(L_w)+\phi^m(S^m(L_1)).$$ 
 As $\phi$ is contained in a compact group, all the limit points of $\{\phi^m\}$ keep $V_\alpha$ and $V_\beta$ invariant. 
 In particular, $\phi^m(L_w)\subset V_\beta$ for all $m$. Moreover, $\phi^m(S^m(L_1))\to V_\alpha$ as $m\to\infty$. Thus all 
 the limit points of $\{T^m(W\cap V_{\alpha\beta})\}$ contain $V_\alpha$ as $m\to\infty$
 
 The limit points of $\phi^{-m}(S^{-m}(L_1))$ are $\psi(L'_w)$ as $m\to\infty$, where $\psi$ is any limit point of 
 $\{\phi^m\mid -\infty< m \leq -1\}$. Thus the limit points of  $\{T^{-m}(W\cap V_{\alpha\beta})\}$ are contained in 
 $V_\beta$ as $m\to\infty$. As
  $\dim(V_\beta)=2=\dim(W\cap V_{\alpha\beta})$, we get that $T^{-m}(W\cap V_{\alpha\beta})\to V_\beta$ as $m\to\infty$.
 
Now we have that all the limit points of $\{T^m(W)\}$ contain $V_\alpha$ as $m\to\infty$, and contain $V_\beta$ as $m\to -\infty$, i.e.\ for 
any rational subspace $W\in R_{n-1}$, if $T^{m_k}(W)\to H$ then $V_\alpha\subset H$ when $m_k\to \infty$, and $V_\beta\subset H$ when 
$m_k\to -\infty$. Consider the set $\mathcal{A}=\{H\in \mathbb{H}_{n-1}\mid V_{\alpha}\subset H\ \mbox{ or }\ V_\beta\subset H\}$. Then 
$\mathcal{A}$ is a proper closed (compact) subspace of $\mathbb{H}_{n-1}$. As $R_{n-1}$ is dense in $\mathbb{H}_{n-1}$, using similar arguments 
as in Case I, we can find infinitely many rational subspaces in $R_{n-1}$ whose $T$-orbits are disjoint. 

\medskip
 \noindent{\bf Case III}: Now suppose both the eigenvalues $\alpha$ and $\beta$ of $T$ (as above) are complex with  
 $|\alpha|>1$ and $|\beta|<1$. Let $V_{\alpha}$ (resp.\ $V_{\beta}$) be a $T$-invariant $2$-dimensional subspace such that 
 $T|_{V_\alpha}$ (resp.\ $T|_{V_\beta}$) have $\alpha$ and $\bar\alpha$ (resp.\ $\beta$ and $\bar\beta$) as eigenvalues. Let 
 $V_{\alpha \beta}=V_{\alpha}+V_{\beta}$. Then $\dim(V_{\alpha\beta})=4$ and $T$ keeps $V_{\alpha\beta}$ invariant. 
 Now we have that $T|_{V_{\alpha\beta}}=\phi S=S\phi$ where, $S|_{V_\alpha}=|\alpha|\Id$, $S|_{V_\beta}=|\beta|\Id$, and 
 each of the maps $\phi|_{V_\alpha}$ and $\phi|_{V_\beta}$ generate a relatively compact group in 
 $\GL(V_\alpha)$ and $\GL(V_\beta)$ respectively. In particular $\phi$ is contained in a compact subgroup of $\GL(V_{\alpha\beta})$.

For any rational subspace $W\in R_{n-1}$, $V_\alpha\not\subset W$ and $V_\beta\not\subset W$. Let $L_{\alpha_1}=W \cap V_{\alpha}$ and let 
$L_{\beta_1}=W\cap V_{\beta}$. Then $V_{\alpha}=L_{\alpha_1}+L_{\alpha_2}$ and 
$V_{\beta}=L_{\beta_1}+L_{\beta_2}$ for some one-dimensional subspaces $L_{\alpha_2}$ in $V_\alpha$ and $L_{\beta_2}$ in $V_\beta$. 
Let $V_2=L_{\alpha_2}+ L_{\beta_2}$. Then $\dim(V_2)=2$ and $V_2$ is $S$-invariant. Let $S_2:=S|_{V_2}$. Then 
$S_2\in \GL(V_2)$ and $S_2$ and $S_2^{-1}$ are both proximal. Let $L_2=W\cap V_2$. Then $\dim(L_2)=1$ as neither $L_{\alpha_2}$ nor $L_{\beta_2}$ 
is contained in $W$. As $S_2$ is proximal, we get that $S^m(L_2)\to L_{\alpha_2}$ and 
$S^{-m}(L_2)\to L_{\beta_2}$ in $\Sub^p_{\R^n}$ as $m\to\infty$. Now 
$$
S^m(W\cap V_{\alpha\beta})=S^m(L_{\alpha_1}+ L_{\beta_1}+L_2)=L_{\alpha_1}+L_{\beta_1}+S^m(L_2). 
$$
Therefore, as $m\to\infty$,
\begin{eqnarray*} S^m(W\cap V_{\alpha\beta})\to L_{\alpha_1}+L_{\beta_1}+L_{\alpha_2} & = & V_\alpha+L_{\beta_1},\mbox{ and }\\
S^{-m}(W\cap V_{\alpha\beta})\to L_{\alpha_1}+L_{\beta_1}+L_{\beta_2} & = &L_{\alpha_1}+V_\beta.
\end{eqnarray*}
As $\phi$ keeps $V_\alpha$ and $V_\beta$ invariant, so does every limit point of $\{\phi^m\}_{m\in\Z}$. As $\{\phi^m\}_{m\in\Z}$ is relatively compact and 
$T^m(W\cap V_{\alpha\beta})=\phi^mS^m(W\cap V_{\alpha\beta})$ for all $m\in\Z$, we have that limit points of $\{T^m(W\cap V_{\alpha\beta})\mid m\in\N\}$ 
contain $V_\alpha$ and limit points of $\{T^{-m}(W\cap V_{\alpha\beta})\mid m\in\N\}$ contain $V_\beta$. 
Thus limit points of $\{T^m(W)\mid m\in\N\}$ contain $V_\alpha$ and $\{T^{-m}(W)\mid m\in\N\}$ contain $V_\beta$ for every rational subspace $W\in R_{n-1}$. 

Let $\mathcal{A}=\{H\in \mathbb{H}_{n-1} \mid V_{\alpha}\subset H \ \mbox{ or } \ V_{\beta}\subset H\}$. Then $\mathcal{A}$ is a proper closed (compact) subset 
of $\mathbb{H}_{n-1}$. As $R_{n-1}$ is dense in $\mathbb{H}_{n-1}$, using similar arguments as in Case I, we can find infinitely many rational subspaces in $R_{n-1}$ 
whose $T$-orbits are disjoint. 
 \end{proof}
 
Now we prove \Cref{expansive-t} where we will use \Cref{rational} for a particular class of automorphisms. 

 
\begin{proof}[Proof of \Cref{expansive-t}] Let $G=\mT^n$, the $n$-torus, for any $n\geq 2$ and let $T\in\Aut(G)$. We want to show
that the $T$-action on $\spg$ is not expansive. If $T^m=\Id$ for some $m\in\N$, then by \Cref{expa-prop}, 
the $T$-action on $\spg$ is not expensive as $\spg$ is infinite. Now suppose that $T^m\ne\Id$ for any $m\in\N$. 

Recall that $\mfS_{n-1}$ is the collection of all $(n-1)$-dimensional subtori of $G=\mT^n$, and $\mfH_{n-1}= \mathfrak{S}_{n-1}\cup \{G\}$, 
both of these sets are $T$-invariant subsets of $\spg$, and $\mfH_{n-1}$ is closed in $\spg$, and hence it is compact. Moreover, 
every $H\in \mathfrak{S}_{n-1}$ is isolated in $\mfH_{n-1}$ (cf. \Cref{isol}); in particular, $\{H\}$ is open in $\mfH_{n-1}$. Note that for every $H$ 
in $\mfS_{n-1}$, $T^m(H)\to G$, if $m\to \pm\infty$ unless $T^m(H)=H$ for some $m\in\N$. However, since $\spg$ is countable, 
we are not able to use Theorem 1 of \cite{Re}, even if $T^m(H)\ne H$, $m\in\N$, for infinitely many $H\in\mfS_{n-1}$. 

\medskip
\noindent{\bf Step 1:} Let ${\bf d}$ be the metric on $\spg$. If possible, suppose that the $T$-action on $\spg$ is expansive. Then the 
$T$-action on $\mfH_{n-1}$ is also expansive; suppose $\epsilon>0$ is an expansive constant for this action. For any $H\in \mfS_{n-1} $, 
 there exists $m\in \Z$ such that ${\bf d}(T^m(H),G)>\epsilon$. Let $B(G,\epsilon)$ be the ball of radius $\epsilon$ centered at $G$ in $\spg$.
 Consider the collection $\{\{H\}\mid H\in\mfS_{n-1}\}\cup \{B(G, \epsilon)$\};  
it is an open cover of $\mfH_{n-1}$. As $\mfH_{n-1}$ is compact, there exist $H_1, \dots,H_k$ in $\mfS_{n-1}$ such that 
 $\mfH_{n-1}=B(G,\epsilon)\cup \{H_1\}\cup \dots \cup \{H_k\}$. As the $T$-action on $\spg$ is expansive with an expansive constant $\epsilon$, 
 for any $H\in \mathfrak{S}_{n-1}$, there exist $m\in \Z$ (which depends on $H$) such that $T^m(H)=H_i$ for some $i \in \{1,\dots,k\}$. 
This implies that $T$ has finitely many orbits in $\mfS_{n-1}$. We will now show that $T$ has infinitely many disjoint orbits in $\mfS_{n-1}$, 
which would contradict the expansivity of $T$.  

\medskip
\noindent{\bf Step 2:} Suppose that all the eigenvalues of $T$ have absolute value $1$, i.e.\ $T$ is distal. 
As $T\in\GL(n,\Z)$, some power of $T$ is unipotent (see e.g.\ Lemma 2.5 of \cite{Ab2}). The statement that there are infinitely many 
$(n-1)$-dimensional subtori with disjoint $T$-orbits is equivalent to the statement that there are infinitely many $(n-1)$-dimensional subtori 
with disjoint $T^m$-orbits for any $m\in\N$. Without loss of any generality, we may assume that $T$ is unipotent and that $T\neq \Id$. 
By Proposition 3.10 of \cite{ShY3}, there exist nontrivial  closed connected $T$-invariant subgroups 
$\{e\}=K_0\subsetneq K_1\cdots\subsetneq K_{l+1}=G$ of $\mT^n$ such that $T$ acts trivially on $K_i/K_{i-1}$, $1\leq i\leq l+1$ 
and $T$ does not act trivially on $K_i/K_{i-2}$, $2\leq i\leq l+1$ (see also 
Lemma 2.5 of \cite{Ab2}). 
Note that $l\ne 0$ as $T\neq\Id$. If $\dim(K_l)\leq n-2$, then $\dim(G/K_l)\geq 2$, and we can choose infinitely many distinct tori 
(closed connected subgroups) $B_m$ of co-dimension $1$ in $G/K_l$, $m\in\N$. Let $H_m$ be a subtorus in $G$ containing $K_l$ such that 
$H_m/K_l=B_m$ for each $m$. Then $\dim(H_m)=n-1$ and $H_m$ is $T$-invariant for every $m$. Moreover, $H_m$'s are distinct as $B_m$'s 
are so and $\{H_m\}$ is a $T$-orbit in $\mfS_{n-1}$ for each $m$. 

 Now suppose $\dim(K_l)=n-1$. There exists a closed one-parameter subgroup $C_0=\{x_t\}$ such that $G=C_0\times K_l$. Now
$T(x_t)=x_ty_t$, $t\in\R$, for some nontrivial closed one-parameter subgroup $\{y_t\}\subset K_l$. For $m\in \N$, let $C_m :=\{x_t y_{t/m}\}$.

Suppose $\dim(K_l)=1$. Then $l=1$ and $T|_{K_l}=\Id$.  Then $T(\{x_t\})=\{x_t y_t\}$ and $T^k(\{x_t\})=\{x_t y_t^k\}=\{x_t y_{kt}\}$ 
and the $T$-orbit of $C_0$ is $\{\{x_t y_{kt}\}\mid k\in \Z\}$. Moreover, $T^k(C_m)=\{x_t y_{(k+1/m)t}\}$ and 
the $T$-orbit of $C_m$ is $\{\{x_t y_{(k+1/m)t}\} \mid k\in \Z\}$. Observe that the $T$-orbits of $C_m$'s are disjoint. 

Now suppose $\dim(K_l)\geq 2$. We can choose a $T$-invariant subtorus $H$ such that $K_{l-1}\subset H\subset K_l$ and 
$K_l=\{y_t\}\times H$. Here, $H=K_{l-1}$ if $\dim(K_l/K_{l-1})=1$. Note that $T(y_t)\in y_tK_{l-1}\subset y_tH$, $t\in\R$. 
Let $H_m:=C_mH$, $m\in\N$. Then $\dim(H_m)=n-1$ and $H_m\in\mfS_{n-1}$, $m\in\N$. As $H$ is $T$-invariant, it is easy to see that the 
$T$-orbits of $H_k$ and $H_m$ are disjoint if $k\neq m$. Thus, if $T$ is not distal then $T$ has infinitely many disjoint orbits in $\mfS_{n-1}$.

\medskip
\noindent{\bf Step 3:} Now suppose $T$ admits an eigenvalue with absolute value other than $1$, i.e.\  $T$ is not distal. 
Let $\pi:\R^n\to \mT^n$ be the natural projection with $\ker\pi=\Z^n$. Then we have $T\in\GL(n,\Z)$ 
as a linear automorphism of $\R^n$ with $\pi\circ T=T\circ\pi$. Recall that $R_{n-1}$, the set of all $(n-1)$-dimensional rational subspace 
of $\R^n$, is a $T$-invariant subspace of $\Sub_{\R^n}$, then $\pi(R_{n-1})=\mfS_{n-1}$, and the map $R_{n-1}$ to $\mfS_{n-1}$
induced by $\pi$ is bijective. 

Suppose $T$ does not keep any nontrivial proper rational subspace of $\R^n$ invariant. Then by \Cref{rational} in this case, there are infinitely 
many rational subspaces in $R_{n-1}$ with disjoint $T$-orbits. Then there are infinitely many $(n-1)$-dimensional tori in $\spt$ with disjoint $T$-orbits. 

Suppose $n=2$, i.e.\ $\dim(G)=2$. Then since $T$ does not have an eigenvalue of absolute value $1$, both the eigenvalues of $T$ are real 
with absolute value other than $1$. Thus $T$ does not keep any nontrivial proper rational subspace of $\R^2$ invariant (as such a space 
would have dimension $1$, and it would mean that eigenvalues of $T$ have absolute value $1$). Now by \Cref{rational}, $T$ has infinitely many 
$1$-dimensional rational subspaces with disjoint $T$-orbits. Thus the assertion that $T$ has infinitely many co-dimension one subtori with disjoint 
$T$-orbits holds for all $T\in\Aut(G)$ for $n=2$. 

Suppose $n\geq 3$. Suppose the assertion that every $T\in\Aut(G)$ has infinitely many disjoint orbits in $\mfS_{k-1}$ 
holds for any torus $G$ with dimension $k$ such that $1<k<n$. Now suppose $G$ is such that $\dim(G)=n$. If $T$ is distal, then 
the assertion holds as shown in Step 2. If $T$ does not keep any nontrivial proper rational 
subspace of $\R^n$ invariant, equivalently, if $T$ does not keep any proper subtori invariant, then the assertion holds as shown above. 

Now suppose that $T$ keeps a nonzero proper rational subspace of $\R^n$ invariant. Then 
there is a proper subtorus $H$ on $G$ such that $T(H)=H$. Then $\dim(H)<n$. Let $\T\in\Aut(G/H)$ be the automorphism induced by $T$. 
Suppose the dimension of $G/H$ is greater than or equal to $2$. Since $\dim(G/H)<n$, by the induction hypothesis, $G/H$ has infinitely many 
subtori (say) $B_m$ of co-dimension $1$ with disjoint $\T$-orbits. Let $H_m$ be the subtori of $G$ containing $H$ such that $H_m/H=B_m$. 
Then $\dim(H_m)=n-1$ and $H_m$'s have disjoint $T$-orbits. 

Now suppose $\dim(G/H)=1$. Then the corresponding automorphism $\T$ of $G/H$, and hence, $T$ has a real eigenvalue 
which is $\pm 1$. We can choose a one-dimensional subtorus, say, $M$ of $G$ which is $T$-invariant. Now 
$\dim(G/M)=n-1\geq 2$. Thus arguing as above for $M$ instead of $H$, we have that $G$ has infinitely many 
subt-tori of dimension $n-1$ with disjoint $T$-orbits. By induction, the assertion that there are infinitely many subtori of 
co-dimension $1$ in $G$ with disjoint $T$-orbits holds. 

As noted at the end of Step 1, it follows that the $T$-action on $\spg$ is not expansive. 
\end{proof}

The proof of  \Cref{expansive-t} actually proves a stronger statement that the $T$-action on $\mfH_{n-1}$ is not expansive, where 
$\mfH_{n-1}$ consists of the whole group $G$ and all the $(n-1)$-dimensional subtori of $G$. 

\begin{rem} \label{rem} As shown in the proof of \Cref{expansive-t}, any $T\in\GL(n,\Z)$ admits infinitely many 
subtori of co-dimension 1 in $\mT^n$ with disjoint $T$-orbits. This is equivalent to the statement that $T$ has infinitely many disjoint 
$T$-orbits in $R_{n-1}$, the space of $(n-1)$-dimensional rational subspaces of $\R^n$. Thus \Cref{rational} holds for any $T\in\GL(n,\Z)$ without the 
condition on $T$ mentioned there. 
\end{rem}

The following corollary follows easily from \Cref{expansive-t}. We give a proof for the sake of completeness. 

\begin{cor} \label{Lie-t}
Let $G$ be a connected Lie group such that it contains a central torus of dimension at least $2$. Then $G$ does not admit any automorphism 
that acts expansively on $\spg$.
\end{cor}

\begin{proof} Let $C$ be the maximal central torus in $G$. By the hypothesis, the dimension of $C$ is at least $2$. Let $T\in\Aut(G)$. 
Then $T(C)=C$ and by \Cref{expansive-t}, $T$ does not act expansively on $\Sub^p_C$. As $\Sub^p_C$ is a closed subspace of $\spg$, 
by \Cref{expa-prop}\,(5), we get that $T$ does not act expansively on $\spg$. 
\end{proof}

\noindent{\bf Acknowledgments.} We thank Amala Bhave for valuable discussions.  Debamita Chatterjee acknowledges  
the UGC Non-NET research fellowship, and Himanshu Lekharu acknowledges the CSIR-JRF research fellowship from CSIR, Govt.\ of India.

{\advance\baselineskip-0.5mm
\begin{thebibliography}{99}
\bibitem{Ab1} H.\ Abels. Distal affine transformation groups. {\it J.\ Reine Angew.\ Math.}  299/300 (1978), 294--300.

\bibitem{Ab2} H.\ Abels. Distal automorphism groups of Lie groups.  {\it J.\ Reine Angew.\ Math.}  329 (1981), 82--87.

\bibitem{BC1} H.\ Baik and L.\ Clavier. The space of geometric limits of one-generator closed subgroups of
$\PSL_2(\R)$.  {\it Algebr.\ Geom.\ Topol.} 13 (2013), 549--576.

\bibitem{BC2} H.\ Baik and L.\ Clavier. The space of geometric limits of abelian subgroups of $\PSL_2(\C)$.
{\it  Hiroshima Math.\ J.} 46 (2016), 1--36.

\bibitem{BP} R.\ Benedetti and C.\ Petronio, {\it Lectures on Hyperbolic Geometry}. Universitext, Springer, Berlin, 1992.

\bibitem{Be} D.\ Berend. Multi-invariant sets on tori. {\it Trans.\ AMS} 280 (1983), 509--532.

\bibitem{BrHK} M.\ R.\ Bridson, P.\ de la Harpe and V.\ Kleptsyn. The Chabauty space of closed subgroups of the
three-dimensional Heisenberg group.  {\it Pacific J.\ Math.} 240 (2009), 1--48.

\bibitem{Bry} B.\ F.\ Bryant.  On expansive homeomorphisms. {\it Pac.\ J.\ Math.} 10 (1960), 1163--1167. 

\bibitem{CSh} D.\ Chatterjee and R.\ Shah. Characterisation of distal 
actions of automorphisms on the space of one-parameter subgroups of Lie groups. 
{\it J.\ Austral.\ Math.\ Soc.} 119 (2025), 152--175.

\bibitem{ChoR} M.\ Choudhuri, G.\ Faraco and A.\ K.\ Yadav. On the distality and expansivity of certain maps on
spheres. {\it Dyn. Syst}. 39 (2024), 166--181.

\bibitem{CoGu} J.\ P.\ Conze and Y. Guivarc'h. Remarques sur la distalit\'e dans les espaces vectoriels.  {\it C.\ R.\ Acad.
Sci.\ Paris Ser.\ A 278} (1974), 1083--1086.

\bibitem{E} R.\ Ellis. Distal transformation Groups.  {\it Pacific J.\ Math.} 8 (1958), 401--405.

\bibitem{FP} J.\ Fresnel and M.\ van der Put. Compact subgroups of $\GL(n,\C)$. 
 {\it Rend.\ Semin.\ Mat.\ Univ.\ Padova} 116 (2006), 187--192.

\bibitem{Fu} H.\ Furstenberg. The structure of distal flows. {\it  Amer.\ J.\ Math.} 85 (1963), 477--515.

\bibitem{GR} H.\ Gl\"ockner and C.\ R.\ E.\ Raja. Expansive automorphisms
of totally disconnected locally compact groups.  {it J.\ Group Theory} 20 (2017), 589--619.

\bibitem{HaKa} H.\ Hamrouni and B.\ Kadri. On the compact space of closed subgroups of locally compact groups,
 J.\ Lie Theory 24 (2014), 715--723.

\bibitem{KiSch} B.\ Kitchens and K.\ Schmidt. Automorphisms of compact groups. {\it  Ergod.\ Th.\ \&  Dynam.\ Sys.} 9 (1989),
691--735.

\bibitem{Mo} C.\ C.\ Moore. Distal affine transformation groups.  {\it Amer.\ J.\ Math.} 90 (1968), 733--751.

\bibitem{PaSh} R.\ Palit and R.\ Shah. Distal actions of automorphisms of nilpotent groups $G$ on Sub$_G$ and
applications to lattices in Lie groups,  {\it Glasg.\ Math.\ J.} 63 (2021) 343--362.

\bibitem{PaPrSh} R.\ Palit. M.\ B.\ Prajapati and R.\ Shah.  Dynamics of actions of automorphisms of discrete groups
$G$ on Sub$_G$ and applications to lattices in Lie groups,  {\it Groups, Geom.\ Dyn.} 17 (2023), 185--213.

\bibitem{PHu} I.\ Pourezza and J.\ Hubbard. The space of closed subgroups of $\R^2$. {\it Topology} 18 (1979), 143--146.

\bibitem{PrSh} M.\ B.\ Prajapati and R.\ Shah. Expansive actions of automorphisms of locally compact groups
$G$ on Sub$_G$. {\it Monatsh.\ Math.} 193 (2020), 129--142. 

\bibitem{R} C.\ R.\ E.\ Raja. On the existence of ergodic automorphisms in ergodic $\Z^d$ -actions on compact groups. {\it Ergod.\
Th.\ \& Dynam.\ Sys.}  30 (2010), 1803--1816.

\bibitem{RSh1} C.\ R.\ E.\ Raja and R.\ Shah. Distal actions and shifted convolution property.  {\it Israel J.\ Math.} 177
(2010), 301--318.

\bibitem{RSh2} C.\ R.\ E.\ Raja and R. Shah. Some properties of distal actions on locally compact groups. 
{\it Ergod.\ Th.\  \& Dynam.\ Sys.}  39 (2019), 1340--1360.

\bibitem{Re} W.\ L.\ Reddy. Pointwise expansion homeomorphisms.  {\it J.\ Lond.\ Math.\ Soc.}  2  (1970), 232--236

\bibitem{Sh1} Orbits of distal actions on locally compact groups.  {\it J.\ Lie Theory} 22 (2010), 586--599.

\bibitem{Sh2} R.\ Shah. Expansive Automorphisms on locally compact groups. {\it New York J.\ Math.} 26 (2020), 285--302

\bibitem{ShY3}  R.\ Shah and A.\ K.\ Yadav. Distal action of automorphisms of Lie groups $G$ on
Sub$_G$. {\it Math.\ Proc.\ Camb.\ Phil.\ Soc.} 173 (2022), 457--478.

\bibitem{W} P.\ Walters. {\it An Introduction to Ergodic Theory}. GTM 79. Springer, New York, 1982.

\end{thebibliography}}
\end{document}
