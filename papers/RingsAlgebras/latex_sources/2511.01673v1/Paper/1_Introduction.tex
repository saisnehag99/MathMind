\section{Introduction}

% Motivation

In this paper, we study the \emph{Ideal Chomp Game}, a two-player game whose ``playing field'' is a given fixed ring $R$. The origins of the Ideal Chomp Game lie in the (classical) Chomp Game formulated by David Gale \cite{gale} in 1974. In the classical Chomp Game two players, Alice (A) and Bob (B) chomp pieces of a rectangular chocolate bar. More precisely, they choose a piece and chomp all pieces above and to the right of it (see Figure \ref{fig:chomp}). The player who eats the last piece of chocolate, which according to the rules of the game is always the lowest and leftmost piece, loses the game.

\begin{figure}[h!]
% \label{ChompGame}
    \centering
    \includegraphics[width=0.9\textwidth]{Z_Abbildungen/Spiel1_untenabgeschnitten.jpg}  % Adjust path and width as needed
    \caption{A game of Chomp where player A (red) loses.}
    \label{fig:chomp}
\end{figure}

The Ideal Chomp Game is in spirit similar to the Chomp Game, but played on the algebraic structure of a ring, which allows for a geometric interpretation. It was first studied by C.W. Henson in his paper \cite{henson} in 1970 and is played according to the following set of rules: First, a ring $R$ is fixed on which the two players $A$ and $B$ decide to play the Ideal Chomp Game. The players then start with the zero ideal $I_0 = (0)$ and in each turn increase the ideal $I_{n-1}$ strictly by adding a generator to obtain $I_{n}$. The player who increases the ideal to the entire ring loses. On the geometric side, the move of adding the element $a$ to the set of generators corresponds to the intersection of the current subscheme $$S_{n-1} = \Spec R/I_{n-1} \hookrightarrow \Spec R$$ corresponding to the ideal $I_{n-1}$ with the subscheme corresponding to the ideal $(a)$ . The player who turns $S_n$ into the empty subscheme, i.e. takes the last point away, loses the game. Notice the similarity to the classical Chomp Game.

Indeed, the Classical Chomp Game on a rectangle of size $a \times b$ is essentially the same as the Ideal Chomp Game on the ring $K[x,y]/(x^a, y^b)$ if we additionally restrict the possible moves of the players from polynomials to monomials.

\begin{Ex}
    On $\BZ$ a possible game could look like the following:
    First, Alice plays $f_1 = 100$. Next, Bob plays $f_2 = 2 \in \BZ \setminus (100)$. Now whatever element $f_3 \in \BZ \setminus (2)$ Alice chooses, she will lose since $(f_1,f_2,f_3)$ will generate $\BZ$. Clearly, Alice could have won the Ideal Chomp Game on $\BZ$ by starting with a prime.
\end{Ex}

The main result of our paper is the following complete classification of the Ideal Chomp Game on $\bar{K}$-algebras $R$ up to rank $6$.

%%%
\iffalse
\begin{Ex}
    On the ring $R = K[x,y]/(x,y)^2$ the game could play out as follows: First, Alice chooses $f_1 = x + y \in R \setminus \{0\}$. Then Bob chooses $f_2 = 2x + y \in R \setminus (x+y)$, so $S_2 = (x,y)$. No matter what Alice plays next, she loses.
\end{Ex}
\fi
%%%

\enlargethispage{3mm}

\begin{Thm}
    [Classification of the Ideal Chomp Game on $\bar{K}$-algebras up to rank 6]
    \label{ClassificationThm1}
    Alice has a winning strategy in the Ideal Chomp Game on any $\bar{K}$-algebra $R$ up to rank 6, except if $R$ is isomorphic to \vspace{-3mm}
    \begin{align*}
        R_1 & = \bar{K},\\
        R_4 & = \bar{K}[x,y]/(x,y)^2,\\
        R_{12} & =  \bar{K}[x,y]/(xy,x^3,y^3),\\
        R_{13} & = \bar{K}[x,y]/(x^2,xy^2, y^3),\\
        R_{17} & = \bar{K}[x,y,z,w]/(x,y,z,w)^2.
    \end{align*}
\end{Thm}

\pagebreak
