

\documentclass[11pt,a4paper,reqno]{amsart}
\usepackage{amsmath,amssymb,amsthm,amsfonts}
\usepackage[export]{adjustbox}
\usepackage{amsbsy}
\usepackage[utf8]{inputenc}
\usepackage[normalem]{ulem}
\usepackage{xspace}
\usepackage{cancel}
\usepackage[greek,english]{babel}
\usepackage{url}
\usepackage{wrapfig}
\usepackage{enumitem}
\usepackage{multicol}
\usepackage{moreenum}
\usepackage{xcolor}
\usepackage{longtable}
\usepackage{array}
%\usepackage{amsthm}
%\usepackage{cleveref}


\usepackage[pagewise]{lineno} %\linenumbers
\newcounter{dummy}
\makeatletter
\newcommand\myitem[1][]{\item[#1]\refstepcounter{dummy}\def\@currentlabel{#1}}
\makeatother
\usepackage{subcaption}
\captionsetup[subfigure]{labelfont=rm} % use lower case letters for subfigures
\usepackage{comment}
\usepackage{tikz}
\usetikzlibrary{backgrounds}
\usepackage[bottom=2.5cm,top=2.5cm, left=2.5cm, right=2.5cm]{geometry}

\usepackage{multirow}
\usepackage{array,multirow}

\definecolor{LinkColor}{rgb}{0,0,0} %black
\usepackage[colorlinks=true,linkcolor=LinkColor,citecolor=LinkColor,urlcolor= LinkColor, naturalnames, hyperindex, pdfstartview=FitH, bookmarksnumbered, plainpages]{hyperref}
\usepackage{cleveref}

\makeatletter
\newcommand{\slunlhd}{%
	\mathrel{\mathpalette\sl@unlhd\relax}%
}

\newcommand{\sl@unlhd}[2]{%
	\sbox\z@{$#1\lhd$}%
	\sbox\tw@{$#1\leqslant$}%
	\dimen@=\ht\tw@
	\advance\dimen@-\ht\z@
	\ifx#1\displaystyle
	\advance\dimen@ .2pt
	\else
	\ifx#1\textstyle
	\advance\dimen@ .2pt
	\fi
	\fi
	\ooalign{\raisebox{\dimen@}{$\m@th#1\lhd$}\cr$\m@th#1\leqslant$\cr}%
}
\makeatother


\newtheorem{innercustomthm}{Theorem}[section]
\newenvironment{customthm}[1]
{\renewcommand\theinnercustomthm{#1}%
	\begin{innercustomthm}}
	{\end{innercustomthm}}
\crefname{innercustomthm}{Theorem}{Theorems}




\newtheorem{maintheorem}{Theorem}[section]
\newtheorem{maincorollary}[maintheorem]{Corollary}
\newtheorem{mainquestion}[maintheorem]{Question}

\renewcommand\themaintheorem{\Alph{maintheorem}}

\newtheorem{theorem}{Theorem}[section]
\newtheorem{corollary}[theorem]{Corollary}
\newtheorem{lemma}[theorem]{Lemma}
\newtheorem{proposition}[theorem]{Proposition}




\theoremstyle{definition}
\newtheorem{definition}[theorem]{Definition}
\newtheorem{remark}[theorem]{Remark}
\newtheorem{example}[theorem]{Example}

\newcommand{\ChAngel}[1]{{\textcolor{blue}{#1}}}
\newcommand{\ChAnn}[1]{{\textcolor{wildstrawberry}{#1}}}
\newcommand{\ChAndreas}[1]{{\textcolor{olive}{#1}}}
\newcommand{\ChSugandha}[1]{{\textcolor{red}{#1}}}


\newcommand{\cut}{\textsf{cut}\xspace}
\newcommand{\SG}[1]{\text{SG}[#1]}
\newcommand{\SL}{\operatorname{SL}}
\newcommand{\GL}{\operatorname{GL}}
\newcommand{\PSL}{\operatorname{PSL}}
\newcommand{\PGL}{\operatorname{PGL}}
\newcommand{\rL}{\operatorname{PGL}}
\newcommand{\Irr}{\operatorname{Irr}}
\newcommand{\Aut}{\operatorname{Aut}}
\newcommand{\Out}{\operatorname{Out}}
\newcommand{\Inn}{\operatorname{Inn}}
\newcommand{\Syl}{\operatorname{Syl}}
\newcommand{\Ker}{\operatorname{Ker}}
\newcommand{\FF}{\mathbb{F}}
\newcommand{\Z}{\textup{Z}}
\newcommand{\U}{\textup{U}}
\newcommand{\V}{\textup{V}}
\newcommand{\C}{\textup{C}}
\newcommand{\N}{\textup{N}}
\newcommand{\ZZ}{\mathbb{Z}}
\newcommand{\QQ}{\mathbb{Q}}
\newcommand{\GK}{\Gamma_{\textup{GK}}}
\renewcommand{\O}{\operatorname{O}}
\newcommand{\F}{\textup{F}}
\newcommand{\Frob}{\textup{Fr}} 
\newcommand{\tr}{\textup{tr}} 
\newcommand{\PQ}{(\textup{PQ})\xspace}

\setlength\parindent{10pt}

\newtheorem*{theoremA1}{Theorem A[(i)]}
\newtheorem*{theoremA2}{Theorem A[(ii)]}
\newtheorem*{theoremB1}{Theorem B[(i)]}
\newtheorem*{theoremB2}{Theorem B[(ii)]}

\newcommand{\GEN}[1]{\langle #1 \rangle}
\newcommand{\pmatriz}[1]{\left(\begin{array} #1 \end{array}\right)}
\newcommand{\qand}{\quad \text{and} \quad}
\newenvironment{proofof}{\par\noindent \textit{Proof of }}{\qed\par\bigskip}


\DeclareMathOperator{\diag}{diag}
\DeclareMathOperator{\lcm}{lcm}


\author[S. Chahal]{Seema Chahal}
\address{(Seema Chahal) Department of Mathematics, Indian Institute of Technology Roorkee, Roorkee (Uttarakhand)-247667, India.}
\email{\href{mailto:seema_r@ma.iitr.ac.in}{seema\_r@ma.iitr.ac.in}}
\author[S. Maheshwary]{Sugandha Maheshwary}
\address{(Sugandha Maheshwary) Department of Mathematics, Indian Institute of Technology Roorkee, Roorkee (Uttarakhand)-247667, India.}
\email{\href{mailto:msugandha@ma.iitr.ac.in}{msugandha@ma.iitr.ac.in}}





\thanks{The second author gratefully acknowledges the support by Science  \& Engineering Research Board (SERB),  DST (Department of Science and Technology), India (SRG/2023/000180).}

\keywords{finite semisimple group algebras, primitive central idempotents, group codes.}

% 94B60(1980–now)Other types of codes
%94B05(1980–now)Linear codes (general theory)
%11T71(1980–now)Algebraic coding theory; cryptography (number-theoretic aspects)
%94B65(1991–now)Bounds on codes
%16S34(1991–now)Group rings [See also 20C05, 20C07], Laurent polynomial rings (associative algebraic aspects)
%20C05(1973–now)Group rings of finite groups and their modules 

\subjclass[2010]{16S34, 20C05, 11T71, 94B05, 94B65}
\date{}

\title{On metacyclic $p$-group codes }
\begin{document}
	\maketitle
	\begin{abstract}
		In this article, we study the metacyclic $p$-group codes arising from finite semisimple group algebras. In \cite{CM25}, we studied group codes arising from metacyclic groups with order divisible by two distinct odd primes. In the current work, we focus on metacyclic $p$-group codes, as a result of which we are also able to extend the results of \cite{CM25} for metacyclic 	groups with order divisible by any two primes, not necessarily odd or distinct. Consequently, existing results on group algebras of some important classes of groups, including dihedral and quaternion groups, have been extended. Additionally, we provide left codes for the undertaken group algebras. %The work in this article substantially contributes to the study of  $p$-group codes, basing upon metacyclic $p$-group codes.
		 Finally, we construct non-central codes using units motivated by  Bass and bicyclic units, which  are inequivalent to any abelian group codes and yield  best known parameters.
	\end{abstract}
	
	\section{Introduction}
	The theory of group codes, initiated by Berman \cite{Ber67} and MacWilliams~\cite{Wil70}, studies ideals of semisimple group algebras. Since these ideals are determined by idempotents, their explicit description plays a central role in the construction and analysis of group codes. In particular, primitive central idempotents (pcis) yield information about central codes. Group codes form a rich class of linear codes.  For instance, cyclic codes can be understood as group codes arising from cyclic groups, Reed-Solomon codes over field $\mathbb{F}_p$ are group codes of elementary abelian $p$-groups \cite{Pas88}; the binary Golay code $[24,12,8]$ can be obtained as an ideal in a group algebra over a finite field~\cite{Wol80}.  
	
	In group codes, abelian group codes are well studied and cover many classical families of linear codes. However non-abelian codes are also of interest because of their potential applications in code-based cryptography (\cite{DK15}, \cite{DK16}). Among the non-abelian group codes, metacyclic codes form an asymptotically good family of codes \cite{BMS20}. Particularly for dihedral  codes over $\mathbb{F}_q$, Dutra et al.~\cite{DFPM09} investigated the codes under certain restrictions on $q$. Under similar restrictions, Assuena and Milies~(\cite{APM17}, \cite{APM19}, \cite{Ass22}) considered split metacyclic groups of order $p_1^m p_2^n$, where $p_1$ and $p_2$ are distinct odd primes. They also proposed constructions of certain non-central codes with good parameters. Gupta and Rani~(\cite{GR22b}, \cite{GR22a}, \cite{GR23}) applied the theory of strong Shoda pairs to obtain pcis for dihedral groups and constructed corresponding codes, again under restrictive hypotheses on $q$. More recently, Vedenev~\cite{Ved25} carried out a comprehensive study of group codes from non-abelian split metacyclic group algebras.  
	
	In our earlier work~\cite{CM25}, we mainly worked with the pcis of $\mathbb{F}_qG$, where $G$ is a metacyclic group of order $p_1^m p_2^l$, with $p_1$ and $ p_2$  distinct odd primes. Unlike the previously existing work as cited above, we assumed almost no restriction on $q$ and hence extended the known results in this direction. In the current article, we further extend the results of \cite{CM25} by including the cases where $p_1$ and $p_2$ are any primes, not necessarily odd or distinct. This is done by studying the pcis in metacyclic $p$-group algebras, for any prime $p$. In particular, for metacyclic $p$-groups which  have a maximal cyclic subgroup, we also obtain the structures for their respective group algebras. Consequently, we improve several existing results on dihedral $2$-codes as well (cf.~\cite{DFPM09}, \cite{GR22b}, \cite{GR22a}). Furthermore, the results are generalised for metacyclic groups with order divisible by more than two primes and dihedral as well as Quaternion groups of arbitrary orders. 
	
	Throughout the article, we consider various kinds of groups. For each of these groups, we compute a complete set of pcis as well as left idempotents and study all the parameters for the associated group codes. We also provide an $\mathbb{F}_q$-basis for these codes. 
	The computation of pcis is based on strong Shoda pair theory and wherever possible, we provide unified treatment to the codes depending upon the type of the corresponding  strong Shoda pairs. This also facilitates us to give the explicit structure of the considered group algebras. Finally, we construct non-central codes via conjugation of idempotents with suitable units, motivated by well known Bass and bicyclic units of $\mathbb{Z}G$. Hence, we obtain non-central codes with improved parameters as compared to central codes.  
	We include illustrations through explicit construction of codes whose parameters are at par with the best known linear codes.
	

	
	

	\section{Notation and Preliminaries}
	Throughout the article, we use the notation which is in accordance with \cite{CM25}. For better accessibility, we restate the notation and include some fundamental results in this section. 
	
	Let $\mathbb{F}_q$ denote the field with $q$ elements and let $\mathbb{F}_qG$ be the finite semisimple group algebra of a group $G$ over $\mathbb{F}_q$, so that $q$ is relatively prime to $|G|$, the order of $G$. For $\alpha = \sum_{g \in G} \alpha_g g \in \mathbb{F}_qG$, the weight of $\alpha$ is cardinality of the set $\{ g \in G \mid \alpha_g \neq 0 \}$ and is denoted by  $wt(\alpha)$. The Hamming distance between $\alpha$ and $\beta=\sum_{g \in G} \beta_g $ in $\mathbb{F}_qG$ is $d(\alpha,\beta) = |\{ g \in G \mid \alpha_g \neq \beta_g \}|$, which satisfies $d(\alpha,\beta) = wt(\alpha - \beta)$, and hence $wt(\alpha) = d(\alpha,0)$. The weight of an ideal $I \subseteq \mathbb{F}_qG$ is defined as  $\min\{ wt(\alpha) \mid \alpha \in I, \ \alpha \neq 0 \}$. As  stated in the introduction, group codes are nothing but the ideals of $\mathbb{F}_qG$, which are determinable via their idempotents. If $e$ is a pci of $\mathbb{F}_qG$, then $\mathbb{F}_qG e$ is the corresponding central linear $[n,k,d]$ code, where  $n = |G|$, $k = \dim_{\mathbb{F}_q}(\mathbb{F}_qG e)$, the $\mathbb{F}_q$
	dimension of $\mathbb{F}_qG e$ and $d = d(\mathbb{F}_qG e),$ the weight of $\mathbb{F}_qG e$.  
	
	Denote the set of irreducible characters of $G$ over $\mathbb{F}_q$  by $\Irr(G)$.
	If $H$ and $K$ are subgroups of $G$ such that $H/K$ is cyclic, then for a generator $\gamma \in \Irr(H/K)$, the $q$-cyclotomic coset of $\gamma$ is given by $C_q(\gamma)=\{\gamma,\gamma^q,\gamma^{q^2},...,\gamma^{q^{o-1}}\},$ where $o$ is the multiplicative order of $q$ modulo $|H/K|$. Let $\mathcal{C}(H/K)$ be the set of $q$-cyclotomic cosets of $\Irr(H/K)$ containing the generators of $\Irr(H/K)$. The action $g * C = g^{-1} C g, \quad g \in N_G(H) \cap N_G(K),\; C \in \mathcal{C}(H/K),$
	defines the set $\mathcal{R}(H/K)$ of distinct orbits.  Denote the stabilizer of any element of $\mathcal{C}(H/K)$ by $\mathcal{E}_G(H/K).$ For $C = C_q(\chi) \in \mathcal{R}(H/K)$, define 
	\begin{equation}
		\epsilon_C(H,K) = \frac{1}{[H:K]}\,\widehat{K} \sum \limits_{\overline{h} \in H/K} \operatorname{tr}(\chi(\overline{h}))\,h^{-1},
	\end{equation}
	where $\widehat{K} = \frac{1}{|K|}\sum \limits_{k \in K} k$ and $\operatorname{tr} = \operatorname{tr}_{\mathbb{F}_q(\xi_{[H:K]})/\mathbb{F}_q}$ with $\xi_{[H:K]}$ a primitive $[H:K]$-th root of unity. The sum of distinct $G$-conjugates of $\epsilon_C(H,K)$ is denoted $e_C(G,H,K)$.\\
	Consider a pair $(H, K)$ of subgroups of $G$ such that $H$ is normal subgroup of $G$ and $H/K$
	is cyclic as well as a maximal abelian subgroup of  $N_G(K)/K$. Then by (\cite{OdRS04}, Corollary~3.6) and \cite{BdR07}, $(H, K)$ is a strong Shoda pair of $G$ and 
	$e_C(G,H,K)$ is a pci of $\mathbb{F}_qG.$  Further,
	\begin{equation}\label{equation 2}
		\mathbb{F}_qGe_C(G,H,K)\cong M_{[G:H]}(\mathbb{F}_{q^{o/[E:H]}}),
	\end{equation}
	where $E=\mathcal{E}_G(H/K)$ and $o$ is the multiplicative order of $q$ modulo $[H:K].$ \\
	Two strong Shoda pairs of a group are said to be inequivalent, if their corresponding pcis are distinct. We shall denote the set of all  inequivalent strong Shoda pairs of a group $G$ by $\mathcal{S}(G).$   
	Clearly, $(G, G) \in \mathcal{S}(G)$ and for a normal subgroup $K$ of $G$, the pair $(G, K) \in \mathcal{S}(G)$ if and only if $G/K$ is cyclic. The following theorem provides parameters of the codes associated with the pcis  corresponding to strong Shoda pair of type $(G,K)$.
	\begin{theorem}\label{parmameters $G=H$}
		Let $\mathbb{F}_q G$ be a finite semisimple group algebra. If $K$ is a normal subgroup of $G$ such that $G/K=\langle gK\rangle$, for some $g
		\in G$, then the code corresponding to the \linebreak pci(s) $e:=e_C(G, G, K)$,  $C \in \mathcal{R}(G/K)$, satisfy the following:
		\begin{itemize}
			\item[(i)] $\dim_{\mathbb{F	}_q}(\mathbb{F}_q G e) = o_{|G/K|}(q)$;
			\item[(ii)] The set $\mathcal{B}:= \{e, eg, \ldots, eg^{o_{|G/K|}(q)-1}\}$ is an $\mathbb{F}_q$-basis for  $\mathbb{F}_q G e$;
			\item[(iii)] $2|K| \leq d \leq \operatorname{wt}(e)$,  where $d$ denotes the minimum distance of the code and $\operatorname{wt}(e)$ denotes the weight of the idempotent $e$.
			
		\end{itemize}
		In particular, if $|G/K|=p^j$ where $j \in \mathbb{N}$ and  $p$ is an odd prime such that $o_{p^j}(q)=\phi(p^j)$,  then the minimum distance of code generated by $e$ is $2|K|$.
		
	\end{theorem}
	\begin{proof}
		As stated in preliminaries, we have that $\mathbb{F}_qGe_C(G,H,K)\cong M_{[G:H]}(\mathbb{F}_{q^{o/[E:H]}}),$
		where $E=\mathcal{E}_G(H/K)$ and $o$ is the multiplicative order of $q$ modulo $[H:K].$  So $\mathbb{F}_qGe_C(G, G, K)\cong \mathbb{F}_{q^{o_{|G/K|}(q)}}$ and we get the desired dimension.  
		Now we verify that $\mathcal{B}$ is an $\mathbb{F}_q$-basis for  $\mathbb{F}_q G e$, where  $e:=e_C(G, G, K)$.
		By dimension consideration, it is sufficient to show that the set $\mathcal{B}$ is linearly independent. Let $|G/K|=k$.  If $\mathcal{B}$ is a linearly dependent set, then
		\begin{equation}
			\sum\limits_{\mu_g=0}^{o_k(q)-1}\beta_{\mu_g}g^{\mu_g}e=0,
		\end{equation} 
		for some non-zero coefficients $\beta_{\mu_g},$ which yields that $ge$ is a root of non-zero polynomial of degree at most $o_{k}(q)-1$. This is not possible because $\mathbb{F}_qGe$ is the smallest field containing $ge$ and has degree   $o_{k}(q)$ over $\mathbb{F}_q$. 
		For distance bound firstly, observe that $\widehat{K}e_C(G, G, K)=e_C(G, G, K)$, so that $\mathbb{F}_qGe_C(G, G, K)\subseteq\mathbb{F}_qG\widehat{K}$. 
		Any element $\alpha \in \mathbb{F}_q G e_C(G, G, K)$ can be written as $\alpha = \left( \sum \limits_{t \in T} \alpha_t t \right) \widehat{K}$, with $\alpha_t \in \mathbb{F}_q$, where $T$ denotes the transversal of $K$ in $G$. If only one coefficient $\alpha_t$ is non-zero, say $\alpha = \alpha_t t \widehat{K}$ for some $t \in T$, then $\mathbb{F}_q G e_C(G, G, K) \supseteq \mathbb{F}_q G \widehat{K}$, which implies $k=o_k(q)$,  a contradiction. Hence, at least two coefficients must be non-zero, implying that each non-zero codeword has weight at least $2|K|$.\\
		Now if $|G/K|=p^j$ such that $o_{p^j}(q)=\phi(p^j)$ then from (\ref{equation 3}), the expression for $e=\widehat{K}[1-\widehat{\langle g^{p^{j-1}}\rangle }]$ and  $(1-g^{p^{j-1}})e=(1-g^{p^{j-1}})\widehat{K}$, which implies $d \leq 2|K|$.
			\end{proof}
			The above result is proved in a general setting for an arbitrary finite group $G$.  Henceforth, we focus on codes generated by the pcis corresponding to strong Shoda pairs $(H,K)$ where $H$ is a proper subgroup of $G$.\\
		It may be noted that metacyclic groups are normally monomial and hence the algorithms given in \cite{BM14} and \cite{BM16} to compute $\mathcal{S}(G)$ are applicable for these groups.  
		Let $G$ be a metacyclic group of the form $C_{p_1^m} \rtimes C_{p_2^l}$, where $p_1$ and $p_2$ are distinct primes, and $C_{p_2^l}$ acts faithfully on $C_{p_1^m}$. Then,  $G$ can be presented as
	\begin{equation}\label{equation 3}
		G = \langle a,b \mid a^{p_1^m} = b^{p_2^l} = 1,\ b^{-1}ab = a^r \rangle,
	\end{equation} 
	where $m,l,r \in \mathbb{N}$ are such that $o_{p_1^m}(r) = p_2^l$.  
	By  \cite{JOdRV13} we have that,   \begin{equation}\label{equation 4}
		\mathcal{S}(G) = \{(G,G)\} \cup \{(G,\langle a,b^{p_2^{j_2}} \rangle) \mid j_2 = 1,\dots,l\} \cup \{(\langle a\rangle, \langle a^{p_1^{j_1}}\rangle) \mid j_1 = 1,\dots,m\}.
	\end{equation}
	The following lemma, analogous to~\cite[Lemmas~3.1 and~3.2]{CM25} shall be useful in the study of metacyclic group codes of even length, particularly for computing traces.
	\begin{lemma}\label{lemma_trace_zero 2}  
	For $i, q  \in \mathbb{N}$, where $q$ is a power of some odd prime, we have the following:
	\begin{enumerate}
		\item[(1)]  If $q=1 + 2^{i_0} c $, with $c$ odd and $i_0\geq 2$, then  $
		o_{2^i}(q) = 
		\begin{cases}
			2^{i - i_0 }, & ~\mathrm{if } ~i > i_0 \\
			1, & ~\mathrm{otherwise} \\
			
		\end{cases} ~~{\mathrm {and}}
		$
		   
		  $$	\tr(\xi_{2^i}) =\sum \limits_{j=0}^{o_{2^i}(q)-1}\xi_{2^i}^{q^j}= 0, \quad \text{if and only if } i > i_0.$$
		\item[(2)] If $q=-1 + 2^{i_0} c $, with $c$ odd and $i_0\geq 2$, then $
		o_{2^i}(q) = 
		\begin{cases}
			2^{i - i_0 }, & ~~\mathrm{if}~ i > i_0 \\
			2, & ~~\mathrm{otherwise} \\
			
		\end{cases}~~{\mathrm {and}}
		$
		
		$$	\tr(\xi_{2^i}) =\sum \limits_{j=0}^{o_{2^i}(q)-1}\xi_{2^i}^{q^j}= 0, \quad \text{if and only if } i >  i_0~ \mathrm{or}~i=2.$$
	\end{enumerate}
\end{lemma}

\begin{proof} If $q$ is a power of an odd prime and $i\in \mathbb{N}$, then
	the expression of order $o_{2^i}(q)$ as in the statement follows from (\cite{SBR07}, Section 2.2). \\
	Let  \( q = 1+2^{i_0}c  \), where \( c \) is an odd integer and  $i_0\geq 2$. Clearly, if $i \leq i_0$, then $o_{2^i}(q)=1$  and $\text{tr}(\xi_{2^i}) \neq  0.$
	Suppose $i > i_0$, so that  $o_{2^i}(q) = 2^{i - i_0 }$. We see that the sets $J:=\{ q^j \mid 0 \leq j < o_{2^i}(q) \}$
	and  \(K:=\{ 1 + 2^{i_0}k \mid 0 \leq k < 2^{i-i_0} \} \) contain same elements  modulo \( 2^i \). This is because the cardinalities of the sets $J$ and $K$ are same and for any \( q^j \in J \),  i.e., $0 \leq j < 2^{i - i_0 }$,  we can write
	\[
	q^j= (1+2^{i_0}c )^j \equiv 1 + 2^{i_0}k_j \mod 2^i,
	\]
	for some \( 0 \leq k_j < 2^{i-i_0} \), so that $1 + 2^{i_0}k_j \in K$ . Hence, 
	\[
	\tr(\xi_{2^i}) = \sum \limits_{j=0}^{o_{2^i}(q)-1}\xi_{2^i}^{q^j}=\sum \limits_{k=0}^{2^{i-i_0}-1} \xi_{2^i}^{1 + 2^{i_0}k }=\xi_{2^i} \sum\limits_{k=0}^{2^{i-i_0}-1} \left( \xi_{2^i}^{2^{i_0}} \right)^k=0,
	\] as $ \xi_{2^i}^{2^{i_0}}  \neq 1 $ when $i > i_0$. 
	Therefore,
	$
	\tr(\xi_{2^i}) = 0, ~\text{if~and ~only ~if} ~i > i_0.
	$
	\\
	Now, if \( q = -1 + 2^{i_0}c \), with \( c \) odd and $i_0\geq 2$, then   analogously we obtain that  
	$ \tr(\xi_{2^i}) = 0, $ for $i > i_0$.
	For \(  i \leq  i_0 \), we have  $o_{2^i}(q)=2$, and hence $
	\tr(\xi_{2^i}) = \xi_{2^i} + \xi_{2^i}^{-1}=0 $, if and only if $i=2.$
\end{proof}

		\section{Metacyclic 2-group codes}
		%\subsection{Metacyclic $2$-group codes having maximal cyclic subgroup}
		 In this section, we study metacyclic $2$-groups. Specifically, we work with metacyclic $2$-groups  which possess a maximal cyclic subgroup. As per (\cite{Huppert1}, I, Satz 14.9(b)), a metacyclic group of order $2^{n+1}$, where $n \geq 3$ which has a maximal cyclic subgroup, is isomorphic to  one of the following: 
		 	\begin{enumerate}
		\item[(i)] $D_{2^{n+1}} := \langle a, b \mid a^{2^{n}} = 1,\ b^2 = 1,\ b^{-1}ab = a^{-1} \rangle$ (dihedral).
		\item[(ii)] $Q_{2^{n+1}}:= \langle a, b \mid a^{2^{n}} = 1,\ b^2 = a^{2^{n-1}},\ b^{-1}ab = a^{-1} \rangle$  (generalized quaternion).
		
		
		
		\item[(iii)] $SD_{2^{n+1}} := \langle a, b \mid a^{2^{n}} = 1,\ b^2 = 1,\ b^{-1}ab = a^{-1 + 2^{n-1}} \rangle$  (semi-dihedral).
		
		\item[(iv)] $G_{2^{n+1}} := \langle a, b \mid a^{2^{n}} = 1,\ b^2 = 1,\ b^{-1}ab = a^{1 + 2^{n-1}} \rangle$ (ordinary metacyclic).
	\end{enumerate}
	
	
	It has been proved in~\cite[Theorem~5]{DFPM09}, that  
	$\mathbb{F}_q D_{2^{n+1}} \ \cong\ \mathbb{F}_q Q_{2^{n+1}}.$ An alternate way to prove this is via theory of strong Shoda pairs. We rather apply this theory to find out when   group algebras $\mathbb{F}_q D_{2^{n+1}}$ and $\mathbb{F}_q SD_{2^{n+1}}$ are isomorphic.  It turns out that $\mathbb{F}_q G_{2^{n+1}}$ is never isomorphic to  $\mathbb{F}_q SG_{2^{n+1}}$.
		\begin{theorem}\label{$F_qD_{2^{n+1}}$ Isomorphic $F_qD_{2^{n+1}}^-$}
		Let $n\geq 3$ and let $q$ be a prime power of some odd prime.
%		both $\mathbb{F}_q D_{2^{n+1}}$ $(\cong \mathbb{F}_q Q_{2^{n+1}})$ and $\mathbb{F}_q SD_{2^{n+1}}$ are semisimple group algebras.
 Then,
		\[
		\mathbb{F}_q D_{2^{n+1}}  (\cong \mathbb{F}_q Q_{2^{n+1}}) \cong \mathbb{F}_q SD_{2^{n+1}}
		\quad \text{if and only if} \quad q \not \equiv -1 \mod{2^{n-1}}.
		\]
	\end{theorem}
	
	\begin{proof}
	

	A set of strong Shoda pairs of $G$ , where $G=D_{2^{n+1}},~ Q_{2^{n+1}},~ SD_{2^{n+1}},$ computed using algorithm given in  \cite{BM14}, is given by 
	\[
	\mathcal{S}(G)=
	\big\{
	(G, G),\ 
	(G,\langle a \rangle),\ 
	(G, \langle a^2, b \rangle),\ 
	(G, \langle a^2, ab\rangle)
	\big\}
	\cup 
	\big\{
	(\langle a \rangle, \langle a^{2^j} \rangle)\ | \ \ 2 \leq j \leq n
	\big\}.
	\] 
	
	We shall observe that the difference in structure of group algebras possibly occurs, only due to the component corresponding to $(\langle a \rangle, \langle 1 \rangle)$. It follows from  the results stated in Section 2 that if $G=D_{2^{n+1}}$ or $Q_{2^{n+1}}$, then 

		\[
	\mathbb{F}_q G \cong
	\begin{cases}
	4\mathbb{F}_q ~ \bigoplus
		 \limits_{j=2}^{n} 
		\frac{\phi(2^j)}{o_{2^j}(q)} M_2\!\left( \mathbb{F}_{q^{\,o_{2^j}(q)}} \right), ~ \text{if } -1 \in \langle q \rangle \mod{2^n}, \\
		4\mathbb{F}_q ~\bigoplus 
		\limits_{j=2}^{j_0} 
		\frac{\phi(2^j)}{o_{2^j}(q)} \, M_2\!\left( \mathbb{F}_{q^{\,o_{2^j}(q)/2}} \right)
		\bigoplus
		 \limits_{j=j_0+1}^{n} 
		\frac{\phi(2^j)}{2o_{2^j}(q)} M_2\!\left( \mathbb{F}_{q^{\,o_{2^j}(q)}} \right),~\text{otherwise}
	\end{cases}
	\]
	\noindent where $j_0$ is such that  
	\begin{equation}\label{DQ}
		-1 \in \langle q \rangle \mod\ 2^{j_0} ~~\mathrm{but}~ -1 \notin \langle q \rangle \mod\ 2^{j_0+1}
	\end{equation}
	
	
	\noindent and \[
	\mathbb{F}_q SD_{2^{n+1}} \cong
	\begin{cases}
	4	\mathbb{F}_q~  \bigoplus
		\limits_{j=2}^{n} 
		\frac{\phi(2^j)}{o_{2^j}(q)}  
		M_2\!\left( \mathbb{F}_{q^{o_{2^j}(q)}} \right),~~ \text{if} -1+2^{n-1} \in \langle q \rangle\mod 2^n\\
		4\mathbb{F}_q~ \bigoplus
		\limits_{j=2}^{j_1} 
		\frac{\phi(2^j)}{o_{2^j}(q)} \, 
		M_2\!\left( \mathbb{F}_{q^{\,o_{2^j}(q)/2}} \right)
		\bigoplus 
		\limits_{j=j_1+1}^{n} 
		\frac{\phi(2^j)}{2o_{2^j}(q)} \,
		M_2\!\left( \mathbb{F}_{q^{\,o_{2^j}(q)}} \right),
		~ \text{otherwise}
		
	\end{cases}
	\]
	where $j_1$ is such that
	
	\begin{equation}\label{SD}
		-1+2^{n-1} \in \langle q \rangle\mod 2^{j_1} ~~\text{but}~
		-1+2^{n-1} \notin \langle q \rangle  \mod\ 2^{j_1+1}.
	\end{equation}
	
	
	Note that $j_0\leq n-1$ and if $j_0 <n-1$, then $j_1=j_0$. Now, if $j_0=n-1$, then 	$\mathbb{F}_q G   \cong \mathbb{F}_q SD_{2^{n+1}}$ if and only if $j_1=j_0$ and the conditions $	-1 \notin \langle q \rangle  \mod\ 2^{n}$ and $	-1+2^{n-1} \notin \langle q \rangle  \mod\ 2^{n}$ are equivalent. We prove that
	\begin{equation}\label{eq 8}
		-1 \in \langle q \rangle \mod{2^n} 
		\iff 
		-1 + 2^{n-1} \in \langle q \rangle \mod{2^n}
	\end{equation}

precisely when $ q \not \equiv -1 \mod{2^{n-1}}$.

Observe that  $U(2^n) \cong C_2 \times C_{2^{n-2}} \cong \langle -1 \rangle \times \langle 5 \rangle$, and hence $U(2^n)$ contains exactly three elements of order $2$, namely $-1$, $-1+2^{n-1}$ and $1-2^{n-1}$. %The elements $-1$ and $-1+2^{n-1}$ simultaneously belong to $\langle q \rangle$ if and only $ q \not \equiv -1 \mod{2^{n-1}}$.
	
	\smallskip
	\noindent\textbf{Case (i):} Suppose $q \equiv 1 \mod{4}$.  
	Then $\langle q \rangle = \langle 5^k \rangle$ for some $k$. Clearly, $-1 \notin \langle q \rangle$.  
	Since $5^{2^{n-3}} \equiv 1 - 2^{n-1} \mod{2^n}$, we have $1 - 2^{n-1} \in \langle 5 \rangle$.  
	Consequently, $-1 \cdot (1 - 2^{n-1}) = -1 + 2^{n-1} \in \langle -5 \rangle \not\subseteq \langle q \rangle$.  
	Hence, (\ref{eq 8}) holds.
	
	\smallskip
	\noindent\textbf{Case (ii):} Suppose $q \equiv -1 \mod{4}$.  
	We claim that (\ref{eq 8}) holds if and only if $q \not \equiv -1 \mod{2^{n-1}}$.  
	
	First, assume $q \equiv -1 \mod{2^{n-1}}$.  
	Then $q^2 \equiv 1 \mod{2^n}$, so $\langle q \rangle$ has only one element of order $2$. If this element is either $-1$ or $-1+2^{n-1}$, (\ref{eq 8}) does not hold, otherwise we must have the order $2$ element to be $q=1-2^{n-1}$ which  contradicts the assumption  $q \equiv -1 \mod{2^{n-1}}$.    
	
	
	Conversely, if $q \not\equiv -1 \mod{2^{n-1}}$, then by \Cref{lemma_trace_zero 2} we have $o_{2^n}(q) > 2$.  
	Thus $\langle q \rangle$ contains at least two elements of order $2$, which forces the third element of order $2$ to also lie in $\langle q \rangle$.  
	Hence, the claim holds.
\end{proof}


	

	Since $\mathbb{F}_q D_{2^{n+1}}$ and $\mathbb{F}_q Q_{2^{n+2}}$ are always isomorphic, and $\mathbb{F}_q D_{2^{n+1}} \cong \mathbb{F}_q SD_{2^{n+1}}$ if and only if $ q \not \equiv -1 \mod 2^{n-1}$, it follows from the proof of  \Cref{$F_qD_{2^{n+1}}$ Isomorphic $F_qD_{2^{n+1}}^-$} that, except for the component corresponding to the strong Shoda pair $(\langle a \rangle, 1)$, the codes generated by the pcis of $\mathbb{F}_q D_{2^{n+1}}$ and $\mathbb{F}_q SD_{2^{n+1}}$ are equivalent. Moreover, the computation of the pcis corresponding to $(\langle a \rangle, 1)$ is somewhat similar in both $\mathbb{F}_q D_{2^{n+1}}$ and $\mathbb{F}_q SD_{2^{n+1}}$. Therefore, we consider the codes generated by the pcis of $\mathbb{F}_q D_{2^{n+1}}$  and $\mathbb{F}_q G_{2^{n+1}}$ only.\\
	
			We are now in position to write the pcis of the groups under consideration.
	
	\subsection{\underline{$\mathbf{D_{2^{n+1}}}$}} 
	
	
	$D_{2^{n+1}} := \langle a, b \mid a^{2^{n}} = 1,\ b^2 = 1,\ b^{-1}ab = a^{-1} \rangle$ with $n \geq 3.$
	%\[\mathcal{S}(D_{2^{n+1}}):=\{(D_{2^{n+1}},D_{2^{n+1}}), ~(D_{2^{n+1}},\langle a \rangle),~(SD_{2^{n+1}}, \langle a^2, b \rangle),\ (SD_{2^{n+1}}, \langle a^2, ab\rangle)\}
	%\]
	%\[\{(\langle a \rangle, \langle a^{2^j} \rangle)_{j=2}^n\}\].
	\begin{theorem}\label{prop_Dihedral}
		Let	$\mathbb{F}_q$ be a finite field containing $q$ elements, where $q$ is power of some odd prime so that $q$ is of the form $q= \pm  1 + 2^{i_0}c$, where $c$ is odd and $i_0 \geq 2$. The pcis of $\mathbb{F}_qD_{2^{n+1}}$ are as in \Cref{tab3:my_label}.%\ChSugandha{[Check the differences in table, subsequently the proof content]}: 
		\begin{small}
			\begin{table}[hbt!]
				
				
				\begin{tabular}{|ccl|}			
					\hline 
					
					\hline
					&&	if ~$q= \pm 1+2^{i_0}c$ with $c$ odd. \\
					
					$e_1$ &$:=$&$e_C(D_{2^{n+1}},D_{2^{n+1}},D_{2^{n+1}}), C\in \mathcal{R}(D_{2^{n+1}}/D_{2^{n+1}}) $\\
					
					&$=$ &$\widehat{D_{2^{n+1}}}$   \\&&\\
				
					
				
					$e_{2}$ &$:= $ & $e_C(D_{2^{n+1}},D_{2^{n+1}}, \langle a\rangle),C\in \mathcal{R}(D_{2^{n+1}}/ \langle a\rangle)$ \\
					
					
					&$=$&$\widehat{\langle a \rangle}-\widehat{D_{2^{n+1}}}$
					\\&&\\
										
					$e_{3}$ &$:= $ & $e_C(D_{2^{n+1}},D_{2^{n+1}}, \langle a^2,b\rangle),C\in \mathcal{R}(D_{2^{n+1}}/ \langle a^2,b\rangle)$\\
					
					&$=$&$\widehat{\langle a^2,b \rangle }-\widehat{D_{2^{n+1}}}$ \\&&\\
					
					
					
					
					$e_{4}$ &$:= $ & $e_C(D_{2^{n+1}},D_{2^{n+1}}, \langle a^2,ab\rangle),C\in \mathcal{R}(D_{2^{n+1}}/ \langle a^2,ab\rangle)$\\
					&$=$&$\widehat{\langle a^2,ab \rangle}-\widehat{D_{2^{n+1}}}$ \\

					\hline 	
					
					\hline 
					&&	if ~$q= 1+2^{i_0}c$ with $c$ odd. \\
%					\underline{\textbf{$-1\in\langle q\rangle $}}&&\\
					
					
					
					$e_{2^{j},k}$ &$:= $ & $e_C(D_{2^{n+1}},\langle a \rangle,\langle a^{2^{j}} \rangle),C=C_q(\gamma^k)\in \mathcal{R}(\langle a \rangle/\langle a^{2^{j}} \rangle),2 \leq j \leq n$ \\
					%	&  &  \\
					&$=$&$\begin{cases}
						\frac{1}{2^{j}} \widehat{\langle a^{2^{j }}\rangle}\sum\limits_{\mathfrak{i}=0}^{2^{j}-1} [\tr(\xi_{2^{j}}^{k\mathfrak{i}})]a^{-\mathfrak{i}},~~~~~~~~~~~~~~~\mathrm{if} ~ 1\leq j \leq i_0\\
						\frac{1}{2^{j}} \widehat{\langle a^{2^{j }}\rangle}\sum\limits_{\mathfrak{i}'=0}^{2^{i_0}-1}[\tr(\xi_{2^{j}}^{k\mathfrak{i}'{2^{j-i_0}}})]a^{-\mathfrak{i}'{2^{j-i_0}}},~~\mathrm{otherwise}.
					\end{cases}$
					\\
%					\underline{\textbf{$-1\not\in\langle q \rangle$}}&&\\
%					
%					%	&&\\
%					
%					$e_{2^{j},k}$ &$:= $ & $e_C(D_{2^{n+1}},\langle a \rangle,\langle a^{2^{j}} \rangle),C=C_q(\gamma^k)\in \mathcal{R}(\langle a \rangle/\langle a^{2^{j}} \rangle), 2\leq j \leq n$ \\
%					%	&  &  \\
%					
%					&$=$&$\begin{cases}
%						\frac{1}{2^{j}} \widehat{\langle a^{2^{j }}\rangle}\sum\limits_{\mathfrak{i}=0}^{2^{j}-1} [\tr(\xi_{2^{j}}^{k\mathfrak{i}}) + \tr(\xi_{2^{j}}^{-k\mathfrak{i}})]a^{-\mathfrak{i}},~~~~~~~~~~~~~~~~~~~~~\mathfrak{i}\neq 2^{j-2},~ 3\cdot 2^{j-2}~\mathrm{if} ~ 1\leq j \leq i_0\\
%						\frac{1}{2^{j}} \widehat{\langle a^{2^{j }}\rangle}\sum\limits_{\mathfrak{i}'=0}^{2^{i_0}-1}[\tr(\xi_{2^{j}}^{{k\mathfrak{i}'2^{j-i_0}}})+\tr(\xi_{2^{j}}^{{-k\mathfrak{i}'2^{j-i_0}}})]a^{-\mathfrak{i}'{2^{j-i_0}}},\mathfrak{i}'\neq 2^{j-2},~ 3\cdot 2^{j-2}~\mathrm{otherwise}.
%					\end{cases}$				\\
%					%		&  &  \\
					
					
					\hline
					
					\hline 
					
					&&	if ~$q= -1+2^{i_0}c$ with $c$ odd. \\
					\underline{\textbf{$-1\in\langle q\rangle\mod 2^j $}}&&\\
					
					
					%	&&\\
					$e_{2^{j},k}$ &$:= $ & $e_C(D_{2^{n+1}},\langle a \rangle,\langle a^{2^{j}} \rangle),C=C_q(\gamma^k)\in \mathcal{R}(\langle a \rangle/\langle a^{2^{j}} \rangle),2\leq j \leq n$ \\
					%	&  &  \\
					
					&$=$&$\begin{cases}
						\frac{1}{2^{j}} \widehat{\langle a^{2^{j }}\rangle}\sum\limits_{\mathfrak{i}=0}^{2^{j}-1} [\tr(\xi_{2^{j}}^{k\mathfrak{i}})]a^{-\mathfrak{i}},~~~~~~~~~~~~~~~\mathrm{if} ~ 3\leq j \leq i_0\\
						\widehat{\langle a^2 \rangle }-	\widehat{\langle a^4 \rangle },~~~~~~~~~~~~~~~~~~~~~~~~~~~~~~\mathrm{if}~ j=2\\
						\frac{1}{2^{j}} \widehat{\langle a^{2^{j }}\rangle}\sum\limits_{\mathfrak{i}'=0}^{2^{i_0}-1}[\tr(\xi_{2^{j}}^{k\mathfrak{i}'{2^{j-i_0}}})]a^{-\mathfrak{i}'{2^{j-i_0}}},~ \mathfrak{i}'\neq 2^{j-2},~ 3\cdot 2^{j-2}	,~\mathrm{if} ~~j>i_0.
					\end{cases}$
					\\
					
					
					
					
					\underline{\textbf{$-1\not\in\langle q \rangle \mod 2^j$}}&&\\
					
					%	&&\\
					%	&&\\
					$e_{2^{j},k}$ &$:= $ & $e_C(D_{2^{n+1}},\langle a \rangle,\langle a^{2^{j}} \rangle),C=C_q(\gamma^k)\in \mathcal{R}(\langle a \rangle/\langle a^{2^{j}} \rangle),2\leq j \leq n$ \\
					%	&  &  \\
					
					&$=$&$\begin{cases}
						\frac{1}{2^{j}} \widehat{\langle a^{2^{j }}\rangle}\sum\limits_{\mathfrak{i}=0}^{2^{j}-1} [\tr(\xi_{2^{j}}^{k\mathfrak{i}}) +\tr(\xi_{2^{j}}^{-k\mathfrak{i}})]a^{-\mathfrak{i}},~~~~~~~~~~~~~~~\mathrm{if} ~ 3 \leq j \leq i_0, ~ \mathfrak{i}\neq 2^{j-2},~ 3\cdot 2^{j-2}\\
						\widehat{\langle a^2 \rangle }-	\widehat{\langle a^4 \rangle },~~~~~~~~~~~~~~~~~~~~~~~~~~~~~~~~~~~~~~~~~~~\mathrm{if}~ j=2\\
						\frac{1}{2^{j}} \widehat{\langle a^{2^{j }}\rangle}\sum\limits_{\mathfrak{i}'=0}^{2^{i_0}-1}[\tr(\xi_{2^{j}}^{k\mathfrak{i}'{2^{j-i_0}}}) + \tr(\xi_{2^{j}}^{-k\mathfrak{i}'{2^{j-i_0}}})]a^{-\mathfrak{i}'{2^{j-i_0}}},~\mathfrak{i}'\neq 2^{j-2},~ 3\cdot 2^{j-2}	,~\mathrm{if} ~j>i_0.
					\end{cases}$
					\\
					
					\hline
					
					
				
				\end{tabular}
				
				\caption{Pcis of  $D_{2^{n+1}}$}
				\label{tab3:my_label}
				
					For $2 \leq j \leq n$, 
				the possible choices of $k$ yield $\tfrac{\varphi(2^j)}{o_{2^j}(q)}$ distinct idempotents  when $-1 \in \langle q\rangle \mod{2^j}$ and $\tfrac{\varphi(2^j)}{2o_{2^j}(q)}$ distinct idempotents when $-1 \notin \langle q\rangle \mod{2^j}$, $\varphi$ being the Euler totient function.
				
			\end{table}
		
		\end{small}
	
	
	
		
			\end{theorem}	
			\begin{proof}
					If $q \equiv 1\mod 4$, then (\cite{CM25}, Proposition 4.1) holds for any prime $p$ (including $p=2$). This is because, in this case, 	in view of \Cref{lemma_trace_zero 2}, the result in (\cite{CM25}, Lemma 3.2) holds for any prime $p$ (not necessarily odd).\\
					Hence, assuming  $q \equiv -1 \mod 4$, we write the pcis of $\mathbb{F}_qD_{2^{n+1}}$ corresponding to $(\langle a \rangle,\langle a^{2^{j}} \rangle) \in \mathcal{S}(D_{2^{n+1}})$, where  $1\leq j\leq n$, i.e., $e_C(D_{2^{n+1}},\langle a \rangle, \langle a^{2^{j}} \rangle)$, where $C\in \mathcal{R}(\langle a\rangle/\langle a^{2^{j}} \rangle)$ for   $1\leq j \leq n$.
%					 As $\langle a \rangle/\langle a^{2^{j}} \rangle$ is cyclic group of order $2^{j}$,  we have that
%			the number of $q$ cyclotomic classes of $\Irr(\langle a \rangle/ \langle a^{2^{j}} \rangle)$ containing generators equals $\frac{\phi(2^{j})}{o_{2^j}(q)}$. 
	If $-1\in \langle q\rangle$ then 
			$\mathcal{R}(\langle a \rangle/ \langle a^{2^{j}} \rangle)=\mathcal{C}(\langle a \rangle/ \langle a^{2^{j}} \rangle)$ 
			and
			 in this case  we have  
				$$\begin{aligned}
					e_C(G,\langle a \rangle,\langle a^{2^{j}} \rangle)=\epsilon_C(\langle a \rangle,\langle a^{2^{j}} \rangle)
					=\frac{1}{2^{j}}\widehat{\langle a^{2^{j}} \rangle}\Sigma_{\mathfrak{i}=0}^{2^{j}-1}\tr(\xi_{2^{j}}^\mathfrak{i}})){a^{-\mathfrak{i}},
				\end{aligned}$$
					where
					$$\tr(\xi_{2^{j}}^{k\mathfrak{i}})=(\xi_{2^{j}}^{k\mathfrak{i}})^{q^0}+(\xi_{2^{j}}^{k\mathfrak{i}})^q+(\xi_{2^{j}}^{k\mathfrak{i}})^{q^2}+....+(\xi_{2^{j}}^{k\mathfrak{i}})^{q^{o_{2^j}(q)-1}}.$$
				
				Now, if $1\leq j \leq i_0$, $j \neq 2$ then $\tr(\xi_{2^{j}}^{k\mathfrak{i}})\neq 0$,  by \Cref{lemma_trace_zero 2} and if $j=2, $ then 
				$$\begin{aligned}
					\epsilon_C(\langle a \rangle , \langle a^4 \rangle)&=\frac{1}{4}\widehat{\langle a^4 \rangle}\sum \limits_{\mathfrak{i}=0}^{3}\tr(\gamma^{k}(\overline{a^\mathfrak{i}})){a^{-\mathfrak{i}}}
					%&=\frac{1}{4}\widehat{\langle a^4 \rangle}[2-2a^{-2}]
					=	
					\widehat{\langle a^4 \rangle}-\widehat{\langle a^2 \rangle}.
				\end{aligned}$$
					Assume $i_0<j\leq m$. In this case, $\xi_{2^{j}}^{k\mathfrak{i}}$ is $2^{i}$-th primitive root of unity, if $\gcd(\mathfrak{i},2^{j})=2^{j-i}$. 
%						Thus the summation in the trace term contains sub-blocks of the form 
%					$(\xi_{2^{i}})^{q^0}+(\xi_{2^{i}})^q+(\xi_{2^{i}})^{q^2}+....+(\xi_{2^{i}})^{q^{o_{2^{i}}(q)-1}}$, each of length $o_{2^{i}}(q)$.
					 Hence,  $$\tr(\xi_{2^{j}}^{k\mathrm{i}})=\frac{o_{2^{j}}(q)}{o_{2^{i}}(q)}[(\xi_{2^{i}})^{q^0}+(\xi_{2^{i}})^q+(\xi_{2^{i}})^{q^2}+....+(\xi_{2^{i}})^{q^{o_{2^{i}}(q)-1}}].$$
						Note that by \Cref{lemma_trace_zero 2}, the above term is zero if and only if  $i>i_0$ or $i=2$. Therefore, the terms which do not vanish are precisely the ones where $i_0\geq  i$ and $i \neq 2$, i.e., $\mathfrak{i}$ is a multiple of $2^{j-i_0}$ but not an odd multiple of $2^{j-2}$. So we obtain $$e_C(D_{2^{n+1}},\langle a \rangle,\langle a^{2^{j}} \rangle)=  \frac{1}{2^{j}} \widehat{\langle a^{2^{j }}\rangle}\sum\limits_{\mathfrak{i}'=0}^{2^{i_0}-1}[\tr(\xi_{2^{j}}^{tk\mathfrak{i}'{2^{j-i_0}}})]a^{-\mathfrak{i}'{2^{j-i_0}}}, \mathfrak{i}'\neq 2^{j-2},~ 3\cdot 2^{j-2}.$$
					If $-1\notin\langle q\rangle\mod{2^j}$, then
				\[
				\begin{aligned}
					e_C(D_{2^{n+1}},\langle a\rangle,\langle a^{2^{j}}\rangle)
					&=\frac{1}{2^{j}}\widehat{\langle a^{2^{j}}\rangle}
					\sum\limits_{\mathfrak{i}=0}^{2^{j}-1}\big[\tr(\xi_{2^{j}}^{k\mathfrak{i}})+\tr(\xi_{2^{j}}^{-k\mathfrak{i}})\big]a^{-\mathfrak{i}}\\
					&=\frac{1}{2^{j}}\widehat{\langle a^{2^{j}}\rangle}
					\sum\limits_{\mathfrak{i}=0}^{2^{j}-1}\tr\big(\xi_{2^{j}}^{k\mathfrak{i}}+\xi_{2^{j}}^{-k\mathfrak{i}}\big)\,a^{-\mathfrak{i}}.
				\end{aligned}
				\]
				
				Note that \(\xi_{2^{j}}^{k\mathfrak{i}}+\xi_{2^{j}}^{-k\mathfrak{i}}\neq 0\) precisely when
				\(\mathfrak{i}\neq 2^{j-2},\,3\cdot 2^{j-2}\).
			Therefore,	by \Cref{lemma_trace_zero 2}, we have 
				
				\begin{itemize}
					\item[(i)] If \(1\le j\le i_0\), then
					\[
					e_C(D_{2^{n+1}},\langle a\rangle,\langle a^{2^{j}}\rangle)
					=\frac{1}{2^{j}}\widehat{\langle a^{2^{j}}\rangle}
					\sum\limits_{\substack{\mathfrak{i}=0\\ \mathfrak{i}\neq 2^{j-2},\,3\cdot2^{j-2}}}^{2^{j}-1}
					\tr\big(\xi_{2^{j}}^{k\mathfrak{i}}+\xi_{2^{j}}^{-k\mathfrak{i}}\big)\,a^{-\mathfrak{i}},
					\]
					and for these \(\mathfrak{i}\) the trace is non-zero.
					
					\item[(ii)] If \(i_0<j\le m\), write \(\mathfrak{i}=\mathfrak{i}'2^{\,j-i_0}\) with
					\(\mathfrak{i}'=0,\dots,2^{i_0}-1\). Then the sum reduces to
					\[
					e_C(D_{2^{n+1}},\langle a\rangle,\langle a^{2^{j}}\rangle)
					=\frac{1}{2^{j}}\widehat{\langle a^{2^{j}}\rangle}
					\sum\limits_{\mathfrak{i}'=0}^{2^{i_0}-1}
					\tr\big(\xi_{2^{j}}^{k\mathfrak{i}'2^{\,j-i_0}}+\xi_{2^{j}}^{-k\mathfrak{i}'2^{\,j-i_0}}\big)\,a^{-\mathfrak{i}'2^{\,j-i_0}},
					\]
					where the vanishing indices correspond to \(\mathfrak{i}'=2^{\,i_0-2}\) and
					\(\mathfrak{i}'=3\cdot 2^{\,i_0-2}\). Thus the effective summation excludes
					\(\mathfrak{i}'=2^{\,i_0-2},\,3\cdot 2^{\,i_0-2}\), and for the remaining \(\mathfrak{i}'\)
					the trace is non-zero.
				\end{itemize}
				 \end{proof}
					Using  \Cref{prop_Dihedral} we obtain $\dim_{\mathbb{F}_q}(\mathbb{F}_q D_{2^{n+1}} e_{2^j,k})$, an  $\mathbb{F}_q$-basis of $\mathbb{F}_q D_{2^{n+1}} e_{2^j,k}$ and bounds on the distance $d:=d(\mathbb{F}_qD_{2^{n+1}}e_{2 ^{j },k})$ exactly as obtained in Corollaries 4.3 and 4.5 of \cite{CM25} with $p=2$. The parameters and the basis remain unchanged except that the upper bound improves in the current case because of the reduced support.
					
			
%			%For $\mathbb{F}_qD_{2^{n+1}}$, the parameters of such codes are listed next, omiting the details owing to similarity in arguments with Theorems 3.4-3.5 of \cite{CM25}.
%				On the lines of (\cite{CM25}, Theorems 3.4-3.5) we have following corollary to Theorem \ref{prop_Dihedral}.
%	
%	\begin{corollary}\label{Cor_distance_dihedral}%\ChSugandha{[Check the changes in light of changes in above theorem]}
%		In the foregoing notation, 
%		
%		
%		
%		\begin{enumerate}
%			\item  $\dim_{\mathbb{F	}_q}(\mathbb{F}_qD_{2^{n+1}}e_{2^{j},k})=\begin{cases}
%				2o_{2^{j}}(q),~\mathrm{if} -1  \in \langle q\rangle, \\
%				4o_{2^{j}}(q),~\mathrm{if}  -1\not  \in \langle q\rangle.
%			\end{cases}$
%			\item 	An $\mathbb{F}_q$-basis of $\mathbb{F}_qD_{2^{n+1}}e_{2^j,k}$ is given by $\mathcal{B}$, where \\
%			
%			$\mathcal{B}=\begin{cases}
%				\{a^{\eta_a}b^{\eta_b}e_{2^{j},k}~|~0\leq \eta_a< o_{2^j}(q),~0\leq \eta_b \leq 1\},~~~~ \mathrm{if}~ -1\in \langle q\rangle ,\\
%				\{a^{\eta_a}b^{\eta_b}\epsilon_C(\langle a \rangle,\langle a^{2^{j}} \rangle)^t~|~t\in \{1,b\},~0\leq \eta_a< o_{2^j}(q),~0\leq \eta_b \leq 1\}, ~~~\mathrm{if} -1\not \in \langle q\rangle.
%				
%			\end{cases}$
%			
%			%$	4o_{p^{j}}(q),~ -1\not  \in \langle q\rangle $, and
%			\item $d:=d(\mathbb{F}_qD_{2^{n+1}}e_{2 ^{j },k})$ satisfies the following:\\
%			\begin{itemize}
%				\item[(i)] if $q=1+2^{i_0}c$ with $c$ odd then \\
%				$$\begin{cases}
%					2^{n-j +1}\leq d \leq  2^n,~~~~~~~\mathrm{if}~ 1\leq j \leq i_0,\\
%					2^{n-j+1 }\leq d \leq 2 ^{n-j +i_0},~\mathrm{otherwise}.
%				\end{cases}$$
%				\item[(ii)]if $q=-1+2^{i_0}c$ with $c$ odd then 
%				
%					\[
%				d =
%				\begin{cases}
%					\begin{aligned}
%						&2^{n-j+1} \leq d \leq 2^n, && \text{if } ~1\leq j \leq i_0,\\
%						&2^{n-1}, && \text{if } j=2,\\
%						&2^{n-j+1} \leq d \leq 2^{n-j}(2^{i_0}-2), && \text{otherwise } 
%					\end{aligned}
%					& \quad \text{if } -1\in \langle q \rangle \mod{2^j}, \\[1cm]
%					
%					\begin{aligned}
%						&2^{n-j+1} \leq d \leq 2^{n}-2, && \text{if } 3\leq j \leq i_0,\\
%						&2^{n-j+1} \leq d \leq 2^{n-j}(2^{i_0}-2), && \text{otherwise} 
%					\end{aligned}
%					& \quad \text{if } -1\notin \langle q \rangle \mod{2^j}.
%				\end{cases}
%				\]
%				
%				
%	\end{itemize}
%			
%		\end{enumerate}   	
%	\end{corollary}
%
%	
%	
%	
%		On the lines of (\cite{CM25}, Theorems -3.5) we have following corollary to Theorem \ref{prop_Dihedral}.
%By \cite[Corollary~4.3 and Corollary~4.5]{CM25}, the formulas for 
%$\dim(\mathbb{F}_q D_{2^n} e_{p^j,k})$, the $\mathbb{F}_q$-basis, 
%and the lower distance bound remain unchanged; the upper bound still 
%arises from the support of the idempotent, but improves here since the 
%support is reduced.

	\begin{remark}
		It is worth noting that the dihedral codes discussed in \cite{GR22b}, \cite{GR22a} and \cite[Section~4]{DFPM09} can be obtained as special cases of results in this subsection, when the group under consideration is a dihedral group of order $2^{n+1}$. Each of these works considers different conditions on the field size $q$: $q$ is of the form $8c \pm 1$ with $c$ odd in \cite{GR22a}, $o_{2^n}(q) = 1$ or $2$ in \cite{GR22b}, and $o_{2^n}(q) = \phi(2^n)$ in \cite{DFPM09}. Consequently, the corresponding results on code dimension and minimum distance in those works follow as special cases. In \Cref{prop_Dihedral}, we provide a unified treatment of these computations for all such choices of $q$.
	\end{remark}
%\begin{proposition}
%	Let $G$ be a finite non-abelian metacyclic group given by
%	\begin{equation} \label{eq:metacyclic}
%		G = \langle a, b \mid a^n = 1,\; b^t = a^\ell,\; b^{-1}ab = a^r \rangle,
%	\end{equation}
%	where $n, t, r, \ell \in \mathbb{N}$ are such that 
%	\[
%	r^t \equiv 1 \pmod{n}, \qquad \ell r \equiv \ell \pmod{n}, \qquad \ell \mid n.
%	\]
%	
%	Then the left ideal $\mathbb{F}_q G \cdot e_{b_j}e$ is minimal if and only if the ideal $\mathbb{F}_q G \cdot e$ is minimal as a two-sided ideal, where $e_{b_j}$ is a primitive central idempotent of $\mathbb{F}_q\langle b \rangle$ and $e$ is a primitive central idempotent of $\mathbb{F}_qG$.
%\end{proposition}
%
%\begin{proof}
%	Suppose that $\mathbb{F}_q G \cdot e_{b_j}e$ is not minimal. Then there exist nonzero orthogonal idempotents $f_1, f_2$ such that
%	\[
%	e_{b_j}e = f_1 + f_2.
%	\]
%	
%	Since
%	\[
%	0 = (1 - e_{b_j})(e_{b_j}e) = (1 - e_{b_j})(f_1 + f_2),
%	\]
%	multiplying on the right by $f_1$ we obtain
%	\[
%	(1 - e_{b_j}) f_1 = 0 \quad \implies \quad f_1 = e_{b_j} f_1.
%	\]
%	Similarly, $(1 - e_{b_j}) f_2 = 0 \implies f_2 = e_{b_j} f_2$. Hence
%	\[
%	f_1 = e_{b_j} f_1 = f_1 e_{b_j}, \qquad f_2 = e_{b_j} f_2 = f_2e_{b_j} .
%	\]
%	Thus,
%	\[
%	f_1 = f_1e_{b_j}= e_{b_j}f_1, \qquad f_2 = f_2e_{b_j} = e_{b_j}f_2.
%	\]
%	
%	Write
%	\[
%	f_t = \sum_k \alpha_k a^k + \sum_{i,j} \beta_{ij} a^i b^j.
%	\]
%	Since $bab^{-1} = a^r$, for each $k$, $1 \leq k \leq n - 1$, we have
%	\[
%	b^x a^k = a^{kr^x} b^x,
%	\]
%	As $e_{b_j}=	\epsilon_C(\langle b \rangle ,\langle b^j \rangle) = \frac{1}{[\langle b \rangle :\langle b^j \rangle]}\,\widehat{b^j} \sum_{i=0}^{j-1} \tr(\chi(\overline{b}^i))\,b^{-i}$. Let $[\langle b \rangle :\langle b^j \rangle]=j_1$
%	
%	We have $e_{b_j}a^ke_{b_j}=\sum_{i=o}^{o(b)-1}\tr(\xi_{j_1}^i)a^{kr^{i}}e_{b_j}$
%	
%	where
%	\[
%	\sum_{i=o}^{o(b)-1}\tr(\xi_{j_1}^i)a^{kr^{i}}  \in Z(\mathbb{F}_q G).
%	\]
%	Therefore, there exist $\lambda_1, \lambda_2 \in Z(\mathbb{F}_q G) \cap \mathbb{F}_q \langle a \rangle$ such that
%	\[
%	f_1 = \lambda_1 e_{b_j}, \qquad f_2 = \lambda_2 e_{b_j}.
%	\]
%	Since $f_t = f_t e$, we may assume $\lambda_t \in Z(\mathbb{F}_q G) \cdot e \cap \mathbb{F}_q \langle a \rangle$ for $t = 1,2$.
%	
%	Now,
%	\[
%	f_1 f_2 = (\lambda_1 \lambda_2) e_{b_j }= 0,
%	\]
%	and as $\lambda_1 \lambda_2 \in \mathbb{F}_q \langle a \rangle$, we conclude that $\lambda_1 \lambda_2 = 0$. Since $e$ is a primitive idempotent, we must have $\lambda_1 = 0$ or $\lambda_2 = 0$, implying $f_1 = 0$ or $f_2 = 0$, a contradiction. Thus $\mathbb{F}_q G \cdot e_{b_j}e$ is minimal.
%	
%	Conversely, if $\mathbb{F}_q G \cdot e$ is not minimal, then there exist nonzero orthogonal central idempotents $e_1, e_2$ such that
%	\[
%	e = e_1 + e_2.
%	\]
%	Then
%	\[
%	e_{b_j}e = e_{b_j}e_1 + e_{b_j}e_2, \qquad (e_{b_j}e_1)(e_{b_j}e_2) = 0.
%	\]
%	Hence the left ideal $\mathbb{F}_q G \cdot e_{b_j}e$ is not minimal.
%\end{proof}
% The semisimple  group algebra $\mathbb{F}_q G$ can be expressed as a direct sum of minimal left ideals, each of which is generated by some primitive idempotent. %$e_i$.
% , i.e.,
%	$\mathbb{F}_q G \;=\; \langle e_1 \rangle \oplus \cdots \oplus \langle e_m \rangle \oplus \langle e_{m+1} \rangle \oplus \cdots \oplus \langle e_s \rangle .
%	$
%	The set $\{e_1, \ldots, e_s\}$ thus forms a complete family of primitive orthogonal idempotents of $\mathbb{F}_q G$.
	 It has also been observed that non-central group codes play an equally significant role; in fact, they often yield codes that are inequivalent to abelian codes and may even possess better parameters. Such codes correspond to left (or right) ideals of the group algebra, and every left (right) ideal $I \subseteq \mathbb{F}_q G$ can be generated by a suitable idempotent. A complete set of pairwise orthogonal irreducible left idempotents of $\mathbb{F}_qD_{2^{n+1}}$ follows from  above theorem and  (\cite{APM19}, Proposition~2.5).
	 
	 
	 
\begin{corollary}\label{left pcis F_qD_{2^{n+1}}}
	The semisimple group algebra $\mathbb{F}_qD_{2^{n+1}}$ decomposes into minimal left ideals generated by a complete set of primitive orthogonal idempotents, given by:
	
	
	\begin{itemize}
		\item[(i)] Four central idempotents: $e_1, e_2, e_3, e_4$.
		\item[(ii)] $2 \dfrac{\phi(2^j)}{\kappa o_{2^j}(q)}$ left  idempotents: $e_{2^j,k}\widehat{\langle b \rangle}$ and $e_{2^j,k}(1-\widehat{\langle b \rangle})$ with $2 \leq j \leq n$,	where $\kappa=2$ for $r+1\leq j \leq n$, if $r$ is such that $-1 \in \langle q \rangle \mod{2^r}$ but $-1 \notin \langle q \rangle \mod{2^{r+1}}$ and $\kappa=1$ in all other cases.
	\end{itemize}
\end{corollary}
%	If $q \equiv 1 \pmod{4}$, then the number of idempotents 
%	
%	
%	
%	\underline{$q \equiv 1 \pmod{4}$}, left irreducible idempotents of  $\mathbb{F}_qD_{2^{n+1}}$ are as follows:	$4 \;+\; 2 \sum_{j=2}^{n}\frac{\phi(2^j)}{o_{2^j}(q)}$	
%	\begin{itemize}
%		\item[(i)] Four central idempotents:	$e_1, e_2, e_3, e_4$, and 
%		\item[(ii)] $2 \dfrac{\phi(2^j)}{ko_{2^j}(q)}$ left irreducible idempotents: $e_{2^j,k}\widehat{b}$ and $e_{2^j,k}(1-\widehat{b})$ with $2 \leq j \leq n$,
%		
%	\end{itemize}
%	
%	
%	\begin{enumerate}
%		\item If $q \equiv 1 \pmod{4}$, left irreducible idempotents of  $\mathbb{F}_qD_{2^{n+1}}$ are as follows:	%$4 \;+\; 2 \sum_{j=2}^{n}\frac{\phi(2^j)}{o_{2^j}(q)}$
%	
%		\begin{itemize}
%			\item[(i)] Four central irreducible idempotents $e_1, e_2, e_3, e_4$ as in  \Cref{tab3:my_label}.
%			\item[(ii)] $2 \dfrac{\phi(2^j)}{o_{2^j}(q)}$ left irreducible idempotents of the form $e_{2^j,k}\widehat{b}$ and $e_{2^j,k}(I-\widehat{b})$ with $1 \leq j \leq n$.
%		\end{itemize}
%		
%		\item If $q \equiv -1 \pmod{4}$ and $r$ is such that  $r$ is such that $-1 \in \langle q \rangle \mod{2^r}$ but $-1 \notin \langle q \rangle \mod{2^{r+1}}$ then left irreducible idempotents are 
		
%		then $\mathbb{F}_qD_{2^{n+1}}$ has 
%		\[
%		4 \;+\; 2 \sum_{j=2}^{r}\frac{\phi(2^j)}{o_{2^j}(q)} 
%		\;+\; \sum_{j=r+1}^{n}\frac{\phi(2^j)}{2o_{2^j}(q)}
%		\]
%		left irreducible idempotents, where $r$ is such that $-1 \in \langle q \rangle \pmod{2^r}$ but $-1 \notin \langle q \rangle \pmod{2^{r+1}}$. They are described as follows:
%		\begin{itemize}
%			\item[(i)] Four central irreducible idempotents $e_1, e_2, e_3, e_4$ as in \Cref{tab3:my_label}.
%			\item[(ii)] $2 \dfrac{\phi(2^j)}{o_{2^j}(q)}$ left irreducible idempotents of the form $e_{2^j,k}\widehat{b}$ and $e_{2^j,k}(I-\widehat{b})$, with $2 \leq j \leq r$.
%			\item[(iii)] $ \dfrac{\phi(2^j)}{o_{2^j}(q)}$ left irreducible idempotents of the form $e_{2^j,k}\widehat{b}$ and $e_{2^j,k}(I-\widehat{b})$, with $r+1 \leq j \leq n$.
%		\end{itemize}
%	\end{enumerate}

%\begin{proof}
%Clearly, the given idempotents are minimal by \Cref{prop_Dihedral} and (\cite{APM19}, Proposition~2.5).  
%Now we show that they form a complete set of primitive left idempotents.
%
%First, observe that for every $j$ and $k$, we always have
%\[
%\mathbb{F}_q D_{2^{n+1}} e_{2^j,k} \;=\; 
%\mathbb{F}_q D_{2^{n+1}} e_{2^j,k} \widehat{b} 
%\;\oplus\; 
%\mathbb{F}_q D_{2^{n+1}} e_{2^j,k} (I-\widehat{b}).
%\]
%Hence the $\mathbb{F}_q$-dimension of each summand
%$\mathbb{F}_q D_{2^{n+1}} e_{2^j,k} \widehat{b}$ and 
%$\mathbb{F}_q D_{2^{n+1}} e_{2^j,k} (I-\widehat{b})$
%is exactly one half of the dimension of 
%$\mathbb{F}_q D_{2^{n+1}} e_{2^j,k}$.
%
%\smallskip
%\noindent
%The number of distinct $e_{2^j,k}$ is determined by the Wedderburn decomposition of $\mathbb{F}_q D_{2^{n+1}}$ in both cases.  
%
%Case 1: $q \equiv 1 \pmod{4}$.  
%In this case, $-1 \in \langle q \rangle \pmod{2^n}$, and for each $2 \leq j \leq n$, the number of distinct $e_{2^j,k}$ by  (\ref{DQ}) is
%$\frac{\phi(2^j)}{o_{2^j}(q)}.$
%
%\smallskip
%\noindent
%Case 2: $q \equiv -1 \pmod{4}$.  
%Then, by (\ref{DQ}), there exists $j$ such that
%$-1 \in \langle q \rangle \pmod{2^j}$ but $-1 \notin \langle q \rangle \pmod{2^{j+1}}$.
%Consequently, for $2 \leq i \leq j$, the number of distinct $e_{2^i,k}$ is
%
%$\frac{\phi(2^i)}{o_{2^i}(q)},$
%
%while for $j+1 \leq i \leq n$, it reduces to
%$\frac{\phi(2^i)}{2\,o_{2^i}(q)}.$
%
%
%\smallskip
%In either case, after performing the splitting with $\widehat{b}$ and $(I-\widehat{b})$, 
%we obtain precisely all primitive left idempotents of $\mathbb{F}_q D_{2^{n+1}}$.  
%Finally, summing their $\mathbb{F}_q$-dimensions shows that the total dimension equals $2^{n+1}$, so no further primitive idempotents exist.  
%
%This proves completeness of the above list.
%\end{proof}





We next consider the group algebra $\mathbb{F	}_qG_{2^{n+1}}$.
	\subsection{\underline{$\mathbf{G_{2^{n+1}}}$}}
$G_{2^{n+1}} := \langle a, b \mid a^{2^{n}} = 1,\ b^2 = 1,\ b^{-1}ab = a^{1+2^{n-1}} \rangle$, $n \geq 3.$ 

	\begin{theorem}\label{prop_G_{2^{n+1}}}
		Let	$\mathbb{F}_q$ be a finite field containing $q$ elements, where $q$ is power of some odd prime so that $q$ is of the form $q= \pm  1 + 2^{i_0}c$, where $c$ is odd and $i_0 \geq 2$. The pcis of $\mathbb{F}_qG_{2^{n+1}}$ are as in \Cref{tab4:my_label}.
		
		\begin{small}
			
			
			\begin{table}
				
				
				\begin{tabular}{|ccl|}
					
					\hline 
					
					\hline
					&&	if ~$q= \pm 1 +2^{i_0}c$ with $c$ odd. \\
					
					$e_1$ &$:=$&$e_C(G_{2^{n+1}}, G_{2^{n+1}}, G_{2^{n+1}}), C\in \mathcal{R}(G_{2^{n+1}}/G_{2^{n+1}}) $\\
					
					&$=$ &$\widehat{G_{2^{n+1}}}$   \\&&\\
					
					
					
					
					$e_{2}$ &$:= $ & $e_C(G_{2^{n+1}},G_{2^{n+1}}, \langle a\rangle),C\in \mathcal{R}(G_{2^{n+1}}/ \langle a\rangle)$ \\
					
					
					&$=$&$\widehat{\langle a \rangle}-\widehat{G_{2^{n+1}}}$
					\\&&\\
					
					
					$e_{3}$ &$:= $ & $e_C(G_{2^{n+1}},G_{2^{n+1}}, \langle a^2,b\rangle),C\in \mathcal{R}(G_{2^{n+1}}/ \langle a^2,b\rangle)$\\
					
					&$=$&$\widehat{\langle a^2,b \rangle }-\widehat{G_{2^{n+1}}}$ \\&&\\
					
					
					
					
					$e_{4}$ &$:= $ & $e_C(G_{2^{n+1}}, G_{2^{n+1}}, \langle a^2,ab\rangle),C\in \mathcal{R}(G_{2^{n+1}}/ \langle a^2,ab\rangle)$\\
					&$=$&$\widehat{\langle a^2,ab \rangle}-\widehat{G_{2^{n+1}}}$ \\

					\hline 
					
					\hline 
					&&	if ~$q= 1+2^{i_0}c$ with $c$ odd. \\
					
					
					
					%	&&\\
					$e_{2^{j},k}$ &$:= $ & $e_C(G_{2^{n+1}}, G_{2^{n+1}}, K),C=C_q(\gamma^k)\in \mathcal{R}(G_{2^{n+1}}/K)$,\\ &&where $K=\langle a^{2^{j}}, b \rangle$ or $\langle a^{2^{{j}-1}}b\rangle,~2\leq j \leq n-1$ \\
					
					&$=$&$\begin{cases}
						\frac{1}{2^{j}} \widehat{K}\sum\limits_{\mathfrak{i}=0}^{2^{j}-1} [\tr(\xi_{2^{j}}^{k\mathfrak{i}})]a^{-\mathfrak{i}},~~~~~~~~~~~~~~~\mathrm{if} ~ 2\leq j \leq i_0\\
						\frac{1}{2^{j}} \widehat{K}\sum\limits_{\mathfrak{i}'=0}^{2^{i_0}-1}[\tr(\xi_{2^{j}}^{k\mathfrak{i}'{2^{j-i_0}}})]a^{-\mathfrak{i}'{2^{j-i_0}}},~\mathrm{otherwise}.
					\end{cases}$
					\\
					
					
					&&\\
					
					$e_{2^n,k}$ &$:= $ & $e_C(G_{2^{n+1}},\langle a \rangle,\langle 1 \rangle),C=C_q(\gamma^{k})\in \mathcal{R}(\langle a \rangle/\langle 1 \rangle)$ \\
					%	&  &  \\
				%	\underline{\textbf{$2^{n-1}+1 \in\langle q\rangle $}}&&\\
					&$=$&$\begin{cases}
						\frac{1}{2^{i_0}} \sum\limits_{\mathfrak{i}=0}^{2^{i_0}-1} [\tr(\xi_{2^{i_0}}^{k\mathfrak{i}}) ]a^{-\mathfrak{i}},~~~~~~~~~~~~~\mathrm{if}~n=i_0\\
						\frac{1}{2^{n}}\Sigma_{\mathfrak{i}'=0}^{2^{i_0}-1}[\tr(\xi_{2^{n}}^{{k\mathfrak{i}'2^{n-i_0}}})]a^{-\mathfrak{i}'{2^{n-i_0}}}~\mathrm{if}~n>i_0.
					\end{cases}$
					\\
%					\underline{\textbf{$2^{n-1}+1 \notin\langle q\rangle $}}&&\\
%					&$=$&$\begin{cases}
%						\frac{1}{2^{n}} \sum\limits_{\mathfrak{i}=0}^{2^{n}-1} [\tr(\xi_{2^{n}}^{k\mathfrak{i}}) + \tr(\xi_{2^{n}}^{(2^{n-1}+1)k\mathfrak{i}})]a^{-\mathfrak{i}},~~~~~~~~~~~~~~~~~~~\mathrm{if}~n=i_0\\
%						\frac{1}{2^{n}}\sum\limits_{\mathfrak{i}'=0}^{2^{i_0}-1}[\tr(\xi_{2^{n}}^{{k\mathfrak{i}'2^{n-i_0}}})+\tr(\xi_{2^{n}}^{{(2^{n-1}+1)k\mathfrak{i}'2^{n-i_0}}})]a^{-\mathfrak{i}'{2^{n-i_0}}}~\mathrm{if}~n>i_0.
%					\end{cases}$\\
%					
					
					
					
					
					\hline
					
					\hline 
					&&	if ~$q= -1+2^{i_0}c$ with $c$ odd. \\
					
					
					
					
					$e_{2^{j},k}$ &$:= $ & $e_C(G_{2^{n+1}}, G_{2^{n+1}}, K),C=C_q(\gamma^k)\in \mathcal{R}(G_{2^{n+1}}/K)$,\\ &&where $K=\langle a^{2^{j}}, b \rangle$ or $\langle a^{2^{{j}-1}}b\rangle,~2\leq j \leq n-1$ \\
					
					
					&$=$&$\begin{cases}
						\frac{1}{2^{j}} \widehat{K}\sum\limits_{\mathfrak{i}=0}^{2^{j}-1} [\tr(\xi_{2^{j}}^{k\mathfrak{i}})]a^{-\mathfrak{i}},~~~~~~~~~~~~~\mathrm{if} ~ 2\leq j \leq i_0, j \neq 2\\
						\widehat{\langle a^2, b \rangle }-	\widehat{\langle a^{4}, b \rangle},~~~~~~~~~~~~~~~~~~~\mathrm{if} ~K=\langle a^{4}, b\rangle \\
						
						\widehat{\langle a^{4}b \rangle }-	\widehat{\langle a^2b \rangle },~~~~~~~~~~~~~~~~~~~~~~\mathrm{if}~ K=\langle a^2b\rangle \\
						\frac{1}{2^{j}} \widehat{K}\sum\limits_{\mathfrak{i}'=0}^{2^{i_0}-1}[\tr(\xi_{2^{j}}^{k\mathfrak{i}'{2^{j-i_0}}})]a^{-\mathfrak{i}'{2^{j-i_0}}},~ \mathfrak{i}'\neq 2^{j-2},~ 3\cdot 2^{j-2}	,~\mathrm{if} ~~j>i_0.
					\end{cases}$
					\\
					&&\\
					$e_{2^n,k}$ &$:= $ & $e_C(G_{2^{n+1}},\langle a \rangle,\langle 1 \rangle),C=C_q(\gamma^{k})\in \mathcal{R}(\langle a \rangle/\langle 1 \rangle)$ \\
					%	&  &  \\
					\underline{\textbf{$2^{n-1}+1 \in\langle q\rangle\mod 2^n $}}&&\\
					&$=$&$\begin{cases}
						\frac{1}{2^{i_0}} \sum\limits_{\mathfrak{i}=0}^{2^{i_0}-1} [\tr(\xi_{2^{i_0}}^{k\mathfrak{i}}) ]a^{-\mathfrak{i}}, ~~~~~~~~~~~~~~~~~~~~~~~~~~~~~~~~~~~~~~~~~~\mathrm{if}~n=i_0\\
						\frac{1}{2^{n}}\sum\limits_{\mathfrak{i}'=0}^{2^{i_0}-1}[\tr(\xi_{2^{n}}^{{k\mathfrak{i}'2^{n-i_0}}})]a^{-\mathfrak{i}'{2^{n-i_0}}},~ \mathfrak{i}'\neq 2^{j-2},~ 3\cdot 2^{j-2}~~~~~~~~\mathrm{if}~n>i_0.
					\end{cases}$
					\\
					\underline{\textbf{$2^{n-1}+1 \notin\langle q\rangle \mod 2^n$}}&&\\
					&$=$&$\begin{cases}
						\frac{1}{2^{i_0}} \sum\limits_{\mathfrak{i}=0}^{2^{i_0}-1} [\tr(\xi_{2^{i_0}}^{k\mathfrak{i}}) + \tr(\xi_{2^{i_0}}^{(2^{i_0-1}+1)k\mathfrak{i}})]a^{-\mathfrak{i}}, ~~~~\mathfrak{i}~						 \mathrm{is ~even~~~~~~~if}~n=i_0 
\\
						
						\frac{1}{2^{n}}\sum\limits_{\mathfrak{i}'=0}^{2^{i_0}-1}[\tr(\xi_{2^{n}}^{{k\mathfrak{i}'2^{n-i_0}}})+\tr(\xi_{2^{n}}^{{(2^{n-1}+1)k\mathfrak{i}'2^{n-i_0}}})]a^{-\mathfrak{i}'{2^{n-i_0}}},~~~~\mathrm{if}~n>i_0\\~~~~~~~~~~~~~~~~~~~~~~~~~~~~~~~~~~~~~~~~~~~~~~~~~~~~~~~~~~~~~~~~~~~ \mathfrak{i}'\neq 2^{j-2},~ 3\cdot 2^{j-2}
					\end{cases}$
					\\
					%		&  &  \\
					
					
					\hline
					
					
					
					
					
					
					
					
					
					
					\hline
				\end{tabular}
				
				\caption{Pcis of  $G_{2^{n+1}}$}
				\label{tab4:my_label}
		
				The possible choices of $k$ yield $\tfrac{\varphi(2^j)}{o_{2^j}(q)}$ distinct idempotents for $2 \leq j \leq n$ when $1+2^{n-1} \in \langle q\rangle \mod{2^j}$, and $\tfrac{\varphi(2^j)}{2o_{2^j}(q)}$ choices when $1+2^{n-1} \notin \langle q\rangle \mod{2^j}$.
	\end{table}		\end{small}
		
		
		
		
	\end{theorem}	
	\begin{proof}%\ChSugandha{[this proof is undiscussed yet.]}
			The complete list of strong Shoda pairs of $G:=G_{2^{n+1}}$ is given by

				\[
			\mathcal{S}(G) =
			\big\{
			(G, K)  : K\in \{G, \langle a \rangle,\langle a^2, b \rangle,\langle a^2, ab \rangle,  \langle a^{2^j}, b \rangle , \langle a^{2^{j-1}}b \rangle\ | \ \ 2 \leq j \leq n-1\}	\}\cup	\{(\langle a \rangle, \langle 1 \rangle) 
			\big\}.
			\]
		
			
			
			
		
All the strong Shoda pairs of  \(G_{2^{n+1}}\), except $(\langle a \rangle, \langle 1 \rangle) $, are of the type 
\((G_{2^{n+1}}, K)\) with \(K\) a proper subgroup of \(G_{2^{n+1}}\). Therefore, in view of \Cref{parmameters $G=H$}
%we compute the primitive central idempotents of \(\mathbb{F}_q G_{2^{n+1}}\) 
%corresponding to the strong Shoda pairs \((G_{2^{n+1}},K)\), where 
%\[
%K=\langle a^{2^{j}}, b \rangle \quad \text{or} \quad K=\langle a^{2^{j-1}}b\rangle, 
%\quad 2\leq j \leq n-1.
%\]
%That is, we compute the idempotents
%\[
%e_{2^{j},k} := e_C(G_{2^{n+1}}, G_{2^{n+1}}, K), \qquad C\in \mathcal{R}(G_{2^{n+1}}/K).
%\]
%
%%As \(G_{2^{n+1}}/K\) is cyclic of order \(2^{j}\), it follows that 
%%\(\Irr(G_{2^{n+1}}/K)\) is also cyclic of the same order, generated by some 
%%\(\gamma\). Hence the number of \(q\)-cyclotomic classes of 
%%\(\Irr(G_{2^{n+1}}/K)\) containing generators equals 
%%\(\frac{\varphi(2^{j})}{o_{2^j}(q)}\), and 
%%\(\mathcal{C}(G_{2^{n+1}}/K)\) consists of elements of the form 
%%\(C_q(\gamma^{k})\) for suitable odd integers \(k\). 
%%There are precisely \(\frac{\varphi(2^{j})}{o_{2^j}(q)}\) such choices of \(k\) 
%%yielding distinct idempotents.
% Note that in this case
%\(\mathcal{R}(G_{2^{n+1}}/K)=\mathcal{C}(G_{2^{n+1}}/K)\).
%
%Thus, for \(C=C_q(\gamma^{k})\in \mathcal{R}(G_{2^{n+1}}/K)\), we have
%\[
%e_C(G_{2^{n+1}}, G_{2^{n+1}}, K)
%= \epsilon_C(G_{2^{n+1}}, K)
%= \frac{1}{2^{j}} \, \widehat{K} \sum_{\mathfrak{i}=0}^{2^{j}-1} 
%\operatorname{tr}\!\big(\xi_{2^j}^{k\mathfrak{i}}\big)\,a^{-\mathfrak{i}},
%\]
%where
%\[
%\operatorname{tr}(\xi_{2^{j}}^{k\mathfrak{i}})
%= (\xi_{2^{j}}^{k\mathfrak{i}})^{q^0}
%+ (\xi_{2^{j}}^{k\mathfrak{i}})^{q}
%+ (\xi_{2^{j}}^{k\mathfrak{i}})^{q^2}
%+ \cdots 
%+ (\xi_{2^{j}}^{k\mathfrak{i}})^{q^{\,o_{2^j}(q)-1}}.
%\]
%
%If \(q \equiv 1 \pmod{4}\), then, using \Cref{lemma_trace_zero 2} and 
%arguing as in the proof of Theorem~3.3 of \cite{CM25}, for the Shoda pair 
%\((G, \langle a, b^{p_2^{j_2}}\rangle)\), we obtain the required expression.
%			
%			Now, if $q \equiv -1 \pmod{4}$, then for $1 \leq j \leq i_0$, $j \neq 2$, we have 
%			$\operatorname{tr}(\xi_{2^{j}}^{k\mathfrak{i}})\neq 0$ by \Cref{lemma_trace_zero 2}. 
%			If $j=2$, then
%			$$
%			\begin{aligned}
%				\epsilon_C(G,K)
%				&=\frac{1}{4}\,\widehat{K}\sum_{\mathfrak{i}=0}^{3}
%				\operatorname{tr}\!\big(\gamma^{k}(\overline{a^\mathfrak{i}})\big)\,a^{-\mathfrak{i}} \\[6pt]
%				&=
%				\begin{cases}
%					\widehat{\langle a^{2^{j-1}}, b \rangle} - \widehat{\langle a^{2^j}, b \rangle},
%					& \text{if } K=\langle a^{4}, b\rangle, \\[6pt]
%					\widehat{\langle a^{2^j}b \rangle} - \widehat{\langle a^{2^{j-1}}b \rangle},
%					& \text{if } K=\langle a^{2}b\rangle.
%				\end{cases}
%			\end{aligned}
%			$$
%			
%			Assume $i_0 < j \leq l$. 
%			Arguing as in the proof of \Cref{prop_Dihedral}, 
%			one sees that $\tr(\xi_{2^{j}}^{k\mathfrak{i}})$ is nonzero precisely when 
%			$i \leq i_0$ and $i \neq 2$, that is, when $\mathfrak{i}$ is a multiple of 
%			$2^{j-i_0}$ but not an odd multiple of $2^{j-2}$. 
%			This yields the required expression.\\
				we consider the idempotent corresponding to  
		$(\langle a \rangle, \langle 1 \rangle)$ only. 
		
		
		If $2^{\,n-1}+1 \in \langle q \rangle \mod{2^n}$,  then  $e_C(G_{2^{n+1}},\langle a \rangle,\langle 1 \rangle)=\epsilon_C(\langle a \rangle,\langle 1 \rangle)$ 
		and the expression is obtained directly. On the other hand, if $2^{\,n-1}+1 \notin \langle q \rangle \mod{2^n}$, 
		then by the same observation on order $2$ elements in $U(2^n)$ as done  in the proof of 
		\Cref{$F_qD_{2^{n+1}}$ Isomorphic $F_qD_{2^{n+1}}^-$}, 
		this situation occurs precisely when 
		\[
		q \equiv -1 \mod{2^{\,n-1}}
		\quad \text{and} \quad
		q \equiv -1 \mod{4}.
		\]
%		On the other hand, if $2^{\,n-1}+1 \notin \langle q \rangle \mod{2^n}$, 
%		then this case arises only when 
%		$$
%		q \equiv -1 \mod{2^{\,n-1}} 
%		\quad \text{and} \quad 
%		q \equiv -1 \mod{3}.
%		$$ because  
%			 $U(2^n) \cong C_2 \times C_{2^{n-2}} \cong \langle -1 \rangle \times \langle 5 \rangle$, there are only three elements of order $2$, namely $-1$, $-1+2^{n-1}$, and $1-2^{n-1}$.
%		
%		\smallskip
%		\noindent\textbf{Case (i):} Suppose $q \equiv 1 \mod{4}$.  
%		Then $\langle q \rangle = \langle 5^k \rangle$ for some $k$ and arguing as in proof of \Cref{$F_qD_{2^{n+1}}$ Isomorphic $F_qD_{2^{n+1}}^-$} we obtain that neither of $-1$ nor $-1+2^{n-1} \in \langle q \rangle.$ So  $1+2^{n-1}\in \langle q \rangle$. 
%			\smallskip
%		\noindent\textbf{Case (ii):} Suppose $q \equiv -1 \pmod{4}$.  
%		Claim  holds if and only if $q \not \equiv -1 \pmod{2^{n-1}}$.  
%		
%		First, assume $q \equiv -1 \mod{2^{n-1}}$.  
%		Then $q^2 \equiv 1 \pmod{2^n}$, so $\langle q \rangle$ has only one element of order $2$. If this element is either $-1$ or $-1+2^{n-1}$,  otherwise we must have the order $2$ element to be $q=1-2^{n-1}$ which  contradictions the assumption  $q \equiv -1 \mod{2^{n-1}}$.    
%		
%		
%		Conversely, if $q \not\equiv -1 \pmod{2^{n-1}}$, then by \Cref{lemma_trace_zero 2} we have $o_{2^n}(q) > 2$.  
%		Thus $\langle q \rangle$ contains at least two elements of order $2$, which forces the third element of order $2$.
%		

By  \Cref{lemma_trace_zero 2}, we obtain that
 for $n=i_0$,  $e_{{2^n}, k}=	\frac{1}{2^{n}} \sum\limits_{\mathfrak{i}=0}^{2^{n}-1} [\tr(\xi_{2^{n}}^{k\mathfrak{i}}) + \tr(\xi_{2^{n}}^{(1+2^{n-1})k\mathfrak{i}})]a^{-\mathfrak{i}} $, where
 $\tr(\xi_{2^{n}}^{k\mathfrak{i}}) + \tr(\xi_{2^{n}}^{(1+2^{n-1})k\mathfrak{i}}) =\tr(\xi_{2^{n}}^{k\mathfrak{i}} + \xi_{2^{n}}^{(2^{n-1}+1)k\mathfrak{i}}) \neq 0$ if and only if $\mathfrak{i}$ is even. 
  And, for $n > i_0$, 
 $e_{{2^n}, k}=	\frac{1}{2^{n}}\sum\limits_{\mathfrak{i}'=0}^{2^{i_0}-1}[\tr(\xi_{2^{n}}^{{k\mathfrak{i}'2^{n-i_0}}})+\tr(\xi_{2^{n}}^{{(1+2^{n-1})k\mathfrak{i}'2^{n-i_0}}})]a^{-\mathfrak{i}'{2^{n-i_0}}},~~ \mathfrak{i}'\neq 2^{j-2},~ 3\cdot 2^{j-2}$ with $\tr(\xi_{2^{n}}^{{k\mathfrak{i}'2^{n-i_0}}}+\xi_{2^{n}}^{{(1+2^{n-1})k\mathfrak{i}'2^{n-i_0}}})\neq 0.$

 

			
	\end{proof}
\begin{remark}From $\mathcal{S}(G_{2^{n+1}})$ given in above proof it follows that
\begin{equation}\label{weddG_{2^{n+1}}}
	\mathbb{F}_q G_{2^{n+1}} \cong
	\begin{cases}	4\mathbb{F}_q ~~ \bigoplus	\limits_{j=2}^{n-1} 
		
		2\frac{\phi(2^j)}{o_{2^j}(q)} 
		\mathbb{F}_{q^{\,o_{2^j}(q)}} ~  \bigoplus~
		\frac{\phi(2^n)}{o_{2^n}(q)} 
		M_2\!\left(\mathbb{F}_{q^{\,o_{2^n}(q)/2}}\right),
		~ \text{if $1+2^{n-1} \in \langle q \rangle ~\rm {mod} ~2^n$},
		\\
		4\mathbb{F}_q ~~ \bigoplus	\limits_{j=2}^{n-1} 
		
		2\frac{\phi(2^j)}{o_{2^j}(q)} 
		\mathbb{F}_{q^{\,o_{2^j}(q)}} ~ \bigoplus~
		\frac{\phi(2^n)}{2o_{2^n}(q)} \,
		M_2\!\left(\mathbb{F}_{q^{\,o_{2^n}(q)}}\right),
		~  \text{otherwise}.
	\end{cases}
\end{equation}
\end{remark}
%	Since all the above strong Shoda pair of $G_{2^{n+1}}$ is of the form $(G, K)$ type except the one which is of the form $(\langle a \rangle,  \langle 1 \rangle)$, so we need to give parameters only for the pcis corresponding to strong Shoda pair $(\langle a \rangle, \langle 1 \rangle),$ namely $e_{2^n,k} .$\\
	

	\begin{corollary}\label{Cor_distance_{M_{2^{n+1}}}}In the foregoing notation, 
			\begin{enumerate}
			\item $\dim_{\mathbb{F}_q}(\mathbb{F}_qG_{2^{n+1}}e_{2^{n},k})=\begin{cases}
				2o_{2^{n}}(q),~\mathrm{if} ~1+2^{n-1} \in \langle q\rangle, \\
				4o_{2^{n}}(q),~\mathrm{if} ~ 1+2^{n-1}\not  \in \langle q\rangle.
			\end{cases}$
			\item 	An $\mathbb{F}_q$-basis of $\mathbb{F}_qG_{2^{n+1}}e_{2^n,k}$ is given by $\mathcal{B}$, where \\
			
			$\mathcal{B}=\begin{cases}
				\{a^{\eta_a}b^{\eta_b}e_{2^{n},k}~|~0\leq \eta_a< o_{2^n}(q),~0\leq \eta_b \leq 1\},~~~~ \mathrm{if}~ 1+2^{n-1}\in \langle q\rangle ,\\
				\{a^{\eta_a}b^{\eta_b}\epsilon_C(\langle a \rangle,\langle a^{2^{n}} \rangle)^t~|~t\in \{1,b\},~0\leq \eta_a< o_{2^n}(q),~0\leq \eta_b \leq 1\}, ~~~\mathrm{if}~ 1+2^{n-1}\not \in \langle q\rangle.
				
			\end{cases}$
			
			%$	4o_{p^{j}}(q),~ -1\not  \in \langle q\rangle $, and
			\item $d:=d(\mathbb{F}_qG_{2^{n+1}}e_{2 ^{n },k})$ satisfies the following:
			\begin{itemize}
				\item[(i)] if $q=1+2^{i_0}c$, where $c$ is odd and $i_0 \geq 2$, then $2 \leq d \leq 2 ^{ i_0}$,
%				$$\begin{cases}
%					2 \leq d \leq  2^n,~~~~~~~\mathrm{if}~ n= i_0,\\
%					2 \leq d \leq 2 ^{ i_0},~\mathrm{otherwise}.
%				\end{cases}$$
				\item[(ii)]if $q=-1+2^{i_0}c$, where $c$ is odd and $i_0 \geq 2$, then  $d$ satisfies the following:\\
				
				\underline{\textbf{$2^{n-1}+1 \in\langle q\rangle \mod 2^n$}}~~~~~~	$\begin{cases}	
						\begin{aligned}
						&2 \leq d \leq 2^{i_0}, && \mathrm{if}~ n=i_0>2,\\
						&d=2, &&  \mathrm{if}~ n=2,\\
						&2 \leq d \leq (2^{i_0}-2), &&  \mathrm{if}~ i_0<n
					\end{aligned}
					
				
				\end{cases}$
			
			\vspace{0.5cm}
			
			
				
					\underline{\textbf{$2^{n-1}+1 \notin\langle q\rangle \mod 2^n$}}~~~~~~	$\begin{cases}	
					\begin{aligned}
						&2 \leq d \leq 2^{i_0-1}, &&  \mathrm{if}~ n=i_0>2,\\
						
						&2 \leq d \leq (2^{i_0}-2), &&  \mathrm{if}~ i_0<n.
					\end{aligned}
					
					
				\end{cases}$
				
%				\[
%				d =
%				\begin{cases}
%					\begin{aligned}
%						&2 \leq d \leq 2^{i_0}, && \text{if } n=i_0>2,\\
%						&d=2, && \text{if } n=2,\\
%						&2 \leq d \leq (2^{i_0}-2), && \text{if } i_0<n
%					\end{aligned}
%						& \quad \text{if } 1+2^{\,n-1}\in \langle q \rangle \pmod{2^n}, \\[1cm]
%					
%					\begin{aligned}
%						&2 \leq d \leq 2^{i_0-1}, && \text{if } n=i_0>2,\\
%						&2 \leq d \leq (2^{i_0}-2), && \text{if } i_0<n
%					\end{aligned}
%					& \quad \text{if } 1+2^{\,n-1}\notin \langle q \rangle \pmod{2^n}.
%				\end{cases}
%				\]
					\end{itemize}
			
		\end{enumerate}   	
	\end{corollary}

	\begin{corollary}\label{left pcis F_qG_{2^{n+1}}}
		The semisimple group algebra $\mathbb{F}_qG_{2^{n+1}}$ decomposes into minimal left ideals generated by a complete set of primitive orthogonal idempotents, given by:
		\begin{itemize}
			\item[(i)]$4 \;+\; 2 \sum\limits_{j=2}^{n-1}\frac{\phi(2^j)}{o_{2^j}(q)}$ central  idempotents corresponding to strong Shoda pairs of type $(G_{2^{n+1}},K),$ as listed in \Cref{tab4:my_label}.
			\item[(ii)] $2 \dfrac{\phi(2^n)}{\kappa o_{2^n}(q)}$ left  idempotents: $e_{2^n,k}\widehat{\langle b \rangle}$ and $e_{2^n,k}(1-\widehat{\langle b \rangle})$,	where $\kappa=2$ for $1+2^{n-1} \notin \langle q \rangle \mod 2^n$ and $\kappa=1$ in all other cases.
	\end{itemize}
	\end{corollary}
%		\begin{enumerate}
%			\item If $1 + 2^{n-1} \in \langle q \rangle \pmod{2^n}$, then a complete family of left idempotents of $\mathbb{F}_qG_{2^{n+1}}$ is given by
%			
%		
%%			\[
%%			2 \;+\; 2 \sum_{j=1}^{n-1}\frac{\phi(2^j)}{o_{2^j}(q)}+\frac{\phi(2^{n-1})}{o_{2^{n-1}}(q)}+2\frac{\phi(2^n)}{o_{2^n}(q)}
%%			\]
%		
%			\begin{itemize}
%				\item[(i)]  $2 \;+\; 2 \sum\limits_{j=1}^{n-1}\frac{\phi(2^j)}{o_{2^j}(q)}+\frac{\phi(2^{n-1})}{o_{2^{n-1}}(q)}$ are (left)central irreducible idempotents corresponding to Shoda pair of type $(G_{2^{n+1}},K)$as in \Cref{tab4:my_label}.
%				\item[(ii)] $2\frac{\phi(2^n)}{o_{2^n}(q)}$ left irreducible idempotents of the form $e_{2^n,k}\widehat{b}$ and $e_{2^nk}(I-\widehat{b})$.
%			\end{itemize}
%			
%			\item If $q \equiv -1 \pmod{4}$ such that $1+2^{n-1}\notin \langle q \rangle \mod 2^n$, then  a complete family of left idempotents of  $\mathbb{F}_qG_{2^{n+1}}$  are as follows:
%%			\[
%%		2 \;+\; 2 \sum_{j=1}^{n-1}\frac{\phi(2^j)}{o_{2^j}(q)}+\frac{\phi(2^{n-1})}{o_{2^{n-1}}(q)}+\frac{\phi(2^n)}{o_{2^n}(q)}
%%		\]
%		
%		\begin{itemize}
%			\item[(i)]  $2 \;+\; 2 \sum\limits_{j=1}^{n-1}\frac{\phi(2^j)}{o_{2^j}(q)}+\frac{\phi(2^{n-1})}{o_{2^{n-1}}(q)}$ are (left)central irreducible idempotents corresponding to Shoda pair of type $(G_{2^{n+1}},K)$as in \Cref{tab4:my_label}.
%			\item[(ii)] $\frac{\phi(2^n)}{o_{2^n}(q)}$ left irreducible idempotents of the form $e_{2^n,k}\widehat{b}$ and $e_{2^nk}(I-\widehat{b})$.
%		\end{itemize}
%		\end{enumerate}
%	\end{corollary}

%\begin{proof}
%	The result follows directly from \Cref{weddG_{2^{n+1}}} and the argument used in the proof of \Cref{left pcis F_qD_{2^{n+1}}}, together with the fact that the primitive central idempotents corresponding to Shoda pairs $(G,K)$ account for the complete decomposition of the commutative component $\mathbb{F}_q(G/G')$. Hence these are also primitive left idempotents.
%\end{proof}
	A complete classification of metacyclic 2-groups is available in \cite{XZ06} and hence  one can similarly do the computations for any metacyclic $2$-groups. 


	\begin{remark}
		In \cite{CM25}, we determined the pcis of 
		$\mathbb{F}_q G$ for split metacyclic groups of order $p_1^m p_2^l$, 
		assuming $p_1$ and $p_2$ to be distinct odd primes. In view of results in this section, the case when $p_1=p_2=2$ is handled. Moreover, 
		using \Cref{lemma_trace_zero 2} and following the same reasoning as in the proofs of 
	\Cref{prop_Dihedral} and \Cref{prop_G_{2^{n+1}}},  the result of  \cite[Theorem~3.3]{CM25} 
	also extends to the case when one of the primes is $2$.
	\end{remark}
	

%	\subsection{Metacyclic group codes of even length}
	
%		In \cite{CM25}, we determined the primitive central idempotents (pcis) of 
%	$\mathbb{F}_q G$ for split metacyclic groups of order $p_1^m p_2^l$ 
%	under the assumption that $p_1$ and $p_2$ are distinct odd primes.  
%	Using \Cref{lemma_trace_zero 2}, 
%	%together with the strong Shoda pairs of $G$ given in (\ref{equation 3}), 
%	and applying the same reasoning as in the proofs of \Cref{prop_Dihedral} 
%	and \Cref{prop_G_{2^{n+1}}}, one can extend \cite[Theorem~3.3]{CM25} 
%	to the case where one of the primes may be $2$.  

	
		
%		We now record the cases where the expressions differ from those in 
%		\cite[Theorem~3.3]{CM25}.
%		
%	%	\ChSugandha{[Rewritten, so check details.]}
%		\begin{itemize}
%			\item[(i)] For a metacyclic group $G$ of order $2^l p_{1}^m$, with presentation
%			\[
%			G := \langle a,b \mid a^{p_{1}^m} = b^{2^l} = 1,\; b^{-1}ab = a^r\rangle,
%			\]
%			where $m,l,r \in \mathbb{N}$ satisfy $o_{p_{1}^m}(r) = 2^l$, assume
%			\underline{\textbf{$q \equiv -1 \pmod{4}$}}, with $q = -1 + 2^{i_0}c$, $c$ odd, $i_0 \geq 2$. Then
%			\[
%			e_{2^{j},k} := e_C(G,G,\langle a,b^{2^j}\rangle), \quad 
%			C = C_q(\gamma^k) \in \mathcal{R}(G/\langle a ,b^{2^j}\rangle), \; 1 \leq j \leq l,
%			\]
%			is given by
%			\[
%			e_{2^{j},k} =
%			\begin{cases}
%				\frac{1}{2^j}\,\widehat{\langle a,b^{2^j} \rangle}\;
%				\sum\limits_{\mathfrak{i}=0}^{2^j-1}\big[\tr(\xi_{2^j}^{k\mathfrak{i}})\big]\,b^{-\mathfrak{i}}, 
%				& \text{if } 1 \leq j \leq i_0,\, j \neq 2, \\[6pt]
%				\widehat{\langle a, b^{4} \rangle}-\widehat{\langle a, b^{2} \rangle}, 
%				& \text{if } j=2, \\[6pt]
%				\frac{1}{2^j}\,\widehat{\langle a,b^{2^j} \rangle}
%				\sum\limits_{\mathfrak{i}'=0}^{2^{i_0}-1}
%				\big[\tr(\xi_{2^j}^{k\mathfrak{i}'2^{j-i_0}})\big]\,
%				b^{-\mathfrak{i}'2^{j-i_0}}, & \text{if } j>i_0, \;
%				\mathfrak{i}' \neq 2^{j-2},\, 3\cdot 2^{j-2}.
%			\end{cases}
%			\]
%			
%			\item[(ii)] For a metacyclic group $G$ of order $2^m p_{2}^l$, with presentation
%			\[
%			G := \langle a,b \mid a^{2^m} = b^{p_2^l} = 1,\; b^{-1}ab = a^r\rangle,
%			\]
%			where $m,l,r \in \mathbb{N}$ satisfy $o_{2^m}(r) = p_2^l$, assume that 
%			$\mathbb{F}_q G$ is semisimple, so $q = \pm 1 + 2^{i_0}c$, with $c$ odd, $i_0 \geq 2$.  
%			If \underline{\textbf{$q \equiv -1 \pmod{4}$}}, then
%			\[
%			e_{2^{j},k} := e_C(G,\langle a \rangle,\langle a^{2^{j}} \rangle), \quad
%			C = C_q(\gamma^{k}) \in \mathcal{R}(\langle a \rangle/\langle a^{2^{j}} \rangle),\; 1 \leq j \leq m,
%			\]
%			is given by
%			\[
%			e_{2^{j},k} =
%			\begin{cases}
%				\frac{1}{2^{j}} \,\widehat{\langle a^{2^{j }}\rangle}
%				\sum\limits_{\mathfrak{i}=0}^{2^{j}-1} 
%				\Big[ \sum\limits_{t\in \tau}\tr(\xi_{2^{j}}^{k\mathfrak{i}})\Big] a^{-\mathfrak{i}}, 
%				& \text{if } 1 \leq j \leq i_0,\, j \neq 2, \\[6pt]
%				\widehat{\langle a^4 \rangle}-\widehat{\langle a^2 \rangle}, 
%				& \text{if } j=2, \\[6pt]
%				\frac{1}{2^{j}} \,\widehat{\langle a^{2^{j }}\rangle}
%				\sum\limits_{\mathfrak{i}'=0}^{2^{i_0}-1}
%				\Big[\sum\limits_{t\in \tau}\tr(\xi_{2^{j}}^{k\mathfrak{i}'2^{j-i_0}})\Big]\,
%				a^{-\mathfrak{i}'2^{j-i_0}}, & \text{if } j>i_0,\;
%				\mathfrak{i}' \neq 2^{j-2},\, 3\cdot 2^{j-2}.
%			\end{cases}
%			\]
%			Here $\tau$ denotes a transversal of $\langle r^{\omega_0}\rangle$ in $\langle r \rangle$, 
%			where $\omega_0$ is the smallest positive integer such that $r^{\omega_0}\in \langle q \rangle$.
%		\end{itemize}
	


In the next section, we further extend our analysis to the remaining case when $p_1$ and $p_2$ are not  distinct but odd primes, that is, we consider metacyclic $p$-groups, where $p$ is an odd prime.

	\section{Metacyclic $p$-group codes, $p\neq 2$}
%\ChSugandha{	[This section is to discussed afresh]} 
Cyclic and abelian $p$-group codes have been largely studied. For instance, the results of Arora and Pruthi \cite{PMA97} on cyclic $p$ codes were extended by Ferraz and Polcino Milies \cite{FRP07} to the results on abelian $p$-group codes.  In this section, we take a step further and study non-abelian metacyclic $p$-group codes, where $p$ is an odd prime.

Following Section 3, we first study metacyclic $p$-group codes for the groups with maximal cyclic subgroup. This is followed by assessing some arbitrary $p$-group codes.
\subsection{Metacyclic $p$-group codes having maximal cyclic subgroup, $p \neq 2$}
Up to isomorphism, there is a unique non-abelian metacyclic group of order $p^{n+1}$, where $p$ is an odd prime and $n \geq 2$, which possesses a maximal cyclic subgroup (c.f. (\cite{Huppert1}, I, Satz 14.9(a))). For this group, we provide a complete list of idempotents using strong Shoda pairs in the next theorem. 
\begin{theorem}\label{pcis_$p^n$}
For $G_{p^{n+1}}= \langle a, b \mid a^{p^{\,n}} = 1,\; b^p = 1,\; b^{-1}ab= a^{\,p^{\,n-1}+1} \rangle$, where $p$ is an odd prime and $n \geq 2$, the strong Shoda pairs of $G_{p^{n+1}}$ along with respective pcis of semisimple group algebra $\mathbb{F}_qG_{p^{n+1}}$ are as listed in \Cref{tab:pcis_G_pn}.
\begin{table}[ht]
	\centering
	\small
	\begin{tabular}{|p{4.2cm}|p{10.8cm}|}
		\hline
		\textbf{Strong Shoda pairs} & \textbf{Primitive central idempotents in $\mathbb{F}_q G_{p^{n+1}}$} \\
		\hline
		
		$(G_{p^{n+1}}, G_{p^{n+1}})$ & 
		$e_0 = \widehat{G_{p^{n+1}}}$ \\
		\hline
		
		$(G_{p^{n+1}}, \langle a \rangle)$ & 
		$e_{1,k} = \tfrac{1}{p}\,\widehat{\langle a \rangle}\;
		\sum\limits_{i=0}^{p-1} \tr(\xi_p^{\,ki})\,b^{-i}$ \\
		\hline
		
		$(G_{p^{n+1}}, \langle a^{p^j}, a^{ip^{j-1}} b \rangle)$, 
		
		$0 \leq i \leq p-1$, $1 \leq j \leq n-1$ 
		& 
		$e_{(i,j),k} =
		\begin{cases}
			\frac{1}{p^j}\,\widehat{\langle a^{p^j}, a^{ip^{j-1}} b \rangle}\;
			\sum\limits_{r=0}^{p^j-1} \tr(\xi_{p^j}^{\,kr})\,a^{-r}, & 1 \leq j \leq i_0, \\[6pt]
			\frac{1}{p^j}\,\widehat{\langle a^{p^j}, a^{ip^{j-1}} b \rangle}\;
			\sum\limits_{r=0}^{p^{i_0}-1} \tr(\xi_{p^j}^{\,kr})\,a^{-rp^{\,j-i_0}}, & j > i_0,
		\end{cases}$ \\
		\hline
		
		$(\langle a \rangle, \langle a^{p^n}\rangle)$ 
		& 
		$e_{p^n,k} =
		\begin{cases}
			\frac{1}{p^{i_0}} \sum\limits_{t \in T}\;\sum\limits_{r=0}^{p^{i_0}-1} 
			\tr(\xi_{p^{i_0}}^{\,krt})\,a^{-r}, & n = i_0, \\[6pt]
			\frac{1}{p^{\,n}} \sum\limits_{t \in T}\;\sum\limits_{r=0}^{p^{i_0}-1} 
			\tr(\xi_{p^{\,n}}^{\,kr t})\,a^{-rp^{\,n-i_0}}, & {\,n} > i_0,
		\end{cases}$ \\[6pt]
		& where $T$ is a transversal of $\langle (1+p^{n-1})^{\omega_0}\rangle$ in 
		$\langle 1+p^{n-1}\rangle$, with $\omega_0$ being the least integer such that 
		$(1+p^{n-1})^{\omega_0} \in \langle q \rangle$. \\
		\hline
		
	\end{tabular}
	\caption{Pcis of $\mathbb{F}_q G_{p^{n+1}}$}
	\label{tab:pcis_G_pn}
	\normalsize
	
	
	
\end{table}
\end{theorem}
The following corollaries are immediate from \Cref{pcis_$p^n$}
\begin{corollary}
	The structure of $	\mathbb{F}_qG$, where $G:=G_{p^{n+1}}$, is
\begin{equation*}\label{wedder p^n}
	\mathbb{F}_qG \cong
	\begin{cases}
		\mathbb{F}_q 
		\bigoplus \dfrac{\phi(p)}{o_p(q)}\,\mathbb{F}_{q^{o_p(q)}}
		\bigoplus  \limits_{j=1}^{n-1}p\dfrac{\phi(p^j)}{o_{p^j}(q)}\,\mathbb{F}_{q^{o_{p^{j}}(q)}} 
		\bigoplus \dfrac{\phi(p^{\,n})}{o_{p^{\,n}}(q)}  
		M_p\!\left( \frac{\mathbb{F}_{q^{o_{p^{\,n}}(q)}}}{p} \right), 
		 \mathrm{if}~ 1+p^{\,n-1} \in \langle q \rangle ~\mathrm{mod}~ p^{n}, \\
		\mathbb{F}_q 
		\bigoplus \dfrac{\phi(p)}{o_p(q)}\,\mathbb{F}_{q^{o_p(q)}}
		\bigoplus  \limits_{j=1}^{n-1}p\dfrac{\phi(p^j)}{o_{p^j}(q)}\,\mathbb{F}_{q^{o_{p^{j}}(q)}} 
		\bigoplus \dfrac{\phi(p^{\,n})}{p o_{p^{\,n}}(q)}  
		M_p\!\left( \mathbb{F}_{q^{o_{p^{\,n}}(q)}} \right),
		\text{otherwise.}
	\end{cases}
\end{equation*}
\end{corollary}

%Its presentation can be given by 
%\[
%G = \langle a, b \mid a^{p^{\,n-1}} = 1,\; b^p = 1,\; b^{-1}ab= a^{\,p^{\,n-2}+1} \rangle,
%\] $n \geq 3$.
%
%
%By the algorithm given in \cite{BM14} and by using the structure of $G$, we have
%\[
%\mathcal{S}(G) = 
%\Big\{
%(G, G),\; (G, \langle a \rangle),\;  (\langle a \rangle, \langle 1 \rangle)
%\Big\} 
%\cup 
%\Big\{(G, \langle a^{p^j}, a^{ip^{j-1}} b \rangle)\;\Big|\;0\leq i \leq p-1,\; 1 \leq j \leq n-2
%\Big\}
%\].



			\begin{corollary}\label{Cor_distance_{p^n}}In the foregoing notation, 
				\begin{enumerate}
						\item $\dim_{\mathbb{F}_q}(\mathbb{F}_qG_{p^{n+1}}e_{p^n, k})=
				po_{p^{n}}(q)\gcd(\omega_0, p)$, where  $\omega_0$ is the least integer such that 
				$(1+p^{n-1})^{\omega_0} \in \langle q \rangle$. 
						\item 	The set  
							$\mathcal{B}=
						\{a^{\eta_a}b^{\eta_b}e_{p^n,k}~|~0\leq \eta_a< \gcd(\omega_0,p)o_{p^{n}}(q),~0\leq \eta_b \leq p-1\}$, where  $\omega_0$ is the least integer such that 
						$(1+p^{n-1})^{\omega_0} \in \langle q \rangle$, is an $\mathbb{F}_q$-basis of $\mathbb{F}_qG_{p^{n+1}}e_{p^n, k}$.
						\item $d=\begin{cases}
						2 \leq d \leq p^{n} ~~\mathrm{if}~~ n\leq i_0\\
						2 \leq d \leq p^{i_0} ~~\mathrm{otherwise}.
						\end{cases}$
						
						
					\end{enumerate}   	
			\end{corollary}
	In view of \Cref{tab:pcis_G_pn} and  the Wedderburn decomposition of $\mathbb{F}_qG_{p^{n+1}}$, we have following:
	
	\begin{corollary}\label{left pcis F_qG_{p^{n+1}}}
	The semisimple group algebra $\mathbb{F}_qG_{p^{n+1}}$, with $G_{p^{n+1}}$ as defined in \Cref{pcis_$p^n$}, decomposes into minimal left ideals generated by a complete set of primitive orthogonal idempotents, given by:
	\begin{itemize}
			\item[(i)] $1 \;+\frac{\phi(p)}{o_p(q)}\,+ p\sum \limits_{j=1}^{n-1}\frac{\phi(p^j)}{o_{p^j}(q)}$ central  idempotents corresponding to strong Shoda pairs of type $(G_{p^{n+1}},K),$ as listed in \Cref*{tab:pcis_G_pn}.
			\item[(ii)] $2\frac{\phi(p^{n})}{\kappa o_{p^{n}}(q)}$ left  idempotents: $e_{p^{n},k}\widehat{\langle b \rangle}$ and $e_{p^{n},k}(1-\widehat{\langle b \rangle})$,	where $\kappa=p$ for $1+p^{n-1} \notin \langle q \rangle \mod p^{n}$ and $\kappa=1$ in all other cases.
		\end{itemize}
	\end{corollary}
%	\begin{itemize}
%		\item If $1+p^{n-2} \in \langle q \rangle \mod p^{n-1}$, then $\mathbb{F}_qG$ has 
%		\[
%		1 \;+\frac{\phi(p)}{o_p(q)}\,+ p\sum_{j=1}^{n-2}\frac{\phi(p^j)}{o_{p^j}(q)}+2\frac{\phi(p^{n-1})}{o_{p^{n-1}}(q)}
%		\]
%		left irreducible idempotents:
%		\begin{itemize}
%			\item[(i)]  $	1 \;+\frac{\phi(p)}{o_p(q)}\,+ p\sum_{j=1}^{n-2}\frac{\phi(p^j)}{o_{p^j}(q)}$ are (left)central irreducible idempotents corresponding to Shoda pair of type $(G, K)$as in the above table.
%			\item[(ii)] $2\frac{\phi(p^{n-1})}{o_{p^{n-1}}(q)}$ left irreducible idempotents of the form $e_{3,k}\widehat{b}$ and $e_{3,k}(I-\widehat{b})$.
%		\end{itemize}
%		
%		\item If  $1+p^{n-2}\notin \langle q \rangle \mod p^{n-1}$, then $\mathbb{F}_qG$ has 
%		\[
%		1 \;+\frac{\phi(p)}{o_p(q)}\,+ p\sum_{j=1}^{n-2}\frac{\phi(p^j)}{o_{p^j}(q)}+2\frac{\phi(p^{n-1})}{p\cdot o_{p^{n-1}}(q)}
%		\]
%		left irreducible idempotents:
%		\begin{itemize}
%			\item[(i)] $	1 \;+\frac{\phi(p)}{o_p(q)}\,+ p\sum_{j=1}^{n-2}\frac{\phi(p^j)}{o_{p^j}(q)}$ are (left)central irreducible idempotents corresponding to Shoda pair of type $(G, K)$as in the above table.
%			\item[(ii)]  $2\frac{\phi(p^{n-1})}{p\cdot o_{p^{n-1}}(q)}$ left irreducible idempotents of the form $e_{3,k}\widehat{b}$ and $e_{3,k}(I-\widehat{b})$.
%		\end{itemize}
%	\end{itemize}
Note that
the group 
	$G_{2^{n+1}} = \langle a, b \mid a^{2^{n}} = 1,\ b^2 = 1,\ b^{-1}ab = a^{1 + 2^{n-1}} \rangle$ ($G_{p^{n+1}}$ of \Cref{pcis_$p^n$} with $p = 2$), 
	 is the metacyclic $2$-group listed as (iv) in Subsection~3.2 and the results obtained in \Cref{pcis_$p^n$} coincide with those derived earlier.


	\subsection{Metacyclic $p$-group codes, $p\neq 2$}
		The classification of non-abelian  metacyclic $p$-groups of order $p^n, n \geq 3$ has been provided in \cite{Ste96}. One can, in principle use the presentation to compute the strong Shoda pairs and hence the idempotents for any of these groups. The strong Shoda pairs of all groups (not necessarily metacyclic) of order  $p^n, n \leq 4$ have been provided in  \cite{BM14} and \cite{BM15} using which the structure of their respective group algebras has been provided in \cite{GM19}. We list the strong Shoda pairs for metacyclic groups of order $p^5$ and consequently provide the description of their group algebras. 
		\begin{proposition}\label{prop_metacyclic_codes_p^5} For an odd prime $p$, there are four non isomorphic non-abelian metacyclic groups of order $p^5$ given by :
		
		\[
		\begin{aligned}
			G_1 &= \langle a, b \mid a^{p^2} = 1,\; b^{p^3} = 1,\; bab^{-1} = a^{p+1} \rangle,\\
			G_2 &= \langle a, b \mid a^{p^3} = 1,\; a^{p^2}=b^{p^2} ,\; bab^{-1} = a^{p+1} \rangle,\\
			G_3 &= \langle a, b \mid a^{p^3} = 1,\; a^{p^2}=b^{p^2} ,\; bab^{-1} = a^{p^2+1} \rangle,\\
			G_4 &= \langle a, b \mid a^{p^4} = 1,\; a^{p}=b^{p} ,\; bab^{-1} = a^{p^3+1} \rangle.
		\end{aligned}
		\]
		A complete set  $\mathcal{S}(G_i)$ of $G_i, 1 \leq i \leq 4$ is as listed below:
		
		\begin{small}
		
			\begin{description}
			\item[$\mathcal{S}(G_1)$] 	$
		%	\mathcal{S}(G_1) 
		=
			\Big\{ (G_1, \langle a, b^{p^i} \rangle) \Big\}_{i=0}^2\cup \Big\{ (G_1, \langle a^i b, a^p \rangle) \Big\}_{i=0}^{p-1}
			\cup \Big\{ (G_1, \langle ab^{ip}, a^p \rangle) \Big\}_{i=0}^{p-1}
			\cup \Big\{ (G_1, \langle ab^{-ip^2}, a^p \rangle) \Big\}_{i=1}^{p-1}
			\cup \Big\{ (\langle a^2 b, b^p \rangle,\; \langle a^{ip} b^{p^2} \rangle) \Big\}_{i=1}^{p-1}
			\cup \Big\{ (\langle a^2 b, b^p \rangle,\; \langle a^{ip} b^p, b^{p^2} \rangle) \Big\}_{\substack{0 \leq i \leq p-1 \\ i \neq 2}}
			\cup \Big\{ (\langle a^2 b, b^p \rangle,\; \langle a^{2p+2} b^{1-2p}, b^{p^2} \rangle) \Big\}.
$
		\item[$\mathcal{S}(G_2)$] 	$
	=
		\Big\{ (G_2, \langle a, b^{p^i} \rangle) \Big\}_{i=0}^{2}
		\cup \Big\{ (G_2, \langle a^k b, a^p \rangle) \Big\}_{k=0}^{p-1}
		\cup \Big\{ (G_2, \langle ab^{kp}, a^p \rangle) \Big\}_{k=1}^{p-1}
		\cup \Big\{ (\langle a^{-1}b, a^p \rangle,\; \langle a^{kp} b^p, b^{p^2} \rangle) \Big\}_{k=0}^{p-1}
		\cup \Big\{ (\langle a^{-1} b, b^p \rangle,\; \langle a^{-1} b^{1-2p}, b^{p^2} \rangle) \Big\}
		\cup \Big\{ (\langle ab^{p^2-p} \rangle, \langle 1 \rangle) \Big\}.$
		\item[$\mathcal{S}(G_3)$] =	$
		\Big\{ (G_3, \langle a^k b, b^{p^2} \rangle) \Big\}_{k=0}^{p^2-1}
		\cup \Big\{ (G_3, \langle a b^{kp}, b^{p^2} \rangle) \Big\}_{k=1}^{p-1}
		\cup \Big\{ (G_3, \langle a^k b, a^p \rangle) \Big\}_{k=0}^{p-1}
		\cup \Big\{ (G_3, \langle a, b^{p^i} \rangle) \Big\}_{i=0}^{2}
		\cup \linebreak\Big\{ (\langle a, b^{p} \rangle,\; \langle a^{p(p-1)} b^{p(pk+1)} \rangle) \Big\}_{k=0}^{p-1}.$
			\item[$\mathcal{S}(G_4)$] =	$
		\Big\{ (G_4, \langle a b^{-1}, b^{p^i} \rangle) \Big\}_{i=0}^{3}
		\cup \Big\{ (G_4, \langle b \rangle) \Big\}
		\cup \Big\{ (\langle a^p, a^{-1} b^2 \rangle,\; \langle 1 \rangle) \Big\}
		\cup \Big\{ (G_4, \langle a b^{k p^i - 1} \rangle) \Big\}_{\substack{1 \leq k \leq p-1 \\ 0 \leq i \leq 2}}.
		$
	\end{description}
\end{small}
\end{proposition}
As a direct consequence we obtain the structure of $\mathbb{F}_qG_i, 1 \leq i \leq 4$ and observe that no two of them are isomorphic.
\begin{enumerate}
	\item[$\mathbb{F}_qG_1 \cong$] $	
 \mathbb{F}_q  \bigoplus \delta  (p+1) \mathbb{F}_{q^{o_p(q)}} \bigoplus  \delta p \Big[
	\mathbb{F}_{q^{o_{p^2}(q)}} \bigoplus   \mathbb{F}_{q^{o_{p^3}(q)}} \Big]  \bigoplus M_{p}\!\left(\mathbb{F}_{q^{ o_{p}(q)}}\right)
	\bigoplus$ \\
	  $ \delta(p-1)\Big[ M_{p}\left(\mathbb{F}_{q^{ \frac{o_{p^3}(q)}{p}}}\right)  
	\bigoplus  M_{p}\!\left(\mathbb{F}_{q^{ \frac{o_{p^2}(q)}{p}}}\right)\Big] .
$
		\item[$\mathbb{F}_qG_2 \cong $] 	$ \mathbb{F}_q  \bigoplus \delta \Big[ (p+1) \mathbb{F}_{q^{o_p(q)}} \bigoplus  
		p\mathbb{F}_{q^{o_{p^2}(q)}}  \bigoplus  (p-1) M_{p}\!\left(\mathbb{F}_{q^{ \frac{o_{p^2}(q)}{p}}}\right) \Big] 
		\bigoplus \delta \Big[ M_{p}\!\left(\mathbb{F}_{q^{ \frac{o_{p^2}(q)}{p}}}\right)  \bigoplus  M_{p}\!\left(\mathbb{F}_{q^{ o_{p}(q)}}\right)\Big].$
			\item[$\mathbb{F}_q G_3 \cong$] 	$ \mathbb{F}_q  \bigoplus  \delta \Big[ (p+1) \mathbb{F}_{q^{o_p(q)}} \bigoplus  
			p(p+1)\mathbb{F}_{q^{o_{p^2}(q)}} \bigoplus  p M_{p}\!\left(\mathbb{F}_{q^{ \frac{o_{p^3}(q)}{p}}}\right) \Big].$
				\item[$\mathbb{F}_q G_4 \cong$] 	$  \mathbb{F}_q \bigoplus \delta \Big[ (p+1) \mathbb{F}_{q^{o_p(q)}} \bigoplus  
				p\mathbb{F}_{q^{o_{p^2}(q)}} \bigoplus p\mathbb{F}_{q^{o_{p^3}(q)}}\bigoplus    M_{p}\!\left(\mathbb{F}_{q^{ \frac{o_{p^4}(q)}{p}}}\right) \Big].$
		
\end{enumerate}
	
			By using  \Cref{prop_metacyclic_codes_p^5}, one can easily obtain the complete list of pcis of $\mathbb{F}_qG_i, 1 \leq i \leq 4.$ For instance, consider the group  $G_1$ and its strong Shoda pairs $\Big\{	(\langle a^2 b, b^p \rangle,\; \langle a^{ip} b^{p^2} \rangle)\Big\}_{i=1}^{p-1}$. Then, we have 
			$$\epsilon_C(\langle a^2 b, b^p \rangle,\; \langle a^{ip} b^{p^2} \rangle )=\frac{1}{p^3}\widehat{\langle a^{ip} b^{p^2} \rangle}[(\Sigma_{\mathfrak{i}=0}^{p-1}\tr(\xi_{p^3}^{kp^2\mathfrak{i}}){{(a^2b)}^{-p^2\mathfrak{i}}})]$$ and  since $(a^2b)^{-p^2\mathfrak{i}}
			\in \mathcal{Z}(G_1)$, it follows that $e_C(G_1, \langle a^2 b, b^p \rangle,\; \langle a^{ip} b^{p^2} \rangle )=\epsilon_C(\langle a^2 b, b^p \rangle,\; \langle a^{ip} b^{p^2} \rangle).$ Consequently, we obtain a minimal group code of length $p^5$ and dimension $p  o_{p^3}(q)$, with minimum distance satisfying 
			\[
			2p \;\leq\; d \;\leq\; p^2,
			\]
			for all admissible choices of $i$ and $k$.
		\subsection{Good $p$-group codes}
\begin{example}
	Over the field $\mathbb{F}_2$, for non-abelian  groups of order $27$, the pcis corresponding to $(G,K)$-types yield codes with parameters $[27, 2, 18]$ by using \Cref{parmameters $G=H$}, which coincide with the best-known binary linear codes of these parameters \cite{Gras}.
\end{example}
\begin{example}\label{example_dihedral_central}
	\underline{$\mathbb{F}_3G$, where	$G:=D_{8} = \langle a, b \mid a^{4} = b^2 = 1,\; a^b = a^{-1} \rangle$.}\\
	Consider the left idempotent $(1-e)\widehat{b}$ where,
	$e := e_C(G, G, G)$. Then the code generated by $(1-e)\widehat{b}$ has parameters $[8, 3, 4]$, which is very close to the best-known $[8, 3, 5]$ code.
\end{example}
We now also consider a non metacyclic $2$-group code.
\begin{example}\label{example_order16}
	\underline{$\mathbb{F}_3 G$, where $G \cong C_2 \times Q_8$}.\\
Let $G$ be presented as	\[
	G := \langle a, b, c \mid a^{4} = b^{4} = c^{2} = 1,\; ba = a^{3}b,\; ca = ac,\; cb = bc, \;a^{2} = b^{2} \rangle.
	\]
We have that
	\[
	\mathcal{S}(G) = \{(G, G),\; (G, \langle a, b\rangle),\; (G, \langle b^i c \rangle),\; (G, \langle a^2, a^i b, a^j c\rangle) (\langle a, c \rangle, \langle a^i c \rangle) \mid 0 \leq i, j \leq 1 \}.
	\]
	Consider
	$e=1-(e_1 + e_2+e_3),$ where $e_1=e_C(G, G, G)=\widehat{G}$, $e_2=e_C(G, G, \langle a, b \rangle)$ and $e_3=e_C(G,\langle a,c\rangle, \langle c \rangle).$ 
	The code generated by $e$ is a $[16, 10, 4]$ code, and is a best-known code.
\end{example}

\section{Metacyclic group codes of arbitrary length}
So far we have considered metacyclic codes of groups whose order is divisible by at most two primes.  In this section, we  investigate metacyclic codes of length divisible by more than two primes. The direct products of metacyclic groups of relatively prime order are also considered.

\begin{theorem}\label{thm:character_sum_vanish}
	Let $p_1, p_2$ be distinct odd primes with $p_1 < p_2$ such that $p_1 \nmid (p_2-1)$, and  let $q$ be a natural number relatively prime to both $p_1$ and $p_2$. For $m, l \in \mathbb{N}$, suppose 	\[
	o_{p_1^{m}}(q) = \frac{\phi(p_1^{m})}{\delta_1}
~\text{and}~
	o_{p_2^{\ell}}(q) = \frac{\phi(p_2^{\ell})}{\delta_2},
	\]
	where $\delta_1 = p_1^{\,i_0^{(1)}-1}\delta_1'$ and  $\delta_2 = p_2^{\,i_0^{(2)}-1}\delta_2'$ with $\gcd(\delta_1', p_1) = \gcd(\delta_2', p_2) = 1$. 
	 If $n = p_1^{m} p_2^{\ell}$, then 
	 for every integer  $k$ with $\gcd(k,n) = 1$ and for every $j_1, j_2$ such that $1 \leq j_1 \leq m$ and $1 \leq j_2 \leq l$,
	\[
	\sum_{i=0}^{o_{p_1^{j_1}p_2^{j_2}}(q)-1} \xi_{p_1^{j_1}p_2^{j_2}}^{k q^i} = 0 
	\quad \Longleftrightarrow \quad 
	j_1 > i_0^{(1)} ~~\text{or}~~\ j_2 > i_0^{(2)}.
	\]
\end{theorem}
\begin{proof}
	Set $J = \{ q^i \mid 0 \leq i < o_{p_1^{j_1}p_2^{j_2}}(q)\}, 
	$ so that 	$\sum_{i=0}^{o_{p_1^{j_1}p_2^{j_2}}(q)-1} \xi_{p_1^{j_1}p_2^{j_2}}^{\,k q^i}
	= \sum_{q^i \in J} \xi_{p_1^{j_1}p_2^{j_2}}^{\,k q^i}.
	$
	Denote $ \mathrm{lcm}\!\left(\frac{p_1-1}{\delta_1'}, \frac{p_2-1}{\delta_2'}\right)$ by $\lambda$ and using division algorithm write $i=\lambda u+v$ with $0 \leq v \leq \lambda$.\\
		\emph{Case (i).} Suppose $i_0^{(1)} < j_1 \leq m$ and $i_0^{(2)} < j_2 \leq \ell$.  
	In this case 
	$o_{p_1^{j_1} p_2^{j_2}}(q) 
	= p_1^{j_1 - i_0^{(1)}} p_2^{j_2 - i_0^{(2)}} \lambda
	$ and $0 \leq u < p_1^{j_1 - i_0^{(1)}} p_2^{j_2 - i_0^{(2)}}$. We thus have for $q^i \in J$, $q^i=(q^\lambda)^u q^v$
which modulo $p_1^{j_1}p_2^{j_2}$ equals $(1 + p_1^{i_0^{(1)}} p_2^{i_0^{(2)}} c)^u q^v $ and 
	$(1 + p_1^{i_0^{(1)}} p_2^{i_0^{(2)}} c)^u q^v 
	\equiv q^v + p_1^{i_0^{(1)}} p_2^{i_0^{(2)}} w\mod~p_1^{j_1}p_2^{j_2}$, where $0 \leq w < p_1^{j_1-i_0^{(1)}}p_2^{j_2-i_0^{(2)}} $. Thus $J$ can be described as
	\[
	J = \bigl\{\, q^v + p_1^{i_0^{(1)}}p_2^{i_0^{(2)}} w 
	\ \bigm|\ 
	0 \leq v < \lambda,\ 
	0 \leq w < p_1^{j_1-i_0^{(1)}}p_2^{j_2-i_0^{(2)}} \bigr\}.
	\]
		It follows that
	\[
	\sum_{q^i \in J}\xi_{p_1^{j_1}p_2^{j_2}}^{\,k q^i}
	= \sum_{v=0}^{\lambda-1}\xi_{p_1^{j_1}p_2^{j_2}}^{\,k q^v}
	\sum_{w=0}^{p_1^{j_1-i_0^{(1)}}p_2^{j_2-i_0^{(2)}}-1}
	\bigl(\xi_{p_1^{j_1}p_2^{j_2}}^{\,k p_1^{i_0^{(1)}}p_2^{i_0^{(2)}}}\bigr)^w.
	\]
	The inner geometric sum vanishes since the base is a nontrivial root of unity of order $p_1^{j_1-i_0^{(1)}}p_2^{j_2-i_0^{(2)}}$. Hence the entire sum equals zero.
	
	\medskip
		\emph{Case (ii).} Suppose $i_0^{(1)} < j_1 \leq m$ and $1 \leq j_2 \leq i_0^{(2)}$ (the argument is symmetric if $i_0^{(2)} < j_2 \leq \ell$ and $1 \leq j_1 \leq i_0^{(1)}$). 
		In this case $o_{p_1^{j_1} p_2^{j_2}}(q) = p_1^{j_1 - i_0^{(1)}} \lambda.$ Hence, for  $q^i \in J$, writing $i = \lambda u + v$ where $0 \leq u < p_1^{j_1-i_0^{(1)}}.$
Therefore, $q^i
	\equiv (1 + p_1^{i_0^{(1)}}p_2^{i_0^{(2)}}c)^u q^v
	\equiv q^v + p_1^{i_0^{(1)}}p_2^{i_0^{(2)}} w 
	\mod{p_1^{j_1}p_2^{j_2}},$
	for	some $c,w \in \mathbb{Z}$.
	Therefore
	\[
	J = \bigl\{\, q^v + w p_1^{i_0^{(1)}} p_2^{i_0^{(2)}}
	\ \bigm|\ 
 0 \leq v < \lambda ,	0 \leq w < p_1^{j_1-i_0^{(1)}}\bigr\}.
	\]
	It follows that
	\[
	\sum_{q^i \in J} \xi_{p_1^{j_1}p_2^{j_2}}^{\,k q^i}
	= \sum_{v=0}^{\lambda-1} \xi_{p_1^{j_1}p_2^{j_2}}^{\,k q^v}
	\sum_{w=0}^{p_1^{j_1-i_0^{(1)}}-1}
	\bigl(\xi_{p_1^{j_1}p_2^{j_2}}^{\,k p_1^{i_0^{(1)}}	p_2^{i_0^{(2)}}}\bigr)^w.
	\]
	The inner sum vanishes since $\xi_{p_1^{j_1}p_2^{j_2}}^{\,k p_1^{i_0^{(1)}}p_2^{i_0^{(2)}}}$ is a nontrivial root of unity of order $p_1^{j_1-i_0^{(1)}}$. Hence the whole sum equals zero. 
	
	\medskip
	
	\emph{Case (iii).} Suppose $1 \leq j_1 \leq i_0^{(1)}$ and $1 \leq j_2 \leq i_0^{(2)}$.  
	In this case $o_{p_1^{j_1}p_2^{j_2}}(q)$ is coprime to $p_1p_2$.  and the sum 
	$\sum \limits_{i=0}^{o_{p_1^{j_1}p_2^{j_2}}(q)-1} \xi_{p_1^{j_1}p_2^{j_2}}^{k q^i} \neq 0.$
	\end{proof}
In view of \Cref*{lemma_trace_zero 2}, similar proof can be extended if one of the prime is equal to $2$. We provide the results without details. 
\begin{proposition}\label{thm:char_sum_vanish_2p_refined}
	Let $p$ be an odd prime and let $q$ be a power of a prime with $\gcd(q,2p)=1$. Write $q=\pm1+2^{\,i_0}c,$ where $ c$ is odd and $ i_0\ge2.$
	For $ m,\ell\in\mathbb{N}$, let $n=2^{m}p^{\ell}$ . Suppose
	$o_{p^{\ell}}(q)=\frac{\varphi(p^{\ell})}{\delta_p}$ with $ 
	\delta_p=p^{\,i_0^{(p)}-1}\delta_p'$, where $\gcd(\delta_p',p)=1.	$ 
	
	If $1\le j_1\le m$, $1\le j_2\le\ell$, then for every $k\in\mathbb{Z}$ with $\gcd(k,n)=1$,
	\[
	\sum_{t=0}^{o_{2^{j_1}p^{j_2}}(q)-1}\xi_{2^{j_1}p^{j_2}}^{\,kq^t}=0
	\]
	holds exactly in the following cases:
	\begin{enumerate}
		\item $j_2>i_0^{(p)}$; or
		\item $j_1>i_0$; or
		\item $j_1=2$ and $q\equiv -1+2^{\,i_0}c$, $c$ odd .
	\end{enumerate}
	\end{proposition}
\begin{remark}
	\begin{itemize}
	\item[(i)] Let 	$D_{2n} = \langle a, b~ | ~a^n = b^2 = 1, \; b^{-1}ab = a^{-1} \rangle,$ of order $2n$, where $n \geq 3.$\\
		A complete list of strong Shoda pairs of $G:=D_{2n}$ is given by  	\[
	%	\mathcal{S}(G) :=
%		\begin{cases}
%			\{(D_{2n}, D_{2n}), \ (D_{2n}, \langle a \rangle), \ (\langle a \rangle, \langle a^{v} \rangle) \mid v \neq 1, v \mid n, \  \}, & \text{if $n$ is odd}, \\[6pt]
%			\{(D_{2n}, D_{2n}), \ (D_{2n}, \langle a \rangle), \ (D_{2n}, \langle a^2, b \rangle), \ (D_{2n}, \langle a^2, ab \rangle), \ (\langle a \rangle, \langle a^{v} \rangle) ~|~  v >2, v \mid n\}, & \text{if $n$ is even}.
%		\end{cases}
%		\]
	%	\[
		\mathcal{S}(G) :=
		\begin{cases}
			\{(G, G), \ (G, \langle a \rangle), \ (\langle a \rangle, \langle a^{v} \rangle) \mid v \neq 1, \ v \mid n \}, & \text{if $n$ is odd}, \\[6pt]
			\{(G, G), \ (G, \langle a \rangle), \ (G, \langle a^2, b \rangle), \ (G, \langle a^2, ab \rangle), \ (\langle a \rangle, \langle a^{v} \rangle) \mid v > 2, \ v \mid n\}, & \text{if $n$ is even}.
		\end{cases}
		\]
		
		Thus, using \Cref*{thm:character_sum_vanish} and \Cref{thm:char_sum_vanish_2p_refined}, one can extend the computations of~\cite[Section~4]{CM25} to the case where $n$ is the product of two distinct primes, which can further be extended to the case arbitrary $n$ and $q$ such that $n$ and $q$ are arbitrary. Thereby, extending \cite{DFPM09, GR22b, GR22a, GR23, Mar15}.
		
			\item[(ii)]Likewise for $Q_{4m} = \langle a, b \mid a^{2m} = 1, \; b^2 = a^{m},  b^{-1}ab = a^{-1} \rangle$, the generalized quaternion group, 	of order $4m$,  $m \geq 2$, similar results may be obtained using 
		a complete list of strong Shoda pairs of $G:=Q_{4m}$  given by  
%		\[
%	\mathcal{S}(Q_{4m}) :=
%		\begin{cases}
%			\{(Q_{4m}, Q_{4m}), \ (Q_{4m}, \langle a \rangle), \ (Q_{4m}, \langle a^2 \rangle), \ (\langle a \rangle, \langle a^{v} \rangle) |  v >2, ~v \mid 2m, \ \}, ~ \text{if $m$ is odd}, \\[6pt]
%			\{(Q_{4m}, Q_{4m}), \ (Q_{4m}, \langle a \rangle), \ (Q_{4m}, \langle a^2, b \rangle), \ (Q_{4m}, \langle a^2, ab \rangle), \ (\langle a \rangle, \langle a^{v} \rangle) |v>2, v \mid 2m,   \}, ~ \text{if $m$ is even}.
%		\end{cases}
%		\]
		\[
		\mathcal{S}(G) :=
		\begin{cases}
			\{(G, G), \ (G, \langle a \rangle), \ (G, \langle a^2 \rangle), \ (\langle a \rangle, \langle a^{v} \rangle) \mid v > 2, \ v \mid 2m \}, & \text{if $m$ is odd}, \\[6pt]
			\{(G, G), \ (G, \langle a \rangle), \ (G, \langle a^2, b \rangle), \ (G, \langle a^2, ab \rangle), \ (\langle a \rangle, \langle a^{v} \rangle) \mid v > 2, \ v \mid 2m \}, & \text{if $m$ is even}.
		\end{cases}
		\]
		
	Consequently, this generalizes the study of quaternion 
	 codes of order $4m$ to arbitrary $m$ (see~\cite{GY21}).
	\end{itemize}
\end{remark}

\begin{example}\label{example_dihedral_central}
	\underline{$\mathbb{F}_5D_{12}, D_{12}= \langle a, b \mid a^{6}=b^2=1,\; a^b=a^{-1}\rangle.$} \\
Set $
	e :=e_C(D_{12}, D_{12}, \langle a^2, ab \rangle) \;+\; \widehat{b} e_C(D_{12}, \langle a \rangle, \langle 1 \rangle).
$
	Then the group code $\mathbb{F}_5 D_{12}\, e$ has parameters $[12,3,8]$, which is best known.
\end{example}
\begin{remark}
Let $G_1$ and $G_2$ be two groups with $\gcd(|G_1|, |G_2|)=1$. If $(H_1, K_1)$ and $(H_2, K_2)$ are respectively the strong Shoda pairs of $G_1$ and $G_2$, then one can  check that 
 $(H_1 \times H_2, K_1 \times K_2)$ is a strong Shoda pair of $G_1 \times G_2$. Consequentely, $\mathcal{S}(G_1 \times G_2)$ is obtainable from $\mathcal{S}(G_1)$ and $\mathcal{S}(G_2)$. This enables us to work with groups of order divisible by more than two primes.
\end{remark}
%\begin{proposition}\label{Shoda_pairs}
%	Let $(H_1, K_1)$ and $(H_2, K_2)$ be strong Shoda pairs of finite groups $G_1$ and $G_2$, respectively. If $\gcd(|G_1|, |G_2|) = 1$, then $(H_1 \times H_2, K_1 \times K_2)$ is a strong Shoda pair of $G_1 \times G_2$.
%\end{proposition}
%\begin{proof}
%	Since $K_i \trianglelefteq H_i$ and $H_i/K_i$ is cyclic for $i = 1,2$, we have 
%	$
%	K := K_1 \times K_2 \trianglelefteq H := H_1 \times H_2,
%	\quad \text{and} \quad 
%	H/K \cong (H_1/K_1) \times (H_2/K_2),
%	$
%	which is cyclic because the orders of $H_1/K_1$ and $H_2/K_2$ are coprime.  
%	Also, $N_{G_1 \times G_2}(K) \cong N_{G_1}(K_1) \times N_{G_2}(K_2),
%	\quad \text{so} \quad
%	N_{G_1 \times G_2}(K)/K \cong (N_{G_1}(K_1)/K_1) \times (N_{G_2}(K_2)/K_2).$
%	Since each $H_i/K_i$ is a maximal abelian subgroup of $N_{G_i}(K_i)/K_i$, it follows that $H/K$ is a maximal abelian subgroup of $N_{G_1 \times G_2}(K)/K$ (again because of coprime orders $G_1$ and $G_2$, every subgroup of $G_1 \times G_2$ is of the form $L_1 \times L_2$, where $L_1$ and $L_2$ are subgroups of $G_1$ and $G_2$ respectively).  
%	
%	Let $\varepsilon = \varepsilon(H_1 \times H_2, K_1 \times K_2)$ be the element of $\mathbb{F}_q[G_1 \times G_2]$ corresponding to the pair $(H_1 \times H_2, K_1 \times K_2)$.  
%	Since $\mathbb{F}_q[G_1 \times G_2] \cong \mathbb{F}_q[G_1] \otimes \mathbb{F}_q[G_2]$, we have
%	$	\varepsilon = \varepsilon_1 \otimes \varepsilon_2,$
%	where $\varepsilon_i = \varepsilon(H_i, K_i) \in \mathbb{F}_q[G_i]$.  
%	The $G_1 \times G_2$-conjugates of $\varepsilon$ are of the form 
%	$	\varepsilon^{(g_1, g_2)} = \varepsilon_1^{g_1} \otimes \varepsilon_2^{g_2},
%	\quad (g_1, g_2) \in G_1 \times G_2.
%	$
%	Suppose $(g_1, g_2)$ and $(g_1', g_2') \in G_1 \times G_2$ yield distinct conjugates. Then
%	$
%	(\varepsilon_1^{g_1} \otimes \varepsilon_2^{g_2})(\varepsilon_1^{g_1'} \otimes \varepsilon_2^{g_2'}) 
%	= (\varepsilon_1^{g_1} \cdot \varepsilon_1^{g_1'}) \otimes (\varepsilon_2^{g_2} \cdot \varepsilon_2^{g_2'}) = 0,
%	$
%	unless $\varepsilon_1^{g_1} = \varepsilon_1^{g_1'}$ and $\varepsilon_2^{g_2} = \varepsilon_2^{g_2'}$.  
%	That is, distinct conjugates are orthogonal, which shows that $(H_1 \times H_2, K_1 \times K_2)$ satisfies condition (SS3).  
%	Hence, all three conditions (SS1)–(SS3) are satisfied, and $(H_1 \times H_2, K_1 \times K_2)$ is a strong Shoda pair of $G_1 \times G_2$.
%\end{proof}

%Consequently, we obtain the following corollary.
%\begin{corollary}\label{Shoda_pairs_general}
%	Let $(H_i,K_i)$ be strong Shoda pairs of finite groups $G_i$ for $i=1,\dots,k$. 
%	Assume the orders of the groups are pairwise coprime, i.e.
%	\[
%	\gcd(|G_i|,|G_j|)=1 \quad \text{for all } i\neq j.
%	\]
%	Then
%	\[
%	\big(H_1 \times H_2 \times \cdots \times H_k,\; K_1 \times K_2 \times \cdots \times K_k\big)
%	\]
%	is a strong Shoda pair of 
%	\[
%	G_1 \times G_2 \times \cdots \times G_k.
%	\]
%\end{corollary}
%\begin{remark}
%	Thus, if we consider 
%	$	G_1 = \langle a,b \mid a^{p_1^{m_1}} = b^{p_2^{l_1}} = 1,\ a^b = a^{r_1} \rangle,
%	$
%	with $a^b = b^{-1}ab$ and $m_1, l_1, r_1 \in \mathbb{N}$ such that 
%	$
%	o_{p_1^{m_1}}(r_1) = p_2^{l_1},
%	$
%	and $G_2 = \langle c,d \mid c^{p_3^{m_2}} = d^{p_4^{l_2}} = 1,\ c^d = c^{r_2} \rangle,
%	$
%	with $c^d = d^{-1}cd$ and $m_2, l_2, r_2 \in \mathbb{N}$ such that 
%	$
%	o_{p_3^{m_2}}(r_2) = p_4^{l_2},
%	$
%	where $p_1, p_2, p_3,$ and $p_4$ are distinct primes.  
%	
%	Then, using \Cref{Shoda_pairs}, we obtain a complete list of strong Shoda pairs of $G_1 \times G_2$, and hence a complete list of primitive central idempotents (pcis). Moreover, by applying \Cref*{lemma_order_composite} and \Cref*{thm:character_sum_vanish}, one can explicitly determine the support of these pcis. Thus can do the similar computations. Hence, we will skip the details.  
%\end{remark}

We illustrate the above discussion with an explicit example.
\begin{example}\label{example:G1xG2}
	Consider the groups
	$G_1 \;=\; \langle a_1,\,b_1 \mid a_1^{7^3}=1,\; b_1^{3}=1,\; b_1a_1b_1^{-1}=a_1^{18}\rangle$ and $G_2 \;=\; \langle a_2,\,b_2 \mid a_2^{11}=1,\; b_2^{5}=1,\; b_2a_2b_2^{-1}=a_2^{4
	}\rangle.$
		Here, $	
	\mathcal{S}(G_1)\;=\;\{(G_1,G_1),\ (G_1,\langle a_1\rangle),\ (\langle a_1\rangle,\langle a_1^{7^{j_1}}\rangle)_{j_1=1}^{3}\}$ and $	\mathcal{S}(G_2)\;=\;\{(G_2,G_2),\ (G_2,\langle a_2\rangle),\ (\langle a_2\rangle,\langle 1\rangle)\}.$
	Thus for $G=G_1 \times G_2$, the strong Shoda pairs of $G$ are listed in \Cref*{tab:pcis_G1xG2} along with the  expressions of pcis in $\mathbb{F}_2G$ computed  using \Cref*{thm:character_sum_vanish}. 
	\begin{center}
		\begin{table}[ht]
			\centering
			\begin{tabular}{|c|l|}
				\hline
				\textbf{Strong Shoda Pair $(H,K)$} & \textbf{Primitive central idempotent $e$ in $\F_2[G]$} \\
				\hline
				
				$(G,G)$ & \(\displaystyle e_0=\widehat{G}.\) \\[6pt]
				\hline
				
				$(G,\ \langle a_1\rangle\times G_2)$ & 
				\(
				\displaystyle e_{1,k} \;=\; \frac{1}{3}\,\widehat{\langle a_1\rangle\times G_2}\;
				\sum_{i=0}^{2}\tr\!\big(\xi_{3}^{k i}\big)\,(b_1,  1  )^{-i}
				\)
				
				\\
				\hline
				
				$(G,\ G_1\times\langle a_2\rangle)$ &
				\(
				\displaystyle e_{2,k} \;=\; \frac{1}{5}\,\widehat{G_1\times\langle a_2\rangle}\;
				\sum_{i=0}^{4}\tr\!\big(\xi_{5}^{k i}\big)\,(1, b_2)^{-i},
				\)
				\\[10pt]
				\hline
				
				$(G,\ \langle a_1\rangle\times\langle a_2\rangle)$ &
				\(
				\displaystyle e_{3,k}
				= \frac{1}{15}\,\widehat{\langle a_1\rangle\times\langle a_2\rangle}
				\sum_{i=0}^{14}
				\tr\!\big(\xi_{15}^{k i}\big)
				(b_1, b_2)^{-i},
				\)
				\\[12pt]
				\hline
				
				$(\langle a_1\rangle\times G_2,  \langle a_1^{7^{j_1}}\rangle)\times G_2 )_{j_1=1}^{3}$ &
				\(
				\displaystyle e_{4,k}^{j_1} \;= \frac{1}{7^{j_1}}\widehat{ \langle a_1^{7^{j_1}}\rangle)\times G_2}\sum_{i=0}^{6}\tr\!\big(\xi_{7^{j_1}}^{7^{j_1-1}ki}\big)\,(a_1, 1 )^{-i7^{j_1-1}}
				\)
				\\[8pt]
				\hline
				$(\langle a_1\rangle\times G_2,  \langle a_1^{7^{j_1}}\rangle)\times \langle a_2 \rangle )_{j_1=1}^{3}$ &
				\(
				\displaystyle e_{5,k}^{j_1} \;= \frac{1}{7^{j_1}}\widehat{\langle a_1^{7^{j_1}}\rangle)\times \langle a_2 \rangle}\sum_{i=0}^{6}\tr\!\big(\xi_{5\cdot7^{j_1}}^{7^{j_1-1}ki}\big)\,(a_1, b_2 )^{-i7^{j_1-1}}
				\)
				\\ \hline
				
				
				$(G_1\times\langle a_2\rangle, G_1\times \langle 1 \rangle)$ &
				\(
				\displaystyle e_{6,k} \;=\; \frac{1}{11}\widehat{G_1}\sum_{i=0}^{10}\tr\!\big(\xi_{11}^{ki}\big)(a_2)^{-i},
				\)
				\\[8pt]
				\hline
				$(G_1\times\langle a_2\rangle, \langle a_1 \rangle \times \langle 1 \rangle )$ &
				\(
				\displaystyle e_{7,k} \;=\; \frac{1}{33}\sum_{i=0}^{32}\tr\!\big(\xi_{33}^{ki}\big)(b_1, a_2)^{-i},
				\)
				\\[8pt]
				\hline
				
				$(\langle a_1\rangle\times\langle a_2\rangle, \langle a_1^{7^{j_1}}\rangle \times \langle 1 \rangle)_{j_1=1}^{3}$ &
				\(
				\displaystyle e_{8,k}^{j_1} \;= \frac{1}{7^{j_1}\cdot 11}\widehat{\langle a_1^{7^{j_1}}\rangle}
				\sum_{i=0}^{6}
				\tr\!\big(\xi_{7^{j_1}\cdot 11}^{i7^{j_1-1}k}\big)(a_1, a_2)^{-i7^{j_1-1}}.
				\)
				\\[12pt]
				\hline
					\end{tabular}
			\caption{Pcis of $\mathbb{F}_2(G_1 \times G_2)$.}
			\label{tab:pcis_G1xG2}
		\end{table}
			\end{center}
	Further, the codes generated by the pcis  corresponding to strong Shoda pairs $(H, K)$, where $H$ is proper subgroup of $G$ are provided in \Cref*{tab:params_G1xG2_noncentral}.
	\begin{table}[ht]
		\centering
		\setlength{\tabcolsep}{8pt}
		\renewcommand{\arraystretch}{1.2}
		\begin{tabular}{|l|c|c|c|}
			\hline
			\textbf{Strong Shoda pair $(H,K)$} & \textbf{Idempotent} & \textbf{Dimension $k$} & \textbf{Distance $d$} \\ \hline
			$(\langle a_1\rangle\times G_2,\ \langle a_1^{7^{j_1}}\rangle\times G_2)_{j_1=1}^{3}$ 
			& $e_{4,k}^{j_1}$ & $ 9\cdot 7^{j_1-1}$ & $110 \cdot 7^{3-j_1}  \leq d \leq 330 \cdot 7^{3-j_1}$\\ \hline
			$(\langle a_1\rangle\times G_2,\ \langle a_1^{7^{j_1}}\rangle\times\langle a_2\rangle)_{j_1=1}^{3}$ 
			& $e_{5, k}^{j_1}$ & $36\cdot 7^{j_1-1}$ & $22 \cdot 7^{3-j_1} \leq d \leq 66\cdot 7^{3-j_1}$ \\ \hline
			$(G_1\times\langle a_2\rangle,\ G_1\times \langle 1\rangle)$ 
			& $e_{6, k}$ & $50$ & $6\cdot 7^3 \leq d \leq 10\cdot 7^3$ \\ \hline
			$(G_1\times\langle a_2\rangle,\ \langle a_1\rangle\times \langle 1\rangle)$ 
			& $e_{7, k}$ & $50$ & $2\cdot 7^3 \leq d \leq 33\cdot 7^3$ \\ \hline
			$(\langle a_1\rangle\times\langle a_2\rangle,\ \langle a_1^{7^{j_2}}\rangle\times \langle 1\rangle)_{j_1=1}^{3}$ 
			& $e_{8, k}^{j_1}$ & $450\cdot 7^{j_1-1}$ & $2\cdot 7^{3-j_1} \leq d \leq 6\cdot 7^{3-j_1}$ \\ \hline
		\end{tabular}
		\caption{Parameters for codes corresponding to pcis of $\mathbb{F}_2(G_1 \times G_2).$}
		\label{tab:params_G1xG2_noncentral}
	\end{table}
	
\end{example}
\section{Some Non-Central Good Codes}
In this section, we aim to construct non-central codes arising from the central ones.  As observed in \cite{CM25}, non-central codes often exhibit better distance parameters and they are not always equivalent to abelian codes. Since such codes correspond to left ideals, we need to consider left idempotents. In view of \Cref{left pcis F_qD_{2^{n+1}}}, \Cref{left pcis F_qG_{2^{n+1}}}, and \Cref{left pcis F_qG_{p^{n+1}}}, it is sufficient to consider left group codes generated by idempotents of the form 
$e_C(G, H, K)\widehat{\langle b \rangle}$,
where \( H \) is a proper subgroup of the metacyclic group \( G \). Since the construction of these codes essentially depends on explicit expressions of the idempotents, we perform the computations for the idempotents \( e_{p_1^{j_1}, k_1}\in \mathbb{F}_qG \), which were obtained in our earlier work~\cite{CM25} for $G$ as in (\ref{equation 3}). We first consider codes generated by 
$\mathbb{F}_q G e_{p_1^{j_1}, k_1} \widehat{\langle b^\beta \rangle}, \quad 1 \leq \beta \leq p_2^l - 1.$
\begin{theorem}\label{thm:noncentral-code}
	Let $C=\mathbb{F}_q G e_{p_1^{j_1}, k_1}\widehat{\langle b^\beta \rangle}$ be a non-central code.  
	If $\dim$ and $d$ respectively denote its dimension and minimum distance, then
	\[
	\dim = o_{p_1^{j_1}}(q)\, p_2^{\lambda+\lambda_0},
	\]
	and
	\[
	d=
	\begin{cases}
		2 p_1^{m-j_1}\, p_2^{\,l-\lambda} \;\leq d \leq\; p_1^m \, p_2^{\,l-\lambda}, & 1 \leq j_1 \leq i_0^{(1)}, \\[4pt]
		2 p_1^{m-j_1} p_2^{\,l-\lambda} \;\leq d \leq\; p_1^{\,m-j_1+i_0^{(1)}} \, p_2^{\,l-\lambda}, & \text{otherwise},
	\end{cases}
	\]
	where $\gcd(\beta, p_2^l)=p_2^\lambda$ and $\gcd(\omega_0, p_2^l)=p_2^{\lambda_0}$.
\end{theorem}

\begin{proof}
	Let $f := e_{p_1^{j_1}, k_1}\widehat{\langle b^\beta \rangle}$ and $o := o_{p_1^{j_1}}(q)$.  
	
	Case 1: $r \in \langle q \rangle$.
	We claim that
	\[
	\mathcal{B} := \{\, a^{\eta_a} b^{\eta_b} f \;\mid\; 0 \leq \eta_a < o,\; 0 \leq \eta_b < p_2^\lambda \,\}
	\]
	is a basis for the left ideal $\mathbb{F}_q G f$.  	Indeed, for any $\alpha \in \mathbb{F}_q G f$, write $\alpha = \alpha' \widehat{\langle b^\beta \rangle}$ with  
	\[
	\alpha' = \sum_{\eta_b=0}^{p_2^l-1} \sum_{\eta_a=0}^{o-1} 
	\alpha_{\eta_a\eta_b}\, a^{\eta_a} b^{\eta_b} e_{p_1^{j_1}, k_1}.
	\]
	Partitioning $\eta_b$ by a transversal $T$ of $\langle b^\beta \rangle$ in $\langle b\rangle$ shows that $\mathcal{B}$ spans $\mathbb{F}_q G f$.  
	Linear independence follows from the independence of $\{a^{\eta_a} e_{p_1^{j_1},k_1}\}_{0\leq \eta_a<o}$.  
	Thus $|\mathcal{B}| = o\, p_2^\lambda$, giving $\dim = o\, p_2^\lambda$.  
	
	Case 2: $r \notin \langle q \rangle$. 
	By the same argument as in \cite[Theorem~3.5]{CM25}, the $\mathbb{F}_q$ basis of $\mathbb{F}_qGe_{p_1^{j_1},k_1}$ is
	\[
	\mathcal{B}' = \{\, a^{\eta_a} b^{\eta_b} \, \epsilon_C(\langle a \rangle,\langle a^{p_1^{j_1}}\rangle)^t \widehat{\langle b^\beta \rangle} 
	\;\mid\; 0\leq \eta_a<o,\; 0\leq \eta_b < p_2^\lambda,\; t\in \tau \,\},
	\]
	where $\tau$ is a transversal of $\langle a, b^{\omega_0}\rangle$ in $G$.  
	Hence $\dim = o\, p_2^{\lambda+\lambda_0}$.
	
For the distance bound, let $K = \langle a^{p_1^{j_1}} \rangle$ and $N = \langle b^\beta \rangle$, so that $NK \leq G$. By the method of \Cref{parmameters $G=H$}, the code parameters satisfy the stated bounds.

\end{proof}
\subsection*{Non-central codes using units}
We first recall some well known units of integral group ring  $\mathbb{Z}G$ which remain units in semisimple group ring  $\mathbb{F}_q G$  (for instance see \cite{DJ15}, 1.2.4).
\medskip

\noindent
\textbf{Bicyclic units.} For $g,h \in G$, with $\tilde{h}=1+h+\cdots+h^{|h|-1}$, define
\[
b(g, \tilde{h}) = 1 + (1 - h)g\tilde{h}, 
\quad 
b(\tilde{h}, g) = 1 + \tilde{h}g(1 - h).
\]
These are units with inverses
$b(g, \tilde{h})^{-1} = b(-g, \tilde{h})$ and $b(\tilde{h}, g)^{-1} = b(\tilde{h}, -g)$.

\medskip

\noindent 
\textbf{Bass units.} Consider $x \in G$, an element of order $n$. Let $k$ be relatively coprime to  $ n$ and let $m>0$ be such that $k^m \equiv 1 \mod{n}$. Then
\[
u_{k,m}(x) = \Big(1 + x + \cdots + x^{k-1}\Big)^m + \frac{1-k^m}{n}\,\tilde{x}
\]
is invertible, with inverse $u_{l,m}(x^k)$ where $kl \equiv 1 \mod{n}$.\\
\textbf{Alternating units.} Let $g \in G$, an element of odd order $n$ and let $k$ coprime to $2n$. Then,$$u_k(g) := 1 + g + g^2 + \cdots + g^{k-1}$$ is a  unit.  Consider $u_{k}(g)$ in  $\mathbb{F}_q G$ where $q$ is a power of $2$
and  $g \in G$ be an element of order  $p^m$, $p\neq 2$. If $k <p$ and let $k_1 \in \mathbb{Z}_{>0}$ is such that 
\[
k k_1 \equiv 1 \mod{p^m},
\] then
\[
u_k(g)^{-1} =
\begin{cases}
	u_{k_1}(g^k), & \text{if } k_1 \text{ is odd}, \\
	u_{k_1}(g^k) + \tilde{g}, & \text{if } k_1 \text{ is even}.
\end{cases}
\] This is because
$u_k(g) \cdot u_{k_1}(g^k) = \left( \sum_{i=0}^{k-1} g^i \right) \left( \sum_{j=0}^{k_1 - 1} g^{kj} \right) = \sum_{t=0}^{k k_1 - 1} g^t.$
Since \( k k_1 \equiv 1 \mod{p^m} \) and \( g \) has order \( p^m \). Thus, the sum above is over \( k k_1 \equiv 1 \mod{p^m} \) consecutive powers of \( g \), which implies
\[
u_k(g) \cdot u_{k_1}(g^k) =
\begin{cases}
	1, & \text{if } k_1 \text{ is odd}, \\
	1 + \tilde{g}, & \text{if } k_1 \text{ is even}.
\end{cases}
\]
Reconsidering $G$ as in (\ref{equation 3}) and its idempotents $e:=e_{p_1^{j_1}, k_1}$ as  given in (\cite{CM25}, Theorem 3.3), we have that, if $u \in \mathbb{F}_q G e$ is a unit, then 
$ue\widehat{\langle b^\beta \rangle}u^{-1}
$ yields non-central idempotents since conjugation is an automorphism. It has been observed that suitable choices of units can improve the parameters of such codes \cite{APM19}. Futher, 
$e + s\,\widehat{\langle b \rangle}a^k (1 - \widehat{\langle b \rangle})e$ is a unit with inverse
$e - s\,\widehat{\langle b \rangle}\,a^k (1 - \widehat{\langle b \rangle})\, e$
where,  $s \in \mathbb{F}_q \setminus \{0\}$ and  $1 \leq k \leq o(a) - 1$, $o(a)$ being the order of $a$.\\
For any unit $u$ defined above,
we denote $u^{-1}e\widehat{\langle b^\beta \rangle}  u$ by $ e^{\beta u}$ and $e \widehat{\langle b^\beta \rangle} $ by $ e^{\beta}$. In the ongoing notation we have the following corollary to \Cref{thm:noncentral-code}.
\begin{corollary}The dimension and distance of $\mathbb{F}_qGe^{\beta u}$ satisfy the following:
	\begin{enumerate}
	\item $\dim_{\mathbb{F}_q}(\mathbb{F}_qG e^{\beta u})=o_{p_1^{j_1}}(q) \cdot p_2^{\lambda + \lambda_0}
	$
	\item  
	$d(\mathbb{F}_qGe^{\beta }) \leq d(\mathbb{F}_qGe^{\beta u}).$
	
\end{enumerate}   	
\end{corollary}
%\begin{corollary} Let	$\mathbb{F}_q$ field containing $q$ elements, $G$-split metacyclic group of order $p_1^mp_2^l$,  with presentation
%	$	G = \langle a, b \mid a^{p_1^m} = e, b^{p_2^l} = e, \, a^b = a^{r} \rangle
%	$, where $o_{p_1^m}(q)=p_2^l.$ 
%	\label{Cor_distance_non-central codes}In the foregoing notation, 
%	\begin{enumerate}
%		\item $\dim(\mathbb{F}_qG e_{p_1^{j_1}, k_1}^{\beta u}=o_{p_1^{j_1}}(q) \cdot p_2^{\lambda + \lambda_0}
%		$
%		\item $d:=d(\mathbb{F}_qG e_{p_1^{j_1}, k_1}^{\beta u}$,  
%		$d(\mathbb{F}_qGe_{p_1^{j_1}, k_1}^{\beta }) \leq d(\mathbb{F}_qGe_{p_1^{j_1}, k_1}^{\beta u}).$
%		
%	\end{enumerate}   	
%\end{corollary}
\begin{proof}The dimension follows from \Cref{thm:noncentral-code}. Since conjugation is an automorphism the dimensions of code generated by the non-central idempotents constructed by units are the same as code generated by idempotents  $ e^{\beta}$. 
	Also if we consider $u=e+ s\widehat{\langle b^\beta \rangle}a^k(1-\widehat{\langle b^\beta \rangle})e$, then clearly
	$u^{-1}eu=(1-a(1-\widehat{ \langle b^\beta \rangle}))e\widehat{\langle b^\beta \rangle}.$
	Now if we consider $u=u_{k, m}(a)$ or $u=u_{k}(a)$ then  $u_{k, m}(a)e\widehat{\langle b^\beta \rangle}=e\widehat{\langle b^\beta \rangle}u_{k, m}(a^r),\; r \in \mathbb{Z}$, which implies idempotent $ e^{\beta u}$  contains $ e^{\beta}$ as a factor.  So, $d(\mathbb{F}_qGe^{\beta}) \leq d( e^{\beta u})$.
\end{proof}
Now we will give some examples where code generated by above methods  indeed improve the distance parameter.
\begin{example}\label{example_dihedral_central}\underline{$\mathbb{F}_3D_{14}$ and $\mathbb{F}_5D_{14}$, $D_{14}= \langle a, b ~|~ a^{7}=b^2=1, a^b=a^{-1} \rangle.$}\\
		We see that the pcis of $\mathbb{F}_3D_{14}$ are $e_1:=\widehat{D_{14}}$, $e_2:=\widehat{\langle a \rangle}-\widehat{D_{14}}$ and $e_{7,1}:=1-\widehat{\langle a \rangle}$. 
	
	If we consider  $\mathbb{F}_3D_{14}$, then the code generated by the idempotent $\widehat{\langle b \rangle}e_{7,1}$ is a $[10,6,4]$. However, by considering the idempotent $e_{7,1}(\widehat{\langle b \rangle}+\widehat{\langle b \rangle}a(1-\widehat{\langle b \rangle}))$, we get a $[14, 6, 6]$ code, which is best known and also  not equivalent to abelian code.\\
	Now if we consider  $\mathbb{F}_5D_{14}$, then code generated by  the idempotent $e_{7,1}(\widehat{\langle b \rangle}+\widehat{\langle b \rangle}a(1-\widehat{\langle b \rangle}))$, we get a $[14, 6, 7]$ code, which is best known and also  not equivalent to abelian code.\\ 
\end{example}
\begin{example}\label{example_order39}\underline{$\mathbb{F}_2G$ and $\mathbb{F}_5G$, $G$-the non-abelian metacyclic group of order $39$}, \\
	
	$G:= \langle a, b ~|~ a^{13}=b^3=1, a^b=a^{9} \rangle$.\\
	
	The code generated by the pci of $\mathbb{F}_2G$ corresponding to the strong Shoda pair $(\langle a \rangle,  \langle 1\rangle) $, namely $e_{13,1}=e_C(G, \langle a \rangle,  \langle 1\rangle)$ yields a $[39, 36, 2]$, which is a best known code.\\
	Now the non-central idempotent, namely $e_{13,1}\widehat{\langle b \rangle}$, gives a $[39, 12, 6]$ code.
	However, by following the technique given above and considering the idempotent $e_{13,1}(\widehat{\langle b \rangle}+\widehat{\langle b \rangle}a(1-\widehat{\langle b \rangle}))$, we get a $[39, 12, 10]$ code, which is not equivalent to abelian code. Now if we consider $u=1+a+a^2$ as proved above, $ue_{13, 1} $ is a unit in $\mathbb{F}_2Ge_{13, 1}$. So $e_{13, 1}^{1u}$  gives a code with parameters $[39, 12, 12]$.  \\
Similarly, if we consider the code generated by the non-central idempotent 
$e_{13,1}\widehat{\langle b \rangle}$ of $\mathbb{F}_5G$, we obtain a $[39,12,6]$-code. 
However, when we take $e_{13,1}(\widehat{\langle b \rangle} + \widehat{\langle b \rangle}a(1 - \widehat{\langle b \rangle}))$, 
the resulting code has parameters $[39,12,17]$, which is very close to the 
best-known $[39,12,18]$-code. Thus, the distance parameter increases 
significantly.

\end{example}


\begin{example}\label{example_order57}
	\underline{$\mathbb{F}_2G$, where $G$ is the non-abelian metacyclic group of order $57$},\\
	\[
	G := \langle a, b \mid a^{19} = b^3 = 1,~ a^b = a^{7} \rangle.
	\]
	The non-central idempotent $e_{19,1}\widehat{\langle b \rangle}$ gives a $[57, 18, 6]$ code. However, by following the technique described above and considering the idempotent
$
	e_{19,1}(\widehat{\langle b \rangle} + \widehat{\langle b \rangle}a(1 - \widehat{\langle b \rangle})),
$
	we obtain a $[57, 18, 14]$ code, which is not equivalent to any abelian code.
	
	Now, if we consider the unit $u = 1 + a + a^2$, then $u e_{19,1}$ is a unit in $\mathbb{F}_2 G e_{19,1}$. So the conjugated idempotent $e_{19,1}^{1u}$ gives a code with parameters $[57, 18, 16]$. Again, this code is inequivalent to any abelian code and is close to the best-known code with parameters $[57, 18, 17]$. It is apparent that the distance parameter increases. 
\end{example}
\begin{example}\label{example_order20}
	\underline{$\mathbb{F}_3G$, where $G$ is the non-abelian metacyclic group of order $20$},\\
	\[
	G := \langle a, b \mid a^{5} = b^4 = 1,~ a^b = a^{2} \rangle.
	\]
	The non-central idempotent $e_{5,1}\widehat{\langle b \rangle}$ generates a code with parameters $[20,4,8]$.  
	Now consider the unit $u = 1 + a $. Since $u e_{5,1}$ is a unit in $\mathbb{F}_3 G e_{5,1}$, the conjugated idempotent $e_{5,1}^{1u}$ yields a code with parameters $[20,4,12]$. 
	This code coincides with a best-known code. Clearly, adjoining the unit improves the minimum distance from $8$ to $12$ while keeping the length and dimension fixed.
\end{example}
%Let $e$ denote a central idempotent corresponding to a strong Shoda pair $(H,K)$, 
%and let $u$ be any unit in the simple component $\mathbb{F}_q G e$. 
%Then the element
%$u^{-1} e \, \widehat{b} u$ is again an idempotent, and is non-central since conjugation by $u$ is an automorphism of $\mathbb{F}_q Ge$. 
%We denote this new idempotent by 
%$e^{u} := u^{-1} e \, \widehat{b} u$.
%
%	\subsection{Non-central codes using units}
%Let $u \in \mathbb{F}_q G e$ be a unit. Then, as observed earlier, conjugating the idempotent $e\,\widehat{b}$ by $u$ produces a non-central idempotent
%$e^u := u^{-1} e\,\widehat{b}\,u,$
%which generates a non-central code. Suitable choices of units can improve the parameters of such codes~\cite{APM19}.  Two important families of units in $\mathbb{Z}G$—Bass and bicyclic units—remain units in $\mathbb{F}_q G$ for any group $G$.
%\medskip
%
%\noindent
%\textbf{Bicyclic units.} For $g,h \in G$, with $\hat{h}=1+h+\cdots+h^{|h|-1}$, define
%\[
%b(g, \hat{h}) = 1 + (1 - h)g\hat{h}, 
%\quad 
%b(\hat{h}, g) = 1 + \hat{h}g(1 - h).
%\]
%These are units with inverses
%$b(g, \hat{h})^{-1} = b(-g, \hat{h})$ and $b(\hat{h}, g)^{-1} = b(\hat{h}, -g)$.
%
%\medskip
%\noindent
%\textbf{Bass units.} If $x \in \mathbb{F}_q G$ is a unit of order $n$, let $k$ and are relatively coprime $ n$ and $m>0$ such that $k^m \equiv 1 \pmod{n}$. Then
%\[
%u_{k,m}(x) = \Big(1 + x + \cdots + x^{k-1}\Big)^m + \frac{1-k^m}{n}\,\hat{x}
%\]
%is invertible, with inverse $u_{l,m}(x^k)$ where $kl \equiv 1 \pmod{n}$.\\
%
%
%\medskip
%\noindent
%Analogously, we consider two special units of $\mathbb{F}_q G$ to obtain non-central codes not equivalent to abelian codes. \\
%First one is alternating units in $\mathbb{F}_q G$ where $q = 2^k$.
%Let $g \in G$ be an element of order  $p^m$ where $p$ is an odd prime. Let $k$ be an odd number such that $k <p$ and let $k_1 \in \mathbb{Z}_{>0}$ satisfy
%\[
%k k_1 \equiv 1 \pmod{p^m}.
%\]
%Define $u_k(g) := 1 + g + g^2 + \cdots + g^{k-1}$\\
%Then \( u_k(g) \) is a unit in \( \mathbb{F}_q G \), and its inverse is given by
%\[
%u_k(g)^{-1} =
%\begin{cases}
%	u_{k_1}(g^k), & \text{if } k_1 \text{ is odd}, \\
%	u_{k_1}(g^k) + \widehat{g}, & \text{if } k_1 \text{ is even}.
%\end{cases}
%\]
%Because,
%$u_k(g) \cdot u_{k_1}(g^k) = \left( \sum_{i=0}^{k-1} g^i \right) \left( \sum_{j=0}^{k_1 - 1} g^{kj} \right) = \sum_{t=0}^{k k_1 - 1} g^t.$
%Since \( k k_1 \equiv 1 \mod{p^m} \) and \( g \) has order \( p^m \). Thus, the sum above is over \( k k_1 \equiv 1 \mod{p^m} \) consecutive powers of \( g \), which implies
%\[
%u_k(g) \cdot u_{k_1}(g^k) =
%\begin{cases}
%	1, & \text{if } k_1 \text{ is odd}, \\
%	1 + \widehat{g}, & \text{if } k_1 \text{ is even}.
%\end{cases}
%\]
%Similar to bicyclic units, the elements
%$e + s\,\widehat{b}a^k (1 - \widehat{b})e$
%are units inside the ideal $(\mathbb{F}_q G) e$, with inverses
%$e - s\,\widehat{b}\,a^k (1 - \widehat{b})\, e$
%where $1 \leq k \leq o(a) - 1$ and $s \in \mathbb{F}_q \setminus \{0\}$, where $o$ is order of $a$.
%Based on such units, we can now construct non-central idempotents.  
%If $u$ is a unit in $(\mathbb{F}_q G) e$, then
%\[
%u^{-1} e\,\widehat{b}\, u
%\]
%is a non-central idempotent, since conjugation by $u$ is an automorphism.
%\subsection{Construction of Non-Central Idempotents}
%Let $G$ be metacyclic group 
%\begin{theorem}\label{thm:noncentral-code}
%	Let $C=\mathbb{F}_q G e_{p_1^{j_1}, k_1}\widehat{b^\beta}$ be a non-central code.  
%	If $\dim$ and $d$ respectively denote its dimension and minimum distance, then
%	\[
%	\dim = o_{p_1^{j_1}}(q)\, p_2^{\lambda+\lambda_0},
%	\]
%	and
%	\[
%	d=
%	\begin{cases}
%		2 p_1^{m-j_1}\, p_2^{\,l-\lambda} \;\leq d \leq\; p_1^m \, p_2^{\,l-\lambda}, & 1 \leq j_1 \leq i_0^{(1)}, \\[4pt]
%		2 p_1^{m-j_1} p_2^{\,l-\lambda} \;\leq d \leq\; p_1^{\,m-j_1+i_0^{(1)}} \, p_2^{\,l-\lambda}, & \text{otherwise},
%	\end{cases}
%	\]
%	where $\gcd(\beta, p_2^l)=p_2^\lambda$ and $\gcd(\omega_0, p_2^l)=p_2^{\lambda_0}$.
%\end{theorem}
%
%\begin{proof}
%	Let $f := e_{p_1^{j_1}, k_1}\widehat{b^\beta}$ and $o := o_{p_1^{j_1}}(q)$.  
%	
%	\noindent\textbf{Case 1: $r \in \langle q \rangle$.}  
%	We claim that
%	\[
%	\mathcal{B} := \{\, a^{\eta_a} b^{\eta_b} f \;\mid\; 0 \leq \eta_a < o,\; 0 \leq \eta_b < p_2^\lambda \,\}
%	\]
%	is a basis for the left ideal $\mathbb{F}_q G f$.  
%	
%	Indeed, for any $\alpha \in \mathbb{F}_q G f$, write $\alpha = \alpha' \widehat{b^\beta}$ with  
%	\[
%	\alpha' = \sum_{\eta_b=0}^{p_2^l-1} \sum_{\eta_a=0}^{o-1} 
%	\alpha_{\eta_a\eta_b}\, a^{\eta_a} b^{\eta_b} e_{p_1^{j_1}, k_1}.
%	\]
%	Partitioning $\eta_b$ by a transversal $T$ of $\langle b^\beta \rangle$ in $\langle b\rangle$ shows that $\mathcal{B}$ spans $\mathbb{F}_q G f$.  
%	Linear independence follows from the independence of $\{a^{\eta_a} e_{p_1^{j_1},k_1}\}_{0\leq \eta_a<o}$.  
%	Thus $|\mathcal{B}| = o\, p_2^\lambda$, giving $\dim = o\, p_2^\lambda$.  
%	
%	\noindent\textbf{Case 2: $r \notin \langle q \rangle$.}  
%	By the same argument as in \cite[Thm.~3.5]{CM25}, the basis is
%	\[
%	\mathcal{B}' = \{\, a^{\eta_a} b^{\eta_b} \, \epsilon_C(\langle a \rangle,\langle a^{p_1^{j_1}}\rangle)^t \widehat{b^\beta} 
%	\;\mid\; 0\leq \eta_a<o,\; 0\leq \eta_b < p_2^\lambda,\; t\in \tau \,\},
%	\]
%	where $\tau$ is a transversal of $\langle a, b^{\omega_0}\rangle$ in $G$.  
%	Hence $\dim = o\, p_2^{\lambda+\lambda_0}$.
%	
%
%	Let $K=\langle a^{p_1^{j_1}} \rangle$, $N=\langle b^\beta \rangle$, so $NK \leq G$.  
%	By the method of \Cref{parmameters $G=H$}, the code parameters satisfy the stated bounds.
%\end{proof}
%	\textbf{Construction of non-central idempotents.}  
%If $u$ is any unit in $\mathbb{F}_q G e_{p_1^{j_1}, k_1}$, then
%$u^{-1} e_{p_1^{j_1}, k_1} \widehat{b^\beta} u
%$
%is a non-central idempotent, since conjugation is an automorphism. Let us denote $u^{-1}e_{p_1^{j_1}, k_1}  \widehat{b^\beta}  u$ by $ e_{p_1^{j_1}, k_1}^{\beta u}$, where $u$ is unit of any type as defined above and $e_{p_1^{j_1}, k_1}  \widehat{b^\beta} $ by $ e_{p_1^{j_1}, k_1}^{\beta}$.
%
%\begin{corollary} Let	$\mathbb{F}_q$ field containing $q$ elements, $G$-split metacyclic group of order $p_1^mp_2^l$,  with presentation
%	$	G = \langle a, b \mid a^{p_1^m} = e, b^{p_2^l} = e, \, a^b = a^{r} \rangle
%	$, where $o_{p_1^m}(q)=p_2^l.$ 
%	\label{Cor_distance_non-central codes}In the foregoing notation, 
%	\begin{enumerate}
%		\item $\dim(\mathbb{F}_qG e_{p_1^{j_1}, k_1}^{\beta u})=o_{p_1^{j_1}}(q) \cdot p_2^{\lambda + \lambda_0}
%		$
%		\item $d:=d(\mathbb{F}_qG e_{p_1^{j_1}, k_1}^{\beta u}$,  
%		$d(\mathbb{F}_qGe_{p_1^{j_1}, k_1}^{\beta }) \leq d(\mathbb{F}_qGe_{p_1^{j_1}, k_1}^{\beta u}).$
%		
%	\end{enumerate}   	
%\end{corollary}
%\begin{proof}
%	Since conjugation is an automorphism the dimensions of code generated by the non-central idempotents constructed by units are the same as code generated by idempotents  $ e_{p_1^{j_1}, k_1}^{\beta}$.
%	Also if we consider $u=e_{p_1^{j_1}, k_1} + s\widehat{b^\beta}a^k(1-\widehat{b^\beta})e_{p_1^{j_1}, k_1}$, then clearly
%	$u^{-1}e_{p_1^{j_1}, k_1}u=(1-a(1-\widehat{ b^\beta}))e_{p_1^{j_1}, k_1}\widehat{b^\beta}.$
%	Now if we consider $u=u_{k, m}(a)$ or $u=u_{k}(a)$ then  $u_{k, m}(a)e_{p_1^{j_1}, k_1}\widehat{b^\beta}=e_{p_1^{j_1}, k_1}\widehat{b^\beta}u_{k, m}(a^r)$, this implies idempotent $ e_{p_1^{j_1}, k_1}^{\beta u}$  contains $ e_{p_1^{j_1}, k_1}^{\beta}$ as a factor.  So, $d(\mathbb{F}_qGe_{p_1^{j_1}, k_1}^{\beta}) \leq d( e_{p_1^{j_1}, k_1}^{\beta u}))$.
%\end{proof}
%
%
%%\subsection{Construction of non-central idempotents.}  
%%\begin{theorem} Let	$\mathbb{F}_q$ field containing $q$ elements, $G$-split metacyclic group of order $p_1^mp_2^l$,  with presentation
%%	$	G = \langle a, b \mid a^{p_1^m} = e, b^{p_2^l} = e, \, a^b = a^{r} \rangle
%%	$, where $o_{p_1^m}(q)=p_2^l.$ 
%%	
%%	\label{Cor_distance_non-central codes}In the foregoing notation, 
%%	\begin{enumerate}
%%		\item $\dim_{\mathbb{F}_q}(\mathbb{F}_qG e_{p_1^{j_1}, k_1}^{\beta u})=o_{p_1^{j_1}}(q) \cdot p_2^{\lambda + \lambda_0}
%%		$
%%		\item $d:=d(\mathbb{F}_qG e_{p_1^{j_1}, k_1}^{\beta u})$,  
%%		$d(\mathbb{F}_qGe_{p_1^{j_1}, k_1}^{\beta }) \leq d(\mathbb{F}_qGe_{p_1^{j_1}, k_1}^{\beta u}).$
%%		
%%	\end{enumerate}   	
%%\end{theorem}
%%\begin{proof}
%%	Since conjugation is an automorphism, the codes generated by the non-central idempotents constructed from units have the same dimension as those generated by $e_{p_1^{j_1}, k_1}^{\beta}$, so the dimension follows from \Cref{lemma:noncentral-code}.  
%%	For $u=e_{p_1^{j_1}, k_1}+s\widehat{b^\beta}a^k(1-\widehat{b^\beta})e_{p_1^{j_1}, k_1}$, we have 
%%	\[
%%	u^{-1}e_{p_1^{j_1}, k_1}u = (1-a(1-\widehat{b^\beta}))\,e_{p_1^{j_1}, k_1}\widehat{b^\beta}.
%%	\]
%%	Moreover, if $u=u_{k,m}(a)$ or $u=u_{k}(a)$, then 
%%	\[
%%	u_{k,m}(a)e_{p_1^{j_1}, k_1}\widehat{b^\beta} 
%%	= e_{p_1^{j_1}, k_1}\widehat{b^\beta}u_{k,m}(a^r),
%%	\] 
%%	so the idempotent $e_{p_1^{j_1}, k_1}^{\beta u}$ contains $e_{p_1^{j_1}, k_1}^{\beta}$ as a factor.  
%%	Hence,
%%	\[
%%	d\bigl(\mathbb{F}_qGe_{p_1^{j_1}, k_1}^{\beta}\bigr) 
%%	\leq d\bigl(e_{p_1^{j_1}, k_1}^{\beta u}\bigr).
%%	\]
%%\end{proof}
%%	\begin{lemma}\label{lemma:noncentral-code}
%%				Let $C=\mathbb{F}_q G e_{p_1^{j_1}, k_1}\widehat{b^\beta}$ be a non-central code.  
%%				If $\dim$ and $d$ respectively denote its dimension and minimum distance, then
%%				\[
%%				\dim = o_{p_1^{j_1}}(q)\, p_2^{\lambda+\lambda_0},
%%				\]
%%				and
%%				\[
%%				d=
%%				\begin{cases}
%%					2 p_1^{m-j_1}\, p_2^{\,l-\lambda} \;\leq d \leq\; p_1^m \, p_2^{\,l-\lambda}, & 1 \leq j_1 \leq i_0^{(1)}, \\[4pt]
%%					2 p_1^{m-j_1} p_2^{\,l-\lambda} \;\leq d \leq\; p_1^{\,m-j_1+i_0^{(1)}} \, p_2^{\,l-\lambda}, & \text{otherwise},
%%				\end{cases}
%%				\]
%%				where $\gcd(\beta, p_2^l)=p_2^\lambda$ and $\gcd(\omega_0, p_2^l)=p_2^{\lambda_0}$.
%%			\end{lemma}
%%			
%%			\begin{proof}
%%				Let $f := e_{p_1^{j_1}, k_1}\widehat{b^\beta}$ and $o := o_{p_1^{j_1}}(q)$.  
%%				
%%				\noindent Case 1: $r \in \langle q \rangle$. 
%%				We claim that
%%				\[
%%				\mathcal{B} := \{\, a^{\eta_a} b^{\eta_b} f \;\mid\; 0 \leq \eta_a < o,\; 0 \leq \eta_b < p_2^\lambda \,\}
%%				\]
%%				is a basis for the left ideal $\mathbb{F}_q G f$.  
%%				
%%				Indeed, for any $\alpha \in \mathbb{F}_q G f$, write $\alpha = \alpha' \widehat{b^\beta}$ with  
%%				\[
%%				\alpha' = \sum_{\eta_b=0}^{p_2^l-1} \sum_{\eta_a=0}^{o-1} 
%%				\alpha_{\eta_a\eta_b}\, a^{\eta_a} b^{\eta_b} e_{p_1^{j_1}, k_1}.
%%				\]
%%				Partitioning $\eta_b$ by a transversal $T$ of $\langle b^\beta \rangle$ in $\langle b\rangle$ shows that $\mathcal{B}$ spans $\mathbb{F}_q G f$.  
%%				Linear independence follows from the independence of $\{a^{\eta_a} e_{p_1^{j_1},k_1}\}_{0\leq \eta_a<o}$.  
%%				Thus $|\mathcal{B}| = o\, p_2^\lambda$, giving $\dim = o\, p_2^\lambda$.  
%%				
%%				\noindent Case 2: $r \notin \langle q \rangle$. 
%%				By the same argument as in \cite[Thm.~3.5]{CM25}, the basis is
%%				\[
%%				\mathcal{B}' = \{\, a^{\eta_a} b^{\eta_b} \, \epsilon_C(\langle a \rangle,\langle a^{p_1^{j_1}}\rangle)^t \widehat{b^\beta} 
%%				\;\mid\; 0\leq \eta_a<o,\; 0\leq \eta_b < p_2^\lambda,\; t\in \tau \,\},
%%				\]
%%				where $\tau$ is a transversal of $\langle a, b^{\omega_0}\rangle$ in $G$.  
%%				Hence $\dim = o\, p_2^{\lambda+\lambda_0}$.
%%				
%%			The distance bound follows from the observation that 
%%			let $K = \langle a^{p_1^{j_1}} \rangle$ and $N = \langle b^\beta \rangle$, so that $NK \leq G$. 
%%			By applying the method described in \Cref{parmameters $G=H$}, the parameters of the code satisfy the stated bounds.
%%			
%%			\end{proof}
%%
%%		
%		
%				
%			
%		
%			Now we will give some examples where code generated with the help of above units, actually increses the distance parameter.
%			\begin{example}\label{example_order39}\underline{$\mathbb{F}_2G$ and $\mathbb{F}_5G$, $G$-the non-abelian metacyclic group of order $39$}, \\
%				
%				$G:= \langle a, b ~|~ a^{13}=b^3=1, a^b=a^{9} \rangle$.\\
%				
%				The code generated by the pci of $\mathbb{F}_2G$ corresponding to the strong Shoda pair $(\langle a \rangle,  \langle 1\rangle) $, namely $e_{13,1}=e_C(G, \langle a \rangle,  \langle 1\rangle)$ yields a $[39, 36, 2]$, which is a best known code.\\
%				Now the non-central idempotent, namely $e_{13,1}\widehat{b}$, gives a $[39, 12, 6]$ code.
%				However, by following the technique given above and considering the idempotent $e_{13,1}(\widehat{b}+\widehat{b}a(1-\widehat{b}))$, we get a $[39, 12, 10]$ code, which is not equivalent to abelian code. Now if we consider $u=1+a+a^2$ as proved above, $ue_{13, 1} $ is a unit in $\mathbb{F}_2Ge_{13, 1}$. So $e_{13, 1}^{u}$  gives a code with parameters $[39, 12, 12]$.  \\
%				Also if we consider the code generated by the non-central idempotent  $e_{13,1}(\widehat{b}+\widehat{b}a(1-\widehat{b}))$ of $\mathbb{F}_5G$, we get a $[39, 12, 17]$, which is a best known code. 
%				Clearly, distance parmeter increses.
%				Also if we consider  $e_{13, 1}^{u}+I,$  where $I$ denotes the identity element of $\mathbb{F}_5G$ yields a $[39, 27, 5]$, which is inequivalent to abelian code and is closer to the best-known code with parameters $[39, 27, 6]$. \\
%			\end{example}
%			
%			\begin{example}\label{example_dihedral_central}\underline{$\mathbb{F}_3D_{14}$ and $\mathbb{F}_5D_{14}$}\\
%				
%				$D_{14}= \langle a, b ~|~ a^{7}=b^2=1, a^b=a^{-1} \rangle$.\\
%				
%				We see that the primitive central idempotents of $\mathbb{F}_3D_{14}$ are $e_1:=\widehat{D_{14}}$, $e_2:=\widehat{\langle a \rangle}-\widehat{D_{14}}$ and $e_{7,1}:=1-\widehat{\langle a \rangle}$. $\mathbb{F}_3D_{10}e_{7,1}$ 
%				
%				If we consider  $\mathbb{F}_3D_{14}$, then the code generated by the idempotent $\hat{b}e_{7,1}$ is a $[10,6,4]$. However, by considering the idempotent $e_{7,1}(\widehat{b}+\widehat{b}a(1-\widehat{b}))$, we get a $[14, 6, 6]$ code, which is best known and also  not equivalent to abelian code.\\
%				Now if we consider  $\mathbb{F}_5D_{14}$, then code generated by  the idempotent $e_{7,1}(\widehat{b}+\widehat{b}a(1-\widehat{b}))$, we get a $[14, 6, 7]$ code, which is best known and also  not equivalent to abelian code.\\ 
%			\end{example}
%			\begin{example}\label{example_order57}
%				\underline{$\mathbb{F}_2G$, where $G$ is the non-abelian metacyclic group of order $57$},\\
%				\[
%				G := \langle a, b \mid a^{19} = b^3 = 1,~ a^b = a^{7} \rangle.
%				\]
%				The non-central idempotent $e_{19,1}\widehat{b}$ gives a $[57, 18, 6]$ code.
%				
%				However, by following the technique described above and considering the idempotent
%				\[
%				e_{19,1}(\widehat{b} + \widehat{b}a(1 - \widehat{b})),
%				\]
%				we obtain a $[57, 18, 14]$ code, which is not equivalent to any abelian code.
%				
%				Now, if we consider the unit $u = 1 + a + a^2$, as proved above, then $u e_{19,1}$ is a unit in $\mathbb{F}_2 G e_{19,1}$. So the conjugated idempotent $e_{19,1}\widehat{b}^u$ gives a code with parameters $[57, 18, 16]$. Again, this code is inequivalent to any abelian code and is closer to the best-known code with parameters $[57, 18, 17]$.
%				
%				Clearly, the distance parameter increases. 
%			\end{example}
%			\begin{example}\label{example_order20}
%				\underline{$\mathbb{F}_3G$, where $G$ is the non-abelian metacyclic group of order $20$},\\
%				\[
%				G := \langle a, b \mid a^{5} = b^4 = 1,~ a^b = a^{2} \rangle.
%				\]
%				The non-central idempotent $e_{5,1}\widehat{b}$ generates a code with parameters $[20,4,8]$.  
%				Now consider the unit $u = 1 + a + a^2$. Since $u e_{5,1}$ is a unit in $\mathbb{F}_3 G e_{5,1}$, the conjugated idempotent $e_{5,1}\widehat{b}^u$ yields a code with parameters $[20,4,12]$. 
%				This code coincides with a best-known code. Clearly, adjoining the unit improves the minimum distance from $8$ to $12$ while keeping the length and dimension fixed.
%			\end{example}
%			\begin{example}\label{example_order81}
%				\underline{$\mathbb{F}_2 G$, where $G$ is a non-abelian metacyclic group of order $81$}.\\
%				\[
%				G := \langle a, b \mid a^{27} = b^3 = 1,\; a^b = a^{10} \rangle.
%				\]
%				The non-central idempotent $e\widehat{b}$ generates a code with parameters $[81, 20, 6]$, 
%				where 
%				\[
%				e := e_C(G, G, \langle a^3, b \rangle) + e_C(G, \langle a \rangle).
%				\] 
%				If we instead consider 
%				\[
%				e\bigl(\widehat{b} + \widehat{b}a(1 - \widehat{b})\bigr),
%				\]
%				we obtain a $[81, 20, 12]$ code, showing that the minimum distance increases under conjugation by a unit.  
%				
%				Now let $u = 1 + a + a^3$. Since $ue$ is a unit in $\mathbb{F}_2 G e$, the conjugated idempotent $e\widehat{b}^u$ yields a code with parameters $[81, 20, 24]$, very close to the best-known $[81, 20, 26]$.  
%				
%				Thus, adjoining a suitable unit improves the minimum distance from $6$ to $24$, while preserving the length and dimension.
%			\end{example}
%		
%
%
%
%%\begin{theorem} 
%%	If $dim$ and $d$ respectively denotes the dimension and minimum distance for the non-central code $\mathbb{F}_q Ge_{p_1^{j_1}, k_1} \widehat{ b^\beta }$, then 
%%		\[
%%	dim= o_{p_1^{j_1}}(q) \cdot p_2^{\lambda + \lambda_0}\]
%%	\[ d=
%%	\begin{cases}2 p_1^{m-j_1} \cdot p_2^{l-\lambda} \leq d \leq  p_1^m \cdot p_2^{l-\lambda} ~~\text{for} ~~1 \leq j_1 \leq i_0^{(1)}\\
%	%	2 p_1^{m-j_1} p_2^{l-\lambda} \leq d \leq  p_1^{m-j_1 +i_0^{(1)}} \cdot p_2^{l-\lambda}~~\text{otherwise} 
%	%	\end{cases} 
%%	\] where $\gcd(\beta, p_2^l)=p_2^\lambda ~~\text{and}~~ \gcd(\omega_0, p_2^l)=p_2^{\lambda_0}$.
%%\end{theorem}
%%\begin{proof}
%%	We first consider the case when $r \in \langle q \rangle$.
%%For brevity, let $f := e_{p_1^{j_1}, k_1}\widehat{b^\beta}$ and  $o:= o_{p_1^{j_1}}(q)$. We claim that the set
%%\[
%%\mathcal{B} = \{ f, af, a^2 f, \dots, a^{o-1} f, bf, abf, \dots, a^{o- 1} bf, \dots, b^{p_2^\lambda-1} f, ab^{p_2^\lambda-1}f, \dots, a^{o- 1} b^{p_2^\lambda -1}f \}
%%\]
%%is a basis for the left ideal $\mathbb{F}_q G f$.
%%
%%Let $\alpha \in \mathbb{F}_q G f$. Then $\alpha = \alpha' \widehat{b^\beta}$, where $\alpha' \in \mathbb{F}_q G e_{p_1^{j_1}, k_1}$. From (\cite{CM25}, Theorem 3.5), we can write
%%\[
%%\alpha' = \sum_{\eta_b = 0}^{p_2^l- 1} \sum_{\eta_a = 0}^{o- 1} \alpha_{\eta_a \eta_b} \, a^{\eta_a} b^{\eta_b} \, e_{p_1^{j_1}, k_1}
%%\]
%%and hence
%%\[
%%\alpha = \sum_{\eta_b = 0}^{p_2^l- 1} \sum_{\eta_a = 0}^{o- 1} \alpha_{\eta_a \eta_b} \, a^{\eta_a} b^{\eta_b}e_{p_1^{j_1}, k_1}\widehat{b^\beta} = \sum_{t \in T} \sum_{\eta_a = 0}^{o- 1} \alpha_{\eta_a t} \, a^{\eta_a} b^{t}e_{p_1^{j_1}, k_1}\widehat{b^\beta} ,
%%\] where $T$ is transversal of $\langle b^\beta \rangle$ in $\langle b \rangle$. Thus $\mathcal{B}$ spans $\mathbb{F}_q G f$.\\
%%Now, to prove that $\mathcal{B}$ is linearly independent. Consider
%%\[
%% \sum_{\eta_b = 0}^{p_2^\lambda- 1} \sum_{\eta_a = 0}^{o- 1} \alpha_{\eta_a \eta_b} \, a^{\eta_a} b^{\eta_b} e_{p_1^{j_1}, k_1}\widehat{b^\beta}  =0
%%\]
%%\[
%%\sum_{\eta_b = 0}^{p_2^\lambda- 1}\left( \sum_{\eta_a = 0}^{o- 1} \alpha_{\eta_a \eta_b} \, a^{\eta_a} e_{p_1^{j_1}, k_1}\right) b^{\eta_b} \,\widehat{b^\beta} =0
%%\]
%%Because of different support we have,
%%\[\left( \sum_{\eta_a = 0}^{o- 1} \alpha_{\eta_a \eta_b} \, a^{\eta_a} e_{p_1^{j_1}, k_1}\right) b^{\eta_b} \,\widehat{b^\beta} =0 \text{ for each $\eta_b$}.\] 
%%Now because of linearly independence of the set $\{ e_{p_1^{j_1}, k_1},a e_{p_1^{j_1}, k_1},\ldots, a^{o-1} e_{p_1^{j_1}, k_1}\}$, we have $\alpha_{\eta_a \eta_b}=0~~ \forall ~~\eta_a~~ \text{and}~~ \eta_b.$ Hence, $\mathcal{B}$ is a basis for  left ideal $\mathbb{F}_q G f.$\\
%%Now if $r \notin \langle q \rangle$ then by similar argument as given in the proof of (\cite{CM25}, Theorem 3.5) we can prove that the set $\mathcal{B}':=\{a^{\eta_a}b^{\eta_b}  \epsilon_C(\langle a \rangle,\langle a^{p_1^{j_1}} \rangle)^t\widehat{b^\beta}~|~0\leq \eta_a< o,~0\leq \eta_b < p_2^{\lambda},~t\in \tau \}$ is a basis for left ideal $\mathbb{F}_q G f$, where $\tau$ is a transversal of $\langle a, b^{\omega_0}\rangle$ in $G$, ~$\omega_0$ being the smallest positive integer such that $r^{\omega_0}\in \langle q \rangle$. \\
%% We now derive bounds on the minimum distance of the code generated by the idempotent 
%% $e_{p_1^{j_1}, k_1}\,\widehat{b^\beta}$. 
%% Set $K=\langle a^{p_1^{j_1}}\rangle$ and $N=\langle b^\beta \rangle$, so that $NK$ forms a subgroup of $G$. 
%% Applying the same technique as in \Cref{parmameters $G=H$}, we obtain the desired distance bound.
%%As, $\widehat{NK}e_{p_1^{j_1}, k_1}\widehat{b^\beta}=\widehat{N}\widehat{K}e_{p_1^{j_1}, k_1}\widehat{b^\beta}=\widehat{N}\widehat{b^\beta}e_{p_1^{j_1}, k_1}=e_{p_1^{j_1}, k_1}\widehat{b^\beta}$,  we have that  $\mathbb{F}_qGe_{p_1^{j_1}, k_1}\widehat{b^\beta} \subseteq \mathbb{F	}_qG\widehat{NK}$.  Let $T'$ be a right transversal of $NK$ in $G$. An arbitrary element $\alpha \in \mathbb{F}_qG$ can be written as $\alpha = \sum_{t \in T'}t\alpha_t $ with $\alpha_t \in \mathbb{F}_qNK$ so the elements of $\mathbb{F}_qGe_{p_1^{j_1}, k_1}\widehat{b^\beta}$ are of the form $(\sum_{t \in T'}t\alpha_t' )\widehat{NK}$, $\alpha_t' \in \mathbb{F}_q.$ If only one coefficient $\alpha_t'$ were non-zero, then we would have $\widehat{NK} \in \mathbb{F}_qGe_{p_1^{j_1}, k_1}\widehat{b^\beta}$ which implies $ \mathbb{F}_qG\widehat{NK}\subseteq \mathbb{F}_qGe_{p_1^{j_1}, k_1}\widehat{b^\beta}$ but $\mathbb{F}_qGe_{p_1^{j_1}, k_1}\widehat{b^\beta} \subseteq \mathbb{F}_qG\widehat{NK}.$ This implies $\mathbb{F}_qGe_{p_1^{j_1}, k_1}\widehat{b^\beta}= \mathbb{F}_qG\widehat{NK}$. Consequently, $ \dim \mathbb{F}_qGe_{p_1^{j_1}, k_1}\widehat{b^\beta}= \dim \mathbb{F}_qG\widehat{NK}$ as $\mathbb{F}_q$-vector subspace. Dimension of  $\mathbb{F}_qG\widehat{NK}=|\frac{G}{NK}|=p_1^{j_1}\cdot p_2^{\lambda}$.  Now if $r \in \langle q \rangle$ then by above we have that dimension of $\mathbb{F}_qGe_{p_1^{j_1}, k_1}\widehat{b^\beta}=o \cdot p_2^\lambda $. This implies $p_1^j = o$, which is a contradiction.\\
%%Similarly, when  $r \notin \langle \Bar{q} \rangle$ we will again have a contradiction. This implies that  at least two coefficients should be non-zero. So, $wt((\mathbb{F}_qGe_{p_1^{j_1}, k_1}\widehat{b^\beta})) \geq 2|NK|=2p_1^{m-j_1}\cdot p_2^{l-\lambda}$ in both the cases. Also $wt(e_{p_1^{j_1}, k_1}\widehat{b^\beta})=p_1^{m-j_1+i_0^{(1)}}p_2^{l-\lambda}$ in both the cases for $i_0^{(1)} < j_1 \leq p_1^m$ and $wt(e_{p_1^{j_1}, k_1}\widehat{b^\beta})=p_1^{m}\cdot p_2^{l-\lambda}$ for $1 \leq j_1 \leq i_0^{(1)}$ . So, we get the required bound.
%
%%\end{proof}






		
			
		
			
			\bibliographystyle{amsalpha}
			\bibliography{Codes}
			
			
			
			
		\end{document}