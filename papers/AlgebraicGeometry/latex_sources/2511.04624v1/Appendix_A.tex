Most of the material in this section is well known to experts. 
However, since there is no reference treating this exact setting, we include it here for completeness.
The following statement generalizes \cite{Milne}, Lemma 12.4.


\begin{proposition}\label{prop:group_like}
Let $\Spec(S_0)$ be connected (so that $S_0$ only has the trivial idempotents $0_S$ and $1_S$).
Then the group-like elements of $S_0[D]$ are linearly independent and exactly the elements of $D$.
\end{proposition}



\begin{proof}
    As elements of $D$ are group-like, we only have to take care of the converse direction. Thus let $\alpha = \sum_i a_i d_i \in S_0[D]$ be group like, where $a_i \in S_0$ and $d_i \in D$. For $\alpha \neq 0$ it holds
    \begin{align*}
        \sum_i a_i d_i \otimes d_i \equ \sum_i a_i \Delta(d_i) \equ \Delta(\alpha) \equ \alpha \otimes \alpha \equ \sum_{i, j} a_i a_j d_i \otimes d_j
    \end{align*}
    and the $d_i \otimes d_j$ are linearly independent for all $i, j$, this implies that $a_i a_j = 0$ for $i \neq j$ and $a_i^2 = a_i$. As $S_0$ does not contain any non-trivial idempotent element, it follows that $a_j = 0$ for all $j \neq i$, that is, $\alpha = d_i \in D$ (so that $a_i = 1$ for exactly one $i$).
\end{proof}




Hence, we can interpret $D$ as the character group of the group algebra $S_0[D]$ (if $\Spec(S_0)$ is connected). In general, elements $d \in D$ correspond to characters $\chi^d: G \to \BG_m = \Spec(\CO_{S_0}[t, t^{-1}])$, so that we may view the coaction as
\begin{align*}
    S \to S \otimes_{S_0} S_0[D], \ S_d \ni f \mapsto f \otimes \chi^d, 
\end{align*}
and therefore
\begin{align*}
    S_0[D] \equ \bigoplus_{d\in D} S_0 \chi^d .
\end{align*}
In particular, we can deduce the following equation for homogeneous elements (cf.\  \cite{CRB}, Construction 1.2.2.2):
\begin{align*}%\label{eq:grading_character}
    f \in S_d \ \iff \ f(gx) \equ \chi^d(g) f(x) \ \ \text{ for all } g \in G, x \in  \Spec(S).
\end{align*}

The following result is due to \cite{Water} (Theorems on pages 5, 9, and 14 and § 2.2). 

\begin{proposition}\label{prop:G_functor_represent}
    Let $S$ be a $D$-graded integral domain, where $D$ is a finitely generated abelian group.
    \begin{enumerate}[label=(\roman*)]
        \item The affine group scheme $G = \Spec(S_0[D])$ represents the exact functor
        \begin{align*}
            G_D\colon \mathrm{Alg}_{S_0} \to \mathrm{Grp},\ \ R \mapsto \Hom_{\mathrm{Grp}}(D, R^\times).
        \end{align*}
        \item $G$ is a finite product of copies of 
        \begin{align*}
            \BG_{m,S_0} \deq \Spec(\CO_{S_0}[T, T^{-1}])
        \end{align*}
        and various 
        \begin{align*}
            \mu_{n_i, S_0} \deq \Spec(\CO_{S_0}[T]/(T^{n_i}-1)).
        \end{align*}
        In particular, $G$ is connected if and only if $D$ is torsion-free.
    \end{enumerate}
\end{proposition}

