
\section{Introduction}

\subsection{Aims and presentation}

Let $ku$ be the $p$-localized connective cover of the complex topological $K$-theory spectrum $KU$, and $\ell$ its Adams summand. In this article, we lift the computation by Angelveit, Hill and Lawson in \cite{angeltveit2010topological} of $\THH_*(\ell)$ into a computation of $\THH_*(ku)$. The equation
\begin{equation}
  \THH(A;B) \simeq B\wedge_A\THH(A)
\end{equation}
combined with a Bockstein spectral sequence computing the coefficient $B$ yields a Bockstein spectral sequence computing $\THH_*(A;B)$; the Bockstein spectral sequence computed in \cite{angeltveit2010topological} has the form
\begin{equation}
  \THH_*(\ell;H\Z_{(p)})\otimes P(v_1) \Rightarrow \THH_*(\ell)
\end{equation}
with $P(v_1) = \ell_*$ a polynomial algebra.

The same construction can be used over $ku$, with a Bockstein spectral sequence of the form
\begin{equation}
  \THH_*(ku;H\Z_{(p)})\otimes P(u) \Rightarrow \THH_*(ku).
\end{equation}
However, these two spectral sequence cannot be compared directly using the injection $\ell\rightarrow ku$, since the generator of $\THH_*(\ell;H\Z_{(p)})$ supporting the differentials in the first spectral sequence have image zero in the second spectral sequence.

To lift the computation to $ku$, we need a more subtle technique. We use the relation $u^{p-1} = v_1$, and consider the cofiber of the multiplication by $v_1$ in $ku$, denoted $ku/v_1$. Since we also have $\ell/v_1 \simeq H\Z_{(p)}$, the morphism
\begin{equation}
  \THH_*(\ell;H\Z_{(p)}) \rightarrow \THH_*(ku; ku/v_1)
\end{equation}
is non-trivial. Moreover, multiplication by $v_1$ in $ku$ yield a third Bockstein spectral sequence, of the form
\begin{equation}
  \THH_*(ku; ku/v_1) \otimes P(v_1) \Rightarrow \THH_*(ku)
\end{equation}
which can then be compared \emph{via} the morphism with the Bockstein spectral sequence computing $\THH_*(\ell)$. To compute the Bockstein spectral sequence associated to the multiplication by $u$, we need to explore its relationship with the Bockstein spectral sequence associated to the multiplication by $v_1$. This is done using what we will call a \emph{gathered spectral sequence}. The technique developed for this computation is general and could be used in other computation where some power of a multiplicative element is better understood than the element itself.

Another computation of $\THH_*(ku)$, using different techniques, was carried out in \cite{lee2022integral}.


\subsection{Notations and conventions}

We will use the following notations to describe various algebras:
\begin{itemize}
\item $P(x)$ is a polynomial algebra over a generator $x$,
\item $P_n(x)$ is a truncated polynomial algebra at height $n$, that is the quotient of $P(x)$ by the relation $x^n=0$,
\item  $\Gamma(x)$ is a divided power algebra, which is generated additively by the divided power of $x$, denoted $\gamma_i x$ for any $i\geq 0$, and with the multiplicative relations:
  \begin{equation*}
    \gamma_i x\cdot \gamma_j x = \binom{i+j}{i} \gamma_{i+j} x,
  \end{equation*}
\item $E(x)$ is an exterior algebra, which means $P_2(x)$.
\end{itemize}
The base ring for these algebras will be determined in most case by the context in which they appear. When computing homology with coefficient in $\F_p$ or modulo $p$ homotopy, the base ring will be $\F_p$. When computing homology with coefficients in $\Z$, $\Z_{(p)}$ or $\Z_p$ (the integers, the $p$-localized integers or the $p$-completed integers), it will be $\Z$, $\Z_{(p)}$ or $\Z_p$. When computing $\THH$, it will be the base ring for the coefficient spectrum. If we need to specify the base ring, we will note it in a subscript: $P_\Q(x)$, $E_\Q(x)$, etc.

When writing spectral sequences, we will use tensor products $\otimes$ of these algebras. One of these tensor products will be written $\botimes$, it will separate the algebras generated by classes whose bidegree lies on the $x$-axis -- on the left of $\botimes$ -- and those generated by classes whose bidegree lies on the $y$-axis -- on the right of $\botimes$.

When we write generators in the form $v_0^h\sigma u\mu_N$, we will use the conventions
\begin{equation}
  v_0^0\sigma u \mu_N = \sigma u\mu_N \quad v_0^1\sigma u \mu_N = v_0\sigma u\mu_N \quad \sigma u \mu_0 = \sigma u
\end{equation}
even when $v_0$ is not a proper multiplicative element. The same convention applies with $\sigma u$ replaced by $\sigma v_1$.

\subsection{Aknowlegdements}

The content of this article is part of my PhD, which I am grateful to have done under the supervision of Christian Ausoni at the University Paris 13. I would also like to thank my current institution, the Sino-French Institute of the Renmin University of China, for supporting my work.

%
%
%
%
