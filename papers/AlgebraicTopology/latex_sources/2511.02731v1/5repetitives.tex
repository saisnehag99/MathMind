\section{Coherent Auslander-Reiten diagrams} \label{sec:equiv-repet}

Let $k$ be a field. To every $k$-linear Hom-finite Krull-Schmidt category $\mathsf{A}$ there is associated an important piece of information $\Gamma(\mathsf{A})$ called the Auslander-Reiten quiver. It has vertices the iso-classes of indecomposables in $\mathsf{A}$, and for $M,N \in \mathsf{ind}\, \mathsf{A}$, the number of arrows $[M] \to [N]$ equals $\mathrm{dim}_k\, \mathrm{Irr}(M,N)$, where $\mathrm{Irr} = \mathrm{rad}/\mathrm{rad}^2$ denotes the space of irreducible morphisms (c.f. \cite[ch. IV]{AssSimSko06}). This quiver provides a very detailed (but normally not complete) description of the category $\mathsf{ind}\, \mathsf{A}$. 

Recall that a \emph{translation quiver} $(\Gamma,\tau)$ consists of a locally finite quiver (only a finite number of adjacent arrows to any vertex) $\Gamma$ together with an injective map $\tau: \Gamma_0' \to \Gamma_0$ defined on a subset $\Gamma_0' \subset \Gamma_0$ such that, for any $x \in \Gamma_0'$ and $y \in \Gamma_0$, arrows $y \to x$ are in bijection with arrows $\tau x \to y$. The map $\tau$ is called the \emph{translation} of $\Gamma$, and a \emph{polarization} of $\Gamma$ is any choice of an injective map $\mu: \Gamma_1' \to \Gamma_1$ with $\Gamma_1' = \{y \to x \mid x \in \Gamma_0'\}$ sending $y \to x$ to an arrow $\tau x \to y$. A \emph{morphism of translation quivers} is a morphism of quivers commuting with $\tau$ and $\mu$. Any full subquiver of $\Gamma$ consisting of a vertex $x$, its translate $\tau x$ and their (common) neighbors is called a \emph{mesh} in $\Gamma$. Finally, a translation quiver is called \emph{stable} if $\Gamma_0' = \Gamma_0$.

\begin{fact}[\cite{Hap88}] \label{fact:Happel1}
Let $Q$ be a finite acyclic quiver and $A = kQ$ its path algebra.
\begin{enumerate}
    \item The quiver $\Gamma_Q := \Gamma(\mathsf{D}^b(\mathsf{mod}\, A))$ is a stable translation quiver with translation $\tau$ given by the derived Nakayama functor $-\otimes_A^\mathbb{L}DA$.
    \item The meshes in $\Gamma_Q$ correspond bijectively to the Auslander-Reiten triangles (up to isomorphism) in $\mathsf{D}^b(\mathsf{mod}\, A)$.
\end{enumerate}
\end{fact}

Let $Q$ and $A=kQ$ as above. A key observation in representation theory is that $\mathrm{dim}_k\, \mathrm{Irr}(P(y),P(x))$ counts the number of arrows $x \to y$ in $Q$, where $P(x) = Ae_x$ denotes the indecomposable projective at the vertex $x \in Q$. Hence $Q^\mathrm{op} \cong \Gamma(\mathsf{proj}\, A)$ and there is a natural inclusion $Q^\mathrm{op} \subset \Gamma_Q$ corresponding to $\mathsf{proj}\, A \subset \mathsf{D}^b(\mathsf{mod}\, A)$. 

We borrow from \cite{MiyYek01} the following

\begin{nota}
A connected component of $\Gamma_Q = \Gamma(\mathsf{D}^b(\mathsf{mod}\, A))$ is called \emph{irregular\footnote{A more precise terminology would be to call these \emph{non-regular}, c.f. \cite[sec. VIII.2]{AssSimSko06}.}} if it is isomorphic to the connected component containing $Q^\mathrm{op}$ (i.e. the indecomposable projectives). We denote $\Gamma_Q^\mathrm{irr}$ the disjoint union of all irregular components of $\Gamma_Q$.
\end{nota}

To any quiver $Q$ there is associated the \emph{repetitive quiver} $\ZQ$, which is a stable translation quiver. The set of vertices of $\ZQ$ is $(\ZQ)_0 = \mathbb{Z} \times Q_0$, and for each arrow $\alpha: v \to w$ in $Q$ and $n\in\mathbb{Z}$, there are two arrows $(n,\alpha): (n,v) \to (n,w)$ and $(n,\alpha^*): (n,w) \to (n+1,v)$ in $\ZQ$. The translation of $\ZQ$ is given by $\tau(n,v) =(n-1,v)$, and it comes equipped with a canonical polarization $\mu(n,\alpha) = (n-1,\alpha^*)$. We identify $Q$ with the subquiver $\{0\} \times Q \subset \ZQ$, and so we often write $(0,v) = v$, $(n,v) = \tau^{-n}v$ and $(n,\alpha) = \tau^{-n}\alpha$. Observe that $\tau$ is an automorphism of $\ZQ$, and not only of $(\ZQ)_0$.

We will also need to consider a \emph{bigger repetitive quiver} $\mathbb{Z} \times \ZQ := \coprod_{i \in \mathbb{Z}} \mathbb{Z}Q$, i.e. consisting of countable copies $\{i\} \times \ZQ$ of $\ZQ$ with $i \in \mathbb{Z}$. It comes equipped with an automorphism $\sigma$ that acts only on the first factor by $\sigma(i) = i+1$. The quiver $\mathbb{Z} \times \ZQ$ is a stable translation quiver with translation $\tau$ and polarization $\mu$ that extend those of $\ZQ \cong \{0\}\times \ZQ$ and commute with $\sigma$.

\begin{fact}[\cite{Hap88}] \label{fact:Happel2}
Let $Q$ be a finite acyclic quiver and $A = kQ$ its path algebra.
\begin{enumerate}
    \item If $A$ has finite-representation type (i.e. $Q$ is Dynkin), then there is a canonical isomorphism of translation quivers $\Gamma_Q^\mathrm{irr} \cong \ZQ^\mathrm{op}$ which is the identity on $Q$. Moreover, in this case $\Gamma_Q^\mathrm{irr} = \Gamma(\mathsf{D}^b(\mathsf{mod}\, A))$.
    \item If $A$ has infinite-representation type (i.e. $Q$ is non-Dynkin), then there is a canonical isomorphism of translation quivers $\Gamma_Q^\mathrm{irr} \cong \mathbb{Z}\times\ZQ^\mathrm{op}$ which is the identity on $Q$ and sends the suspension $\Sigma$ to $\sigma$.
\end{enumerate}
\end{fact}

\subsection{Representations of the Auslander-Reiten quiver} \label{subsec:ARquiver} Let $Q$ be a finite acyclic quiver and $A=kQ$ its path algebra. We explain here an intrinsic way to manipulate the Auslander-Reiten quiver $\Gamma_Q=\Gamma(\mathsf{D}^b(\mathsf{mod}\, A))$. 

From now on we use derived $\infty$-categories, so we can interpret $\Dd{b}{kQ}$ as a stable $\infty$-category of representations $\Dd{b}{k}^Q$. Moreover, since $Q \cong \Gamma(\mathsf{proj}\, A)^\mathrm{op}$, then $\Dd{b}{kQ}$ consists of contravariant representations of the indecomposable projectives. Under this identification, the inclusion $Q \subset \Gamma_{Q^\mathrm{op}}$ induces a restriction functor 
$$\mathrm{res}: \Dd{b}{k}^{\Gamma_{Q^\mathrm{op}}} \xrightarrow{\ \ \quad} \Dd{b}{k}^Q \simeq \Dd{b}{kQ}$$
which evaluates a representation $Y: \Gamma_{Q^\mathrm{op}} \to \Dd{b}{k}$ at the indecomposable projectives, i.e. $(\mathrm{res}\, Y)_v = Y_{P(v)}$. Conversely, there is a way to recover the values of a representation $X: Q \to \Dd{b}{k}$ in terms of the $P(v)$'s. This is by means of the Yoneda-type equivalence $\mathbb{R}\iHom(P(v),X) \simeq X_{v}$, where $\mathbb{R}\iHom(-,X): \Dd{b}{kQ} \to \Dd{b}{k}$ is the derived functor of the internal hom-complex. There is a functor
\begin{align*}
\Dd{b}{k}^Q \simeq \Dd{b}{kQ} &\xrightarrow{\ \ \quad} \Dd{b}{k}^{\Gamma_{Q^\mathrm{op}}} \\
X \quad &\xmapsto{\ \ \quad} \ \widetilde{X} = \mathbb{R}\underline{\mathrm{Hom}}(-,X)
\end{align*}
which is a homotopy right-inverse of the restriction. Indeed, this functor extends a representation $X$ of $Q$ to a representation $\widetilde{X}$ of $\Gamma_{Q^\mathrm{op}}$; from taking values in the indecomposable projectives to taking values in every indecomposable of $\Dd{b}{kQ}$. We call the extended representations $\widetilde{X} = \mathbb{R}\underline{\mathrm{Hom}}(-,X) \in \Dd{b}{k}^{\Gamma_{Q^\mathrm{op}}}$ \emph{coherent Auslander-Reiten diagrams}. From these, one can exploit properties of the Auslander-Reiten quiver at the level of representations:

\begin{example}
Let $X \in \Dd{b}{k}^Q \simeq \Dd{b}{kQ}$ and consider the Auslander-Reiten translation $\tau$ in $\mathsf{D}^b(kQ)$ (i.e. the derived Nakayama functor) and its inverse $\tau^-$. There are natural equivalences:
$$(\tau X)_v \simeq \mathbb{R}\iHom(P(v),\tau X) \simeq \mathbb{R}\iHom(\tau^-P(v),X) \simeq \widetilde{X}(\tau^-P(v)).$$
That is, one can recover the translate of $X$ in $\mathsf{D}^b(kQ)$ from the coherent Auslander-Reiten diagram $\widetilde{X} = \mathbb{R}\underline{\mathrm{Hom}}(-,X)$ and the translation of $\Gamma_Q$.
\end{example}

Coherent Auslander-Reiten diagrams admit the following abstract description:

\begin{lemm} \label{lemm:cohAR}
Let $X \in \Dd{b}{kQ}$ and write $\widetilde{X} = \mathbb{R}\underline{\mathrm{Hom}}(-,X) \in \Dd{b}{k}^{\Gamma_{Q^\mathrm{op}}}$. Then:
\begin{enumerate}
    \item Every mesh in $\Gamma_Q$ starting in $\tau M$ and ending in $M$ is sent by $\widetilde{X}$ to a cofiber sequence of the form
    $$\widetilde{X}(M) \to \textstyle\bigoplus\limits_{i=1}^n \widetilde{X}(E_i) \to \widetilde{X}(\tau M).$$
    \item If $Q$ is non-Dynkin, and $\sigma$ is the automorphism of $\Gamma_Q^\mathrm{irr}$ corresponding to the suspension (c.f. {\normalfont{\Cref{fact:Happel2}}}), then $\widetilde{X}$ sends $\sigma M$ to the suspension $\Sigma\widetilde{X}(M)$.
\end{enumerate}
\end{lemm}
\begin{proof}
Follows from the fact that $\widetilde{X}=\mathbb{R}\underline{\mathrm{Hom}}(-,X): \Dd{b}{kQ}^\mathrm{op} \to \Dd{b}{k}$ is an exact functor of stable $\infty$-categories.
\end{proof}

\begin{rema} \label{rema:cohAR}
By the description of the Auslander-Reiten quiver in \Cref{fact:Happel1,fact:Happel2}, it follows that one can reach every vertex of the irregular part $\Gamma_Q^\mathrm{irr}$ from the indecomposable projectives using meshes and the automorphism $\sigma$.
Hence, the previous lemma tells us how to extend $X$ to $\widetilde{X}$ in $\Gamma_Q^\mathrm{irr}$ using cofibers and suspension.
\end{rema}

\subsection{Mesh representations of the repetitive quiver} We give an abstract version of coherent Aulander-Reiten diagrams for arbitrary stable $\infty$-categories. The main construction (\Cref{theo:ZQ-mesh}) is given for diagrams of shape $\ZQ$ and it is based on the ideas of \Cref{lemm:cohAR} and \Cref{rema:cohAR}. By iterated abstract reflection functors, we are able to identify $\mathcal{C}^Q$ with a suitable \emph{mesh $\infty$-category} of $\ZQ$.

\begin{nota}
Let $Q$ be a finite quiver and $\{v_1,...,v_n\} \subset Q_0$ an (ordered) finite set of vertices. For $?\in\{+,-\}$, we define $Q^?(v_1,...,v_n)$ inductively as $Q^?(\varnothing) := Q$ and $Q^?(v_1,...,v_k) := Q^?(v_1,...,v_{k-1})^?(v_k)$.
\end{nota}

Let $Q$ be a finite quiver. An ordering $Q_0 = \{v_1,...,v_n\}$ of its vertices is called \emph{admissible} if for each $i$ the vertex $v_i$ is a source in $\sigma_{v_{i-1}}\cdots\sigma_{v_1}Q$. 
We observe that an admissible ordering exists if and only if $Q$ is acyclic, see \cite[Lemma 3.1.1]{Kra08}.

\begin{cons}[Knitting] \label{cons:knitting}
Let $Q$ be a finite acyclic quiver and $Q_0=\{v_1,...,v_k\}$ an admissible ordering of its vertices.
\begin{itemize}[align=parleft, labelwidth=\widthof{(Knitting to both sides)}, leftmargin=\dimexpr\labelwidth+\labelsep\relax+0.25cm]
    \item[(Knitting to the right)] For $n \in \mathbb{N} \cup \{\infty\}$, we define ${_nQ}$  inductively:
    $$_0Q := Q, \quad _nQ := {_{n-1}Q}^+(\tau^{-n+1}v_1,...,\tau^{-n+1}v_k), \quad \text{and} \quad _\infty Q := \bigcup\limits_{n\geq 0}{_nQ}.$$
    \item[(Knitting to the left)] For $n \in \mathbb{N} \cup \{\infty\}$, we define ${_{-n}Q}$  inductively:
    $$_0Q := Q, \quad _{-n}Q := {_{-n+1}Q}^-(\tau^{n-1}v_k,...,\tau^{n-1}v_1), \quad \text{and} \quad _{-\infty} Q := \bigcup\limits_{n\geq 0}{_{-n}Q}.$$
    \item[(Knitting to both sides)] For $n \in \mathbb{N} \cup \{\infty\}$, we define ${_n\underline{Q}}$  inductively:
    \begin{align*}
    _0\underline{Q} := Q, \quad _n\underline{Q} &:= {_{n-1}\underline{Q}}^+(\tau^{-n+1}v_1,...,\tau^{-n+1}v_k)^-(\tau^{n-1}v_k,...,\tau^{n-1}v_1), \\ \text{and} \quad _\infty \underline{Q} &:= \bigcup\limits_{n\geq 0}{_n\underline{Q}}.  
    \end{align*}
\end{itemize}  
\end{cons}

\begin{prop}
Let $Q$ be a finite acyclic quiver. There are natural isomorphisms of quivers $_\infty Q \cong \mathbb{N}Q$, $_{-\infty} Q \cong -\mathbb{N}Q$ and $_\infty \underline{Q} \cong \ZQ.$
\end{prop}
\begin{proof}
We only prove the first isomorphism, the others being analogous. Using the result \cite[Proposition VIII.1.5]{AssSimSko06}, it will follow from the next three simple facts: (1) $(_\infty Q, \tau)$ is a translation quiver; (2) $Q$ is a section of ${_\infty Q}$ in the sense of \cite[Definition VIII.1.2]{AssSimSko06}; and (3) for $v_i \in Q_0$, $\tau^nv_i$ is defined in ${_\infty}Q$ if and only if $n \leq 0$.

Let us prove (1), (2) and (3) now. The last one is clear. For (2): $Q$ is acyclic by assumption; by definition, every vertex $v$ in ${_\infty Q}$ is such that $\tau^nv \in Q_0$ for a unique $n \in \mathbb{Z}$; and the convexity of $Q$ in ${_\infty Q}$ follows directly from the fact that all predecessors of $v_i \in Q_0$ are in $Q$. Finally, for (1), one needs to show that there is a bijection $\mathrm{arr}(v,w) \isoarrow \mathrm{arr}(w,\tau^{-1}v)$ for any $v,w \in {_\infty Q}_0$. It is clear that such bijections exist for $v$ in the step $\Gamma^+(v)$ where $\tau^{-1}v$ is added (for $\Gamma$ the subquiver of ${_\infty Q}$ previous to the addition of $\tau^{-1}v$). Arrows $\bullet \to \tau^{-1}v$ are only added in such a step, so $\mathrm{arr}(w,\tau^{-1}v)$ is never changed. But because we are adding vertices according to the admissible ordering, it is also true that no arrows $v \to \bullet$ can be added in the next steps. Thus, the bijection $\mathrm{arr}(v,w) \isoarrow \mathrm{arr}(w,\tau^{-1}v)$ still holds in ${_\infty Q}$.
\end{proof}

\begin{rema}
The previous proposition also shows that \Cref{cons:knitting} does not depend on the particular choice of an admissible ordering.
\end{rema}

Let $\mathcal{C}$ be a stable $\infty$-category along this section.

\begin{rema}
We observe that $_\infty Q$, $_{-\infty} Q$ and $_\infty \underline{Q}$ are defined as direct unions in $\sSet$ (i.e. filtered colimits of inclusions).
$$Q={_0 \underline{Q}} \xhookrightarrow{\quad} {_1 \underline{Q}} \xhookrightarrow{\quad} \cdots \xhookrightarrow{\quad} {_n \underline{Q}} \xhookrightarrow{\quad} \cdots \quad\quad \text{with \ $\varinjlim\limits_{n\geq 0} {_n \underline{Q}} = {_\infty \underline{Q}} \cong \ZQ$}.$$ 
Applying $\mathcal{C}^{(-)} = \Fun(-,\mathcal{C})$, we get an inverse system of (stable) $\infty$-categories of representations and restriction functors between them with
$$\mathcal{C}^\ZQ = \varprojlim(\mathcal{C}^Q=\mathcal{C}^{_0 \underline{Q}} \ \xleftarrow{\ \quad} \ \mathcal{C}^{_1 \underline{Q}} \ \xleftarrow{\ \quad} \ \cdots \ \xleftarrow{\ \quad} \ \mathcal{C}^{_n \underline{Q}} \ \xleftarrow{\ \quad} \ \cdots),$$
and the canonical functor from the limit $\mathcal{C}^\ZQ \to \mathcal{C}^{_n \underline{Q}}$ is the restriction along $_n \underline{Q} \subset \ZQ$.
\end{rema}

We will now define inductively mesh $\infty$-categories $\mathcal{C}^{\Gamma,\, \mathrm{mesh}}$ for subquivers $\Gamma$ of $\ZQ$ and reach $\mathcal{C}^\ZQmesh$ in the limit.

\begin{defi}
Let $\Gamma$ be a finite quiver with a mesh $\infty$-category $\mathcal{C}^{\Gamma,\, \mathrm{mesh}}$ and a forgetful functor $\mathcal{C}^{\Gamma,\, \mathrm{mesh}} \to \mathcal{C}^\Gamma$. For $?\in \{+,-\}$ and $v \in \Gamma_0$, we denote $\mathcal{C}^{\Gamma^?(v),\, v\text{-}\mathrm{mesh}}$ the \emph{(local) mesh $\infty$-category} at $v$ obtained from \Cref{def:mesh-subcat}. We define the \emph{(total) mesh $\infty$-category} $\mathcal{C}^{\Gamma^?(v),\, \mathrm{mesh}}$ of $\Gamma^?(v)$ by the following (homotopy) pullback.
\begin{equation} \label{eq:def-total-mesh}
\begin{tikzcd}
\mathcal{C}^{\Gamma^?(v),\, \mathrm{mesh}} \ar[r] \ar[d] \ar[rd,phantom,"\PBho"] & \mathcal{C}^{\Gamma^?(v),\, v\text{-}\mathrm{mesh}} \ar[d] \\[0.5em]
\mathcal{C}^{\Gamma,\, \mathrm{mesh}} \ar[r] & \mathcal{C}^{\Gamma}
\end{tikzcd}
\end{equation}
The composition $\mathcal{C}^{\Gamma^?(v),\, \mathrm{mesh}} \to \mathcal{C}^{\Gamma^?(v),\, v\text{-}\mathrm{mesh}} \to \mathcal{C}^{\Gamma^?(v)}$ gives a new forgetful functor.

In particular, letting $\mathcal{C}^{_0 \underline{Q},\, \mathrm{mesh}} = \mathcal{C}^Q$, the construction \eqref{eq:def-total-mesh} gives inductively mesh $\infty$-categories $\mathcal{C}^{_n \underline{Q},\, \mathrm{mesh}}$, forgetful functors $\mathcal{C}^{_n \underline{Q},\, \mathrm{mesh}} \to \mathcal{C}^{_n \underline{Q}}$ and restriction functors $\mathcal{C}^{_n \underline{Q},\, \mathrm{mesh}} \to \mathcal{C}^{_{n-1} \underline{Q},\, \mathrm{mesh}}$ for all $n\geq 1$.

We define the \emph{mesh $\infty$-category} $\mathcal{C}^\ZQmesh$ to be the (homotopy) limit of mesh $\infty$-infinity categories
$$\mathcal{C}^\ZQmesh = \varprojlim(\mathcal{C}^Q=\mathcal{C}^{_0 \underline{Q},\, \mathrm{mesh}} \xleftarrow{\ \ } \mathcal{C}^{_1 \underline{Q},\, \mathrm{mesh}} \xleftarrow{\ \ } \cdots \xleftarrow{\ \ } \mathcal{C}^{_n \underline{Q},\, \mathrm{mesh}} \xleftarrow{\ \ } \cdots).$$
The forgetful functors $\mathcal{C}^{_n \underline{Q},\, \mathrm{mesh}} \to \mathcal{C}^{_n \underline{Q}}$ induce a forgetful functor $\mathcal{C}^\ZQmesh \to \mathcal{C}^\ZQ$.
\end{defi}

\begin{rema}
\eqref{eq:def-total-mesh} is a pullback in $\infCAT$, and by \Cref{theo:Qpv-mesh,theo:Qmv-mesh}, the right vertical map is an equivalence. Thus $\mathcal{C}^{\Gamma^?(v),\, \mathrm{mesh}} \to \mathcal{C}^{\Gamma,\, \mathrm{mesh}}$ is also an equivalence. In particular, we get equivalences $\mathcal{C}^{_n \underline{Q},\, \mathrm{mesh}} \isoarrow \mathcal{C}^{_{n-1} \underline{Q},\, \mathrm{mesh}}$ for all $n \geq 1$. 
\end{rema}

\begin{theo} \label{theo:ZQ-mesh}
Let $Q$ be a finite acyclic quiver. Restriction along the inclusion $Q \subset \ZQ$ induces a natural equivalence
$\mathcal{C}^\ZQmesh \isoarrow \mathcal{C}^Q$.
\end{theo}
\begin{proof}
The functor $\mathcal{C}^\ZQmesh \to \mathcal{C}^Q$ is the transfinite op-composition in $\infCAT$ of the countable sequence of equivalences
\begin{equation*} 
\mathcal{C}^Q=\mathcal{C}^{_0 \underline{Q},\, \mathrm{mesh}} \xleftarrow{\ \simeq\ } \mathcal{C}^{_1 \underline{Q},\, \mathrm{mesh}} \xleftarrow{\ \simeq\ } \cdots \xleftarrow{\ \simeq\ } \mathcal{C}^{_n \underline{Q},\, \mathrm{mesh}} \xleftarrow{\ \simeq\ } \cdots
\end{equation*}
Thus it is also an equivalence.
\end{proof}

\begin{rema}
There is an analogy between the above construction and the Auslander algebra $\mathsf{\Lambda}_Q$ of a Dynkin quiver. The Gabriel quiver of $\mathsf{\Lambda}_Q$ is $\Gamma(\mathsf{mod}\, kQ)^\mathrm{op}$ and there is a natural equivalence $\mathsf{mod}\, kQ \simeq \mathsf{proj}\, \mathsf{\Lambda}_Q$ (see \cite[sec. VI.5]{AusReiSma97}).
\end{rema}

Now we give a particularly useful description of the mesh $\infty$-category of $\ZQ$: it can be also described as pairs $(X,(c_{\tau^nv})_{\tau^nv\in\ZQ})$ consisting of a representation $X:\ZQ \to \mathcal{C}$ and, for each mesh of $\ZQ$ (corresponding to a vertex $\tau^nv$), a cofiber sequence 
$$\begin{tikzcd} X_{\tau^nv} \ar[r] \dar &[-0.8em] \bigoplus\limits_{\tau^nv\to w} X_w \ar[d] \\[-0.3em] 0 \rar & X_{\tau^{n-1}v} \end{tikzcd}$$
realizing the value of $X$ at the vertex $\tau^{n-1}v$ as the cofiber of $X_{\tau^nv} \to \bigoplus\limits_{\tau^nv\to w} X_w$.

\begin{prop} \label{prop:desc-ZQmesh}
Let $Q$ be a finite acyclic quiver. There is a (homotopy) pullback 
\begin{equation} \label{eq:ZQ-mesh}
\begin{tikzcd}
\mathcal{C}^\ZQmesh \ar[r] \ar[d] \ar[rd,phantom, "\PBho",pos=0.55]  & \prod\limits_{\tau^nv}\mathcal{C}^{\square,\, \mathrm{cof}} \ar[d,"(l^*)_{\tau^nv}"] \\[0.5em]
\mathcal{C}^\ZQ \ar[r] & \prod\limits_{\tau^nv}\mathcal{C}^{\Lambda^2_1},
\end{tikzcd} 
\end{equation}
where the functor $\mathcal{C}^\ZQ \to \prod\limits_{\tau^nv}\mathcal{C}^{\Lambda^2_1}$ sends $X \mapsto (X_{\tau^nv} \to \textstyle\bigoplus\limits_{\tau^nv \to w} X_w \to X_{\tau^{n-1} v})_{\tau^nv}$.
\end{prop}

The proof uses the following purely categorical lemma about pullbacks.

\begin{lemm}
Consider the following diagram consisting of three pullback squares in an arbitrary $\infty$-category with finite limits:
$$\begin{tikzcd}
p \arrow[r, "\varepsilon_1"] \arrow[d, "\varepsilon_2"'] \ar[rd,phantom, "\lrcorner", very near start] & p_1 \arrow[d, "\alpha_1"'] \arrow[r, "\delta_1"] \ar[rd,phantom, "\lrcorner", very near start] & e_1 \arrow[d, "\beta_1"] \\
p_2 \arrow[r, "\alpha_2"] \arrow[d, "\delta_2"'] \ar[rd,phantom, "\lrcorner", very near start] & c \arrow[r, "\gamma_1"'] \arrow[d, "\gamma_2"] & d_1 \\
e_2 \arrow[r, "\beta_2"] & d_2. &
\end{tikzcd}$$
Then the following square is also a pullback:
$$\begin{tikzcd}
p \ar[r,"{(\varepsilon_1\delta_1,\varepsilon_2,\delta_2)}"] \ar[d,"\alpha_1\varepsilon_1"'] \ar[rd,phantom, "\lrcorner", very near start] &[1em] e_1\times e_2 \ar[d,"\beta_1\times\beta_2"] \\
c \ar[r,"{(\gamma_1,\gamma_2)}"'] & d_1\times d_2.
\end{tikzcd}$$
\end{lemm}
\begin{proof}
It follows by the following pasting of pullbacks:
$$\begin{tikzcd}
p \arrow[r, "{(\varepsilon_1,\varepsilon_2)}"] \arrow[d, "\alpha_1\varepsilon_2"'] \ar[rd,phantom, "\lrcorner", very near start] & p_1\times p_2 \arrow[d, "\alpha_1\times\alpha_2"'] \arrow[r, "\delta_1\times\delta_2"] \ar[rd,phantom, "\lrcorner", very near start] & e_1\times e_2 \arrow[d, "\beta_1\times\beta_2"] \\
c \arrow[r, "{(1,1)}"']                                                             & c\times c \arrow[r, "\gamma_1\times\gamma_2"']                                         & d_1\times d_2.                                  
\end{tikzcd}$$
The right one is a product of pullbacks. For the left one: denoting $\pi_i:p_1\times p_2 \to p_i$ the projections, the upper-left pullback in the hypothesis can be turned into an equalizer
$$\begin{tikzcd}
p \arrow[r, "{(\varepsilon_1,\varepsilon_2)}"] & p_1\times p_2 \arrow[r, "\varepsilon_1\pi_1", shift left] \arrow[r, "\varepsilon_2\pi_2"', shift right] & c
\end{tikzcd}$$
by \cite[Proposition 7.6.4.23]{Lur25}, and this equalizer can be turned into the desired pullback by \cite[Proposition 7.6.4.22]{Lur25}.
\end{proof}

\begin{proof}[Proof {\normalfont{(of \Cref{prop:desc-ZQmesh})}}]
Let $\Gamma$ be the quiver resulting from $Q$ after adding the $n^\text{th}$ mesh of $\ZQ$ and assume there is a (homotopy) pullback 
$$\begin{tikzcd}
\mathcal{C}^{\Gamma,\, \mathrm{mesh}} \ar[r] \ar[d] \ar[rd,phantom, "\PBho",pos=0.6]  & (\mathcal{C}^{\square,\, \mathrm{cof}})^n \ar[d,"(l^*)_i"] \\[0.5em]
\mathcal{C}^\Gamma \ar[r] & (\mathcal{C}^{\Lambda^2_1})^n.
\end{tikzcd}$$
Adding the $(n+1)^\text{th}$ mesh, we have the following diagram of (homotopy) pullbacks
$$\begin{tikzcd}
{\mathcal{C}^{\Gamma^?(v),\, \mathrm{mesh}}} \arrow[d] \arrow[r] \ar[rd,phantom, "\PBho",pos=0.58] & {\mathcal{C}^{\Gamma^?(v),\, v\text{-}\mathrm{mesh}}} \arrow[d] \arrow[r] \ar[rd,phantom, "\PBho",pos=0.5] & {\mathcal{C}^{\square,\, \mathrm{cof}}} \arrow[d] \\
{\mathcal{C}^{\Gamma^?(v),\, \Gamma\text{-}\mathrm{mesh}}} \arrow[d] \arrow[r] \ar[rd,phantom, "\PBho",pos=0.5] & \mathcal{C}^{\Gamma^?(v)} \arrow[d] \arrow[r] & \mathcal{C}^{\Lambda^2_1} \\
{\mathcal{C}^{\Gamma,\, \mathrm{mesh}}} \arrow[r] \arrow[d] \ar[rd,phantom, "\PBho",pos=0.6] & \mathcal{C}^{\Gamma} \arrow[d] & \\
{(\mathcal{C}^{\square,\, \mathrm{cof}})^n} \arrow[r] & (\mathcal{C}^{\Lambda^2_1})^n &
\end{tikzcd}$$
where we introduce the notation $\mathcal{C}^{\Gamma^?(v),\, \Gamma\text{-}\mathrm{mesh}}$ for the pullback $\mathcal{C}^{\Gamma^?(v)} \times_{\mathcal{C}^\Gamma} \mathcal{C}^{\Gamma,\, \mathrm{mesh}}$.
By the previous lemma, we get a (homotopy) pullback square 
$$\begin{tikzcd}
\mathcal{C}^{\Gamma^?(v),\, \mathrm{mesh}} \ar[r] \ar[d] \ar[rd,phantom, "\PBho",pos=0.6]  & (\mathcal{C}^{\square,\, \mathrm{cof}})^{n+1} \ar[d,"(l^*)_i"] \\[0.5em]
\mathcal{C}^{\Gamma^?(v)} \ar[r] & (\mathcal{C}^{\Lambda^2_1})^{n+1}.
\end{tikzcd}$$
We know such a (homotopy) pullback exists for $n=0$ by definition (\Cref{def:mesh-subcat}). Therefore by induction, it exists for all $n\geq 0$. The limit of all these pullback is again a pullback, and it is precisely that of the statement. 
\end{proof}

\begin{coro} \label{coro:equiv-repet} 
Let $Q$ and $Q'$ be finite acyclic quivers. Any translation isomorphism of  repetitive quivers $f:\ZQ \cong \ZQ'$ induces a stable equivalence
$\tilde{f^*}:\mathcal{C}^{Q'} \simeq \mathcal{C}^Q$, i.e. a natural equivalence for all $\mathcal{C}$ stable $\infty$-categories.
\end{coro}
\begin{proof}
Let $f: \ZQ \xrightarrow{\, \cong\, } \ZQ'$ be a translation isomorphism. This induces an isomorphism $f^*:\mathcal{C}^{\ZQ'} \xrightarrow{\, \cong\, } \mathcal{C}^\ZQ$ and a correspondence between meshes of $\ZQ$ and of $\ZQ'$ that we can express as an isomorphism $\rho: \prod_{\tau^n f(v)} \mathcal{C}^{\Lambda^2_1} \xrightarrow{\cong} \prod_{\tau^n v} \mathcal{C}^{\Lambda^2_1}$. More precisely, the identity maps between the copy of $\mathcal{C}^{\Lambda^2_1}$ indexed by $\tau^n f(v) \in \ZQ'$ and the copy indexed by $\tau^n v\in\ZQ$ induce an isomorphism $\rho: \prod_{\tau^n f(v)} \mathcal{C}^{\Lambda^2_1} \xrightarrow{\cong} \prod_{\tau^n v} \mathcal{C}^{\Lambda^2_1}$. Similarly, one has an isomorphism $\rho': \prod_{\tau^n f(v)} \mathcal{C}^{\square,\, \mathrm{cof}} \xrightarrow{\cong} \prod_{\tau^n v} \mathcal{C}^{\square,\, \mathrm{cof}}$, and there is an isomorphism of cospans
$$\begin{tikzcd}
\mathcal{C}^{\ZQ'} \rar \ar[d,"f^*"',"\cong"] & \prod_{\tau^n f(v)} \mathcal{C}^{\Lambda^2_1} \ar[d,"\rho"',"\cong"] & \prod_{\tau^n f(v)} \mathcal{C}^{\square,\, \mathrm{cof}} \ar[d,"\rho'"',"\cong"] \lar \\
\mathcal{C}^{\ZQ} \rar & \prod_{\tau^n v} \mathcal{C}^{\Lambda^2_1} & \prod_{\tau^n v} \mathcal{C}^{\square,\, \mathrm{cof}}. \lar
\end{tikzcd}$$
By the pullback \eqref{eq:ZQ-mesh}, this induces an isomorphism $\mathcal{C}^{\ZQ',\, \mathrm{mesh}} \cong \mathcal{C}^\ZQmesh$, and composing with the equivalences from \Cref{theo:ZQ-mesh}, we get the desired equivalence:
\begin{equation*}
\mathcal{C}^{Q'} \xleftarrow{\ \simeq\ } \mathcal{C}^{\ZQ',\, \mathrm{mesh}} \cong \mathcal{C}^{\ZQmesh} \xrightarrow{\ \simeq\ } \mathcal{C}^Q.
\end{equation*}

For the naturality, we check that given an exact functor $F:\mathcal{C} \to \mathcal{D}$, the induced square 
$$\begin{tikzcd}
\mathcal{C}^Q \ar[r,"\tilde{f^*}"] \ar[d,"F_*"'] & \mathcal{C}^{Q'} \ar[d,"F_*"] \\
\mathcal{D}^Q \ar[r,"\tilde{f^*}"] & \mathcal{D}^{Q'}
\end{tikzcd}$$
commutes in $\infCAT$. Similarly to \Cref{rema:functoriality}, $(-)^\ZQmesh$ is functorial with respect to exact functors, and the naturality of pullbacks gives a commutative square 
$$\begin{tikzcd}
\mathcal{C}^\ZQmesh \ar[r,"\tilde{f^*}"] \ar[d,"F_*"'] & \mathcal{C}^{\ZQ',\, \mathrm{mesh}} \ar[d,"F_*"] \\
\mathcal{D}^\ZQmesh \ar[r,"\tilde{f^*}"] & \mathcal{D}^{\ZQ',\, \mathrm{mesh}}
\end{tikzcd}$$
in $\infCAT$.
Finally, one just needs to observe that the equivalence $\mathcal{C}^\ZQmesh \xrightarrow{\, \simeq\, } \mathcal{C}^Q$ is natural with respect to exact functors, as it is the composition of the forgetful functor $\mathcal{C}^\ZQmesh \to \mathcal{C}^\ZQ$ and the restriction $\mathcal{C}^\ZQ \to \mathcal{C}^Q$.
\end{proof}

\begin{rema}
We already knew from \Cref{theo:equiv-quivers} that $Q$ and $Q'$ are stably equivalent if and only if $\ZQ \cong \ZQ'$ as translation quivers. However, it is interesting to note that here we get a concrete equivalence $\tilde{f^*}:\mathcal{C}^{Q'} \simeq \mathcal{C}^Q$ from $f:\ZQ \cong \ZQ'$.
\end{rema}

We end this section by relating mesh representations of $\ZQ$ with exact functors from the derived category $\Dd{b}{kQ}$ with $k$ a field. This is done by fixing an inclusion $\Gamma_Q \xhookrightarrow{} \Dd{b}{kQ}$ of the Auslander-Reiten quiver, which amounts to choosing representatives for each iso-class of indecomposable objects and a basis of the space of irreducible morphisms between each two of them. Identifying $\ZQ^\mathrm{op}$ with the connected component of $\Gamma_Q$ containing the indecomposable projectives, we get an inclusion 
$$\iota:\ZQ \subset \Gamma_Q^\mathrm{op} \xhookrightarrow{ \ \ \ } \Dd{b}{kQ}^\mathrm{op}.$$

\begin{lemm}
There is a canonical functor $\tilde{\iota^*}: \Fun^\mathsf{ex}(\Dd{b}{kQ}^\mathrm{op},\mathcal{C}) \to \mathcal{C}^\ZQmesh$ mapping each exact functor to its restriction to the AR quiver, i.e. such that
$$\begin{tikzcd} \Fun^\mathsf{ex}(\Dd{b}{kQ}^\mathrm{op},\mathcal{C}) \ar[rr,"\iota^*"] \ar[dr,"\tilde{\iota^*}"'] & & \mathcal{C}^{\ZQ} \\
& \mathcal{C}^\ZQmesh \ar[ur,"\mathrm{forget}"'] &
\end{tikzcd}$$
commutes in $\infCAT$.
\end{lemm}
\begin{proof}
Identify through $\iota$ vertices and arrows of $\ZQ$ with objects and morphisms of $\Dd{b}{kQ}$. A mesh of $\ZQ$ starting at an indecomposable object $M$ of $\Dd{b}{kQ}$ gives rise to an AR triangle and hence a cofiber sequence of the form $M \to \bigoplus_{i=1}^n E_i \to \tau^-M$, which defines a diagram $c:\square \to \Dd{b}{kQ}^\mathrm{op}$. This produces a commutative diagram
$$\begin{tikzcd}
\Fun^\mathsf{ex}(\Dd{b}{kQ}^\mathrm{op},\mathcal{C}) \ar[r,"c^*"] \ar[d,"\iota^*"'] & \mathcal{C}^{\square,\, \mathrm{cof}} \ar[d,"l^*"] \\
\mathcal{C}^\ZQ \ar[r] & \mathcal{C}^{\Lambda^2_1}
\end{tikzcd}$$
for each mesh of $\ZQ$. Then the universal property of the homotopy pullback \eqref{eq:ZQ-mesh} gives the desired functor $\tilde{\iota^*}: \Fun^\mathsf{ex}(\Dd{b}{kQ}^\mathrm{op},\mathcal{C}) \to \mathcal{C}^\ZQmesh$.
\end{proof}

\begin{coro} \label{rema:Rhom}
The functor $s: \Dd{b}{kQ} \xrightarrow{ } \Dd{b}{k}^{\ZQ},\ X \xmapsto{ } \mathbb{R}\iHom(-,X),$ from \ref{subsec:ARquiver} factors through $\Dd{b}{k}^\ZQmesh$, and this gives a homotopy inverse of the equivalence $\varphi: \Dd{b}{k}^\ZQmesh \isoarrow \Dd{b}{kQ}$ from \Cref{theo:ZQ-mesh}.
\end{coro}
\begin{proof}
Because the $\mathbb{R}\iHom(-,X)$ are exact, the transpose of $$\mathbb{R}\iHom: \Dd{b}{kQ}^\mathrm{op} \times \Dd{b}{kQ} \xrightarrow{ \ \ } \Dd{b}{k}$$ gives a functor 
$$\Upsilon:\Dd{b}{kQ} \xrightarrow{ \ \ } \Fun^\mathsf{ex}(\Dd{b}{kQ}^\mathrm{op},\Dd{b}{k}),\quad X \xmapsto{ \ \ } \mathbb{R}\iHom(-,X).$$ If we let $\mathcal{C}=\Dd{b}{k}$ in the previous lemma, then $s$ is precisely $\iota^*\Upsilon$, and so $s$ factors through $\Dd{b}{k}^\ZQmesh$ using $\phi:=\tilde{\iota^*}\Upsilon:\Dd{b}{kQ} \xrightarrow{ \ \ } \Dd{b}{k}^\ZQmesh$.

Now let us denote $r$ the restriction along $Q\subset \ZQ$, so that $rs \simeq \mathrm{id}$, and let $j:\Dd{b}{k}^\ZQmesh \to \Dd{b}{k}^{\ZQ}$ be the forgetful functor. Then $\phi\varphi = rj\phi \simeq rs \simeq \mathrm{id}$, and thus $\phi = \varphi^{-1}$ (in the homotopy category of $\infCAT$).
\end{proof}