Fix now sequences $\Gamma^\pm$ of Reeb orbits and a relative homology class $\beta\in H_2(Y,\Gamma^+\sqcup\Gamma^-)$. Let $T$ be the corolla with degree $\beta$ and positive/negative exterior edges labeled by $\Gamma^\pm$. Fix $L > \cA_\lambda(\Gamma^+),\cA_\lambda(\Gamma^-)$. Then, choose a $\cP(T)$-integral approximation $\wt\lambda$ and let $d$ be the integer of Definition~\ref{de:auxiliary-datum}. Let $\cBR = \cBR_{\Gamma^+\sqcup \Gamma^-}(d)$ be the space defined in \textsection\ref{subsec:base-space}. Recall that it is a principal torus bundle over the real-oriented blow-up of $\cB_{0,\Gamma^+\sqcup \Gamma^-}(d)$. Let $\cC\to \cBR$ be the pullback of the universal family of $\cB_{0,\Gamma^+\sqcup \Gamma^-}(d)$.
Due to the action bound, we may fix $\kappa_L > 0$ sufficiently small so that non-trivial cylinders have $\lambda$-energy at least $2\kappa_L$. 

\begin{definition}\label{de:family-of-tree}
We define $\cZ = \cZ_{\wt\lambda}(T)$ be the space of tuples $(\varphi,u)$ where 
\begin{enumerate}[label=\roman*),leftmargin=20pt,ref=\roman*]
    \item $\varphi\in \cBR$ lies in the stratum associated to a tree $T_\varphi$ admitting a contraction $T_\varphi\to T$,
    \item $u = ([u_v])_{v\in V(T_\varphi)}$ is a collection of equivalence classes of smooth maps $u_v \cl \dot{C}_v\to \wh Y$ up to translation, where
	\begin{itemize}[leftmargin=15pt]
    \setlength\itemsep{2.5pt}
	\item $u_v$ is $J$-holomorphic near the punctures of $\dot{C}_v$,
	\item if $x \in C_v$ is a positive/negative puncture (mapping to a node) of type $1$, then $u_v$ is positively/negatively asymptotic to the Reeb orbit $\gamma_e$, where $e\in E(T')$ is the associated edge;
    \item the matching isomorphism of $\varphi$ at the edge $e =(v,v')$ intertwines $(\wh u_v)_{z_{v,e}}$ and $(\wh u_{v'})_{z_{v',e}}$,
    \item $\int_{C_v}{u_v}^*_Yd\lambda \geq 0$,
	\item if $C_v$ is unstable, then $\int_{C_v}u^*_Yd\lambda \geq \kappa_L$.
	\end{itemize}
    \end{enumerate}
\end{definition}

\noindent We assume $\cBR$ and $\cC$ are equipped with $G$-invariant metrics $d_\cB$ and $d_\cC$ respectively, the choice of which is irrelevant. Define 
\begin{equation}\label{eq:curlyCnbd}
    \cC_{\ge \epsilon} \coloneqq \{ \zeta \in \cC\mid d_\cC(\zeta,\text{Crit}(\pi)) \ge\epsilon\}
\end{equation}
for $\epsilon > 0$. 
We equip $\cZ$ with the topology generated by the following $\epsilon$-neighborhoods for $\epsilon > 0$. Given $(\varphi,u)\in \cZ$, define $\cN_\epsilon(\varphi,u)$ to be the subset of points $(\varphi',u')$ such that
\begin{itemize}[leftmargin=20pt]
\setlength\itemsep{1pt}
	\item $d_\cB(\varphi,\varphi') < \epsilon$;
	\item the (orbits of the) graphs satisfy 
    \begin{equation*}
    d_{\text{H}}\lbr{\bR^{V(T_\varphi)}\cdot\graph(\varphi,u)|_{\cC_{\ge\epsilon}},\bR^{V(T_{\varphi'})}\cdot\graph(\varphi',u')|_{\cC_{\ge\epsilon}}} < \epsilon
	\end{equation*} 
	in the Gromov-Hausdorff metric, where $T_\varphi$ is the dual graph of the fiber $\cC_\varphi$ and we choose any representatives of the classes $[u_v]$ and $[u'_{v'}]$; 
	\item for $e \in E(T_\varphi)$ with associated Reeb orbit $\gamma_e$ and corresponding node $x_e \in \cC_\varphi$ we have $$d_Y(u'_Y(z),\gamma_e) < \epsilon$$ 
	for any $z \in \cC_{\varphi'}$ with $d_\cC(z,x_e) \leq \epsilon$.
\end{itemize}

\subsubsection{Determining unitary framings}\label{subsec:domain-metrics} 
Abusing notation, we also denote by $\cC \to \cZ$ the pullback of the universal family $\cC\to \cB_{\Gamma^+\sqcup \Gamma^-}(d)$. The $\cG$-action on $\cBR$ lifts to a $\cG$-action on $\cZ$. In contrast to the action on $\cBR$, which is not proper, the additional data of the building means that the lifted action is proper, which will be crucial to reduce to the action of the compact group $G$ later on.

\begin{lemma}\label{lem:palais-proper} The $\cG$-action on $\cZ$, given by 
	$$g \cdot (s,u) = (g\cdot s,u\g g\inv),$$
	is proper in the sense of Palais, \cite{Pal61}.
\end{lemma}

\begin{proof} Let $\wt \cC$ be the real blowup at the images of the sections and the nodes of the fibers of $\cC\to \cV$. Fix a smooth metric on $\wt\cC$. This restricts to a smooth metric on $\cC\inn$. Given now $(s,u)\in \cZ$, we see that the projection $u_Y$ of $u$ to $Y$ extends to a continuous function $\bar{u}_Y\cl \wt\cC_s \to Y$. Given any Riemannian metric on $Y$, we obtain an inequality of Lipschitz numbers 
	$$L(u_Y)\leq L(\bar{u}_Y).$$
	The Lipschitz number of $\bar{u}_Y$ is finite since its differential is bounded away from the ``puncture circles" and we know its behavior near the ``puncture circles". Thus, we may use the same argument as in \cite[Lemma 4.13]{AMS23} to conclude.
 \end{proof}

 \begin{lemma}\label{lem:zero-locus-right}
     Writing $\cZ_{\delbar}\sub \cZ$ for the closed subspace of $J$-holomorphic building, the quotient map 
    \begin{equation}\label{eq:footprint}
        \psi \cl \cZ_{\delbar}/\cG\to \Mbar^{\, J}(T)
    \end{equation}
    is an isomorphism of orbi-spaces.
 \end{lemma}


 \begin{proof} The surjectivity of~\eqref{eq:footprint} follows from the construction of the very ample line bundles in \textsection\ref{subsec:framings}, while injectivity (including the statement about isotropy groups) can be shown as in \cite[Discussion~3.16]{HS22}. Continuity follows from the definition of the metric on $\cZ_{\delbar}$. Thus, it remains to show that~\eqref{eq:footprint} is open or, equivalently, closed.\par
 \vspace{-4pt}
 This can be checked locally on the target. To this end, let $[u,C]\in \Mbar^{\, J}(T)$ be arbitrary.
 By \cite[Proposition~3.26]{Par19}, we can find a divisor $D\sub Y$ so that $u_Y\pf D$ and adding $u_Y\inv(D)$ as marked points stabilizes the domain $C$ of $u$, and so that this is the minimal number of marked points required to stabilize the domain.
 Then, there exists a neighborhood $U \sub \Mbar^{\, J}(T)$ of $[u,C]$ so that for any $[u',C']\in U$, the map $u'_Y$ intersects $D$ transversely and the added points ${u'}\inv(D)$ stabilize $C'$.
 This yields a continuous map 
 $$\ff\cl U \to \Mbar_{0,\#\Gamma^-+\#\Gamma^++ m}/S_m,$$ 
 where $m = \#u\inv(D)$ and the symmetric action permutes the last $m$ marked points.
 Let $\rho_1,\dots,\rho(d)$ be local sections of the universal family 
 $$\cc{\cC} := \cC_{0,\#\Gamma^-+\#\Gamma^++ m}\to \Mbar_{0,\#\Gamma^-+\#\Gamma^++ m}/S_m$$ near $p = \ff([u,C])$, whose images do not meet the nodal points of the fibers and so that for each irreducible component $C_v$ of $C$ we have 
 \begin{equation}\label{eq:number-of-sections}
 \#\{i\mid \rho_i(p) \in C_v\} = \deg(\fL_u|_{C_v}^{\otimes p}).
 \end{equation}
Shrinking $U$ if necessary, we may assume the equality in \eqref{eq:number-of-sections} holds for any point in $U$. Define the holomorphic line bundle $\cc{\cL}\coloneqq \cO_{\cc{\cC}}(\rho_1 +\dots +\rho(d))$. It pulls back to a complex orbi-line bundle over the universal family of $\Mbar^{\, J}(T)$, and has the same multi-degree as $\fL_u^{\otimes p}$ when restricted to the fiber over $[u,C]$. Therefore, as in \cite[Lemma~4.8]{HS22}, the forgetful map $\cZ_{\delbar}\to \Mbar^{\, J}(T)$ is locally the projectivization of a continuous orbi-bundle. In particular, the map is closed.
\end{proof}

\begin{remark}
    The above proof strongly relies on the fact that our curves have genus zero. 
\end{remark}


By \cite[Lemma~4.4]{HS22}, the $\PGL_{d+1}(\bC)$-action on $\cB_{n}(d)$ is proper when restricted to the locus $\cB^{\stb}_{n}(d)\sub\cB_{n+ \Gamma^+,\Gamma^-}(d)$ of curves with stable domain (for any $n \ge 0$). Therefore, \cite[Corollary~4.6]{HS22} asserts the existence of a $\PGL_{d+1}(\bC)$-invariant map 
\begin{equation}\label{eq:reducing-structure-group}
    \zeta\cl \cB^{\stb}_{3d'}(d)\to \PGL_{d+1}(\bC)/\PU(d+1)\cong \fp\fu_{d+1}.
\end{equation}
where the isomorphism is induced by the polar decomposition, \cite[Lemma~3.8(iii)]{HS22}. By averaging, we can choose $\zeta$ to be invariant under the $S_{3d'}$-action on $\cB_{3d'+ \Gamma^+,\Gamma^-}^{\stb}(d)$ given by permuting the marked points labeled by $\{1,\dots,3d'\}$.
Lemma~\ref{lem:zero-locus-right} and \cite[Proposition~3.26]{Par19}, refined as in the proof of \cite[Lemma~4.3]{HS22}, imply that we can find a finite set $\{D_i,r_i\}$ of compact codimension-$2$ submanifolds with boundary $D_i \sub Y$ so that the sets 
$$U_i := \set{(\varphi,u) \in \cZ\mid u\pf D_i,\,\forall C'\sub \cC|_\varphi: \# C'\cap u\inv(D_i) = 3\deg(L_u|_{C'}),\, (\cC|_\varphi,u\inv(D_i)) \text{ is stable}}$$
form an open cover of $\cZ_{\delbar}$. We replace $\cZ$ with $\union{i}{U_i}$ without further mention. Since $\cZ$ is metrizable, there exists a $\cG$-invariant partition of unity $\{\chi_i\}\iI$ subordinate to $\{U_i\}\iI$ by Lemma \ref{lem:palais-proper} and \cite[Corollary~4.12]{AMS23}. Let 
$$\Phi_i \cl U_i \to \cB_{3d'}^{\stb}(d)/S_{3d'}$$ 
be given by forgetting the marked points labeled by $\Gamma^\pm$, adding the intersections as marked points, mapping to $\cBR_{3d'}(d)/S_{3d'}$ and then applying the blow-down map. Finally, define
\begin{equation}\label{}
    \zeta_\cU \cl \cZ\,\to\, \fp\fu_{d+1} 
\end{equation}
by 
\begin{equation}\label{eq:construction-of-slice-map}
\zeta_\cU(\varphi,u) = \s{i}{\chi_i(\varphi,u)\,\zeta(\Phi_i(\varphi,u))}.\end{equation}

In \cite{HS22}, the collection $\{(U_i,D_i,\chi_i)\}_i$ was called a good covering; see Definition~3.10 op. cit. While the above discussion shows that we can do the same in our setting, this definition is too restrictive for the inductive construction in \textsection\ref{subsec:unstructured-flow-cat}.
A prototype of the issue one faces is that once we fix divisors for one-leveled buildings, we need to find new divisors for two-leveled buildings such that the restriction to the one leveled components has the same intersection combinatorics with the already chosen divisors.
Thus, we replace it with the following variation.

\begin{definition}\label{de:good-covering}
    A \emph{good covering} of $\cZ_{\delbar}$ consists of 
    \begin{enumerate}[label=\roman*),leftmargin=20pt,ref=\roman*]
        \item a finite collection $\{U_i\}_i$ of $\PGL_{d+1}(\bC)$-invariant open subsets of $\cZ$ that cover $\cZ_{\delbar}$,
        \item for each $i$ smooth $\cG$-equivariant sections $\sigma_{i,j}\cl U_i \to \cC\inn|_{U_i}$ for $1 \leq j \le d+2$ together with divisors $D_{ij}\sub Y$ so that 
        \begin{itemize}[leftmargin=5pt]
            \item $\sigma_{i,j}$ and $\sigma_{i,j'}$ have disjoint images for $j \neq j'$
            \item for each $j$ we have $u(\sigma_{i,j}(\varphi,u)) \in D_{i,j}$ and $u\pf D_{i,j}$ near $\sigma_{i,j}(\varphi,u)$ for any $(\varphi,u)\in U_i$,
        \end{itemize}
        \item a continuous $\cG$-invariant function $\chi_i \cl \cZ \to [0,1]$ with support in $U_i$
    \end{enumerate}
     so that $\s{i}{\chi_i}$ is positive on $\cZ_{\delbar}$.
\end{definition}

\begin{remark}
    Note that the divisors need not be distinct, i.e., in the construction above, we can take $D_{i,j} = D_i$ for any $j$. Thus the existence of a good covering follows from the discussion above Definition \ref{de:good-covering}.
\end{remark}

The discussion before Definition~\ref{de:good-covering} carries over to the definition of good coverings and yields the following statement.

\begin{lemma}\label{lem:map-to-lie-algebra}
    A good covering $\cU$ together with the map in \eqref{eq:reducing-structure-group} determines a $\cG$-invariant map $\zeta_\cU\cl \cZ \to \fp\fu_{d+1}$.\qed
\end{lemma}

