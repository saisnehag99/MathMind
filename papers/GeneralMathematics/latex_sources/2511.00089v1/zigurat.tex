\documentclass[12pt]{article}
%\usepackage[french]{babel}
\usepackage{amssymb}
\usepackage{amsmath}
\usepackage{amssymb}
\usepackage{amsthm}
\usepackage{graphicx}
\usepackage{amsthm}
\usepackage{graphicx}
%\usepackage{psfrag}
\usepackage[dvipsnames]{xcolor}
\usepackage{graphics}
\usepackage{amsthm}
%\usepackage[dvips]{graphicx}
\font\fiverm=cmr10 scaled 700
\input pictex.tex
% macros
\newcommand{\esp}{\hspace{0.01cm}}
\newcommand{\Q}{\mathbb Q}
\newcommand{\N}{\mathbb N}
\newcommand{\C}{\mathbb C}
\newcommand{\clo}{\mathrm{S}^1}
\newcommand{\anal}{\mathrm{Diff}_+^{\omega}(\mathrm{S}^1)}
\newcommand{\D}{\displaystyle}
\newcommand{\M}{\mathcal{M}}
\theoremstyle{definition}

\newtheorem{thm}{Theorem}[section]

\newtheorem{prop}[thm]{Proposition}

\newtheorem{lem}[thm]{Lemma}

\newtheorem{rem}[thm]{Remark}

\newtheorem{defn}[thm]{Definition}

\newtheorem{cor}[thm]{Corollary}

\newtheorem{ex}[thm]{Example}

\newtheorem{qs}[thm]{Question}

\hyphenation{nu-me-ra-ble} \setlength{\oddsidemargin}{-0.05in}
\setlength{\evensidemargin}{-0.05in}
\newcommand{\dos}{\mathrm{Diff}^2_+(\mathrm{S}^1)}
\newcommand{\ceroundos}{\mathrm{Diff}^2_+([0,1])}
\newcommand{\dosabe}{\mathrm{Diff}^2_+([a,b[)}
\newcommand{\ce}{\mathrm{C}}
\newcommand{\vs}{\vspace{1cm}}
\newcommand{\sgn}{\vartriangleleft}
\setlength{\textwidth}{6.5in}
\setlength{\topmargin}{-0.5in}
\setlength{\textheight}{8.9in}

\input epsf

\def\contentsname{Table of contents}

\def\refname{References}

\begin{document}

\date{}
\author{Andr\'es Navas}

\title{Parametric proofs of the Pythagorean theorem \\
via ziggurats and pyramids}
\maketitle

\noindent{\bf Abstract.}  We propose two new proofs of the Pythagorean theorem via area rearrangement arguments starting from  
very simple geometric configurations. The constructions depend on an angular parameter, each choice of which yields a proof. 
For specific values of the parameter, we recover some classical and more recent proofs.

\vspace{0.2cm}

\noindent{\bf Keywords:} Pythagorean theorem, area rearrangement, trigonometric proof.

\vspace{0.2cm}

\noindent{\bf MCS 2020:} 
01-02, %research exposition
51-03, %history of geometry
97G30, %area and volume (educational aspects)
97G99, %geometry education, none of the above
97U99. %educational material


%%%%%%%%%%%%%%%%%%%%%%%%%%%%%%%%%%%%%%%%%%%%%%%%%%%%%%%%%%%%%%%%%%%%%%%%%%%%%%%%%

\vspace{0.7cm}

\noindent{\bf \Large{Introduction}}

\vspace{0.35cm}

In recent years, there has been some interest in (re)discovering and collecting different proofs of the Pythagorean theorem, 
much beyond the classical compilation \cite{Lo}. Here we propose a couple of new proofs, which come from interesting geometric 
configurations that have remained unnoticed by the community. Actually, these are two parametric families of proofs, each of which 
depends on an angle involved in the corresponding construction. 

For the first family, the parameter is the angle of inclination of a ziggurat. Here, by a {\em ziggurat} of side $\ell$ and 
angle $60^{\mathrm{o}} \leq \theta < 180^{\mathrm{o}}$ (an $(\ell,\theta)$-{\em ziggurat}, for short) we mean the isosceles 
trapezium having a basis and the two ``non-parallel'' sides of length $\ell$, with both angles on the basis equal to $\theta$. 

\vspace{-0.28cm}

\begin{center}
\includegraphics[width=5.5cm]{Zigurat1}
\end{center}

\vspace{-0.28cm}

Note that, for $\theta = 90^{\mathrm{o}}$, this becomes a square. Also, for $\theta = 60^{\mathrm{o}}$, this degenerates to an equilateral 
triangle. The first aim of this note is to give a direct proof (with no use of the Pythagorean theorem) of the following result.

\vspace{0.4cm}

\noindent{\bf Theorem A.}
Let $\triangle ABC$ be a triangle with a right angle at $C$ and sides $a,b$ and $c$ (opposed to $A,B$ and $C$, respectively). 
For each $60^{o} \leq \theta \leq 135^{o}$, the sum of the areas of the $(a,\theta)$ and $(b,\theta)$-ziggurats equals the area 
of the $(c,\theta)$-ziggurat.

\begin{center}
\includegraphics[width=8cm]{Zigurat2}
\end{center}

For $\theta = 90^{\mathrm{o}}$, this corresponds exactly to the Pythagorean theorem. However, for each angle $\theta$, the corresponding claim 
is also equivalent to it. Indeed, the area of the $(\ell,\theta)$-ziggurat equals some constant $C(\theta)$ times the area of the square of side $\ell$, 
so that dividing all the terms in the equality involving the areas of the ziggurats by $C(\theta)$, we recover the equality involving the 
squares.\footnote{At this point, it is worth mentioning that even in Euclid's {\em Elements} one can find such a remark. Specifically, 
Proposition 31 of Book VI states: ``In right-angled triangles, the figure on the side opposite to the right angle equals the sum of the 
similar and similarly described figures on the sides containing the right angle'', and this for any figure~! (of course, we know today 
that non-measurable figures are forbidden).} 
Here, the value of the (positive) constant $C(\theta)$ is irrelevant, yet it can explicitly (and easily) be computed:
 $C(\theta) = \sin (\theta) \, (1 - \cos (\theta)).$

If we consider ziggurats of an angle $90^{\mathrm{o}} < \theta \leq 135^{\mathrm{o}}$ and we extend its non parallel sides until they meet, we 
obtain isosceles triangles over the sides of our original triangle $\triangle ABC$, each having two angles equal to $180^{\mathrm{o}} - \theta$ at 
the basis. The area of such a triangle equals some universal constant $D(\theta)$ times that of the corresponding ziggurat. (Again, the precise 
value of $D(\theta)$ is irrelevant, yet it is easily seen to be equal to $\ell^2 / (4 \cos (\theta) (1-\cos (\theta)))$.)  We thus obtain the consequence 
of Theorem A below. To simplify the statement, we call $(\ell,\theta)$-pyramid an isosceles triangle with basis $\ell$ and angles $\theta$ at the basis. 

\vspace{0.4cm}

\noindent{\bf Theorem B.}
Let $\triangle ABC$ be a triangle with a right angle at $C$ and sides $a,b$ and $c$ (opposed to $A,B$ and $C$, respectively). For each 
$45^{\mathrm{o}} \leq \theta < 90^{\mathrm{o}}$, the sum of the areas of the $(a,\theta)$ and $(b,\theta)$-pyramids equals the area 
of the $(c,\theta)$-pyramid. 

\vspace{0.4cm}

In the last section of this note we give a direct proof of the statement above, without passing through ziggurats (and avoiding the use 
of the Pythagorean theorem). The reader will note that both proofs rely on a similar argument (coming from \cite{Na1}), namely, putting the 
polygonal piece (the ziggurat or the pyramid) on the hypotenuse toward the interior of the original triangle, and then looking for (scaled) 
rotated versions of the latter that naturally arise in the resulting configuration. 

\begin{center}
\includegraphics[width=6.5cm]{Zigurat3}
\end{center}

%%%%%%%%%%%%%%%%%%%%%%
\section{A general geometric configuration}

Perhaps more interesting than producing a new proof of the Pythagorean theorem is to create a new geometric configuration for it  
(we will come back to this point later). To accomplish such a task here using ziggurats, we 
fix an angle $60^{\mathrm{o}} \leq \theta \leq 135^{\mathrm{o}}$, and we build external ziggurats over the sides $a$ and $b$ and an internal 
ziggurat over the side $c$, all of them with angle $\theta$. Denote these ziggurats as $D' CA D''$, $E' CB E''$ and $F AB G$, as depicted 
below. Also, let $C'$ be such that $E' C'$ has length $b$ and is parallel to $D' C$. Note that $E'C'D'C$ is a parallelogram (which degenerates 
to a segment for $\theta = 135^{\mathrm{o}}$). We will denote $\alpha$ and $\beta$ the angles of $\triangle ABC$ at $A$ and $B$, respectively.
 
\vspace{-0.2cm}
 
\begin{center}
\includegraphics[width=8cm]{Zigurat4}
\end{center} 

 %\vspace{0.05cm}
 
 \noindent{\bf Lemma.} For $\theta \neq 90^{\mathrm{o}}$, the quadrilaterals $C'D'D''F$ and $C'E'E''G$ are parallelograms 
 (which degenerate to line segments for $\theta = 90^{\mathrm{o}}$).
 
 \vspace{0.35cm}
 
 \noindent{\em Proof.} The triangle $\triangle D''FA$ is a rotated copy of $\triangle ABC$ (it arises from a rotation by an angle $\theta$ about $A$). 
 Therefore, $FD''$ has length $a$,  which is also the length of $C'D'$. We claim that these two segments are also parallel (this implies that 
$C'D'D''F$ is a parallelogram, and the proof for $C'E'E''G$ is analogous). Indeed, this is easily checked by looking at the slopes of these  
segments with respect to $AB$. Namely, on the one hand, $C'D'$ is parallel to the segment $E'C$, whose line is obtained from that of $AB$ 
first by negatively rotating (at $B$) of an angle $\beta$ and later positively rotating (at $C$) of an angle $\theta$. Thus, the slope in this case 
is $\theta - \beta$. On the other hand, the line of $FD''$ is obtained from that of $AB$ first by positively rotating (at $A$) of an angle 
$\theta$ and later negatively rotating (at $F$) of an angle $\beta$. Thus, the slope in this case is again $\theta - \beta$.
$\hfill\square$
 
\vspace{-0.1cm} 
 
\begin{center}
\includegraphics[width=10cm]{Zigurat5}
\end{center}

\vspace{-0.2cm}

\noindent{\bf Observation.} In the proof above, we did not use the hypothesis that the angle at $C$ is a right angle. The 
lemma thus provides a very general configuration. Also, we did not use the restriction on the angle $\theta$, though  
for angles outside the range, the ziggurats may (auto-)intersect. We will come back to the latter point further on. 

%%%%%%%%%%%%%%%%%%%%%%
\section{The proof argument}

The proof of Theorem A will follow from computing the area of the polygon $ABE''GC'FD''$ (denoted by 
$\mathcal{P}$ in what follows) in two different ways.

On the one hand, $\mathcal{P}$ consists of the $(c,\theta)$-ziggurat plus the triangles $\triangle GBE'' $, $\triangle AFD''$ and 
$\triangle FGC'$. The first two of these are congruent to $\triangle ABC$, and the latter is similar to it, 
by the lemma above. Moreover, the similarity ratio is easily seen to be equal to $(1-2\cos(\theta))$. Therefore,  
$$area(\mathcal{P}) = area \big( (c,\theta)\mbox{-ziggurat} \big) + \big( 2 + (1-2\cos(\theta))^2 \big) \times area (\triangle ABC).$$

On the other hand, $\mathcal{P}$ consists of the $(a,\theta)$ and $(b,\theta)$-ziggurats plus the triangle $\triangle ABC$ and the 
parallelograms $C'D'D''F$, $C'E'E''G$ and $E'C'D'C$. Note that $C'D'D''F$ (resp. $C'E'E''G$) has sides of length $a$ and 
$b (1-2\cos(\theta))$ (resp. $a(1-2\cos (\theta))$ and $b$). Using that $\angle ACB = 90^{\mathrm{o}}$ and chasing angles, 
one easily concludes that 
$$area (E'C'D'C) = ab \, \sin (270^{\mathrm{o}} - \theta)$$
and 
$$area (C'D'D''F) = area (C'E'E''G) = ab \, (1-2 \cos (\theta)) \sin (270^{\mathrm{o}} - \theta).$$ 
Since $ab = 2 \, area (\triangle ABC)$, one obtains that $area(\mathcal{P})$ also equals 
the sum of the areas of the $(a,\theta)$ and $(b,\theta)$-ziggurats plus 
 $$\big( 1+ 4 \, (1-2\cos(\theta)) \sin (270^{\mathrm{o}} - \theta) 
 + 2 \sin (270^{\mathrm{o}} - 2 \theta)  \big) \times area (\triangle ABC).$$

A careful but elementary trigonometric manipulation gives 
\begin{small}
\begin{eqnarray*}
1+ 4 \, (1-2\cos(\theta)) \sin (270^{\mathrm{o}} - \theta) + 2 \sin (270^{\mathrm{o}} - 2 \theta)
&=& 
1 -  4 (1-2\cos(\theta)) \cos (\theta) - 2 \cos (2 \theta) \\
&=&
1 - 4 \cos (\theta) + 8 \cos^2(\theta)- 2 \, (2 \cos^2(\theta)-1) \\
&=&
3 - 4 \cos (\theta) + 4 \cos^2 (\theta) \\
&=& 
2 + (1-2\cos(\theta))^2. 
\end{eqnarray*}
\end{small}Finally, putting things together, this allows us to conclude that 
$$area \big( (c,\theta)\mbox{-ziggurat} \big)  
= 
area \big( (a,\theta)\mbox{-ziggurat} \big)  + area \big( (b,\theta)\mbox{-ziggurat} \big) ,$$
as claimed.

%%%%%%%%%%%%%%%%%%%%%%
\section{A short discussion on the use of trigonometry}
\label{sec-conf} 

The reader might be afraid of the trigonometric manipulations in the arguments above, 
as they seem to implicitly use the (trigonometric version of the) Pythagorean theorem. A careful check 
shows that  we used the definition of the trigonometric functions for acute angles, their extensions to arbitrary 
angles, as well as the formulae for the trigonometric functions of sums and differences of angles. 
However, in the last sequence of equalities, we also used the identity 
\begin{equation}\label{eq-problema}
\cos (2 \theta) = 2 \cos^2 (\theta) - 1.
\end{equation}
The problem is that this is based on the Pythagorean identity \, $\sin^2 (\theta) + \cos^2 (\theta) = 1$, 
because the natural formula that arises from the definitions is 
$$\cos (2 \theta) = cos^2 (\theta) - \sin^2 (\theta);$$ 
compare the Remark below.

In order to circumvent this circularity issue, we propose a ``proof (almost) without words'' 
of the identity (\ref{eq-problema}) based on the relation \, $\sin (2 \theta) = 2 \sin (\theta) \cos (\theta)$ \, 
(which is properly available without the use of the Pythagorean theorem; see \cite{Zi}). 
Namely, in the image below, $\triangle ABC$ is similar to $\triangle OB'C'$, and the equality 
$\frac{AB}{CB} = \frac{OB'}{B'C'}$
becomes 
$$1 + \cos (2 \theta) 
= AB 
= CB \cdot \frac{OB'}{B'C'} 
= \sin (2 \theta) \cdot \frac{\cos (\theta)}{\sin (\theta)},$$
from where the identity (\ref{eq-problema}) follows directly.

\vspace{-0.2cm}
\begin{center}
\includegraphics[width=8.4cm]{Zigurat6}
\end{center}
\vspace{-0.2cm}

It is worth pointing out that the previous argument gives a new trigonometric proof of the Pythagorean theorem, at least for angles no 
larger than $45^{\mathrm{o}}$ (it is not hard to modify the argument for angles between $45^{\mathrm{o}}$ and $90^{\mathrm{o}}$; 
alternatively, one can use the formulae for trigonometric functions of double angles). This is along the lines of \cite{Zi} and  
particularly \cite{Lu}. Since there has been a lot of activity in this direction in recent years (see for instance \cite{JJ}), we 
take this opportunity to include a remark that is somehow unrelated to our geometric construction. 

\vspace{0.25cm}

\noindent {\bf Remark.} If we want to show the Pythagorean theorem for acute triangles using trigonometry, 
a very simple argument would consist in just letting $\alpha' = \alpha$ in the identity 
\begin{equation}\label{eq-stupid}
\cos (\alpha - \alpha') = \cos (\alpha) \cos (\alpha') + \sin (\alpha) \sin (\alpha').
\end{equation}
However, as claimed in \cite{Zi}, this is not allowed, as the trigonometric ratios are not defined for the zero angle... Nevertheless, 
nothing prevents us from using a continuity argument, namely by varying $\alpha'$ to make it converge to $\alpha$ from below. 
It is very instructive to look at what happens geometrically when doing this. Indeed, a standard diagram that shows equality 
(\ref{eq-stupid}) is depicted on the left below (see the appendix of \cite{She} for more on this). When passing to the limit, 
this becomes the diagram on the right, from where the Pythagorean relation emerges very clearly once again 
(compare to the Proof 41 in \cite{Bo} as well as \cite{Shi}).

\begin{center}
\includegraphics[width=7.9cm]{Zigurat7} 
\quad
\includegraphics[width=7.9cm]{Zigurat8}
\end{center}

%%%%%%%%%%%%%%%%%%%%%%
\section{Further remarks on the geometric configuration}

We will not pursue the direction above since, as we suggested from the very beginning, our goal is not to 
show what are the limits of the different approaches, but rather to highlight new geometric configurations (compare 
\cite{Ma} and references therein). We hence prefer to go into a deeper analysis 
of the picture that arises for specific angles $\theta$. The reader can check that in all the cases described below, 
the trigonometric computations involved in the proof become elementary (in particular, the 
configurations can be drawn with a rule and a compass).

\vspace{0.5cm}

\noindent{\bf {\underline{$\theta = 90^{\mathrm{o}}$:}}} For this choice, the classical configuration below emerges. 
The proof argument then becomes very close (equal\,?) to that of the proofs listed as 24, 63, 69 and 70 in \cite{Bo}.

\begin{center}
\includegraphics[width=8cm]{Zigurat9}
\end{center}

\vspace{0.2cm}

\noindent{\bf {\underline{$\theta = 60^{\mathrm{o}}$:}}} For this choice, the configuration that arises and the proof argument 
are those developed in \cite{Na1} (see  \cite{Na2} for an animation). 
Note that the area of the parallelogram $CE'C'D'$ coincides with that of $\triangle ABC$. 
One directly concludes that the sum of the areas of the equilateral triangles over $a$ and $b$ equals the area of the 
equilateral triangle over $c$.

\begin{center}
\includegraphics[width=7cm]{Zigurat10}
\end{center}

\noindent{\bf {\underline{$\theta = 120^{\mathrm{o}}$:}}} For this choice, each ziggurat corresponds to half of the 
corresponding regular hexagon. Again, the area of the parallelogram $CE'C'D'$ coincides with that of $\triangle ABC$, 
while the areas of each $C'D'D''F$ and $C'E'E''G$  are twice this. The configuration shows that the sum of the areas of 
the regular hexagons over $a$ and $b$ equal the area of the regular hexagon over $c$ (obviously, this also follows 
directly from the case $\theta = 60^{\mathrm{o}}$ above). 

\begin{center}
\includegraphics[width=8cm]{Zigurat11}
\end{center}


\noindent{\bf {\underline{$\theta = 135^{\mathrm{o}}$:}}} For this choice, each ziggurat corresponds to a precise fraction (the ``basis'')  
of the corresponding regular octagon. One can readily check that the equality of areas that arose in the proof of Theorem A 
boils down to the elementary algebraic identity below:
$$2 + (1+\sqrt{2})^2 \, = \, 1 + 2 \,  \sqrt{2} \,  (1+\sqrt{2}).$$
Again, the argument yields a ``direct proof'' of the fact that 
the sum of the areas of the regular octagons over $a$ and $b$ equals the area of 
the regular octagon over $c$.

\begin{center}
\includegraphics[width=8cm]{Zigurat12}
\end{center}


\noindent{\bf {\underline{$\theta = 108^{\mathrm{o}}$:}}} For this choice, each ziggurat corresponds to the ``basis'' of a regular pentagon. 
Having in mind that the length of the diagonal of such a pentagon equals $\varphi$ times the length of its side (where $\varphi$ denotes 
the golden mean $(1+\sqrt{5})/2$), it is straightforward to check that the equality of areas in the proof comes from an 
algebraic relation that ultimately reduces to $\varphi^2 = 1 + \varphi$. We leave this task to the reader. (Hint: use that
 $\sin (18^{\mathrm{o}}) = 1/2\varphi$ and $\cos (36^{\mathrm{o}}) = \varphi / 2$.)

The fraction of the total area covered by the basis of a regular pentagon is easily seen to be equal to $(1+\varphi)/(2+\varphi)$. With this, 
the argument provides once again a ``direct proof'' of the fact that the sum of the areas of the regular pentagons over $a$ and $b$ equals the 
area of the regular pentagon over $c$. We will come back to this point later.

\begin{center}
\includegraphics[width=7cm]{Zigurat13}
\end{center}

It is not difficult to produce a  Geogebra applet  to explore the configuration. Fixing the segment $AB$, one may 
play with two free parameters: the point $C$ (that moves along the circle of diameter $AB$) and the angle $\theta$. 
The reader will notice that one can relax the restriction $60^{\mathrm{o}} \leq \theta \leq \mathrm{135}^{\mathrm{o}}$ 
letting $\theta$ vary arbitrarily. 
Of course, for the ``forbidden'' values of $\theta$, the configuration degenerates: for $\theta < 60^{\mathrm{o}}$,  
all the ziggurats have self-intersections, and for $\theta > 135^{\mathrm{o}}$, the ziggurats over $a$ and $b$ intersect. Despite this, 
the configuration that arises still allows one to prove the Pythagorean theorem, but with a more careful analysis. A good way to proceed 
is by using coordinate systems and/or plane geometry via complex numbers. We leave to the reader the task of further exploring 
these degenerate configurations; see also the Remark in the last section.
%A Geogebra applet  to explore the configuration will remain available at XXX 
%In there, the segment $AB$ is fixed, and there are two free parameters: the point $C$ (that moves along 
%the circle of diameter $AB$) and the angle $\theta$. 
%The reader will notice that we do not impose the restriction 
%$60^{\mathrm{o}} \leq \theta \leq \mathrm{135}^{\mathrm{o}}$, but we let $\theta$ vary arbitrarily. 

%%%%%%%%%%%%%%%%%%%%%%%%%%%%%%%%%%%%%%%%%%%%%%%%%%%%%%%%%%%%%%%%%%%%%%%%%%%%%

\section{Pyramids instead of ziggurats}

To give a direct proof of Theorem B, let us draw external pyramids over the sides $a$ and $b$ and an internal one over the side $c$. Let us 
denote by $A'$, $B'$ and $C'$ the corresponding vertices, as drawn below. Note that the length of the equal sides of the pyramid 
over $a$ (resp. $b$, $c$) equals $a$ (resp. $b$, $c$) times the constant factor $r = r_{\theta}= 1 / (2 \cos (\theta))$. If we positively rotate 
$\triangle ABC$ about $A$ and then apply a homothety of ratio $r$, we obtain $\triangle AC'B'$. In particular, the length of $B'C'$ equals $ra$.
Analogously, if we negatively rotate $\triangle ABC$ about $B$ and later we apply a homothety of ratio $r$, we obtain $\triangle C'BA'$. Hence, the 
length of $C'A'$ equals $rb$. As a consequence, $CA'C'B'$ has opposite sides of equal length, and therefore it is a parallelogram (which 
degenerates to a segment for a specific value of $\theta$). Note that so far we have not used the fact that the angle at $C$ is a right angle.

Let us now definitively introduce the value $90^{\mathrm{o}}$ for the angle at $C$, and let us chase angles. As before, we can think of 
the polygon $\mathcal{P} = ABA'C'B'$ in two different ways:

\noindent - As the union of $\triangle ABC$, the $(a,\theta)$ and $(b,\theta)$-pyramids, and the parallelogram $CA'C'B'$. As such, 
its area equals the sum of the areas of the two small pyramids plus 
$$\frac{ab}{2} + ra \cdot rb \cdot \sin (2 \theta - 90^{\mathrm{o}}) = ab \left( \frac{1}{2} - r^2 \cos (2 \theta) \right).$$

\noindent - As the union of the $(c,\theta)$-pyramid and the triangles $\triangle AC'B'$ and $\triangle B A' C'$. Viewed this way, its area 
equals that of the $(c,\theta)$-pyramid plus 
$$2 \, \frac{ra \cdot rb}{2} = r^2 ab.$$

\noindent Using that $r = 1 / (2 \cos (\theta)))$, one easily checks that the last two expressions above coincide, thus establishing the result. Note 
that, again, one is forced to use the trigonometric identity $\cos (2\theta) = 2\cos^2 (\theta) - 1$ in this computation.

\begin{center}
\includegraphics[width=7cm]{Zigurat14}
\end{center} 

%\vspace{0.2cm}

\noindent{\bf Remark.} The idea of similarities can also be exploited in the context of ziggurats provided they incorporate a new parameter, 
namely the ratio $r$ between the non-parallel sides and the basis. The reader can easily handle the resulting configuration, which still 
leads to a proof of the Pythagorean theorem.

\vspace{0.3cm}

As for the case of the ziggurats, there are some configurations arising from the argument above that we would like to highlight. 

\vspace{0.2cm}

\noindent{\bf {\underline{$\theta = 45^{\mathrm{o}}$:}}} For this choice, we obtain the configuration below, which is along the lines 
of the classical ones, yet it is slightly different. Actually, the proof that arises seems to be new, yet it is very close to Proof 64 in \cite{Bo}.

\begin{center}
\includegraphics[width=9cm]{Zigurat15}
\end{center} 

\vspace{0.2cm}

\noindent{\bf {\underline{$\theta = 60^{\mathrm{o}}$:}}} Nothing is new here, as an $(\ell,60^{\mathrm{o}})$-pyramid is the 
same as an $(\ell,60^{\mathrm{o}})$-ziggurat. We retrieve again the configuration and the proof argument from \cite{Na1}. 

\vspace{0.2cm}

\noindent{\bf {\underline{$\theta = 72^{\mathrm{o}}$:}}} For this choice, the $(\ell,72^{\mathrm{o}})$-pyramid is a golden triangle (with basis 
$\ell$).  The argument above then ``directly'' establishes that the sum of the areas of the golden triangles over $a$ and $b$ equals the area 
of the golden triangle over $c$. Now, noting that a regular pentagon is the union of two basic ziggurats minus a ``central'' golden triangle, 
this allows establishing the same statement for the regular pentagons over the sides $a,b,c$ in a ``more transparent'' way 
than the one exhibited just using the ziggurats of angle $108^{\mathrm{o}}$. It may be interesting to explore whether this can 
lead to a more natural proof just using a dissection argument (without relying on the Bolyai-Gerwien-Wallace theorem). 

\vspace{0.4cm}

\noindent{\bf Acknowledgments.} 
I'm indebted to Hugo Caerols, Eduardo Reyes and Michele Triestino for discussions and corrections, 
and to Patrick Popescu-Pampu for his interest and for pointing out to me the reference \cite{Ma}. 
I would like to thank all the school students participating in the ``Entrenamiento Matem\'atico Ol\'impico'' 
at USACH for creating a pleasant atmosphere for thinking about the content of this Note.  

This work was supported by the DICYT Regular Project 042532NF$_{-}$REG ``On the growth of certain 
dynamical cocycles'' (Vicerrector\'ia de Investigaci\'on, Innovaci\'on y Creaci\'on VRIC). In particular, the images 
were produced by Juan Pablo Pincheira, a Master student at USACH. Also, the YouTube video linked 
in \cite{Na2} was produced (several years ago) by Nicolé Geyssel, at that time also a Master student at 
USACH. I would like to express my warm gratitude to both of them.

%%%%%%%%%%%%%%%%%%%%%%%%%%%%%%%%%%%%%%%%%%%%%%%%%%%%%%%%%%%%%%%%%%%%%%%%%%%%%

\begin{footnotesize}

\begin{thebibliography}{Dillo 83}

\bibitem[Bo]{Bo} {\sc A. Bogomolny.} {\em Cut the knot: Pythagorean theorem.}  
Personal academic blog: https://www.cut-the-knot.org/pythagoras/

\bibitem[JJ]{JJ} {\sc N. Jackson \& C. Johnson.} 
Five or ten new proofs of the Pythagorean theorem.  
{\em The American Math. Monthly} {\bf 131}, no. {\bf 9} (2024), 739-752.

\bibitem[Lo]{Lo} {\sc E. Loomis.} {\em The Pythagorean Proposition.}  Rutgers University (1927). 

\bibitem[Lu]{Lu} {\sc N. Luzia.} Other trigonometric proofs of Pythagoras theorem. 
Preprint (2015); arXiv:1502.06628.

\bibitem[Ma]{Ma} {\sc M. Maex.} 
A synthetic proof of the spherical and hyperbolic Pythagorean theorem on models in Euclidean and Minkowski space. 
Preprint (2025); arXiv:2509.03314.

\bibitem[Na1]{Na1} {\sc A. Navas.} 
The Pythagorean theorem via equilateral triangles. 
{\em Mathematics Magazine} {\bf 93}, no. {\bf 5} (2020), 343-346.

\bibitem[Na2]{Na2} {\sc A. Navas.} 
Encore une preuve du th\'eor\`eme de Pythagore. 
{\em Images des Math\'ematiques} (2019).

\bibitem[She]{She} {\sc C. Shene.} 
A new approach to proving the pythagorean theorem. 
Preprint (2023); available at https://pages.mtu.edu/~shene/VIDEOS/GEOMETRY/004-Pythagorean-Thm/Pythagorean.pdf.

\bibitem[Shi]{Shi} {\sc S. Shirali.} 
Two new proofs of the Pythagorean theorem, part i. 
{\em At Right Angles} (2023), 7–12.

\bibitem[Zi]{Zi} {\sc J. Zimba.} 
On the possibility of trigonometric proofs of the Pythagorean theorem. 
{\em Forum geometricorum} {\bf 9} (2009),  275-278.

 \end{thebibliography}

\vspace{0.1cm}

\noindent Andr\'es Navas \\ 

\noindent Departamento de Matem\'atica y Ciencia de la Computaci\'on\\

\noindent Universidad de Santiago de Chile (USACH)\\ 

\noindent Alameda 3363, Santiago, Chile\\ 

\noindent email: andres.navas@usach.cl

\end{footnotesize}

\end{document}

%%%%%%%%%%%%%%%%%%%%%%%%%%%%%%%%%%%%%%%%%%%%%%%%%%%%%%%%%%%%%%%%%%%%%%%%%%%%%%%%%%%%%%%%%