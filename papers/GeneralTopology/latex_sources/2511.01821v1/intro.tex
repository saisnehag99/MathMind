\section{Introduction}

\subsection{Context} 
Contact manifolds are smooth manifolds equipped with a maximally integral distribution, their contact structure. Although such manifolds appear in the literature as early as \cite{Lie72}, the systematic study of contact manifolds is younger than that of symplectic manifolds. While any contact manifold $(Y,\xi)$ is odd-dimensional, given a contact form $\lambda$ we can define the symplectization of $(Y,\lambda)$ to be the product $(\bR\times Y,d(e^s\lambda))$. While the symplectization is not a complete invariant of contact manifolds, \cite{Cou14}, it allows many tools from symplectic geometry to be imported into contact topology, in particular, the theory of pseudo-holomorphic curves.
Early applications were constructions of symplectic capacities, \cite{EH87}, and the Weinstein conjecture in dimension $3$, \cite{Hof93}.\par 
Symplectic field theory is the ambitious vision of \cite{EGH00}, proposing an intricate algebraic framework based on moduli spaces of punctured pseudo-holomorphic curves. We refer to \cite{HS24} for a survey on the history and possible application of SFT; we could not do them justice here.
The main problem in realizing the SFT framework is the lack of transversality of the relevant moduli spaces. The basic theory and estimates were worked out in \cite{HWZI,HWZII}, while compactness was established in \cite{BEH03,CM05}. The goal of a uniform theory to deal with the transversality issues motivated the development of polyfolds, \cite{HWZ21,FFGW16,FH18}. A construction of Kuranishi charts for all genera was given by \cite{Ish18}, but utilizing them to obtain algebraic invariants is yet to be done.\par


In genus zero, considering curves with one positive puncture, SFT postulates the existence of a Floer homology theory, called \emph{contact homology}, with generators given by Reeb orbits. In \cite{BH18} and \cite{BH23}, Bao--Honda give constructions that are suitable for computations, cf. \cite{Avd23}, while \cite{Par19} gives a more abstract construction of contact homology using the framework developed in \cite{Par16}. However, as in Hamiltonian Floer theory, contact homology uses only the information of rigid moduli spaces of curves, i.e., those of dimension zero, and of the existence of suitable moduli spaces in dimension one. Floer homotopy theory has the goal of extracting invariants also from higher-dimensional moduli spaces. It was proposed by \cite{CJS95} in 1995 and reworked in the Morse--Bott context by \cite{Z24,CK23,Bon24}. Both \cite{LT18} and \cite{AB24} give different approaches, respectively, the latter placing greater emphasis on bordism theories. Flow categories and the associated homotopical structures have been constructed in Hamiltonian Floer theory, \cite{BX22,Rez22}, symplectic cohomology, \cite{Rez24,CK23}, Lagrangian Floer theory, \cite{Lar21,PS24,BC25} as well as \cite{LS14} in a somewhat different context. Apart from \cite{TT25}, which uses generating families instead of pseudo-holomorphic curves, no Floer homotopy theoretic constructions have been given in the context of contact topology. 


 
\subsection{Main results}
The construction of our flow category relies fundamentally on the global Kuranishi charts for moduli spaces of punctured pseudo-holomorphic curves that we build in \textsection\ref{sec:prelim-aux}. To keep the notation light here, we summarize the background on contact topology in \textsection\ref{subsec:background} and define our pseudo-holomorphic curves and their degenerations in \textsection\ref{subsec:buildings}. For this introduction, let us simply say that we consider the SFT compactifications $\Mbar^{\,J}_{\sft}(\Gamma^+,\Gamma^-;\beta)$ of moduli spaces of genus-zero $J$-holomorphic curves of relative homology class $\beta$ in the symplectisation of a contact manifold $(Y,\lambda)$ that are asymptotic at their positive and negative punctures to Reeb orbits in $\Gamma^+$ and $\Gamma^-$ respectively. 

\begin{intthm}[Theorem~\ref{thm:leveled-gkc}]\label{prop:gkc-exists}
    Let $(Y,\xi)$ be a closed contact manifold equipped with a non-degenerate contact form $\lambda$. Suppose $J$ is a $\lambda$-adapted almost complex structure on $(Y,\xi)$ and $\Gamma^\pm$ are finite sequences of Reeb orbits.
    \begin{enumerate}[ \normalfont 1),leftmargin=20pt,ref=\arabic*]
        \item The moduli space $\Mbar^{\,J}_{\sft}(\Gamma^+,\Gamma^-;\beta)$ admits a global Kuranishi chart with corners $\cK$ of the correct virtual dimension.
        \item If $\Gamma^+$ and $\Gamma^-$ consist of good Reeb orbits, then the orientation line $\fo_\cK$ of $\cK$ is canonically isomorphic to $\fo(\bR)\dul\otimes\bigotimes\limits_{\gamma\in \Gamma^+}\fo_{\gamma}\otimes\bigotimes\limits_{\gamma\in \Gamma^-}\fo\dul_{\gamma}$.
    \end{enumerate}
\end{intthm}

While much of the construction is similar to previous constructions of global Kuranishi charts, the crucial first step is different: the construction of very ample line bundles on the closed domain curves. The original idea in \cite{AMS21}, somewhat rephrased in \cite{HS22} and \cite{AMS23}, is to construct a very ample line bundle on the domain of pseudo-holomorphic curve by pulling back a sufficiently positive differential form, respectively, line bundle on the target manifold. In the contact case, where $Y$ could have vanishing $H^2$ and such a line bundle could only be pulled back to the punctured domain, this is not an option. However, in genus zero, all that is needed is the assignment of a positive degree to each irreducible component of a domain so that these degrees add correctly when a node is smoothed. This is carried out in \textsection\ref{sssec:framings_of_buildings}.

\begin{remark}[Technical remark]
    The proof of Theorem~\ref{prop:gkc-exists} passes through the moduli spaces considered in \cite{Par19}. In contrast to the compactifications usually considered in the SFT literature, these curves do not come with levels, that is, one quotients the maps on each irreducible component separately by the translation action on the target. In the process, we give a geometric way of recovering the usual moduli spaces considered in SFT from those used by \cite{Par19}, which may be of independent interest.
\end{remark}

Theorem~\ref{prop:gkc-exists} does not assume $\Gamma^+$ to be a single Reeb orbit. Hence, our construction yields the foundations necessary for rational SFT as outlined in \cite{Lat22} and used in \cite{Sie19,MZ20,MZ23}. We do not pursue this direction and instead establish that (at least genus zero) SFT moduli spaces fit into the flow category framework of \cite{AB24}. However, we have to generalize their definition of flow category in two ways. First, we allow the objects to be orbifolds and replace the cartesian product (in the definition of the compositions) by a form of fiber product. Secondly, the flow category comes with symmetric actions on the objects and morphism spaces. The details can be found in \textsection\ref{sec:sliced-flow-cats}. Restricting to the case of points and trivial symmetric actions recovers the classical definition. Tangential structures such as stable complex structures or framings can be defined for this definition of flow categories exactly as in \cite{AB24} and are discussed in \textsection\ref{subsec:sliced-flow-cat}. In particular, we show that these flow categories are the objects of a stable $\infty$-category (Theorem~\ref{prop:flow-stable}), extending one of the main results of \cite{AB24} to our setting. We expect that most of the flow category constructions in the literature, such as \cite{PS25}, can be adapted.

\begin{remark}
     Replacing the objects of the flow category by orbifolds was proposed by Mohammed Abouzaid. It differs from the definition of a Morse--Bott flow category in \cite{Z24,CK23,Bon24} in the way the composition maps are defined and because the group action depends on the object. This new definition appears naturally in our setup because we equip our curves with asymptotic markers at each puncture but do not constrain the markers to be mapped to a fixed base point on the Reeb orbit. Curiously enough, not choosing a base point is almost forced on us by the global Kuranishi chart construction.
\end{remark}



Given a contact manifold $(Y,\xi)$ with non-degenerate contact form $\lambda$, define $\cP$ to be the set of finite sequences of Reeb orbits. Given $L > 0$, we let $\cP_{\le L}\sub \cP$ be the subset of sequences where each element has action $\leq L$. We identify a Reeb orbit $\gamma$ with the associated orbifold $B\gamma$ given by the manifold 
$$E\gamma = \{\wt\gamma\mid \wt\gamma \text{ a constant-speed parametrization of }\gamma\}$$
equipped with the $S^1$-action that rotates the domain. This orbifold is equivalent to $[*/\bZ_m]$, where $m$ is the multiplicity of the Reeb orbit. To a sequence $\Gamma = (\gamma_1,\dots,\gamma_k)$ of Reeb orbits, we associate the product $B\Gamma = B\gamma_1\times \dots\times B\gamma_k$.

\begin{intthm}\label{thm:sft-flow-category} For any choice of $\lambda$-adapted almost complex structure $J$ there exists a stably complex rel--$C^1$ flow category $\scM^{Y,\lambda}$ with objects given by finite sequences of Reeb orbits and morphism spaces based on moduli spaces of punctured $J$-holomorphic curves.
\end{intthm}

We emphasize that we obtain a stable complex structure on $\scM^{Y,\lambda}$ despite including bad Reeb orbits in the objects of our symmetric flow category. The non-orientability of these Reeb orbits is compensated for by certain index bundles that are part of the stable complex structure; see \S\ref{subsec:lift-of-objects}. Theorem~\ref{thm:sft-flow-category} follows from a telescope construction using Theorem~\ref{thm:flow-cat} and Proposition~\ref{prop:colimit-of-flow-cats}. Concretely, in Theorem~\ref{thm:flow-cat} and Theorem~\ref{thm:stable-complex-structure}, we construct a symmetric flow category $\scM^{Y,\lambda}_{\leq L}$ and its stably complex lift with objects given by $\cP_{\le L}$. Proposition~\ref{prop:colimit-of-flow-cats} then asserts that for $L < L'$, we can find a stably complex symmetric bimodule from $\scM_{\leq L}^{Y,\lambda}$ to $\scM_{\leq L}^{Y,\lambda}$.

The flow category we construct should be regarded as an enhancement of contact homology after forgetting the natural dga structure on contact homology. A sketch of the relation is given in \S\ref{subsec:contact homology}. We expect that the dga structure can be encoded by utilizing the natural symmetric monoidal structure of concatenation on the objects $\scM^{Y,\lambda}_{\le L}$ but do not pursue it in this paper. 

\begin{remark}
    Instead of the flow category constructed here, one could use Theorem~\ref{prop:gkc-exists} to build a contact flow multi-category, where the objects are Reeb orbits, while moduli spaces of curves with one positive puncture and multiple negative punctures constitute the multi-morphisms. Since the foundational aspects of flow multi-categories are still under development, we chose to pursue a bar-construction-style flow category for contact manifolds.  
\end{remark}

\begin{remark} 
The global Kuranishi chart of $\Mbar^J_{\sft}(\Gamma^+,\Gamma^-;\beta)$ depends on a choice of \emph{perturbation datum}, see Definition~\ref{de:auxiliary-datum}, similar to other constructions of global Kuranishi charts, \cite{AMS21,HS22,BX22,Rez22}. One key step of Theorem~\ref{thm:sft-flow-category} is an inductive construction of such perturbation data, following \cite{BX22}. However, by carefully choosing these data and not using smoothing theory, we can avoid some of the additional steps.
\end{remark}

% \begin{remark} We expect that $\scM^{Y,\lambda}$ admits a lift to a framed flow category whenever $c_1(\xi) = 0$.\end{remark}

Restricting to cylinders, we can drop the restriction on the action in geometrically nice cases.

\begin{corollary}
    Suppose $(Y,\lambda)$ has no contractible Reeb orbits. Then, there exists a \emph{cylindrical contact flow category} $\scM^{cyl}$ with objects given by \emph{all} Reeb orbits of $\lambda$ and morphism spaces given by compactifications of moduli spaces of cylinders.
\end{corollary}

Let now $(\wh X,d\lambda)$ be an exact symplectic cobordism from $(Y^-,\lambda^-)$ to $(Y^+,\lambda^+)$ so that the primitive $\lambda$ is of the form $e^{\pm s}\lambda^\pm$ near the respective end of $\wh X$. Let $J$ be an $\omega$-compatible almost complex structure on the completion of $X$ so that $J = J^\pm$ over the completed ends for some cylindrical almost complex structure. 

\begin{intthm}[{Theorem~\ref{thm:flo_bim}}] Given an action bound $L$, any exact symplectic cobordism induces a rel--$C^1$ flow bimodule $\scN^{\wh X}$ from $\scM^{Y^-,\hspace{-0.2pt}\lambda^-}_{\le L}$ to $\scM^{Y^+,\hspace{-0.2pt}\lambda^+}_{\leq L}$.
\end{intthm}

\begin{corollary}[{Lemma~\ref{lem:trivial-cobordism-diagonal-bimodule}}]\label{cor:diagonal-bimodule}
     If $(\wh X,\omega)$ is the trivial cobordism $(\wh Y,d(e^s\lambda))$, then $\scN^{\wh X}$ is equivalent to the diagonal bimodule given suitable choices of auxiliary data.
\end{corollary}

The diagonal bimodule should be thought of as the identity morphism in $\infty$-category $\Flow^\Sigma$. Thus, the corollary can be seen as a (weak) naturality statement of our construction.

To prove invariance up to equivalence of the choice of contact structure and almost complex structure of the flow category $\scM^{Y,\lambda}$, one would have to further prove the flow-categorical analogs of the composition of chain homotopies and invariance under deformations using flow bordisms. This is delegated to future work.


\subsection*{Acknowledgments}
The authors thank Mohammed Abouzaid, John Pardon, and Chris Woodward for valuable input and discussions. They are grateful to Russell Avdek, Kristen Hendricks, Roman Krutowski, and Noah Porcelli for comments on an earlier draft.
 S.C. thanks Julian Chaidez, Sheel Ganatra, Srijan Ghosh, Eric Kilgore, Mohan Swaminathan, Kyler Siegel and Sushmita Venugopalan for valuable discussions. S.C. is grateful to Chris Kottke for clarifications on fibers of generalized blowup maps. S.C. also thanks the Max Planck Institute for Mathematics for its hospitality. A.H. is grateful to Russell Avdek for discussions on contact topology and SFT, to Janko Latschev for explaining his understanding of the algebraic framework of SFT, and to Nick Sheridan for sharing his method to deal with orientation signs.\par 
\noindent A.H. is supported by ERC grant ROGW, No. 864919. 

