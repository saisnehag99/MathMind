\section{Extensions and Outlook}

After our complete classification of $\bar{K}$-algebras of Krull dimension 0 up to rank~$6$ and our partial results on $\bar{K}$-algebras of Krull dimension 0 and higher rank in Section \ref{ClassificationIdealChomp}, we will provide explicit examples and ideas on how to approach the Ideal Chomp Game on $\bar{K}$-algebras of higher Krull dimensions. Furthermore, we will state questions that are still open to the author's best knowledge along the way.

\subsection{Henson's Theorem}
\label{sec:henson}

In 1970 Henson solved the Ideal Chomp Game on a class of rings that are close to being principal ideal domains, extending the result that player A has a winning strategy on principal ideal domains that are not fields (Proposition \ref{principalmaxidealwin}) to these rings. We discuss this result and give an example of a class of rings on which the Ideal Chomp Game is solved using Henson's result. Furthermore, we discuss its limitations.

\begin{Def}
\label{hensoncondition}
    A ring $R$ satisfies Henson's condition (with respect to $a\in R$) if $R$ is an integral domain and there exists an element $a \in R$ such that $R/(a)$ is a PID, but not a field.
\end{Def}

\begin{Thm}
[Henson's Theorem]
\label{hensontheorem}
    Let $R$ be a ring satisfying Henson's condition. Then the first player has a winning strategy in
    the Ideal Chomp Game on $R$, beginning with the move $a^2$.
\end{Thm}

\begin{skproof}
    The main idea of the proof is to define so-called \emph{special ideals}. These are ideals $I \ideal R/(a^2)$ that can be written as a power of some maximal ideal, i.e. $I = \Fm^b$ for some $b \in \BN_{\geq 1}$. Using some more technical lemmata, one shows that player A can always reach such a special ideal when playing correctly while prohibiting player B from reaching such a special ideal. Since any maximal ideal is in particular special, player A will be the player to build some maximal ideal. Hence, player B loses the game. The complete proof can be found in \cite{henson}.
\end{skproof}

Some examples of rings that satisfy Henson's condition w.r.t. $x$ are $K[x,y]$ for an arbitrary field $K$, $\BZ[x]$ or more generally $R[x]$ for any principal ideal domain $R$.

We now discuss an explicit winning strategy for player A on $\bar{K}[x,y]$. The needed background on primary ideals can be found in \cite{atiyah-macdonald}.

\begin{Ex}[A winning strategy for player A on $\bar{K}\lbrack x,y\rbrack$]
    Player A starts by playing $a_1 = x^2 \in \bar{K}[x,y]$ according to Henson's Theorem \ref{hensontheorem}. Player $B$ then plays $$a_2 = f(x,y) = p(y) + x\cdot q(y) \in \bar{K}[x,y]/ (x^2).$$
    Let $R = \bar{K}[x,y]/(x^2)$, for all $n \geq 2$ the ideal $I_n = (a_2, \dots,a_n) \ideal R$ is always understood as an ideal in $R$ and all the $a_i$'s satisfy $a_i \in R \setminus (a_2,\dots, a_{i-1})$.\\
    Notice that all maximal ideals of $R$ are of the form $\Fm_b = (x,y-b)$ for some $b \in \bar{K}$. We call an ideal \emph{special} if it is of the form $\Fm_b^k$ for some $b \in \bar{K}$ and some $k \geq 1$. The special ideals in $R$ are precisely of the form $$\Fm_b^k = 
        ((y-b)^{k}, x(y-b)^{k-1}).$$

    In particular, $I_2 = (a_2) = (f)$ is not special as it is principal while all special ideals are not.

    Let $$f(x,y) = p(y) + x \cdot q(y) = c_p \cdot \prod_{i=1}^n (y - b_i)^{s_i} + x \cdot c_q \cdot \prod_{i = 1}^n (y - b_i)^{t_i}$$ where $c_p, c_q \in \bar{K}$. We discuss a special case where some primary ideal over $f$ is easy to find. The general approach to finding a response of player A to a polynomial player B plays can be found in \cite{henson}.

    Suppose there exists $b_i$ with $y-b_i \mid p,q$. Let $r = \min\{s_i,t_i\}$. Then the $\Fm_{b_i}$-primary ideal over $f$ is $\left((y-b_i)^{r}\right)$ and player A can choose $$a_3 = (y-b_i)^r + x(y-b_i)^{r-1}$$ to build the ideal $$I_3 = (f,a_3) = \Fm_{b_i}^r.$$ Hence, player A was able to build a special ideal. Player A continues to do so in the following rounds and finally wins the game.

\end{Ex}


Let us discuss one more explicit example:

\begin{Ex}
    
    Let player A start with $a_1 = x^2$ and player B play $a_2 = x(y-1) + (y-2)$. Then player A will respond with $a_3 = y-2$ which yields the ideal $$I_3 = (x(y-1), (y-2), x^2) = (x, y-2),$$ since $x = x(y-1) - x(y-2) \in I_3$. As this is a maximal ideal, player A wins.
\end{Ex}

Notice, however, that the use of Henson's theorem is limited when exploring the Ideal Chomp Game on higher dimensional rings:

\begin{Lem}
    Let $R$ be a Noetherian local ring of Krull dimension larger than or equal to three. Then $R$ does not satisfy Henson's condition.
\end{Lem}

\begin{myproof}
    By corollary 11.18 from \cite{atiyah-macdonald}, the dimension of $R/(x)$ for $x$ in the unique maximal ideal $\Fm \ideal R$ is $$\dim(R/(x)) = \dim(R) - 1.$$ Since principal ideal domains are of Krull dimension less than or equal to $1$, we conclude.
\end{myproof}

\subsection{The Ideal Chomp Game on higher dimensional $\bar{K}$-Algebras}
\label{weiteresKalgebren}

We already classified a large portion of Ideal Chomp Games on $\bar{K}$-algebras of Krull dimension 0 in Theorem \ref{ClassificationThm}. What remains towards a complete classification is a classification of the Ideal Chomp Game on local $\bar{K}$-algebras of finite rank $\geq 7$ and of local $\bar{K}$-algebras of Krull dimension~$\geq 1$. In the following, we discuss some partial results in the Ideal Chomp Game on $\bar{K}$-algebras with Krull dimension~$\geq 1$.

\subsubsection{$\bar{K}$-Algebras of Krull Dimension 1}

Besides Henson's Theorem (see Theorem \ref{hensontheorem}), there is another result on winning strategies in the Ideal Chomp Game on $\bar{K}$-algebras of Krull dimension 1:

\begin{Prop}
\label{ellipticCurves}
    Let $f = 0$ be an affine Weierstrass equation in $\bar{K}[x,y]$. Then player B has a winning strategy on $\bar{K}[x,y]/(f)$.
\end{Prop}

\begin{myproof}
    See Brandenburg's Paper ``Algebraic games -- Playing with groups and rings'' (\cite{brandenburg}), Proposition 5.7.
\end{myproof}

For $\bar{K}$-algebras of Krull dimension 1, a classification of the Ideal Chomp Game on them is missing.

\begin{Ques}
    Can we classify the Ideal Chomp Game on $\bar{K}$-algebras of Krull dimension 1?
\end{Ques}

\subsubsection{$\bar{K}$-Algebras of Krull Dimension 2}

For the Ideal Chomp Game on $\bar{K}$-algebras of Krull Dimension 2 we don't have any results that classify the game on a larger portion of these algebras. Hence, we will just look at one specific example.

\begin{Ex}
    In the Ideal Chomp Game on $\bar{K}[x,y]$ player A has a winning strategy. This can be obtained from both Henson's Theorem (see Theorem \ref{hensontheorem}) and the proposition on winning strategies on elliptic curves (see Proposition \ref{ellipticCurves}).
\end{Ex}

For most Ideal Chomp Games on $\bar{K}$-algebras, or more generally on most rings, of Krull dimension 2 it is not known which player has a winning strategy. In particular, the following question is open for any choice of your favorite ring that does not satisfy Henson's condition or is a non-trivial Cartesian product of rings, e.g. for $K[x,y,z]/(z^{a})$ for $a \in \BN_{\geq 2}$.

\begin{Ques}
    Who has a winning strategy in the Ideal Chomp Game on your favorite (local) $\bar{K}$-algebra/ring of Krull dimension 2? 
\end{Ques}

\subsubsection{$\bar{K}$-Algebras of Krull Dimension $\geq 3$}
\label{sec:KDgeq3}

By Proposition \ref{CartesianProdCor} player A has a winning strategy on rings of the form $R = \prod_{i=1}^{n} R_i$ with $R_i$ of Krull dimension 1 and $n \geq 2$. These rings are of Krull dimension~n. In particular, there are examples of rings of Krull dimension n where we know which player has a winning strategy. For player A we even have examples of rings for any given Krull dimension $\geq 1$ that cannot be written as a nontrivial Cartesian product of rings and where player A has a winning strategy:

\begin{Ex}
    Given $n \in \BN$, consider the scheme $S$ we receive by gluing together the affine space $\BA^{n-1}$ with $\BA^1$ at the origin in $\BA^n$. Then the coordinate ring $R \subseteq \bar{K}[z_1,z_2,z_3,z_4]$ of $S$ cannot be written as a nontrivial Cartesian product since $S$ is connected. Furthermore, player A has the winning strategy to play $z_4 - 1$ in his first move to reduce $R$ to $R/(z_4-1) \cong K$ where the isomorphism follows from the fact that $S \cap \{z_4 = 1\} = \{(0,0,0,1)\}$ is a single point.
\end{Ex}

Since there are so many ways for the first player to reduce $\bar{K}[x_1,\dots,x_n]$ to a $\bar{K}$-algebra of Krull dimension $n-1$, we conjecture the following regarding this open question:
\begin{Conj}
\label{IdealChompGameConjecture}
    Player A has a winning strategy on $\bar{K}[x_1,\dots,x_n]$ for any $n \geq 1$. 
\end{Conj}

Similarly, we ask the question whether or not a similar statement holds for player B. We expect this question to be more difficult than the previous one.

\begin{Ques}
    Can we find a ring $R_n$ in any Krull dimension $n \in \BN$ where player~B has a winning strategy?
\end{Ques}

Finally, we show that there are infinitely many Krull dimensions in which player A and player B, respectively, have a winning strategy.

\begin{Lem}
    There exists a ring $R_n$ in every Krull dimension $n \in \BN_{\geq 0}$ such that player A has a winning strategy in the Ideal Chomp Game on $R_n$.
\end{Lem}

\begin{myproof}    
    Since the ring $$R_n = \bar{K}[x_0]/(x_0^2) \times  \bar{K}[x_1,\dots,x_n]$$ is a non-trivial cartesian product of Notherian rings, player A has a winning strategy on each $R_n$ by Proposition \ref{CartesianProdCor}. As $\bar{K}[x_0]/(x_0^2)$ has Krull dimension 0 and $\bar{K}[x_1,\dots,x_n]$ has Krull dimension $n$ by ..., $R_n$ has Krull dimension $n$. This concludes the proof.
\end{myproof}

\begin{Lem}
    There exist infinitely many Krull dimensions $n \in \BN$ such that there exists a ring $R_n$ of Krull dimension $n$ where player B has a winning strategy.
\end{Lem}

\begin{myproof}
    Assume there was a Krull dimension $n \in \BN$ such that player A has a winning strategy on every ring of Krull dimension $n$. Then player B has a winning strategy in the Ideal Chomp Game on $\bar{K}[x_1,\dots, x_{n+1}]$ since any element player A can choose in their first turn reduces the ring to a ring of Krull dimension $n$. Hence, player B has a winning strategy on infinitely many rings of pairwise different Krull dimension.
\end{myproof}

 