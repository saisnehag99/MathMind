\subsection{Orientations}\label{subsec:orientation}

The \emph{determinant line} of a global Kuranishi chart $\cK = (G,\cT,\cE,\fs)$, is the real line bundle
\begin{equation}
    \det(\cK) := \det(T\cT) \otimes \det(\fg)\dul \otimes \det(\cE)\dul
\end{equation}
on $\cT$ restricted to the zero locus $\fs\inv(0)$. 
Its \emph{orientation line} is the sheaf of $\bZ_2$-torsors 
\begin{equation}
\orl(\cK) := (\det(\cK)\sm 0)/\bR_{> 0}.\end{equation}
in degree $\vdim(\cK)$. Given $n \in \bZ$, we define $\orl(n) \coloneqq \bZ_2[-n]$.

\begin{definition}
    An \emph{orientation} of $\cK$ is an isomorphism $\orl(\vdim\cK)\cong \orl(\cK)$.
\end{definition} 

 We will throughout use the isomorphism 
 \begin{equation}\label{eq:multiplied-with-dual}
     \orl\otimes\orl\dul\cong\fo(0): v\otimes f \mapsto f(v)
 \end{equation}
 to trivialize the multiplication of an orientation line with its dual. Given a finite-dimensional vector space $V$, we let $\orl(V)$ be the $\bZ_2$-torsor in degree $\dim V$ associated to $H_{\dim V}(V,V\sm\{0\};\bZ)$. Given a Cauchy--Riemann operator $D$ we define its orientation line to be 
\[\orl(D)\coloneqq \orl(\ker D)\otimes \orl(\coker D)\dul.\]

\begin{remark}
   Given a finite-dimensional vector space $V$, the zero map $D_0 = 0 \cl V\to V$ has orientation line $\orl(D_0) \cong \orl(V)\orl(V)\dul$. On the other hand, it is homotopic to the identity $D_1 = \ide$ with orientation line $\orl(D_1) =\orl(0)$. Our choice of trivialization in~\eqref{eq:multiplied-with-dual} ensures that the canonical isomorphism $\orl(D_0) \cong \orl(D_1)$ is orientation-preserving.
\end{remark}

\begin{lemma}
    Let $\cK_\alpha$ be the global Kuranishi chart of Theorem~\ref{thm:pardon-gkc}. Then there exists a canonical isomorphism 
    \begin{equation}
        \orl(\cK) \cong \orl(\delbar_J) \otimes \orl(\bR)\dul\otimes\orl(2|\Gamma^+|-2|\Gamma^-|-6).
    \end{equation} 
\end{lemma}

\begin{proof}
    Observe first that $\cK_\alpha$ admits a well-defined vector lift of its tangent micro-bundle, given by $$T\cT_\alpha = T_{\cT_\alpha/\cBR_\alpha}\oplus \pi^*T\cBR_\alpha,$$ 
    where $\pi \cl \cT_\alpha\to \cBR_\alpha$ is the forgetful map. We will call it from now on simply the tangent bundle and will omit the subscript $\alpha$. Recall that $\cBR$ is a torus bundle over a blow-up of the complex manifold $\cB\sub \Mbar_{0,|\Gamma^+|+|\Gamma^-|}(\bP^d,d)$. Thus, 
    $$\orl(\cBR)\cong \orl(\cB)\otimes \orl(S^1)^{\otimes \Gamma^+\sqcup \Gamma^-}\cong \orl(\fp\fg\fl)\otimes\orl(\Mbar_{0,|\Gamma^+|\sqcup \Gamma^-})\otimes \orl(S^1)^{\otimes \Gamma^+\sqcup \Gamma^-}$$
    canonically. Meanwhile, for $(\varphi,u,w)\in \cT$ we have
    $$(T_{\cT/\cBR})_{(\varphi,u,w)} = \ker\lbr{D_u^\conn + \mu_k(-)|_{\graph(\varphi,u)}\cl C^\infty(\dot{C},u^*T\wh Y) \oplus E_k \to \Omega^{0,1}(\dot{C},u^*T\wh Y)}$$
    which agrees with the index of the Cauchy--Riemann operator $D_u^\conn + \mu_k(-)|_{\graph(\varphi,u)}$. By \cite[Lemma~3.2]{Bao23}, there exists a canonical isomorphism 
    \begin{equation}\label{eq:orientation-vertical-tangent-bundle}
        \orl(D_u^\conn + \mu_k(-)|_{\graph(\varphi,u)}) \cong \orl(D_u^\conn) \otimes \orl(E_k).
    \end{equation}
    Combining these two isomorphisms with the polarization isomorphism $\cG\cong G \times \fg$, we obtain the canonical isomorphisms (over the locus of curves with smooth domains) 
   \begin{align*}
       \orl(\cK_\alpha) \,&\cong\, \orl(D^\conn) \orl(E_k)\orl(\bR)\dul\orl(\cBR)\orl(\wh\fg)\dul\orl(\fg)\dul\orl(E_k)\dul\\
       &\cong\, \orl(D^\conn) \orl(E_k)\orl(\bR)\dul\orl(\fp\fg\fl)\orl(\Mbar_{0,|\Gamma^+|+|\Gamma^-|})\orl(S^1)^{\otimes \Gamma^+\sqcup \Gamma^-}(\orl(S^1)^{\otimes \Gamma^+\sqcup \Gamma^-})\dul\orl(\fg)\dul\orl(\fg)\dul\orl(E_k)\dul\\
       &\cong\,\orl(D^\conn)\orl(\bR)\dul \orl(\Mbar_{0,|\Gamma^+|+|\Gamma^-|})\orl(\fp\fg\fl)\orl(\fp\fg\fl)\dul\orl(E_k)\dul\orl(E_k)\\&\cong\,\orl(D^\conn)\orl(\bR)\dul \orl(\Mbar_{0,|\Gamma^+|+|\Gamma^-|}),
   \end{align*}
  where we omitted the tensor product. Note that we use the Koszul sign rule when switching two orientation lines.
\end{proof}

We associate to any based Reeb orbit $(\gamma,b_\gamma)$ a virtual vector space $V_{\gamma} =(V_\gamma^+,V_\gamma^-)$ as follows.

\begin{definition}[{\cite[Definition~2.46]{Par19}}] Let $\wt\gamma$ be the constant-speed parametrization determined by a base point $b\in \ol\gamma$. Pull back the complex bundle $\wt\gamma^*\xi \oplus \bC \to S^1$ to a bundle $\cV\to \bC\units$. The Lie derivative $\cL_R$ and the trivial connection on $\bC$ pull back to yield a connection $\conn$ on $\cV$. Let $(\wt\cV,\delbar)$ be an extension of $(\cV,\conn^{0,1})$ to all of $\bC$. We define 
\begin{equation}
    V_{\gamma,b} \coloneqq \text{Ind}(\wt\cV,\delbar)
\end{equation}
    to be the index bundle of this extension.
\end{definition}

By \cite[Lemma~2.47]{Par19}, such extensions $(\wt\cV,\delbar)$ exist, and any two extensions differ by the direct sum with a complex vector space (up to isomorphism). In particular, the associated orientation line $\fo(V_{\gamma,b})$
is independent of the choice of extension. Recall that a Reeb orbit $\gamma$ is \emph{good} if it is \emph{not} an even multiple cover of a simple Reeb orbit $\ol\gamma$ with $|\sigma(\cA_{\ol\gamma})\cap (-1,0)| \equiv 1$ mod $2$. Equivalently, $\gamma$ is good if the action of $\bZ/m_\gamma$ on $\orl(V_{\gamma,b})$ is trivial. Thus, for good Reeb orbits, we have a canonical isomorphism $\orl(V_{\gamma,b})\cong \orl(V_{\gamma,b'})$ for any two base points $b,b'$ and we can set 
\begin{equation}
    \orl_\gamma \coloneqq \orl(V_{\gamma,b}).
\end{equation}
By a straightforward generalization of \cite[Lemma~2.51]{Par19} to the case with several positive punctures, the orientation line $\orl(\delbar_{J}) = \orl(D^\conn)$ is canonically isomorphic to 
$$\orl(\Gamma^+;\Gamma^-) \coloneqq \bigotimes\limits_{\gamma\in \Gamma^+}\orl_\gamma\otimes \bigotimes\limits_{\gamma\in \Gamma^-}\orl_\gamma\dul$$
This completes the proof of Theorem~\ref{thm:pardon-gkc}\eqref{gkc-orientation}.
