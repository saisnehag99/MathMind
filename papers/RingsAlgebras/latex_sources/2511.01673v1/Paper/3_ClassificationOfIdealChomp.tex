\section{Classification of the Ideal Chomp Game on $\bar{K}$-algebras up to rank $6$}
\label{ClassificationIdealChomp}

We completely classify the Ideal Chomp Game on $\bar{K}$-algebras up to rank 6 and establish a classification of the Ideal Chomp Game on a large portion of Noetherian $\bar{K}$-algebras. Recall Theorem \ref{ClassificationThm1}:

\begin{Thm*}
    [Classification of the Ideal Chomp Game on $\bar{K}$-algebras up to rank 6]
    \label{ClassificationThm2}
    Player A wins on all $\bar{K}$-algebras up to rank 6, except for local $\bar{K}$-algebras isomorphic to 
    \begin{align*}
        R_1 & = \bar{K},\\
        R_4 & = \bar{K}[x,y]/(x,y)^2,\\ R_{12} & =  \bar{K}[x,y]/(xy,x^3,y^3),\\
        R_{13} & = \bar{K}[x,y]/(x^2,xy^2, y^3),\\ R_{17} & = \bar{K}[x,y,z,w]/(x,y,z,w)^2.
    \end{align*}
\end{Thm*}

The proof strategy of Theorem \ref{ClassificationThm2} relies on the Structure Theorem for Artinian Rings (see Theorem 8.7 in \cite{atiyah-macdonald}) which allows us to split any $\bar{K}$-algebra of rank up to 6 into a Cartesian product of local $\bar{K}$-algebras, Poonen's classification of local $\bar{K}$-algebras up to rank 6 (see Table \ref{poonentable}), proving that player B has a winning strategy on $R_1,R_4,R_{12}, R_{13}$ and $R_{17}$ and various reduction steps which show that player A can reduce the game on any $\bar{K}$-algebra of rank up to 6 apart from $R_1, R_4, R_{12}, R_{13}, R_{17}$ to one of these five rings. Thereby, player A is the second player to move on a ring where the second player has a winning strategy. Hence, player A has a winning strategy.

\subsection{Proof of the Classification}

We proceed as follows: In Proposition \ref{Bwins} we prove that player B has a winning strategy on the rings $R_1, R_4, R_{12}, R_{13}$ and $R_{17}$ where Table \ref{poonentable} claims that player B has a winning strategy. In Proposition \ref{AwinsLocal} we then use Table \ref{reducingAlgebras} together with Proposition \ref{Bwins} to also establish all the claimed winning strategies of player A. Altogether these results show Theorem \ref{ClassificationThm2} which gives a complete classification of all Ideal Chomp Games on $K$-algebras up to rank $6$. Some technical statements are deferred from the proofs of these propositions into lemmata.

\begin{Lem}
\label{localalgebras}
    In the game on a local $\bar{K}$-algebra $\bar{K}[x_1,x_2,\dots,x_n]/I$ with the unique maximal ideal $(x_1,\dots,x_n)$ corresponding to the origin $(0,0,...,0)$, any player who plays a polynomial with non-zero constant coefficient loses immediately. In particular, this holds for all the $K$-algebras in Poonen's table \ref{poonentable}.
\end{Lem}

\begin{myproof}
    Let $f$ be a polynomial with non-zero constant coefficient. Then $(0,\dots,0) \notin V(f)$, so in particular $(0,\dots, 0) \notin V((f) + I) = V(f) \cap V(I)$. If $I + (f)$ wasn't the entire ring $\bar{K}[x_1,\dots,x_n]$, then $I + (f)$ would be contained in some maximal ideal $\Fm$. However, $I$ is only contained in $(x_1,\dots,x_n)$ by our assumption that $(x_1,\dots,x_n)$ is the unique maximal ideal in $\bar{K}[x_1,\dots,x_n]/I$. We conclude that $I + (f) = \bar{K}[x_1,\dots,x_n]$, so any player who plays a polynomial with non-zero constant coefficient loses immediately.
\end{myproof}

\begin{Prop}
\label{Bwins}
    Player B has a winning strategy in the Ideal Chomp Game on $R_1, R_4, R_{12}, R_{13}$ and $R_{17}$.
\end{Prop}

\begin{myproof}
    \begin{enumerate}[itemsep = 1ex]
        \item On $R_1$ player B has a winning strategy by Proposition \ref{fieldwin}.

        \item Player B has a winning strategy on $R_4$:\vspace*{1ex}
        
        Suppose player $A$ plays the polynomial $f(x,y) \in R_4$ in the first move. Then $f$ has a representative $\tilde{f} \in \bar{K}[x,y]$ of the form $\tilde{f} = ax + by + c$. By Lemma \ref{localalgebras} we can assume that $c = 0$ as player A would else lose immediately. As $f \neq 0$, we have that $a \neq 0$ or $b \neq 0$. Since the situation is symmetric, let us assume, without loss of generality, that $a \neq 0$. Then player B can play $y$ in his next turn to return the ring $$R_4/(f) + (y) = R_4/(ax + by,y) = R_4/(x,y) \cong \bar{K}$$ to player A who immediately loses in his next move by Proposition \ref{fieldwin}.

        \item Player B has a winning strategy on $R_{12}$:\vspace*{1ex}
        
        Suppose that player A plays $f \in R_{12}\setminus \{0\} = \bar{K}[x,y]/(xy,x^3,y^3)\setminus \{0\}$. Then $f$ has a representative $\tilde{f} \in \bar{K}[x,y]$ of the form $\tilde{f} = a_0 + a_1 x + a_2 y + a_3 x^2 + a_4 y^2$. By Lemma \ref{localalgebras} player A loses immediately if the constant coefficient $a_0$ is non-zero. Hence, we can assume that $\tilde{f}$ has constant coefficient $a_0 = 0$, i.e. $\tilde{f} = a_1x + a_2y + a_3x^2 + a_4y^2$.\vspace*{1ex}
    
        \textit{Case 1:} Suppose that at least one of $a_1,a_2$ is non-zero. Without loss of generality (the situation is symmetric), assume that $a_1 \neq 0$. Since $$\tilde{f} \cdot x = a_1x^2 + a_2xy + a_3x^3 + a_4xy^2 \equiv a_1x^2,$$ we find that $$x^2 \in (xy, x^3,y^3) + (f) = I_0 + (f) = I_1.$$ Hence, $I_1 = (x^2,xy,y^3) + (a_1x + a_2y + a_4 y^2)$ with $a_1 \neq 0$. As a result, it is a valid move for player $B$ to play $y$: the minimal degree of an element of $(x^2, xy, y^3)$ is 2, hence, y would have to lie in $(a_1x + a_2y + a_4y^2)$ which it doesn't since $a_1 \neq 0$. Once player B has played $y$, the game's ideal is $I_2 = (x,y)$, i.e. player A loses in the next turn by Proposition \ref{fieldwin}.\vspace*{1ex}

        \textit{Case 2:} Suppose that $a_1,a_2 = 0$. Then $\tilde{f} = a_3x^2 + a_4y^2$ with at least one of the coefficients $a_3,a_4$ being nonzero. Without loss of generality (the situation is symmetric) assume that $a_3 \neq 0$. Then player B can play $y^2$ in the next turn which reduces the game to the game on $R_4 = \bar{K}[x,y]/(x,y)^2$. This game is won by the second player, i.e. player B, by our analysis of winning strategies on $R_4$.

        \item Player B has a winning strategy on $R_{13}$:\vspace*{1ex}

        Suppose that player A plays $f \in \bar{K}[x,y]/(x^2, xy^2, y^3) \setminus \{0\}$. Then $f$ has a representative of the form $\tilde{f} = a_0 + a_1x + a_2y + a_3 xy + a_4 y^2$. By Lemma \ref{localalgebras} we know that player A loses if the constant term is non-zero, so we may assume $a_0 = 0$. As a preparation we compute:
    
        $$\tilde{f}(x,y) \cdot x = a_2 xy + a_3 x^2y$$
        $$\tilde{f}(x,y) \cdot y = a_1 xy + a_2y^2.$$
    
        \textit{Case 1:} Suppose $a_1 = a_2 = 0$. Then $\tilde{f} = a_3xy + a_4y^2$.\vspace*{1ex}
        
        \hspace*{3mm}\textit{Case 1.1:} If $a_3 \neq 0$, then $y^2 \notin I_1 = I_0 + (\tilde{f})$, so player B may add $y^2$ to the ideal to obtain $I_2 = I_0 + (f) + (y^2) = (x,y)^2$. Hence, the game is reduced to the game on $R_4 = \bar{K}[x,y]/(x,y)^2$. In this case, player B wins by our previous analysis of winning strategies on $R_4$.\vspace*{1ex}
        
        \hspace*{3mm}\textit{Case 1.2:} Otherwise, if $a_3 = 0$, then $\tilde{f} = a_4y^2$. Hence, player B can play $xy$ to reduce the game to the game on $\bar{K}[x,y]/I_0 + (\tilde{f}) + (xy) = \bar{K}[x,y]/(x,y)^2$. Again, player B wins by our previous analysis of winning strategies on $R_4$.\vspace*{1ex}

        \textit{Case 2:} Suppose $a_1 = 0, a_2 \neq 0$. Then by the preparatory computations, $y^2 \in I_1$ and player B can play $x$ since $a_2 \neq 0$. Thereby, player B reduces the game to $$R_{13}/(\tilde{f},x) = \bar{K}[x,y]/(x,y^3,a_2y) = K[x,y]/(x,y) \cong \bar{K}$$\vspace*{1ex} and wins by Proposition \ref{fieldwin}.\vspace*{1ex}

        \textit{Case 3:} Suppose $a_1 \neq 0, a_2 = 0$. Then by the preparatory computations, $xy \in I_1$, but $y^2 \notin I_1$. If player B now plays $y^2$, he reduces the game to $\bar{K}[x,y]/(x,y)^2$ which he wins by our previous analysis of winning strategies on $R_4$.\vspace*{1ex}

        \textit{Case 4:} Finally, suppose $a_1,a_2 \neq 0$.

        Then player B can play $y$. Thereby, he reduces the game to $$R_{13}/I_1 + (y) = \bar{K}[x,y]/(x,y) \cong \bar{K}$$ and wins by Proposition \ref{fieldwin}.

        \item Player B has a winning strategy on $R_{17}$:\vspace*{1ex}

        Any element $f \in R_{17}\setminus\{0\}$ player A plays can be represented by a polynomial of the form $\tilde{f} = a_0 + a_1x + a_2 y + a_3 z + a_4 w$. By Lemma \ref{localalgebras} $a_0 = 0$ if player A tries to avoid losing immediately. Therefore, there exists $i \in \{1,2,3,4\}$ with $a_i \neq 0$. But then taking the quotient $\bar{K}[x,y,z,w]/(x,y,z,w)^2 + (f)$ reduces the rank of $\bar{K}[x,y,z,w]/(x,y,z,w)^2$ over $\bar{K}$ by exactly 1. Looking at Table~\ref{reducingAlgebras} we notice that in all games on $\bar{K}$-algebras with $\rank(R) = 3$, one move suffices to reduce them to the algebras $R_1$ or $R_4$ which are second player wins by what we proved above. Hence, all $\bar{K}$-algebras of rank three are first player wins, making $R_{17} = K[x,y,z,w]/(x,y,z,w)^2$ a second player win.\vspace{1ex}  

        Notice that our proof for the winning strategy on $R_{17}$ only works once we have established that all $\bar{K}$-algebras of rank 3 can be reduced to $R_1$ are $R_4$ by a single move which will happen in Proposition \ref{AwinsLocal} without using the winning strategy on $R_{17}$.
        
    \end{enumerate}
\end{myproof}

For proving that player A has a winning strategy on all the other local $\bar{K}$-algebras up to rank $6$ we need the following auxiliary result:

\begin{Lem}
\label{ringIso1}
     The following rings are isomorphic over any field $K$: $$K[y,z]/(y,z)^2 \cong K[x,y,z]/(x,y,z)^2 +(x + y).$$
\end{Lem}

\begin{myproof}
    Let\vspace*{-5mm} $$\phi: K[y,z]/(y,z)^2 \to K[x,y,z]/(x,y,z)^2 + (x+y),\quad f(y,z) \mapsto f(y,z),$$
    $$\psi: K[x,y,z]/(x,y,z)^2 + (x+y) \to K[y,z]/(y,z)^2,\quad f(x,y,z) \mapsto f(-y,y,z).$$ It is straightforward to check that $\phi$ and $\psi$ are both $K$-algebra homomorphisms. We show that they are two-sided inverses of one another.\\
    Indeed, let $f \in K[x,y,z]/(x,y,z)^2 + (x+y)$. Then $$\phi \circ \psi(f(x,y,z)) = \phi(f(-y,y,z)) = \phi(g(y,z)) = g(y,z) = f(-y,y,z) = f(x,y,z)$$ since $x + y \equiv 0$ in $K[x,y,z]/(x,y,z)^2 + (x+y)$. Conversely, let $f \in K[y,z]/(y,z)^2$. Then $$\psi \circ \phi(f(y,z)) = \psi(f(y,z)) = f(y,z).$$ This concludes the Lemma.
\end{myproof}

The following table summarizes the reduction moves we need for the next Proposition \ref{AwinsLocal}.

\renewcommand{\arraystretch}{1.2}
\setlength\LTleft{0pt plus 1fill}
\setlength\LTright{0pt plus 1fill}
\enlargethispage{3mm}
\vspace*{-3mm}

\begin{table}[h!]
\begin{minipage}[t]{0.48\linewidth}
  \smallskip
    \begin{longtable}{|c|c|c|c|}
        \hline
        \(n\) & ring & move & resulting ring\\
        \hline
        2 & \(R_2\) & \(x\) & \(R_1\)\\
        \hline
        3 & \(R_3\) & \(x\) & \(R_1\)\\
        \hline
        4 & \(R_5\) & \(x\) & \(R_1\)\\
        & \(R_6\) & \(y^2\) & \(R_4\)\\
          & \(R_7\) & \(xy\)& \(R_4\)\\
          & \(R_{7,*}\) & \(x^2\) & \(R_4\)\\
          & \(R_8\) & \(z\) & \(R_4\)\\
        \hline
        5 & \(R_9\) & \(x\) & \(R_1\)\\
          & \(R_{10}\) & \(y^2\) & \(R_4\)\\
          & \(R_{11}\) & \(y^2\) & \(R_4\)\\
          & \(R_{14}\) & \(z\) & \(R_4\)\\
          & \(R_{15}\) & \(z\) & \(R_4\)\\
          & \(R_{15,*}\) & \(z\) & \(R_4\)\\
          & \(R_{16}\) & \(x\) & \(R_4\)\\
        \hline
        6 & \(R_{18}\) & \(x\) & \(R_1\)\\
          & \(R_{19}\) & \(y^2\) & \(R_4\)\\
          & \(R_{20}\) & \(y^2\) & \(R_4\)\\
          & \(R_{21}\) & \(y^3\) & \(R_{12}\)\\
          & \(R_{22}\) & \(x^3\) & \(R_{12}\)\\
          & \(R_{23}\) & \(y^3\) & \(R_{13}\)\\
          & \(R_{24}\) & \(y^3\) & \(R_{13}\)\\
          & \(R_{25}\) & \(xy^2\) & \(R_{13}\)\\
          & \(R_{25,*}\) & \(x^2\) & \(R_{13}\)\\
          & \(R_{25,**}\) & \(y^3\) & \(R_{13}\)\\
          \hline
    \end{longtable}
    \label{reductionMoves}
    \end{minipage}\hfill
    \begin{minipage}[t]{0.48\linewidth}
    \smallskip
    \begin{longtable}{|c|c|c|c|}
        \hline
        \(n\) & ring & move & resulting ring\\
        \hline
          6 & \(R_{26}\) & \(xy\) & \(R_{12}\)\\
          & \(R_{27}\) & \(z\) & \(R_{4}\)\\
          & \(R_{28}\) & \(z\) & \(R_{4}\)\\
          & \(R_{29}\) & \(z\) & \(R_{4}\)\\
          & \(R_{29,*}\) & \(z\) & \(R_{4}\)\\
          & \(R_{30}\) & \(x\) & \(R_4\)\\
          & \(R_{31}\) & \(x\) & \(R_4\)\\
          & \(R_{31,*}\) & \(x\) & \(R_4\)\\
          & \(R_{32}\) & \(z\) & \(R_4\)\\
          & \(R_{33}\) & \(x\) & \(R_{12}\)\\
          & \(R_{34}\) & \(x+y\)& \(R_4\)\\
          & \(R_{34,*}\) & \(x+y\) & \(R_4\)\\
          & \(R_{35}\) & \(x+y\) & \(R_4\)\\
          & \(R_{36}\) & \(z\) & \(R_4\)\\
          & \(R_{36,*}\) & \(z\) & \(R_4\)\\   
          & \(R_{37}\) & \(z\) & \(R_4\)\\
          & \(R_{37,*}\) & \(z\) & \(R_4\)\\
          & \(R_{38}\) & \(w^2\) & \(R_{17}\)\\
          &\(R_{39}\) & \(zw\) & \(R_{17}\)\\
          & \(R_{39,*}\) & \(z^2\) & \(R_{17}\)\\
          & \(R_{40}\) & \(y^2\) & \(R_{17}\)\\
          & \(R_{41}\) & \(xy\) & \(R_{17}\)\\
          & \(R_{41,*}\) & \(x^2\) & \(R_{17}\)\\
          & \(R_{42}\) & \(v\) & \(R_{17}\)\\
        \hline
\end{longtable}
\end{minipage}
\setcounter{table}{1}
\caption{Winning moves for player A on local $\bar{K}$-algebras of rank $\leq 6$.}
\label{reducingAlgebras}
\end{table}

\begin{Prop}
\label{AwinsLocal}
    Player A has a winning strategy on all other rings from Table~\ref{poonentable}, that is on all rings $R_i$ with $i \notin \{1,4,12,13,17\}$.
\end{Prop}

\begin{myproof}
    Consider Table \ref{reducingAlgebras}. It shows that every such $\bar{K}$-algebra $R$ can be reduced to a $\bar{K}$-algebra $R'$ on which the second player has a winning strategy by Proposition \ref{Bwins}. Hence, player A can play the given move in column ``move'' to reduce the current game to a game where the second player, now player A, has a winning strategy. We conclude that player A has a winning strategy on all rings $R_i$ with $i \notin \{1,4,12,13,17\}$ given in Table \ref{poonentable}.

    The only cases where the isomorphism class of the resulting $\bar{K}$-algebra after the move is not immediately clear are $R_{34}, R_{34,*}$ and $R_{35}$. We discuss them here:

    \begin{enumerate}
        \item On $R_{34} = K[x,y,z]/(xy, xz, y^2, z^2, x^3)$ player A has a winning strategy starting with $x + y$. Since
        $$(x+y) \cdot x = x^2 + xy \equiv x^2 \text{ and } (x+y) \cdot z \equiv yz,$$ 
        we have $x^2, yz \in I_0 + (x+y).$ Hence, $$R_{34}/(x+y) = \bar{K}[x,y,z]/(x,y,z)^2 + (x+y),$$ so by Lemma \ref{ringIso1} we conclude that player A reduces the game to a game isomorphic to $K[y,z]/(y,z)^2 \cong R_4$ where the second player (then player A) has a winning strategy.

        \item On $R_{34,*} = \bar{K}[x,y,z]/(xy, xz, yz, y^2-z^2, x^3)$ player A has a winning strategy starting with $x + y$ and, thereby, reducing the game to
         \begin{align*}
             R_{34,*}/(x+y) = \bar{K}[x,y,z] / (xy, xz, yz, y^2 - z^2, x^3) + (x+y) =\\= \bar{K}[x,y,z]/(x,y,z)^2 + (x+y) \cong \bar{K}[y,z]/(y,z)^2 \cong R_4
         \end{align*} by Lemma \ref{ringIso1} and since in $R_{34,*}$ we have $$(x+y) \cdot y = xy + y^2 = y^2 \text{ and } (x+y)x = x^2 + xy = x^2.$$

         \item On $R_{35} = \bar{K}[x,y,z]/(xy, xz, yz, x^2 + y^2 - z^2)$ player A has a winning strategy starting with $x+y$. Since $$(x+y)x = xy + x^2 \equiv x^2 \text{ and }(x+y)y = xy + y^2 \equiv y^2,$$ we have that $x^2, y^2, z^2 \in I_0 + (x+y)$. Hence, by Lemma \ref{ringIso1} $R_{35}/(x+y) = \bar{K}[x,y,z]/(x,y,z)^2 + (x+y) \cong \bar{K}[y,z]/(y,z)^2 \cong R_4$.
    \end{enumerate}

     We also verified all the reductions given in Table \ref{reducingAlgebras} using Sage. The Sage Worksheet can be found under the link provided in the bibliography \cite{karl}.
\end{myproof}

\begin{Cor}
    Table \ref{poonentable} classifies the Ideal Chomp Game on all local $\bar{K}$-algebras up to rank $6$.
\end{Cor}

We conclude this section by classifying the Ideal Chomp Game on all (not necessarily local) $\bar{K}$-algebras up to rank $6$ and provide partial results on $\bar{K}$-algebras of Krull dimension~0 and higher rank.

\begin{Prop}
    Player A has a winning strategy on every non-local $\bar{K}$-algebra of finite rank.
\end{Prop}

\begin{myproof}
    Since finite-rank $\bar{K}$-algebras are Artinian rings, the Structure Theorem for Artinian Rings (see e.g. Theorem 8.7 in \cite{atiyah-macdonald}) implies that any finite-rank $\bar{K}$-algebra can be written as a finite direct product of local finite-rank $\bar{K}$-algebras. Since the $\bar{K}$-algebras are non-local, this direct product is non-trivial. Hence, player A has a winning strategy on all non-local $\bar{K}$-algebras of finite rank by Proposition \ref{CartesianProdCor}.
\end{myproof}

Returning to our main classification result, we obtain:

\begin{Thm}
    [Classification of the Ideal Chomp Game on $\bar{K}$-algebras up to rank 6]
    \label{ClassificationThm}
    Player A wins on all $\bar{K}$-algebras up to rank 6, except for local $\bar{K}$-algebras isomorphic to
    \begin{myitemize}
        \item $R_1 = \bar{K}$,
        \item $R_4 = \bar{K}[x,y]/(x,y)^2$, \item $R_{12} =  \bar{K}[x,y]/(xy,x^3,y^3)$,
        \item $R_{13} = \bar{K}[x,y]/(x^2,xy^2, y^3)$ and \item $R_{17} = \bar{K}[x,y,z,w]/(x,y,z,w)^2$.
    \end{myitemize}
\end{Thm}

\begin{myproof}
    By Corollary \ref{CartesianProdCor} player A wins on all finite-rank $\bar{K}$-algebras that are not local themselves, including those up to rank $6$. By Proposition \ref{Bwins} and Proposition \ref{AwinsLocal} $\bar{K}$-algebras isomorphic to $R_1, R_4, R_{12}, R_{13}$ and $R_{17}$ are the only local $\bar{K}$-algebras up to rank 6 where player B has a winning strategy. This concludes the proof.
\end{myproof}