\ifdefined\isMainDocument
\else
    \documentclass[10pt,oneside,reqno]{amsart}
    \usepackage{xr}
    \externaldocument{../main} % Main の .aux を参照
    \usepackage{latexsym}
\usepackage{amscd}
\usepackage{amsmath}
\usepackage{amssymb}
\usepackage[mathscr]{euscript}
\usepackage{graphicx}
\usepackage{geometry}
\usepackage{mathtools}
\usepackage[all]{xy}
\usepackage[dvipsnames]{xcolor}
\usepackage[colorlinks,final,hyperindex]{hyperref}
\usepackage{tikz}
\usepackage{tikz-cd}
\usetikzlibrary{decorations.pathmorphing,decorations.markings,arrows,calc,shapes.geometric,arrows.meta,positioning}
\usepackage{enumerate}
\usepackage{multirow}

\usepackage[bb=dsserif]{mathalpha}
\usepackage{bm}

\newcommand{\QQ}{\ensuremath{\mathbb{Q}}}
\newcommand{\DD}{\ensuremath{\mathbb{D}}}
\newcommand{\CC}{\ensuremath{\mathbb{C}}}
\newcommand{\RR}{\ensuremath{\mathbb{R}}}
\newcommand{\ZZ}{\ensuremath{\mathbb{Z}}}
\newcommand{\NN}{\ensuremath{\mathbb{N}}}

\newcommand{\g}{\ensuremath{\mathfrak{g}}}
\renewcommand{\SS}{\ensuremath{\mathbb{S}}}

\newcommand{\id}{\ensuremath{\mathrm{id}}}

\makeatletter

\newcommand{\bbone}{\mathbb{1}}

\newcommand{\calA}{\mathcal{A}}
\newcommand{\calB}{\mathcal{B}}
\newcommand{\calC}{\mathcal{C}}
\newcommand{\calD}{\mathcal{D}}
\newcommand{\calE}{\mathcal{E}}
\newcommand{\calF}{\mathcal{F}}
\newcommand{\calG}{\mathcal{G}}
\newcommand{\calH}{\mathcal{H}}
\newcommand{\calI}{\mathcal{I}}
\newcommand{\calJ}{\mathcal{J}}
\newcommand{\calK}{\mathcal{K}}
\newcommand{\calL}{\mathcal{L}}
\newcommand{\calM}{\mathcal{M}}
\newcommand{\calN}{\mathcal{N}}
\newcommand{\calO}{\mathcal{O}}
\newcommand{\calP}{\mathcal{P}}
\newcommand{\calQ}{\mathcal{Q}}
\newcommand{\calR}{\mathcal{R}}
\newcommand{\calS}{\mathcal{S}}
\newcommand{\calT}{\mathcal{T}}
\newcommand{\calU}{\mathcal{U}}
\newcommand{\calV}{\mathcal{V}}
\newcommand{\calW}{\mathcal{W}}
\newcommand{\calX}{\mathcal{X}}
\newcommand{\calY}{\mathcal{Y}}
\newcommand{\calZ}{\mathcal{Z}}
\newcommand{\kk}{\Bbbk}

\newcommand{\scrY}{\mathscr{Y}}
\newcommand{\scrV}{\mathscr{V}}
\newcommand{\scrP}{\mathscr{P}}

\newcommand{\Rep}{\mathop{\mathrm{Rep}}\nolimits}
\newcommand{\alg}{\mathop{\mathrm{alg}}\nolimits}
\newcommand{\Span}{\mathop{\mathrm{Span}}\nolimits}
\newcommand{\proj}{\mathop{\mathrm{proj}}\nolimits}
\newcommand{\Hom}{\mathop{\mathrm{Hom}}\nolimits}
\newcommand{\PHom}{\mathop{\mathrm{PHom}}\nolimits}
\newcommand{\Tw}{\mathop{\mathrm{Tw}}\nolimits}
\newcommand{\Dim}{\mathop{\mathrm{Dim}}\nolimits}
\newcommand{\Imm}{\mathop{\mathrm{Im}}\nolimits}
\newcommand{\Aus}{\mathop{\mathrm{Aus}}\nolimits}
\newcommand{\Ann}{\mathop{\mathrm{Ann}}\nolimits}
\newcommand{\APC}{\mathop{\mathrm{APC}}\nolimits}
\newcommand{\Barr}{\mathop{\mathrm{Bar}}\nolimits}
\newcommand{\Cone}{\mathop{\mathrm{Cone}}\nolimits}
\newcommand{\modu}{\mathop{\mathrm{mod}}\nolimits}
\newcommand{\dgMod}{\mathop{\mathrm{dgMod}}\nolimits}
\newcommand{\soc}{\mathop{\mathrm{soc}}\nolimits}
\newcommand{\dg}{\mathop{\mathrm{dg}}\nolimits}
\newcommand{\red}{\mathop{\mathrm{red}}\nolimits}
\newcommand{\Map}{\mathop{\mathrm{Map}}\nolimits}
\newcommand{\inj}{\mathop{\mathrm{inj}}\nolimits}
\newcommand{\op}{\mathop{\mathrm{op}}\nolimits}
\newcommand{\Ker}{\mathop{\mathrm{Ker}}\nolimits}
\newcommand{\Tor}{\mathop{\mathrm{Tor}}\nolimits}
\newcommand{\Tot}{\mathop{\mathrm{Tot}}\nolimits}
\newcommand{\Ext}{\mathop{\mathrm{Ext}}\nolimits}
\newcommand{\sg}{\mathop{\mathrm{sg}}\nolimits}
\newcommand{\Sl}{\mathop{\mathfrak{sl}}\nolimits}
\newcommand{\nc}{\mathop{\mathrm{nc}}\nolimits}
\newcommand{\Tr}{\mathop{\mathrm{Tr}}\nolimits}
\newcommand{\Irr}{\mathop{\mathrm{Irr}}\nolimits}
\newcommand{\HC}{\mathop{\mathrm{HC}}\nolimits}
\newcommand{\HH}{\mathop{\mathrm{HH}}\nolimits}
\newcommand{\THH}{\mathop{\mathrm{TH}}\nolimits}
\newcommand{\Perf}{\mathop{\mathrm{Perf}}\nolimits}
\newcommand{\del}{\partial}
\newcommand{\ev}{\mathop{\mathrm{ev}}\nolimits}
\newcommand{\co}{\mathop{\mathrm{co}}\nolimits}
\newcommand{\pt}{\mathop{\mathrm{pt}}\nolimits}
\newcommand{\wt}{\mathop{\mathrm{wt}}\nolimits}
\newcommand{\pCY}{\mathop{\mathrm{pCY}}\nolimits}
\newcommand{\gr}{\mathop{\mathrm{gr}}\nolimits}
\newcommand{\coker}{\mathop{\mathrm{coker}}\nolimits}


\newcommand{\Top}{\mathop{\mathrm{Top}}\nolimits}
\newcommand{\Set}{\mathop{\mathrm{Set}}\nolimits}
\newcommand{\Vect}{\mathop{\mathrm{Vect}}\nolimits}
\newcommand{\Mod}{\mathop{\mathrm{Mod}}\nolimits}
\newcommand{\Ob}{\mathop{\mathrm{Ob}}\nolimits}
\newcommand{\Ind}{\mathop{\mathrm{Ind}}\nolimits}
\newcommand{\Sym}{\mathop{\mathrm{Sym}}\nolimits}
\newcommand{\Alt}{\mathop{\mathrm{Alt}}\nolimits}
\newcommand{\End}{\mathop{\mathrm{End}}\nolimits}
\newcommand{\Ch}{\mathop{\mathrm{Ch}}\nolimits}


\newcommand{\Ass}{\mathop{\mathrm{Ass}}\nolimits}
\newcommand{\Com}{\mathop{\mathrm{Com}}\nolimits}
\newcommand{\Lie}{\mathop{\mathrm{Lie}}\nolimits}
\newcommand{\colim}{\mathop{\mathrm{colim}}\nolimits}

\newcommand{\Op}{\mathop{\mathsf{Operads}}\nolimits}
\newcommand{\Diop}{\mathop{\mathsf{Dioperads}}\nolimits}
\newcommand{\PDiop}{\mathop{\mathsf{PDioperads}}\nolimits}
\newcommand{\Coop}{\mathop{\mathsf{Cooperads}}\nolimits}
\newcommand{\Codiop}{\mathop{\mathsf{Codioperads}}\nolimits}
\newcommand{\PCodiop}{\mathop{\mathsf{PCodioperads}}\nolimits}

\newcommand{\antish}{\ensuremath{\mbox{!`}}}


\tikzset{->-/.style={decoration={
			markings,
			mark=at position #1 with {\arrow{>}}},postaction={decorate}}}
\tikzset{-w-/.style={decoration={
			markings,
			mark=at position #1 with {\arrow{Stealth[fill=white,scale=1.4]}}},postaction={decorate}}}
\tikzset{->-/.default=0.65}
\tikzset{-w-/.default=0.65}
\tikzstyle{bullet}=[circle,fill=black,inner sep=0.5mm]
\tikzstyle{circ}=[circle,draw=black,fill=white,inner sep=0.5mm]
\tikzstyle{vertex}=[circle,draw=black,thick,inner sep=0.5mm]
\tikzstyle{dot}=[draw,circle,fill=black,minimum size=0.5mm,inner sep = 0mm, outer sep = 0mm]
\tikzset{darrow/.style={double distance = 4pt,>={Implies},->},
	darrowthin/.style={double equal sign distance,>={Implies},->},
	tarrow/.style={-,preaction={draw,darrow}},
	qarrow/.style={preaction={draw,darrow,shorten >=0pt},shorten >=1pt,-,double,double
		distance=0.2pt}}
\newcommand\tikcirc[1][2.5]{\tikz[baseline=-#1]{\draw[thick](0,0)circle[radius=#1mm];}}
\newcommand{\tikzfig}[1]{\begin{tikzpicture}[auto,baseline={([yshift=-.5ex]current bounding box.center)}]#1\end{tikzpicture}}
\usetikzlibrary{decorations.pathreplacing}

\usepackage{amsthm}
\usepackage{thmtools}
\usepackage[noabbrev,capitalize]{cleveref}
\newtheorem{theorem}{Theorem}
\newtheorem{lemma}[theorem]{Lemma}
\newtheorem{proposition}[theorem]{Proposition}
\newtheorem{corollary}[theorem]{Corollary}
\newtheorem{conjecture}[theorem]{Conjecture}
\newtheorem{Claim}[theorem]{Claim}

\newtheorem*{theorem*}{Theorem}

\theoremstyle{definition}
\newtheorem{definition}{Definition}
\newtheorem{example}{Example}
\newtheorem{cexample}{Counterexample}
\newtheorem{exercise}{Exercise}
\newtheorem{question}{Question}
\newtheorem{remark}{Remark}

\addtolength{\textheight}{1.75cm}
\addtolength{\voffset}{-0.5cm}
\addtolength{\footskip}{+0.6cm}
\geometry{margin=1.2in}


\usepackage[backend=biber, maxbibnames=99, bibencoding=utf8, doi=false, isbn=false, url=false, style=alphabetic]{biblatex}
\addbibresource{refs.bib}

    \begin{document}
\fi

\section[$\AdDMR_0$, $\addmr$, and $\adls$]{Adjoint double shuffle scheme $\AdDMR_0$} \label{sect:AdDMR}
% and its bigraded version  $\adls$}



The purpose of this paper is to study AdMZVs. To begin with, we define AdMZVs more precisely. Let $\bullet\in \{\shuffle,*\}$. For $k_1,\ldots,k_r\in\mathbb{Z}_{>0}$ and $l\in\mathbb{Z}_{\ge 0}$, $\bullet$-AdMZVs are defined by
\begin{align*}
\bAdmzv{l}{k_1,\ldots,k_r} :=& \sum_{i = 0}^r (-1)^{k_{i+1}+\cdots+k_r + l} \zeta^{\bullet}(k_1,\ldots,k_i)\zeta^{\bullet}_l(k_r,\ldots,k_{i+1}) \in \mathcal{Z}.
\end{align*}
Here, 
\[
\zeta_{l}^{\bullet}(k_r,\ldots,k_{i+1}) = (-1)^l \sum_{\substack{l_{i+1}+\cdots+l_r = l \\ l_{i+1},\ldots,l_r\ge 0}}
\left(\prod_{j = i+1}^r \binom{k_j + l_j+1}{l_j}\right)
\zeta^{\bullet}(k_r+l_r,\ldots,k_{i+1}+l_{i+1}).
\]
It should be noted that the images of these two types of AdMZVs in $\mathcal{Z}/\zeta(2)\mathcal{Z}$ are identical, i.e.,
\begin{align}
\sAdmzv{l}{k_1,\ldots,k_r}. \equiv \hAdmzv{l}{k_1,\ldots,k_r} \mod \zeta(2)\mathcal{Z} \label{eq:gen-AdMZV}
\end{align}
holds. 
We define AdMZVs as the image of $\bullet$-AdMZVs in $\mathcal{Z}/\zeta(2)\mathcal{Z}$ and denote $\sAdmzv{l}{k_1,\ldots,k_r} \mod \zeta(2)\mathcal{Z}$ by simply $\Admzv{l}{k_1,\ldots,k_r}$.
In this section, we review Jarossay's work \cite{Jarossay}.
We consider a generating function for AdMZVs as follows:
\begin{align*}
\Phi_{\mathrm{Ad}_{\shuffle}} :=& \sum_{x_0^lx_1x_0^{k_r-1}x_1\cdots x_0^{k_1-1}x_1 \in X^{*}} \sAdmzv{l}{k_1,\ldots,k_r} x_0^lx_1x_0^{k_r-1}x_1\cdots x_0^{k_1-1}x_1 + \text{(addtional terms).}%\\
%\Phi_{\mathrm{Ad}_*} :=& \sum_{l \ge 0} \sum_{Y_{k_1}\cdots Y_{k_r} \in Y^*} \hAdmzv{l}{k_1,\ldots,k_r} Y_{k_r}\cdots Y_{k_1}.
\end{align*}
The fundamental idea of Jarossay's work is the following equality $\Phi_{\mathrm{Ad}_{\shuffle}} = \Phi_{\shuffle}^{-1}x_1\Phi_{\shuffle}$.
From this perspective, he developed the property of $\Phi_{\shuffle}^{-1}x_1\Phi_{\shuffle}$ and proposed the $\Q$-linear relation among AdMZVs. He named these relations \textbf{adjoint double shuffle relations}.
Additionally, he introduced an affine scheme called the \textit{adjoint double shuffle scheme} and raised the question of whether two affine schemes, $\DMR_0$ and the adjoint double shuffle scheme, are actually isomorphic.
From this observation, we then consider an affine scheme \textit{the adjoint double shuffle scheme} $\AdDMR_0$.
 
\nsubsection{The generating function of AdMZVs and adjoint double shuffle relations}

A key idea in Jarossay's work is the occurrences of the AdMZVs in the coefficient of $\Phi_{\shuffle}^{-1}x_1\Phi_{\shuffle}$.
The following proposition formalizes this observation.
\begin{prop}[{\cite[Proposition 3.2.2]{Jarossay}}]
For $l\in \mathbb{Z}_{\ge 0}$ and $k_1,\ldots,k_r\in\mathbb{Z}_{>0}$, the following holds:
\[
\langle \Phi_{\shuffle}^{-1}x_1\Phi_{\shuffle} \mid x_0^lx_1x_0^{k_r-1}x_1\cdots x_0^{k_1-1}x_1\rangle = \sAdmzv{l}{k_1,\ldots,k_r} .
\]
\end{prop}
\begin{proof}
The claim follows from the identities below.
\begin{align*}
\langle \Phi_{\shuffle}^{-1}x_1\Phi_{\shuffle} \mid x_0^{k_r-1}x_1\cdots x_0^{k_1-1}x_1x_0^l \rangle
=& \sum_{j = 0}^r \langle\Phi_{\shuffle}^{-1} \mid x_0^lx_1\cdots x_0^{k_1-1}x_1\cdots x_0^{k_i-1}\rangle \langle \Phi_{\shuffle}\mid x_0^{k_{i+1}-1}x_1 \cdots x_0^{k_r-1}x_1\rangle \\
=& \sum_{j = 0}^r(-1)^{k_1 + \cdots + k_i + l}\langle \Phi_{\shuffle} \mid x_0^{k_i-1}x_1 \cdots x_0^{k_1}x_1x_0^l \rangle\\
&\times \langle \Phi_{\shuffle}\mid x_0^{k_{i+1}-1}x_1 \cdots x_0^{k_r-1}x_1\rangle \displaybreak[3] \\
=&\sum_{j = 0}^r(-1)^{k_1 + \cdots + k_i + l} \zeta_l^{\shuffle}(k_1,\ldots,k_j)\zeta^{\shuffle}(k_r,\ldots,k_{j+1})\\
=&\sAdmzv{l}{k_r,\ldots,k_1}.
\end{align*}
Here, we use the antipode property of the group-like element, namely, $S_{X}^{\vee}(\Phi_{\shuffle}) = \Phi_{\shuffle}^{-1}$ in  the second equality and
\begin{align*}
\langle \Phi_{\shuffle} \mid x_0^{k_r-1}x_1\cdots x_0^{k_1-1}x_1x_0^l \rangle =  \zeta_{l}^{\shuffle} (k_1,\ldots,k_r) 
\end{align*}
in the third equality.
\end{proof}
Since $\Delta_{\shuffle}(\Phi_{\shuffle}) = \Phi_{\shuffle} \otimes \Phi_{\shuffle}$,
it follows that
\begin{align}
\Delta_{\shuffle}(\Phi_{\shuffle}^{-1}x_1\Phi_{\shuffle} ) = \Phi_{\shuffle}^{-1}x_1\Phi_{\shuffle}  \otimes 1 + 1\otimes \Phi_{\shuffle}^{-1}x_1\Phi_{\shuffle}. \label{eq:ad-shuffle}
\end{align}
Since the condition $\Delta_{\shuffle}(\Phi_{\shuffle}) = \Phi_{\shuffle} \otimes \Phi_{\shuffle}$ corresponds to the shuffle product relations of $\shuffle$-regularized MZVs, the condition \eqref{eq:ad-shuffle} corresponds to some $\Q$-linear relations among $\shuffle$-AdMZVs which are related to the binary operation $\shuffle$.
We call these $\Q$-linear relations in \eqref{eq:ad-shuffle}  \textbf{adjoint shuffle relations}.

We next investigate other $\Q$-linear relations among AdMZVs which correspond to the harmonic product relations.
Let $l\in \mathbb{Z}_{\ge 0}$, $k_1,\ldots,k_r\in\mathbb{Z}_{>0}$, and $T$ be an indeterminate. We define $\mathbf{q}_{\#}^{\vee} :  \ide{h}^{\vee} \rightarrow \mathbb{Q}[[T]]\cotimes \Q\langle\langle Y \rangle\rangle$ as a $\mathbb{Q}$-linear map by
\[
\mathbf{q}_{\#}^{\vee} (x_0^{k_1-1}x_1\cdots x_1x_0^{k_{r+1}-1})
=
\begin{cases}
T^{k_1-1}y_{k_2}\cdots y_{k_{r}} &\text{if }k_{r+1}=1\text{ and}\\
0&\text{otherwise, this case includes $\text{(argument)} =1$}.
\end{cases}
\]
From \eqref{eq:gen-AdMZV}, it follows that
\begin{align*}
\mathbf{q}_{\#}^{\vee}(\Phi_{\shuffle}^{-1}x_1\Phi_{\shuffle}) \equiv \sum_{w = y_{k_1}\cdots y_{k_r} \in Y^*}\left( \sum_{l\ge 0} T^l \zeta_{\mathrm{Ad}}(k_r,\ldots,k_1,l)\right) w \mod \zeta(2)\mathcal{Z} \cotimes \ide{h}^{\vee}.
\end{align*}
On the other hand, we define a $\Q[[T]]$-module morphism $\sft_*^{\vee}$ of $\mathbb{Q}[[T]] \cotimes \Q\langle \langle Y \rangle \rangle$ by
\[
y_k \mapsto \sum_{j=0}^{k-1} \binom{k-1}{j} T^{j} y_{k-j}.
\]
Then, it follows that $\sft_*^{\vee}$ is the coalgebra homomorphism with respect to $\Delta_*$.
Since, for $\Phi \in \ide{h}^{\vee}$, $\sft_*^{\vee}(\Phi)$ is explicitly given by
\begin{align}
\sft_*^{\vee}(\Phi) = \sum_{w = y_{k_1}\cdots y_{k_r} \in Y^*} \sum_{l\ge 0} T^l \sum_{\substack{l_1 + \cdots +l_r = l \\ l_1,\ldots, l_r\ge 0}} \prod_{j=1}^r \binom{k_j + l_j -1}{l_j} \langle\Phi \mid y_{k_1+l_1}\cdots y_{k_r+l_r} \rangle w,\label{eq:sft-exc}
\end{align}
this leads to the following relation between $\mathbf{q}_{\#}^{\vee}(\Phi_{\shuffle}^{-1}x_1\Phi_{\shuffle})$ and $\Phi_*$:
\begin{align*}
\mathbf{q}_{\#}^{\vee}(\Phi_{\shuffle}^{-1}x_1\Phi_{\shuffle}) =& 
\sum_{w = y_{k_1}\cdots y_{k_r} \in Y^*}\left( \sum_{l\ge 0} T^l \zeta_{\mathrm{Ad}}^{\shuffle}(k_r,\ldots,k_1;l)\right) w  \displaybreak[3] \\
\stackrel{\eqref{eq:gen-AdMZV}}{\equiv} &  \sum_{w = y_{k_1}\cdots y_{k_r} \in Y^*}\left( \sum_{l\ge 0} T^l \zeta_{\mathrm{Ad}}^{*}(k_r,\ldots,k_1;l)\right) w \displaybreak[3] \\
\equiv &  \sum_{w = y_{k_1}\cdots y_{k_r} \in Y^*}\left( \sum_{l\ge 0} T^l  \sum_{j = 0}^r (-1)^{k_1+\cdots+k_j + l} \zeta^*(k_r,\ldots,k_{j+1})\zeta^*(k_1,\ldots,k_j ; l)  \right) w \displaybreak[3] \\
\equiv& \sum_{w = y_{k_1}\cdots y_{k_r} \in Y^*} \sum_{ j = 0}^r   \\
& \left(\sum_{l\ge 0} T^l \sum_{\substack{l_1 +\cdots + l_r = l \\l_1,\ldots,l_r\ge 0 }} \left(\prod_{i = 1}^j \binom{k_i + l_i -1}{l_i}\right) \zeta^{*}(k_1 + l_1,\ldots,k_j + l_j) \right) \inv(y_{k_j}\cdots y_{k_1}) \\
&\times  \zeta^*(k_r,\ldots,k_j) y_{k_{j+1}}\cdots y_{k_r} \displaybreak[3] \\
\equiv & \sum_{w = y_{k_1}\cdots y_{k_r} \in Y^*}  \sum_{ j = 0}^r \\
&  \left(\sum_{l\ge 0} T^l \sum_{\substack{l_1 +\cdots + l_r = l \\ l_1,\ldots,l_r\ge 0} } \left(\prod_{i = 1}^j \binom{k_i + l_i -1}{l_i}\right) \langle \Phi_* \mid y_{k_j + l_j}\cdots y_{k_1+l_1}\rangle \right) \inv(y_{k_j}\cdots y_{k_1}) \\
&\times  \langle \Phi_* \mid y_{k_{j+1}}\cdots y_{k_r} \rangle y_{k_{j+1}}\cdots y_{k_r} \displaybreak[3] \\
\stackrel{\eqref{eq:sft-exc}}{\equiv} & \sum_{w = y_{k_1}\cdots y_{k_r} \in Y^*}  \sum_{ j = 0}^r \\
&\left(\langle \sft_*^{\vee}(\Phi_*) \mid y_{k_j}\cdots y_{k_1} \rangle \inv(y_{k_j}\cdots y_{k_1})\right)\times 
\left( \langle \Phi_* \mid y_{k_{j+1}}\cdots y_{k_r} \rangle  y_{k_{j+1}}\cdots y_{k_r}\right) \displaybreak[3] \\ 
\equiv &\inv \circ \sft_*^{\vee}(\Phi_*)\Phi_*
\end{align*}
modulo $\zeta(2)\mathcal{Z} \cotimes \ide{h}^{\vee}$.
Here, $\inv$ is an anti-automorphism on $\Q\langle \langle Y \rangle\rangle$ defined by $y_k \mapsto (-1)^k y_k$ ($k\in\mathbb{Z}_{>0}$).
Since $\sft_*^{\vee}$ is the coalgebra homomorphism with respect to $\Delta_*$, it follows that the following equality holds:
\begin{align}
\Delta_*(\mathbf{q}_{\#}^{\vee}(\Phi_{\shuffle}^{-1}x_1\Phi_{\shuffle})) = \mathbf{q}_{\#}^{\vee}(\Phi_{\shuffle}^{-1}x_1\Phi_{\shuffle})\otimes \mathbf{q}_{\#}^{\vee}(\Phi_{\shuffle}^{-1}x_1\Phi_{\shuffle}). \label{eq:ad-harmonic}
\end{align}
Here, we regard $\Delta_*$ as $\Q[[T]]$-module via the coefficient expansion.
Since the condition $\Delta_{*}(\Phi_{*}) = \Phi_{*} \cotimes \Phi_{*}$ corresponds to the harmonic product relations of $*$-regularized MZVs, the Condition \eqref{eq:ad-harmonic} gives rise to $\Q$-linear relations among $*$-AdMZVs, which correspond to the binary operation $*$.
Jarossay showed this result in a more general setting:
\begin{thm}[{\cite[Proposition 3.2.5]{Jarossay}}]\label{thm:t-adic}
For $\Phi \in \DMR_0(R)$, we have
\begin{enumerate}
\item\label{thm:t-adic-1} $\Delta_{\shuffle}(\Phi^{-1}x_1\Phi) = \Phi^{-1}x_1\Phi \otimes 1 + 1\otimes \Phi^{-1}x_1\Phi$, and
\item\label{thm:t-adic-2} $\Delta_{*}(\mathbf{q}_{\#}^{\vee}(\Phi^{-1}x_1\Phi)) = \mathbf{q}_{\#}^{\vee}(\Phi^{-1}x_1\Phi) \otimes \mathbf{q}_{\#}^{\vee}(\Phi^{-1}x_1\Phi)$. Here, we regard \(\Delta_*\) as a $R[[T]]$-module morphism via the coefficient expansion.
\end{enumerate}
\end{thm}



\begin{rmk}
Let $t$ be an indeterminate. Ono, Seki and Yamamoto (\cite{Ono-Seki-Yamamoto}) proposed the $t$-adic SMZVs, which is the $t$-adic completion of $\sum_{0\le l<l'} \Admzv{l}{k_1,\ldots,k_r} t^l$, i.e. a $t$-adic SMZV is defined by 
\[
\zeta_{\widehat{\S}}(k_1,\ldots,k_r) := \sum_{0\le l} \Admzv{l}{k_1,\ldots,k_r} t^l \in \mathcal{Z}/\zeta(2)\mathcal{Z} \cotimes \mathbb{Q}[[t]].
\]
Independently, Jarossay introduced the $\Lambda$-adjoint multiple zeta values \cite[Definition~2.3.1(ii)]{Jarossay}. 
They are the same completion, obtained by replacing the indeterminate $t$ with $\Lambda$.
Ono, Seki, and Yamamoto also discussed certain analogs of the double shuffle relations among $t$-adic SMZVs, called the double shuffle relations of $t$-adic SMZVs.
It should be noted that when a linear relation among $t$-adic SMZVs is given, comparing the coefficients of the indeterminate $t$ gives a linear relation among AdMZVs.

Based on the discussion up to this point, we derive two families of linear relations among AdMZVs, Jarossay's one and Ono-Seki-Yamamoto's one.
However, the difference between them is only the shuffle relations.
To be more precise, Theorem \ref{thm:t-adic}. (\ref{thm:t-adic-1}) gives Jarossay's shuffle linear relations, while Ono, Seki, and Yamamoto's shuffle linear relation is
\begin{align*}
\langle \Phi \mid x_0^lx_1(u\shuffle vx_i)\rangle = -\sum_{l_1+l_2 = l} \langle \Phi \mid x_0^{l_1}x_i(x_0^{l_2}\shuffle S_X(v))x_1u\rangle
%\label{eq:AppB-1}
\end{align*}
for all $u$, $v\in X$ and $i = 0$, $1$. 
Jarossay proved that these two types of linear shuffle relations are equivalent (see \cite[Proposition 3.4.1]{Jarossay}).
\end{rmk}



Motivated by Theorem \ref{thm:t-adic}, Jarossay introduced the adjoint double shuffle scheme, defined by the adjoint double shuffle relations.
\begin{df}\label{def:AdDMR}
We define $\AdDMR_0(R)$ as the set of $\Phi \in R\cotimes \ide{h}^{\vee}$ satisfying the following properties:
\begin{enumerate}
%\item\label{def:AdDMR-1} There exists $\phi \in R\cotimes \ide{h}^{\vee}$ with $\Delta_{\shuffle}(\phi) = \phi\otimes \phi$, $\langle \phi \mid\mathbf{1}\rangle = 1$ and $\langle \phi \mid x_0\rangle = \langle \phi \mid x_1\rangle = 0$ such that $\Phi = \phi^{-1}x_1\phi$ holds,
\item\label{def:AdDMR-2} $\Phi - x_1 \in \ide{F}_2^{\ge 4}$,
\item\label{def:AdDMR-4} $\Delta_{\shuffle}(\Phi) = \Phi \otimes 1 + 1\otimes \Phi$, and
\item\label{def:AdDMR-3} $\Delta_{*}(\Phi_{\#}) = \Phi_{\#} \otimes  \Phi_{\#} $, where $\Phi_{\#} := \mathbf{q}_{\#}^{\vee}(\Phi)$.
\end{enumerate}
We define the adjoint double shuffle scheme as a functor $\AdDMR_0 : \Qalg\rightarrow \cat{Set}$ ; $R \mapsto \AdDMR_0(R)$.
%We can check that $\AdDMR_0$ is the affine scheme and $\mathcal{O}(\AdDMR_0) = \mathcal{Z}_{\mathrm{Ad}}^f$. 
\end{df}

By Theorem \ref{thm:t-adic}, there is a morphism of an affine scheme 
\[
\Ad(x_1)  : \DMR_0 \rightarrow \AdDMR_0,
\]
that is, for each $\Q$-algebra $R$, we define $\Ad^R(x_1) : \DMR_0(R) \rightarrow \AdDMR_0(R) ; \phi \mapsto \phi^{-1}x_1\phi$.
Jarossay's question asks whether this natural morphism is an isomorphism.
\begin{qu}[{\cite[Quetion 3.2.8]{Jarossay}}]\label{qu:Jarossay}
Is the morphism of affine schemes
\[
\Ad(x_1) : \DMR_0 \longrightarrow \AdDMR_0
\]
a naturally isomorphism?
\end{qu}


To verify Question~\ref{qu:Jarossay}, we focus on the tangent spaces of $\DMR_0$ and $\AdDMR_0$.
Proving that $\DMR_0$ is an affine group scheme, Racinet showed that the tangent space $\dmr$ at $1$ is a Lie algebra and that the image of $\dmr$ under a certain exponential map acts transitively on $\DMR_0$. 
Consequently, studying $\dmr$ is essentially as informative as studying $\DMR_0$, so we focus on $\dmr$.
Since the natural map $\Ad^{\Q}(x_1): \DMR_0(\Q)\rightarrow \AdDMR_0(\Q)$ sends $1$ to $x_1$, we compare $\dmr$ with the tangent space of $\AdDMR_0(\Q)$ at $x_1$, denoted $\addmr$.
By Racinet, \cite[Definitions in Section 3.3.1]{Racinet}, $\dmr$ is written as
\[
\dmr = \{\psi \in \ide{F}_2^{\ge 3} \mid \Delta_*(\psi_{\star}) = \psi_{\star} \otimes 1 + 1\otimes \psi_{\star} \},
\]
where $\psi_{\star} = \mathbf{q}(\psi) + \sum_{n\ge 2} \frac{(-1)^n}{n} \langle \psi \mid x_0^{n-1}x_1\rangle y_1^n$. 
On the adjoint side, we set as follows:
\begin{df}\label{def:addmr}
We define 
\begin{align*}
\addmr := \{\Psi \in  \ide{h}^{\vee} \mid x_1 + \varepsilon \Psi \in \AdDMR_0(\Q[\varepsilon]/(\varepsilon^2))\}.
\end{align*}
Equivalently, $\addmr$ is the $\Q$-linear space of all $\Psi\in \ide{h}^{\vee}$ such that
\begin{enumerate}
%\item\label{def:addmr--1}  $\wt \Psi \ge 4$,
\item\label{def:addmr--2}  $\Psi \in \ide{F}_2^{\ge 4}$, and 
\item\label{def:addmr--3}  $\Delta_{*}(\Psi_{\#}) = \Psi_{\#} \otimes 1 + 1 \otimes \Psi_{\#}$.
\end{enumerate}
\end{df}
Each $\Q$-linear spaces are decomposed by weight as
\[
\dmr = \prod_{k\ge0}\dmr^{(k)},\qquad
\addmr = \prod_{k\ge0}\addmr^{(k)}.
\]
Here, $\dmr^{(k)}$ (resp. $\addmr^{(k)}$) denotes the homogeneous subspace of weight $k$ in $\dmr$ (resp. $\addmr$).
Then, the table below shows $\dim_\Q \dmr^{(k)}$ up to $k=10$, and $\dim_\Q \addmr^{(k)}$ up to $k=11$ respectively.
\begin{center}
\begin{tabular}{c|ccccccccccccc}
$k$ & $0$ & $1$ & $2$ & $3$ & $4$ & $5$ & $6$ & $7$ & $8$ & $9$ & $10$ & $11$ & $\cdots$ \\
\hline
$\dim \dmr^{(k)}$ & $0$ & $0$ & $0$ & $1$ & $0$ & $1$ & $0$ & $1$ & $1$ & $1$ & $1$ &  & $\cdots$ \\
$\dim \addmr^{(k)}$ & $0$ & $0$ & $0$ & $0$ & $2$ & $2$ & $3$ & $3$ & $4$ & $5$ & $6$ & $7$ & $\cdots$ 
\end{tabular}
\end{center}

Since the derivation $\ad(x_1)$ of $\Ad(x_1)$ raises the weight by one (see next subsection), we expect $\dim \dmr^{(k)} = \dim \addmr^{(k+1)}$ for all $k\in\mathbb{Z}_{\ge 0}$.
However, these data above indicate $\dim \dmr^{(k)} \neq \dim \addmr^{(k+1)}$ for $3 \le k \le 10$, thus Question \ref{qu:Jarossay} is negative.

\begin{comment}
\textcolor{red}{このJarossayの問題について確認するために$\DMR_0(\Q)$の$1$における接平面$\dmr$と、$\AdDMR_0(\Q)$の$x_1$における接平面$\addmr$に着目する。ここで$\AdDMR_0(\Q)$の``$x_1$"における接平面について考察する理由は$1$は$\Ad(x_1)^{\Q} : \DMR_0(\Q)\rightarrow \AdDMR_0(\Q)$により$x_1$に移ることにある。
具体的には$\dmr$と$\addmr$はexplicitにこの様に書き下せる。
($\dmr$と$\addmr$のexplicit formula)
また、$\Ad(x_1) : \DMR_0 \rightarrow \AdDMR_0$の微分写像は$\ad(x_1) : \dmr\rightarrow \addmr ; \psi \mapsto [x_1,\psi]$となる(詳しくは次のsubsectionで)
このとき、$\dmr$と$\addmr$はweight direct product decompositionを持つ。
(式を書く)
ここで、$\dmr^{(k)}$のup to $k = 10$までの次元と$\addmr^{(k)}$のup to $k = 11$までの次元は以下のように書き下せる。
(次元の表)
$\ad(x_1)$がweightを一つ上げる写像であることを考慮すると$\dim \dmr^{(k)} = \dim \addmr^{(k+1)} $が期待されるが、この表からは期待する次元にはならないことがうかがえる。
}
To verify Question \ref{qu:Jarossay}, we computed the dimensions of the Lie algebras associated with $\DMR_0$ and $\AdDMR_0$, and this calculation shows that Question \ref{qu:Jarossay} is probably negative (Appendix \ref{section:computational data}). However, by imposing some additional conditions, we obtain a refined formulation of Jarossay's problem.\footnote{Professor Minoru Hirose at Kagoshima University kindly provided the proto type of the source codes related to the adjoint double shuffle relations; we are grateful for his support.}
\end{comment}



\nsubsection{Affine group scheme $\TM_1$ and corresponding Lie algebra}

From now on, we refine Jarossay's question.
To discuss $\AdDMR_0$, we introduce an affine group scheme $\TM_1$. This setup is based on \cite[Section 1.1.1]{Jarossay2015} and \cite[Proposition A.1.1]{Jarossay2020}.
We define $\TM_1(R) := \{ \Phi \in R\widehat{\otimes}\ide{h}^{\vee} \mid \langle \Phi \mid x_0^k \rangle = 0 \text{ for all $k\in\mathbb{Z}_{\ge 0}$},  \langle \Phi \mid x_1 \rangle = 1 \}.$
We consider the binary operation $\circledast_1$ \cite[Definition 1.1.3]{Jarossay2015} on $\TM_1$ defined by
\[
\Phi_1 \circledast_{1} \Phi_2 = \kappa_{\Phi_1}(\Phi_2)
\]
for $\Phi_1$, $\Phi_2 \in \TM_1(R)$.

\begin{prop}\label{prop;TM1}
A pair $(\TM_1(R), \circledast_1)$ forms a group.
Additionally, let $\TM_1$ be a functor $\Qalg\rightarrow \cat{Grp}$ ; $R\mapsto \TM_1(R)$. 
Then, $\TM_1$ is the prounipotent affine group scheme.
\end{prop}

\begin{proof}

Let $\Phi\in \TM_1(R)$ and define $\tau^R : \TM_1(R) \rightarrow \mathrm{End}_{R\text{-alg}}(R\cotimes\ide{h}^{\vee}) : \Phi \mapsto \kappa_{\Phi}$.
For $m\in\mathbb{Z}_{\ge 0}$, we define the $R$-submodules $R\cotimes \ide{h}^{\vee}_{m}$ of $R\cotimes \ide{h}^{\vee}$ as the ideals $(x_0,x_1)^m$. Since $\kappa_{\Phi}$ is the endomorphism of $R\cotimes \ide{h}^{\vee}$ as an $R$-algebra and $\kappa_{\Phi}(x_i) - x_i$ do not have terms with weight $1$, we have
\begin{enumerate}
\item\label{enumi:prouni-1} $\tau^R(\Phi_1 \circledast_1 \Phi_2) = \tau^R(\Phi_1) \circ \tau^R(\Phi_2)$,
\item\label{enumi:prouni-2} $\tau^R(\Phi_1) = \tau^R(\Phi_2)$ if and only if  $\Phi_1 =  \Phi_2$,
\item\label{enumi:prouni-3} $\kappa_{\Phi} (R\otimes \ide{h}^{\vee}_m) \subset R\otimes \ide{h}^{\vee}_m$, and
\item\label{enumi:prouni-4} the following composition of the $R$-module map
\[
\xymatrix{
R\cotimes \ide{h}^{\vee} \ar[r]^{\kappa_{\Phi}|_{R\cotimes \ide{h}^{\vee}_m}} & R\cotimes \ide{h}^{\vee}_m  \ar@{->>}[r]&  R\cotimes \ide{h}^{\vee}_m/  \ide{h}^{\vee}_{m+1}
}
\]
is the identity.
\end{enumerate}
Therefore, if we show that $\TM_1(R)$ equips the structure of a group with respect to $\circledast_1$, then $\tau^R$ is the $R$-module representation of $\TM_1(R)$ and satisfies the conditions of the prounipotency.

Given these preliminary considerations, we show the group structure of $\TM_1(R)$.
By the definition of $\circledast_1$, we can immediately check that $x_1$ is the identity element of $\TM_1(R)$.
The associativity law of $\circledast_1$ is given as follows:
Let $\Phi_1$, $\Phi_2$, and $\Phi_3 \in \TM_1(R)$. Since $\kappa_{\Phi_1 \circledast_1 \Phi_2} = \tau^R (\Phi_1 \circledast_1 \Phi_2) = \tau^R (\Phi_1) \circ \tau^R (\Phi_2) = \kappa_{\Phi_1} \circ \kappa_{\Phi_2}$ holds, we have
\[
(\Phi_1\circledast_1 \Phi_2) \circledast_1 \Phi_3 
= \kappa_{\Phi_1\circledast_1 \Phi_2}(\Phi_3)
=\kappa_{\Phi_1} \circ \kappa_{\Phi_2} (\Phi_3)
=\kappa_{\Phi_1} (\Phi_2 \circledast_1 \Phi_3)
=\Phi_1 \circledast_1(\Phi_2 \circledast \Phi_3).
\]
Lastly, we show the existence of the inverse $\Phi'$ of $\Phi\in \TM_1(R)$.
We shall illustrate it constructively.
By the definition of $\TM_1(R)$, the coefficients of $\Phi'$ in weight $1$  must be $\langle \Phi' \mid x_1 \rangle = 1$ and $\langle \Phi' \mid x_0 \rangle = 0$.
%We put $w\in X^*$ with the weight degree at least $2$ and assume $\Phi'$ is determined up to degree $\deg w$.
Let $w\in X^*$ with $2\le \wt w$. Since $\kappa_{\Phi}$ satisfies the \eqref{enumi:prouni-3} and \eqref{enumi:prouni-4} as mentinoned in this proof, we have
\[
 \langle \kappa_{\Phi}(\Phi') \mid w \rangle = \langle \Phi' \mid w \rangle - \sum_{\wt w' < \wt w} c_{w',w} \langle \Phi' \mid w' \rangle
\]
for $c_{w',w} \in R$. Therefore, by recursively putting $\langle \Phi' \mid w \rangle =  \sum_{\wt w' < \wt w} c_{w',w} \langle \Phi' \mid w' \rangle$, we find out $\Phi'$ to be the inverse of $\Phi$ with respect to $\circledast_1$.


\end{proof}


\begin{rmk}
One may notice that the product $\circledast_1$ on $\TM_1(R)$ is defined so that $\Ad^R(x_1) : (\TM(R) , \circledast) \rightarrow (\TM_1(R),\circledast_1)$ is a group homomorphism \cite[Proposition 1.1.4 (ii)]{Jarossay2015}, where $\TM(R):=\{\Phi\in R\widehat{\otimes}\ide{h}^{\vee}\mid \langle \Phi,1\rangle=1\}$, equipped with the product $\circledast$.
One also notes that $\DMR_0(R)$ is a subgroup of $\TM(R)$.
\end{rmk}





We now define the corresponding Lie algebra of $\TM_1$. We define
\[
\tm_1 := \ker(\TM_1(\Q[\varepsilon]) \xrightarrow{\varepsilon = 0} \TM_1(\Q)).
\]
A direct calculation shows
\[
\tm_1= \{\Psi \in  \ide{h}^{\vee} \mid \langle \Psi \mid x_1 \rangle = 0,  \langle \Psi \mid x_0^l \rangle = 0 \text{ ($l\in\mathbb{Z}_{\ge 0}$)}\}.
\]
For $\Psi_1$, $\Psi_2 \in \tm_1$, we define $\Q$-bilinear binary operation $\{\mathchar`-,\mathchar`-\}_1$ \cite[Appendix A.1.2]{Jarossay2020} on $\tm_1$ by
\[
\{\Psi_1,\Psi_2\}_{1} := d_{\Psi_1}(\Psi_2) - d_{\Psi_2}(\Psi_1).
\]
Here, for $\Psi \in \ide{h}^{\vee}$, we define the derivation $d_{\Psi} : \ide{h}^{\vee} \rightarrow \ide{h}^{\vee}$ by
\[
d_{\Psi}(x_0) = 0 \;\;,\;\;d_{\Psi}(x_1)= \Psi.
\] 
Then, we have the following.

\begin{prop}
A pair $(\tm_1,\{\mathchar`-,\mathchar`-\}_1)$ froms a Lie algebra.
\end{prop}

\begin{proof}
Let $\der( \ide{h}^{\vee})$ be a derivation on $\ide{h}^{\vee}$. Then, $\der (\ide{h}^{\vee})$ is the corresponding Lie algebra of an affine group scheme $\Aut(\ide{h^{\vee}}) : \Qalg \rightarrow \Grp ; R\mapsto \Aut_{R\text{-\textbf{alg}}}(R\cotimes \ide{h^{\vee}})$.

We focus on the corresponding Lie algebra homomorphism $d\tau : \tm_1\rightarrow \der(\ide{h}^{\vee})$ of $\tau^\Q : \TM_1(\Q) \rightarrow \Aut_{\Qalg}(\ide{h}^{\vee})$ defined by $\tau_{1 + \varepsilon \Psi} = \id + \varepsilon d\tau(\Psi) $ ($1 + \varepsilon \Psi \in \ker (\TM_1(\Q[\varepsilon]) \xrightarrow{\varepsilon = 0} \TM_1(\Q))$).
Then by construction, $d\tau$ is a Lie algebra homomorphism and the image of $d\tau$ is a derivation on $ \ide{h}^{\vee}$.
Since 
\begin{align*}
\tau(1 + \varepsilon)(x_0) = &\kappa_{1 + \varepsilon \Psi}(x_0) = x_0 = x_0 + \varepsilon \cdot 0,\\
\tau(1 + \varepsilon)(x_1) = &\kappa_{1 + \varepsilon \Psi}(x_1) = 1 + \varepsilon \Psi,
\end{align*}
we have $d\tau(\Psi)(x_0) = 0$ and $d\tau(\Psi)(x_1) = \Psi$. This implies that $d\tau(\Psi) = d_{\Psi}$ holds.
Let $\langle\mathchar`-, \mathchar`-\rangle$ be the Lie bracket equipped in $\tm_1$.
Then, we have
\begin{align*}
\langle\Psi_1 ,\Psi_2\rangle =  d\tau (\langle \Psi_1, \Psi_2\rangle)(x_1)
 = [d\tau(\Psi_1), d\tau(\Psi_2)](x_1) = d_{\Psi_1}(\Psi_2) - d_{\Psi_2}(\Psi_1) = \{\Psi_1,\Psi_2\}_1.
\end{align*}
The second equality above follows from the fact that $d\tau$ is a Lie algebra homomorphism.
Therefore, $\langle\mathchar`-,\mathchar`-\rangle = \{\mathchar`-,\mathchar`-\}_1$ holds.

\end{proof}

\begin{cor}\label{cor:tm-freeLie}
The intersection of $\Q$-linear spaces $\tm_1 \cap  \ide{F}_2$ is a Lie subalgebra of $\tm_1$.
\end{cor}

\begin{proof}
Let $\Psi_1,\Psi_2\in \tm_1\cap \ide{F}_2$.
Because $d_{\Psi_1}$ is a derivation of the Lie algebra, it follows that $d_{\Psi_1}(\Psi_2)\in \tm_1\cap \ide{F}_2$. Consequently, ${\Psi_1,\Psi_2}\in \tm_1\cap \ide{F}_2$.
\end{proof}

\begin{comment}
By Proposition \ref{prop:unipotent}. (\ref{enumi:unipotent-1}) provides the exponential map $\exp^{\circledast} : \tm_1\rightarrow \TM_1$ which satisfies
\[
\tau\circ \exp^{\circledast_1}) = \exp \circ d\tau.
\]
Substituting $x_1$ into the above, $\tau(f)(\phi) = \phi$ for $\phi \in \TM_1(R)$ implies
\[
\exp^{\circledast} = \tau(\exp^{\circledast_1})(x_1) = \exp (d\tau)(x_1).
\]
\end{comment}

\nsubsection{A Lie algebra $\Fad$ and an affine group scheme $\FAd$}
For $f \in \ide{h}^{\vee}$, we define a $\Q$-linear map $\ad(f) : \ide{h}^{\vee}\rightarrow \ide{h}^{\vee} ; g \mapsto [f,g]$ and a $\Q$-linear space
\[
\ide{F}_2^{\ad_{x_1}} :=  \left\{ \Psi \in \ide{F}^{\ge 3}\middle|
\Psi^{00} = 0
\right\},
\]

which is inspired by the following Schneps's proposition.

\begin{prop}[{\cite[Proposition 2.2]{Schneps}}]\label{prop:Schneps}
Let $\Psi \in  \ide{h}^{\vee}$.
Then, the conditions
\begin{itemize}
\item $\Psi \in  \ide{F}_2^{\ge 3}$ and $\Psi^{00} = 0$ holds, and
\item there exists $\psi \in\ide{F}_2^{\ge 2}$ such that
\[
\Psi = [x_1,\psi],
\]
\end{itemize}
are equivalent.
\end{prop}


This proposition implies
\[
 \Fad  = \left\{ \Psi \in \ide{F}_2^{\ge 3} \middle|
\begin{matrix}
\text{${}^{\exists}\psi \in \ide{F}_2^{\ge 2}$ such that }\Psi = [x_1,\psi]
\end{matrix}
\right\}.
\]

\begin{prop}[{\cite[Proposition A.1.1]{Jarossay2020}}]\label{prop:Fad}
The $\Q$-linear space $\Fad$ is a Lie subalgebra of $\tm_1$.
\end{prop}

\begin{proof}
We focus on $\ide{F}_2^{\ge 2}$.
Then, $\ide{F}_2^{\ge 2}$ forms a Lie algebra with respect to $\{\mathchar`-,\mathchar`-\}$. Here, for $\phi$, $\psi \in \ide{F}_2^{\ge 2}$, we define $\{\phi,\psi\} := d_{[x_1,\phi]}(\psi) - d_{[x_1,\psi]}(\phi) + [\phi,\psi]$ as in \cite[Proposition 3.5, Corollary 3.6, and Proposition 3.7]{Racinet}.
We can easily see that the image of the mapping $\ad(x_1) : \ide{F}_2^{\ge 2} \rightarrow \tm_1 ; \phi \mapsto [x_1,\phi]$ is equal to $\Fad$ by Proposition \ref{prop:Schneps}. Therefore, it suffices to show that $\ad(x_1)$ is a Lie algebra homomorphism.

Let $\psi_1$, $\psi_2 \in \ide{F}_2^{\ge 2}$. Then we have
\begin{align*}
\ad(x_1)(\{\psi_1,\psi_2\}) =& [x_1,\{\psi_1,\psi_2\}]\\
=&[x_1, d_{[x_1,\psi_1]}(\psi_2) - d_{[x_1,\psi_2]}(\psi_1) + [\psi_1,\psi_2]]\\
=&[x_1,d_{[x_1,\psi_1]}(\psi_2)] - [x_1,d_{[x_1,\psi_2]}(\psi_1)] + x_1\psi_1\psi_2 + \psi_2\psi_1 x_1 - x_1\psi_2\psi_1 - \psi_1\psi_2x_1\\
=&[x_1,d_{[x_1,\psi_1]}(\psi_2)]  + [x_1,\psi_1]\psi_2 + \psi_1x_1\psi_2 - \psi_2[x_1 , \psi_1] + \psi_2x_1\psi_1\\
&-[x_1,d_{[x_1,\psi_2]}(\psi_1)] -[x_1,\psi_2]\psi_1 - \psi_2x_1\psi_1 + \psi_1[x_1,\psi_2] - \psi_1x_1\psi_2\\
=&[x_1,d_{[x_1,\psi_1]}(\psi_2)] +  [d_{[x_1,\psi_1]}(x_1),\psi_2] - [x_1,d_{[x_1,\psi_2]}(\psi_1)] - [d_{[x_1,\psi_2]}(x_1),\psi_1]\\
=&d_{[x_1,\psi_1]}(\ad(x_1)(\psi_2)) - d_{[x_1,\psi_2]}(\ad(x_1)(\psi_1)) \\
=&\{\ad(x_1)(\psi_1), \ad(x_1)(\psi_2) \}_1
\end{align*}
as claimed.
\end{proof}

\begin{rmk}
We can check $\ker \ide{ad}(x_1) = \Q\langle\langle x_1 \rangle \rangle $.
Thus, putting $\widetilde{\ide{F}}_2 := \{\Phi \in \ide{h}^{\vee} \mid \langle \Phi \mid x_1^n \rangle = 0\text{ for $n \in \mathbb{Z}_{>0}$}  \}$, we have an isomorphism of Lie algebras $\ide{ad}(x_1) : \widetilde{\ide{F}}_2^{\ge 2} \rightarrow \Fad$.
\end{rmk}
Let us define
\[
\FAd(R) := 
\left\{
\Phi \in R\cotimes \ide{h}^{\vee} 
\middle|
\begin{matrix}
\Phi - x_1 \in \ide{F}_2^{\ge 3}\text{ and}\\
\text{ ${}^{\exists} \phi \in \exp(\ide{F}_2)$ such that }\Phi=\phi^{-1}x_1\phi.
\end{matrix}
\right\}
\]
and a functor $\FAd : \Qalg \rightarrow \Set ; R \mapsto \FAd(R)$.
In a similar way to Proposition \ref{prop:Fad}, we can check that $\FAd(R)$ is a subgroup of $\TM_1(R)$, and this implies that $\FAd$ is a closed affine subscheme of $\TM_1$ as an affine group scheme.
Additionally, a corresponding Lie algebra of $\FAd$ is $\Fad$.

Let $f$ be the inverse map of $\ad(x_1)\mid_{\widetilde{\ide{F}}_2^{\ge 2}}$. 
The defining equations that define $\FAd$ are as follows:
\begin{prop}\label{prop:Adjoint}
Let $\Phi-x_1\in \ide{F}_2^{\ge 3}$ and write
\[
\Phi-x_1=\sum_{k\ge 3}\Phi^{(k)} \qquad (\Phi^{(k)} \text{: homogeneous of weight }k).
\]
Define sequences $\{U_n\}_{n\ge 3}\subset \ide{F}_2$ and 
$\{\psi_n\}_{n\ge 2}\subset \ide{F}_2$ recursively by
\[
U_3:=0,\qquad U_4:=0,
\]
and for $n\ge 5$ set
\[
U_n
:=\sum_{r\ge 2}\frac{(-1)^r}{r!}
\!\!\!\sum_{\substack{m_1,\dots,m_r\ge 2\\ m_1+\cdots+m_r=n-1}}
\!\!\!
\ad(\psi_{m_1}) \circ\cdots \circ\ad(\psi_{m_r})(x_1),
\]
then put, for every $n\ge 3$,
\[
\psi_{n-1}:=f(\Phi^{(n)} - U_n).
\]
Let $\psi:=\sum_{m\ge 2}\psi_m$ and $\phi:=\exp(-\psi)$. If $(U_n - \Phi^{(n)})^{00} = 0$ for $n\ge 3$, then $\Phi = \phi^{-1}x_1\phi$. Thus, $\Phi \in \FAd(\Q)$.
Conversely, if $\Phi \in \FAd(\Q)$, then $(U_n - \Phi^{(n)})^{00} = 0$ for all $n \in \mathbb{Z}_{\ge 3}$.
\end{prop}

\begin{proof}
Recall the adjoint identity:
\[
\exp(-\psi)\,x_1\,\exp(\psi)
= \exp(\ad(-\psi))(x_1)
= x_1+\sum_{r\ge 1}\frac{1}{r!}\,\ad(\psi)^{\,r}(x_1).
\]
Write $\psi=\sum_{m\ge 2}\psi_m$ with $\psi_m$ homogeneous of weight $m$.
The weight $n$ part of $\ad(\psi)^{\,r}(x_1)$ is the sum over
$r$-tuples $(m_1,\ldots,m_r)$ with $m_i\ge 2$ and $m_1+\cdots+m_r=n-1$:
\[
\bigl(\ad(\psi)^{\,r}(x_1)\bigr)^{(n)}
=
\sum_{r\ge 0} \frac{1}{r!}\sum_{\substack{m_1,\dots,m_r\ge 2\\ m_1+\cdots+m_r=n-1}}
\ad(-\psi_{m_1})\circ \cdots \circ\ad(-\psi_{m_r})(x_1).
\]
Separating the $r=1$ gives
\begin{equation}\label{eq:weight-n-expansion}
\bigl(\exp(-\psi)\,x_1\,\exp(\psi)\bigr)^{(n)}
=
\ad(x_1)(\psi_{n-1})+
\sum_{r\ge 2}\frac{(-1)^r}{r!}\sum_{\substack{m_1,\dots,m_r\ge 2\\ m_1+\cdots+m_r=n-1}}\ad(\psi_{m_1})\circ \cdots \circ\ad(\psi_{m_r})(x_1).
\end{equation}
By definition, the second term is exactly $U_n$.

We first show that if $(\Phi^{(n)} - U_n)^{00}=0$ for all $n$, then $\Phi \in \FAd(\Q)$.
To begin with, we inductively construct $\psi_n\in \ide{F}_2$ so that, putting $\psi=\sum_{m\ge2}\psi_m$ and $\phi:=\exp(-\psi)\in \exp(\ide{F}_2)$, we obtain $\Phi=\phi^{-1}x_1\phi$.
For $n=3,4$, since $U_3=U_4=0$, set $\psi_2:=f(\Phi^{(3)})$ and $\psi_3:=f(\Phi^{(4)})$.
Assume $\psi_2,\dots,\psi_{n-2}$ have been constructed. Then $U_{n+1}$ is determined by $\psi_2,\dots,\psi_{n-2}$.
By the assumption $(\Phi^{(n+1)}-U_{n+1})^{00}=0$ and Proposition \ref{prop:Schneps}, there exists
\[
\psi_n:=f\!\bigl(\Phi^{(n+1)}-U_{n+1}\bigr)\in \ide{F}_2
\quad\text{with}\quad
\ad(x_1)(\psi_n)=\Phi^{(n+1)}-U_{n+1}.
\]
This completes the inductive construction of $\psi$ and $\phi=\exp(-\psi)$.

Next, we prove that for $n\ge3$,
\[
(\phi^{-1}x_1\phi\bigr)^{(n)}=\Phi^{(n)}.
\]
By \eqref{eq:weight-n-expansion} and the definition of $U_n$, it follows that
\[
\bigl(\exp(-\psi)\,x_1\,\exp(\psi)\bigr)^{(n)}
=\ad(x_1)(\psi_{\,n-1})+U_n
=\bigl(\Phi^{(n)}-U_n\bigr)+U_n
=\Phi^{(n)}.
\]
So we have $\Phi=\phi^{-1}x_1\phi$.

Conversely, let $\Phi=\exp(-\psi)\,x_1\,\exp(\psi) \in \Fad(\Q)$ with
$\psi=\sum_{m\ge 2}\psi_m$.
Then \eqref{eq:weight-n-expansion} yields
\[
\Phi^{(n)} - U_n=\ad(x_1)(\psi_{n-1}),
\]
so $(\Phi^{(n)} - U_n)^{00}=0$ for all $n\ge 3$.
\end{proof}

\begin{rmk}
Considering coefficient expansion, Proposition \ref{prop:Adjoint} holds over any $\Q$-algebra $R$.
\end{rmk}

We now refine Question~\ref{qu:Jarossay}.

\begin{qu}\label{qu:Jarossay}
Is there a natural isomorphism of affine group schemes $\Ad^R(x_1) : \DMR_0 \rightarrow \AdDMR_0 \times_{\TM_1} \FAd$?
\end{qu}

Here, the fiber product over $\TM_1$ is an intersection in the sense of affine schemes. Note that $\AdDMR_0$ and $\FAd$ is emmbeded in $\TM_1$.
As in the previous subsection, we consider tangent spaces of $\DMR_0$ and $\AdDMR_0 \times_{\TM_1} \FAd$.
We now state the question in terms of tangent spaces.

\begin{qu}\label{qu:Jarossay-Lie}
Is the $\Q$-linear map $\ad(x_1) : \dmr \rightarrow \addmr \cap \Fad ; \psi \mapsto [x_1,\psi]$ isomorphic?
\end{qu}
Comparing dimensions of the homogeneous components of $\dmr$ and $\addmr\cap\FAd$ we obtain
\begin{center}
\begin{tabular}{c|ccccccccccccc}
$k$ & $0$ & $1$ & $2$ & $3$ & $4$ & $5$ & $6$ & $7$ & $8$ & $9$ & $10$ & $11$ & $\cdots$ \\
\hline
$\dim \dmr^{(k)}$ & $0$ & $0$ & $0$ & $1$ & $0$ & $1$ & $0$ & $1$ & $1$ & $1$ & $1$ &  & $\cdots$ \\
$\dim \addmr^{(k)} \cap \Fad$
& $0$ & $0$ & $0$ & $0$ & $1$ & $0$ & $1$ & $0$ & $1$ & $1$ & $1$ & $1$ & $\cdots$ \\
\end{tabular}
\end{center}


\ifdefined\isMainDocument
\else
    %\bibliographystyle{amsplain}
    %\bibliography{reference-adjoint}
    \end{document}
\fi