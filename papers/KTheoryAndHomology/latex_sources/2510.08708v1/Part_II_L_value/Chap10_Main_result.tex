\begin{notation}
    We use the same notations as Section \ref{Sec: zeta integral}. 
\end{notation}

\subsection{Connection to automorphic $L$-functions}
\label{SS: connection to auto L-function}
In this subsection, we state the main result in terms of automorphic $L$-functions. 

\par Recall that we let $n = a + b + r + s$. Let $\pi = \pi_{f} \otimes \pi_{\infty}$ be a \textit{generic} cuspidal automorphic representation of $G(\AAA)$ whose central character is
\begin{align*}
    \omega_{\pi} \colon Z_{G}(\QQ) & \backslash Z_{G}(\AAA) \rightarrow \CC^{\times} \\
                             & z \longmapsto |z|^{-n} \nu(z), 
\end{align*}
where $\nu$ is a finite-order Hecke character of sign $(-1)^{n}$. Moreover, let the archimedean component $\pi_{\infty}$ of $\pi$ be a discrete series with Blattner parameter $(1+s-r+b, -1-a+s-r, s-r)$. The cuspidal automorphic representation $\pi$ can be viewed as a direct summand of
\begin{equation*}
    \mathrm{H}_{B, !}^{2}(\Sh_G, V(2))_{\CC} := \mathrm{Im}(\mathrm{H}^{2}_{\dR,c}(\Sh, V(2))_{\CC} \rightarrow \mathrm{H}^{2}_{\dR}(\Sh_G,V(2))_{\CC}, 
\end{equation*}
which is the interior cohomology of the Shimura variety of infinite level with coefficients in the local system associated to the representation $V(2)$. Let $E(\pi_f)$ be the rational field of $\pi$, which is a number field. 

\par Let $M_{B}(\pi_f, V(2))$ (see Definition \ref{def: M_B and M_dR}) be the $\pi_f$-isotypic subspace of $\mathrm{H}_{B, !}^{2}(\Sh_G, V(2))$. Similarly, let $M_{\dR}(\pi_f, V(2))$ (see Definition \ref{def: M_B and M_dR}) be the $\pi_f$-isotypic subspace of 
\begin{equation*}
    \mathrm{H}_{\dR, !}^{2}(\Sh_G, V(2)) := \mathrm{Im}(\mathrm{H}^{2}_{\dR,c}(\Sh, V(2)) \rightarrow \mathrm{H}^{2}_{\dR}(\Sh_G,V(2)).
\end{equation*}

\par Let $\mathcal{K}(V(2))$ (see equation (\ref{eq: K(pi, V)})) be the $\QQ[G(\AAA_f)]$-submodule of $\Ext^{1}_{\mathrm{MHS}_{\RR}^{+}}(1, \mathrm{H}^{2}_{B,!}(\Sh_G,V(2))_{\RR})$ generated by   the image of $\mathcal{E}is^{n}_H$ with domain $\mathcal{B}_{n, \QQ}$. Let $\mathcal{K}(\pi_f, V(2))$ be the $\pi_f$-isotypic subspace of $\mathcal{K}(V(2))$ which is a $1$-dimensional $E(\pi_f)$-subspace of the rank one $E(\pi_f)\otimes_{\QQ}\RR$-module $\Ext^{1}_{\mathrm{MHS}_{\RR}^{+}}(1, M_{B}(\pi_f, V(2))_{\RR})$. In $\Ext^{1}_{\mathrm{MHS}_{\RR}^{+}}(1, M_{B}(\pi_f, V(2))_{\RR})$, there is another $E(\pi_f)$-subspace $\mathcal{D}(\pi_f, V(2))$ called the Deligne $E(\pi_f)$-structure (see Definition \ref{def: Delign-rational structure}) coming from the comparison between Betti and de Rham realizations of the motive. The main result in this section gives a measure of the difference between $\mathcal{K}(\pi_f,V(2))$ and $\mathcal{D}(\pi_f,V(2))$ in terms of a non-critical automorphic $L$-value. 

\begin{notation}
    We use the notation $|\cdot|_{f}$ to denote the product over all finite absolute values:
    \begin{equation*}
        \prod_{v < \infty} |\cdot|_{v}. 
    \end{equation*}
\end{notation}

\begin{lemma}
\label{Lemma: relate L-function of contra of pi to L-function of pi}
    Let $S$ be a finite set of places of $\QQ$ (excluding $p = 2$ and $p = \infty$) such that $\pi_{p}$ is unramified for $p \notin S$ and let 
    \begin{equation*}
            L^{S}(s, \pi |\mu|_{f}^{-2}, \mathrm{std}) = \prod_{p \notin S} L_{p}(s, \pi_p |\mu|_{p}^{-2}, \mathrm{std})
    \end{equation*}
    be the partial standard $L$-function of $\pi|\mu|_{f}^{-1}$. Then we have for $s \in \CC$
    \begin{equation*}
            L^{S}(s, (\tilde{\pi} \times \nu_1 )|\mu|_{f}^{-2}, \mathrm{std}) = L^{S}(s, \pi, \mathrm{std}). 
    \end{equation*}
\end{lemma}

\begin{proof}
    First, let $\eta$ be the endomorphism on $E_{p}^{3}$ that maps $(x_1, x_2, x_3) \in E_p^{3}$ to $(\overline{x_1}, \overline{x_2}, -\overline{x_3})$.
    It induces an endomorphism on $G(\Qp)$ and $H(\Qp) = \GL_2(\Qp) \boxtimes E_{p}^{\times}$ by 
    \begin{align*}
        & \eta \colon \begin{pmatrix}
                    a_{11} & a_{12} & a_{13} \\
                    a_{21} & a_{22} & a_{23} \\
                    a_{31} & a_{32} & a_{33} 
                \end{pmatrix} \in G(\Qp) \mapsto \begin{pmatrix}
                                                    \overline{a_{11}} & \overline{a_{12}} & -\overline{a_{13}} \\
                                                        \overline{a_{21}} & \overline{a_{22}} & -\overline{a_{23}} \\
                                                        -\overline{a_{31}} & -\overline{a_{32}} & \overline{a_{33}} 
                                                   \end{pmatrix}, \\
        & \eta \colon (\begin{pmatrix}
                    a & b \\
                    c & d
                \end{pmatrix}, z) \in \GL_2(\Qp) \boxtimes E_{p}^{\times} \mapsto (\begin{pmatrix}
                                                                                        a & -b \\
                                                                                        -c & d
                                                                                    \end{pmatrix}, \overline{z}).
    \end{align*}
    Hence, the action of $\eta$ on $\GL_{2}(\Qp)$ is given by conjugating by $t = \begin{pmatrix}
                                                                                        1 & 0 \\
                                                                                        0 & -1
                                                                                    \end{pmatrix}$. 
    \par Second, it follows from \cite[Proposition 7.9]{PS18} that if we define $\pi_p^{\eta}(g) = \pi_p(\eta g \eta^{-1})$ and let $\omega_{\pi_p}$ be the central character of $\pi_p$, then we have 
        \begin{equation*}
            \tilde{\pi}_p \cong \pi_p^{\eta} \otimes (\omega_{\pi_p}^{-1}|_{Z(\Qp)} \circ \mu).
        \end{equation*}
        This shows that 
    \begin{align*}
        (\tilde{\pi}_p \times \nu_1) |\mu|_{f}^{-2} &\cong \pi_p^{\eta} \otimes (\nu_{1} \circ \mu)  \otimes (\omega_{\pi_p}^{-1}|_{Z(\Qp)} \circ \mu) |\mu|_{f}^{-2} \\  
            &\cong \pi_p^{\eta}, 
    \end{align*}
    where the last isomorphism holds for $\nu_1 \circ \mu = (\omega_{\pi_p}|_{Z(\Qp)} \circ \mu) |\mu|^{2}$. 
    \par Third, by the unramified computation carried out in \cite[\S 3.4 and \S 3.6]{PS18}, we have that for unramified $W_p, \Phi_p$ and $\nu_p$, 
    \begin{align*}
        L_{p}(s, \pi_p, \mathrm{std}) &= I_{p}(W_p, \Phi_p, \nu_p, s) \\ 
        &= \int_{U_{2}(\QQ_p) \backslash H(\QQ_p)}  \Phi_{p}((0,1)g_{1,p}) W_{p}(g_p) |\det(g_{1,p})|_{p}^{s}dg_{p}. 
    \end{align*}
    Hence, for $W_{p}$ associated to $\pi_{p}$, we have 
    \begin{align*}
        L_{p}(s, (\tilde{\pi}_p \times \nu_1) |\mu|_{f}^{-2}, \mathrm{std}) & = L_{p}(s, \pi_p^{\eta}, \mathrm{std}) \\
        & = \int_{U_{2}(\QQ_p) \backslash H(\QQ_p)}  \Phi_{p}((0,1)g_{1,p}) W_{p}(tg_pt^{-1}) |\det(g_{1,p})|_{p}^{s}dg_{p} \\ 
        & = \int_{U_{2}(\QQ_p) \backslash H(\QQ_p)}  \Phi_{p}((0,1)tg_{1,p}t^{-1}) W_{p}(g_p) |\det(g_{1,p})|_{p}^{s}dg_{p} \\
        & = \int_{U_{2}(\QQ_p) \backslash H(\QQ_p)}  \Phi_{p}((0,1)g_{1,p}) W_{p}(g_p) |\det(g_{1,p})|_{p}^{s}dg_{p} \\
        & = L_{p}(s, \pi_{p}, \mathrm{std}), 
    \end{align*}
    where the fourth identity holds for unramified $\Phi_{p}$. 
\end{proof}

\begin{theorem}
\label{Thm: auto_L-value}
Let $n = a + b + r + s$. Under the conditions
\begin{enumerate}
     \item $0 \le -r \le a$ and  $0 \le -s \le b$, 
     \item $a + r \neq b + s$, 
     \item $a > 0$ and $b > 0$,
     \item $r \neq 0$ or $s \neq 0$, 
\end{enumerate}
the relation between $\mathcal{K}(\pi_f,V(2))$ and $\mathcal{D}(\pi_f,V(2))$ is as follows: 
\begin{itemize}
    \item if $a \equiv b + 1\, (\mathrm{mod} \, 2)$, then
        \[
            \mathcal{K}(\pi_f,V(2)) =  W(\pi)  c^{-}(\pi_f, V(2)) \pi^{-(2a + 2b + r + s + 3)} W_{0, b + s - a - r}(8\sqrt{2}\pi D^{-\frac{3}{4}})^{-1} L(n + 1, \pi, \mathrm{std}) \mathcal{D}(\pi_f, V(2));
        \]
    \item if $a \equiv b \, (\mathrm{mod} \, 2)$, then 
        \[
            \mathcal{K}(\pi_f,V(2)) =  W(\pi)  c^{-}(\pi_f, V(2)) \pi^{-(2a + 2b + r + s + 4)} W_{0, b + s - a - r}(8\sqrt{2}\pi D^{-\frac{3}{4}})^{-1} L(n + 1, \pi, \mathrm{std}) \mathcal{D}(\pi_f, V(2)).
        \]
\end{itemize}
The notations in the above formulas are as follows: 
\begin{itemize}
    \item $W(\pi) \in \CC^{\times}/\overline{\QQ}^{\times}$ is a non-zero constant up to non-zero algebraic numbers depending on the normalization of the Whittaker{\textendash}Fourier coeffcients of automorphic forms belongs to $\pi$, 
    \item $W_{0,b + s -a - r}$ is the classical Whittaker function on $\RR$, 
    \item $\mathrm{c}^{-}(\pi_f, V(2))$ is the Deligne period \cite[\S 1.7]{Deligne79} of the motive $M(\pi_f, V(2))$, which is non-zero, 
    \item $L(s, \pi, \mathrm{std})$ is the standard $L$-function of the automorphic representation $\pi$. 
\end{itemize}
\end{theorem}

\begin{proof}
    \par First, it follows from Corollary \ref{corollary: Omega pairing in terms of omega paring with rho} that 
    \begin{equation*}
         \langle \Omega, \tilde{v}_{K} \rangle_{B} = \langle \omega_{\Psi}, [\rho] \rangle_{B} \otimes \mathbf{1}. 
    \end{equation*}
    By equation (\ref{eq:pairing}), $\langle \omega_{\Psi}, [\rho] \rangle_{B}$ is equal to 
    \begin{equation}
    \label{fomrula: summation}
         -\frac{1}{2(a + b + r + s + 1)} \sum\limits_{j = A}^{B} (-1)^{b + r + s} (2j - b -r - s + 1) C_j\int \Xi_{1 + a + 2b - 2j + r + s, a + b + r + s - 2j}(\phi_f).
    \end{equation}
     Corollary \ref{corollary: identify integral of Xi and I} implies that 
    \begin{equation*}
     \int \Xi_{1 + a + 2b - 2j + r + s, a + b + r + s - 2j}(\phi_f) =  c \cdot I(\varphi, \Phi, \nu, 1 + a + b + r + s), 
    \end{equation*}
    for $\Phi = \Phi_f \otimes \Phi_{\infty}, \varphi = \Psi_{f} \otimes X^{1 + a + 2b - 2j + r + s}_{(1, -1, 0)}{\Psi}_{\infty}, \nu = (\nu_1 = |\cdot|^{-n} \nu, \nu_2 = 1)$ and 
    \begin{equation}
    \label{eq: chap10 formula for c}
        c = (-1)^{j} 2i \pi^{1 +a + b + r + s} \Gamma(1 + a + b + r + s)^{-1}, 
    \end{equation}
    in which 
    \begin{equation*}
        \Phi_{\infty}(x, y) := (ix - y)^{j}(ix + y)^{a + b + r + s + j} e^{-\pi(x^2 + y^2)}. 
    \end{equation*}
    By Remark \ref{remark: arch_zeta_integral}, only the term $j = b + s$ is left in the summation of formula (\ref{fomrula: summation}). Hence, the pairing $\langle \omega_{\Psi}, [\rho] \rangle_{B}$ is equal to
    \begin{equation*}
        \frac{(-1)^{b + s + r + 1}(b + s - r + 1)}{2(a + b + r + s + 1)} C_{b + s} \cdot c \cdot I(\varphi, \Phi, \nu, 1 + a + b + r + s),
    \end{equation*}
    for $\Phi = \Phi_f \otimes \Phi_{\infty}, \varphi = \Psi_{f} \otimes X^{1 + a + r - s}_{(1, -1, 0)}{\Psi}_{\infty}, \nu = (\nu_1 = |\cdot|^{-n} \nu, \nu_2 = 1)$ 
    in which 
    \begin{equation*}
        \Phi_{\infty}(x, y) := (ix - y)^{b + s}(ix + y)^{a + r + 2b + 2s} e^{-\pi(x^2 + y^2)}. 
    \end{equation*}

    
    \par Second, it follows from Proposition \ref{Prop:factorize} that the pairing $\langle \omega, [\rho] \rangle_{B}$ equals 
    \begin{align*}
         & \frac{(-1)^{b + r + s + 1}(b + s - r + 1)}{2(a + b + r + s + 1)}  c \cdot C_{b + s} W_{\varphi}(1)  \prod_{p \in S} I_{p}(W_{\varphi, p}, \Phi_p, \nu_p, 1 + a + b + r + s) \\ 
        &\times I_{\infty}(W_{\varphi, \infty}, \Phi_\infty, \nu_\infty, 1 + a + b + r + s) L^{S}(1 + a + b + r + s, (\tilde{\pi} \times \nu_1 )|\mu|_{f}^{-2}, \mathrm{std}). 
    \end{align*} 
    By the definition of the Whittaker period (see Definition \ref{def: Whittaker period}), we can see that 
    \begin{equation*}
        W_{\varphi}(1) \sim W(\pi). 
    \end{equation*}
    It follows from Lemma \ref{Lemma:C_nonvanish} and Lemma \ref{Lemma: relate L-function of contra of pi to L-function of pi} that 
    \begin{align*}
        \langle \omega_{\Psi}, [\rho] \rangle_{B}  \sim c \cdot W(\pi)  \prod_{p \in S} I_{p}(W_{\varphi, p}, \Phi_p, \nu_p, 1 + a + b + r + s) & I_{\infty}(W_{\varphi, \infty}, \Phi_\infty, \nu_\infty, 1 + a + b + r + s)  \\
        & \times L^{S}(a + b + r + s + 1, \pi, \mathrm{std}). 
    \end{align*}
    Then by Proposition \ref{Prop:algebracity} and Corollary \ref{corollary: local L-factor algebraicity}, if we make a careful choice of $W_{\varphi, p}$ and $\Phi_p,$ according to the proposition, we have 
    \begin{equation*}
        \langle \omega_{\Psi}, [\rho] \rangle_{B}  \sim c \cdot W(\pi) I_{\infty}(W_{\varphi, \infty}, \Phi_\infty, \nu_\infty, 1 + a + b + r + s)L(a + b + r + s + 1, \pi, \mathrm{std}). 
    \end{equation*}
    \par Third, we can see from the archimedean computation in Proposition \ref{Prop:Arch_Zeta} that 
    \begin{align*}
        \langle \omega_{\Psi}, [\rho] \rangle_{B}  \sim & c \cdot W(\pi)  \pi^{-(3a + 3b + 2r + 2s + 5)} W_{0, b + s - a - r}(8\sqrt{2}\pi D^{-\frac{3}{4}})^{-1} \Gamma(a + b + r + s + 1)\\
                & \times \Gamma( \frac{1}{2}a + \frac{3}{2}b + s + 2) \Gamma(\frac{1}{2}a + \frac{1}{2} b + r + 2) L(a + b + r + s + 1, \pi, \mathrm{std}). 
    \end{align*}
    By the formula \ref{eq: chap10 formula for c} for $c$, we have 
    \begin{align*}
         \langle \omega_{\Psi}, [\rho] \rangle_{B}  \sim & W(\pi)  \pi^{-(2a + 2b + r + s + 4)} W_{0, b + s - a - r}(8\sqrt{2}\pi D^{-\frac{3}{4}})^{-1} \\
                        & \times \Gamma( \frac{1}{2}a + \frac{3}{2}b + s + 2) \Gamma(\frac{1}{2}a + \frac{1}{2} b + r + 2) L(a + b + r + s + 1, \pi, \mathrm{std}). 
    \end{align*}
    Recall that we have the following two formulas for $\Gamma$-values: for $n \in \ZZ_{\ge 0}$, 
    \begin{align*}
        & \Gamma(\frac{1}{2} + n) = \frac{(2n)!}{4^{n}n!} \sqrt{\pi}, \\
        & \Gamma(n) = n!. 
    \end{align*}
    So it can be seen that if $a \equiv b + 1\, (\mathrm{mod} \, 2)$,
    \begin{align*}
        &  \Gamma\left( \frac{1}{2}a + \frac{3}{2}b + s + 2\right) \sim \pi^{\frac{1}{2}}, \\ 
        &  \Gamma\left(\frac{1}{2}a + \frac{1}{2} b + r + 2\right) \sim \pi^{\frac{1}{2}}; 
    \end{align*}
    and if $a \equiv b \, (\mathrm{mod} \, 2)$,  
     \begin{align*}
        & \Gamma\left( \frac{1}{2}a + \frac{3}{2}b + s + 2\right)  \in \ZZ,  \\ 
        & \Gamma\left(\frac{1}{2}a + \frac{1}{2} b + r + 2\right)  \in \ZZ.  
    \end{align*}
    Hence, we can get that if $a \equiv b + 1\, (\mathrm{mod} \, 2)$
    \begin{equation*}
         \langle \omega_{\Psi}, [\rho] \rangle_{B}    \sim W(\pi)  \pi^{-(2a + 2b + r + s + 3)} W_{0, b + s - a - r}(8\sqrt{2}\pi D^{-\frac{3}{4}})^{-1} L(a + b + r + s + 1, \pi, \mathrm{std});
    \end{equation*}
    and if 
    $a \equiv b \, (\mathrm{mod} \, 2)$
    \begin{equation*}
         \langle \omega_{\Psi}, [\rho] \rangle_{B}    \sim W(\pi)  \pi^{-(2a + 2b + r + s + 4)} W_{0, b + s - a - r}(8\sqrt{2}\pi D^{-\frac{3}{4}})^{-1} L(a + b + r + s + 1, \pi, \mathrm{std}). 
    \end{equation*}
    \par Finally, combine it with the formula 
    \begin{equation*}
        \mathcal{K}(\pi_{f}, V(2)) = \frac{\langle \Omega, \tilde{v}_{K} \rangle_{B}}{\langle \Omega, \tilde{v}_{D} \rangle_{B}} \mathcal{D}(\pi_f, V(2)). 
    \end{equation*}
    and Proposition \ref{prop: periods}, 
    we complete the proof. 
\end{proof}

\begin{remark}
\label{remark: after auto L-value}
    \begin{itemize}
        \item 
        The condition $(1)$ guarantees that we have the branching from $G$ to $H$. The condition $(2)$ makes sure that we are in the weight $\le -3$ situation of Beilinson's conjectures (see \cite[Conjecture (6.1)]{Nekovar94}). The condition $(3)$ implies the local system associated to $V$ be regular so that we have the vanishing theorem. Condition $(4)$ guarantees that we have the vanishing on the boundary theorem. 
        \item Under the condition of the theroem, it follows from Proposition \ref{Prop: twisted base change} and \cite[Page 151]{GJ72} that the point $s = 1 + n = 1 + a + b + r + s$ lies in the region of absolute convergent of $L(s, \pi, \mathrm{std})$. 
    \end{itemize}
\end{remark}

\subsection{Connection to motivic $L$-functions}
\label{SS: conenction to motivic L-function}
In this subsection, we relate the automorphic $L$-functions to motivic $L$-functions and compare Theorem \ref{Thm: auto_L-value} with Beilinson's conjectures. 

\begin{notation}
    \begin{itemize}
        \item Recall that for any $V \in \mathrm{Rep}_{\QQ}(G)$, there is a canonical $l$-adic local system $\mu_{l}(V)$ on $S$ (with infinite level structure). We use $V$ to denote the $l$-adic local system. 
        \item We can define $l$-adic cohomology $\mathrm{H}_{\text{\'et}}^{2}(S_{\overline{\QQ}}, V(2))$, compactly supported $l$-adic cohomology $\mathrm{H}_{\text{\'et}, c}^{2}(S_{\overline{\QQ}}, V(2))$ and interior $l$-adic cohomology 
        \begin{equation*}
            \mathrm{H}_{\text{\'et}, !}^{2}(S_{\overline{\QQ}}, V(2)) := \mathrm{Im}(\mathrm{H}_{\text{\'et}, c}^{2}(S_{\overline{\QQ}}, V(2)) \rightarrow \mathrm{H}_{\text{\'et}}^{2}(S_{\overline{\QQ}}, V(2))). 
        \end{equation*}
    \end{itemize}
\end{notation}

\begin{remark}
    The interior cohomology $\mathrm{H}_{!,\text{\'et}}^{2}(S_{\overline{\QQ}}, V(2))$ is isomorphic to the intersection cohomology $\mathrm{IH}^{2}_{\text{\'et}}(S^{*}_{\overline{\QQ}}, V(2))$, where $S^{*}$ is the Baily-Borel compactification of $S$. (see \cite[Corollary 2.3.3.4]{CavicchiThesis}.) 
\end{remark}

\begin{definition}
       The comparison isomorphism 
                \begin{equation*}
                    \mathrm{H}_{\text{\'et}, !}^{2}(S_{\overline{\QQ}}, V(2)) \cong \mathrm{H}_{B, !}^{2}(S(\CC), V(2)) \otimes_{\QQ} \QQ_{l}, 
                \end{equation*}
                suggests we define the following $E(\pi_f) \otimes_{\QQ} \QQ_{l}$-module as in Definition \ref{def: M_B and M_dR}:
                \begin{equation*}
                     M_{\text{\'{e}t}}(\pi_{f}, V(2)) := \Hom_{\QQ[G(\AAA_f)]}(\Res_{E(\pi_f)/\QQ}\pi_f, \mathrm{H}_{\text{\'{e}t}, !}^{2}(S_{\overline{\QQ}}, V(2))). 
                \end{equation*}
                This has an action of the absolute Galois group $\Gal(\overline{\QQ}/\QQ)$ induced by the action of $\Gal(\overline{\QQ}/\QQ)$ on $S_{\overline{\QQ}}$. 
\end{definition}

\begin{remark}
    The $\Gal(\overline{\QQ}/\QQ)$-module $M_{\text{\'{e}t}}(\pi_{f}, V(2))$ is the $l$-adic \'{e}tale realization of the Grothendieck motive associated to $\pi$ constructed in \cite[Theorem 5.6]{Wild_15}. 
\end{remark}

\begin{definition}[Motivic $L$-function]
\label{Def: mot L-function}
    Define $L(M_{\text{\'{e}t}}(\pi_{f}, V(2)), s)$, the motivic $L$-function for $s \in \CC$, as the following partial Euler product (without archimedean factor): 
    \begin{equation*}
        L(M_{\text{\'{e}t}}(\pi_{f}, V(2)), s) := \prod_{v < \infty} L_{v}(M_{\text{\'{e}t}}(\pi_{f}, V(2)), s), 
    \end{equation*}
    where 
    \begin{itemize}
        \item if $v = p \neq l$, we define the local $L$-factor as 
              \begin{equation*}
                  L_{v}(M_{\text{\'{e}t}}(\pi_{f}, V(2)), s) := \det(1 - Fr_{p} p^{-s} | M_{\text{\'{e}t}}(\pi_{f}, V(2))^{I_{p}})^{-1}, 
              \end{equation*}
              and $I_{p}$ is the inertia group at the prime $p$ and $Fr_{p}$ is geometric Frobenius at $p$; 
        \item if $v = l$, we choose a prime $l^{\prime} \neq l$ and use $l^{\prime}$ to define $M_{\text{\'{e}t}}(\pi_{f}, V(2))$ and the motivic $L$-factor $L_{v}(M_{\text{\'{e}t}}(\pi_{f}, V(2)), s)$. It makes sense by the \textit{independence of $l$} in Proposition \ref{Prop: indep of l}. 
        %\item if $v = l$, we define the local $L$-factor as
        %      \begin{equation*}
        %          L_{v}(M_{\text{\'{e}t}}(\pi_{f}, V(2)), s) := \det(1 - Fr_{l} l^{-s} | D_{crys} (M_{\text{\'{e}t}}(\pi_{f}, V(2))|_{G_{l}}))^{-1}
        %      \end{equation*}
        %      here $Fr_{l}$ is geometric Frobenius at $l$, $G_{l}$ is the local Galois group $\Gal(\overline{\QQ_{l}} / \QQ_{l})$ at $l$ and 
        %      the Fontaine functor $D_{cris}$ is defined as follows: for $l$-adic finite dimensional $G_{l}$ representation $V$, 
        %      \begin{equation*}
        %          D_{cris}(V): = (V \otimes_{\QQ_{l}} B_{cris})^{G_{l}}
        %      \end{equation*}
        %      here $B_{cris}$ is the crystalline period ring defined by Fontaine (see \cite[Definition 7.7.]{FonatineOuyang22}). 
    \end{itemize}
\end{definition}

\begin{convention}
    As explained in Remark \ref{remark: pairing}, we can view the motivic $L$-function as 
    \begin{equation*}
        L_{v}(M_{\text{\'{e}t}}(\pi_{f}, V(2)), s) \cdot \mathbf{1}
    \end{equation*}
    where $\mathbf{1}$ is unit of $E(\pi_f) \otimes_{\QQ} \overline{\QQ_{l}}$. Hence, we now view $L_{v}(M_{\text{\'{e}t}}(\pi_{f}, V(2)), s)$ as scalar-valued. 
\end{convention}

\begin{proposition}
\label{Prop: indep of l}
    For any finite place $v$, the $L$-factor $L_{v}(M_{\text{\'{e}t}}(\pi_{f}, V(2)), s)$ is a polynomial in $\frac{1}{(Nv)^{s}}$ with coefficients in $\overline{\QQ}$.
    In other words, the definition of the $L$-factor is independent of the choice of $l$. 
\end{proposition}

\begin{proof}
    This follows from a theorem of Gabber (see \cite[Theorem 1]{Fujiwara00}). 
\end{proof}

Now if we fix an embedding $\overline{\QQ} \hookrightarrow \CC$, then the $L$-factor $L_{v}(M_{\text{\'{e}t}}(\pi_{f}, V(2)), s)$ makes sense as a $\CC$-valued function. 

\begin{proposition}
\label{Prop: convergence region of motivic L-function}
    The infinite product in the definition of $L(M_{\text{\'{e}t}}(\pi_{f}, V(2)), s)$ is absolutely convergent when $\mathrm{Res}(s) > \frac{-a - b - r - s}{2}$. 
\end{proposition}

\begin{proof}
    It follows from the fact the weight of $M_{\text{\'{e}t}}(\pi_{f}, V(2))$ is $-2 - a - b - r - s$ (see Corollary \ref{Corollary: Hodge decomp for motives}) and \cite[(1.5)]{Nekovar94}. 
\end{proof}

\begin{proposition}
\label{Prop:rel_L_ftn}
        If we let $L^{S}(M_{\text{\'{e}t}}(\pi_f, V(2)), s)$ be the partial motivic $L$-function away from finitely many bad primes in $S$,  then we have 
        \begin{equation*}
            L^{S}(M_{\text{\'{e}t}}(\pi_f, V(2)), s) = L^{S}(s + n + 1, \pi, std),
        \end{equation*}
        for $s \in \CC$. 
\end{proposition}

\begin{proof}
     In our setting, we use the ``homological normalization" of $V(2)$, but in \cite{LR92}, ``cohomological normalization" is used. Hence, the motivic $L$-function in \cite{LR92} is the 
     \begin{equation*}
        L^{S}(M_{\text{\'{e}t}}(\pi_f, V(2))(-2-n), s)
    \end{equation*}
     in our setting. It is proved in \cite[Theorem A, P291]{LR92} that 
     \begin{equation*}
          L^{S}(M_{\text{\'{e}t}}(\pi_f, V(2))(-2-n), s) = L^{S}(s - 1, \pi, \mathrm{std}).
     \end{equation*}
     Hence, we have 
     \begin{equation*}
         L^{S}(M_{\text{\'{e}t}}(\pi_f, V(2)), s - n - 2) =  L^{S}(M_{\text{\'{e}t}}(\pi_f, V(2))(-2-n), s) = L^{S}(s - 1, \pi, \mathrm{std}). 
     \end{equation*}
     This implies
     \begin{equation*}
          L^{S}(M_{\text{\'{e}t}}(\pi_f, V(2)), s) = L^{S}(s + n + 1, \pi, \mathrm{std}). 
     \end{equation*}
\end{proof}

\begin{remark}
    \par Let us check the compatibility of motivic normalization with automorphic normalization. 
    \par First, we check from the motivic side. It follows from the weight of $M_{B}(\pi_f, V(2))$ is $w = -2 - n$ that\footnote{The notation ``$\leftrightarrow$'' means that equality holds after multiply by the remaining Euler factors.}
    \begin{equation*}
        L^{S}(M_{\text{\'{e}t}}(\pi_f, V(2)), s) \leftrightarrow L^{S}(M_{\text{\'{e}t}}(\pi_f, V(2)), -1 - n - s). 
    \end{equation*}
    Then by Proposition \ref{Prop:rel_L_ftn}, we can see that  
    \begin{align*}
        & L^{S}(M_{\text{\'{e}t}}(\pi_f, V(2)), s) = L^{S}(s + n + 1, \pi, \mathrm{std}), \\ 
        & L^{S}(M_{\text{\'{e}t}}(\pi_f, V(2)), -1 - n - s) = L^{S}(-s, \pi, \mathrm{std}). 
    \end{align*}
    So by the  argument in Lemma \ref{Lemma: relate L-function of contra of pi to L-function of pi}, we get
    \begin{equation}
    \label{eq: ftn_motiv}
        L^{S}(s + n + 1, \pi, \mathrm{std}) = L^{S}(1 + s, \tilde{\pi}, \mathrm{std}) \leftrightarrow L^{S}(-s, \pi, \mathrm{std}).
    \end{equation}
    Hence, the center of $L(s, \pi, \mathrm{std})$ is at $s = \frac{1}{2}$. 
    
    \par Second, we check from the automorphic side. 
    The central character of $\pi$ is 
    \begin{equation*}
        z \in Z(\AAA_\QQ) \mapsto |z|^{-n} \nu, 
    \end{equation*}
    where $\nu$ is a finite order Hecke character of sign $(-1)^{n}$. Hence, the automorphic representation $\pi |\mu|^{\frac{n}{2}}$ is unitary. 
    Therefore, by the argument in Lemma \ref{Lemma: relate L-function of contra of pi to L-function of pi}, we have 
    \begin{equation*}
        L^{S}(s, \pi, \mathrm{std}) = L^{S}(s, (\pi|\mu|^{\frac{n}{2}})|\mu|^{-\frac{n}{2}}, \mathrm{std}) = L^{S}\left(s - \frac{n}{2}, \pi|\mu|^{\frac{n}{2}}, \mathrm{std}\right). 
    \end{equation*}
    The automorphic representation $\pi|\mu|^{\frac{n}{2}}$ is unitary, so by \cite[(5.1), P163]{Gelbart&PS84}, we have 
    \begin{equation*}
        L^{S}\left(s - \frac{n}{2}, \pi|\mu|^{\frac{n}{2}}, \mathrm{std}\right) \leftrightarrow L^{S}\left(1 + \frac{n}{2} - s, \tilde{\pi}|\mu|^{-\frac{n}{2}}, \mathrm{std}\right). 
    \end{equation*}
    It can be seen from the  argument in Lemma \ref{Lemma: relate L-function of contra of pi to L-function of pi} that 
    \begin{equation*}
        L^{S}\left(1 + \frac{n}{2} - s, \tilde{\pi}|\mu|^{-\frac{n}{2}}, \mathrm{std}\right) = L^{S}(1  - s, \pi|\mu|^{\frac{n}{2}}, \mathrm{std}).
    \end{equation*}
    Thus
    \begin{equation}
         L^{S}(s, \pi, \mathrm{std}) \leftrightarrow  L^{S}(1 - s, \pi, \mathrm{std}),
    \end{equation}
    which means the center of $L(s, \pi, \mathrm{std})$ is at $s = \frac{1}{2}$. Hence, the automorphic normalization is compatible with the motivic normalization. 
\end{remark}

Using the previous proposition, we can state our main theorem in terms of motivic $L$-functions. 
\begin{theorem}
\label{Them_mot_L_function}
    Let $n = a + b + r + s$. Under the conditions 
    \begin{enumerate}
         \item $0 \le -r \le a$ and  $0 \le -s \le b$,     
         \item $a + r \neq b + s$, 
         \item $a > 0$ and $b > 0$,
         \item $r \neq 0$ or $s \neq 0$, 
    \end{enumerate}
    the relation between $\mathcal{K}(\pi_f,V(2))$ and $\mathcal{D}(\pi_f,V(2))$ is as follows: 
\begin{itemize}
    \item if $a \equiv b + 1\, (\mathrm{mod} \, 2)$, then
        \[
            \mathcal{K}(\pi_f,V(2)) =  W(\pi)  c^{-}(\pi_f, V(2)) \pi^{-(2a + 2b + r + s + 3)} W_{0, b + s - a - r}(8\sqrt{2}\pi D^{-\frac{3}{4}})^{-1} L(M_{\text{\'{e}t}}(\pi_f, V(2)), 0) \mathcal{D}(\pi_f, V(2));
        \]
    \item if $a \equiv b \, (\mathrm{mod} \, 2)$, then 
        \[
            \mathcal{K}(\pi_f,V(2)) =  W(\pi)  c^{-}(\pi_f, V(2)) \pi^{-(2a + 2b + r + s + 4)} W_{0, b + s - a - r}(8\sqrt{2}\pi D^{-\frac{3}{4}})^{-1}  L(M_{\text{\'{e}t}}(\pi_f, V(2)), 0) \mathcal{D}(\pi_f, V(2)).
        \]
\end{itemize}
The notations in the above formulas are as follows: 
\begin{itemize}
    \item $W(\pi) \in \CC^{\times}/\overline{\QQ}^{\times}$ is a non-zero constant up to nonzero algebraic numbers depending on the normalization of the Whittaker{\textendash}Fourier coefficients of automorphic forms belonging to $\pi$, 
    \item $W_{0,b + s -a - r}$ is the classical Whittaker function on $\RR$, 
    \item $\mathrm{c}^{-}(\pi_f, V(2))$ is the Deligne period \cite[\S 1.7]{Deligne79} of the motive $M(\pi_f, V(2))$ which is non-zero, 
    \item $L(M_{\text{\'{e}t}}(\pi_f, V(2)), s)$ is the motivic $L$-function of the automorphic representation $\pi$. 
\end{itemize}
\end{theorem}

\begin{proof}
    Since for a finite place $v$ of $\QQ$, the value of local $L$-factor $L_{v}(M_{\text{\'{e}t}}(\pi_f, V(2), s)$ at $s = 0$ and the value of $L_{v}(s, \pi, \mathrm{std})$ at $s = 1 + a + b + r + s$ belongs to $\overline{\QQ}^{\times}$, we only need to consider the partial $L$-functions. By Proposition \ref{Prop:rel_L_ftn}, we have 
    \begin{equation*}
        L^{S}(M_{\text{\'{e}t}}(\pi_f, V(2)), 0) = L^{S}(1 + n, \pi, std).
    \end{equation*}
\end{proof}

\begin{remark}
    The point $s = 0$ of the motivic $L$-function $L(M(\pi_f, V(2)), s)$ is exactly what Beilinson conjectured in his conjectures \textnormal{\cite[Conjecture (6.1)]{Nekovar94}}, and the shape of the theorem is similar to the rank $1$ case of the weak Beilinson's conjectures \cite[(6.6)]{Nekovar94}. It only remains to prove 
    \begin{itemize}
        \item if $a \equiv b + 1\, (\mathrm{mod} \, 2)$, then
        \begin{equation*}
            W(\pi)  c^{-}(\pi_f, V(2)) \pi^{-(2a + 2b + r + s + 3)} W_{0, b + s - a - r}(8\sqrt{2}\pi D^{-\frac{3}{4}})^{-1} \in \overline{\QQ}^{\times};
        \end{equation*}
        \item if $a \equiv b \, (\mathrm{mod} \, 2)$, then 
        \begin{equation*}
            W(\pi)  c^{-}(\pi_f, V(2)) \pi^{-(2a + 2b + r + s + 4)} W_{0, b + s - a - r}(8\sqrt{2}\pi D^{-\frac{3}{4}})^{-1}  \in \overline{\QQ}^{\times}.
        \end{equation*}
    \end{itemize}
    We will leave this to future work. 
\end{remark}

\begin{corollary}
\label{corollary: nontrivial_class}
        The $E(\pi_f)$-rational structure $\mathcal{K}(\pi_f, V(2))$ is non-trivial. In other words, the morphism in Construction \ref{Construction: motivic classes}
              \begin{equation*}
                  \mathcal{E}is_{M}^{n} \colon \mathcal{B}_{n} \rightarrow \mathrm{H}_{M}^{3}(S, V(2))
              \end{equation*}
        is non-trivial.
\end{corollary}

\begin{proof}
        It follows from Proposition \ref{Prop: convergence region of motivic L-function} (or Remark \ref{remark: after auto L-value}) that 
        \begin{equation*}
            L(M(\pi_f, V(2)), 0) \neq 0. 
        \end{equation*}
        It follows from the fact that $W(\pi) \neq 0$, $ W_{0,r-s}(8\sqrt{2}D^{-\frac{3}{4}})^{-1} \neq 0$, $\mathrm{c}^{-}(\pi_f,V(2)) \neq 0$ and $\mathcal{D}(\pi_f, V(2))$ is non-trivial and Theorem \ref{Them_mot_L_function} that $\mathcal{K}(\pi_f,V(2))$ is non-trivial.
\end{proof}

\begin{remark} 
    \begin{itemize}
        \item Theorem \ref{Them_mot_L_function} is a $w \le -3$ counterpart of what is proved in \cite[Theorem 8.18]{PS18}, where they proved a case that the motivic sheaf $V$ is trivial and the weight of the motive is $-2$. 
        \item In \cite{LSZ22}, the authors constructed an Euler system for $\GU(2,1)$ based on the motivic class $\mathcal{E}is_{M}^n(\phi_f)$ for $\phi_f \in \mathcal{B}_{n}$ being non-trivial, so our theorem of non-triviality of $\mathcal{K}(\pi_f, V(2))$ answers a question raised in their paper, which gives a necessary condition for their Euler system to work. 
    \end{itemize}
\end{remark}
