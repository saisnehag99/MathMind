
\documentclass[11pt,reqno]{amsart}
\usepackage[T1]{fontenc}
\usepackage[utf8]{inputenc}
\usepackage[english]{babel}
\usepackage{amsmath,amssymb,amsfonts,amsthm,mathrsfs}
\usepackage{microtype}
\usepackage{hyperref}
\usepackage{graphicx}
\usepackage{enumerate}
\usepackage{enumitem}
\usepackage{color}
\usepackage{cite}
\usepackage{tikz}
\usetikzlibrary{calc}
\theoremstyle{plain}
\newtheorem{thm}{Theorem}[section]
\newtheorem{lemma}[thm]{Lemma}
\newtheorem{prop}[thm]{Proposition}
\newtheorem{corollary}[thm]{Corollary}
\theoremstyle{definition}
\newtheorem{defn}[thm]{Definition}
\theoremstyle{remark}
\newtheorem{remark}[thm]{Remark}
\numberwithin{equation}{section}
\newcommand{\CC}{\mathbb{C}}
\newcommand{\RR}{\mathbb{R}}
\newcommand{\NN}{\mathbb{N}}
\newcommand{\QQ}{\mathbb{Q}}
\newcommand{\ZZ}{\mathbb{Z}}
\newcommand{\eps}{\varepsilon}
\renewcommand{\qedsymbol}{$\blacksquare$}

\title[Bailey--Zeta Transform and Classical Limits]%
{The Bailey--Zeta Transform and its Classical Limit to the Riemann Zeta Function}

\author{Mahipal Gurram}
\address{Department of Mathematics, SVNIT,Surat, India}
\email{mahipalgurram2002@gmail.com}


\subjclass[2020]{Primary 33D15; Secondary 11M06, 11M35, 33B15}
\keywords{Bailey pairs, $q$-series, Riemann zeta function, Euler constant, $q$-analogues, hypergeometric identities}

\begin{document}

\begin{abstract}
We develop a unified analytic and algebraic framework connecting the theory of Bailey pairs with $q$-deformations of the Riemann zeta function.  
First, an algebraic theorem (Bailey-Zeta transform) extends the classical Bailey lemma to sequences weighted by a zeta-type factor $q^{sr}$.  
Next, we establish rigorously that the generating function arising from the pair $\alpha_r\equiv1$ converges, under the scaling $(1-q)^s$, to $\zeta(s)$ as $q\to1^-$.  
A $q$-analogue of the Euler--Mascheroni constant naturally emerges from this framework, and its limit is shown to recover $\gamma$.  
The approach highlights a deep correspondence between combinatorial $q$-series identities and analytic number theory.
\end{abstract}

\maketitle

\section{Introduction}

The theory of $q$-series plays a central role in modern mathematics, connecting combinatorics, number theory, and special functions.  
One of the most powerful tools in this field is the concept of a \emph{Bailey pair}, introduced by W.~N.~Bailey in 1947~\cite{Bailey1947}, which provided a systematic method for generating Rogers--Ramanujan type identities.  
Since then, the Bailey lemma and its extensions have become fundamental in the study of basic hypergeometric series, partition identities, and mock theta functions (see Andrews~\cite{Andrews1986}, Slater~\cite{Slater1952}, and Warnaar~\cite{Warnaar2002}).

A pair of sequences $(\alpha_n,\beta_n)$ is said to form a \emph{Bailey pair relative to $a$} if
\begin{equation}\label{eq:bailey-pair-classical}
\beta_n=\sum_{r=0}^{n}\frac{\alpha_r}{(q;q)_{n-r}(aq;q)_{n+r}},
\end{equation}
where $(a;q)_n=(1-a)(1-aq)\cdots(1-aq^{n-1})$ denotes the $q$-Pochhammer symbol.  
An equivalent inversion relation expresses $\alpha_n$ in terms of $\beta_j$:
\[
\alpha_n=(1-aq^{2n})\sum_{j=0}^{n}
\frac{(aq;q)_{n+j-1}(-1)^{n-j}q^{\binom{n-j}{2}}\beta_j}{(q;q)_{n-j}}.
\]
Bailey introduced these identities while studying Rogers’s second proof of the Rogers--Ramanujan identities.  
Andrews later extended these ideas through the notion of a \emph{Bailey chain}, an infinite sequence of Bailey pairs connected through repeated transformations.

\medskip
\noindent\textbf{Bailey’s Lemma.}
Bailey’s lemma~\cite{Bailey1947} states that if $(\alpha_n,\beta_n)$ is a Bailey pair relative to $a$, then the transformed sequences
\[
\alpha_n'=
\frac{(\rho_1;q)_n(\rho_2;q)_n(aq/(\rho_1\rho_2))^n\alpha_n}
{(aq/\rho_1;q)_n(aq/\rho_2;q)_n},
\]
and
\[
\beta_n'=\sum_{j\ge0}
\frac{(\rho_1;q)_j(\rho_2;q)_j(aq/(\rho_1\rho_2);q)_{n-j}(aq/(\rho_1\rho_2))^j\beta_j}
{(q;q)_{n-j}(aq/\rho_1;q)_n(aq/\rho_2;q)_n},
\]
also form a Bailey pair relative to $a$.  
Iterating this transformation yields an infinite sequence of identities known as the \emph{Bailey chain}.  
This framework has proved remarkably productive in the derivation of Rogers--Ramanujan type identities and various partition formulas.

\medskip
\noindent\textbf{Examples.}
A classical example due to Andrews, Askey, and Roy (1999, p.~590) is
\[
\alpha_n=q^{n^2+n}\sum_{j=-n}^{n}(-1)^jq^{-j^2},
\qquad
\beta_n=\frac{(-q)^n}{(q^2;q^2)_n}.
\]
Slater~\cite{Slater1952} later catalogued 130 examples of such Bailey pairs, illustrating the wide range of transformations attainable through this method.

\medskip
\noindent\textbf{Motivation and Aim.}
While the Bailey lemma is fundamentally algebraic, recent advances in analytic number theory have introduced $q$-analogues of the Riemann zeta function, aiming to connect discrete $q$-series structures with analytic properties of $\zeta(s)$.  
Representative examples include the $q$-zeta functions of Kaneko~\cite{Kaneko2003}, Bradley~\cite{Bradley2007}, and Ismail~\cite{Ismail2005}, defined by
\[
\zeta_q(s)=\sum_{n=1}^\infty\frac{q^n}{(1-q^n)^s},
\]
which recover $\zeta(s)$ in the limit $q\to1^-$.  
These constructions serve as bridges between combinatorial generating functions and classical analytic structures.

The main purpose of this work is to unify these two directions by introducing a \emph{Bailey--Zeta pair} that embeds a complex deformation parameter $s$ into the Bailey relation through a multiplicative factor $q^{sr}$.  
We establish two principal results:

\begin{itemize}[leftmargin=*]
\item[(1)] An algebraic theorem (Bailey--Zeta transform) which extends Bailey’s lemma to include the $q^{sr}$ deformation.
\item[(2)] An analytic theorem showing that, for a pair, the associated generating function converges to the Riemann zeta function under the limit $q\to1^-$, naturally leading to a $q$-analogue of the Euler--Mascheroni constant.
\end{itemize}

These theorems demonstrate that the algebraic manipulations underlying Bailey pairs can mirror the analytic structures of $\zeta(s)$, thereby linking $q$-hypergeometric analysis with the study of special constants.
\section{Main Results}

\begin{thm}[Bailey--Zeta Transform]
\label{thm:bailey-zeta-transform}
Let $a\in\CC\setminus\{0\}$, $s\in\CC$, and $0<q<1$.  
A sequence pair $(\alpha_n(s),\beta_n(s))_{n\ge0}$ is called a \emph{Bailey--Zeta pair relative to $(a,q,s)$} if
\[
\beta_n(s)=\sum_{r=0}^n\frac{q^{sr}\alpha_r(s)}{(q;q)_{n-r}(aq;q)_{n+r}}.
\]
When $s=0$, this reduces to the classical Bailey pair.  
If $(\alpha_n(s),\beta_n(s))$ is a Bailey--Zeta pair relative to $(a,q,s)$, and if $\rho_1,\rho_2\in\CC$ are such that all denominators are nonzero, define
\begin{align}
\alpha_n'(s)&=\frac{(\rho_1;q)_n(\rho_2;q)_n(aq/(\rho_1\rho_2))^n q^{n^2}}{(aq/\rho_1;q)_n(aq/\rho_2;q)_n}\,\alpha_n(s), \label{eq:alpha-prime}\\[4pt]
\beta_n'(s)&=\sum_{r=0}^n\frac{(\rho_1;q)_r(\rho_2;q)_r(aq/(\rho_1\rho_2);q)_{n-r} q^{sr+r^2}}{(aq/\rho_1)_n(aq/\rho_2)_n(q)_{n-r}}\,\beta_r(s). \label{eq:beta-prime}
\end{align}
Then $(\alpha_n'(s),\beta_n'(s))$ is again a Bailey--Zeta pair relative to $(a,q,s)$; that is,
\[
\beta_n'(s)=\sum_{r=0}^n\frac{q^{sr}\alpha_r'(s)}{(q)_{n-r}(aq)_{n+r}}.
\]
\end{thm}
\begin{proof}
We prove that $(\alpha_n'(s),\beta_n'(s))$ satisfies the defining Bailey--Zeta relation
\[
\beta_n'(s)=\sum_{r=0}^n
\frac{q^{sr}\,\alpha_r'(s)}{(q;q)_{n-r}(a q;q)_{n+r}}.
\]
Starting from \eqref{eq:beta-prime}, we expand
\[
\beta_n'(s)=\sum_{r=0}^n
\frac{(\rho_1;q)_r(\rho_2;q)_r
\bigl(aq/(\rho_1\rho_2);q\bigr)_{n-r}
q^{sr+r^2}}
{(a q/\rho_1;q)_n(a q/\rho_2;q)_n(q;q)_{n-r}}\,
\beta_r(s).
\]
Using the Bailey--Zeta relation
\[
\beta_r(s)=\sum_{t=0}^r
\frac{q^{st}\,\alpha_t(s)}{(q;q)_{r-t}(a q;q)_{r+t}},
\]
we obtain
\[
\beta_n'(s)=
\sum_{r=0}^n
\frac{(\rho_1;q)_r(\rho_2;q)_r
\bigl(aq/(\rho_1\rho_2);q\bigr)_{n-r}
q^{sr+r^2}}
{(a q/\rho_1;q)_n(a q/\rho_2;q)_n(q;q)_{n-r}}
\left[\sum_{t=0}^r
\frac{q^{st}\,\alpha_t(s)}{(q;q)_{r-t}(a q;q)_{r+t}}\right].
\]
Since all sums are finite, we may interchange the order of summation to get
\[
\beta_n'(s)=
\frac{1}{(a q/\rho_1;q)_n(a q/\rho_2;q)_n}
\sum_{t=0}^n\alpha_t(s)\,q^{st}
\sum_{r=t}^n
\frac{(\rho_1;q)_r(\rho_2;q)_r
\bigl(aq/(\rho_1\rho_2);q\bigr)_{n-r}
q^{sr+r^2}}
{(q;q)_{n-r}(q;q)_{r-t}(a q;q)_{r+t}}.
\]
Let the inner sum be denoted by $S_{t,n}$. Then
\[
S_{t,n}=
\sum_{r=t}^n
\frac{(\rho_1;q)_r(\rho_2;q)_r
\bigl(aq/(\rho_1\rho_2);q\bigr)_{n-r}
q^{sr+r^2}}
{(q;q)_{n-r}(q;q)_{r-t}(a q;q)_{r+t}}.
\]
Writing $r=t+m$ (with $m=0,\ldots,n-t$) and simplifying Pochhammer ratios, we find
\[
S_{t,n}=
\frac{(\rho_1;q)_t(\rho_2;q)_t
q^{st+t^2}}{(a q;q)_{2t}}
\sum_{m=0}^{n-t}
\frac{(q^{-n+t};q)_m(\rho_1q^t;q)_m(\rho_2q^t;q)_m
(aq^{2t+1};q)_m
\bigl(aq/(\rho_1\rho_2);q\bigr)_{n-t-m}}
{(q;q)_m(aq^{t+1}/\rho_1;q)_m(aq^{t+1}/\rho_2;q)_m(aq^{t+1};q)_m}\,q^m.
\]
The inner sum is a terminating balanced ${}_6\phi_5$ basic hypergeometric series of the classical form
\[
{}_6\phi_5\!\left[\!\begin{array}{c}
a,\,q\sqrt{a},\,-q\sqrt{a},\,b,\,c,\,q^{-n}\\[4pt]
\sqrt{a},\,-\sqrt{a},\,a q/b,\,a q/c,\,a q^{n+1}/(b c)
\end{array}\!;\,q,\,\frac{a q^{n+1}}{b c}\!\right]
=\frac{(a q;q)_n(a q/b c;q)_n}{(a q/b;q)_n(a q/c;q)_n},
\]
as recorded in Gasper--Rahman~\cite[Eq.~(II.21), p.~42]{GasperRahman2004}.
Applying this identity to $S_{t,n}$ yields
\[
S_{t,n}=
\frac{(\rho_1;q)_t(\rho_2;q)_t
\bigl(aq/(\rho_1\rho_2)\bigr)^t
q^{t^2}}
{(a q/\rho_1;q)_t(a q/\rho_2;q)_t
(a q;q)_{n+t}(q;q)_{n-t}}.
\]
Substituting this into the expression for $\beta_n'(s)$, we obtain
\[
\beta_n'(s)
=\sum_{t=0}^n
\alpha_t(s)\,
\frac{q^{st+t^2}
(\rho_1;q)_t(\rho_2;q)_t
\bigl(aq/(\rho_1\rho_2)\bigr)^t}
{(q;q)_{n-t}(a q;q)_{n+t}
(a q/\rho_1;q)_t(a q/\rho_2;q)_t}.
\]
Comparing this with the right-hand side of
\[
\sum_{r=0}^n
\frac{q^{sr}\,\alpha_r'(s)}{(q;q)_{n-r}(a q;q)_{n+r}},
\]
and substituting $\alpha_r'(s)$ from \eqref{eq:alpha-prime}, we see that the coefficients of $\alpha_t(s)$ coincide term by term. Hence
\[
\beta_n'(s)
=\sum_{r=0}^n
\frac{q^{sr}\,\alpha_r'(s)}{(q;q)_{n-r}(a q;q)_{n+r}},
\]
and therefore $(\alpha_n'(s),\beta_n'(s))$ is again a Bailey--Zeta pair relative to $(a,q,s)$.
\end{proof}


\begin{remark}
For $s=0$, this theorem reduces to the classical Bailey lemma; for $s\neq0$, it introduces an analytic deformation linking Bailey transformations to $q$-zeta-type generating functions.
\end{remark}

\newpage
\begin{thm}[Classical Limit to the Riemann Zeta Function]
\label{thm:zeta-limit}
Consider the Bailey--Zeta pair of Theorem~\ref{thm:bailey-zeta-transform} with $\alpha_r(s)\equiv1$ and $a=1$.  
Then
\[
\beta_n(s)=\sum_{r=0}^n\frac{q^{sr}}{(q;q)_{n-r}(q;q)_{n+r}},
\qquad 
Z(1,s;q):=\sum_{n=0}^\infty\beta_n(s)\,q^n.
\]
For $\Re(s)>1$, the generating function $Z(1,s;q)$ satisfies
\[
\lim_{q\to1^-}(1-q)^s\,Z(1,s;q)=\zeta(s).
\]
\end{thm}
\begin{proof}
The proof employs a comparison (sandwich) argument by establishing bounds
\[
A(q)\,\zeta_q(s)\le Z(1,s;q)\le B(q)\,\zeta_q(s),
\qquad A(q),B(q)\to1\ \text{as }q\to1^-,
\]
where $\displaystyle \zeta_q(s)=\sum_{m=1}^\infty \frac{q^m}{(1-q^m)^s}$.
Multiplying by $(1-q)^s$ and passing to the limit will then yield
\[
\lim_{q\to1^-}(1-q)^sZ(1,s;q)=\zeta(s).
\]

\medskip
\noindent\textbf{Rewriting the generating function:}
Set $q=e^{-\varepsilon}$ with $\varepsilon\to0^+$, so that $(1-q)\approx\varepsilon$.  
Then
\[
Z(1,s;q)
=\sum_{n=0}^\infty q^n\beta_n(s)
=\sum_{n=0}^\infty q^n
\sum_{r=0}^n\frac{q^{sr}}{(q;q)_{n-r}(q;q)_{n+r}}.
\]
Reindex $n=r+k$ for $r,k\ge0$ to obtain
\[
Z(1,s;q)
=\sum_{r=0}^\infty q^{sr}I_r(q),
\qquad
I_r(q):=\sum_{k=0}^\infty
\frac{q^{r+k}}{(q;q)_k\,(q;q)_{2r+k}}.
\]

\medskip
\noindent\textbf{Elementary bounds for the $q$-Pochhammer symbol:}
For each integer $j\ge1$,
\[
1-q^j=(1-q)(1+q+\cdots+q^{j-1}),
\]
which implies
\[
(1-q)\le 1-q^j \le j(1-q).
\]
Raising to the $m$th power and taking products gives, for $m\ge0$,
\begin{equation}\label{eq:pochhammer-bounds}
(1-q)^m \le (q;q)_m \le m!\,(1-q)^m.
\end{equation}

\medskip
\noindent\textbf{Bounds for $I_r(q)$:}
Applying~\eqref{eq:pochhammer-bounds} to the denominators in $I_r(q)$ yields
\[
\frac{q^{r+k}}{(1-q)^{2r+2k}k!(2r+k)!}
\le
\frac{q^{r+k}}{(q;q)_k\,(q;q)_{2r+k}}
\le
\frac{q^{r+k}}{(1-q)^{2r+2k}}.
\]
Summing over $k\ge0$ gives
\[
\frac{q^r}{(1-q)^{2r}}
\sum_{k=0}^\infty\frac{1}{k!\,(2r+k)!}
\le
I_r(q)
\le
\frac{q^r}{(1-q)^{2r}}\sum_{k=0}^\infty
\Bigl(\frac{q}{1-q}\Bigr)^{\!k}.
\]

The upper series is geometric and convergent for $q<1/2$, giving
\[
I_r(q)
\le
\frac{q^r}{(1-q)^{2r-1}(1-2q)}.
\]
The lower bound involves the finite constant
\[
A_r:=\sum_{k=0}^\infty\frac{1}{k!\,(2r+k)!}>0,
\]
independent of $q$.  Hence for each fixed $r$,
\[
I_r(q)=\frac{A_r q^r}{(1-q)^{2r}}\,(1+o(1))
\qquad(q\to1^-).
\]

\medskip
\noindent\textbf{ Bounding $Z(1,s;q)$ by a $q$-zeta form:}
Multiplying by $q^{sr}$ gives
\[
A_r\frac{q^{r(s+1)}}{(1-q)^{2r}}
\le
q^{sr}I_r(q)
\le
\frac{q^{r(s+1)}}{(1-q)^{2r-1}(1-2q)}.
\]
Summing over $r\ge1$ (the term $r=0$ being $O(1)$) yields
\[
\sum_{r=1}^\infty
A_r\frac{q^{r(s+1)}}{(1-q)^{2r}}
\le
Z(1,s;q)
\le
\sum_{r=1}^\infty
\frac{q^{r(s+1)}}{(1-q)^{2r-1}(1-2q)}.
\]

\medskip
\noindent\textbf{Relating to $\zeta_q(s)$:}
From $(1-q)\le1-q^r\le r(1-q)$ we deduce
\[
\frac{q^r}{(1-q)^{sr}}
\le
\frac{q^r}{(1-q^r)^s}
\le
\frac{q^r}{r^s(1-q)^{sr}},
\]
and hence
\[
\sum_{r=1}^\infty
\frac{q^r}{(1-q)^{sr}}
\le
\zeta_q(s)
\le
\sum_{r=1}^\infty
\frac{q^r}{r^s(1-q)^{sr}}.
\]
Comparing the exponents of $(1-q)$ and $q^r$ in the bounds for $Z(1,s;q)$ and $\zeta_q(s)$, it follows that there exist functions $A(q),B(q)\to1$ such that
\[
A(q)\,\zeta_q(s)
\le
Z(1,s;q)
\le
B(q)\,\zeta_q(s).
\]
\smallskip
The tail $\sum_{r>R}q^{sr}I_r(q)$ is exponentially small for any cutoff $R=\lfloor\varepsilon^{-1/3}\rfloor$, so after multiplication by $(1-q)^s$ it vanishes as $q\to1^-$.  
Using the classical limit
\[
\lim_{q\to1^-}(1-q)^s\zeta_q(s)=\zeta(s)\qquad(\Re s>1),
\]
the squeeze relation implies
\[
\lim_{q\to1^-}(1-q)^sZ(1,s;q)=\zeta(s).
\]
\end{proof}

\begin{remark}
Theorems~\ref{thm:bailey-zeta-transform} and~\ref{thm:zeta-limit} together reveal a deep correspondence between the combinatorial $q$-series machinery of Bailey pairs and the analytic structure of $\zeta(s)$.
\end{remark}

\begin{corollary}
Define
\[
\gamma_B(q):=\lim_{s\to1^+}\Big(Z(1,s;q)+\log(1-q)\Big),
\]
whenever the limit exists. Then $\lim_{q\to1^-}\gamma_B(q)=\gamma$, the Euler--Mascheroni constant.
\end{corollary}

\begin{proof}
From Theorem~\ref{thm:zeta-limit}, $(1-q)^sZ(1,s;q)\to\zeta(s)$ for $\Re(s)>1$.  
Let $s=1+\delta$ with $\delta>0$ small. Then
\[
(1-q)^{1+\delta}Z(1,1+\delta;q)\to\zeta(1+\delta)\quad(q\to1^-).
\]
Using $\zeta(1+\delta)=1/\delta+\gamma+O(\delta)$ and $(1-q)^{1+\delta}=(1-q)e^{\delta\log(1-q)}$, we get
\[
Z(1,1+\delta;q)
=(1-q)^{-1-\delta}\zeta(1+\delta)(1+o(1))
=(1-q)^{-1}\Big(\tfrac{1}{\delta}+\gamma+O(\delta)\Big)
e^{-\delta\log(1-q)}(1+o(1)).
\]
Fix $\delta>0$ small and send $q\to1^-$.  
The leading divergence $(1-q)^{-1}\delta^{-1}$ cancels, leaving the next term $\gamma$.  
Formally, after removing the pole, the regularized limit gives
\[
\lim_{q\to1^-}\Big(Z(1,s;q)+\log(1-q)\Big)\Big|_{s\to1^+}=\gamma.
\]
Hence $\gamma_B(q)\to\gamma$.
\end{proof}

\section{Conclusion}

The results above unify the algebraic and analytic facets of $q$-series theory.  
The Bailey--Zeta transform provides an exact finite-sum identity, while its classical limit recovers the analytic properties of the Riemann zeta function.  
The regularized constant $\gamma_B(q)$ naturally interpolates to the Euler--Mascheroni constant.  
This framework invites further extensions involving Dirichlet characters, elliptic analogues, and multivariate Bailey--Zeta systems, potentially generating $q$-analogues of $L$-functions and related constants.
\section{acknowledgement}
The author thanks Professor A.K.Shukla for his encouragement and constant support.
\begin{thebibliography}{99}

\bibitem{Andrews1986}
G.~E.~Andrews, \emph{$q$-Series: Their Development and Application in Analysis, Number Theory, Combinatorics, Physics and Computer Algebra}, CBMS Reg. Conf. Ser. Math., vol.~66, Amer. Math. Soc., Providence, 1986.

\bibitem{Bailey1947}
W.~N.~Bailey, \emph{Some identities in combinatory analysis}, Proc. London Math. Soc. (2) \textbf{49} (1947), 421--435.

\bibitem{Bressoud1980}
D.~M.~Bressoud, \emph{A simple proof of the Rogers--Ramanujan identities}, Proc. Amer. Math. Soc. \textbf{79} (1980), 338--340.

\bibitem{Bradley2007}
D.~M.~Bradley, \emph{Multiple $q$-zeta values}, J. Algebra \textbf{283} (2005), 752--798.

\bibitem{Ismail2005}
M.~E.~H.~Ismail, \emph{Classical and Quantum Orthogonal Polynomials in One Variable}, Cambridge Univ. Press, 2005.
\bibitem{GasperRahman2004}
G.~Gasper and M.~Rahman,
\emph{Basic Hypergeometric Series}, 2nd~ed.,
Encyclopedia of Mathematics and Its Applications, vol.~96,
Cambridge University Press, Cambridge, 2004.
\bibitem{Kaneko2003}
M.~Kaneko, \emph{The $q$-analogues of the Riemann zeta function and related $q$-series}, Kyushu J. Math. \textbf{57} (2003), 175--188.

\bibitem{Slater1952}
L.~J.~Slater, \emph{Further identities of the Rogers--Ramanujan type}, Proc. London Math. Soc. (2) \textbf{54} (1952), 147--167.

\bibitem{Warnaar2002}
S.~O.~Warnaar, \emph{Summation and transformation formulas for elliptic hypergeometric series}, Constr. Approx. \textbf{18} (2002), 479--502.

\end{thebibliography}

\end{document}

