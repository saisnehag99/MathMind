\newpage

%====================================================================
% 9. Conclusion
%====================================================================

\section*{Conclusion}

We implemented a scheme combining the three-fold broad geometry
(Broad--BCT, \S\ref{sec:broad3}), the \emph{Robust\,Kakeya} block (\S\ref{sec:kakeya}),
the kernel analysis with $12$-fold integration by parts (\S\ref{sec:kernel}),
and the \emph{narrow cascade} (\S\ref{sec:narrow}). 
The \emph{Robust\,Kakeya} and \emph{narrow} regimes are used
mutually exclusively; the tube-packing block (\S\ref{sec:tube-pack}) 
is also applied only when the high-density conditions fail.

For the kernel block we obtained the estimate
\[
   \|K\|_{L^{2}\to L^{2}} \ \lesssim\ \lambda^{-9/2}D^{-3},
   \qquad D=\lambda^{1/12},
\]
which, together with the contributions of the other blocks, yields in the summary balance
\[
   \sigma_\lambda=-\frac{2557}{576} \approx -4.44,
   \qquad
   \sigma_D=-3
\]
(see \S\ref{subsec:balance-table-final}).
Both cumulative exponents are negative, which gives an improvement
over the trivial scaling benchmark.

The appendices contain geometric and algebraic lemmas
(App.~\ref{app:det}), technical estimates 
(App.~\ref{app:C3}--\ref{app:C4}), and comments on the variant with 
$x$–dependent amplitude (App.~\ref{app:damp}),
which makes the arguments self-contained.

\medskip
\noindent
Thus the asserted estimates in Theorem~\ref{thm:main} 
are fully justified, and the regime separation and balance
of exponents can be used in a broader context of space–time decompositions. 
The approach naturally extends to a wide class of decoupling problems, in particular,
to anisotropic variants for other hypersurfaces, and may serve as a starting point for new
“robust’’ and “narrow’’ scenarios in problems of geometric harmonic analysis.

