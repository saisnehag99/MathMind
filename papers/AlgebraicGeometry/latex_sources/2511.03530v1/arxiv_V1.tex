

\documentclass[english,reqno]{amsart}


%standard packages
%------------------------------
%The following three packages are all included in amsart
\usepackage{amsmath}
\usepackage{amssymb} %to produce blackboard bold characters(\supsetneq)
%\usepackage{amsthm} %to define theorem-like structures
%-------------------------------
\usepackage[new]{old-arrows}
\usepackage{enumitem} %lay­out of the three ba­sic list en­vi­ron­ments
\usepackage{mathtools} %an extension package to amsmath + loading amsmath
\usepackage[table]{xcolor} %color of tables
\usepackage[all]{xy} %xymatrix commutative diagram
\usepackage{tikz} %pictures
\usepackage{tikz-cd}%implies
%\usepackage{geometry} %define page dimensions
\usepackage{indentfirst} %Indent first paragraph
\usepackage{babel} %lan­guages
\usepackage{setspace} %set­ting the spac­ing be­tween lines
\usepackage{stmaryrd}


%setting of hyperlinks
\usepackage[colorlinks,linkcolor=red,anchorcolor=green,citecolor=blue]{hyperref} %color of hyperlinks
\hypersetup{linktocpage = true} %color of hyperlinks of TOC

%special packages and notations (rarely used)
\usepackage{rotating} %Rotate an image, table or paragraph
\def\no{n{\raisebox{0.2ex}{\textsuperscript{o}}}} %numbering


%various tables and their settings
\usepackage{ytableau} %young tableau
\usepackage{longtable} %long table
\newcolumntype{M}[1]{>{\centering\arraybackslash}m{#1}} %define dimension for long stable

 
%setting of fonts
\usepackage{mathpazo}
\usepackage{mathrsfs}
\DeclareFontFamily{OMS}{rsfs}{\skewchar\font'60}
\DeclareFontShape{OMS}{rsfs}{m}{n}{<-5>rsfs5 <5-7>rsfs7 <7->rsfs10 }{}
\DeclareSymbolFont{rsfs}{OMS}{rsfs}{m}{n}
\DeclareSymbolFontAlphabet{\scr}{rsfs}
\DeclareSymbolFontAlphabet{\scr}{rsfs}

% a beautiful font (with XeLatex)
\usepackage[T1]{fontenc}
%\usepackage[utf8]{inputenc}
%\usepackage[no-math]{fontspec}
%\setromanfont[Ligatures={Common}, Numbers={OldStyle}, Variant=01]{Linux Libertine O}

%pagestyle
\pagestyle{plain}%header empty; the footer contains the page number.
%\geometry{left=2.5cm,right=2.5cm,top=2.5cm,bottom=2cm} %dimension of pages
%\renewcommand{\baselinestretch}{1.2} %line spacing
%\setlength{\parindent}{0in} %indent
\sloppy %avoiding overfulls


%mathcal letters
\def\tr{{\rm tr}}
\newcommand{\ric}{\textnormal{Ric}}
\newcommand\cA{{\mathcal A}}
\newcommand\cB{{\mathcal B}}
\newcommand\cC{{\mathcal C}}
\newcommand\cD{{\mathcal D}}
\newcommand\cE{{\mathcal E}}
\newcommand\cF{{\mathcal F}}
\newcommand\cG{{\mathcal G}}
\newcommand\cH{{\mathcal H}}
\newcommand\cI{{\mathcal I}}
\newcommand\cK{{\mathcal K}}
\newcommand\cL{{\mathcal L}}
\newcommand\cN{{\mathcal N}}
\newcommand\cM{{\mathcal M}}
\newcommand\cO{{\mathcal O}}
\newcommand\cP{{\mathcal P}}
\newcommand\cQ{{\mathcal Q}}
\newcommand\cR{{\mathcal R}}
\newcommand\cS{{\mathcal S}}
\newcommand\cT{{\mathcal T}}
\newcommand\cU{{\mathcal U}}
\newcommand\cV{{\mathcal V}}
\newcommand\cW{{\mathcal W}}
\newcommand\cX{{\mathcal X}}
\newcommand{\ddbar}{\sqrt{-1}\partial\bar\partial}
%mathbb letters
\newcommand\bbB{{\mathbb B}}
\newcommand\bbC{{\mathbb C}}
\newcommand\bbF{{\mathbb F}}
\newcommand\bbK{{\mathbb K}}
\newcommand\bbN{{\mathbb N}}
\newcommand\bbP{{\mathbb P}}
\newcommand\bbQ{{\mathbb Q}}
\newcommand\bbR{{\mathbb R}}
\newcommand\bbS{{\mathbb S}}
\newcommand\bbZ{{\mathbb Z}}

%mathscr letters
\newcommand\sA{{\mathscr A}}
\newcommand\sB{{\mathscr B}}
\newcommand\sD{{\mathscr D}}
\newcommand\sE{{\mathscr E}}
\newcommand\sF{{\mathscr F}}
\newcommand\sG{{\mathscr G}}
\newcommand\sI{{\mathscr I}}
\newcommand\sL{{\mathscr L}}
\newcommand\sM{{\mathscr M}}
\newcommand\sN{{\mathscr N}}
\newcommand\sO{{\mathscr O}}
\newcommand\sQ{{\mathscr Q}}
\newcommand\sT{{\mathscr T}}
\newcommand\sV{{\mathscr V}}
\newcommand\sH{{\mathscr H}}


%Special Abb.
\newcommand{\id}{{\rm id}}
\newcommand{\rk}{{\rm rk}}
\newcommand{\codim}{{\rm codim}}
\newcommand{\Cl}{{\rm Cl}}
\newcommand{\im}{{\rm im}}
\newcommand{\Bs}{{\rm Bs}}
\newcommand{\locus}{\rm locus}
\newcommand{\Sec}{\rm Sec}
\newcommand{\defect}{\rm def}
\newcommand{\Aut}{\rm Aut}
\newcommand{\Ext}{\rm Ext}
\newcommand{\Lag}{\rm Lag}
\newcommand{\Gr}{\rm{Gr}}
\newcommand{\Br}{{\rm Br}}

\newcommand{\ID}{{\rm{I}d}}

%Objects in math
\DeclareMathOperator*{\bs}{Bs}
\DeclareMathOperator*{\Irr}{Irr}
\DeclareMathOperator*{\Sym}{Sym}
\DeclareMathOperator*{\pic}{Pic}
\DeclareMathOperator*{\reg}{reg}
\DeclareMathOperator*{\Sing}{Sing}
\DeclareMathOperator*{\Pic0}{Pic^0}
\DeclareMathOperator*{\Nm}{Nm}
\DeclareMathOperator*{\red}{red}
\DeclareMathOperator*{\num}{num}
\DeclareMathOperator*{\norm}{norm}
\DeclareMathOperator*{\nons}{nons}
\DeclareMathOperator*{\lin}{lin}
\DeclareMathOperator*{\supp}{Supp}
\DeclareMathOperator*{\NS}{NS}
\DeclareMathOperator*{\Exc}{Exc}
\DeclareMathOperator*{\PGL}{PGL}

%Moduli spaces and objects attacted
\newcommand{\NEX}{\overline{\mbox{NE}}(X)}
\newcommand{\NAX}{\overline{\mbox{NA}}(X)}
\newcommand{\NE}[1]{\ensuremath{\overline{\mbox{NE}}(#1)}}
\newcommand{\NA}[1]{\ensuremath{\overline{\mbox{NA}}(#1)}}
\newcommand{\chow}[1]{\ensuremath{\mathcal{C}(#1)}}
\newcommand{\chown}[1]{\ensuremath{\mathcal{C}_n(#1)}}
\newcommand{\douady}[1]{\ensuremath{\mathcal{D}(#1)}}
\newcommand{\Chow}[1]{\ensuremath{\mbox{\rm Chow}(#1)}}
\newcommand{\Hilb}[1]{\ensuremath{\mbox{\rm Hilb}(#1)}}

\newcommand{\Univ}{\ensuremath{\mathcal{U}}}
\newcommand{\RC}[1]{\ensuremath{\mbox{\rm RatCurves}^n(#1)}}

%sheaves operators
\newcommand\Ctwo{\ensuremath{C^{(2)}}}
\newcommand{\Proj}[1]{\ensuremath{\mbox{\rm Proj}(#1)}}
\newcommand{\Hombir}[2]{\ensuremath{\mbox{$\rm{Hom}_1$}(#1,#2)}}
\newcommand{\cHom}[2]{\ensuremath{\mathcal{H}om_{\mathcal{O}_X}(#1,#2)}}
\newcommand{\sHom}[2]{\ensuremath{\mathscr{H}om_{\mathscr{O}_X}(#1,#2)}}
\newcommand{\cExt}{{\mathcal Ext}}
\newcommand{\cTor}{{\mathcal Tor}}
\newcommand{\defeq}{{\vcentcolon=}}

%math operators
\newcommand\bp{{\bar\partial}}

%Morphisms
\newcommand{\meom}[3]{\ensuremath{#1\colon #2\dashrightarrow #3}}
\newcommand{\morp}[3]{\ensuremath{#1\colon #2\rightarrow #3}}

%Theorem type newenviroment 
% Theorem type environments
\theoremstyle{plain}
\newtheorem{thm}{Theorem}[section]
\newtheorem{lemma}[thm]{Lemma}
\newtheorem{prop}[thm]{Proposition}
\newtheorem{cor}[thm]{Corollary}
\newtheorem{defn}[thm]{Definition}
\newtheorem{claim}[thm]{Claim}
\newtheorem{setup}[thm]{Setup}

\theoremstyle{remark}
\newtheorem{example}[thm]{Example}
\newtheorem{remark}[thm]{Remark}
\newtheorem{notation}[thm]{Notation}

\newcommand\dbar{{\overline{\partial}}}


\def\Cl{\operatorname{Cl}}
\def\div{\operatorname{div}}
\def\mod{\operatorname{mod}}
\def\deg{\operatorname{deg}}
\def\im{\operatorname{Im}}
\def\Supp{\operatorname{Supp}}
\def\dim{\operatorname{dim}}
\def\codim{\operatorname{codim}}
\def\chr{\operatorname{char}}
\def\Ex{\operatorname{Ex}}
\def\lct{\operatorname{lct}}
\def\Spec{\operatorname{Spec}}
\def\Proj{\operatorname{Proj}}
\def\Bl{\operatorname{Bl}}
\def\WDiv{\operatorname{WDiv}}
\def\Frac{\operatorname{Frac}}
\def\ker{\operatorname{ker}}
\def\coker{\operatorname{coker}}
\def\Pic{\operatorname{Pic}}
\def\Alb{\operatorname{Alb}}
\def\alb{\operatorname{alb}}
\def\NS{\operatorname{NS}}
\def\NE{\overline{\operatorname{NE}}}
\def\Tr{\operatorname{Tr}}
\def\min{\operatorname{min}}
\def\max{\operatorname{max}}
\def\Sup{\operatorname{Sup}}
\def\Inf{\operatorname{inf}}
\def\Eff{\operatorname{\overline{Eff}}}
\def\Mov{\operatorname{\overline{Mov}}}
\def\NM{\operatorname{\overline{NM}}}
\def\SNM{\operatorname{\overline{SNM}}}
\def\NF{\operatorname{\overline{NF}}}
\def\Nef{\operatorname{Nef}}
\def\NA{\operatorname{\overline{NA}}}
\def\Univ{\operatorname{Univ}}
\def\Hom{\operatorname{Hom}}
\def\ev{\operatorname{ev}}
\def\pt{\operatorname{pt}}
%\def\NDiv#1{\operatorname{N^1(#1)}}
%\def\NCur#1{\operatorname{N_1(#1)}}
\def\vol{\operatorname{vol}}
\def\hor{\operatorname{\hor}}
\def\ver{\operatorname{\ver}}
\def\Bs{\operatorname{Bs}}
\def\SBs{\operatorname{SBs}}
\def\mult{\operatorname{mult}}
\def\coeffi{\operatorname{coeffi.}}
\def\NT{\operatorname{Non-Terminal}}
\def\ex{\operatorname{ex}}
\def\nex{\operatorname{nex}}
\def\Center{\operatorname{Center}}
\def\gcd{\operatorname{gcd}}
\def\discrep{\operatorname{discrep}}
\def\pr{\operatorname{pr}}
\def\sm{\operatorname{\textsubscript{\rm sm}}}
\def\sing{\operatorname{\textsubscript{\rm sing}}}
%\def\orb{\operatorname{\textsubscript{orb}}}
\def\orb{\mathrm{orb}}
\def\ver{\operatorname{\textsubscript{\rm ver}}}
\def\hor{\operatorname{\textsubscript{\rm hor}}}

\def\red{\operatorname{\textsubscript{\rm red}}}
\def\Diff{\operatorname{Diff}}
\def\dis{\displaystyle}

\def\orb{\mathrm{orb}}

%itemize&enumerate-->nonindent
\setlist[itemize]{leftmargin=*}
\setlist[enumerate]{leftmargin=*}

%numbering of lists
\renewcommand{\labelenumi}{(\arabic{enumi})} %numbering of enumerate
\renewcommand{\labelenumii}{(\theenumi.\arabic{enumii})} %subenumerate
\numberwithin{equation}{section} %numbering of equations
\renewcommand\thesubsection{\thesection.\Alph{subsection}} %subsection

%depth of TOC
\setcounter{tocdepth}{1} %include till subsections in TOC

%quotient
%\newcommand{\bigslant}[2]{{\raisebox{.2em}{$#1$}\left/\raisebox{-.2em}{$#2$}\right.}}



\makeatletter

\ifnum\@ptsize=0 \addtolength{\hoffset}{-0.3cm} \fi \ifnum\@ptsize=2 \addtolength{\hoffset}{0.5cm} \fi 


%informations of paper
\title{Title} 
%\date{\today}

\subjclass[2010]{}
\keywords{}

 
\author{Fu Xin}
\address{School of sciences, Institute for Theoretic Sciences, Westlake University, Hangzhou, Zhejiang Province, 310030, China}
\email{fuxin54@westlake.edu.cn}

\author{Ou Wenhao}
\address{ Institute of Mathematics, Academy of Mathematics and Systems Science, Chinese Academy of Sciences, Beijing, 100190, China}
\email{wenhaoou@amss.ac.cn}
 

\begin{document}

\begin{abstract}
In \cite{Ou2024}, the orbifold Bogomolov-Gieseker  inequality is proved for a stable reflexive sheaf on a compact K\"ahler variety with klt singularities.  
In this paper,  we  give a characterization on the stable reflexive sheaf  when the Bogomolov-Gieseker equality holds. 
\end{abstract}


\title{Orbifold Bogomolov-Gieseker inequalities on compact K\"ahler varieties
%Bogomolov-Gieseker inequality for log terminal K\"ahler variety
}

\maketitle

\tableofcontents

%\vspace{-0.2cm} 
 

 


\section{Introduction}

 
The theory of  holomorphic vector bundles is a central object in complex algebraic geometry and complex analytic geometry. 
The notion of stable vector bundles on complete curves was introduced by Mumford  in  \cite{Mumford1963}. 
Such notion of stability was then extended to torsion-free sheaves on any projective manifolds (see \cite{Takemoto1972}, \cite{Gieseker1977}), and is now known as the slope stability. 
An  important property of stable vector bundles is the following Bogomolov-Gieseker inequality, involving the Chern classes of the vector  bundle.
\begin{thm}
\label{thm:BG-inequality-intro}
Let $Z$ be a projective manifold of dimension $n$,   let  $H$ be  an ample divisor, and let $\cF$ be a $H$-stable vector bundle of rank $r$ on $Z$. 
Then 
\[   \Big(c_2(\cF)-\frac{r-1}{2r}c_1(\cF)^2 \Big)  \cdot  H^{n-2} \geq  0. \]
\end{thm}
When  $Z$ is a surface, the inequality was proved in \cite{Bogomolov1978}. 
In higher dimensions, one may apply Mehta-Ramanathan   theorem in  \cite{MehtaRamanathan1981/82}   to  reduce to the case of surfaces, by taking hyperplane  sections. 
Later in \cite{Kawamata1992}, as a part of the proof for the three-dimensional abundance theorem, Kawamata extended the inequality to orbifold Chern  classes of  reflexive sheaves on projective surfaces with quotient singularities. 
The technique of taking hypersurface sections then  allows us to deduce Bogomolov-Gieseker inequalities for reflexive sheaves on projective varieties which have quotient singularities in codimension 2.   



On the analytic side, let $(Z,\omega)$ be a compact  K\"ahler manifold, and $(\cF, h)$ a Hermitian  holomorphic vector bundle on $Z$.  
L\"ubke proved that  if $h$ satisfies the Einstein condition, then the following inequality holds (see \cite{Lub82}),
\[   \int_Z \Big(c_2(\cF,h)-\frac{r-1}{2r}c_1(\cF,h)^2 \Big)  \wedge  \omega^{n-2} \geq 0. \] 
It is now well understood that if $\cF$ is slope stable, then it admits a Hermitian-Einstein metric. 
The case when $Z$ is a complete curve was proved by Narasimhan-Seshadri  in \cite{NarasimhanSeshadri1965}, the case of projective surfaces was proved by Donaldson in \cite{Donaldson1985}, and the case of arbitrary compact K\"ahler manifolds was proved by Uhlenbeck-Yau in \cite{UhlenbeckYau1986}.   
%An advantage of the analytic method is to give an insight to the equality condition. 
Simpson extended the existence of  Hermitian-Einstein  metric to stable Higgs bundles, on compact and certain non compact K\"ahler manifolds, see \cite{Simpson1988}. 
Furthermore,  in \cite{BandoSiu1994},  Bando-Siu introduced the notion of admissible metrics and proved the existence of admissible  Hermitian-Einstein  metrics on stable reflexive sheaves. 


For compact K\"ahler varieties which has quotient singularities,  an orbifold version of Donaldson-Uhlenbeck-Yau theorem was proved by Faulk in \cite{Faulk2022}.    
If the variety $Z$ has quotient singularities only in codimension 2,  
in \cite{Ou2024}, the second author constructed a projective bimeromorphic map $\rho\colon X\to Z$, so that $X$ has quotient singularities, and the indeterminacy locus of $\rho^{-1}$ has codimension at least $3$ in $Z$. 
By using Faulk's theorem on $X$, 
we can then deduce Bogomolov-Gieseker inequalities for orbifold Chern classes of stable reflexive sheaves on $Z$. 

When the variety $Z$ is smooth, the theorem of  Donald-Uhlenbeck-Yau also characterizes the condition when the equality holds in the Bogomolov inequalities.  
This part was not proved in \cite{Ou2024} for singular spaces. 
We focus on this problem in this paper, and  prove the following theorem.  
For the precise definition of the orbifold Chern classes $\hat{c}$, we refer to \cite[Section 9]{Ou2024}. 

\begin{thm}
\label{thm:unitary-flat} 
Let $(Z,\omega)$ be a compact K\"ahler variety of dimension $n$ with klt singularities, 
and let $\mathcal{F}$ be a $\omega$-stable reflexive sheaf on $Z$. 
Then the following two conditions are equivalent. 
\begin{enumerate}
    \item $\hat{c}_2(\mathcal{F})\cdot [\omega]^{n-2} = \hat{c}_1(\mathcal{F})^2  \cdot [\omega]^{n-2} = 0$.
    \item  There is a finite quasi-\'etale cover $p\colon Z'\to Z$, such that the reflexive pullback $(p^*\mathcal{F})^{**}$ is a unitary flat vector bundle. 
\end{enumerate}
\end{thm}

 

We outline the proof as follows. 
We follow the method of \cite{CGNPPW}, and  combine it with the work of \cite{GPSS}. 
By taking an appropriate bimeromorphic map $\rho\colon X\to Z$ constructed in \cite{Ou2024}, 
we may assume there is an orbifold structure $\mathfrak{X}$ on $X$, so that the pullback $\mathcal{E} := \rho^*\mathcal{F}$ induces an orbifold vector bundle $\mathcal{E}_{\orb}$ on $\mathfrak{X}$. 
Then there is a sequence of orbifold K\"ahler forms $\{\omega_i\}$, which converges to $\rho^*\omega_Z$. 
We may identify each $\omega_i$ as a K\"ahler current on $X$. 
Faulk's theorem implies that there is an orbifold Hermitian-Einstein metric $h_i$  on $\mathcal{E}$ with respect to $\omega_i$.  
If we can prove that such a sequence $\{h_i\}$ converges to some Hermitian-Einstein metric $h_\infty$ with respect to $\rho^*\omega_Z$, then we can deduce Theorem \ref{thm:unitary-flat} by classic curvature calculus.  
We notice that, the Einstein condition is expressed as an elliptic PDE. 
Therefore, if we can obtain certain uniform $L^\infty$ bounds on $\{h_i\}$, 
then we can conclude by  using classic elliptic analysis. 
In order to get such $L^\infty$ bounds, there are two main ingredients. 
The first one is uniform geometric estimates on $(X,\omega_i)$, which was essentially proved in a   series of recent groundbreaking  works: \cite{GPS24},  \cite{GPSS24} and \cite{GPSS}. 
Such estimates are highly non trivial, since the family $\{\omega_i\}$ is degenerating. 
The second main ingredient is essentially proved in the enlightening paper \cite{CGNPPW}. 
Up to renormalizing $h_i$, we can control $\|h_i\|_{L^\infty}$ by $\|h_i\|_{L^1}$,  uniformly in $i$.   
In the end, we can adapt the method of \cite{Simpson1988} to obtain the desired convergence. 






 
There are other approaches to  Bogomolov-Gieseker inequalities and Hermitian-Einstein metrics on singular spaces, see for example \cite{Wu21},   \cite{ChenWentworth2024},  \cite{Chen2025} and \cite{GuenanciaPaun2024}.  
In particular, when $X$ is a klt threefold,  \cite{GuenanciaPaun2024} 
proved  Bogomolov-Gieseker inequalities as well as 
the equality conditions, with a different method. On the other hand, in a very interesting recent preprint \cite{ZZZ2025} (c.f. see also \cite{LZZ17}), the orbifold Bogomolov-Gieseker  inequality is obtained for semi-stable Higgs sheaves on compact K\'ahler varieties with klt singularities. 
It will be interesting to characterize the equality condition there.

The paper is organized as follows. 
In Section \ref{section:preliminaries}, we introduce the notation for the paper, and recall some known results, such as finiteness of fundamental groups for klt singularities, 
and Simpson's operations in his paper \cite{Simpson1988}. 
In Section \ref{section:uniform-kahler}, we establish uniform geometric estimates for a degenerating family of orbifold smooth K\"ahler forms, following the method of \cite{GPSS}. 
In the last section, we establish  mean value type inequalities as in \cite{CGNPPW}, and finish the proof of the main theorem. 
\\


\noindent\textbf{Acknowledgment.} 
The authors are grateful to professor Bin Guo, Jian Song,  Chuanjing Zhang for conversations.  Xin Fu is  supported by National Key R\&D Program of China 2024YFA1014800 and NSFC No. 12401073. Wenhao Ou is supported by the National Key R\&D Program of China (No. 2021YFA1002300). 





 

\section{Preliminaries}\label{section:preliminaries}
We  fix some notation and prove some elementary results in this section.  


\subsection{Complex analytic varieties and complex orbifolds}    
A  complex analytic variety  $X$
is a reduced and irreducible  complex analytic space.   
We will denote by $X_{\sm}$ its smooth locus and by $X_{\sing}$ its singular locus.  
A smooth complex analytic variety is also called a complex manifold.    
A complex analytic variety $X$ is said to have klt singularities, 
if for every point $x\in X$, 
there is a neighborhood $U$ of $x$ and a  divisor $\Delta$ on $U$ such that $K_U+\Delta$ is $\mathbb{Q}$-Cartier and $(U,\Delta)$ is a klt pair in the sense of \cite[Definition 2.34]{KollarMori1998}.   


We refer to \cite{Grauert1962} for the notion of K\"ahler spaces.    
Assume that $(X,\omega)$ is a  K\"ahler  manifold. 
We will denote by $\Lambda$ the contraction with $\omega$. 
Then the Laplace-Beltrami operator $\Delta$ with respect $\omega$ satisfies 
$\Delta =  2\sqrt{-1}\Lambda\partial\bar{\partial}$.   
We denote 
\[\Delta'=\frac{1}{2}\Delta =  \sqrt{-1}\Lambda\partial\bar{\partial}. \]



Let $X$ be a complex analytic variety of dimension $n$,  
and let $\omega$ be some closed positive $(1,1)$-current with bounded local potentials on $X$. 
Assume that $\omega$ is continuous on some dense Zariski open subset $U\subseteq X$.  
Then we will denote by $\int_{(X,\omega)}$ the integration over $U$ with respect to the volume form $\omega^n$



A complex orbifold $\mathfrak{X}$ with quotient space $X$ is defined by the following data. 
There is an open covering $\{U_i\}$ of $X$, there are complex manifolds $V_i$, there are  finite groups $G_i$ acting holomorphically on $V_i$, such that $V_i/G_i \cong U_i$. 
We require further that the $(V_i,G_i)$ are compatible along the overlaps. 
For more details, we refer to, for example,   \cite[Section 3.1]{DasOu2023}.  %or \cite[Section 2.C]{Ou2024}. 
Throughout this paper, we always assume that the actions of $G_i$ are faithful. 
We call the branched locus of the orbifold structure  $\mathfrak{X}$ the subset $Z\subseteq X$, over which the natural morphisms $V_i\to X$ are branched. 
The set $Z$ is always a closed analytic subset of $X$.   
An object $h$, for example a current, a function, etc., is called orbifold smooth, if $h|_{X\setminus Z}$ is smooth, and if the pullback of $h|_{X\setminus Z}$  on any orbifold chart $V_i$ extends to a  smooth object on $V_i$. 





%We say that a normal compact complex analytic variety $X$ is $\mathbb{Q}$-factorial, if for any reflexive sheaf $\mathcal{L}$ of rank one on $X$, there is some integer $m> 0$ such that $(\mathcal{L}^{\otimes m})^{**}$ is locally free.



A finite morphism $f\colon X\to Y$ between  complex analytic varieties is called quasi-\'etale, if it is surjective and \'etale over an open subset of $Y$, 
whose complement has codimension at least 2. 
The following two  results on finite morphisms   are very useful for this paper. 
%theorem will be important us to estimate the derivatives of $G$ on $Y$ (c.f Lemma \ref{lemma:GBLOW}).

\begin{thm}
\label{thm:GR-cover} 
Let $X$ be a complex analytic variety,  
and let $X^\circ \subseteq X$ be a dense Zariski open subset.  
Assume that $X^\circ$ is normal and we have a finite \'etale morphism $p\colon Y^\circ \to X^\circ$. 
Then $p$ extends to a finite morphism $p\colon Y\to X$ with $Y$ normal, which is  unique up to isomorphism.
\end{thm}

\begin{proof} 
Let $r\colon X'\to X$ be the normalization. 
By assumption, $r$ is an isomorphism over $X^\circ$. 
Hence up to replacing $X$ by $X'$, we may assume that $X$ is normal. 
In this case, the theorem is proved in {\cite[Th\'eor\`eme XII.5.4]{SGA1}}.  
\end{proof} 
\begin{lemma}
    \label{lemma:smooth-finite-cover}
Let $Y\subseteq \mathbb{C}^n$ be an open ball centered at the origin, and let $\Delta$ be a divisor on $Y$ which is the union of some coordinate hyperplanes. 
Assume that $\pi\colon W\to Y$ is a finite surjective morphism which is \'etale over $Y\setminus \Delta$. 
Then there is a finite morphism $p\colon Y'\to Y$ such that the following properties hold. 
There is an endomorphism $\rho\colon \mathbb{C}^n\to \mathbb{C}^n$, which can be written in coordinates as 
\[
\rho\colon (z_1,...,z_n) \mapsto (z_1^{a_1},...,z_n^{a_n})  \mbox{ for some integers } a_1,...,a_n>0,  
\]
such that $Y'=\rho^{-1}(Y)$ and $p=\rho|_{Y'}$. 
In addition, the morphism $p$ factors through $\pi$.  
\end{lemma}


\begin{proof}
Without loss of the generality, we may assume that $\Delta$ is defined by  $z_1\cdots z_k=0$ for some integer $1 \le k \le n$.
Let $Y^\circ = Y\setminus \Delta$  and $W^\circ = \pi^{-1}(Y^\circ)$. 
Then the fundamental group   $\pi_1(Y^\circ)$ is isomorphic to $\mathbb{Z}^k$, which is generated by the loops $\gamma_1,...,\gamma_k$ around a general point  the components of $\Delta$.  
Moreover, the morphism $\pi|_{W^\circ}$ corresponds to a subgroup $H$ of  $\pi_1(Y^\circ)$ which has finite index. 
It follows that the subgroup $H'$ generated by $\gamma_1^d,...,\gamma_k^d$ is contained in $H$ for some integer $d>0$ sufficiently large.  

We impose $a_1=\cdots = a_k = d$ and $a_{k+1}=\cdots =a_n=1$ in the definition of $\rho$. 
Let $Y'^\circ = \rho^{-1}(Y^\circ)$. 
Then the natural surjective morphism  $Y'^\circ \to Y^\circ$ is \'etale and corresponds to the subgroup $H'$ of $ \pi^{-1}(Y^\circ)$.  
Hence it factors through $W^\circ$. 
By Theorem \ref{thm:GR-cover}, there is a normal variety   $Y''$ with a finite morphism  $Y''\to W$ which extends $Y^{'\circ} \to W^\circ$.  
It is also clear that the natural finite morphism $Y''\to Y$ extends  $Y^{'\circ} \to Y^\circ$.  
By the uniqueness of Theorem \ref{thm:GR-cover}, we can identify  $Y''$ as   $Y'=\rho^{-1}(Y)$.  
This completes the proof of the lemma.
\end{proof} 


We also need the following result on extensions of coherent subsheaves. 



\begin{lemma}\label{lemma:extension-subsheaf} 
Let $Z$ be a normal complex analytic variety and $\cF$ a reflexive coherent sheaf. 
Let $Z^\circ \subseteq Z$ be a Zariski open subset whose complement has codimension at least 2, and $\cE \subseteq \cF|_{Z^\circ}$ a saturated coherent subsheaf. 
Then $\cE$ extends to a coherent saturated subsheaf  of $\cF$  on $Z$.
\end{lemma} 


\begin{proof}
It is enough to prove the extension locally on $Z$. 
Hence by embedding $\cF$ in a free coherent sheaf as a saturated subsheaf, we may assume that $\cF$ is free. 
Furthermore, by removing from $Z^\circ$ some analytic subset of codimension at least two, 
we may assume that $\cE$ is a subbundle of $\cF|_{Z^\circ}$. 
Then there is an induced  morphism $f\colon Z^\circ \to  M$, where $M=G(n,m)$ is the Grassmannian variety, with $n=\mathrm{rank}\, \cF$ and  $m=\mathrm{rank}\, \cE$.  
By applying \cite[Main Theorem]{Siu1975} to the normal variety $Z$, as explained in \cite[page 441]{Siu1975},  we obtain that $f$ extends to a meromorphic map from $Z$ to $M$. 
Let $\Gamma \subseteq Z\times M$ be the closure of the graph of $f$, 
and we denote   $p_1\colon \Gamma \to Z$ and $p_2\colon \Gamma \to M$.  
Then $p_1$ induces a bimeromorphic map from $\Gamma$ to $Z$. 


Let $U$ be the universal vector bundle on $M$. By pulling back to $\Gamma$ \textit{via} $p_2$, we get a subbundle   $\cG$ of $p_1^*\cF$ on $\Gamma$. 
Since   $p_1$ is proper, the direct image $(p_1)_* \cG$ is a coherent  subsheaf of $\cF$ on $X$, extending $\cE$.
This completes the proof of the lemma.
\end{proof}













\subsection{Local and regional fundamental groups of klt singularities}  

Thanks to the Minimal Model Program for projective morphism over a germ of a complex analytic variety, see \cite{DasHaconPaun2022} or \cite{Fujino2022}, 
we can establish the following theorems.   
The first one is on fundamental groups around a klt singularities.  
We recall that the \'etale fundamental group $\pi_1^{\acute{e}t}$ of a topological space is the profinite completion of the fundamental group $\pi_1$. 

\begin{thm}
\label{thm:klt-local}
Let $(x\in X)$ be a germ of  complex analytic variety such that $X$ has klt singularities.  
Then, up to shrinking $X$,  the regional fundamental group $\pi_1^{reg}(X)$ is finite. 
In another word, $\pi_1(X_{\sm})$ is finite. 
\end{thm}

We note that, in the case when the singularity is algebraic, 
it is proved in \cite{Xu2014} that the local \'etale fundamental group $\pi_1^{\acute{e}t}(X\setminus \{x\})$ is finite. 
Later in \cite{Braun2021}, it is shown that the regional fundamental group is finite. 


\begin{proof}
In the proofs of \cite[Theorem 1]{Xu2014} and of \cite[Theorem 1]{Braun2021},  
the assumption that the singularity is algebraic is to ensure the existence of plt blowups, 
which extracts    a Koll\'ar component, see \cite[Lemma 1]{Xu2014}. 
Once we get a Koll\'ar component, we can apply the local-global principal to conclude the finiteness theorems.  
The ``local'' part is the fundamental groups of klt singularities, 
and the ``global'' part is the fundamental groups of weakly Fano pairs.  
The tools for  the proof of plt blowups are the theorems in \cite{BCHM10}. 
More precisely, they are  
the existence of MMP  for projective birational morphisms on klt pairs (see \cite[Theorem 1.2]{BCHM10}), 
and the finite generation of log canonical rings (see \cite[Corollary 1.1.2]{BCHM10}). 
In the case of complex analytic varieties, we can apply \cite[Theorem 1.4]{DasHaconPaun2022} or \cite[Theorem 1.7]{Fujino2022} in the place of \cite[Theorem 1.2]{BCHM10}, 
and we can apply \cite[Theorem 1.3]{DasHaconPaun2022} or \cite[Theorem 1.8]{Fujino2022} in the place of \cite[Corollary 1.1.2]{BCHM10}. 
In particular, plt blowups exist on a germ of analytic klt singularity.  
We also note that the ``global'' part   remains the same even we pass to the analytic setting, since the underlying variety of a weakly Fano pair is always a projective variety.  
Hence, once we can extract a Koll\'ar component,  
the same argument in \cite[Theorem 1]{Braun2021} proves the theorem.  
\end{proof}


The following theorem  was proved in \cite[Theorem 1.5]{GrebKebekusPeternell2016b} in the case of projective variety.  
With Theorem \ref{thm:klt-local} in hand, we can adapt its method in the setting of complex analytic varieties. 


\begin{thm}
\label{thm:klt-cover}
Let $X$ be a compact complex analytic variety with klt singularities.  
Then there is a finite quasi-\'etale cover $f\colon X' \to X$ such that the following property holds. 
If $\iota\colon X'_{\sm} \to X'$ is the natural inclusion, then the induced morphism  $\iota_* \colon  \pi_1^{\acute{e}t}(X'_{\sm}) \to  \pi_1^{\acute{e}t}(X')$  of \'etale fundamental groups 
is an isomorphism. 
\end{thm}


\begin{proof} 
We first consider a sequence of finite Galois surjective morphisms 
\[
\cdots \to X_k \to \cdots \to X_0 = X,
\]
such that every variety $X_k$ is normal and every  morphism $\varphi_k\colon X_k\to X_{k-1}$ is quasi-\'etale. 
The Zariski's purity theorem then implies that $X_k\to X$ is finite \'etale over $X_{\sm}$. 
We claim that there is  an integer $N> 0$, such that $\varphi_k$ is \'etale over $X_{k-1}$ entirely for $k\geq N$. 
Let $x\in X$ be a point.  
Then by Theorem \ref{thm:klt-local}, there is an open neighborhood $U_x$ of $x$, 
such that  $\pi_1((U_x)_{\sm})$ is finite.  
Then for each $k$, there is a positive integer $m_k$, such that if $V\subseteq X_k$ is a connected component of the preimage of $U_x$, then the natural morphism $V\to U_x$ has degree $m_k$.  We note that $m_k$ is independent of the choice of $V$, since every $\varphi_k$ is Galois.  
It follows that $m_k$ is bounded from above by the order of $\pi_1((U_x)_{\sm})$.   
In addition, $m_{k+1}\ge m_k$ for any $k\ge 0$.  
As a consequence, there is some integer $N(x)>0$, such that if $k\geq N(x)$, then $m_k=m_{k+1}$. 
For such integers $k$, the morphism $\varphi_{k+1}$ is a trivial cover over $(U_x)_{\sm}$.  
It follows that $\varphi_{k+1}$ is a trivial cover over $U_x$. 
By compactness, we can cover $X$ by finitely many such open subsets $U_{x_1},...,U_{x_m}$. 
Let $N =  \max\{ N(x_1),...,N(x_m) \} + 1$. 
Then $\varphi_k$ is \'etale  for $k\geq N$.  

Now we return to the situation of the theorem. 
Since any connected \'etale cover of $X$ induces a connected \'etale cover of $X_{\sm}$, 
  the natural morphism  $\pi_1^{\acute{e}t}(X_{\sm}) \to \pi_1^{\acute{e}t}(X)$ is surjective.  
If it is not an isomorphism, then the kernel of it induces a  Galois finite \'etale morphism $Z \to X_{\sm}$ of degree greater than $1$.  
By Theorem \ref{thm:GR-cover}, it extends to a finite quasi-\'etale morphism $Y\to X$. 
This morphism is not \'etale by construction.   
%By taking Galois closures, we may assume that $Y\to X$ is Galois. 
Hence, if we assume by contradiction that such a finite cover  in the theorem does not exists, 
then by induction, we can construct  an infinite sequence of finite Galois morphisms 
\[
\cdots \to X_k \to \cdots \to X_0 = X,
\]
such that every  $X_k$ is normal and every   $\varphi_k\colon X_k\to X_{k-1}$ is quasi-\'etale but not \'etale.
We obtain a  contradiction to  the first  paragraph. 
This completes the proof of the theorem. 
\end{proof}



\subsection{Simpson's operations}    
Let  $\mathcal{E}$ be a holomorphic vector bundle on a K\"ahler manifold $(X,\omega)$, let $h$ a fixed smooth Hermitian metric on $\mathcal{E}$.  
There is a definite positive Hermitian form on the bundle $\mathcal{E}nd(\mathcal{E}) \cong \mathcal{E}^*\otimes \mathcal{E}$ defined by $\langle A,B\rangle = \mathrm{Tr}(AB^*)$, where $\mathrm{Tr}$ is the trace and $B^*$ is the adjoint of $B$ with respect to $h$.   
Let  $End(\mathcal{E})$ be the set of measurable endomorphism of $\mathcal{E}$, that is,  the set of measurable global section of $\mathcal{E}nd(\mathcal{E})$. 
We denote by $End_h(\mathcal{E}) \subseteq End(\mathcal{E})$   the subset of   self-adjoint  endomorphisms of   with respect to $h$.   
 
Let $\psi: \mathbb R\to \mathbb R$ and 
$\Psi: \mathbb R\times \mathbb R\to \mathbb R$ be two smooth functions. 
They induce maps 
\begin{equation}\label{can1}
\psi: End_h(\mathcal{E})\to End_h(\mathcal{E}), \qquad \Psi: End_h(\mathcal{E})\to End\big(End(\mathcal{E})\big)
\end{equation}
as follows.   
Let $s\in End_h(\mathcal{E})$. 
On a small coordinate subset $U\subset X$, there is an $h$-unitary frame $(e_1,\dots, e_r)$ of $\mathcal{E}$ with respect to which $s$ is diagonal, say $s(e_i)= \lambda_i e_i$ for some real functions $\lambda_i$ defined on $U$.  
Then we set \[\psi(s)(e_i):= \exp(\lambda_i)e_i\] 
for each $i= 1,\dots, r$,  and we obtain in this way a  global Hermitian endomorphism $\psi(s)$.

 Given an $End(\mathcal{E})$-valued  $(p, q)$-form   $A$,  
 we can locally write $\displaystyle A= \sum_{i,j} a^j_i e^i\otimes e_j$, where the coefficients $a^i_j$ are  $(p, q)$-forms on $U$, and  $(e^1,\dots, e^r)$ is the basis on $\mathcal{E}^*$ dual to $(e_1,\dots, e_r)$. 
Then we define
\begin{equation}
\Psi(s)(A)|_U
%:= \sum_{i,j} \Psi(\lambda_i, \lambda_j)a^i_j e_i\otimes e^j   
:=  \sum_{i,j} \Psi(\lambda_i, \lambda_j)a^j_i e^i\otimes e_j,  
\end{equation} 
and particularly $\Psi(s)$  defines globally an endomorphism of $End(\mathcal{E})$, which is self-adjoint with respect to the Hermitian form $\langle \cdot , \cdot  \rangle$.    
By definition, we have the following property.  
\begin{equation}\label{can3}
\mbox{If }  \Psi\geq 0  \mbox{ then }  \langle \Psi (s) A, A\rangle \geq 0  \mbox{ for any }  A\in End(\mathcal{E}).    
\end{equation}




In the following statements, 
for any linear subspace $S$ of $End(\mathcal{E})$,  
we denote by $L^p(S)\subseteq S$ the subspace of elements  which is $L^p$.  
The subspace $L_1^p(S)\subseteq S$ is defined as the subspace of elements $s$ 
such that both $s$ and $\bar\partial s$ are $L^p$.  
For any positive real number $b$, we define $L^p_b(S)$ (respectively $L^p_{1,b})$ the set of elements $s$ in $L^p(S)$ (respectively in $L_1^p(S))$ such that 
$|s|\le b$. 


%We say that an element $\sigma \in End(\mathcal{E})$ belongs to the space $W^{1, p}$ if both $\sigma$ and $\bar\partial \sigma$ are in $L^p$.  For each positive real number $b> 0$  denote by $L^p_b$ and $L^{1, p}_b$ the subspace of  $End(\mathcal{E})$ which are in $L^p$ and $L^{1, p}$ respectively, whose  $L^\infty$  norm is moreover bounded by $b$.
%\medskip



%\noindent In this context we have the following  result.


\begin{lemma}\label{lemma:simpsonmor} \cite[Proposition 4.1]{Simpson1988} 
Let $\psi: \mathbb R\to \mathbb R$ and  $\Psi: \mathbb R\times \mathbb R\to \mathbb R $ be two smooth functions. 
Let $b>0$ be a real number. 
Then the following properties  hold. 
\begin{enumerate}
 

  \item For any $p\ge 1$, there is some $b'>0$ such that the following map is continuous  
    $$\psi: L^p_{b}(End_h(\mathcal{E}))\to L^p_{b'}(End_h(\mathcal{E})). $$ 
 

  \item For any $1\leq q\leq p$, we have a  nonlinear map
    $$\Psi: L^p_b(End_h(\mathcal{E}))\to \Hom\big(L^p(End (\mathcal{E})), L^q(End (\mathcal{E}))\big), $$
which is moreover continuous in case when $q< p$.   


\item For any $1\leq q\leq p$, there is some $b'>0$ so that we have the following map  
$$\psi: L^p_{1,b}(End_h(\mathcal{E}))\to L^q_{1,b'}(End_h(\mathcal{E})),$$
    which is continuous if  $q<p$. 
    The formula $\bar\partial \psi(s) =   \psi'(s)(\bar\partial s)$ holds in this context.
\end{enumerate} 
\end{lemma}



 
Now we consider the function $$\Phi (x, y) =\frac{e^{x-y}-1}{x-y}, $$ 
as in \cite{CGNPPW}.  
In the next lemma, we collect some elementary results of $\Phi$ without proof.  

\begin{lemma}\label{lemma:Phi-monotone} 
The following properties hold. 
\begin{enumerate}
\item If $ \alpha< \beta $ are real numbers,  then 
\begin{equation*}%\label{eqn:Phi-interval}
    \Phi(x,y)\geq \frac{\exp(\alpha-\beta )-1}{ \alpha -\beta} \mbox{ for any }
    \alpha \leq x,y \leq \beta.
\end{equation*}  

\item We fix $(x,y)\in \mathbb{R}^2$ and we let 
\[\sigma(\lambda) = \lambda \Psi(\lambda x, \lambda y)  = \frac{\exp(\lambda (x-y)) -1 }{x-y}.\] 
Then $\sigma$ is an increasing function in $\lambda$. 
When $\lambda$ tends to $+\infty$, 
$\sigma(\lambda)$ converges to $\frac{1}{y-x}$ if  $x<y$, 
and tends to $+\infty$ if $x\geq y$.      
\end{enumerate} 
\end{lemma}
 




\begin{lemma}\label{lemma:A}  
Let $S\in End_h(\mathcal{E})$ be a definite positive smooth global section of $\mathcal{E}nd(\mathcal{E})$. 
Let $s= \log S$, which is also a smooth section in $End_h(\mathcal{E})$.   
We still denote by $\partial$ the $(1,0)$-part of the Chern connection on $(\mathcal{E},h)$. 
Then 
\[\langle(\partial S)S^{-1}, \partial s\rangle_\omega  
=
\langle \Phi(s)(\partial s), \partial s\rangle_\omega. \]
Here the Hermitian form $\langle\cdot , \cdot \rangle_\omega$ on the space of differential forms with values in $\mathcal{E}nd(\mathcal{E})$ is   induced by the K\"ahler form $\omega$ on $X$.   
\end{lemma}





\begin{proof}
They following calculation can be found in \cite[Lemma 2.1]{UhlenbeckYau1986}, see also       \cite[Section 4.2]{CGNPPW}.  
Locally, let $(e_1,...,e_r)$ be a smooth $h$-unitary basis of $\cE$, which diagonalizes $S$, and hence $s$. 
Let $(e^1,...,e^r)$ be the dual basis of $\cE^*$. 
Then we can write $S= \sum \exp({\lambda_i})   {e}^i \otimes e_i$ and $s= \sum \lambda_i  e^i \otimes e_i$, for smooth real-valued functions $\lambda_i$.  
We write $\partial e_i  = A_{i}^j e_j$, where $A_i^j$ are smooth $(1,0)$-forms.  
It follows that $\partial e^i  = - A^{i}_j e^j$. 
Then we have 
\[   
\partial S = \sum \exp({\lambda_i}) \partial \lambda_i  e^i \otimes e_i 
+ \sum (\exp({\lambda_i}) - \exp({\lambda_j})) A_{i}^j e^i \otimes e_j,
\]
and 
\[   
\partial s = \sum  \partial \lambda_i  e^i \otimes e_i + 
\sum ( {\lambda_i} - {\lambda_j}) A_i^j e^i \otimes e_j.
\]
Hence 
\[
\langle(\partial S)S^{-1}, \partial s\rangle_\omega  =  \sum \|\partial \lambda_i\|_\omega^2 
+ 
 \sum (\lambda_i-\lambda_j) ( \exp({\lambda_i}-{\lambda_j}) -1 )\|A_i^j\|^2_\omega,
\]
and 
\[\langle \Phi(s)(\partial s), \partial s\rangle_\omega = \sum \|\partial \lambda_i\|_\omega^2 
+  \sum \frac{\exp(\lambda_i-\lambda_j)-1}{\lambda_i-\lambda_j} \cdot  (\lambda_i-\lambda_j)^2 \|A_i^j\|^2_\omega.
\] 
This completes the proof of the lemma. 
\end{proof}


We observe that the previous constructions are still valid in the setting of  K\"ahler  orbifold.  In the following lemma, we assume that $X$ is the quotient space of a K\"ahler orbifold. 


\begin{lemma}\label{lemma:etabound} 
Let $X$ be the quotient space of a compact K\"ahler  orbifold $(\mathfrak{X},\omega_\orb)$, 
and let $(\mathcal{E},h)$ an orbifold Hermitian vector bundle on $\mathfrak{X}$.  
We identify $\omega_\orb$ with a K\"ahler current $\omega$ on $X$ with continuous local potentials. 
Assume that $H$ is a $h$-self-adjoint endomorphism of $\mathcal{E}$ such that $h_{HE}:=h\cdot H$ is a Hermitian-Einstein metric with respect to $\omega_\orb$. 
Let $\eta= \log H$. 
Assume that $\int_{(X,\omega)} \Tr(\eta) =0$. 
Then the following    equality holds, 
\begin{equation}\label{alt5-general}
0 =  
\int_{(X,\omega)}\left\langle \Phi(\eta)(\partial\eta), \partial\eta\right\rangle_\omega 
+ \int_{(X,\omega)} \Tr \big(\eta \circ  \Lambda\Theta_h\big),  
\end{equation} 
where $\Theta_h$ is the Chern curvature tensor of $h$. 
\end{lemma}

\begin{proof}
We can decompose $\eta$ as follows, 
\begin{equation}\label{zz-general}
\eta= -\frac{\rho}{r}\cdot  \ID+ s,
\end{equation} 
where $r$ is the rank of $\mathcal{E}$,  $\Tr s=0$, and $\rho = \Tr \eta$ is an orbifold smooth function on $X$ such that  $\int_{(X,\omega)} \rho =0$. 
The Hermitian-Einstein equation for the metric $h_{HE}$ 
%(notations as in Section $3)$ 
writes
\begin{equation}\label{eqn:HE}
\gamma \cdot \ID- \Lambda\Theta_h= \Lambda \bar\partial\left((\partial H)H^{-1}\right),   
\end{equation}  
where $\gamma$ is an appropriate constant. 
We multiply both side by $\eta$ at the right, and deduce that 
\[
\gamma \cdot \eta =  \Lambda\Theta_h \circ \eta +  \Lambda \bar\partial \left((\partial H)H^{-1}\right) \circ \eta.  
\]
Now we take the trace and integrate on $(X,\omega)$. Since the trace of $\eta$ has mean value $0$, we deduce that 
%if  $\partial^*$ is the adjoint operator of $\partial$ for the coupling $\langle\cdot , \cdot\rangle_\omega$, then 
 %Now we compose $\eta$ with \eqref{eqn:HE}, then take the trace and integrate over $X$,  it follows from  that 
\[
0 =  \int_{(X,\omega)} \Tr \big(\eta  \Lambda\Theta_h\big)  
+ \int_{(X,\omega)}\Tr \left( \Lambda \bar\partial\big((\partial H)H^{-1}\big)\circ \eta\right)
.\]
For the second summand above,  by integration by part and by noting that $\eta$ is  self-adjoint, we  have  
\[
\int_{(X,\omega)}\Tr \left( \Lambda \bar\partial\big((\partial H)H^{-1}\big)\circ \eta\right)= 
\int_{(X,\omega)} \langle(\partial H)H^{-1}, \partial\eta\rangle_\omega.
\] 
Here, the integration by part  holds, as  all data involved   are orbifold smooth.
It follows that 
\[
0 =  
\int_{(X,\omega)} \langle(\partial H)H^{-1}, \partial\eta\rangle_\omega + \int_{(X,\omega)} \Tr \big(\eta \Lambda\Theta_h\big)
.\]
The lemma  then follows from   Lemma \ref{lemma:A}.
\end{proof}

 




\section{Uniform estimates on K\"ahler currents}
\label{section:uniform-kahler}

%\textcolor{red}{It is good to fix the notation, Let $(Z,\omega_Z)$  be the compact klt variety with smooth K\"ahler form. Let $p:X\rightarrow Z$ be the orbifold modification and $\omega_{\orb}$ be a fixed orbifold smooth metric, then the perturbed metric will be denoted as  $\omega_{\epsilon}:=p^*\omega_Z+\epsilon\omega_{\orb}$, also let $\pi:Y\rightarrow X$ be resolution of singularities.}

%\medskip

In this section, we will prove some uniform geometric estimates on a degenerate family of orbifold smooth K\"ahler currents. 

\subsection{Uniform estimates on $\mathcal{W}$ classes and $\mathcal{AK}$ classes. }
We start by recalling the recent breakthrough   by \cite{GPSS} and \cite{GPSS24},  
on uniform geometric estimates of a very robust family of K\"ahler metrics.   
Firstly, we record a series of definitions from these two papers.


 

\begin{defn}
\label{def:Wclass}
Let $(Y,\theta_Y)$ be a compact K\"ahler manifold of dimension $n$. 
Let $p\geq 1$, $A,K>0$ be real numbers and let $\gamma$ be a non-negative continuous function  on $Y$.   
We say that a K\"ahler form $\omega$ belongs to the class $\mathcal{W}(Y, \theta_Y, n, p, A, K, \gamma)$ if the following properties hold. 
\begin{enumerate}
    \item $ [\omega] \cdot  [\theta_Y]^{n-1} \le A$. 
    \item The $p$-th Nash-Yau entropy is bounded by $K$, i.e.
%
$${\mathcal{N}}_p (\omega) = \frac{1}{V_\omega} \int_Y \left|\log \frac{1}{V_\omega} \frac{\omega^n}{\theta_Y^n} \right|^p \omega^n  \leq K, $$ 
%
where $V_\omega=  [\omega]^n$ is the volume of $(X,\omega)$.  
\item  $\frac{\omega^n}{\theta_Y^n} \geq \gamma$. 
\end{enumerate}


\end{defn}


\begin{defn} \label{def:akclass} Let $X$ be a  compact normal K\"ahler variety of dimension $n$, let $\pi: Y \rightarrow X$ be a log resolution of singularities and let $\theta_Y$ be a smooth K\"ahler form  on   $Y$.   
We fix  constants $A,K>0$, an integer $p>n$,  and a non-negative function  $\gamma \in \mathcal{C}^0(Y)$ such that $\{y\in Y \ | \ \gamma(y)=0\}$ is contained in a proper analytic subvariety of $Y$. 
Then   the set of admissible  K\"ahler currents %on $X$
%
$${\mathcal{AK}}(X, \theta_Y, n, p, A, K, \gamma) $$  
%
is defined to be the set of   K\"ahler currents $\omega$ on $X$ satisfying the following conditions.

\begin{enumerate}

\item $[\omega]$ is a K\"ahler class on $X$ and $\omega$ has bounded local potentials. 
 

\item $[\pi^*\omega] \cdot [\theta_Y]^{n-1}\leq A$ and $[\omega]^n \geq A^{-1}$.  
 

\item The $p$-th Nash-Yau entropy is bounded by $K$, i.e.
%
$${\mathcal{N}}_p (\omega) = \frac{1}{V_\omega} \int_Y \left|\log \frac{1}{V_\omega} \frac{(\pi^*\omega)^n}{\theta_Y^n} \right|^p (\pi^*\omega)^n  \leq K, $$
%
where $V_\omega=  [\omega]^n$. 

 
\item %There exists a non-negative $\gamma \in \mathcal{C}^0(Y)$ such that $\{\gamma=0\}$ is contained in a proper analytic subvariety of $Y$ and  
 $\frac{(\pi^*\omega)^n}{ \theta_Y^n}\geq \gamma.$
 
\item The log volume measure ratio 
%
$$\log \left( \frac{(\pi^*\omega)^n}{\theta_Y^n} \right)$$ has log type analytic singularities.
\end{enumerate} 
\end{defn}


The log type analytic singularities in the item $(5)$ above is defined as follows.


\begin{defn}\label{def:logtype}
A  function $F$ on a  complex $Y$ manifold of dimension $n$ is said to have log type analytic singularities if the following properties hold. 
%\begin{enumerate}
%\item $f=F\cdot G$, where $G$ is a continuous function on $Y$ and $F$ further satisfied the following condition,
%\item 
There exist %holomorphic effective 
smooth prime divisors $D_1,...,D_N$ on $Y$ with simple normal
crossings. 
For $j = 1, ...., N$,   
let $\sigma_j \in H^0(Y,\mathcal{O}_Y(D_j))$ be a defining section of $D_j$, 
and $h_j$ be a smooth hermitian metric on  $\mathcal{O}_Y(D_j)$. 
Locally around every point of $Y$, the function $F$ can be written in the shape
%\item 
$$F =\sum_{k=1}^K a_k(-\log)^k\Big(\prod
^N
_{j=1}
e^
{f_{k,j}}|\sigma_j|^{
2b_{k,j}}_
{h_j}\Big),$$
where $K\geq 1$ is an integer, 
$(-\log)^k$
is the $k$- the composition of $(-\log)$, 
$a_k, b_{k,j} \in \mathbb{R}$  and $f_{k,j} \in C
^{\infty}(Y)$. 
%\end{enumerate}
\end{defn}
  



With the notation of Definition of \ref{def:akclass},  for any $\omega \in \mathcal{AK}(X, \theta_Y, n, p, A, K, \gamma)$, we  follow \cite{GPSS} to set $$\cS_{X, \omega}:=X_{\mathrm{sing}}\cup \,\,\pi \left( \textnormal{Singular set of}\,\,\, \left( \log \frac{(\pi^*\omega)^n}{\theta_Y^n} \right) \right).$$   From the definition of log type singularities, we see that $\cS_{X, \omega}$ is an analytic subvariety of $X$. 
%Later, we will show that the current $\omega$ is  indeed a smooth K\"ahler form on $X\setminus \cS_{X, \omega}$.

\begin{defn}\label{def:metric-completion} % Let $X$ be a compact normal K\"ahler variety of dimension $n$. 
With the notation of Definition \ref{def:akclass},   assume that 
%
$$\omega \in \mathcal{AK}(X, \theta_Y, n, p, A, K, \gamma). $$ 
We define
%
$$(\hat X, d) = \overline{(X\setminus \cS_{X, \omega}, \omega|_{X\setminus \cS_{X, \omega}})}$$
%
to be the metric completion of $\left(X\setminus \cS_{X,\omega}, \omega|_{X\setminus \cS_{X,\omega}} \right)$. We also denote the unique metric measure space  associated to $(X, \omega)$ by 
%
$$(\hat X, d, \omega^n).$$
%

\end{defn}

We remark that $\omega^n$ extends uniquely to a volume measure on $\hat X$ because neither $\omega$ nor $\omega^n$  carries mass on $\cS_{X, \omega}$. 
%With the notation of Definition \ref{def:metric-completion}, 
%Let $X$ be an $n$-dimensional normal K\"ahler variety. For any $p>n$ and any $\omega\in \mathcal{AK}(X, \theta_Y, n,p, A, K, \gamma)$, we let $(\hat X, d, \omega^n)$ the metric measure space associated to $(X, \omega)$. We 
We can consider the Sobolev space $L_1^{2}(\hat X,d, \omega^n)= W^{1,2}(\hat X,d, \omega^n)$ as in \cite[Definition 8.1]{GPSS}. 
%on  the metric measure space $(\hat X, d, \omega^n)$. % associated to $\omega$.

%For application in this paper, we need Heat kernel estimates, so we include the precise definition of Heat kernel for the singular metric space $(\hat X, d,\omega^n)$.

%Let $X$ be an $n$-dimensional normal K\"ahler variety. Let $\omega \in \mathcal{AK}(X, \theta_Y, n, A, p, K, \gamma)$. Since $\omega \in \mathcal{C}^\infty(X\setminus \cS_{X, \omega})$, we would like to define the heat kernel of the Laplacian $\Delta_\omega$  by the following parabolic equation 
%
%$$\partial_{t} H(x, y, t) = \Delta_{\omega, y} H(x, y, t), ~ \lim_{t\rightarrow 0^+} H(x, y, t) = \delta_x(y)$$
%
%for $x, y \in Y^\circ = \pi^{-1}(X\setminus \cS_{X, \omega})$.  \cite{GPSS} proves the existence and uniqueness of $H(x, y, t)$ and how it can be extended naturally to $\hat X \times \hat X \times [0, \infty)$. 

 Guo-Phong-Sturm-Song prove a  package of uniform geometric estimates.  
\begin{thm}\cite[Theorem 3.1]{GPSS}\label{thm:soborbi} Let $\omega\in\mathcal{AK}(X, \theta_Y, n, p, A, K, \gamma)$, then the following properties hold:
\begin{enumerate}

\item There exists a constant $C=C(X, \theta_Y, n, p, A, K, \gamma)>0$ such that  
%
$${\textnormal{diam}}(\hat X, d) \leq C.$$
%
%
In particular, $(\hat X, d)$ is a compact metric space.
 
\item  There exist a constant $q>1$  and a constant  $C_S=C_S(X, \theta_Y, n, p,  A, K, \gamma, q)>0$ such that the following Sobolev inequality  
%
$$
\Big(\int_{\hat X} | u  |^{2q}\omega^n   \Big)^{1/q}\le C_S \left( \int_{\hat X} |\nabla u|^2 ~\omega^n + \int_{\hat X} u^2 \omega^n \right)  
%
$$
%
holds for all $u\in W^{1, 2}(\hat X, d, \omega^n)$. 
 


\item There exists a constant $C_H=C_H(X, \theta_Y, n, p, A, K, \gamma,q)>0$ such that the following trace formula holds for the heat kernel $H$ of $(\hat X, d, \omega^n)$, 
%
$$H(x,x, t) \leq \frac{1}{V_\omega} + \frac{C_H}{V_\omega} t^{-\frac{q}{q-1}}. $$ 

 
\item Let $0=\lambda_0 < \lambda_1 \leq \lambda_2 \leq ... $ be the increasing sequence of eigenvalues of the Laplacian $-\Delta_\omega$ on $(\hat X, d, \omega^n)$. Then there exists $c=c(X, \theta_Y, n, p, A, K, \gamma, q)>0$ such that
%
$$\lambda_k \geq c k^{\frac{q-1}{q}}. $$
%\item \textcolor{red}{Add Green kernel? On the other hand, all we need in this paper is uniform mean value inequality which can be obtained upper stairs on the resolution}
\end{enumerate}
\end{thm}



We recall the definition of  heat kernels.  
\begin{defn}\label{def:Heatkernel} 
Assume that $  \omega\in\mathcal{AK}(X, \theta_Y, n, p, A, K, \gamma)$.  
The heat kernel of the Laplacian $\Delta_{\omega}$ is by the following parabolic equations 
%
$$\partial_{t} H(x, y, t) = \Delta_{\omega, y} H(x, y, t),  \ \lim_{t\rightarrow 0^+} H(x, y, t) = \delta_x(y)$$
for $x, y \in Y^\circ = \pi^{-1}(X\setminus \cS_{X, \omega})$. 
\end{defn}









We will need the following uniform mean value inequality, which is essentially proved in \cite[Lemma 2]{GPS24} and \cite[Lemma 5.1]{GPSS24}. 
%for the set of smooth metrics $\mathcal W$.   

\begin{lemma}\label{lemma:meanvalue} 
Let $\omega\in \mathcal{W}(Y, \theta_Y, n, p, A, K, \gamma)$. 
We assume in addition there is some constant $B>1$ such that $V_{\omega} = [\omega]^n$ is contained in $[B^{-1},B]$.  
Let $a$  and $I$ be positive real numbers. 
Let  $v\in L^1(Y, \omega)$ be a function such that $ |\int_{(Y,\omega)}v| \le I$. 
Assume that $v$ is  $\mathcal{C}^2$-differentiable on the set 
$\{v>  -I\cdot B  -1\}$ and  that  
\[\Delta_{\omega}(v)\geq -a\]
on $\{v>  -I\cdot B  \}$. 
Then we have 
\[ v\leq C\big(1+ \Vert v\Vert_{L^1(Y,\omega)}\big)
\] 
where $C=C(Y,\theta_Y, n, p , A,K,\gamma,a,B,I)$ is a positive real number. 
\end{lemma} 


\begin{proof}
Let $M = \frac{1}{V_\omega} \int_Y v\cdot \omega^n$ and let $u=v-M$. 
Then $\int_Y u\cdot \omega^n =0$ and $|M|\leq IB$.  
Thus $u$ is $\mathcal{C}^2$-differentiable on the set 
$\{u>    -1\}$ and  that  
$\Delta_{\omega}(u)\geq - a$ 
on $\{u>  0  \}$.   
By \cite[Lemma 5.1]{GPSS24}, there is a constant $C'=C'(Y,\theta_Y, n,p,A,K,\gamma,a)$ such that 
$u \leq C'(1+ \| u \|_{L^1(Y,\omega)})$. 
%We note that  \[ |M| \leq \frac{1}{V_\omega} \| v \|_{L^1(Y,\omega)} \leq B^{-1}  \| v \|_{L^1(Y,\omega)},\] 
We note that $   \| u \|_{L^1(Y,\omega)} \leq  \| v \|_{L^1(Y,\omega)} + |M| \cdot V_\omega$. 
Hence we have  
\[
v \le IB + C'(1 +  \| v \|_{L^1(Y,\omega))} + IB^2) 
\]
This completes the proof of the lemma. 
\end{proof}





\subsection{Uniform estimates for degenerating families of orbifold K\"ahler forms}

In the remainder of this section, we consider the following situation. 
Let $(Z,\omega_Z)$ be a compact K\"ahler variety of dimension $n$. 
Assume that $\rho\colon X\to Z$ and $\pi\colon Y\to X$ are projective bimeromorphic morphisms, such that $Y$ is smooth and $X$ is the quotient space of some K\"ahler  orbifold $\mathfrak{X}$.  
Let $\theta_Y$ be a K\"ahler form on $Y$.  
We assume that the $\rho\circ \pi$-exceptional locus is a snc divisor, 
and that there is a divisor $D\geq 0$ with the same support, such that $-D$ is relatively ample over $Z$.    
In particular, we fix a smooth Hermitian metric $h_D$ on $\mathcal{O}_Y(D)$ so that $(\rho\circ\pi)^*\omega_Z - \delta \cdot \Theta_{h_D}$ is a K\"ahler form for all $\delta>0$ small enough, where $\Theta$ stands for the Chern curvature.  
We assume further that $\pi(D)$ contains the branched locus of $\mathfrak{X}$, 
denote by $s_D\in H^0(Y,\mathcal{O}_Y(D))$ a global section defining $D$. 
%Then we can write $(\rho\circ\pi)^*\omega_Z + \delta \ddbar \log|s_D|_{h_D}$ for the previous K\"ahler form. 
We also suppose that $\pi(D)$ contains the branched locus of $\mathfrak{X}$.     
Let $\omega_\orb$ be an orbifold K\"ahler form on $\mathfrak{X}$. 
By abuse of notation, we also denote by $\omega_\orb$ the induced K\"ahler current on $X$. 
For any $\epsilon>0$, we set $\omega_\epsilon = \rho^*\omega_Z + \epsilon\cdot \omega_\orb$. 
They are considered as K\"ahler currents on $X$, which are orbifold smooth on $\mathfrak{X}$.  
Our objective is to show   the following theorem. 

\begin{thm}
\label{thm-orbifold-AK-property} 
There exists constants $C,C_S,C_H,c>0$, all independent of $\epsilon$, 
such that for all $\epsilon>0$ small enough, 
the consequences of Theorem \ref{thm:soborbi} hold for $\omega_\epsilon$. 
\end{thm}
%they satisfy the same properties in Theorem \ref{thm:soborbi},  uniformly for all $\epsilon>0$ small enough.   
We follow the method of \cite[Section 7]{GPSS}, and will approximate  $\omega_\epsilon$ by certain family  $\{\omega_{j}\}_{j\gg 0}$ of  smooth K\"ahler forms on $Y$.  
The key is to show that, for all $j$ sufficiently large, 
$\omega_j$ belong to the same class $  \mathcal{W}(Y, \theta_Y, n, p, A, K, \gamma)$, where $A,K,p,\gamma$ are independent of $\epsilon$ and $j$.    
For more details, see Lemma \ref{lemma:unisob}.  



%Our first main theorem generalizes Theorem \ref{thm:soborbi} to an orbifold smooth  K\"ahler current $\omega_{\orb}$, defined on a compact K\"ahler variety $X$ with quotient singularities. 
In our situation, it is routine to verify that the currents $\omega_\epsilon$ satisfy the items $(1)$-$(4)$ of Definition \ref{def:logtype},  uniformly for all  $\epsilon>0$ small enough.  
%The difficulty in this situation is that, 
%if we fix a log resolution $\pi:Y\rightarrow X$, then in general 
However, for the item $(5)$, the log volume ratio $\log \left(\frac{\pi^*\omega^n_{\epsilon}}{\theta_Y}\right)$  does not have log type analytic singularities.
%, i.e. the item $(5)$ in the Definition of \ref{def:logtype} is not satisfied. Nevertheless,
Fortunately, in the following lemma, we observe that the log volume ratio is the sum of two functions, one has log type analytic singularities, and the other one is a continuous function $G_\epsilon$.  
For this function $G_\epsilon$, we can estimate the blow-up rates of its derivatives with respect to the distance to the divisor of $D$. 
With a smoothing argument by using convolutions, we can still find the desired approximations.
%of the volume ratio as in \cite{GPSS}.

 



%In the following lemma, we decompose the log volume ratio $\log\frac{\pi^*\omega_{\orb}^n}{\theta_Y^n}$ into a sum of two parts. 

\begin{lemma} \label{l:logtype} 
%Let  $X$ be the quotient space of a compact K\"ahler orbifold, let  $\omega_{\orb}$ be a K\"ahler current on $X$ which is an  orbifold smooth K\"ahler form,   let  $\pi:Y\rightarrow X$  be a log resolution of the branched locus of $X$, and let $\theta_Y$ be a K\"ahler form on $Y$.  
There   are effective divisor $E_1,E_2$ without common components, 
and a constant  $a>0$ such that the following properties hold. 
The  supports of $E_1$ and $E_2$  are contained in the one of $D$.  
For $i=1,2$, let $s_{E_i}\in H^0(Y,\mathcal{O}_Y(E_i))$ be a global  section  defining $E_i$, 
let $h_{E_i}$ be a smooth Hermitian metric on $E_i$. 
We set  
\[
F_{\log} = a (\log |s_{E_1}|_{h_{E_1}} -  \log |s_{E_2}|_{h_{E_2}}). 
\]
For any   K\"ahler current $\omega$ on $X$ which is an orbifold K\"ahler form on $\mathfrak{X}$,   
we can write  
\[\log\frac{\pi^*\omega^n}{\theta_Y^n}=F_{\log} +  G\] 
where $G$ is  a continuous  function on $Y$, smooth away from $D$. 


In addition, 
%assume that  $D$ is an effective divisor on $Y$ whose support is equal to the singular locus of $\log\frac{\pi^*\omega_{\orb}^n}{\theta_Y^n}$.   Let $s_D\in H^0(Y,\mathcal{O}_Y(D))$ be a section defining $D$, and let $h_D$ be a smooth  Hermitian metric on $\mathcal{O}_Y(D)$.  Then 
for any integer $k>0$, there  are positive integers $C_k, N_k$ such that
$\|\nabla^k G\|_{\theta_Y}\leq C_k |s_D|_{h_D}^{-N_k}$, where $\nabla$ means the covariant derivatives with respect to $\theta_Y$.
\end{lemma}  

 
%{ \color{red} It seems there is a unified proof, by using Lemma 3.7 below. I write it after your proof.}

%\begin{proof} It suffices to prove the lemma locally on $X$, so let us fix a point $p\in X$ and also fix a an orbifold chart $f:(V,G)\rightarrow  X$ for $p$. We discuss two different cases.

%\textbf{Case 1:} there is no codimension $1$ ramification locus for the local orbifold chart. Fix an orbifold chart  near $p$ and also fix a local generator of the line bundle $mK_{V}$, where $m$ is a positive integer possibly strictly larger than $1$.  Then by the definition of orbifold smoothness,  $\omega_{\orb}^n$ is a smooth volume form on $V$ and $f^*m\Omega_X$ is a non-vanishing section of $mK_{V}$, hence $f^*\frac{\omega_{\orb}^n}{\Omega_X\wedge\bar\Omega_X}$ is a smooth function on $V$. In particular, $\frac{\omega_{\orb}^n}{\Omega_X\wedge\bar\Omega_X}$ is a no-where vanishing continuous function near $p\in X$. Now we rewrite the volume ratio as: $$\frac{\pi^*\omega_{\orb}^n}{\theta_Y^n}=\frac{\pi^*\omega^n}{\pi^*\Omega_X\wedge\pi^*\bar\Omega_X}\frac{\pi^*\Omega_X\wedge\pi^*\bar\Omega_X}{\Omega_Y\wedge\bar\Omega_Y}\frac{\Omega_Y\wedge\bar\Omega_Y}{\theta_Y^n}.$$
%For the three functions on the RHS, the first one is a strictly positive continuous function, the second one has log type singularities and the third one is smooth. The lemma is proved in this case.

%\textbf{Case 2:} Now we treat the case when the local orbifold chart $(V,G)$ near $p\in X$ has codimension one ramification locus with ramification index $r$. \textcolor{red}{Take the codimension one ramification locus $\Delta$ on the regular locus of a neighbourhood of $U\subset X$ and then take the closure of $\Delta$ in $X$, which is still denoted by $\Delta$. Then for some positive integer $m$, $m\Delta$ is  a  cartier divisor defined by $g=0$, where $g$ is a holomorphic function. Note that
%$q:=\frac{f^*\Omega_X}{\Omega_{V}}$ is a homomorphic function, then the ratio $m:=q^r/f^*g^{r-1}$ function is holomorphic with the same vanishing order along the reduced divisor defined by $\{q=0\}$. So $m$ is a non-vanishing holomorphic function. Hence
%$$\frac{f^*\Omega_X\wedge f^*\bar\Omega_X}{f^*\omega_{\orb}^n}=\frac{f^*\Omega_X\wedge f^*\bar\Omega_X}{\Omega_{V}\wedge\bar\Omega_{V}}\frac{\Omega_{V}\wedge\bar\Omega_{V}}{f^*\omega_{\orb}^n}$$} For the two terms on the RHS, the second one is a  smooth nonzero functions, and for the first term, we have
%$$\frac{f^*\Omega_X\wedge f^*\bar\Omega_X}{\Omega_{V}\wedge\bar\Omega_{V}}=|q|^2=(|m|^2f^*|g|^{2r-2})^{1/r}.$$
%So $$\frac{f^*\Omega_X\wedge f^*\bar\Omega_X}{f^*\omega_{\orb}^n}=f^*|g|^{2-2/r}G,$$
%where $G$ a priori is a smooth function on the chart $V$ and it is clear from the equation above that function $G$ is invariant under the group action \textcolor{red}{Away from $g=0$ is clear, other case seems unclear? On the other hand, it seems we don't really need G to be continuous, bounded is enough}. We remark that the only difference with case 1 is the new factor $|g|$. Now we  take a log resolution of the pair $\pi:Y\rightarrow (X,\Delta)$ such that $\pi^{-1}\Delta$ and the exceptional divisors are simple normal crossing. Note that the pull back of $|g|^{2-2/r}$ by $\pi$ will be a function of log type, then by repeating the same argument as in case $1$, the lemma is proved.
%\end{proof}





\begin{proof}  
We investigates the  first part of the lemma   locally on $X$.  
Assume that  $f: V\to X$ is an orbifold chart.  
We denote by $\omega_V=f^*\omega $ the smooth K\"ahler form on $V$. 
Let %$\pi:Y\to X$ be a {\color{red}  log} desingularization, and 
$W$ be the normalization of $V\times_X Y$.   

Then we study locally on $Y$. 
By abuse of notation,   we will assume that $Y\subseteq \mathbb{C}^n$ is an open ball centered at the origin.  
Since $D$ is snc, 
we can  assume that  it is a union of coordinates hyperplanes. 
Since the branched locus of $W\to Y$ is contained in the divisor $D$, 
by applying  Lemma \ref{lemma:smooth-finite-cover} to the finite cover $W\to Y$,  we obtain a finite morphism  $p\colon Y'\to Y$.  
We may also assume that $\theta_Y$ is equal to the Euclidean K\"ahler form. 
Let $\theta_{Y'}$ be the Euclidean K\"ahler form on $Y'$.  
Then, by the construction of Lemma \ref{lemma:smooth-finite-cover},  
we have  
\[(p^*\theta_Y)^n = (\prod_{i=1}^n  a_i^2 |z_i|^{2a_i-2}) \cdot \theta_{Y'}^n , \]  
where$(z_1,...,z_n)$ are coordinates on $Y'$  and $a_1,...,a_n$ are positive integers. 


Now we compute the ratio $\frac{\pi^*\omega^n}{\theta_Y^n}$ by pulling it back to $Y'$. 
Up to shrinking $V$, we may assume that $\omega_V^n= \rho\cdot \Theta \wedge \overline{\Theta}$, where $\Theta$ is a nowhere vanishing holomorphic $n$-form independent of $\omega_V$,  and $\rho$ is a smooth nowhere vanishing function on $V$.  
If $q\colon Y'\to V$ is the natural morphism, then the locus where $q$ is not smooth is contained in $p^{-1}(D)$, which is a union of coordinate hyperplanes.  
It follows that $q^*\Theta$ is a holomorphic $n$-form on $Y'$,  whose vanishing locus is contained in $p^{-1}(D)$. 
Hence we can write  
\[
q^*(\Theta\wedge \overline{\Theta}) = A\cdot \varphi_2 \cdot \theta_{Y'}^n
\]
where $A$ is a positive smooth function, and  $\varphi_2$ is of the shape 
\[
\varphi_2 = \prod_{i=1}^n   |z_i|^{2b_i-2} 
\]
for some  positive integers $b_1,...,b_n$.  
Hence we can write 
\[
(q^*\omega_V)^n =   \varphi_1 \cdot \varphi_2 \cdot   \theta_{Y'}^n, 
\] 
where $\varphi_1$ is a smooth positive function.  
Therefore, we can write 
\[
p^*(\frac{\pi^*\omega^n}{\theta_Y^n}) = \frac{(q^*\omega_V)^n}{\theta_{Y'}^n} \cdot \frac{\theta_{Y'}^n}{(p^*\theta_Y)^n} = \psi_1\cdot \psi_2, 
\]
where $\psi_1=\varphi_1$ is a smooth positive function,
and 
\[\psi_2 := \varphi_2\cdot \prod_{i=1}^n |z_i|^{2-2a_i} = \prod_{i=1}^n   |z_i|^{2c_i},  \]  
for some  integers $c_1,..,c_n$.  
We remark that the product $\psi_1\cdot \psi_2$ is invariant under the Galois group of $Y'\to Y$, and so is $\psi_2$. 
Thus, so is $\psi_1$. 
Hence there is a continuous positive function $\eta_1$ on $Y$ whose pullback on $Y'$ is equal to $\psi_1$.   
Similarly, $\psi_2$ descend to a function  $\eta_2$ on $Y$, which has the shape $\eta_2=\prod_{i=1}^n |t_i|^{d_i}$, for some rational numbers $d_1,...,d_n$, 
where $(t_1,...,t_n)$ are coordinates on $Y$.  

It follows that the singularities of the log volume ratio $\log\frac{\pi^*\omega^n}{\theta_Y^n}$ is identical  to those of  $\log \eta_2 = \log (\prod_{i=1}^n |t_i|^{d_i})$.  
Furthermore, from the construction, $\eta_2$ may depend on $Y'$ and $V$, but is independent of $\omega$. 
Hence the $\mathbb{Q}$-divisors locally defined by $\prod_{i=1}^n |t_i|^{d_i}=0$ glue globally into a $\mathbb{Q}$-divisor  $\Delta$,  which depends only on $X$ and $Y$.  
There is a positive integer $m$, such that $m\Delta$ is integral. 
We define $E_1$ and $E_2$ so that  $m\Delta=E_1-E_2$, and define $a= m^{-1}$. 
Then the function 
\[
G  = \log\frac{\pi^*\omega^n}{\theta_Y^n} - F_{\log}
\]
is  continuous on $Y$. 
In addition,  $p^*G$ is smooth on $Y'$. 
This proves the first statement of the lemma. 

For the second part of the lemma, since $Y$ is compact, we only need to prove the estimate locally on $Y$. 
Hence we can still use the previous notation and consider $Y$ as an open ball in $\mathbb{C}^n$.   
We fix some integer $k\geq 0$. 
By pulling back to $Y'$, we see that 
\[
p^*(\nabla^k G)  = \nabla^k(G_1+\log \psi_1). 
\]
for some smooth function $G_1$ on $Y$. 
In particular, $\|p^*(\nabla^k G) \|_{\theta_{Y'}}$ is bounded.  
Without loss of the generality, we can  assume the support of $D$ is defined by  $t_1\cdots t_j =0$ for some integer $  j\le n$. 
Then  $\|\nabla^k G \|_{\theta_{Y}}$ is bounded by $C'_k\cdot |t_1\cdots t_j|^{-m_k}$ for some positive integers $C'_k,m_k$.  
By our assumption on $s_D$, we see that $|s_D|_{h_D}$ can be written as $C''\cdot |t_1|^\alpha_1 \cdots |t_j|^{\alpha_j}$ for some positive integers $\alpha_1,...,\alpha_j$ and some positive smooth function $C''$.  
Hence  $\|\nabla^k G\|_{\theta_Y}\leq C_k |s_D|_{h_D}^{-N_k}$ for some positive integers $C_k,N_k$. 
This completes the proof of the lemma.  
\end{proof}


In the previous lemma, both $E_1$ and $E_2$ are allow to be the zero divisor. 
We will later use convolutions to  approximate the function $G$ above by smooth functions.  
The following proposition provides some estimates on the convolutions.

%Using Lemma \ref{lemma:smooth-finite-cover}, 
%We have the following estimate for derivatives of $G$.
%\begin{lemma}\label{lemma:GBLOW} 
%With the notation of Lemma \ref{l:logtype},  
%for any $k>0$, there is a positive integer $N_k$ such that
%$|\nabla^k G|_{\theta_Y}|\leq |s_D|^{-N_k}$.
%\end{lemma}


%\begin{proof}
%In our situation, assume that $X$ is a variety with quotient singularities and $f: V\to X$ is an orbifold chart.  Let $\pi:Y\to X$ be a log resolution of branched locus of $X$, $W$ the normalization of $V\times_X Y$,  and $G$ certain function on $X$, we want to study the behaviour of $\pi^* G$.   The key property of $G$ relevant to us is that $f^*G$ is a smooth function on orbifold chart $V$.  By using Lemma \ref{lemma:smooth-finite-cover} above, locally around any point of $Y$, we can find a finite cover $Y'\to W\to Y$, such that $Y'$ is smooth. Hence the pullback of $\pi^*G$ from $Y$  to $Y'$ is the same as the pullback of the smooth function $f^*G$ from $V$ to $Y'$, which is smooth. By the specific type of map $\rho$ and chain rule, we deduce that for any $k>0$, there is a  positive integer $N_k$ such that $|\nabla^k G|_{\theta_Y}|\leq |s_D|^{-N_k}$.  \end{proof}



%{ \color{red} In the next lemma, we should say that $G$ is a function on a compact Kahler manifold $Y$, which satisfies the blowup estimates.  We do not need the orbifold $X$ here. Indeed, our $G$ is not defined on $X$ but only on $Y$.  Xin: confused, but anyway should be compatible with G in lemma 3.7}



\begin{prop}\label{prop:appg} Let $G$ be a continuous function on  $Y$ 
which is smooth away from $D$. 
Assume that for any integer $0\leq k\leq 3$,  
there are integers $C'_k >0$ and $a_k\geq 0$, 
such that all covariant derivatives  of $G$ up to order $k$ with respect to $\theta_Y$, 
are bounded by $C'_k\cdot |s_D|_{h_D}^{-a_k}$.   
%, and also assume that $G$ is smooth on a orbifold chart, then there is a log resolution of the pair $\pi:Y\rightarrow (X,\Delta)$, where $\Delta$ is the codimension $1$ ramification locus (with reduced structure) such that 
Then there exists  a family of smooth approximating functions $\{G_\sigma\}_{1 \gg \sigma>0}$ of $ G$ satisfying the following properties:
\begin{enumerate}
\item $ G_\sigma   $ are bounded,  uniformly for all  $\sigma$.  
\item On any compact set $K\subset Y\setminus D$, $G_\sigma$ converge to $G$ uniformly and smoothly as $\sigma\rightarrow 0$.
%, where $D:=E_{exc}+\pi^{-1}\Delta$. Here $E_{exc}$ is the exceptional divisors of $\pi$ and $\pi^{-1}\Delta$ is the strict transformation of $\Delta$. 
\item There are  positive integers $C, d$, 
such that $\|\nabla^2 G_\sigma\|_{\theta_Y} \leq C|s_D|_{h_D}^{-d}$ for all $\sigma$ small enough. 
%, where $s_D$ is the canonical section of the line bundle associated with $D$ and $h_D$ is a smooth Hermitian metric of line bundle.
\end{enumerate}
\end{prop}


\begin{proof} 
Let $\theta_{1}\colon \mathbb{R} \to \mathbb{R}_{\geq 0}$ be a   function supported on $[0,1]$, such that $\theta_1(|w|^2)$ is smooth for  $w\in \mathbb{R}^{2n}$  and that $\int_{\mathbb{R}^{2n}}\theta_1(|w|^2) \mathrm{d}w = 1$.   
For $\sigma>0$, we set  $\theta_{\sigma}(u) = \frac{1}{\sigma^{2n}}\theta_1(\frac{u}{\sigma^2})$, 
so that it is supported on $[0,\sigma^2]$ and $\int_{\mathbb{R}^{2n}}\theta_\sigma(|w|^2) \mathrm{d}w = 1$.    
We will use the functions $\theta_{\sigma}$  as convolution kernels to construct approximations of $G$.  
We note that there are  integers $C_0, b_0>0$, such that the derivatives of $\theta_\sigma$ up to order $2$ is bounded by $ C_0 \cdot \sigma^{-b_0}$. 



We  denote  by $|x-y|$ the distance between two points $x,y \in Y$. 
Since $Y$ is compact, there is some $0<\sigma_0<1$ small enough, such that for any $y\in Y$, 
the exponential map $\exp_y$ is an isomorphism from the ball in $\mathbb{R}^{2n}$ of radius $4\sigma_0$ centered at the origin.   
From now on, we only consider $\sigma>0$ which are less than $\sigma_0$,   
and define the  smooth functions  $G_\sigma$ by using convolutions as follows, 
\[G_\sigma (y) = \int_{w\in \mathbb{R}^{2n}}  \theta_\sigma (|w|^2) \cdot G(\exp_y(w)) \mathrm{d}w.  \] 
By our choice of $\sigma_0$, we have the following alternative expression of $G_\sigma$, 
 \[G_\sigma (y) =  \int_{x\in (Y,\theta_Y)} \theta_\sigma (|x-y|^2) \cdot G(x) \cdot \lambda(y,x),  \]
where $\lambda(y,x)^{-1}$ is the Jacobian determinant of the exponential map $\exp_y$ at the point $(\exp_y)^{-1}(x)$. 
Up to shrinking $\sigma_0$, we can assume that $\exp_y^{-1}$ and  $\lambda(y,x)$ are  smooth function on $\{(x,y)\in Y \times Y \ | \ |x-y|<4\sigma_0 \}$. 
From the standard properties of convolutions, we can deduce the items $(1)$ and $(2)$.  


For the item $(3)$, 
we  set \[T_\sigma = \{x\in Y \ | \  \mathrm{dist}\, (x, D) < \sigma\},\] 
where $\mathrm{dist}\, (x, D)$ is the distance from $x$ to $D$. 
Locally around every point of $D$, there is a coordinate neighborhood with holomorphic  coordinates $(z_1,...,z_n)$, on which $D$ is the union of certain coordinate hyperplanes. 
In particular, $|s_D|_{h_D}$ can be written in the shape 
\[ |s_D|_{h_D} =  A\cdot |z_1|^{\alpha_1}\cdots |z_n|^{\alpha_n}\] 
for some smooth positive function  $A$ and for some $\alpha_1,...,\alpha_n \in \mathbb{Z}_{\geq 0}$. 
Therefore, since $Y$ is compact, there are positive constant integers  $C_1, b_1$, 
such that for all $0<\sigma<\sigma_0$,  we have 
\begin{equation} \label{eqn:appg1}
\sigma \geq \frac{1}{2} \mathrm{dist}(x,D)  \geq  C_1 |s_D(x)|_{h_D} ^{b_1}  \mbox{ for all } x\in T_{2\sigma}.      
\end{equation}
In addition, there is a constant $C_2$, such that for any $y\in  Y\setminus T_{2\sigma}$, and for any $t\in Y$ with $|t-y|\le \sigma$, 
we have  
\begin{equation} \label{eqn:appg2}
    |s_D(t)|_{h_D} \geq C_2 |s_D(y)|_{h_D}. 
\end{equation}
To visualize this constant $C_2$, locally around a point of $D$ for example, 
we may let $C_2 = 2^{-(\alpha_1+\cdots + \alpha_n)}$





We fix an open covering of $Y$ by coordinates open subsets.  
It is enough to prove that, 
there are constant integers $C> 0$ and $d \geq 0$, such that  on each of these open subsets, we have 
\[
|\partial_{z_i} \partial_{\bar{z}_j} G_\sigma| \le C\cdot|s_D|_{h_D}^{-d}, 
\] 
for all $i,j$ and all $\sigma$. 
%\textbf{Proof of claim:} 
%In Lemma \ref{l:logtype}, we have proved  that, 
%for a fixed finite open covering of $Y$ by coordinate open subsets,  for any integer $k \geq 0$,  there are integers $C'_k >0$ and $a_k\geq 0$,  such that all partial derivatives of $G$ up to order $k$ are bounded by $C'_k\cdot |s_D|_{h_D}^{-a_k}$.   
The idea to divide the manifold $Y$ into two parts (depending on $\sigma)$ and estimate the derivatives of $G_\sigma$  separately. 

%We always assume $ 0<\sigma<\sigma_0$, where $\sigma_0$ is a fixed small constant $\ll 1$.   


   
Firstly, we assume that  $y \in Y\setminus T_{2\sigma}$.    
Then for any $w\in \mathbb{R}^{2n}$ with $|w|\le \sigma$,  
the partial derivatives with respect to $y$ satisfies 
\[
|\partial_{z_i} \partial_{\bar{z}_j}G(\exp_y(w))-\partial_{z_i} \partial_{\bar{z}_j}G(\exp_y(0))|  
\le \sigma \cdot |\varphi(y,w')| \le |\varphi(y,w')|, 
\]
where $\varphi$ involves the   partial derivatives of $\partial_{z_i} \partial_{\bar{z}_j}G(\exp_y(w))$ with  respect to $w$, and $w'$ is a point lying on the interval $[0,w]$ inside $\mathbb{R}^{2n}$.  
Since $\exp_y(w)$ is a smooth function for $y\in Y$ and $|w|<4\sigma_0$ by our choice of $\sigma_0$, its partial derivatives up to order $3$ are bounded by a constant, whenever $|w|\le \sigma_0$. 
Therefore, by chain rule, if we set $t=\exp_y(w')$, then the term $|\varphi(y,w')|$ can be controlled by the partial derivatives of $G$ up to order $3$ at the point $t$.  
From the estimates on the partial derivatives of $G$, 
we then deduce that 
\[
|\partial_{z_i} \partial_{\bar{z}_j}G(\exp_y(w))-\partial_{z_i} \partial_{\bar{z}_j}G(\exp_y(0))|  \le |\varphi(y,w')| 
\le C_3\cdot C_3'  |s_D(t)|_{h_D}^{-a_3}, 
\] 
for some constant $C_3$. 
%Then for any $x \in Y$ with $|x-y| \le \sigma$, we have 
%\[
%|\partial_{z_i} \partial_{\bar{z}_j}G(x)-\partial_{z_i} \partial_{\bar{z}_j}G(y)|  
%\le \sigma \cdot |\varphi(t)| \le |\varphi(t)|, 
%\]
%where $\varphi$ involves the third partial derivatives of $G$, 
%and $t\in Y$ is a point whose distance to $y$ is less than $\sigma$.   
Since $y\in  Y\setminus T_{2\sigma}$, we have  $|s_D(t)|_{h_D} \geq C_2 |s_D(y)|_{h_D}$, see \eqref{eqn:appg2}.  
Thus  
\begin{eqnarray*}
|\partial_{z_i} \partial_{\bar{z}_j} (G_\sigma - G) (y) | 
&\le&  \int_{w\in \mathbb{R}^{2n}}  \theta_\sigma (|w|^2) \cdot |\partial_{z_i} \partial_{\bar{z}_j} G(\exp_y(w)) - \partial_{z_i} \partial_{\bar{z}_j}G(\exp_y(0))| \mathrm{d}w  \\
&\le&   C_3 C_3'  |s_D(t)|_{h_D}^{-a_3}  \int_{w\in \mathbb{R}^{2n}}  \theta_\sigma (|w|^2) \cdot  \mathrm{d}w   \\
&\le&   C_3 C_3'  \cdot  (C_2|s_D(y)|_{h_D})^{-a_3}. 
\end{eqnarray*}
Hence $\partial_{z_i} \partial_{\bar{z}_j} G_\sigma (y)$ is bounded by $C'_2 |s_D(y)|_{h_D}^{-a_2} + C_3C'_3 \cdot  (C_2|s_D(y)|_{h_D})^{-a_3}$.  

Assume that $y\in T_{2\sigma}$.  
Since $G$ is continuous, we may assume that $|G| $ is bounded by the constant $ C_0'$. 
Then, by considering the partial derivatives with respect to $y$, we have 
\begin{eqnarray*}
|\partial_{z_i} \partial_{\bar{z}_j} G_\sigma   (y)|  
&=& \left|\int_{x\in (Y,\theta_Y)} (\partial_{z_i} \partial_{\bar{z}_j} (\theta_\sigma (|y-\cdot|^2 )\cdot \lambda(y,\cdot))(x) )\cdot G(x) \right| \\
&\le & \int_{ x\in (Y,\theta_Y)}  C_4 \cdot C_0 \cdot \sigma^{-b_0} \cdot |G(x)|  \\
&\le & \int_{x\in (Y,\theta_Y)}  C_4 \cdot C_0 \cdot \sigma^{-b_0} \cdot C'_0,  
\end{eqnarray*} 
where $C_4$ is a constant independent of $\sigma$ and $y$. 
For the first inequality above, we use the estimates on the derivatives of $\theta_\sigma$ to obtain the term $C_0 \sigma^{-b_0}$. 
We also use the fact that the partial derivatives, with respect to $y$ up to order $2$, of $|y-x|^2$ and of  $\lambda(y,x)$,  are bounded by some constant, over the domain $\{(x,y)\in Y\times Y \ | \ |x-y|<\sigma_0 \}$. 
Since  $y\in T_{2\sigma}$, as shown in \eqref{eqn:appg1},  we have 
\[
\sigma^{-b_0}  \le (C_1 |s_D(y)|_{h_D}^{b_1})^{-b_0}. 
\]
Hence $\partial_{z_i} \partial_{\bar{z}_j} G_\sigma   (y)$ is bounded by 
\[C_4\cdot C_0 \cdot C_1^{-b_0}  \cdot C_0' \cdot {Vol}(Y,\theta_Y) \cdot |s_D(y)|_{h_D}^{-b_1b_0},\] 
where $Vol$ is the volume. 
This completes the proof of the proposition. 
\end{proof}

%We observe that $\omega_\epsilon$ is monotone in $\epsilon$.  For uniform estimates of them, in the next lemma, we study the degenerate case when $\epsilon = 0$. 




%\subsection{a}

%In the lemma above, function $G$ is only bounded both from below and above, so  we cannot directly apply Theorem  \cite[]{GPSS}. Fortunately, we observe that the orbifold smoothness of $G$ is sufficient to apply the fundamental result of Guo-Phong-Sturm-Song. Hence we are able to enlarge the metric class \ref{def:akclass} by including orbifold smooth metric such that the uniform geometric estimates still hold. 
%We are interested in the degenerate family of orbifold smooth metrics $\omega_\epsilon:=p^*\omega_Z+\epsilon\omega_{\orb}, 0<\epsilon\leq 1$. 
%We shall prove that the  family of currents $\{\omega_{\epsilon}\}$  satisfies  certain uniform $L^1$ estimates,  as in the items $(2)$ and $(3)$ of Definition \ref{def:akclass}.  For this purpose, we start by verifying that $\omega_\epsilon$ satisfies the assumption of Definition \ref{def:akclass}. 

In the following proposition, we prove that the family of currents $\{\omega_{\epsilon}\}$  satisfies  certain uniform   estimates. 

\begin{prop}\label{prop:VerifyAKclass}  
There exists $A,K,p,\gamma$ such that  $\omega_\epsilon$ satisfies the assumption $(1)$-$(4)$ of Definition \ref{def:akclass} for all $0<\epsilon \leq 1$. 
In other words, the following properties hold. 
\begin{enumerate}

\item  ${\omega_\epsilon}$ has bounded local potentials. 
 

\item $[\pi^*{\omega_\epsilon}] \cdot [\theta_Y]^{n-1}\leq A$ and $[{ \omega_\epsilon}]^n \geq A^{-1}$.  
 

\item The $p$-th Nash-Yau entropy is bounded by $K$, i.e.
%
$${\mathcal{N}}_p ({\omega_\epsilon}) = \frac{1}{V_{\omega_\epsilon}} \int_Y \left|\log \frac{1}{V_{\omega_\epsilon}} \frac{(\pi^*{\omega_\epsilon})^n}{\theta_Y^n} \right|^p (\pi^*{\omega_\epsilon})^n  \leq K, $$
%
where $V_{\omega_\epsilon}=  [{\omega_\epsilon}]^n$. 

 
\item %There exists a non-negative $\gamma \in \mathcal{C}^0(Y)$ such that $\{\gamma=0\}$ is contained in a proper analytic subvariety of $Y$ and  
 $\frac{(\pi^*{\omega_\epsilon})^n}{ \theta_Y^n}\geq \gamma.$

\end{enumerate} 
\end{prop}


\begin{proof} 
The item $(1)$ holds, since the local potentials of $\omega_\epsilon$ are orbifold smooth, and hence bounded.  
The  item $(2)$ follows from the the monotonicity of $\omega_\epsilon$ in $\epsilon$ and  the fact that  $\omega_Z$ is K\"ahler.    
For the item $(4)$, by the monotonicity of $\omega_\epsilon$ again, it is enough to set 
\begin{equation}\label{eqn:gamma}
\gamma = \frac{(\rho\circ \pi)^*\omega_Z^n}{\theta_Y^n}. 
\end{equation}

%It remains to prove the item $(3)$ on Nash-Yau entropy.  We   remark that the total volumes are uniformly bounded for $\omega_\epsilon$.  Hence we me way ignore the term $V_{\omega_\epsilon}$ in the entropy. By the monotonicity  in $\epsilon$ of log volume ratios $\log \frac{\pi^* {\omega_\epsilon^n}}{\theta_Y^n}$,  it suffices to consider the cases when $\epsilon=0,1$.   When $\epsilon=0$, by Lemma \ref{degto0},  for any $\delta>0$, there is some real number $C_\delta>0$ such that   \[|\log \frac{(\rho\circ \pi)^*{\omega_Z^n}}{\theta_Y^n}| \leq  C_\delta \cdot  |s_D|_{h_D}^{-\delta}.  \] When $\epsilon=1$, by Lemma \ref{l:logtype},  for any $\delta>0$, there is some real number $C_\delta' >0$, such that   \[|\log \frac{{\omega_1^n}}{\theta_Y^n}|\leq C'_\delta \cdot  |s_D|_{h_D}^{-\delta}.  \]  As a consequence,  for any $q>1$,   the integrands $|\log\frac{\pi^*{\omega_\epsilon^n}}{\theta_Y^n}|$ are $L^q$ integrable, uniformly in $\epsilon$, with respect to $\theta_Y^n$.   We notice that $\frac{\pi^*{\omega_1^n}}{\theta_Y^n}$ is integrable on $(Y,\theta_Y)$,  since $\omega_1$ is orbifold smooth.  From Lemma \ref{l:logtype}, we see that the poles of $\frac{\pi^*{\omega_1^n}}{\theta_Y^n}$ has finite order along $D$.  It follows that  $\frac{\pi^*{\omega_1^n}}{\theta_Y^n}$ is $L^{1+\delta_0}$ integrable for some $\delta_0>0$.   Since $\omega_\epsilon^n \le \omega_1^n$ whenever $\epsilon \le 1$,  by Holder inequality,  we obtain that the $p$-th Nash-Yau entropy is bounded by some constant $K_p$, for any $p>1$.  \\
 
It remains to prove the item $(3)$ on Nash-Yau entropies.  
We notice that $\frac{\pi^*{\omega_1^n}}{\theta_Y^n}$ is integrable on $(Y,\theta_Y)$,  since $\omega_1$ is orbifold smooth.  
From Lemma \ref{l:logtype}, we see that  $\frac{\pi^*{\omega_1^n}}{\theta_Y^n}$ has polynomial  poles (with rational exponents)  along $D$. 
It follows that  $\frac{\pi^*{\omega_1^n}}{\theta_Y^n}$ is $L^{1+\delta_0}$ integrable for some $\delta_0>0$. 
Then by the monotonicity of $\frac{\pi^*\omega_\epsilon^n}{\theta_Y^n}$ in $\epsilon$, we have 
\begin{equation}\label{eqn:uniform-1+delta}
\int_Y\left(\frac{\pi^*{\omega^n_\epsilon}}{\theta_Y^n}\right)^{1+\delta_0}{\theta_Y^n}\leq C 
\end{equation} for some constant $C$ independent of $\epsilon$.   
Since the volumes $V_{\omega_\epsilon}$ are bounded between  $[\omega_Z]^n$ and $[\omega_1]^n$, up to enlarging the constant $C$, we have 
\begin{equation}\label{eqn:uniform-1+delta'}
\int_Y\left( \frac{1}{V_{\omega_\epsilon}} \cdot \frac{\pi^*{\omega^n_\epsilon}}{\theta_Y^n}\right)^{1+\delta_0}{\theta_Y^n}\leq C 
\end{equation}

For any $p>1$, and  for any smooth function $H$ on $Y$, 
we also have the following elementary inequality 
\begin{equation}\label{eqn:entropy}
    \int_Y|H|^pe^H \theta_Y^n\leq  C'+ C' \int_Y e^{(1+\delta_0)H}\theta_Y^n,
\end{equation} 
where $C'$ is a constant depending only on $(Y,\theta_Y)$,  $p$ and $\delta_0$. 
Hence the $p$-th Nash-Yau entropies of $\omega_\epsilon$ are uniformly bounded.  
\end{proof}



Note that by Lemma \ref{l:logtype}, we can write 
\[\log  \left(\frac{1}{V_{\omega_\epsilon}} \cdot \frac{ \pi^*{\omega_\epsilon^n}}{\theta_Y^n} \right)=F_{\log}+G_\epsilon, \] 
where  $G_\epsilon$ is a bounded continuous function on $Y$, 
and \[F_{\log} = a(\log |s_{E_1}|_{h_{E_1}} - \log |s_{E_2}|_{h_{E_2}} ) \] 
which  depends only on $X$ and $Y$.   
It follows that  $G_\epsilon + \log V_{\omega_\epsilon}$ is increasing in $\epsilon$.  
By \eqref{eqn:uniform-1+delta'} and by comparing with $ G_1$, we can find a positive constant  $  C''>0$, independent of $\epsilon$, such that 
\begin{equation}\label{eqn:zz}
\int_Y e^{(1+\delta_0)F_{\log}}{\theta_Y^n}\leq C'',\,\,\int_Y e^{(1+\delta_0)G_\epsilon}{\theta_Y^n}\leq C''.\end{equation} 

%For the function $\gamma$ in \eqref{eqn:gamma}, we observe that its logarithm has log type analytic singularities. 

%\begin{lemma}
%\label{lemma:log-gamma} 
%There is an effective divisor $E$, whose support is contained in the one of $D$,  such that 
%\begin{equation}\label{eqn:log-gamma-bound}
%\log \gamma \geq  \log |s_E|_{h_E}.   
%\end{equation}
%Here, $s_E\in H^0(Y,\mathcal{O}_Y(E))$ is a section defining $E$, and $h_E$ is a smooth Hermitian metric on $\mathcal{O}_Y(E)$.   
%\end{lemma}

%\begin{lemma}\label{degto0}
%Let $Z$ be a normal K\"ahler variety and $\omega_Z$ be a smooth K\"ahler form  on $Z$. Then there exists a resolution of singularities $\pi:Y\rightarrow Z$ such that $\pi^*|\log\frac{\omega_Z^n}{\Omega_Z\wedge\bar\Omega_Z}|$  {\color{red} Let $\pi\colon Y \to Z$ be a log resolution and let $\theta_Y$ be a K\"ahler form on $Y$. 
%The log volume ratio $|\log \frac{(\rho\circ\pi)^*\omega_Z^n}{\theta_Y^n}|$  is bounded by a function with analytic singularities along the exceptional divisors of $\pi$.  
%There is an effective divisor $E$ whose support is contained in the one of $D$, and a smooth function $\xi$, such that 
%\[
%\log \gamma \geq  \log |s_E|_{h_E} + \xi. 
%\]
%Here, $s_E\in H^0(Y,\mathcal{O}_Y(E))$ is a section defining $E$, and $h_E$ is a smooth Hermitian metric on $\mathcal{O}_Y(E)$.  
%Moreover, $E-aE_1 + aE_2 \geq 0$.    
%\end{lemma}


%\begin{proof} By definition, $Z$ is locally embedded in some large ambient space $\mathbb C^N$ and $\omega_Z$ is the restriction of a K\"ahler metric on $\mathbb C^N$. Since we only need a bound for $|\log\frac{\omega_Z^n}{\Omega_Z\wedge\bar\Omega_Z}|$, now we assume that $\omega_Z$ is the restriction of Euclidean metric $\omega_{Euc}:=\sum_{i=1}^{i=N}dz_i\wedge d\bar z_i$. Then $\omega_Z^n=\sum_{I\subset N,|I|=n}c(n,N)dz_I\wedge d\bar z_I$, where $c(n,N)$ is a constant depending on $n,N$ and $dz_I\wedge d\bar z_I:=\sum dz_{i_1}\wedge ... dz_{i_n}\wedge d\bar z_{i_1}\wedge ... d\bar z_{i_n}$. Since $\frac{dz_I}{\Omega_Z}$ is a holomorphic function on the regular locus, then by normality, it can be extended as a holomorphic function across the singularities. So $\frac{\omega_Z^n}{\Omega_Z\wedge \bar\Omega_Z}=\sum_{k=1}^{c(n,N)}|f_k|^2$, where $f_k$ is a holomorphic function. This completes the proof.
%\end{proof}

%{\color{red} I do not know how to compare $\Omega_Z\wedge\bar\Omega_Z$ with $\theta_Y^n$. I modify the proof as below.}

%\begin{proof} Since $Y$ is compact, the problem is indeed local around every point $y$ of the $\pi$-exceptional locus.  Hence, by restricting ourself on   coordinate open subsets of $Z$,  we may assume that $Z$ is embedded in some open subset $V$ of some $\mathbb{C}^N$, and that $\omega_Z$ is the restriction of a K\"ahler form $\omega_V$ defined on $V$.     We note that $\frac{\pi^*\omega_Z^n}{\theta_Y^n} = 0$ at $y$. Hence its logarithm is negative around $y$.    Let $\omega_{Euc}:= \sqrt{-1}\sum_{i=1}^{N}dz_i\wedge d\bar z_i$ be the Euclidean K\"ahler form on $V$. Then, up to shrinking $Z$ and $V$, there is a constant $C>0$ such that $\omega_V \geq C \omega_{Euc}$.  Therefore, it is enough to prove the lemma in the case when  $\omega_V=\omega_{Euc}$.  Assuming this situation,   we have  \[\omega_V^n=\sum_{I\subset \{1,...,N\},|I|=n}c(n,N)dz_I\wedge d\bar z_I,\]  where $c(n,N)$ is a constant depending on $n,N$,   $dz_I := dz_{i_1}\wedge \cdots \wedge dz_{i_n} $  and $  d\bar z_I:=  d\bar z_{i_1}\wedge \cdots  \wedge d\bar z_{i_n}$ for  $I=\{i_1<\cdots < i_n\}$. We  observe that $\pi^*(d z_I) $ is a holomorphic $n$-form on $Y$.   Hence, locally around $y$, we can write $\frac{(\rho\circ \pi)^*\omega_{Z}^n}{\theta_Y^n} = \sum_I g_I\cdot |f_I|^2$ for some holomorphic functions $f_I$ and some positive smooth function $g_I$.  
%We also notice that the set of common zeros of the functions $f_I$ is exactly the $\pi$-exceptional locus.  This completes the proof of the lemma.  \end{proof}




%\begin{proof} 
%For the lower bound of $\log \gamma$, 
%since $Y$ is compact, 
%it is enough to show that locally around every point $y$ of $Y$, 
%the function $\log \gamma$ can be bounded from below as in \eqref{eqn:log-gamma-bound}.  
%Hence, we may assume that $Z$ is  embedded in some open subset $V$ of some  $\mathbb{C}^N$. 
%Let $\omega_{Euc} = \sqrt{-1}\sum_{i=1}^N dz_i\wedge d\bar z_i $ be the Euclidean K\"ahler form on $\mathbb{C}^N$. 
%Then \[\frac{(\rho\circ \pi)^*\omega_Z^n}{(\rho\circ \pi)^*(\omega_{Euc}|_Z)^n}\] is a smooth positive function on $Y$.  
%Therefore, we may assume that $\omega_Z=\omega_{Euc}|_Z$. 
%Assuming this situation,   we have 
%\[\omega_{Euc}^n=\sum_{I\subset \{1,...,N\},|I|=n}c(n,N)dz_I\wedge d\bar z_I,\] 
%where $c(n,N)$ is a constant depending on $n,N$,   $dz_I := dz_{i_1}\wedge \cdots \wedge dz_{i_n} $  and $  d\bar z_I:=  d\bar z_{i_1}\wedge \cdots  \wedge d\bar z_{i_n}$ for  $I=\{i_1<\cdots < i_n\}$.
%We  observe that $(\rho\circ\pi)^*(d z_I|_Z) $ is a holomorphic $n$-form on $Y$.  
%Hence, locally around $y$, 
%we can write 
%\[\frac{(\rho\circ \pi)^*\omega_{Z}^n}{\theta_Y^n} = \sum_I g_I\cdot |f_I|^2\] 
%for some holomorphic functions $f_I$ and some positive smooth function $g_I$.  
%We also notice that the set of common zeros of the functions $f_I$ is exactly the $\pi$-exceptional locus. 
%We also remark that the common zeros of the $f_I$ are contained the $\rho\circ\pi$-exceptional locus, which is exactly $D$. 
%Hence we can find such a divisor $E$ and an appropriate metric $h_E$ as in the lemma. 
%For the second part of the lemma, 
%Then by Proposition \ref{prop:VerifyAKclass}, 
%we see that  \[F_{\log} + G_\epsilon  \geq \log \gamma \geq \log |s_E|_{h_E}\] for any $0<\epsilon\le 1$.   Since $G_\epsilon$ is continuous, and hence bounded  on $Y$,  we deduce that $F_{\log} - \log |s_E|_{h_E}$ is bounded.  This implies that $E-a E_1 \geq 0$,  and completes the proof of the lemma.
%\end{proof}

%With the notation above, we set 
%\begin{equation}\label{eqn:gamma-}
 %   \gamma^{-} =  |s_E|_{h_E}. 
%\end{equation} 
%Recall that both $E$ and $E_2$ are supported in the support of $D$. 
%Hence, up to replacing $E$ by $E+ \lceil a \rceil E_2$, 
%and up to multiplying $h_E$ by some small enough positive number,  
%we may assume that  
%\begin{equation}\label{eqn:E2-estimate}
%  |s_{E_2}|_{h_{E_2}}^{a}   \geq  \gamma^-. 
%\end{equation}





 
%\begin{thm}\cite[Theorem?]{GPSS}\label{sob} Let $X$ be an $n$-dimensional compact normal K\"ahler space. For any $\omega\in \mathcal{AK}(X, \theta_Y, n, p, A, K, \gamma)$, the metric measure space $(\hat X, d, \omega^n)$ associated to $(X, \omega)$ satisfies the following properties. 

%\begin{enumerate}

%\item There exists $C=C(X, \theta_Y, n, p, A, K, \gamma)>0$ such that  
%
%$${\textnormal{diam}}(\hat X, d) \leq C.$$
%
%
%In particular, $(\hat X, d)$ is a compact metric space.

%\medskip 

%\item  There exist $q>1$ and $C_S=C_S(X, \theta_Y, n, p,  A, K, \gamma, q)>0$ such that 
%
%$$
%\Big(\int_{\hat X} | u  |^{2q}\omega^n   \Big)^{1/q}\le C_S \left( \int_{\hat X} |\nabla u|^2 ~\omega^n + \int_{\hat X} u^2 \omega^n \right) .
%
%$$
%
%for all $u\in W^{1, 2}(\hat X, d, \omega^n)$. 

%\medskip



%\item There exists $C=C(X, \theta_Y, n, p, A, K, \gamma)>0$ such that the following trace formula holds for the heat kernel of $(\hat X, d, \omega^n)$
%
%$$H(x,x, t) \leq \frac{1}{V_\omega} + \frac{C}{V_\omega} t^{-\frac{q}{q-1}}. $$ 

%\medskip

%\item Let $0=\lambda_0 < \lambda_1 \leq %\lambda_2 \leq ... $ be the increasing sequence %of eigenvalues of the Laplacian $-\Delta_\omega$ %%on $(\hat X, d, \omega^n)$. Then there exists %$c=c(X, \theta_Y, n, p, A, K, \gamma)>0$ such %that
%
%$$\lambda_k \geq c k^{\frac{q-1}{q}}. $$

%\end{enumerate}


%\end{thm}



%Once we have Proposition \ref{prop:appg} and Proposition \ref{prop:VerifyAKclass} proved, we follow the argument of \cite{GPSS} to finish the proof of our first main Theorem \ref{thm:soborbi}.  {\color{red}We do not need 3.11 for 3.4.??? Xin: why not needed?}  {\color{blue}theorem 3.4 is never proved in this paper...}


In the following argument, 
we will approximate $\pi^*\omega_\epsilon$ by a family of smooth K\"ahler forms $\omega_{\epsilon,j}$.  
By abuse of notation,  
we will omit the subscript $\epsilon$ and set $\omega:=\omega_\epsilon$. 
We choose a smooth closed $(1,1)$-form $\omega_0 \in [\omega]$. Since $\omega$ is orbifold smooth, it has continuous local potentials. 
Hence  there exists a unique $\varphi\in PSH(X, \omega_0)\cap \mathcal{C}^0(X)$ such that
%
$$\omega= \omega_0+\ddbar \varphi, ~\sup_X \varphi=0. $$
%
We set 
$$Q:=\log \left(\frac{1}{V_\omega} \cdot \frac{(\pi^*\omega)^n}{\theta_Y^n}\right)=F_{\log} + G.  $$
%
Then $G$ satisfies the assumptions of  Proposition \ref{prop:appg}. 
In addition, by Proposition \ref{prop:VerifyAKclass}, we have 
%
$$\mathcal{N}_p(\omega) = \frac{1}{V_\omega} \int_Y |Q|^p (\pi^*\omega)^n =   \int_Y |Q|^p e^Q \cdot \theta_Y^n \leq K .$$
 
%
\begin{lemma}\label{lemma:goodapp}
We can find a sequence of smooth functions $\{Q_j\}_{j\gg 1}$ on $Y$,   
which converges to $Q$, smoothly on any compact subsets of $Y\setminus D$.  
In addition,  the  following properties hold.   
\begin{enumerate}

\item Let $\gamma$ be the function defined in \eqref{eqn:gamma}.  
There is a constant $c>0$, independent of $\epsilon$ and $j$, such that 
for all sufficiently large $j$,  we have
\[  
 e^{Q_j} \geq  c \cdot  \gamma \cdot |s_{E_2}|_{h_{E_2}}^a. 
\]  
 

%\item There exists a constant $K'>0$, independent of $\epsilon$ and  $j$,   such that  for all $j > 0$ sufficiently large, we have % \begin{equation}\label{goodapp}
% \int_Y |Q_j|^p e^{Q_j} \theta_Y^n \leq K'.
% \end{equation}
%
\item % Assume that  $\|e^Q\|_{L^{1+\delta}(Y, \theta_Y)} <\infty$ for some 
For any $\delta\geq 0$ small enough, there exists $K'>0$, independent of $\epsilon$ and 
$j$,   
such that for all $j>0$ sufficiently large, we have 
% 
\begin{equation}\label{app} 
\|e^{Q_j}\|_{L^{1+\delta}(Y, \theta_Y)} \leq K'. 
\end{equation}
% where constants $K',K''$ are independent of $\epsilon$. 

\medskip

\item %Let $D$ be an effective divisor of $Y$ such that the support of $D$ contains  the exceptional locus of $\pi$ and the singular locus of $F$. We can assume that 
% $$\pi^*[\omega] - \delta [D]  $$ { \color{red} maybe we should  write pullback of $\omega_Z - \delta D$ is Kahler  }
% is a K\"ahler class for some sufficiently small $\delta>0$,  which is independent of $\epsilon$. 
%
%Let $\sigma_D\in H^0(Y,\mathcal{O}_Y(D))$ be a defining section of $D$ and $h_D$ be a smooth Hermitian metric on $\mathcal{O}_Y(D)$. 
There exists $N_\epsilon>0$ and $C_\epsilon>0$, possibly depends on $\epsilon$,  
such that  
$$ \sup_j \|\nabla^2 Q_j \|_{\theta_Y} \leq C_\epsilon |s_D|^{-2N_\epsilon}_{h_{{D}}}. $$
\end{enumerate}
%
\end{lemma}
\begin{proof} We approximate the two functions $F_{\log}$ and ${G}$ separately. For the approximation of $F_{\log}$, we set
$$F_j=   \frac{a}{2} \cdot \log \left(\frac{|s_{E_1}|_{h_{E_1}}^2+ j^{-1}}{|s_{E_2}|_{h_{E_2}}^2+ j^{-1}}  \right).  $$ 
Then  $\{F_j\}$  converges to $F_{\log}$ smoothly on any compact subset of $Y\setminus D$.   
We approximate the function $G$ by a sequence of smooth functions $G_j$ according to Proposition \ref{prop:appg}. 
More precisely, we may let $G_j$ be $G_{\frac{1}{j}}$ with the notation in Proposition \ref{prop:appg}.  
Let $Q_j=F_j + G_j$.    
Then we can verify that the item $(3)$ holds. 



For the item (1), we first recall that $ \frac{(\pi^*\omega)^n}{\theta_Y^n} \geq \gamma$ by Proposition \ref{prop:VerifyAKclass}.   
Since $V_{\omega}$ is bounded by positive numbers independent of $\epsilon$, 
we deduce that 
\[Q= F+G\geq \log  (c'\cdot \gamma )\] 
for some constant $c'>0$ independent of $\epsilon$.   
Since $\{G_j\}$ converges to $G$ uniformly on $Y$, 
we may assume that $G_j\geq G-1$. 
Then 
\begin{eqnarray*}
Q_j- \log c'\gamma &\ge&  ( F_j - F )+ (F + G -\log c'\gamma)  -1  \\ 
& \geq &  \frac{a}{2} \log \left( \frac{|s_{E_2}|_{h_{E_2}}^2}{|s_{E_2}|_{h_{E_2}}^2+ j^{-1}}   \right) + 0 - 1. 
\end{eqnarray*} 
We note that $\frac{a}{2} \log (|s_{E_2}|_{h_{E_2}}^2+ j^{-1}) \leq \frac{a}{2} \log (|s_{E_2}|_{h_{E_2}}^2+ 1) $  is bounded from above  by some constant $\lambda$ depending only on $(E_2,h_{E_2},a)$. 
Hence we deduce that  
\[
Q_j- \log c' \gamma  \geq a \log |s_{E_2}|_{h_{E_2}}   -(1+\lambda)
\]
By setting $c=c'\cdot e^{-(1+\lambda)}$, we obtain the item $(1)$.  



For the item $(2)$, we first fix some $\delta \geq 0$ small enough. 
Then the $L^{1+\delta}$-norm of $e^{F+G}$ on $(Y,\theta_Y)$ is bounded by  some constant independent of $\epsilon$, as shown in \eqref{eqn:uniform-1+delta'}.  
Since  $\{G_j\}$ converges to $G$ uniformly on $Y$, we only need to prove that  
the $L^{1+\delta}$-norms of $e^{F_j}$ are bounded by some constant, independent of $\epsilon$ and $j$.   
We have the following estimate 
\begin{equation}
\label{eqn:dominated-conv}
    e^{F_j} \le (|s_{E_1}|_{h_{E_1}}^2+1)^{\frac{a}{2}}  \cdot  |s_{E_2}|_{h_{E_2}}^{-a}.  
\end{equation}
We have seen in \eqref{eqn:zz}, that $e^{(1+\delta)F_{\log}}$ is integrable. 
Since $E_1$ and $E_2$ do not have common component, 
it  follows that $|s_{E_2}|_{h_{E_2}}^{-a(1+\delta)}$ is integrable, and so is the RHS of the inequality above.   
By the dominated convergence theorem, we deduce the following convergence, 
\[
 \|e^{F_j}\|_{L^{1+\delta}(Y, \theta_Y)} \to \|e^{F_{\log}}\|_{L^{1+\delta}(Y, \theta_Y)}. 
\]
By  \eqref{eqn:zz} again, we can deduce a uniform constant $K'$ for the item $(2)$.  
This completes the proof of the lemma.   
\end{proof}

We will now use the smooth functions $Q_j$ to construct smooth forms  approximating  $\omega$. 
Recall that  $\omega=\omega_0+\ddbar\varphi$.
Pulling back to $Y$, we have
\begin{equation}\label{eqn:ss}
 (\pi^*\omega_0+\ddbar\pi^*\varphi)^n=V_\omega\cdot e^Q\theta_Y^n.
\end{equation}
Let $\{\delta_j\}$ be a sequence of positive real numbers in $(0,1)$ converging to $0$. 
We  consider the following perturbed complex Monge-Amp\`ere equation
%
\begin{equation}\label{eq}
%
 (\pi^*\omega_0 + \delta_j \theta_Y + \ddbar \varphi_j)^n = e^{Q_j + c_j} \theta_Y^n, ~~\sup_X \varphi_j = 0,
 %
 \end{equation}
%
where $c_j$ is the normalizing constant satisfying 
%
$$\int_Y e^{Q_j+ c_j} \theta_Y^n = \left(\pi^* [\omega_0] + \delta_{{j}} [\theta_Y] \right)^n. $$
%
Then the solution $\varphi_j$ exists and is smooth by Yau's theorem.  We define 
%
\begin{equation}\label{eqn:omegaj}
%
\omega_j = \pi^*\omega_0 + \delta_j \theta_Y + \ddbar \varphi_j. 
%
\end{equation}

%Now we verify that $\omega_{j}\in\mathcal  W(A,K,p,n,\gamma)$ (c.f \cite[]{GPSS24} for the precise definition of $\mathcal{W}(A,K,p,n,\gamma)$. ) 


\begin{lemma}\label{lemma:unisob} There exist  constants $ {A}^\circ,{K}^\circ,{p}^\circ$ and a non-negative continuous function $ {\gamma}^\circ$ on $Y$, all independent of $\epsilon$ and $j$, satisfying the following property. 
There is an integer $M_\epsilon>0$, such that $\omega_{j}\in  \mathcal{W}(Y, \theta_Y, n, p^\circ, A^\circ, K^\circ, \gamma^\circ)$ whenever $j \geq M_\epsilon$.
\end{lemma}  

 
\begin{proof}
We observe that $[\omega_j]^n$ and $[\omega_j]\cdot [\theta_Y]^{n-1}$ are uniformly bounded. 
This gives a constant $A^\circ$.  
Moreover $[\omega_j]^n\geq [(\rho\circ \pi)^*\omega_Z]^n>0$.  
Next, we will show that $\frac{\omega_j^n}{\theta_Y^n} = e^{Q_j+c_j}$ is bounded from below by some $\gamma^\circ$.  
By the item $(1)$  of Lemma \ref{lemma:goodapp}, 
it is enough to show that, 
for  $j$ sufficiently large,  the following number 
$$ |{c_j}|=\left|\log\frac{\left(\pi^* [\omega_0] + \delta_{{j}} [\theta_Y] \right)^n}{\int_Ye^{Q_j}\theta_Y^n}\right|$$ 
is  bounded, by  a constant independent of $\epsilon$ and $j$. 
%It is clear that the numerator of the RHS is bounded from above by $([\pi^*(\rho^*\omega_Z+\omega_{\orb})]+[\theta_Y])^n$  and from below by $[(\rho\circ \pi)^*\omega_Z]^n$.  Now we estimate the denominator. By \eqref{app} and H\"older inequality, it is clear that $\int_Ye^{Q_j}\theta_Y^n\leq K''$ for $j$ sufficiently large, where  $K''$ is a constant independent of $\epsilon$ and $j$.   Let $U\subseteq Y\setminus D$ be a non-empty relatively compact open subset.  Then $Q_j$ converges uniformly to $Q$ on $U$.  Hence there is some small enough constant $\lambda>0$, independent of $\epsilon$ and $j$, such that the following inequalities hold, 
%   \[
%   \int_Ye^{Q_j}\theta_Y^n\geq -\lambda + \int_U e^Q \theta_Y^n \geq -\lambda + \int_U (\rho\circ \pi)^*\omega_Z^n.  
%   \]  
We recall that $Q_j=F_j+G_j$ such that $\{G_j\}$ converges uniformly to $G$. 
By \eqref{eqn:dominated-conv} and by using the dominated convergence theorems,  
we see that $\int_{(Y,\theta_Y)} e^{Q_j} \to \int_{(Y,\theta_Y)}e^Q$. 
It follows that the sequence $\{c_j\}$ converges to $\log  V_{\omega}$, which is bounded by constants independent of $\epsilon$.   



It remains to prove that, there is some $p^\circ\geq 1$, 
such that the  $p^\circ$-th Nash-Yau entropy of $\omega_j$ 
\[
\mathcal{N}_{p^\circ}(\omega_j) =  \frac{1}{[\omega_j]^n}  \int_Y (Q_j+c_j)^{p^\circ} \cdot e^{Q_j+c_j} \cdot \theta_Y^n
\]
is  bounded by some constant $K^\circ$, for all $j$ sufficiently large.    
Let $p^\circ\geq 1$ be arbitrary.  
We have proved that, for $j$ sufficiently large,  
 $|c_j|$ and $([\omega_j]^n)^{-1}$ are bounded by constants independent of $\epsilon$ and $j$.   
Hence by using \eqref{eqn:entropy} and   \eqref{app}, 
we can deduce a uniform bound for $\mathcal{N}_{p^\circ}(\omega_j)$. 
This completes the proof of the lemma.  
\end{proof}

In order to show that the family $\{\omega_j\}$ converges to $\omega$, we first prove the  following uniform estimates for the potentials $\varphi_j$.  
%Now we will show the sequence of smoothing metric $\omega_{\epsilon,j}$ converge locally and smoothly to $\omega$ away from the divisor $D$. 
\begin{lemma} 
\label{lemma:2nd} There exist $N'_\epsilon, C'_\epsilon>0$, possibly depend on $\epsilon$,  such that for all $j>0$ sufficiently large, 
%
$$\|\varphi_j\|_{L^\infty(Y)} \leq C'_\epsilon,  \ \Delta_{\theta_Y} \varphi_j \leq C_\epsilon'|s_D|_{h_D}^{-2N_\epsilon'}. $$

\end{lemma}
\begin{proof} Thanks to  Lemma \ref{lemma:goodapp},  
we can argue exactly as in  
\cite[Lemma 7.1]{GPSS}. 
%( in particular item $(3)$ is nontrivial) and  
\end{proof}
 

We can then deduce that  the sequence of smoothing metric $\omega_{j}$ converges locally and smoothly to $\omega$ away from the divisor $D$.  
\begin{lemma} \label{omapp} 
%Let $\varphi$ be the potential on $X$ so that $\omega=\omega_0+\ddbar \varphi$. 
For any relatively compact open subset $\mathcal{K} \subset   Y\setminus D$ and for any integer $k\geq 0$, we have 
%For $\mathcal{K} \subset \subset X\setminus \cS_{X, \omega}$ and $k>0$, we have   
%
$$\lim_{j\rightarrow \infty} \|\varphi_j - \pi^*\varphi\|_{L^\infty(\mathcal{K})} =0, $$
%
$$\lim_{j\rightarrow \infty}  \|\omega_j - \omega\|_{\mathcal{C}^k(\mathcal{K})} =0.$$
%
%Here we identify $\omega_j$ and $\pi_*\omega_j$ on $X\setminus \cS_{X, \omega}$. 
%In particular, $\omega$ is a smooth K\"ahler metric on $X\setminus \cS_{X, \omega}$. 
\end{lemma}  
\begin{proof}Note that $\omega_j$ is smooth outside $D$ and the sequence $\{Q_j\}$ converges smoothly outside $D$. By the second inequality of Lemma \ref{lemma:2nd}, we have a uniform $\mathcal{C}^2$ estimates of $\varphi_j$ for any compact set $K$ inside  $Y\setminus D$. 
By Evans-Krylov theory, we can obtain local higher order estimates for $\varphi_j$,  uniformly away from $Y\setminus D$.
\end{proof} 



Now we can conclude Theorem \ref{thm-orbifold-AK-property}. 

\begin{proof}[{Proof of Theorem \ref{thm-orbifold-AK-property}}] 
From Lemma \ref{lemma:goodapp} to Lemma \ref{omapp},  We have proved that for each $\omega_\epsilon$, it admits a sequence of  approximations $\{\omega_j\}$, 
belonging to  the same class $ \mathcal{W}(Y, \theta_Y, n, p^\circ, A^\circ, K^\circ, \gamma^\circ)$. 
By the same argument as in   \cite[Section 8]{GPSS},  
we can prove the statements of Theorem \ref{thm:soborbi}  for the family $\{\omega_\epsilon\}$, uniformly in $\epsilon$.   
\end{proof}









\begin{remark}\label{rmk:heatuni} For orbifold smooth K\"ahler form $\omega=\omega_\epsilon$, the existence of orbifold smooth heat kernel is known to exist \cite[Proposition 4.1]{Chiang90}. 
Following the same lines of \cite[Corollary 10.5]{GPSS}, we can verify that the orbifold heat kernel is identical with the heat kernel in Definition \ref{def:Heatkernel} for $\omega$.  
\end{remark}



We also need the following statement in the next section. 

\begin{lemma}
\label{lemma:convergence-L1} 
Let $\eta$ be a continuous function on $Y\setminus D$ such that $|\eta|$ is bounded by  $-\alpha \cdot \log |s_D|_{h_D} + \beta$, where $\alpha,\beta>0$ are constants. 
Then the following convergence holds, 
\[
\int_{(Y,\omega_j)} \eta  \to  \int_{(Y,\omega)} \eta    \mbox{ when }  j\to +\infty.  
\]
\end{lemma}

\begin{proof}
We have seen in the proof of Lemma \ref{lemma:unisob} that  the sequence $\{c_j\}$ converges. 
Hence for $j$ sufficiently large, there is a constant $\nu$, independent of $j$, 
such that 
\[
e^{Q_j+c_j} \le \nu \cdot  (|s_{E_1}|_{h_{E_1}}^2+1)^{\frac{a}{2}}  \cdot  |s_{E_2}|_{h_{E_2}}^{-a}. 
\] 
We notice that the product of $-\alpha \cdot \log |s_D| + \beta$ with the RHS above is integrable on $(Y,\theta_Y)$.  
Hence we can conclude by using the dominated convergence theorem.  
\end{proof}





%For our purpose, we will only state the Heat kernel estimate. 

%Now by following the argument of \cite[Theorem 3.1]{GPSS} word by word, we can prove the existence and uniqueness of Heat kernel.

%\begin{thm}\label{mainthm:orbi}
%The Heat kernel as in Definition $\ref{def:Heatkernel}$ exists, and there are  constants $q,C>1$ independent of $\epsilon$ such that the following holds for Heat kernel of $\omega = \omega_\epsilon$: 
%$$H(x,x, t) \leq \frac{1}{V_\omega} + \frac{C}{V_\omega} t^{-\frac{q}{q-1}}. $$ 
%Moreover, if there is another solution $\hat H(x,y,t)$ of the equation in Definition \ref{def:Heatkernel} and satisfying
%\begin{enumerate}\item For any  $0<a<b$, there is a constant $C_1=C_1(a,b)$ such that $|\hat H(t,x,y)|<C$ whenever $t\in [a,b]$,
%\item   For any  $0<a<b$,   there is a constant $C_2=C_2(a,b)$ such that $\|\nabla\hat H(t,x,y)\|_{L^2}<C$ whenever $t\in [a,b]$,
%\end{enumerate}
%then $\hat H=H$.
%\end{thm}
%\begin{proof} The existence of $H(t,x,y)$ is already proved by following \cite{GPSS}. It remains to prove the uniqueness. And we set $\omega:=\omega_\epsilon$ for the rest part of the proof of the theorem.
%Take  cut-off functions $\eta_\delta$ as in Lemma \cite[Lemma 8.2]{GPSS} satisfying $\|\nabla\rho_\delta\|_{L^2}\leq \delta^{1/2}$. Fix a time $T>0$, it suffices to prove that the function $$t\in (0,T)\mapsto \varphi(t): = \int_{Y^\circ} H(x,z,t) \hat H(z, y, T-t) \omega^n(z)$$ is a constant. Then by the Dirac property of $H$ and $\hat H$, we see that $\varphi(t)\to \hat H(x,y, T)$ as $t\to 0^+$ and $\varphi(t)\to H(x,y, T)$ as $t\to T^-$. The uniqueness is proved. 

%Set $\varphi_\delta(t): = \int_{Y^\circ} \eta_\delta(z)H(x,z,t) \hat H(y, z, T-t) \omega^n(z)$, then compute by using heat equation, we deduce that
%$$\varphi'_\delta(t): = \int_{Y^\circ} \eta_\delta(z)\big(\Delta_zH(x,z,t) \hat H(y, z, T-t)-H(x,z,t)\Delta_z\hat H(y,z,T-t)\big) \omega^n(z).$$
%Now fix a small $\epsilon_0>0$ and $t\in [\epsilon_0, T-\epsilon_0]$, there is a constant $C>0$ depending  on $\epsilon_0>0, T$ such that
%$$
%|\varphi_\delta'(t)|  =  \Big| \int_{Y^\circ} (-\langle \nabla \eta_\delta, \nabla H\rangle \hat H + \langle \nabla \eta_\delta, \nabla \hat H \rangle H)\omega^n\Big| \le C \delta^{1/2},
%$$
%by using the H\"older's inequality together with the upper bounds of $\| \nabla H\|_{L^2}$, $\| \nabla \hat H\|_{L^2}$, $|H|_{L^\infty}$,$|\hat H|_{L^\infty}$ in item $(1)$ and $(2)$. Integrating this over $[t_1, t_2] \subset [\epsilon_0, T-\epsilon_0]$ gives 
%$$|\varphi_\delta(t_1) - \varphi_\delta(t_2)| \le C  \delta^{1/2}.$$ Letting $\delta\to 0$ yields $\varphi(t_1) = \varphi(t_2)$. Done.
%\end{proof}
%\begin{remark}The proof of uniqueness above is a copy of \cite{GPSS}, we include the proof  here to present the minimum assumptions needed to show the uniqueness, which is not explicit stated there.
%\end{remark}






\section{Uniform $\mathcal{C}^0$-estimates on  Hermitian-Einstein metrics}  
\label{section:HE}

We fix the following notation for this section. 
Let $(Z,\omega_Z)$ be a compact K\"ahler variety of dimension $n$,  
which has quotient singularities in codimension $2$,  
and let $\mathcal{F}$ be a reflexive coherent sheaf on $Z$.  
We assume that $\mathcal{F}$ is $\omega_Z$-stable. 
Let $\rho \colon  X\to Z$ be an orbifold modification so that there is an orbifold structure $\mathfrak{X}$ over $X$.  We denote $\mathcal{E} = (\rho^*\mathcal{F})^{**}$. 
We may assume that there is an orbifold vector bundle $\mathcal{E}_{\orb}$ on $\mathfrak{X}$, which descend to $\mathcal{E}$, away from the $\rho$-exceptional locus and the branched locus of $\mathfrak{X}$,  see \cite[Section 9]{Ou2024}.  
We emphasize that, by construction, the indeterminacy locus of $\rho^{-1}$ has codimension at least $3$ in $Z$, and the codimension $1$ part of the branched locus of $\mathfrak{X}$ is $\rho$-exceptional. 
In addition, we can assume that there is some $\rho$-exceptional $\rho$-ample divisor (see \cite[Remark 8.2]{Ou2024}).      
Let $\omega_{\orb}$ be a K\"ahler current on $X$ which corresponds to  an orbifold K\"ahler form, and let $\omega_\epsilon = \rho^*\omega_Z +  \epsilon\omega_{\orb}$ for all $ 0< \epsilon\leq 1$.   
Without loss of the generality, we  assume that  $\omega_\orb \geq \rho^*\omega_Z$.  
Then the orbifold  vector bundle $\mathcal{E}_{\orb}$ is stable with respect to $\omega_\epsilon$ for all $\epsilon>0$ small enough by \cite[Claim 9.5]{Ou2024}. 
We fix an orbifold smooth Hermitian metric $h$ on $\mathcal{E}_\orb$.  
By abuse of notation, we also denote by $h$ the induced metrics on $\mathcal{E}$, 
which is well-defined at least on some dense Zariski open subset of $X$.  
Let $(\mathcal{L}_\orb, h_{\mathcal{L}})$ be the determinant line bundle of $(\mathcal{E}_\orb,h)$, and $\theta_\mathcal{L}$ be the Chern curvature of $h_{\mathcal{L}}$.  
Then $\theta_{\mathcal{L}}$ can also be viewed as a current on $X$ which is orbifold smooth.

%Let $h_{\epsilon,HE}$ be the orbifold Hermitian-Einstein metric on $\mathcal{E}_{\orb}$ with respect to $\omega_\epsilon$.  By abuse of notation, we also denote by $h_{\epsilon,HE}$ the induced metric on $\mathcal{E}$.  


Let $\pi\colon Y\to X$ be a log resolution of the closed analytic subset $\Sigma\subseteq X$, 
where $\Sigma$ is the union of the 
branched locus of the orbifold structure $\mathfrak{X}$ and  the $\rho$-exceptional locus.   In particular, the $(\rho\circ \pi)$-exceptional locus is a snc divisor.   
%We assume further that $(\rho\circ\pi)^*\mathcal{F}/\mathrm{torsion}$ is locally free on $Y$.    
We note that the $(\rho\circ \pi)$-exceptional locus contains the $\pi$-preimage of the branched locus of the orbifold structure $\mathfrak{X}$, by the construction of $\rho$.    
We choose  an effective divisor $D$ on $Y$ whose support is equal this exceptional divisor, 
so that   $[(\rho\circ \pi)^*\omega_{Z}] - \delta[D]$ is a K\"ahler class on $Y$ for all $\delta>0$ small enough.  
Let $s_D\in H^0(Y,\mathcal{O}_Y(D))$ be a section defining $D$, 
and let  $h_D$ be a smooth Hermitian metric on the line bundle $\mathcal{O}_Y(D)$, 
so that 
\[(\rho\circ \pi)^*\omega_{Z} -\delta \ddbar\log |s_D|_{h_D}\]  
is a K\"ahler form on $Y$ for all $\delta>0$ small enough. 
 

We note that,  throughout this section,  
all possibly singular metrics, functions or currents are indeed smooth objects defined on the largest Zariski open sets where $Y,X,Z$ are isomorphic.   
Therefore, by abuse of notation, we may use the same letter for such objects, 
which are eventually the same  on these isomorphic open sets,  
without specifying the compactifications $Y,X,Z$.   




We introduce some quantities related to Hermitian-Einstein metrics for this section.  
By \cite[Theorem 1]{Faulk2022}, the orbifold vector bundle $\mathcal{E}_{\orb}$ admits an orbifold Hermitian-Einstein metric  $h_{\epsilon,HE}$ with respect to $\omega_\epsilon$.  
We can interpret these metrics as follows,  
% \begin{defn}\label{defn:a} Let $h_{\epsilon,HE}$ be the Hermitian-Einstein metrics, set
\begin{equation*}
h_{\epsilon,HE} =: h\cdot e^{- \frac{1}{r}\rho_\epsilon}\exp(s_\epsilon),\,\,\,H_\epsilon:= e^{- \frac{1}{r}\rho_\epsilon}\exp(s_\epsilon) ,\,\,\,
 S_\epsilon:= \exp(s_\epsilon),
\end{equation*}
where $s_\epsilon$ is an $h$-self-adjoint endomorphism of $\mathcal{E}$ such that $\Tr s_\epsilon= 0$.  
We  have the equality 
\[
\log \Tr H_\epsilon= \log \Tr S_\epsilon- \rho_\epsilon.
\]
%\end{defn}
The Einstein condition implies that $\rho_\epsilon$ satisfies the following equation
\begin{equation}\label{eqn:a}
\Lambda_{\epsilon} \theta_\mathcal{L}+ \Delta_{\epsilon}'\rho_{\epsilon}= 
\frac{1}{Vol(X, \omega_{\epsilon})}\int_{X}c_1(\mathcal{E},h)\wedge (\omega_{\epsilon} )^{n-1},  
\end{equation}
where $Vol$ is the volume, 
 $\Delta_\epsilon$ is the Laplace-Beltrami operator for $\omega_\epsilon$ and $\Delta_\epsilon'= \frac{1}{2}\Delta_\epsilon$.  
Up to adding a constant, we may assume that $\rho_\epsilon$ is the unique  solution so that 
\[\displaystyle \int_{(X,\omega_\epsilon)}\rho_{\epsilon} = 0,\] 
see \cite[Theorem 2.6]{Chiang90}.  
%Note that the RHS of \eqref{eqn:a} is regarded as a constant function on $X$. 
Since $\Lambda_\epsilon\theta_L$ is orbifold smooth, so is the solution  $\rho_\epsilon$. 


The main objective of this section is to  show that, there is  a sequence of  $h_{\epsilon,HE}$, which converges to a Hermitian-Einstein metric with respect to $\rho^*\omega_Z$ as $\epsilon\rightarrow 0$.   
The key   is to prove certain uniform $\mathcal{C}^0$ estimates on the endomorphisms  $H_{\epsilon}$, see Proposition \ref{prop:C0} for the precise statement. 

We remark that the case when $h_{\epsilon,HE}$ are smooth with respect to a degenerate family of K\"ahler forms is addressed in \cite{CGNPPW}. 
In our case, the new difficulty is the lack of uniform geometric estimates of the family   $\{\omega_\epsilon\}$, 
which are proved in Section \ref{section:uniform-kahler}. 
Essentially, 
this is the only  different part for the convergence of $h_{\epsilon,HE}$,   comparing with \cite{CGNPPW}.  We also remark that it may be possible to take subsequential limit by using compactness result on  Hermitian-Yang-Mills connections in  \cite{Tian00}.



\subsection{Uniform mean value type inequalities for Hermitian-Einstein metrics}
The purpose of this subsection is to prove two  uniform mean value type inequalities for the Hermitian-Einstein metrics $h_{\epsilon,HE}$.  
We adapt the method  of \cite[Section 2.2]{CGNPPW}

%The new input is the uniform geometric control for the family of orbifold smooth metric $\omega_\epsilon$ derived in Section \ref{section:uniform-kahler}. 



\begin{lemma}\label{lemma:boundrho}
There exists   positive constants $C,C'>0$, independent of $\epsilon$, such that the following inequalities hold for all $\epsilon>0$ small enough.
\begin{equation*}  C' \log |s_D|_{h_D}^{2}+ |\rho_\epsilon| \leq  C\left(1 + \int_{(X,\omega_\epsilon)}|\rho_\epsilon| \right).  
\end{equation*}  
\end{lemma}  



\begin{proof}  %Throughout the proof, the letter $C$ denotes a constant which may change from line to line, but it is always independent of $\epsilon$.   
We will first establish the following inequality 
\begin{equation}\label{eqn:rho+} 
C' \log |s_D|^{2}+ \rho_\epsilon \leq  C\left(1 + \int_{(X,\omega_\epsilon)}|\rho_\epsilon| \right). 
\end{equation}  
By assumption at the beginning of the section, there is some effective $\rho$-exceptional Cartier divisor $D_X\subseteq X$ so such that $\rho^*\omega_Z-\delta[D_X]$ is a K\"ahler class for some $\delta>0$ small enough. 
It is then an orbifold K\"ahler class as well. 
Let $\sigma \in H^0(X,\mathcal{O}_X(D_X))$ be a section defining $D_X$.
Then, by $\partial\bar{\partial}$-lemma for compact K\"ahler orbifolds (see for example \cite[Section 7]{Baily56}),  there is some orbifold smooth Hermitian metric $\gamma$ on $\mathcal{O}_X(D_X)$, such that $\rho^*\omega_Z+ \delta \ddbar \log|\sigma|_\gamma^{2}$ is an orbifold 
K\"ahler form.  
In particular, $\gamma$ is continuous.  
Since $\theta_\mathcal{L}$ is orbifold smooth,  there is some constant $C_1>0$ such that 
\begin{align*}
\theta_\mathcal{L}\leq C_1(\rho^*\omega_Z+  \delta \cdot \ddbar \log|\sigma|_\gamma^{2}).  
\end{align*}  
%Since $\pi^*\rho^*\omega_Z +  \delta \ddbar \log|s_D|_{h_D}^2$ is a K\"ahler form on $Y$,  and s
Since $\rho^*\omega_Z\leq \omega_\epsilon$,   we deduce that 
\begin{align*}
\theta_\mathcal{L}\leq C_1(\omega_\epsilon+ \delta \cdot \ddbar \log|\sigma|_\gamma^2  ).  
\end{align*}  
We set $\eta= C_1\cdot \delta \cdot  \log|\sigma|_\gamma^2$.  
Then we have 
\begin{equation*}\label{eqn:mean2}
\Lambda_\epsilon \theta_L\leq C_1+ \Delta_\epsilon'\eta.  
\end{equation*}
Combining with \eqref{eqn:a}, we obtain that 
\begin{equation}\label{eqn:mean2'} 
  \Delta_{\epsilon}'(\rho_{\epsilon} + \eta)\geq -C_1 + 
\frac{1}{Vol(X, \omega_{\epsilon})}\int_{X}c_1(\mathcal{E},h)\wedge ( \omega_{\epsilon})^{n-1}.
\end{equation}



Recall that, for a fixed $\epsilon$,  
there is a  sequence of smooth K\"ahler form  $\omega_{j}$ on $Y$ which converges  to  $\omega_\epsilon$ outside $D$, 
see   \eqref{eqn:omegaj}.  
We note that  $\pi^*\eta$  has at most log poles along $D$ since the support of $\pi^*D_X$ is contained in the one of $D$.  
Hence, by \eqref{eqn:uniform-1+delta} and H\"older inequality, 
we see that the integral 
\[
\int_{(X,\omega_\epsilon)} \eta 
\]
is    bounded by constants independent  of $\epsilon$.   
We also recall that   $\int_{(X,\omega_\epsilon)} \rho_\epsilon  =0$.  
Hence, by Lemma \ref{lemma:convergence-L1}, there is a constant $I$ independent of $\epsilon$ and $j$, such that 
\[
|\int_{(Y, \omega_j)} (\eta +\rho_\epsilon)   |\leq I
\]
for all $j$ sufficiently large.  
%where $dV_j$ is the measure induced by $\omega_j$.   
We also observe from \eqref{eqn:omegaj}, that the volume of $(Y,\omega_j)$ is bounded from below by   $[(\rho\circ\pi)^*\omega_Z]^n >0$ and from above by $( [(\rho\circ\pi)^*\omega_Z] + \rho^*[\omega_\orb] + [\theta_Y])^n$.  
Hence there is some constant $B>1$, independent of $\epsilon$ and $j$, such that the volume of $(Y,\omega_j)$ is contained in $[B^{-1},B]$. 

 

Since $\rho_\epsilon(x)+\eta$ goes to $-\infty$  when $y\in Y $ approaches to $D$,  
the set $$K_\epsilon:=\{y\in Y \ |\ \rho_\epsilon+\eta \geq -IB -1\}$$ is a compact subset of  $Y\setminus D $.    
%Without losing of generality, we assume the boundary of $K_\epsilon$ is smooth{\color{red} smoothness not needed?}.  
Let $\Delta_j$ be the Laplace-Beltrami operator with respect to $\omega_j$.  
Then on $K_\epsilon$, we have the following smooth convergence 
\begin{equation*}\label{eqn:laplace-converge}
    \Delta_j (\rho_\epsilon+\eta )\rightarrow \Delta_\epsilon (\rho_\epsilon+\eta).
\end{equation*} 
From  \eqref{eqn:mean2'}, the RHS above is bounded from below by some constant independent of $\epsilon$. 
Hence, there is some constant $a$ independent of $\epsilon$ and $j$, such that for all   $j$ sufficiently large, 
we have the following inequality on $K_\epsilon$, 
$$\Delta_j (\rho_\epsilon+\eta)\geq a.$$
By Lemma \ref{lemma:unisob}, for $j$ sufficiently large, we have $\omega_{j}\in \mathcal{W}(Y, \theta_Y, n, p, A, K, \gamma)  $ for some $K,A,p,\gamma$ independent of $\epsilon$ and $j$.    
Hence, by Lemma \ref{lemma:meanvalue}, there is a constant $C_2$ independent of $\epsilon$ and $j$ such that 
$$\rho_\epsilon+\eta\leq C_2 \left(1+\int_{(Y,\omega_j)}(|\rho_\epsilon|+|\eta|) \right)$$ 
for all $j$ sufficiently large.  
We recall that $\omega_j^n = e^{Q_j+c_j}\cdot \theta_Y^n$ and the sequence of numbers $\{c_j\}$ converges to $\log Vol(X,\omega_\epsilon)$. 
Thus, by using \eqref{app} of Lemma \ref{lemma:goodapp} and H\"older inequality, 
we can obtain a uniform upper bound on  $\| \eta \|_{L^1(Y,\omega_j)}$, for all $j$ sufficiently large.   
It follows that  
$$\rho_\epsilon+\eta \leq C_3  \left(1+\int_{(Y,\omega_j)}|\rho_\epsilon|   \right), $$ 
for some constant $C_3>0$ independent of $\epsilon$ and $j$. 

Since $\rho_\epsilon$ is bounded, by Lemma \ref{lemma:convergence-L1}, 
the integral in the RHS above converges to $\int_{(X,\omega_\epsilon)}|\rho_\epsilon| $ when $j$ tends to $+\infty$.  
%we need to modify $V_j$ to $V_\epsilon$ in the previous inequality.  If $U$ is any neighborhood of $D$, then we have the following convergence  $$\int_{Y\setminus U}|\rho_\epsilon|dV_j\rightarrow \int_{Y\setminus U}|\rho_\epsilon|dV_{\epsilon}.$$   Moreover, by \eqref{app} in Lemma \ref{lemma:goodapp}  and H\"older inequality,  we have  \[\int_U|\rho_\epsilon|dV_j \leq\sup|\rho_\epsilon| \cdot \int_UdV_j\leq C \cdot K'\cdot \sup|\rho_\epsilon| \cdot (\int_U\theta^n_Y)^{\tau^*}. \]  Here $\tau>1$ is some real number independent of $j$,  and  $\tau^{-1}+{\tau^*}^{-1}=1$. Since $\int_U\theta^n_Y $ tends to zero when we shrink $U$ towards $D$,  and since $\rho_\epsilon$ is orbifold smooth and hence bounded, we conclude that   $$\int_Y|\rho_\epsilon|dV_j\rightarrow \int_Y|\rho_\epsilon|dV_{\epsilon}.$$  
Therefore, we obtain that 
\[
\rho_\epsilon + \eta \le C_3 \left(   1 + \int_{(X,\omega_\epsilon)} |\rho_\epsilon|  \right). 
\]

It remains to compare $\eta$ with $\log |s_D|_{h_D}^2$.  
Since the support of $\pi^*D_X$ is contained in the one of $D$, 
and since $\gamma$ is continuous, 
we see that 
\[
A\cdot \log |s_D|_{h_D}^2 \leq   \pi^*\log |\sigma|_\gamma^2 + A'
\]
for some constants $A' \gg A >0$ sufficiently large.  
Hence there is a constant $B', C'>0$ such that $ C'\log |s_D|_{h_D}^2 \le \eta +B'$. 
This completes the proof of  \eqref{eqn:rho+}.  
%$$(\rho_\epsilon+\log|s_D|)\leq C(1+\int_X|\rho_\epsilon|dV_{\epsilon}).$$  

By replacing $\theta_\mathcal{L}$ and $\rho_\epsilon$ by $-\theta_\mathcal{L}$ and $-\rho_\epsilon$ respectively in the previous reasoning, we see that \eqref{eqn:rho+} still holds if we replace $\rho_\epsilon$ by $-\rho_\epsilon$, up to adjusting the constants $C,C'$.  
This completes the proof of the lemma. 
\end{proof}
\begin{remark} We remark that, we do not use the Heat kernel estimates for orbifold metrics directly when deriving the mean value inequality. Since the function $\log|s_D|^2_{h_D}+\rho_\epsilon$ has some log poles, it is not very clear if we can use its Laplacian and the heat kernel to represent this function.
\end{remark}


%%%%%%%%%%%%%%%%%%%%%%%%%%%%%%%%%%% 



%From  the lower bound of the curvature of the initial metric (see the item $(3)$ of Lemma \ref{lemma:existence:initial-metric-lowerbounded}, we deduce that
%\begin{align*}
% \theta_\mathcal{L}\geq -C p^*\omega_{Z}\geq  -C\omega_\epsilon, \  
% \theta_\mathcal{L} \geq  - C (\omega_\epsilon + \ddbar \eta)
%\end{align*}
%By taking the trace with respect to the metric $\omega_\epsilon$, we deduce that
%\begin{align*}
%\Lambda_\epsilon\theta_\mathcal{L}\geq -C,  \  
%\Lambda_\epsilon\theta_\mathcal{L}\geq -C  - \Delta_\epsilon \eta
%\end{align*}
%It further follows from \eqref{eqn:a} that
%\begin{equation}\label{aaaa}
%g:=\Delta_\epsilon\left(- \rho_\epsilon\right)\geq -C,   \ 
%g:=\Delta_\epsilon\left(- \rho_\epsilon\right)\geq -C -\Delta_\epsilon \eta,  
%\end{equation}
%since the RHS of \eqref{eqn:a} takes values inside a bounded set when $\epsilon$ varies inside $[0,1]$. 
%Note that the function $g$ is orbifold smooth. 
%Thus we can use the heat kernel to represent $\rho_\epsilon$ as follows, 
%$$\rho_\epsilon(x)=\rho_\epsilon(x,t)=\int_X \rho_\epsilon(y) H_\epsilon(x,y,t)dV(y)+\int_0^t\int_X g \cdot H_\epsilon(x,y,t-s)dV(y)ds, $$  
%Let $\overline{\rho}_\epsilon =   \rho_\epsilon - \eta$ and $\overline{g}_\epsilon = \Delta_\epsilon (\overline{\rho}_\epsilon) \geq -C$.  
%{\color{red} We can represent $\overline{\rho}_\epsilon$ by Heat kernel as follows, }
%$$\overline{\rho}_\epsilon(x)=\overline{\rho}_\epsilon(x,t)=\int_X \rho_\epsilon(y) H_\epsilon(x,y,t)dV(y)+\int_0^t\int_X \overline{g} \cdot H_\epsilon(x,y,t-s)dV(y)ds, $$ 
%where $t>0$ is any real number.  We remark that the RHS above might be different from $\rho_\epsilon$ a priori.  However, they are all orbifold smooth solutions to the inhomogeneous  heat equation with the same initial data.   Hence by maximum principle, they are equal.  

%We set $t=1$ in this expression.  By the uniform heat kernel estimate for $\omega_\epsilon$ in Theorem \ref{mainthm:orbi}, and by using  \eqref{aaaa}, we deduce that  %\[\rho_\epsilon\geq  -C \int_{X}|\rho_\epsilon|dV_\epsilon - C.\] 
%\[\rho_\epsilon - \eta \geq  -C \int_{X}|\rho_\epsilon|dV_\epsilon - C.\] 
%Here we used that $\eta$ is integrable on $(X,\omega_\epsilon)$, uniformly in $\eta$. 
%This proves the first inequality.  


%Next we focus on the second inequality. 
%we deduce the reversed mean value inequality for $\rho_\epsilon$ with barrier. 
%The difficulty is that the form $\theta_\mathcal{L}$ is not necessarily bounded from above by the inverse image of $ p^*\omega_Z$.   
%We proceeds as follows. 



%\noindent Next we obtain an inequality similar to the one in Lemma \ref{mv1} for the function $\log \Tr(H_t)$. The method will be the same: derive a lower bound for 
%$\Delta_{t}''\log \Tr(H_t)$ and use the corollary in the previous section. 
%\medskip

%\noindent Recall the following identities
%5\begin{equation}\label{rev40}
%\Delta''_t\log\big(\Tr (H_t)\big)= \frac{\Delta''_t\big(\Tr (H_t)\big)}{\Tr (H_t)}- \frac{|\partial \Tr (H_t)|^2}{\big(\Tr (H_t)\big)^2}
%\end{equation}
%%and 
%\begin{equation}\label{rev41}
%\Delta''\big(\Tr (H_t)\big)= \Tr \big(\Lambda_\epsilon\Theta(E, h_{E})\circ %H_t\big)+ \Tr \Lambda_\epsilon\big((D'_tH_t)\circ H_t^{-1}\circ (\bar D H_t)\big)+ C
%\end{equation}
 %cf. \cite{Siu87}, pages 12-14, where the constant $C$ in \eqref{rev41} is given by the HE condition. As shown in the \emph{loc. cit}, we have 
 %\begin{equation}\label{rev42}
%\Tr \Lambda_\epsilon\big((D'_tH_t)\circ H_t^{-1}\circ (\bar\partial H_t)\big)\geq \frac{|\partial \Tr (H_t)|^2}{\Tr (H_t)}
%\end{equation}
%at each point of $X$ from which we obtain the relation
%\begin{equation}\label{new25}
%\Delta_{t}''\left(\log \Tr(H_t)\right)\geq \frac{\Tr\big(\Lambda_\epsilon \Theta(E, h_{E})\circ H_t\big)}{\Tr(H_t)}+ C.
%\end{equation}
%%On the other hand, by using \eqref{ross2} again we infer that we have
%\begin{equation}\label{new33}
%\frac{\Tr\big(\Lambda_t \Theta(E, h_{E})\circ H_t\big)}{\Tr(H_t)} \geq -C
%\end{equation}
%pointwise on $X$.
%Using the curvature bound  \eqref{lowerboundofCur} of reference metric $h_E$  and standard computation for Hermitian-Einstein metric, we have 

We also have the following estimates. 

\begin{lemma}\label{lemma:boundH}
There are  constants $C,C'>0$ such that the inequality 
\begin{equation*} C'\log |s_D|^2_{h_D} + 
\log \Tr H_\epsilon\leq C\big(1+ \int_{(X,\omega_\epsilon)}\log \Tr H_\epsilon  \big),
\end{equation*}
holds for every $\epsilon\in (0, 1]$.
\end{lemma}
 

\begin{proof} By \cite[Formula (1.9.2)]{Siu87} or \cite[Lemma 3.1]{Simpson1988}, 
we have 
\begin{equation*}%\label{eqn:logtrace-laplace-1}
\Delta_{\epsilon}'\left(\log \Tr H_\epsilon\right)\geq - \| \Lambda_\epsilon \Theta_h \|_h - \|  \Lambda_\epsilon \Theta_{h_{\epsilon,HE}}\|_{h_{\epsilon,HE}}, 
\end{equation*} 
where  $\Theta$ stands for the Chern curvature tensor.  
Then Einstein condition implies that $\|\Lambda_\epsilon \Theta_{h_{\epsilon,HE}}\|_{h_{\epsilon,HE}}$ is  bounded by a constant  independent of $\epsilon$.  
The remainder of the proof is similar to the one of Lemma \ref{lemma:boundrho}. 
We will just mention several main step here. 
As   in  the proof of Lemma \ref{lemma:boundrho}, there is some function $\psi:=\delta \cdot \log |\sigma|_\gamma^2$,  which has log poles along $D_X$, such that $\rho^*\omega_Z + \ddbar \psi$ is an orbifold K\"ahler form.  
Since $\Theta_h$ is orbifold smooth,  there is some  constant $A>0$, independent of $\epsilon$,  such that   
\begin{equation*}
-  A(\omega_\epsilon+     \ddbar  \psi) \cdot \mathrm{Id} \leq  \Theta_h \leq A(\omega_\epsilon+   \ddbar \psi ) \cdot \mathrm{Id},   
\end{equation*} 
where the inequalities $\leq$ are considered in the sense of Nakano positivity. 
%$h$-self-adjoint  endomorphisms. 
It follows that, 
\begin{equation*}
-  A(1 +  \Delta_\epsilon' \psi ) \cdot \mathrm{Id} \leq \Lambda_\epsilon \Theta_h \leq A(1 +  \Delta_\epsilon'   \psi ) \cdot \mathrm{Id}, 
\end{equation*}
where the inequalities $\leq$ are considered for $h$-self-adjoint  endomorphisms.  
Since $ \Lambda_\epsilon \Theta_h$ is self-adjoint with respect to $h$,  
we deduce that, if   $\eta:=  \rk (\mathcal{E})^{\frac{1}{2}} \cdot  A\cdot \psi$, then   
\[
\|\Lambda_\epsilon \Theta_h\|_h \leq  \rk (\mathcal{E})^{\frac{1}{2}} \cdot A +\Delta_\epsilon' \eta. 
\]
Hence we get 
\begin{equation*}
\Delta_{\epsilon}\left(\eta + \log \Tr H_\epsilon \right)\geq A'
\end{equation*} 
for some constant $A'$. 
Arguing as in Lemma \ref{lemma:boundrho}, 
where we consider $\log \Tr H_\epsilon $ in the place of $\rho_\epsilon$, 
we deduce that 
\[\eta + \log \Tr H_\epsilon \leq B(1+ \int_{(X,\omega_X)} \log \Tr H_\epsilon ) \] for some constant $B>0$. 
By comparing $\eta$ with $\log |s_D|_{h_D}^2$, 
we can obtain  the inequality of the lemma.  
\end{proof}



%\begin{remark} The main point in Lemma \ref{lemma:meanvalue} is that the constant $C$ is \emph{independent of $\epsilon$}.  \end{remark}
%%%%%%%%%%%%%%%%%%%%%%%%%%%%%%%%%%%%%%%%%%%%%%%%%%%%%%%%%%%%%%%%%%%%%%%%%%%%%%%%%%%%%%%%%%%%%%%%%%%%%%%%%%%





 
\subsection{$\mathcal{C}^0$ estimate of $h_{\epsilon,HE}$ with barrier}
The main purpose of this subsection is to prove the following $\mathcal{C}^0$ estimate of $H_\epsilon$ and $\rho_\epsilon$, from which Theorem \ref{thm:BG-inequality-intro} follows directly. 
Here additional care should be paid to orbifold singularities, which cause no serious trouble after the preparations of previous discussions.
 %, especially the uniform mean value type inequalities.
Recall that  $H_\epsilon \in End_h(\mathcal{E})$ defines a Hermitian-Einstein metric  with respect to $\omega_\epsilon$ by  $h_{\epsilon,HE}=hH_\epsilon$.    
We set $\eta_\epsilon:= \log H_\epsilon$ and recall that 
\[ \rho_\epsilon = -\Tr \eta_\epsilon,  \ \ \ \ \  
\eta_\epsilon =  -\frac{1}{\mathrm{rk}(\mathcal{E})}\rho_\epsilon \otimes \ID+  s_\epsilon,  \ \ \ \ \  S_\epsilon = \exp(s_\epsilon).   
\]
%By Lemma \ref{lemma:etabound},  the following  equality holds for all $\epsilon>0$,   \begin{equation}\label{alt5}
%0 =  \int_{(X,\omega_\epsilon)}\left\langle \Phi(\eta_\epsilon)(\partial\eta_\epsilon), \partial\eta_\epsilon\right\rangle_\epsilon  + \int_{(X,\omega_\epsilon)} \Tr \big(\eta_\epsilon \Lambda_{\epsilon}\Theta_h\big), 
%\end{equation} 
%where  $\Phi(x, y) = \frac{\exp(y-x)- 1}{y-x}$,  $\langle \cdot , \cdot \rangle_\epsilon$ is the inner product induced by $\omega_\epsilon$,  and $\Theta_h$ is the Chern curvature tensor of $(\mathcal{E},h)$.  













 
\begin{prop}\label{prop:C0} There exists  constant $C,C'> 0$, independent of $\epsilon, $ such that the following inequalities  hold
\begin{equation}\label{est1}
\Tr  H_\epsilon \leq C -C' \log |s_D|_{h_D}^2, \qquad
| \rho_\epsilon| \leq  C -C' \log |s_D|_{h_D}^2. 
% \  \int_{X}|\partial s_\epsilon|^2\frac{dV_{t}}{\log\frac{1}{|s_D|^2}}\leq C.
\end{equation}
%hold, where $s_D$ is the product of sections defining the exceptional divisor of the map $p\circ\pi: Y\to Z$ and $|s_D|^2:=|\sigma_D|^2_{h_D}$, where $h_D$ is a smooth metric associated to the line bundle $D$.
\end{prop}
\medskip



\begin{proof}
The key is to prove Lemma \ref{lemma:integral-bound} below. 
Admitting this lemma for the time being.   
Since $S_\epsilon=\exp(s_\epsilon)$ and $\Tr s_\epsilon=0$, we see that $\Tr S_\epsilon \ge 1$. Hence the  second  inequality follows from Lemma \ref{lemma:boundrho}.  
Since $\log \Tr H_\epsilon = \log \Tr S_{\epsilon} - \rho_\epsilon$, 
we can obtain the first inequality by combing  \eqref{eqn:case1-assumption} with Lemma \ref{lemma:boundH}.    
\end{proof}


\begin{lemma}
\label{lemma:integral-bound}  
There exists a constant $C> 0$ independent of $\epsilon$,  such that 
\begin{equation}\label{eqn:case1-assumption}
    \int_{(X,\omega_\epsilon)}\big(|\rho_\epsilon|+ \log {\Tr S_\epsilon}\big)   \leq C
\end{equation}
for all positive $\epsilon$. 
\end{lemma}


%Now we proceed to prove Proposition \ref{prop:C0}. The idea is to use blow-up analysis to conclude the proof \cite{UhlenbeckYau1986,Simpson1988}, where, in the end, the contradiction is from the stability assumption.  We discuss two different cases. It is helpful to recall the notations in Definition \ref{defn:a}.
 
\begin{proof}
The idea is to adapt the methods of \cite{UhlenbeckYau1986} and \cite{Simpson1988},  by using  blow-up analysis. 
%, where, in the end, the contradiction is from the stability assumption.  
%It is helpful to recall the notations in Definition \ref{defn:a}. 
Assume by contradiction that the lemma does not hold.   
Then there exist sequences $(\delta_i)_{i\geq 1}$ and  $(\epsilon_i)_{i\geq 1}$ of numbers in $(0,1)$ 
converging towards zero such that 
\begin{equation} 
\label{contraseq} 
\int_{(X,\omega_i)}\big(|\delta_i\rho_i|+ \delta_i \log {\Tr S_i}\big)  = 1
\end{equation}
%Here $dV_i$ is the volume element corresponding to the  K\"ahler current $\omega_i:= \omega_{\epsilon_i}$. 
Here, we denote  $\omega_i$,  $\rho_i$, $s_i$, $\eta_i$ and $S_i$ for  $\omega_{\epsilon_i}$ $  \rho_{\epsilon_i}$, $s_{\epsilon_i}$, $\eta_{\epsilon_i}$ and $ S_{\epsilon_i}$  respectively.   
Let 
\[u_i:= \delta_i\eta_i = -\frac{\delta_i}{\mathrm{rk}(\mathcal{E})}\rho_i\otimes \ID_E+ \delta_is_i.\] 
We will show that, up to passing to a subsequence, $u_i$ converges to some limit $u_\infty$, which produces a destabilizing subsheaf of $\mathcal{F}$. 
This will contradict  the stability assumption on $\mathcal{F}$.  


In the following reasoning, the capital letters $C$ and $C'$ denote positive real numbers, which may change from line to line. Nevertheless, they are always independent of $i$.   
Since $\det(S_i)=1$, we have $\Tr(S_i)\geq \mathrm{rk}(\mathcal{E})$. 
In particular, $\log\Tr S_i \geq 0$. 
By \eqref{contraseq},  we have 
\begin{equation*}%\label{tw}
\int_{(X,\omega_i)}|\delta_i\rho_i| \leq 1, \, \int_{(X,\omega_i)}\delta_i \log {\Tr S_i}   \leq 1. \end{equation*}
%for some constant $C$ independent of $i$.
Then by Lemma \ref{lemma:boundrho}, we deduce that 
\begin{equation}\label{ss}
|\delta_i\rho_i|\leq C-C'\cdot \delta_i\log|s_D|_{h_D}^2, 
\end{equation}
for some constants $C$ and $C'$ independent of $i$.
Recall that \begin{equation}\label{qq}
\log \Tr  H_i = \log \Tr  S_i - \rho_i,
\end{equation} 
so by \eqref{contraseq}, we get  
\[\int_{(X,\omega_i)}\delta_i |\log {\Tr H_i}|   \leq 1.\]
Then by Lemma \ref{lemma:boundH}, we deduce that
\begin{equation}\label{tw1}
 |\delta_i\log  \Tr H_i |\leq C -C'\cdot \delta_i\log|s_D|_{h_D}^2.  
\end{equation}
Combining  \eqref{ss}, \eqref{qq} and \eqref{tw1}, we obtain that   
\[
\delta_i\log\Tr S_i \leq C- C'\cdot \delta_i\log|s_D|_{h_D}^2.
\]
For a  point $x\in X$, if the largest eigenvalue of $ s_i (x)$ is $\lambda_{i,max}$, then $\lambda_{i,max} \ge 0$ for $\Tr s_i =0$. 
Moreover, since $S_i=\exp(s_i)$, we see that  
\[
\delta_i \cdot \lambda_{i,max} \leq \delta_i \cdot \log \Tr S_i. 
\]
By using $\Tr s_i=0$ again, we have $\lambda_{i,max}^2\ge \frac{1}{\rk (\mathcal{E})^3} \|s_i\|_h^2$. 
Since 
\[
\|u_{i}\|_h\leq  \rk(\mathcal{E})^{-\frac{1}{2}} |\delta_i\rho_i| + \delta_i \| s_i \|_h,
\] 
it follows that  
\begin{equation}\label{ubarrier}
\|u_{i}\|_h\leq  C -C'\cdot \delta_i\log|s_D|_{h_D}^2.
\end{equation} 

\medskip

The important step towards the contradiction we are looking for is the 
following result.
\begin{claim}\label{claim:limit} There exist a subsequence of $(u_i)_{i\geq 1}$ converging weakly to a limit
$u_\infty$ on compact subsets of $X\setminus \pi(D)$ such that the following hold. 
Let $\omega_\infty:= \rho^*\omega_Z$. 
\begin{enumerate} 


\item  The endomorphism $u_\infty$ is non zero and it belongs to the space $L_{1}^2(X,\omega_\infty)$.  
In other words, both $u_\infty$ and $\dbar u_\infty$   are in $L^2(X,\omega_\infty)$.  
 
\item  Let $\Psi:\mathbb R\times \mathbb R\to \mathbb R_{> 0}$ be a smooth, positive function such that $\displaystyle \Psi(a, b)< \frac{1}{b-a}$ holds for any
$a< b$. Then we have 
$$0 \geq  
\int_{(X,\omega_\infty)}\left\langle \Psi(u_\infty)(\partial u_\infty), \partial u_\infty\right\rangle  + \int_{(X,\omega_\infty)} \Tr \big(u_\infty \Lambda_{\infty}\Theta_h\big) $$
where $\Lambda_\infty$ %and $dV_0$ are 
is the contraction with %and the volume element corresponding to 
$\omega_\infty$, 
and $\langle \cdot , \cdot \rangle_\infty$ is the inner product induced by $\omega_\infty$. 
%, respectively. 
\end{enumerate}
\end{claim} 


Admitting the claim for the time being, we will argue as in \cite[Section 5]{Simpson1988}.   
%By \eqref{ubarrier}, $u_\infty$ obtained is indeed bounded, so the item$(2)$ of the claim implies that $u_\infty\in W^{1,2}(X)$ with respect to  $\omega_Z$.  
We remark that, in \cite{Simpson1988}, the functions $\Psi$ are  assumed to be bounded by $\frac{1}{a-b}$ when $a> b$, which are slightly different from our setting.  
However, the same argument remains valid.  
More precisely, in our situation, we  replace $\Phi(\lambda_1, \lambda_2)$ of \cite[Lemma 5.5 and Lemma 5.6]{Simpson1988} by $\Phi(\lambda_2,\lambda_1)$. 
Afterwards, we replace $\Phi_\gamma(y_1,y_2)$ of \cite[Lemma 5.7]{Simpson1988} by 
$\Phi_\gamma(y_1,y_2) = (1-p_\gamma(y_1))\cdot dp(y_1,y_2)$.  
Now, the item (2) of the claim implies that the eigenvalues of $u_{\infty}$ are constant almost everywhere on $X$, by the arguments of \cite[Lemma 5.4 and Lemma 5.5]{Simpson1988}. 
They are not all equal, since by the second inequality of  \eqref{est1}, 
we have $\Tr u_\infty =0$. 
By the same argument as \cite[Lemma 5.7]{Simpson1988},  
we can construct a saturated destabilizing subsheaf of $\mathcal{F}|_{Z^\circ}$ by using $u_\infty$, 
where $Z^\circ\subseteq Z$ is a smooth open subset whose complement has codimension at least 2, such that $\mathcal{F}|_{Z^\circ}$ is locally free.   
Such a destabilizing subsheaf extends to a coherent subsheaf of $\mathcal{F}$ by Lemma \ref{lemma:extension-subsheaf}.  
This contradicts the stability assumption on $\mathcal{F}$, 
and finishes the proof of  Proposition \ref{prop:C0}.
\end{proof}

It remains to prove the previous claim. 

\begin{proof}[Proof of Claim \ref{claim:limit}]  
The proof is quite long, and we will divide it into several steps. \\

\textit{Step 1.} We will first prove some uniform integrability. 
%From \eqref{ubarrier}, we see that, given $\delta> 0$, there exists   a neighborhood  $U_\delta$ of  $\pi(D)$ such that  \begin{equation}\label{eqn:uniform-int} \int_{(U_\delta,\omega_\orb)}\|u_i\|_h \leq \delta \end{equation}  for any   $i$. 
Since $2\omega_\orb \geq \omega_i$ by  our choice of $\omega_\orb$,  
from \eqref{ubarrier}, 
we deduce that 
\begin{equation}\label{eqn:uniform-int}
\|u_i\|_h \omega_i^n  \leq B_1  \cdot \omega_\orb^n    
\end{equation} 
for  all $i$, 
where $B_1$ is a positive function which only has log poles along $D$, and is smooth elsewhere.   
Next, since $h$ is orbifold smooth, there is some constant $A>0$ such that   
$$-A \cdot \omega_{\orb}\cdot \ID\leq \Theta_h\leq A\cdot  \omega_{\orb}\cdot \ID.$$
Hence there is some constant $A'$ such that 
$$-A'\cdot   {\omega_{\orb}\wedge\omega_i^{n-1}} \cdot \ID  
\leq\Lambda_i\Theta_h \cdot \omega_i^n  
\leq  A'\cdot {\omega_{\orb}\wedge\omega_i^{n-1}}  \cdot \ID.$$
Since $2\omega_{\orb} \geq \omega_i$, we get 
\[
\|\Lambda_i\Theta_h \|_h \cdot \omega_i^n\leq A'' \cdot  \omega_{\orb}^n 
\]
for some constant $A''>0$. 
Together with \eqref{ubarrier}, this implies   that 
\begin{equation}
\label{eqn:uni-int-2}    
\|u_i\|_h \cdot \|\Lambda_i\Theta_h\|_h \cdot  \omega_i^n  \leq  B_2 \cdot  \omega_\orb^n
\end{equation}
for all $i$, 
where $B_2$ is a positive function which only has log poles along $D$, and is smooth elsewhere.    

%are  uniformly integrable on $X$. 
%It follows that 


Let $ \Phi(x, y) = \frac{\exp(x-y)- 1}{x-y}$. 
By applying Lemma \ref{lemma:etabound} to $\eta_i=\delta_i^{-1} u_i$, we have  
\begin{equation}\label{hh}
\frac{1}{\delta_i}\int_{(X,\omega_i)}\left\langle \Phi(u_i/\delta_i)(\partial u_i), \partial u_i\right\rangle_i  
+ \int_{(X,\omega_i)} \Tr \big(u_i\Lambda_{i}\Theta_h\big) = 0,   
\end{equation}   
where $\langle \cdot , \cdot \rangle_i$ is the inner product induced by $h$ and $\omega_i$. 
Using  \eqref{eqn:uni-int-2}, we deduce that the following integrals 
\[
\frac{1}{\delta_i}\int_{(X,\omega_i)}\left\langle \Phi(u_i/\delta_i)(\partial u_i), \partial u_i\right\rangle_i  
\]
are uniformly bounded. \\









\textit{Step 2.}  Assume that $K$ is a relatively compact open subset of $X\setminus \pi(D)$. 
In this step, we will  prove a uniform estimate of the $L^2$-norms of $\partial u_i$ on $K$, 
which will imply the convergence of $u_i$,  up to passing to a subsequence.  
By \eqref{ubarrier},  the eigenvalues of $u_i$ on $K$ are contained some segment $[\alpha,\beta]$ independent of $i$. Hence, by Lemma \ref{lemma:Phi-monotone},  we have
\begin{equation*}
\int_{(K,\omega_i)}\|\partial u_i\|_i^2 
\leq C_K\int_{(K,\omega_i)}\left\langle \Phi(u_i)(\partial u_i), \partial u_i\right\rangle_i 
\end{equation*}
for some constant  $C_K$ depending only on $\alpha,\beta$.     
Here the norm $\|\partial u_i\|$ is induced by $h$ and $\omega_i$.  
Since $\delta_i\le 1$, by Lemma \ref{lemma:Phi-monotone}, we deduce that 
\begin{equation*}
\int_{(K,\omega_i)}\|\partial u_i\|_i^2 
\le 
C_K \cdot \frac{1}{\delta_i}\int_{(X,\omega_i)}\left\langle \Phi(u_i/\delta_i)(\partial u_i), \partial u_i\right\rangle_i.     
\end{equation*} 
%It then follows from \eqref{hh} that 
%\begin{equation*}
%\int_{(K,\omega_i)}|\partial u_i|^2 
%\le 
%C_K \cdot  \int_{(X,\omega_i)} \Tr \big(u_i\Lambda_{i}\Theta(E, h_E)\big).     
%\end{equation*}
From Step 1, we know that the RHS above is bounded from above, uniformly in $i$. 
Hence $\int_{(K,\omega_i)}\|\partial u_i\|_i^2 $ is uniformly bounded.  
Since $\omega_\infty$ is a K\"ahler form in a neighborhood of  $\overline{K}$, this implies that  $\int_{(K,\omega_\infty)}\|\partial u_i\|^2 $ is uniformly bounded, where the norm $\|\partial u_i\|$ is induced by $h$ and $\omega_\infty$.  

%For the same reason, and since $\omega_i$ converges smoothly to $\omega_\infty$ on $K$,  up to enlarging $C_K$, we have \begin{equation*}\label{ggg}\int_{(K,\omega_i)}|\Tr \big(u_i\Lambda_{i}\Theta(E, h_E)\big)| \leq C_K   \end{equation*} for all $i $ sufficiently large.    By \eqref{eqn:uniform-int}, we may assume that $K$ is sufficiently large so that $$\int_{(X\setminus \overline{K},\omega_i)}|\Tr \big(u_i\Lambda_{i}\Theta(E, h_E)\big)| \leq 1.$$ Hence for any $i$ sufficiently large,  we have  \begin{equation*}  \int_{(K,\omega_i)}|\partial u_i|^2 \leq C_K + 1.  \end{equation*}

We have proved in \eqref{ubarrier} that the functions $\|u_i\|_h$ are uniformly bounded on $K$. 
By a standard diagonal procedure, up to passing to a subsequence,  we can  assume that the sequence $\{u_i\}$  converges weakly to an endomorphism 
$u_\infty$,  inside $L_1^{2}(X,\omega_\infty)$, on any relatively compact open subsets  of $X\setminus \pi(D)$.  Moreover, by  Rellich–Kondrachov theorem, we have the strong $L^2$ convergence
\[
\Vert u_i-u_{\infty}\Vert_{L^2(K,\omega_\infty)}\to 0  \mbox{ when } i\to +\infty.
\]
for any relatively compact open subset $K\subset X\setminus D$.   



By dominated convergence theorem, \eqref{eqn:uni-int-2} implies the following convergence, 
\begin{equation}\label{eqn:converge-uLambda} 
\int_{(X,\omega_i)} \Tr \big(u_i\Lambda_{i}\Theta_h \big) 
\to 
\int_{(X,\omega_\infty)} \Tr \big(u_\infty\Lambda_{\infty}\Theta_h\big)  
\mbox{ when }  i \to +\infty. 
\end{equation} 
\\


\textit{Step 3.} In this step, we will  show that the limit $u_{\infty}$ is not identically zero. 
By the assumption of \eqref{contraseq}, we have 
\begin{equation*}
\int_{(X,\omega_i)}\delta_i |\rho_i| + \int_{(X,\omega_i)}\delta_i \log {\Tr S_i}  =  1.  
\end{equation*} 
Since $S_i=\exp(s_i)$, we have 
\[
\delta_i\log{\Tr S_i }\leq \delta_i \|s_i\|_h + \delta_i \log \rk(\mathcal{E}). 
\]
From the definition of $u_i$, we  we get
\begin{equation*}
\delta_i\log{\Tr S_i}\leq   \|u_i\|_h+ \rk(\mathcal{E})^{-\frac{1}{2}}\cdot \delta_i |\rho_i| + \delta_i \log \rk(\mathcal{E}).  
\end{equation*}
Combine with the first equality in Step 3, we deduce that 
\begin{equation}\label{10}
(1+\rk(\mathcal{E})^{-\frac{1}{2}})\int_{(X,\omega_i)}\delta_i |\rho_i| + 
\int_{(X,\omega_i)}\|u_i\|_h   \geq 1-\delta_i \cdot  Vol(X,\omega_i) \cdot \log \mathrm{rk}(\mathcal{E}),  
\end{equation} 
where $Vol$ is the volume. 
Since $\delta_i\rho_i = -\Tr u_i$, 
we deduce  the following convergence, almost everywhere on $X$,  
\[
\delta_i \rho_i \to \rho_\infty := -\Tr u_\infty \mbox{ when } i\to +\infty. 
\]
We tend $i$ to the infinity in  \eqref{10}. 
Recall that $\omega_i \le 2\omega_\orb$.   
Thanks to   \eqref{ss} and   \eqref{eqn:uniform-int}, by dominated convergence theorem, 
we can interchange limit symbol and integral symbol for the LHS of   \eqref{10}. 
It follows that $0 \geq 1$. 
This is a contradiction. 
\\


\textit{Step 4.} We will prove the item $(2)$ in this step. 
%This is an immediate consequence of the mild growth of $u_i$ \eqref{ubarrier}. 
%\begin{equation}\label{ddddd}
%\int_{(U_\delta,\omega_i)}|u_i|_h \cdot |\Lambda_i\Theta(E,h_E)|_h  \leq B\delta,
%\end{equation} 
%for some constant $B>0$.
Fix a function $\Psi$ as in the statement of the claim.  
We will show that for each relatively compact open subset $K\subset X\setminus \pi(D)$, 
the following inequality hold for all $i$ sufficiently large. 
\begin{equation}\label{fffffff}
\int_{(K,\omega_i)}\left\langle \Psi(u_i)(\partial u_i), \partial u_i\right\rangle_i  + \int_{(X,\omega_i)} \Tr \big(u_i\Lambda_{i}\Theta_h\big) \leq 0
\end{equation}  
By Lemma \ref{lemma:Phi-monotone}, we note that  
${\delta_i}^{-1}\Phi(\delta_i^{-1}a,\delta_i^{-1}b)$ tends to $\frac{1}{b-a}$ if $b> a$, 
and to $+\infty$ if $ b\le a$.  
As we have seen before that, by   \eqref{ubarrier},   
the eigenvalues of  $u_i$ on $K$ are contained in an bounded segment $[\alpha,\beta]$.  
Hence, for some $i$ sufficiently large, we have 
\[
{\delta_i}^{-1}\Phi(\delta_i^{-1}a,\delta_i^{-1}b) \geq \Psi(a,b)
\]
for any $a,b\in[\alpha,\beta]$.  
We can then deduce \eqref{fffffff}  from \eqref{hh}. 
Thanks to \eqref{eqn:converge-uLambda}, we  obtain that,   
for any $\delta>0$ fixed,   
if $i$ is sufficiently large,  then 
\begin{equation}\label{zzz}
\int_{(K,\omega_i)}\left\langle \Psi(u_i)(\partial u_i), \partial u_i\right\rangle_i + \int_{(X,\omega_\infty)} \Tr \big(u_\infty\Lambda_{\infty}\Theta_h\big) \leq \delta.
\end{equation} 



Since  $u_i\rightarrow u_\infty$ in $L^2_{b}$ on $(K,\omega_\infty)$,  
for some $b$ depending on $K$, 
we can apply the item $(2)$ of Lemma \ref{lemma:simpsonmor}  to show that,   
there is a convergence 
\[
\Psi^{\frac{1}{2}}(u_i)\rightarrow \Psi^{\frac{1}{2}}(u_\infty)    \mbox{ when } i\to \infty, 
\] 
in $\mathcal{C}^0(L^2, L^q)$ for any $q<2$, 
where $\Psi^{\frac{1}{2}}$ is the positive square root of $\Psi$, which is again smooth.   
Hence, from \eqref{zzz}, we  deduce  that, for all $i$ sufficiently large, 
\[
\| \Psi^{\frac{1}{2}}(u_\infty)(\partial u_i)\|^2_{L^q(K,\omega_i)}+ \int_{(X,\omega_\infty)} \Tr \big(u_\infty\Lambda_{\infty}\Theta_h\big) \leq 2\delta.\]
In addition, we have the following weak convergence in $L^q (K,\omega_\infty)$ 
\[
\Psi^{\frac{1}{2}}(u_\infty)(\partial u_i)\rightarrow \Psi^{\frac{1}{2}}(u_\infty)(\partial u_\infty) \mbox{ when }  i\to \infty. 
\]  
By the Hahn-Banach theorem,  the previous inequality implies that 
\[
\| \Psi^{\frac{1}{2}}(u_\infty)(\partial u_\infty)\|^2_{L^q(K,\omega_\infty)}+ \int_{(X,\omega_\infty)} \Tr \big(u_\infty\Lambda_{\infty}\Theta_h\big) \leq 2\delta.
\]


This inequality holds for any $\delta>0$ and any $q<2$. 
If a measurable function satisfies an 
$L^q$ norm inequality which is uniform for $q<2$ then it satisfies the inequality 
for $q=2$.  
Note that $K$ can be arbitrarily large in $X\setminus \pi (D)$. 
Hence we obtain the item $(2)$ of the claim.
This  completes the proof of the claim. 
%\ref{lemma:limit} and hence also the proof of Proposition \ref{prop:C0}.
\end{proof} 
 

% Lemma \ref{lemma:limit} implies that: the eigenvalues of $u_\infty$ are constant a.e. on $X$, and they are not all equal (here the trace of $u_\infty$ is not zero, but this is the case for the average of the trace showing that at least two of the eigenvalues of this endomorphism must be different). Also some appropriate eigenspace of $u_\infty$ defines a destabilising subsheaf of $E$. We just point out the fact that we are working with  $p^*\omega_Z$ or $\pi^*p^*\omega_Z$ rather than with a genuine K\"ahler metric is not relevant. 



%%%%%%%%%%%%%%%%%%%%%%%%%%%%%%%%%%%%%%%%%%%%%%%%%%%%%%%%%
%%%%%%%%%%%%%%%%%%%%%%%%%%%%%%%%%%%%%%%%%%%%%%%%%%%%%%%%%%%%%%%%%%%%%%%%%%%%%%%%%%%%%%%%%%%%%%%%%%%%%%%%%%%%%%%%%%%%%%%%%%%%%%%%%%%%%%%%%%%%





%Next, we shall use the uniform heat kernel estimate of $\omega_\epsilon$ to obtain uniform estimate of orbifold HYM metrics $H_\epsilon$. This is essentially due to Bando-Siu \cite{}. Firstly, we record a series of differential inequalities for HYM heat flow.

%\begin{equation}\label{HYMHEAT}
%\frac{dH}{dt}H^{-1}=-(\sqrt{-1}\Lambda_\epsilon F-\lambda I),
%\end{equation}

%We know that for each fixed orbifold K\"ahler form $\omega_\epsilon$, the heat flow \eqref{HYMHEAT} converges to an orbifold smooth HYM metric $H_\epsilon$. 

%We know that the curvature tensor satisfy the following inequalities.
%\begin{align*}
%\frac{d\Lambda_\epsilon F}{dt}&=\Delta_\epsilon\Lambda_\epsilon F,\\
%\frac{d|\Lambda_\epsilon F|^2}{dt}&=\Delta_\epsilon|\Lambda_\epsilon F|^2-|\nabla_\epsilon\Lambda_\epsilon F|^2,\\
%\frac{d|\Lambda_\epsilon F|}{dt}&\leq\Delta_\epsilon|\Lambda_\epsilon F|,\\
%\frac{d}{dt}\int|\Lambda_\epsilon F|^2&=-\int|\nabla_\epsilon\Lambda_\epsilon F|^2\\
%\int|\Lambda_\epsilon F|(t,y)&\leq \int|\Lambda_\epsilon F|(0,y)\\
%|\Lambda_\epsilon F|(t,x)&\leq \int H_\epsilon(x,y,t)|\Lambda_\epsilon F|^(0,y)
%\end{align*}


\subsection{Equality condition for Bogomolov-Gieseker inequalities}  

We complete the proof of Theorem \ref{thm:BG-inequality-intro} in this subsection.  

\begin{thm}\label{mainthm:con} 
For the sequence of Hermitian-Einstein metrics  $h_{\epsilon,HE}$, we have
\begin{enumerate}
\item 
$\int_{(X,\omega_\epsilon)}\|\Theta_{h_{\epsilon,HE}}\|_{\omega_\epsilon}^2  \leq C $ for some constant $C$ is independent of $\epsilon$, where $\Theta$ represents the Chern curvature tensor.    
\item There is a Zariski open set $U\subset X_{\sm} \setminus \pi(D)$ whose complement has codimension at least 2, 
there is a sequence $\{\epsilon_i\}$ of positive small enough numbers converging to $0$, 
such that $h_{\epsilon_i,HE}$ converge to a Hermitian-Einstein metric $h_{\infty}$ with respect to $\rho^*\omega_Z$, 
locally and smoothly on $U$.  
Moreover, $\Theta_{h_\infty}$ belongs to $L^2(X,\rho^*\omega_Z)$.  
\item Assume that $\hat{c}_2(\mathcal{F})\cdot [\omega_Z]^{n-2} = \hat{c}_1(\mathcal{F})^2  \cdot [\omega_Z]^{n-2} = 0$, 
then the Hermitian-Einstein metric $H_{\infty}$ defined on $U$   is Hermitian flat. 
\item  Assume the condition of   (3) holds, and that $Z$ has klt singularities.  
Then there is a finite quasi-\'etale cover $p\colon Z'\to Z$, such that the reflexive pullback $(p^*\mathcal{F})^{**}$ is a unitary flat vector bundle. 
\end{enumerate}
\end{thm}

\begin{proof}  
%By the monotonicity of $L^2$ norm of $|\Lambda_\epsilon \Theta|$ (\cite[]{BandoSiu1994}) along Donaldson's heat flow, we have $\int|\Lambda_\epsilon \Theta_{{HE,\epsilon}}|^2\leq C$, where $C$ is independent of both $t>0$ and $\epsilon$. Here for different $\epsilon$, we choose the same initial metric for Donaldson's heat flow. 


We recall the following  identity (see for example the proof of  \cite[Theorem 4.4.7]{Kobayashi2014}), where $c_n$ is a constant depending only on $n$, 
$$\Big(2\hat{c}_2(\mathcal{E}) -\hat{c}_1(\mathcal{E})^2 \Big) \cdot [\omega_\epsilon]^{n-2}=c_n\int (\|\Theta_{h_{\epsilon,HE}}\|_{h_{\epsilon,HE},\omega_\epsilon}^2-\|\Lambda_\epsilon \Theta_{h_{\epsilon,HE}}\|_{h_{\epsilon,HE}}^2)\omega_\epsilon^n.$$
The LHS is bounded by constants independent of $\epsilon$.  
The functions  $\|\Lambda_\epsilon \Theta_{h_{\epsilon,HE}}\|_{h_{\epsilon,HE}}^2$ are constant after the Einstein condition, and they are uniformly bounded as well.  
Hence $\|\Theta_{h_{\epsilon,HE}}\|_{L^2(X,\omega_\epsilon)}$ is uniformly bounded. 
This proves the item $(1)$.

We choose $U\subset X_{\sm} \setminus \pi(D)$ as the maximal Zariski open set over which $\rho^*\mathcal{F}$ is locally free.   
We note that $\rho|_U$ is isomorphic, and $\rho^*\omega_Z$ is a smooth K\"ahler form on $U$.  
Since we have   uniform $\mathcal{C}^0$ estimates for $H_\epsilon$ on any compact subsets of $U$ by Proposition \ref{prop:C0}, 
the convergence in the item $(2)$ is a standard consequence of the elliptic theory on the Hermitian-Einstein equations \eqref{eqn:HE}.   
The $L^2$ property for $\Theta_{h_\infty}$ follows from the item $(1)$ 

%Recall that we have the partial modification map $p: X\rightarrow Z$. 

Now we prove the item $(3)$. By the Hermitian-Einstein condition, we have
\[ \Big(c_2(\cE,h_{\epsilon,HE})-\frac{r-1}{2r}c_1(\cE,h_{\epsilon,HE})^2 \Big)  \wedge  \omega_\epsilon^{n-2} \geq 0. \] 
On the other hand,  by assumption, we have 
$$\Big(2r\hat{c}_2(\mathcal{E}_{\orb}) - (r-1)\hat{c}_1(\mathcal{E}_{\orb})\Big)\cdot [\omega_\epsilon]^{n-2}\rightarrow 0  \mbox{ when } \epsilon\to 0.$$
By  the positivity of the integrands,   
for any precompact open subset $K\subset U$, we have
\[   \int_K \Big(c_2(\cE,h_{\epsilon,HE})-\frac{r-1}{2r}c_1(\cE,h_{\epsilon,HE})^2 \Big)  \wedge  \omega_\epsilon^{n-2} \rightarrow  0  \mbox{ when } \epsilon\to 0. \] 
Hence \[   \int_K \Big(c_2(\cF,h_\infty)-\frac{r-1}{2r}c_1(\cF,h_\infty)^2 \Big)  \wedge  \omega_Z^{n-2} = 0. \] 
%Since $\Theta_{h_{\infty}}$ is $L^2$, we deduce that   $$ \int_Z \Big(c_2(\cF,h_\infty)-\frac{r-1}{2r}c_1(\cF,h_\infty)^2 \Big)  \wedge  \omega_Z^{n-2} = 0.$$ 
This implies that the non negative integrand in the LHS is identically $0$. 
Since $K$ can be arbitrarily large in $U$, we deduce that  $h_{\infty}$ is a Hermitian flat. 

Finally, the item (4) follows from Theorem \ref{thm:klt-cover},  by using the argument of the proof of \cite[Theorem 1.14]{GrebKebekusPeternell2016b}. 
This completes the proof of the theorem. 
 \end{proof}

\begin{proof}[{Proof of Theorem \ref{thm:BG-inequality-intro}}] 
By Theorem \ref{mainthm:con} above, we can deduce that the item $(1)$ implies $(2)$. 
For the converse, we first note that $Z'$ also has klt singularities.  
We denote $\mathcal{F}'=(p^*\mathcal{F})^{**}$. 
Let $\rho'\colon X'\to Z'$ be an orbifold modification as in \cite[Theorem 1.2]{Ou2024}. 
Then $\mathcal{E}':=\rho'^*\mathcal{F}'$ is a unitary flat vector bundle on $X'$. 
If $\widetilde{X}$ is the universal cover of $X'$, then there is a trivial bundle $\widetilde{\mathcal{E}}$ with trivial Hermitian metric $\widetilde{h}$, such that $\mathcal{E}'\cong \widetilde{\mathcal{E}}/\pi_1(X')$ for some appropriate unitary representation of $\pi_1(X')$.  
It follows that the trivial metric $\widetilde{h}$ descend to some smooth flat metric $h'$ on $\mathcal{E'}$. Then $h'$ is orbifold smooth with respect to the standard orbifold structure on $X$. 
This implies that $\hat{c}_1(\mathcal{E}')=0$ and $\hat{c}_2(\mathcal{E}')=0$.   
It is then routine to verify the item $(1)$ of the theorem holds.  
\end{proof}


 
% \section{Appendix}




%\section{extension}





\bibliographystyle{alpha}
\bibliography{reference}

 
\end{document}


 



 
  
\subsection{Resolution of coherent sheaves} 

With the same method of Subsection \ref{subsection:hypersurface}, we can prove the following theorem, which is well-known to specialists. 

\begin{thm}
\label{thm:func-reso-loc-free}
Let $X$ be a complex analytic variety, and let $\mathcal{E}$ be a torsion-free coherent sheaf on $X$. 
Then there is a     projective bimeromorphic morphism $r\colon \widetilde{X} \to X$ 
satisfying the following properties. 
\begin{enumerate}
\item $r^*\mathcal{E}/(\mathrm{torsion})$ is locally free.  
\item $\widetilde{X}$ is a smooth complex analytic variety. 
\item $r$ is obtained by a sequence of blowups at  centers contained in the union of $X_{\sing}$ and the non-locally-free locus of $\mathcal{E}$.
\item The construction of  $r$ is functorial. 
\end{enumerate} 
\end{thm}


\begin{proof}
The proof is very similar to the ones of Lemma \ref{lemma:embeded-dimension-reduction} and  Corollary \ref{cor:blowup-hypersurface-sing}. 
We will only give a sketch and leave the details to the reader. 
Since we will show the functoriality of the construction, 
we only need to prove the theorem locally. 
Hence, we can assume that $(o\in X)$ is a germ of complex analytic variety, and that there is  an exact sequence of coherent sheaves 
\[
\mathcal{O}_X^{\oplus p} \overset{\varphi}{\longrightarrow} \mathcal{O}_X^{\oplus q} \longrightarrow \mathcal{E} \to 0.
\]
Then the morphism $\varphi$ is represented by a  $q\times p$ matrix $\Theta(x)$ with entries as holomorphic functions on $X$. 
If $M$ is the maximum of the ranks of $\Theta(x)$ for $x\in X$, then $q-M$ is the rank of $\mathcal{E}$. 
Let $m$ be the minimum of the ranks of $\Theta(x)$ for $x\in X$. 
Up to shrinking $X$ around $o$, we obtain that $\Theta(o)$  has rank $m$. 
Then $q-m$ is the dimension of the residue of $\mathcal{E}$ at $o$. 
Moreover,  $\mathcal{E}$ is locally free if and only if $M-m=0$.   

Assume that $m<M$. 
Then we let $\mathcal{I}\subseteq \mathcal{O}_X(X)$ be the ideal generated by the determinants of all $(m+1)$-minors of the $\Theta(x)$.  
By a similar argument as in the proof of Lemma \ref{lemma:embeded-dimension-reduction}, the ideal $\mathcal{I}$ depends only on $\mathcal{E}$, and is independent of the choice of the exact sequence. 


Let $f_1\colon X_1\to X$ be the blowup at the ideal $\mathcal{I}$. 
We can pullback $\varphi$ by $f_1$, and obtain an exact sequence of coherent sheaf 
\[
\mathcal{O}_{X_1}^{\oplus p} \overset{f_1^*\varphi}{\longrightarrow} \mathcal{O}_{X_1}^{\oplus q} \longrightarrow f_1^*\mathcal{E} \to 0.
\]  
We note that $f_1^*\varphi$ is represented by the matrix $\Gamma=\Theta\circ f_1$.
Let $o_1\in X_1$ be a point.  
By a similar computation as in the proof of Lemma \ref{lemma:embeded-dimension-reduction}, 
$f_1^*\varphi$ factors through a morphism $\psi_1 \colon \mathcal{O}_{X_1}^{\oplus p} \to \mathcal{O}_{X_1}^{\oplus q}$, such that $\psi_1$ is represented by a matrix $\Gamma_1$ whose rank at $o_1$ is at least $m+1$.  
Therefore, the dimension of the residue of $f_1^*\mathcal{E}/(\mathrm{torsion})$ is smaller than the one of the residue of $\mathcal{E}$ at $o$.   

Since $\mathcal{I}$ depends only on $\mathcal{E}$, the previous blowup is functorial. 
By repeating this procedure for finitely many times, and then we take a functorial resolution of singularities in the end, 
we obtain the required morphism $r\colon  \widetilde{X} \to X$. 
This completes the proof of the theorem. 
\end{proof}





Let us start with a  good reference metric of $\mathcal F$.
\begin{lemma}
\label{lemma:existence:initial-metric-lowerbounded}   
Let $(Z, \omega)$ be a normal   compact K\"ahler variety and $\cF$ a coherent reflexive sheaf on $X$. 
Then there exists a Hermitian metric $h$ on $\cF$  with the following properties. 
\begin{enumerate}
    \item For any point $x\in Z$, there is a neighborhood $U$ of $x$, such that $U$ can be identified as a closed subvariety of an open subset $V$ of $\mathbb{C}^N$ for some integer $N>0$. 
    Moreover, there is a surjective morphism  $ \mathcal{O}_U^a  \to \mathcal{F}|_U $ for some integer $a>0$, and there is a smooth Hermitian metric $g$ on $\mathcal{O}_V^a$, such that $h$ is the natural quotient metric of $g$ on $\mathcal{F}|_U$.   
    \item There is a constant $C$ such that the Chern curvature of $h$ satisfies 
    \begin{equation}\label{lowerboundofCur}
       F_{h} \geq C  \omega \cdot \mathrm{Id}_{\mathcal{F}}
    \end{equation}
    \item Let  $f\colon V\to U$  be a  projective generically finite morphism. 
    Assume that $\mathcal{E}:=f^*\mathcal{F}/{\mathrm{torsion}}$ is locally free. 
    Then $f^*h$ induces naturally a smooth metric $h'$ on $\mathcal{E}$. 
    Moreover, the Chern curvature of $\mathcal{E}$ satisfies 
    \[
      F_{h'} \geq C f^*\omega \cdot \mathrm{Id}_{\mathcal{E}}
    \]
\end{enumerate}
\end{lemma}
\begin{proof}Appendix\end{proof}










\section{a}








{\color{red}  
I realized that the $h_\mathcal{L}$ on $\mathcal{L}= \det \mathcal{F}$ induced by $h_{\mathcal{F}}$ has little chance to be orbifold smooth.  
Indeed, locally, assume $\mathcal{E}$ is a reflexive sheaf on $X$ and $g\colon V\to X$ is an orbifold chart, then $g^*\mathcal{E}/(torsion)$ is reflexive  if and only if $\mathcal{E}$ is locally free. 
In general, we need to consider $(g^*\mathcal{E})^{**}$. 
Hence, if we construct a metric as in Lemma 4.1 above, then it will not extend to a smooth metric on $(g^*\mathcal{E})^{**}$, but only on the subsheaf $g^*\mathcal{E}/(torsion)$.  \\

Maybe we will just let the initial metric $h$ be an orbifold smooth metric on $\mathcal{E}$ over $X$. This guarantee the equation (4.2) has orbifold smooth solutions. 
$\Lambda_\epsilon \theta_\mathcal{L}$ do not have lower bound, but it seems that its integrals are uniformly bounded.   
Indeed, being orbifold smooth, we have  
\[\int_{(X,\omega_\epsilon)} \Lambda_\epsilon \theta_\mathcal{L}
= \int_{X} \theta_\mathcal{L} \wedge \omega_\epsilon^{n-1}
\] which  is a continuous function in $\epsilon\in [0,1]$.  
By compactness, let $\theta$ be some orbifold smooth K\"ahler form so that $\theta+\theta_\mathcal{L}$ is again orbifold smooth K\"ahler.
Then $$|\Lambda_\epsilon \theta_\mathcal{L}| \le  |\Lambda_\epsilon (\theta+\theta_\mathcal{L})| +  |\Lambda_\epsilon \theta| = \Lambda_\epsilon (\theta+\theta_\mathcal{L}) +  \Lambda_\epsilon \theta.$$  
Here we have used the triangle inequality, and we can remove the absolute values on the RHS since the summands are $\geq 0$.  
The RHS is integrable on $(X,\omega_\epsilon)$, uniformly on $\epsilon$.  
This should suffice to show the first inequality of Lemma 4.2.  

For the second inequality, it seems the key is to get the following inequality 
$$\pi^*\theta_\mathcal{L}\leq C(\pi^*p^*\omega_Z+ \ddbar \log|s_D|^{2}).$$ 
I believe that it still holds for our new metric on $\mathcal{E}$.  
Indeed, this is equivalent to say that $ C (\pi^*p^*\{\omega_Z\}- \{D\}) - \pi^*c_1(\mathcal{L}))$ is a K\"ahler class.  
It is true since $(\pi^*p^*\{\omega_Z\} - \{D\})$ is K\"ahler  and $C>0$ is large.  
\\
}

\noindent The normalization mentioned before is as follows. 
{\color{red} the definition of $h_\mathcal{L}$ need to be explained, since $\mathcal{F}$ is not locally free.  Are we using (3) of the previous lemma? 
Moreover, everything lives in $X$. 
} 
Let $(\mathcal{L}, h_\mathcal{L}):= (\det \mathcal{F}, \det h)$ and let $\theta_\mathcal{L}$ be the Chern curvature form on of $h_\mathcal{L}$. 
We recall that the metric $h$ satisfies the inequality \eqref{ross2}.
For all $\epsilon>0$ small enough, the following equation
\begin{equation}\label{eqn:a}
\Lambda_{\epsilon} \theta_\mathcal{L}+ \Delta_{\epsilon}''\rho_{\epsilon}= 
\frac{1}{Vol(X, \omega_{\epsilon})}\int_{X}c_1(\mathcal{F},h)\wedge \{\omega_{\epsilon}\}^{n-1}
\end{equation} 
has a unique solution $\rho_{\epsilon}$ such that $\displaystyle \int_{X}\rho_{\epsilon}dV_\epsilon= 0$ (c.f \cite[Theorem 2.6]{Chiang90}). 
Note that the RHS of \eqref{eqn:a} is regarded as a constant function on $X$. 
Since $\Lambda_\epsilon\theta_L$ is orbifold smooth, so is the solution  $\rho_\epsilon$. 
\medskip

As an application of the Lemma \ref{lemma:meanvalue}, we prove a mean value type inequality for $\rho_\epsilon$, which is crucial to get $\mathcal{C}^0$ estimate for $\rho_\epsilon$ later. 








 
 
\vspace{2mm}

\noindent\textbf{Acknowledgment.} 
The authors are grateful to Bin Guo and Jian Song for conversations.  Xin Fu is  supported by National Key R\&D Program of China 2024YFA1014800 and NSFC No. 12401073. Wenhao Ou is supported by the National Key R\&D Program of China (No. 2021YFA1002300). 


\section{The Bogomolov-Gieseker  inequality}
\label{section:BGL}

In this section, we will prove Theorem \ref{thm:BG}.  
The notion of K\"ahler spaces was introduced in \cite{Grauert1962}.  
For   K\"ahler orbifolds and orbifold coherent sheaves, see for example  \cite[Section 2]{Faulk2022},  \cite[Section 2]{Wu23} or \cite[Section 3.1]{DasOu2023}. 

If $X$ is a compact complex analytic variety of dimension $n$, then there is a natural isomorphism  $H_{2n}(X,\mathbb{R}) \cong H_{2n}^{BM}(X,\mathbb{R})$, where $H_{2n}$ is the singular homology and $H_{2n}^{BM}$ is the Borel-Moore homology. 
We note that  $H_{2n}^{BM}(X,\mathbb{R}) \cong \mathbb{R}$ and 
it  is generated by a canonical element $[X]$, called the fundamental class of $X$.  
For more details on Borel-Moore homology, we refer to \cite[Chapter 19]{Fulton} and the references therein. 
As a consequence, for  cohomology classes $\sigma_1,...,\sigma_k \in H^{\bullet}(X, \mathbb{R})$, such that the sum of their degrees is equal to $2n$, we can define the following intersection number,
\[
\sigma_1  \cdots\sigma_k   :=  (\sigma_1 \smallsmile \cdots  \smallsmile  \sigma_k) \smallfrown [X] \in \mathbb{R}.
\]
When $X$ has only quotient singularities, this intersection form can be identified with the orbifold Poincar\'e duality, see \cite[Theorem 3]{Satake1956}. 




\subsection{Orbifold Chern classes}


We will   explain the   orbifold Chern classes $\hat{c}_{k}$ for $k=1,2$ in the statement of Theorem \ref{thm:BG}.   
We also refer to  \cite[Section 5]{GK20}  for a similar discussion.   



Let $X$ be a compact complex analytic variety of dimension $n$,  
which has quotient singularities in codimension 2.  
Let $\mathcal{F}$ be a coherent reflexive sheaf on $X$.  
Then  there is a Zariski open subset $X^\circ $ of $X$, whose complement has codimension at least 3, such that there is a standard complex  orbifold $\mathfrak{X} = (X_j,G'_j)$ whose quotient space is $X^\circ$. 
Being standard means that the action of $G'_j$ is free in codimension 1.  
Furthermore, since reflexive sheaves on complex manifolds are locally free in codimension 2,  up to shrinking $X^\circ$, we can assume that $\mathcal{F}$ induces an orbifold vector bundle $\mathcal{F}_{\orb}$ on $\mathfrak{X}$, see \cite[Remark 3.4.(5)]{DasOu2023}. 


Let $f\colon Y \to X$ be an orbifold modification as in Theorem \ref{thm:main-thm}. 
Then $Y$ has quotient singularities.     
Let $ \mathfrak{Y}=(Y_i,G_i) $ be   the standard orbifold structure on $Y$, 
and let $\mathcal{E} = (f^*\mathcal{F})/(\mathrm{torsion})$.  
Then $\mathcal{E}$ induces a    reflexive  coherent orbifold sheaf $\mathcal{E}_{\orb}$ on $\mathfrak{Y}$, 
see for example \cite[Remark 3.4.(5)]{DasOu2023}.   
By shrinking $X^\circ$, we may assume that $f$ is an isomorphism over it. 


By \cite[Theorem 3.10]{DasOu2023}, there is a functorial projective bimeromorphic morphism $p_i\colon Z_i\to Y_i$, 
such that $Z_i$ is smooth, that $\mathcal{H}_i:= p_i^*\mathcal{E}_i/(\mathrm{torsion})$ is locally free, and that the indeterminacy locus of $p_i^{-1}$ has codimension at least 3. 
The functoriality implies that there is a canonical action of $G_i$ on $Z_i$ such that $p_i$ is $G_i$-equivariant. 
Furthermore, the  $(Z_i,G_i)$'s induce  a complex orbifold $\mathfrak{Z}$ with quotient space $Z$. 
The collection $(\mathcal{H}_i)$ defines an orbifold vector bundle $\mathcal{H}_{\orb}$  on $\mathfrak{Z}$.   
Therefore,  we have well-defined orbifold Chern classes $\hat{c}_1(\mathcal{H}_{\orb})$ and $\hat{c}_2(\mathcal{H}_{\orb})$ in $H^{\bullet}(Z,\mathbb{R})$, see  \cite[Section 2]{Bla96} or \cite[Definition 3.3]{DasOu2023}.
Let $q\colon Z\to X$ be the natural morphism.  
Then the indeterminacy locus  of $q^{-1}$ has codimension at least 3. 
We can now define the orbifold Chern classes $\hat{c}_1(\mathcal{F})$, $\hat{c}_2(\mathcal{F})$ and $\hat{c}_1(\mathcal{F})^2$.  

\begin{defn}
\label{defn:Chern} 
The orbifold Chern class $\hat{c}_1(\mathcal{F})$, $\hat{c}_2(\mathcal{F})$ and $\hat{c}_1(\mathcal{F})^2$ are defined as   linear forms on $H^{\bullet}(X,\mathbb{R})$, so that  
for any class $\sigma \in H^{\bullet}(X,\mathbb{R})$,  we have 
\begin{eqnarray*}
    \hat{c}_k(\mathcal{F}) \cdot \sigma = \hat{c}_k(\mathcal{H}_{\orb}) \cdot q^*\sigma &  \mbox{ and }  &  
\hat{c}_1(\mathcal{F})^2 \cdot \sigma = \hat{c}_1(\mathcal{H}_{\orb})^2 \cdot q^*\sigma, 
\end{eqnarray*}
where $k=1,2$. 
\end{defn}


To see that these classes are well-defined, it is equivalent to show that the intersection numbers are independent of the choice of the modification $q\colon Z\to X$.  
We first observe the following fact. 

\begin{lemma}
\label{lemma:c2-property}
Let  $X$ be compact complex analytic variety of  dimension $n$ and let $\rho \colon \widetilde{X} \to X$ be a proper bimeromorphic morphism.  
Let $\alpha_1,\alpha_2\in H^{2k}(\widetilde{X} ,\mathbb{R})$.  
Assume that there is a closed analytic subset $V\subseteq X$ of codimension at least $k+1$, such that 
$\alpha_1|_{\widetilde{X}^\circ} =  \alpha_2|_{\widetilde{X}^\circ}$, 
where  ${\widetilde{X}^\circ} =  \widetilde{X} \setminus \rho^{-1}(V)$.  
Then  for any class $\sigma \in H^{2n-2k}(X, \mathbb{R})$, we have 
\[
\alpha_1 \cdot \rho^* \sigma = \alpha_2 \cdot \rho^*\sigma. 
\]
\end{lemma}




\begin{proof} 
Let $E = \rho^{-1}(V)$. 
Then there is some class $\delta \in H^{2k}(\widetilde{X},\widetilde{X}\setminus E, \mathbb{R})$ such that 
  $\alpha_1 - \alpha_2$
is equal to the image of $\delta$ in $H^{2k}(\widetilde{X},\mathbb{R})$. 
It follows that 
\begin{eqnarray*} 
    (\alpha_1 - \alpha_2 ) \cdot \rho^*\sigma  
         &=& \delta \cdot \rho^*\sigma  \\
         &=&  (\rho^*\sigma \smallsmile \delta)  \smallfrown [\widetilde{X}] \\ 
         &=& \rho^*\sigma \smallfrown (\delta \smallfrown [\widetilde{X}]).
\end{eqnarray*}
We note that $\delta \smallfrown [\widetilde{X}] \in H_{2n-2k}(E, \mathbb{R})$. 
Then $\rho_*(\delta \smallfrown [\widetilde{X}]) \in H_{2n-2k}(V, \mathbb{R})$. 
By assumption, we have $\dim V \le n-k-1$. 
Hence $\rho_*(\delta \smallfrown [\widetilde{X}]) = 0$. 
By the projection formula,   we have 
\[
 \rho^*\sigma \smallfrown (\delta \smallfrown [\widetilde{X}]) = \sigma \smallfrown \rho_*(\delta \smallfrown [\widetilde{X}]) =0.
\]
This completes the proof of the lemma. 
\end{proof}


We are in position to prove the following statement. 

\begin{prop}
\label{prop:Chern}
The intersection numbers in Definition \ref{defn:Chern} are independent of the choice of $Z$. 
\end{prop}

\begin{proof}
We assume the following setting. 
Let  $q_j\colon Z_j \to X_j$ be proper bimeromorphic morphisms with $j=1,2$. 
There is a closed analytic subset  $V \subseteq X$ of codimension at least 3,  such that  $X^\circ := X\setminus V$ has quotient singularities.  
The restriction of the  reflexive coherent sheaf $\mathcal{F}$ on $X^\circ$ induces an orbifold vector bundle $\mathcal{F}_{\orb}$ on the standard orbifold structure of $X^\circ$. 
Let $Z_j^\circ = q_j^{-1}(X^\circ)$. Then $q_j$ is an isomorphism on $Z_j^\circ$. 
There is an orbifold $\mathfrak{Z}_j$ whose quotient space is equal to $Z_j$, such that $\mathfrak{Z}_j$ is standard over $Z_j^\circ$. 
There is an orbifold vector bundle $\mathcal{H}_{\orb, j}$ on $\mathfrak{Z}_j$, whose restriction over $Z_j^\circ$
is isomorphic to $\mathcal{F}_{\orb}$ over  $X^\circ$. 
Let $\alpha_j \in \{ \hat{c}_1(\mathcal{H}_{\orb,j}), \hat{c}_2(\mathcal{H}_{\orb,j}), \hat{c}_1(\mathcal{H}_{\orb,j})^2 \}$, such that $\alpha_1$ and $\alpha_2$ are the same type of characteristic class. 

To prove the proposition, it  is enough to show that, for any $\sigma \in H^{\bullet}(X,\mathbb{R})$, the following equality holds  
\[
\alpha_1 \cdot q_1^*\sigma  = \alpha_2 \cdot q_2^*\sigma. 
\]
We note that there is a natural  bimeromorphic  map  $\varphi\colon Z_1\dashrightarrow Z_2$ over $X$. 
Let $\widetilde{X} \subseteq Z_1\times Z_2$ be the closure of the  graph of $\varphi$,  
and let  $p_1\colon \widetilde{X} \to Z_1$,  $p_2\colon \widetilde{X} \to Z_2$ and $\rho\colon \widetilde{X} \to X$
be the natural morphisms. 
Then we have 
\[
\alpha_j \cdot q_j^*\sigma  = p_j^*\alpha_j \cdot \rho^*\sigma 
\]
for $j=1,2$. 
We notice that $p_1$, $p_2$ and $\rho$ are   isomorphisms over $X^\circ$. 
Hence if $\widetilde{X}^\circ = \rho^{-1}(X^\circ)$, then we have 
\[
(p_1^*\alpha_1)|_{\widetilde{X}^\circ} = (p_2^*\alpha_2)|_{\widetilde{X}^\circ}. 
\]
By applying Lemma \ref{lemma:c2-property}, we obtain that 
$\alpha_1 \cdot q_1^*\sigma  =\alpha_2 \cdot q_2^*\sigma.$
This completes the proof of the proposition. 
\end{proof}



\begin{remark}
\label{rmk:Chern} 
We gather some   properties on this notion of Chern classes, which can be derived directly from the definition and Lemma \ref{lemma:c2-property}. 
\begin{enumerate}
    \item  If $X$ has quotient singularities and if $\mathcal{F}$ induces an orbifold vector bundle $\mathcal{F}_{\orb}$ on the standard orbifold structure over $X$, then the Chern classes of $\mathcal{F}$ in Definition \ref{defn:Chern} coincides with the linear forms on $H^{\bullet}(X,\mathbb{R})$ induced by the orbifold  Chern classes of $\mathcal{F}_{\orb}$. 

    \item Let $X^\circ \subseteq X$ be an open subset with quotient singularities, such that $\mathcal{F}|_{X^\circ}$ induces an orbifold vector bundle $\mathcal{F}_{\orb}$ on the standard orbifold structure on $X^\circ$, 
    and that $X\setminus X^\circ$ has codimension at least 3. 
    Let $\Delta\in \{\hat{c}_1, \hat{c}_2, \hat{c}_1^2\}$. 
    If there is some  class $\beta \in H^{\bullet}(X, \mathbb{R})$, 
    whose restriction on $X^\circ$ is equal to $\Delta(\mathcal{F}_{\orb})$,  
    then $\Delta(\mathcal{F})$ is the same as the linear form on $H^{\bullet}(X, \mathbb{R})$ 
    defined by the intersection with  $\beta$. 
    % In other words, $\Delta(\mathcal{F}_{\orb}) \cdot \sigma = \beta\cdot \sigma$ for any $\sigma\in H^{\bullet}(X, \mathbb{R})$. 

      
    \item Let $M = H_1\cap \cdots \cap  H_{n-2}$ be the complete intersection surface of basepoint-free Cartier divisors in       general position.  
          Then  $M $   has quotient singularities, $\mathcal{F}|_M$ is reflexive and inducing an orbifold vector bundle on the standard orbifold structure of $M$, and  we have 
          \[
          \Delta(\mathcal{F}) \cdot c_1(H_1)  \cdots  c_1(H_{n-2}) = \Delta(\mathcal{F}|_M),
          \]
          where $\Delta$ is either $\hat{c}_2$ or $\hat{c}_1^2$.
\end{enumerate} 
\end{remark}
 

With the first Chern class in Definition \ref{defn:Chern}, we can extend  the notion of slope stability as follows.  
Let $X$ be a compact complex analytic  variety of dimension $n$,  
which has quotient singularities in codimension 2.  
Let $\mathcal{F}$ be a coherent reflexive sheaf on $X$.  
For a cohomology class $\alpha \in H^2(X,\mathbb{R})$, we can define the slope
\[
\mu_{\alpha} (\mathcal{F}) = \frac{\hat{c}_1(\mathcal{F}) \cdot \alpha^{n-1}}{\mathrm{rank}\, F}.
\]
Then $\mathcal{F}$ is called $\alpha$-semistable if for any nonzero coherent  subsheaf $\mathcal{E}\subseteq \mathcal{F}$, we have $\mu_{\alpha} (\mathcal{E}) \le \mu_{\alpha} (\mathcal{F})$. 
It is called $\alpha$-stable if the inequality is strict whenever $\mathcal{E}$ has smaller rank.  
From the item (2) of  Remark  \ref{rmk:Chern}, we see that this definition coincides with the classic one, 
if $X$ satisfies that  every reflexive coherent sheaf of rank 1 on it 
has a positive reflexive power which is locally free.  


















\subsection{Proof of the Bogomolov-Gieseker  inequality} 

We complete the proof of Theorem \ref{thm:BG} in this subsection. 

%We will need the following  lemma, showing that the stability is an open condition. 

%\begin{lemma}
%\label{lemma:perturbe-stable}
%Let   $\mathfrak{X} =(X_i,G_i)$ be a compact complex orbifold of dimension $n$, such that the actions of the $G_i$'s are faithful.  Let $\mathcal{E}_{\orb} = (\mathcal{E}_i)$ be an orbifold vector bundle on $\mathfrak{X}$.  Let $\omega$ be an orbifold K\"ahler form and $\eta$ a  closed semipositive  orbifold  $(1,1)$-form on $\mathfrak{X}$.  Assume that $\mathcal{E}_{\orb}$  is stable with respect to $\eta$.  Then it is stable with respect to $\eta+ \epsilon \cdot \omega$ for all $\epsilon>0$ small enough. 
%\end{lemma}


%\begin{proof}  
%See \cite[Lemma 3.16]{DasOu2023}.
%\end{proof}


 




%We are now in the position of proving Theorem \ref{thm:BG}. 


%and Theorem \ref{thm:BG-equality}.



\begin{proof}[{Proof of Theorem \ref{thm:BG}}]
Let $f\colon Y\to X$ be an orbifold modification as in Theorem \ref{thm:main-thm}. 
Then $Y$ has quotient singularities. 
It follows that $Y$ is the quotient space of an orbifold  $ \mathfrak{Y} = (Y_i,G_i)_{i\in I}$, 
such that the action of $G_i$ is free in codimension 1. 
Since $f$ is projective, $Y$ is a compact K\"ahler variety.      
Let $\mathcal{E} = (f^*\mathcal{F})/(\mathrm{torsion})$.  
Then $\mathcal{E}$ induces a    reflexive  coherent orbifold sheaf $\mathcal{E}_{\orb}$ on $\mathfrak{Y}$, 
see for example \cite[Remark 3.4.(5)]{DasOu2023}.   
We note that $\mathcal{E}$ is stable with respect to $f^*\omega$.  
By \cite[Remark 3.4.(5)]{DasOu2023} again, we deduce that $\mathcal{E}_{\orb}$ is stable with respect to the class of $f^*\omega$. 



By \cite[Theorem 3.10]{DasOu2023}, there is a functorial projective bimeromorphic morphism $p_i\colon Z_i\to Y_i$, 
such that $Z_i$ is smooth, that $\mathcal{H}_i:= p_i^*\mathcal{E}_i/(\mathrm{torsion})$ is locally free, and that the indeterminacy locus of $p_i^{-1}$ has codimension at least 3. 
In particular, there is a canonical action of $G_i$ on $Z_i$ such that $p_i$ is $G_i$-equivariant, 
and that the $(Z_i,G_i)$'s induce a complex orbifold $\mathfrak{Z}$ with quotient space $Z$.  
The collection $(\mathcal{H}_i)$ defines an orbifold vector bundle $\mathcal{H}_{\orb}$  on $\mathfrak{Z}$.  
There is an induced projective bimeromorphic morphism $p\colon Z \to Y$. 
In particular, $Z$ is a K\"ahler variety.  
Then there is an orbifold K\"ahler form $\eta$ on $\mathfrak{Z}$, see \cite[Lemma 1]{Wu23}. 
Let $q\colon Z\to X$ be the natural morphism. 
We note that $q^*\omega$  can be viewed as  a semipositive orbifold $(1,1)$-form on $\mathfrak{Z}$. 
Moreover,  $\mathcal{H}_{\orb}$ is stable with respect to $q^*\omega$ for $\mathcal{E}_{\orb}$ is stable with respect to  $f^*\omega$.   
By the same argument of \cite[Lemma 3.16]{DasOu2023}, we deduce that,  
for any $\epsilon>0$ small enough,  
$\mathcal{H}_{\orb}$ is stable with respect to the following orbifold K\"ahler form on $\mathfrak{Z}$  
\[
\alpha_\epsilon:= q^*\omega + \epsilon \eta. 
\]
By \cite[Theorem 1]{Faulk2022}, $\mathcal{H}_{\orb}$ admits an orbifold Hermitian-Einstein metric with respect to $\alpha_{\epsilon}$. 
Then, as  in   \cite[Theorem 4.4.7]{Kobayashi2014},   we have  
\begin{equation*}\label{eqn:BG}
\Big(2r\hat{c}_2(\mathcal{H}_{\orb}) - (r-1)\hat{c}_1(\mathcal{H}_{\orb})\Big)\cdot [\alpha_\epsilon]^{n-2}  
\geq 0.
\end{equation*}
The limit of the LHS above, when $\epsilon$ tends to 0, is equal to 
\begin{eqnarray*}
&& \Big(2r\hat{c}_2(\mathcal{H}_{\orb}) - (r-1)\hat{c}_1(\mathcal{H}_{\orb})\Big)\cdot [ q ^*\omega ]^{n-2} \\
&=& \Big(2r\hat{c}_2(\mathcal{F}) - (r-1)\hat{c}_1(\mathcal{F})^2 \Big) \cdot [\omega]^{n-2}
\end{eqnarray*}   
This completes the proof of the theorem.
\end{proof}







\subsection{smooth metrics}


The goal of this section is to construct an initial metric $H_0$ for the heat equation (\ref{equa:heat-equation-metric}) and also a  reference metric with good curvature property. Such a metric will guarantee that $\Lambda F$ is integrable, where $F$ is the Chern curvature.  
We will follow the lines of \cite[Section 3]{BandoSiu1994}. 
The idea is to construct a metric which is locally the restriction of a smooth metric on a vector bundle.
The first step is to construct an appropriate partition of the unity. 

 

\begin{prop}\label{prop:partition}
Let $X$ be an analytic variety. Let $\{U_i\}$ be a finite open covering of $X$. We assume that each $U_i$ can be identified as a closed analytic subvariety of a domain $V_i \subseteq \mathbb{C}^{N_i}$.
Then there is a partition of the unity, subordinate to $ \{U_i\}$, which satisfies the following property.
For  $x\in X$, there is an open neighbourhood $U$ of $x$, such that 
\begin{enumerate}
\item $U$ is contained in $U_i$ whenever $x\in U_i$,
\item if $U \subseteq U_i$, then there is an open subset $W_i\subseteq V_i$ containing $U$ such that  $\rho_i|_U$ is the restriction of some smooth function $\eta_i\colon W_i \to \mathbb{R}$. 
\end{enumerate}
\end{prop}


\begin{proof}
Since a partition of the unity  can be constructed by  taking linear combinations of  products of  bump functions, we will construct bump functions on $U_i$ which satisfy the property in the proposition.

Without loss of generality, we take a bump function $\gamma_1$ on $V_1$, which restricts to a bump function $\rho_1$  on $U_1$. Let $x\in U_1$ be any point. 
Assume that $x\in U_i$. 
Then  there exists open subsets  $x\in W_1\subseteq V_1 \subseteq \mathbb{C}^{N_1}$  and $x\in W_i\subseteq V_i \subseteq \mathbb{C}^{N_i}$,  such that the identity map of $U_i\cap U_1$ induces a transition function
\[ \varphi_{i,1} \colon W_i \to W_1, \] which is holomorphic. 
We set $\eta_i = \gamma_1 \circ \varphi_{i,1}$. 
Let  $U$ be an open neighbourhood $x$ which is contained both in $W_1$ and $W_i$.  
Then $\rho_1|_U = \eta_i|_U$. 
By taking account of all $U_i$, we complete the proof of the lemma.
%
%Moreover, since it is the restriction of a smooth function in   $V_1$, it follows that $\rho_1$, viewed as a function on $U_i\cap U_1 \subseteq V_i\subseteq \mathbb{C}^{N_i}$, is a  smooth $(\mathcal{C}^\infty)$ function, in the sense of \cite[section 3]{Whitney1934}. Hence by  \cite[Theorem 1]{Whitney1934}, $\rho_1$ can be  extended to a smooth function in a neighbourhood of $U_i\cap U_1 \subseteq V_i$.
%
\end{proof}

 

We will also need the following local estimate.


\begin{lemma}
\label{lemma:curvature-local-bound}
Let $(X, \omega)$ be a    K\"ahler subvariety  of a K\"ahler manifold $(V,\omega)$. 
Let  \[ \cF \subseteq \cO_X^a \]  be a saturated coherent subsheaf.  
Let $g $ be a Hermitian metric on $\cO_V^a$ and $h$ its restriction on $\cO_X^a$.
We denote by $h_\cF$ the restriction of $h$  on $\cF$. 
Let $F_{h_\cF}$ be  the Chern curvature. 
Then   for any precompact open subset $U$ of $X$, there is  a constant $C>0$ such that 
\[ \sqrt{-1} F  \leqslant C\omega \cdot \mathrm{Id}_\cF\]  

Similarly, if $\mathcal{F}$ is a torsion-free quotient of $\mathcal{O}_X^a$, and if $h$ is the quotient metric of $h$, then $ \sqrt{-1} F  \geqslant C\omega \cdot \mathrm{Id}_\cF$. 
\end{lemma}

\begin{proof}
%For simplicity, we omit the lower index $\cF$ for the  locus $X_\cF$.
%We will work locally. 
%Let $x\in X_\cF\cap U$ and $Y \subseteq X_\cF \cap U$ an open neighbourhood of $x$.  
We denote   by  $F_h$    and  $F_g$ the corresponding Chern curvatures. 
Then we have  
\[F_h = F_g|_X.\]
Furthermore, we denote by   $\nabla_h$ and $\nabla_{h_{\cF}}$ the  corresponding Chern connections. 
%The following computation can be found in \cite[Section 1.6]{Kobayashi2014}.  
Then there is a   $(1,0)$-form $A$ on $X$, with values in $\cH om(\cF, \cF^{\perp_h })$, such that,  for any smooth section $s$ of $\cF$ on $X$,  we have 
\[  \nabla_h s = \nabla_{h_\cF} s+ As.\]
Then $$ \sqrt{-1} F_{h_\cF} = \sqrt{-1}F_{h}|_{\cF} - \sqrt{-1}A\wedge A^* \leqslant \sqrt{-1}F_h|_{\cF}.$$  

For a precompact open set $U$, then there is a precompact open subset $W\subseteq V$, which contains $U$. 
Then   there is some constant  $C$,   such that,    on $W$,  we have
\[   \sqrt{-1} F_g   \leqslant C\omega \cdot \mathrm{Id}_{\cO_V^{a}}.  \]
Thus  $ \sqrt{-1} F_{h_\cF}   \leqslant C\omega \cdot \mathrm{Id}_\cF$ on $U$. 
This completes the proof of the lemma.
\end{proof}


Now we are ready to construct the desire metric. 


\begin{lemma}
%\label{lemma:existence:initial-metric-lowerbounded}   
Let $(X, \omega)$ be a normal   compact K\"ahler variety and $\cF$ a coherent reflexive sheaf on $X$. 
Then there exists a Hermitian metric $h$ on $\cF$  with the following properties. 
\begin{enumerate}
    \item For any point $x\in X$, there is a neighborhood $U$ of $x$, such that $U$ can be identified as a closed subvariety of an open subset $V$ of $\mathbb{C}^N$ for some integer $N>0$. 
    Moreover, there is a surjective morphism  $ \mathcal{O}_U^a  \to \mathcal{F}|_U $ for some integer $a>0$, and there is a smooth Hermitian metric $g$ on $\mathcal{O}_V^a$, such that $h$ is the natural quotient metric of $g$ on $\mathcal{F}|_U$.   
    \item There is a constant $C$ such that the curvature of $h$ satisfies 
    \[
     \sqrt{-1} F_{h} \geq C  \omega \cdot \mathrm{Id}_{\mathcal{E}}
    \]
    \item Let  $f\colon Y\to U$  be a  projective generically finite morphism. 
    Assume that $\mathcal{E}:=f^*\mathcal{F}/{\mathrm{torsion}}$ is locally free. 
    Then $f^*h$ induces naturally a smooth metric $\overline{h}$ on $\mathcal{E}$. 
    Moreover, the curvature of $\mathcal{E}$ satisfies 
    \[
    \sqrt{-1} F_{\overline{h}} \geq C f^*\omega \cdot \mathrm{Id}_{\mathcal{E}}
    \]
\end{enumerate}
\end{lemma}


\begin{proof}  
There is a finite open covering $\{U_i\} $ of $X$ such that the following properties hold.
\begin{enumerate}
\item Every $U_i$ is precompact and can be identified with an analytic subvariety of a domain $V_i \subseteq \mathbb{C}^{N_i}$.

\item There is a surjective morphism 
\[  \cE_i \to \cF |_{U_i} \to 0
\] 
such that   $ \cE_i$ is a  free coherent sheaf on $U_i$.   
\end{enumerate}
Let $\cE_i^{V_i}$ be a free coherent sheaf on $V_i$ such that $\cE_i = \cE_i^{V_i}|_{U_i}$. 
Let $g_i$ be a Hermitian metric  on  $\cE^{V_i}$.  
By taking quotients, it  induces a Hermitian metric on $h_i$ on  $\cF|_{U_i}$.  
Let $\{\rho_i\}$ be a partition of the unity subordinate to $\{U_i\}$, as in Proposition \ref{prop:partition}. 
We define the Hermitian metric $h = \sum \rho_i h_i$ on $\cF$.   



The items (2) and (3) follow from the item (1) by applying Lemma \ref{lemma:curvature-local-bound}. 
Hence, we only need to show that $h$ satisfies the item (1).    
Up to shrinking $U$, we assume it satisfies the condition of Proposition \ref{prop:partition}. 
Without loss of the generality, we can also assume that $U_1,...,U_k$ are all the open subsets in the open covering $\{U_i\} $  which contains $x$. 
By taking diagonal, there is an embedding 
\[
\iota \colon U \subseteq  V_1 \times \cdots \times V_k =:W
\]
which identifies $U$ as a locally closed subvariety of $W$. 
Let $\mathcal{G}$ be the free coherent sheaf on $W$ defined as the direct sum of the  pullbacks of the free sheaves $\cE_1,...,\cE_k$. 
Let $g$ be the possibly singular Hermitian metric on $\mathcal{G}$ given by the direct sum of $\rho_1 h_1, ..., \rho_k h_k$.
Then the is a natural surjective morphism from   $\mathcal{G}|_U$ to $\mathcal{F}|_U$, and $h$ is just equal to the quotient metric of $g$ on $\mathcal{F}|_U$. 
This proves the item (1), and completes the proof of the lemma.  
\end{proof}












\begin{lemma}\label{lemma:existence:initial-metric-upperbounded}
Let $(X, \omega)$ be a normal   compact K\"ahler variety and $\cF$ a coherent reflexive sheaf on $X$. 
Then there exists a Hermitian metric $H$ on $\cF$ with Chern curvature $F$ such that  
\[ \sqrt{-1} F  \leqslant C\omega \cdot \mathrm{Id}_\cF\] 
for some positive constant $C$.    

Moreover, let $r\colon \widehat{X} \to X$ be a desingularization  by successively blowing up smooth centers. 
Assume  that the torsion-free quotient $  r^*(\cF^*) /({\mathrm{torsion}})$ is locally free, then $H$ extends to a smooth Hermitian   metric $\widehat{H}$ on the locally free sheaf $ (r^*(\cF^*))^*$.
\end{lemma}


\begin{proof} 
There is a finite open covering $\{U_i\}$ of $X$ such that the following properties hold.
\begin{enumerate}
\item Every $U_i$ is precompact and can be identified with an analytic subvariety of a domain $V_i \subseteq \mathbb{C}^{N_i}$.

\item There is a surjective morphism 
\[  \cE_i \to \cF^*|_{U_i} \to 0
\] 
such that   $ \cE_i$ is a  free coherent sheaf on $U_i$.   
\end{enumerate}
Let $\cE_i^{V_i}$ be a free coherent sheaf on $V_i$ such that $\cE_i = \cE_i^{V_i}|_{U_i}$. 
Let $g_i$ be a Hermitian metric  on  $(\cE^{V_i})^*$.  
By restriction, it  induces a Hermitian metric on $h_i$ on  $\cF|_{U_i}$.  
Let $\{\rho_i\}$ be a partition of the unity subordinate to $\{U_i\}$, as in Proposition \ref{prop:partition}. 
We define the Hermitian metric $H = \sum \rho_i h_i$ on $\cF$.  
We denote by  $F$   its Chern curvature.
We will show that $H$ has the properties of the lemma.


Let $x \in X$ be a point. Then there is an open neighbourhood $x \in U$  such that 
\begin{enumerate}
\item $U$ is contained in $U_i$ whenever $x\in U_i$.
\item if $x\in U \subseteq U_i$, then there is an open subset $W_i\subseteq V_i$ containing $U$ such that  $\rho_i|_U$ is the restriction of some smooth function $\eta_i\colon W_i \to \mathbb{R}$. 
\end{enumerate}
Now we consider the diagonal embedding, followed by the product inclusion,  
\[U  {\hookrightarrow}    \prod_{x\in U_i} U_i  \hookrightarrow   \prod_{x\in U_i} W_i. \] 
Let $\iota \colon U \to \prod_{x\in U_i}W_i$ be the composition of this sequence. 
Then it realizes $U$ as a K\"ahler   subvariety of the manifold $\prod_{x\in U_i}W_i,$ up to scaling the K\"ahler form by the cardinality of $\{ U_i\ |\    x \in U_i \}$.

We denote by $p_i\colon\prod_{x\in U_i} W_i \to W_i $ the natural projections. 
By abuse of notation, we still denote by $p_i$ the projection from $\iota(U)$ to $U\subseteq U_i \subseteq W_i$. 
For the reflexive sheaf $\cF|_U$, we have the following sequence of sheaves supported in  $ \iota(U) \subseteq \prod_{x\in U_i} W_i $, such that the first one is the diagonal map, and the second one is the direct sum embedding, 
\[ \iota_*(\cF|_U) \hookrightarrow  \bigoplus_i p_i^*(\cF|_U)  \hookrightarrow   \bigoplus_i  p_i^*(  \ce^i|_U).\]

We remark that 
\[\bigoplus_i p_i^* ( \ce^i|_U )=   \Big(\bigoplus_i p_i^* ((\cE^{V_i})^*|_{W_i} ) \Big) \Big|_{\iota(U)}. \] 
The vector bundle $ \bigoplus_i p_i^* ((\cE^{V_i})^*|_{W_i} ) $ can be equipped with the orthogonal direct sum metric $g' = \sum \eta_i p_i^*g_i$. 
Then the metric $H$ on $\cF|_U$ coincides exactly the restriction of $g'$ on $\iota_*(\cF|_U)$.
Hence by Lemma \ref{lemma:curvature-local-bound}, we have 
\[\sqrt{-1} F   \leqslant  C_x \omega \cdot \mathrm{Id}_\cF\] 
over some open neighborhood of $x$, for some positive number $C_x$. 
 
Since $X$ is compact, it follows that there is a positive constant $C$ such that 
\[\sqrt{-1} F   \leqslant  C \omega \cdot \mathrm{Id}_\cF.\]

For the second part of the lemma,    we will first  work locally. 
Let $\widehat{U}_i = r^{-1}(U_i)$. 
Since $r$ is obtained by blowing smooth centers, we can blow up $V_i$ with the same centers and obtain a manifold $q\colon \widehat{V}_i \to V_i$, such that $\widehat{U}_i$ is the strict transform of $U_i$ in $\widehat{V}_i$. 
We have the following exact sequence, 
 \[  r^*\cE_i \to  r^*\cF^*|_{\widehat{U}_i} \to 0. \]
By taking the dual morphism, we see that  $ (r^*\cF^*)^* |_{\widehat{U}_i}$ is a subsheaf of $ (r^*\cE_i)^*$. 
By the  assumption of freeness,  $ (r^*(\cF^*))^*|_{\widehat{U}_i}$ is  indeed a subbundle of $  (r^*\cE_i)^* $.
The metric $q^*g_i$ on $q^* (\cE^{V_i})^*$  then  induces a metric $\widehat{h}_i$ on $ (r^*\cF^*)^*|_{\widehat{U}_i}$ by restriction. 
We note that $\widehat{h}_i$ is an extension of $h_i$. 
By taking account of all $U_i$, we see that $H$ extends to a smooth Hermitian metric  $\widehat{H}$ on the locally free sheaf $( r^*\cF^*)^*$.
\end{proof}


For such a metric, we have the following statement on integrability. 


\begin{cor}
\label{cor:initial-metric-integral}
With the notation above, let  $n=\dim X$.  
Let $Z \subseteq X$ be the   locus where $r^{-1}$ is not an isomorphism. 
For any K\"ahler form $\eta$ on $\widehat{X}$, we still denote by $\eta$ its restriction on $X\backslash Z$. 
Then on $X\backslash Z$, we have 
\[ ||\sqrt{-1}  \Lambda_\eta F  ||_H  \eta^{n } \leqslant \mathrm{tr}\, \Big( 2nC \omega \mathrm{Id}_\cF   - n \mathrm{tr}\, (\sqrt{-1}F ) \Big) \wedge \eta^{n-1}.  \]
%Here, by abuse of notation, we identify the Chern curvature $F$ of $H$ with the one of the metric $\widehat{H}$ on $(r^*(\cF^*))^*$. 
As a consequence,  
\begin{eqnarray*}
\int_{{X\backslash Z}} ||\sqrt{-1}  \Lambda_\eta F  ||_H  \eta^{n }  &\leqslant&  2nC \cdot \mathrm{rank}\, \cF\cdot [r^*\omega]\wedge [\eta^{n-1}]\\
&& -2n\pi \cdot c_1\Big( (r^*(\cF^*))^* \Big) \wedge  [\eta^{n-1}] 
\end{eqnarray*}
is bounded, where $[\eta]$ and $[r^*\omega]$ are the corresponding cohomology classes on $\widehat{X}$.
\end{cor}


\begin{proof}
We have  
\[ ||\sqrt{-1}  \Lambda_\eta F  ||_H \leqslant ||\sqrt{-1}  \Lambda_\eta(C\omega \mathrm{Id}_{\cF} - F)  ||_H  + ||C\sqrt{-1}  \Lambda_\eta \omega \mathrm{Id}_{\cF} ||_H.
\]
For simplicity, we set $g= \sqrt{-1}  \Lambda_\eta(C\omega \mathrm{Id}_{\cF} - F). $
Since $F$ is the Chern curvature of $H$, we see that  $g$ is selfadjoint with respect to  $H$. 
Thus $ ||g ||_H = \sqrt{\mathrm{tr}\, (g^2)}$.
Since $g$  is positive by Lemma \ref{lemma:existence:initial-metric}, it follows that 
\[ ||g ||_H  \leq \mathrm{tr}\, g = \sqrt{-1} \Lambda_\eta\mathrm{tr}\, (  (C\omega \mathrm{Id}_{\cF} - F)). \] 
Thus \[ 
||\sqrt{-1}  \Lambda_\eta F  ||_H  \eta^{n } \leqslant  n \mathrm{tr}\, (2 C\omega \mathrm{Id}_{\cF} - F) ) \eta^{n-1}.  \]
This proves the first inequality. 

For the estimate on the integral, we note that $F$ can be viewed as the restriction of $\widehat{F}$ on $X\backslash Z$, where $\widehat{F}$ is the Chern connection of $\widehat{H}$ on $( r^*\cF^*)^*$.   
It follows that 
\[ \int_{X\backslash Z} \mathrm{tr}\, (\sqrt{-1} F)\wedge \eta^{n-1} = \int_{\widehat{X}} \mathrm{tr}\, (\sqrt{-1} \widehat{F} )\wedge \eta^{n-1} =2\pi c_1\Big( (r^*(\cF^*))^* \Big) \wedge  [\eta^{n-1}]. \]
This completes the proof.
\end{proof}

\begin{remark}\label{rem:uniform-integrable}
With the notation above, we assume that $\omega_\epsilon$ is a family of K\"ahler metric on $X\backslash Z$, bounded from above by $\eta$. 
Then the corollary indeed implies that 
$\int_{{X\backslash Z}} ||\Lambda_{\omega_\epsilon} F  ||_H  \omega_\epsilon^{n }$
are uniformly integrable, independent of $\epsilon$,  in the following sense. 
We fix an exhaustion $\{X_j\}$ of $X\backslash Z$, consisting of precompact subsets.  
Let  $T_j = {X\backslash (Z \cup X_j) }$.
Then for each $\xi >0$, there is a positive number $m_0$ such that for any $j \geqslant m_0$, we have  
\[
\int_{T_j} || \Lambda_{\omega_\epsilon} F  ||_H  \omega_\epsilon^{n } \leqslant \xi.
\]
Indeed,  as in Corollary \ref{cor:initial-metric-integral}, we have $ || \Lambda_{\omega_\epsilon} F  ||_H  \omega_\epsilon^{n } \leqslant \Theta \wedge \omega_\epsilon^{n-1},$ where we denote $\Theta = \mathrm{tr}\, \Big( 2nC \omega \mathrm{Id}_\cF   - n \mathrm{tr}\, (\sqrt{-1}F ) \Big)$.  
Lemma \ref{lemma:existence:initial-metric} implies that $\Theta$ is positive. 
Hence $0 \leqslant \Theta \wedge \omega{_\epsilon}^{n-1} \leqslant \Theta \wedge \eta^{n-1}.$ 
Since $\Theta\wedge \eta^{n-1}$ is integrable, there is some $m_0$ such that for any $j \geqslant m_0$, we have 
\[
\int_{T_j}  \Theta \wedge \eta^{ n-1}  \leqslant \xi.
\]
This implies the estimate. 
\end{remark} 

 It is clear that if the equality holds, then $$\Big(2r\hat{c}_2(\mathcal{H}_{\orb}) - (r-1)\hat{c}_1(\mathcal{H}_{\orb})\Big)\cdot [\alpha_\epsilon]^{n-2}  
\rightarrow 0^+.$$
Equivalently
$$\int \Big(2r\tr F_\epsilon^2 - (r-1)\tr F_\epsilon\wedge \tr F_\epsilon\Big) \alpha_\epsilon^{n-2}  
\rightarrow 0^+.$$
Note that on any compact set $K\subset U$, $h_\epsilon$ converge smoothly to $h_\infty$, then we have
$$\int_K \Big(2r\tr F^2 - (r-1)\tr F\wedge \tr F\Big) \omega^{n-2}  
= 0.$$
By the HYM equation, the integrand is nonnegative and hence the integrand  must be zero on $K$ and also on $U$.
