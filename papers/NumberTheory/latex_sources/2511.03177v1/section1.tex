\ifdefined\isMainDocument
\else
    \documentclass[10pt,oneside,reqno]{amsart}
    \usepackage{xr}
    \externaldocument{../main} % Main の .aux を参照
    \usepackage{latexsym}
\usepackage{amscd}
\usepackage{amsmath}
\usepackage{amssymb}
\usepackage[mathscr]{euscript}
\usepackage{graphicx}
\usepackage{geometry}
\usepackage{mathtools}
\usepackage[all]{xy}
\usepackage[dvipsnames]{xcolor}
\usepackage[colorlinks,final,hyperindex]{hyperref}
\usepackage{tikz}
\usepackage{tikz-cd}
\usetikzlibrary{decorations.pathmorphing,decorations.markings,arrows,calc,shapes.geometric,arrows.meta,positioning}
\usepackage{enumerate}
\usepackage{multirow}

\usepackage[bb=dsserif]{mathalpha}
\usepackage{bm}

\newcommand{\QQ}{\ensuremath{\mathbb{Q}}}
\newcommand{\DD}{\ensuremath{\mathbb{D}}}
\newcommand{\CC}{\ensuremath{\mathbb{C}}}
\newcommand{\RR}{\ensuremath{\mathbb{R}}}
\newcommand{\ZZ}{\ensuremath{\mathbb{Z}}}
\newcommand{\NN}{\ensuremath{\mathbb{N}}}

\newcommand{\g}{\ensuremath{\mathfrak{g}}}
\renewcommand{\SS}{\ensuremath{\mathbb{S}}}

\newcommand{\id}{\ensuremath{\mathrm{id}}}

\makeatletter

\newcommand{\bbone}{\mathbb{1}}

\newcommand{\calA}{\mathcal{A}}
\newcommand{\calB}{\mathcal{B}}
\newcommand{\calC}{\mathcal{C}}
\newcommand{\calD}{\mathcal{D}}
\newcommand{\calE}{\mathcal{E}}
\newcommand{\calF}{\mathcal{F}}
\newcommand{\calG}{\mathcal{G}}
\newcommand{\calH}{\mathcal{H}}
\newcommand{\calI}{\mathcal{I}}
\newcommand{\calJ}{\mathcal{J}}
\newcommand{\calK}{\mathcal{K}}
\newcommand{\calL}{\mathcal{L}}
\newcommand{\calM}{\mathcal{M}}
\newcommand{\calN}{\mathcal{N}}
\newcommand{\calO}{\mathcal{O}}
\newcommand{\calP}{\mathcal{P}}
\newcommand{\calQ}{\mathcal{Q}}
\newcommand{\calR}{\mathcal{R}}
\newcommand{\calS}{\mathcal{S}}
\newcommand{\calT}{\mathcal{T}}
\newcommand{\calU}{\mathcal{U}}
\newcommand{\calV}{\mathcal{V}}
\newcommand{\calW}{\mathcal{W}}
\newcommand{\calX}{\mathcal{X}}
\newcommand{\calY}{\mathcal{Y}}
\newcommand{\calZ}{\mathcal{Z}}
\newcommand{\kk}{\Bbbk}

\newcommand{\scrY}{\mathscr{Y}}
\newcommand{\scrV}{\mathscr{V}}
\newcommand{\scrP}{\mathscr{P}}

\newcommand{\Rep}{\mathop{\mathrm{Rep}}\nolimits}
\newcommand{\alg}{\mathop{\mathrm{alg}}\nolimits}
\newcommand{\Span}{\mathop{\mathrm{Span}}\nolimits}
\newcommand{\proj}{\mathop{\mathrm{proj}}\nolimits}
\newcommand{\Hom}{\mathop{\mathrm{Hom}}\nolimits}
\newcommand{\PHom}{\mathop{\mathrm{PHom}}\nolimits}
\newcommand{\Tw}{\mathop{\mathrm{Tw}}\nolimits}
\newcommand{\Dim}{\mathop{\mathrm{Dim}}\nolimits}
\newcommand{\Imm}{\mathop{\mathrm{Im}}\nolimits}
\newcommand{\Aus}{\mathop{\mathrm{Aus}}\nolimits}
\newcommand{\Ann}{\mathop{\mathrm{Ann}}\nolimits}
\newcommand{\APC}{\mathop{\mathrm{APC}}\nolimits}
\newcommand{\Barr}{\mathop{\mathrm{Bar}}\nolimits}
\newcommand{\Cone}{\mathop{\mathrm{Cone}}\nolimits}
\newcommand{\modu}{\mathop{\mathrm{mod}}\nolimits}
\newcommand{\dgMod}{\mathop{\mathrm{dgMod}}\nolimits}
\newcommand{\soc}{\mathop{\mathrm{soc}}\nolimits}
\newcommand{\dg}{\mathop{\mathrm{dg}}\nolimits}
\newcommand{\red}{\mathop{\mathrm{red}}\nolimits}
\newcommand{\Map}{\mathop{\mathrm{Map}}\nolimits}
\newcommand{\inj}{\mathop{\mathrm{inj}}\nolimits}
\newcommand{\op}{\mathop{\mathrm{op}}\nolimits}
\newcommand{\Ker}{\mathop{\mathrm{Ker}}\nolimits}
\newcommand{\Tor}{\mathop{\mathrm{Tor}}\nolimits}
\newcommand{\Tot}{\mathop{\mathrm{Tot}}\nolimits}
\newcommand{\Ext}{\mathop{\mathrm{Ext}}\nolimits}
\newcommand{\sg}{\mathop{\mathrm{sg}}\nolimits}
\newcommand{\Sl}{\mathop{\mathfrak{sl}}\nolimits}
\newcommand{\nc}{\mathop{\mathrm{nc}}\nolimits}
\newcommand{\Tr}{\mathop{\mathrm{Tr}}\nolimits}
\newcommand{\Irr}{\mathop{\mathrm{Irr}}\nolimits}
\newcommand{\HC}{\mathop{\mathrm{HC}}\nolimits}
\newcommand{\HH}{\mathop{\mathrm{HH}}\nolimits}
\newcommand{\THH}{\mathop{\mathrm{TH}}\nolimits}
\newcommand{\Perf}{\mathop{\mathrm{Perf}}\nolimits}
\newcommand{\del}{\partial}
\newcommand{\ev}{\mathop{\mathrm{ev}}\nolimits}
\newcommand{\co}{\mathop{\mathrm{co}}\nolimits}
\newcommand{\pt}{\mathop{\mathrm{pt}}\nolimits}
\newcommand{\wt}{\mathop{\mathrm{wt}}\nolimits}
\newcommand{\pCY}{\mathop{\mathrm{pCY}}\nolimits}
\newcommand{\gr}{\mathop{\mathrm{gr}}\nolimits}
\newcommand{\coker}{\mathop{\mathrm{coker}}\nolimits}


\newcommand{\Top}{\mathop{\mathrm{Top}}\nolimits}
\newcommand{\Set}{\mathop{\mathrm{Set}}\nolimits}
\newcommand{\Vect}{\mathop{\mathrm{Vect}}\nolimits}
\newcommand{\Mod}{\mathop{\mathrm{Mod}}\nolimits}
\newcommand{\Ob}{\mathop{\mathrm{Ob}}\nolimits}
\newcommand{\Ind}{\mathop{\mathrm{Ind}}\nolimits}
\newcommand{\Sym}{\mathop{\mathrm{Sym}}\nolimits}
\newcommand{\Alt}{\mathop{\mathrm{Alt}}\nolimits}
\newcommand{\End}{\mathop{\mathrm{End}}\nolimits}
\newcommand{\Ch}{\mathop{\mathrm{Ch}}\nolimits}


\newcommand{\Ass}{\mathop{\mathrm{Ass}}\nolimits}
\newcommand{\Com}{\mathop{\mathrm{Com}}\nolimits}
\newcommand{\Lie}{\mathop{\mathrm{Lie}}\nolimits}
\newcommand{\colim}{\mathop{\mathrm{colim}}\nolimits}

\newcommand{\Op}{\mathop{\mathsf{Operads}}\nolimits}
\newcommand{\Diop}{\mathop{\mathsf{Dioperads}}\nolimits}
\newcommand{\PDiop}{\mathop{\mathsf{PDioperads}}\nolimits}
\newcommand{\Coop}{\mathop{\mathsf{Cooperads}}\nolimits}
\newcommand{\Codiop}{\mathop{\mathsf{Codioperads}}\nolimits}
\newcommand{\PCodiop}{\mathop{\mathsf{PCodioperads}}\nolimits}

\newcommand{\antish}{\ensuremath{\mbox{!`}}}


\tikzset{->-/.style={decoration={
			markings,
			mark=at position #1 with {\arrow{>}}},postaction={decorate}}}
\tikzset{-w-/.style={decoration={
			markings,
			mark=at position #1 with {\arrow{Stealth[fill=white,scale=1.4]}}},postaction={decorate}}}
\tikzset{->-/.default=0.65}
\tikzset{-w-/.default=0.65}
\tikzstyle{bullet}=[circle,fill=black,inner sep=0.5mm]
\tikzstyle{circ}=[circle,draw=black,fill=white,inner sep=0.5mm]
\tikzstyle{vertex}=[circle,draw=black,thick,inner sep=0.5mm]
\tikzstyle{dot}=[draw,circle,fill=black,minimum size=0.5mm,inner sep = 0mm, outer sep = 0mm]
\tikzset{darrow/.style={double distance = 4pt,>={Implies},->},
	darrowthin/.style={double equal sign distance,>={Implies},->},
	tarrow/.style={-,preaction={draw,darrow}},
	qarrow/.style={preaction={draw,darrow,shorten >=0pt},shorten >=1pt,-,double,double
		distance=0.2pt}}
\newcommand\tikcirc[1][2.5]{\tikz[baseline=-#1]{\draw[thick](0,0)circle[radius=#1mm];}}
\newcommand{\tikzfig}[1]{\begin{tikzpicture}[auto,baseline={([yshift=-.5ex]current bounding box.center)}]#1\end{tikzpicture}}
\usetikzlibrary{decorations.pathreplacing}

\usepackage{amsthm}
\usepackage{thmtools}
\usepackage[noabbrev,capitalize]{cleveref}
\newtheorem{theorem}{Theorem}
\newtheorem{lemma}[theorem]{Lemma}
\newtheorem{proposition}[theorem]{Proposition}
\newtheorem{corollary}[theorem]{Corollary}
\newtheorem{conjecture}[theorem]{Conjecture}
\newtheorem{Claim}[theorem]{Claim}

\newtheorem*{theorem*}{Theorem}

\theoremstyle{definition}
\newtheorem{definition}{Definition}
\newtheorem{example}{Example}
\newtheorem{cexample}{Counterexample}
\newtheorem{exercise}{Exercise}
\newtheorem{question}{Question}
\newtheorem{remark}{Remark}

\addtolength{\textheight}{1.75cm}
\addtolength{\voffset}{-0.5cm}
\addtolength{\footskip}{+0.6cm}
\geometry{margin=1.2in}


\usepackage[backend=biber, maxbibnames=99, bibencoding=utf8, doi=false, isbn=false, url=false, style=alphabetic]{biblatex}
\addbibresource{refs.bib}

    \begin{document}
\fi



\section{Introduction}\label{section:Introduction}

\nsubsection{Notations}

Throughout this paper, let $X = \{x_0,x_1\}$ be a set of letters, $X^*$ the free monoid generated by $X$ and $\ide{h}:= \mathbb{Q}\langle X\rangle$ be free associative $\mathbb{Q}$-algebra generated by $X$.
We define a weight $\wt w$ on $X^*$ by the number of letters in $w\in X^*$.
The weight is extended to $\ide{h}$. 
Let $\ide{h}=\bigoplus_{m\ge0}\ide{h}^{(m)}$ be a direct sum decomposition with respect to weight, where $\ide{h}^{(m)}$ is the $\mathbb{Q}$-linear subspace of homogeneous elements of weight $m$.
We define $\ide{h}^{\vee}=\varprojlim_{n}\,\ide{h}/\bigl(\bigoplus_{m\ge n}\ide{h}^{(m)}\bigr)$.
It is known that, we can consider $\ide{h}^{\vee}$ as a completed free associative $\mathbb{Q}$-algebra $\mathbb{Q}\langle\langle X \rangle\rangle$ generated by $X$.


%\subsection*{Multiple zeta values and double shuffle relations}
\nsubsection{Multiple zeta values and extended double shuffle relations}


For a tuple of integers $(k_1,\ldots,k_r)\in\mathbb{Z}_{>0}^r$ with $k_r>1$, the \textit{multiple zeta value} (MZV) is a real number defined by
\[
\zeta(k_1,\ldots,k_r) := \sum_{0<m_1<\cdots <m_r} \frac{1}{m_1^{k_1}\cdots m_r^{k_r}}.
\]
We set $\zeta(\emptyset) = 1$.
The condition $k_r>1$ ensures the convergence of the above multiple series. Let $\mathcal{Z}$ be the $\mathbb{Q}$-linear space of MZVs.
When $r=1$, $\zeta(k_1)$ coincides with the special values of the Riemann zeta function.
MZVs were first studied by Euler and Goldbach.
They studied double zeta values (the case $r=2$) and proved the $\Q$-linear relations among MZVs which is known as the sum formula nowadays.
Around 1990, Hoffman \cite{Hoffman-92} and Zagier \cite{Zagier-92} rediscovered MZVs and their applications.
Multiple zeta values can be written as iterated integral expressions \cite{Zagier-94} and these expressions allow us to interpret them as the periods of mixed Tate motives (\cite{Deligne-Goncharov}).
Brown \cite{Brown2012} showed that every period of mixed Tate motives over $\mathbb{Z}$ is a $\Q[(\pi i)^{-1}]$-linear combination of MZVs.

The extended double shuffle relations (Definition \ref{df:eds}) are a family of $\mathbb{Q}$-linear relations among MZVs.
These relations arise from the combination of two kinds of product-to-sum relations, where the first one originates from the iterated integral expression, and the second one originates from the multiple series expressions.
By $\mathcal{Z}^f$, we denote the $\Q$-algebra generated by the formal symbols $\zeta^f(k_1,\ldots,k_r)$ ($r\in\mathbb{Z}_{>0}$, $k_1,\ldots,k_r\in\mathbb{Z}_{>0}$) which satisfy the extended double shuffle relations (the $\Q$-algebra $\mathcal{Z}^f$ was first introduced in \cite{Ihara-Kaneko-Zagier} as $R_{\mathrm{EDS}}$). The $\Q$-algebra $\mathcal{Z}^f$ is called the formal multiple zeta spaces, and its generators $\zeta^f(k_1,\ldots,k_r)$ are called the formal multiple zeta values. 
The $\mathbb{Q}$-algebra $\mathcal{Z}^f$ is equipped with the shuffle product (see Section \ref{section:setup}).
Conjecturally, the map $\mathcal{Z}^f\to\mathcal{Z}$ sending $\zeta^f(k_1,\ldots,k_r)$ to $\zeta^{\shuffle}(k_1,\ldots,k_r)$ (for $r\ge1$, $k_i\in\mathbb{Z}_{>0}$, see Section \ref{section:setup} for $\zeta^{\shuffle}(k_1,\ldots,k_r)$. Note that $\zeta^{\shuffle}(k_1,\ldots,k_r)$ is well defined even$k_r=1$).
) is a $\mathbb{Q}$-algebra isomorphism.
 This conjecture suggests that the extended double shuffle relations generate all $\mathbb{Q}$-algebraic relations among MZVs.

Let $\Set$ denote the category of sets, and $\Qalg$ the category of unital, commutative, and associative $\mathbb{Q}$-algebras.
The extended double shuffle relations are conjectured to generate all $\Q$-linear relations among MZVs.  
From the perspective of the extended double shuffle relations, Racinet \cite{Racinet} introduced the affine scheme $\DMR_0 : \Qalg \rightarrow \Set; (R \mapsto \DMR_0(R))$ whose coordinate ring is $\mathcal{Z}^f/\zeta^f(2)\mathcal{Z}^f$. 
Note that we can regard $\DMR_0(R)$ as the set of elements $\Phi\in R\cotimes \ide{h}^{\vee}$ whose coefficients satisfy the extended double shuffle relations.
For instance, let
\[
\Phi_{\shuffle} := \sum_{w \in X^{*}} \zeta^{\shuffle}(w)w = 1 + \zeta^{\shuffle}(2) x_0x_1  - \zeta^{\shuffle}(2) x_1x_0 + \zeta^{\shuffle}(3)x_0x_0x_1 +  \zeta^{\shuffle}(1,2)x_1x_0x_1+\cdots
\]
and consider $\overline{\Phi}_{\shuffle}$ as the image of $\Phi_{\shuffle}$ in $(\mathcal{Z}/\zeta(2)\mathcal{Z}) \cotimes \ide{h}^{\vee}$. Then $\overline{\Phi}_{\shuffle} \in \DMR_0(\mathcal{Z}/\zeta(2) \mathcal{Z})$.
A remarkable property of $\DMR_0$ is that it possesses a natural structure of an affine group scheme, with a group law given by the Ihara product $\circledast$ (see Section \ref{section:review-on-DMR}).
The group $\DMR_0$ is conjectured to be isomorphic to the prounipotent part of the motivic Galois group $U^{\rm{dR}}$ (\cite[Question~~3.31]{Furusho2}).  
\begin{comment}
Indeed, the prounipotent part $U^{\rm{dR}}$ is also expected to be isomorphic to the Grothendieck-Teichm\"{u}ller group $\mathrm{GRT}$. %Belyi (\cite{Belyi}) 
Combining Brown's theorem \cite{Brown2012} and Ihara's theorem \cite{Ihara}, it shows the existence of the embedding $U^{\rm{dR}} \hookrightarrow \mathrm{GRT}_1$.
Furusho \cite{Furusho} showed $\mathrm{GRT}_1 \hookrightarrow \DMR_0$. Hence $U^{\rm{dR}} \hookrightarrow \DMR_0$.
Moreover, the Kashiwara-Vergne group $\mathrm{KRV}$, which arises from the Kashiwara-Vergne conjecture, is also expected to be isomorphic to $U^{\rm{dR}}$ 
\end{comment}

\nsubsection{Adjoint multiple zeta values}


In this paper, we mainly focus on \textit{adjoint multiple zeta values} (AdMZVs) and the $\Q$-linear relations among them, termed as adjoint double shuffle relations.
These concepts were introduced by Jarossay \cite{Jarossay}.
The purpose of this study is to investigate the adjoint double shuffle relations along with the theory of $\DMR_0$.


For $(k_1,\ldots,k_r) \in \mathbb{Z}_{>0}^r$ and $l \in \mathbb{Z}_{>0}$, we define AdMZVs by
\begin{align*}
\Admzv{l}{k_1,\ldots,k_r} :=& \sum_{i = 0}^r (-1)^{k_{i+1}+\cdots+k_r + l} \zeta^{\shuffle}(k_1,\ldots,k_i)\zeta^{\shuffle}_l(k_r,\ldots,k_{i+1}) \mod \zeta(2),
\end{align*}
where
\[
\zeta_{l}^{\shuffle}(k_r,\ldots,k_{i+1}) = (-1)^l \sum_{\substack{l_{i+1}+\cdots+l_r = l \\ l_{i+1},\ldots,l_r\ge 0}}
\left(\prod_{j = i+1}^r \binom{k_j + l_j+1}{l_j}\right)
\zeta^{\shuffle}(k_r+l_r,\ldots,k_{i+1}+l_{i+1}).
\]
%It is known that $\zeta^{\shuffle}_l$ appears in certain coefficients of $\Phi_{\shuffle}$.
For each $\mathbf{k} \in \mathbb{Z}_{>0}^r$ and $l\in \mathbb{Z}_{>0}$, AdMZVs $\Admzv{l}{\mathbf{k}}$ is a generalizations of the \textit{symmetric multiple zeta values} (SMZVs) \cite[Section 4]{Kan1} $\zeta_{\S}(\mathbf{k})$. Indeed, $\zeta_{\S}(\mathbf{k}) = \Admzv{0}{\mathbf{k}}$ holds.


SMZVs are conjectured to share the same $\Q$-linear relations among finite multiple zeta values $\zeta_{\A}(\mathbf{k})$, referred to as the Kaneko-Zagier conjecture.
Let $P$ be the set of prime numbers, and we define a $\Q$-algebra $\A$ by
\[
\A := \left. \prod_{ p\in P} \mathbb{F}_p\middle/ \bigoplus_{ p\in P}\mathbb{F}_p \right.
\]
Here, $\mathbb{F}_p$ is the $p$-th finite field.
This $\Q$-algebra $\A$ is referred to as the ring of integers modulo infinitely large primes, or ``Poor man's ad\`{e}le ring".
For $(k_1,\ldots,k_r) \in \mathbb{Z}_{>0}^r$, we define a \textit{finite multiple zeta value} (FMZV) $\za(k_1,\ldots,k_r)$ by
\[
\za(k_1,\ldots,k_r) = \left( \sum_{0<m_1<\cdots<m_r < p} \frac{1}{m_1^{k_1}\cdots m_r^{k_r}} \mod p \right)_p \in \A.
\]
Let $\mathcal{Z}^{\A}$ be a $\Q$-algebra generated by $1$ and FMZVs.
Kaneko-Zagier conjecture suggests that there is a well-defined $\Q$-algebra homomorphism.
\begin{align}
\mathcal{Z}^{\A} \rightarrow \mathcal{Z}/\zeta(2)\mathcal{Z} ; \zeta_{\A}(k_1,\ldots,k_r) \mapsto \zeta_{\S}(k_1,\ldots,k_r). \label{eq:Kaneko-Zagier}
\end{align}
Yasuda \cite{Yasuda} showed that SMZVs span the $\Q$-linear space $\mathcal{Z}/\zeta(2)\mathcal{Z}$.
This implies that the above $\Q$-algebra homomorphism \eqref{eq:Kaneko-Zagier} is a surjection under the assumption of its well-definedness.
Recently, the ring $\A$, to which FMZVs belong, has attracted attention.
Rosen first developed an $\A$-analog of periods \cite{Rosen}, and later constructed an $\A$-analog of algebraic numbers \cite{Rosen1}.
Subsequently, Kaneko–Matsusaka–Seki proposed an $\A$-analog of Euler's constant
\cite{Kaneko-Matsusaka-Seki}.


Jarossay studied the element $\overline{\Phi}_{\shuffle}^{-1}x_1\overline{\Phi}_{\shuffle} \in (\mathcal{Z} / \zeta(2)\mathcal{Z}) \cotimes \ide{h}^{\vee}$ and conducted extensive research on $\overline{\Phi}_{\shuffle}^{-1}x_1\overline{\Phi}_{\shuffle}$ in several papers (\cite{Jarossay}, \cite{Jarossay2020}) and found some non-trivial properties.
Indeed, Jarossay obtained the $\Q$-linear relations among AdMZVs referred to as the \textit{adjoint double shuffle relations}, by adopting Racinet's dual formulation of the double shuffle relations to $\overline{\Phi}_{\shuffle}^{-1}x_1\overline{\Phi}_{\shuffle}$.
According to these properties of  $\overline{\Phi}_{\shuffle}^{-1}x_1\overline{\Phi}_{\shuffle}$, Jarossay introduced the \textit{adjoint double shuffle affine scheme} whose defining equation is the adjoint double shuffle relations $\AdDMR_0$.
Related to $\AdDMR_0$, he posed the question, \textit{``Are the two affine schemes, $\DMR_0$ and the adjoint double shuffle scheme, actually isomorphic?"} (\cite[Question 3.2.8]{Jarossay}).
By \cite[Proposition 1.3.6 (i)]{Jarossay2015} and \cite[Proposition 3.2.7]{Jarossay}, there is a closed immersion 
\begin{align}
\Ad(x_1) : \DMR_0 \rightarrow \AdDMR_0 \label{eq:Adjoint-map}
\end{align}
 defined by $\DMR_0(R) \rightarrow \AdDMR_0(R) ; \phi \mapsto \phi^{-1}x_1\phi$ for each $R \in \Qalg$.


\begin{comment}
\textcolor{red}{
Conjecture \ref{conj:Jarossay-1} suggests $U^{\mathrm{dR}} \cong \AdDMR_0\times_{\TM_1} \FAd$ through the conjecture $U^{\mathrm{dR}} \cong \DMR_0$.
This suggests that progress on the relations among AdMZVs, FMZVs, and SMZVs can probably contribute to the isomorphism problem for $U^{\mathrm{dR}}$.
Furthermore, we expect that this will provide a starting point for the Formal Kaneko–Zagier conjecture proposed by Bachmann and Risan \cite{Risan}, which concerns these families of $\Q$-linear relations.
%Furthermore, for a unified understanding of various kinds of multiple zeta values.
}
\end{comment}

\nsubsection{Verification, Question, and our result}

The purpose of this paper is to investigate Jarossay's question mentioned above.
However, computational calculation suggests that the answer is likely to be negative.
To assess this indication, we analyze the tangent space $\dmr$ of $\DMR_0(\Q)$ at $1$ and the tangent space $\addmr$ of $\AdDMR_0(\Q)$ at $x_1$.
Then, the map \eqref{eq:Adjoint-map} induces the following embedding map:
\begin{align}
\ad(x_1) : \dmr\rightarrow \addmr ; \psi \mapsto [x_1,\psi].
\end{align}
We shall consider the direct product decomposition of the weight homogeneous of $\dmr$ and $\addmr$, namely,
\begin{align*}
\dmr =& \prod_{k\ge 0} \dmr^{(k)}\\
\addmr =& \prod_{k\ge 0} \addmr^{(k)}.
\end{align*}
Then we observe that $\dim \dmr^{(k)} \neq \dim \addmr^{(k+1)}$ for $k \in \mathbb{Z}_{\ge 0}$ (see Appendix~\ref{section:computational data}).
(Remineded that the adjoint map increases the weight by $1$).


To refine Jarossay's question, we first define the affine scheme $\FAd:\Qalg\rightarrow\Set$ as follows: for $R\in\Qalg$, set
\[
\FAd(R) := \{ \Phi \in R\cotimes \ide{h}^{\vee} \mid  \text{${}^{\exists} \phi \in \ide{h}^{\vee}$ with $\Delta_{\shuffle}(\phi) = \phi \otimes \phi$ such that  $\Phi = \phi^{-1}x_1 \phi$ }  \}.
\]
Here, $\Delta_{\shuffle} : \ide{h}^{\vee} \rightarrow \ide{h}^{\vee} \cotimes \ide{h}^{\vee}$ is a $\Q$-algebraic homomorphism defined by $x_i \mapsto  x_i \otimes 1 + 1\otimes x_i$ for $i = 0$, $1$. 
For each $R \in \Qalg$, the set $\FAd(R)$ is a group under the operation $\circledast_1$ (see Section \ref{sect:AdDMR}) with a unit $x_1$ and this describes that $\FAd$ is an affine group scheme.
Let $\Fad$ denote the tangent space of $\FAd(\Q)$ at $x_1$.
Then $\Fad$ is a Lie algebra with respect to the bracket $\{\mathchar`-,\mathchar`-\}_1$ (see Section \ref{sect:AdDMR}),
and it can be described explicitly by
\[
\Fad = \{ \phi \in R\cotimes \ide{h}^{\vee} \mid \text{${}^{\exists} \phi \in \ide{h}^{\vee}$ with $\Delta_{\shuffle}(\phi) = \phi \otimes 1 + 1\otimes \phi$ such that  $\Phi = [x_1 ,\phi]$ } \}.
\]

We next consider the fiber product (intersection in the sense of affine schemes) $\AdDMR_0 \times_{\TM_1} \FAd$ (for $\TM_1$, see Section \ref{sect:AdDMR}). We can see that the tangent space of $(\AdDMR_0 \times_{\TM_1} \FAd)(\Q)$ at $x_1$ is equal to $\addmr \cap \Fad$.
Considering the weight homogeneous decomposition of $\dmr = \prod_{k\ge 0} \dmr^{(k)}$ and $\addmr \cap \Fad = \prod_{k\ge 0} \addmr^{(k)} \cap \Fad$, we observe $\dim \dmr^{(k)} = \dim \addmr^{(k+1)} \cap \Fad$ up to $k = 10$ (Appendix~\ref{section:computational data}). 
According to this observation, we expect that the following isomorphism of $\Q$-linear space $\dmr \cong \addmr \cap \Fad$ (Question \ref{qu:Jarossay-Lie})  might be true. If this holds, $ \addmr \cap \Fad$ forms a Lie algebra with respect to $\{\mathchar`-,\mathchar`-\}_1$.
Since $\dmr$ and $\addmr \cap \Fad$ are expected to be pronilpotent Lie algebras, the correspondence between
prounipotent affine group schemes and pronilpotent Lie algebras further suggest that the following isomorphism of affine schemes
$\DMR_0 \cong \AdDMR_0 \times_{\TM_1} \FAd$ would be true (Question \ref{qu:Jarossay}). If this holds, $ \AdDMR_0 \times_{\TM_1} \FAd$ forms an affine group scheme with respect to $\circledast_1$.


According to this observation, we obtain partial positive results toward Question \ref{qu:Jarossay-Lie}.
We utilize the following $\Q$-linear space:
\begin{align*}
V_{\strprty}:=& \left\{\Psi \in \ide{h}^{\vee} \middle|
\begin{matrix}
\text{$\langle \Psi \mid w \rangle = 0$ for $w \in X^*$ with $\wt w\le 1$}\\
 \Psi^{11} + \Psi^{10} + \Psi^{01} = 0
\end{matrix}
\right\},
\end{align*}
where we define
\[
\Psi = x_0\,\Psi^{00}\,x_0+ x_0\,\Psi^{01}\,x_1+ x_1\,\Psi^{10}\,x_0+ x_1\,\Psi^{11}\,x_1 .
\]
for $\Psi \in \ide{h}^{\vee}$ such that the coefficients of $1$, $x_0$, and $x_1$ are $0$.
The defining equations of $V_{\strprty}$ are related to explicit parity formulas \cite{Hirose-parity}.
Further details on $V_{\strprty}$ are given in Section \ref{sect:Main Theorem}.
Below, we describe a sketch of our main theorem.
\begin{mt}[simplified version]\label{mt:Lie-str}
An intersection of $\Q$-linear space $\addmr \cap \Fad \cap V_{\strprty}$ forms a Lie algebra equipped with a certain bracket $\{\mathchar`-,\mathchar`-\}_1$ (see Section \ref{sect:AdDMR}).
\end{mt}

It was previously unknown whether the relations among AdMZVs, FMZVs, or SMZVs possess a natural algebraic structure.
Our main theorem provides the first evidence that one can obtain a nontrivial algebraic structure from the $\Q$-linear relations among AdMZVs.



\begin{comment}
\nsubsection{Adjoint multiple zeta values and adjoint double shuffle relations}

Inspired by the Kaneko-Zagier conjecture, the ring $\mathscr{A}$, in which finite multiple zeta values are defined, is expected to be an analogue of $\mathbb{R}$.
Indeed, Rosen \cite{Rosen1} constructed $\mathscr{A}$-analogues of periods and algebraic numbers.
Kaneko, Matsusaka, and Seki \cite{Kaneko-Matsusaka-Seki} proposed an $\mathscr{A}$-analogue of Euler's constant.
The author \cite{Anzawa-Funakura} has also investigated $\mathscr{A}$ and constructed transcendental elements of $\mathscr{A}$ under the generalized Riemann hypothesis (GRH) by using the $q$-Fibonacci sequence.
Recently, Luca and Zudilin \cite{Luca-Zudilin} refined the main theorem of \cite{Anzawa-Funakura}.
They removed the GRH assumption and constructed another transcendental element of $\mathscr{A}$ from Frobenius traces of elliptic curves (\cite{Luca-Zudilin-2}).
\end{comment}






\begin{comment}
\nsubsection{今後の展望}

Our investigation predicts three works.
First is to solve the formal Kaneko-Zagier conjecture.

Second is to determine the $\Q$-linear relations which


\textcolor{red}{ここから書き直し}

\nsubsection{Interpretation of $\mathcal{Z}/\zeta(2)\mathcal{Z}$ as a coordinate ring of some algebraic groups}


The motivic Galois group $G^{\rm{dr}}$ is an affine group scheme over $\Q$ defined as the Tannakian fundamental group of the category of the mixed Tate motives over $\Z$. 
It is known that, via the embedding $G^{\mathrm{dR}}\hookrightarrow \ide{h}^{\vee}$, the group law on $G^{\mathrm{dR}}$ corresponds to the Ihara product $\circledast$ (see Section \ref{section:review-on-DMR}).
Moreover, there is a canonical semidirect product decomposition
\[
G^{\mathrm{dR}} \cong U^{\mathrm{dR}} \rtimes \mathbb{G}_m.
\]
Here, $\mathbb{G}_m$ denotes the multiplicative group scheme. The closed pro-unipotent subgroup scheme $U^{\mathrm{dR}}\subset G^{\mathrm{dR}}$ is called the pro-unipotent part of $G^{\mathrm{dR}}$.
Deligne and Goncharov constructed a motivic lift of MZVs from mixed Tate motives over $\mathbb{Z}$ \cite{Deligne-Goncharov}.
Let $\mathcal{Z}^{\ide{m}}$ denote the $\mathbb{Q}$-linear space spanned by motivic MZVs. We note that $\mathcal{Z}^{\ide{m}}$ is a $\mathbb{Q}$-algebra equipped with the shuffle product, and there is a natural surjection $\mathcal{Z}^{\ide{m}} \twoheadrightarrow \mathcal{Z}$ (conjecturally, a $\mathbb{Q}$-algebra isomorphism).
Then the coordinate ring of $U^{\mathrm{dR}}$ is $\mathcal{Z}^{\ide{m}}/(\zeta^{\ide{m}}(2))$, where $\zeta^{\ide{m}}(2)\in\mathcal{Z}^{\ide{m}}$ denotes the motivic element sent to $\zeta(2)$ by the natural surjection $\mathcal{Z}^{\ide{m}}\twoheadrightarrow\mathcal{Z}$.
Therefore, $\mathcal{Z}^{\ide{m}}/\zeta^{\ide{m}}(2)\mathcal{Z}^{\ide{m}}$ possesses the structure of Hopf algebra and $\mathcal{Z}^{\ide{m}}$ possesses the structure of the ($\mathcal{Z}^{\ide{m}}/\zeta^{\ide{m}}(2)\mathcal{Z}^{\ide{m}}$)-Hopf comodule.
The coaction on $\mathcal{Z}^{\ide{m}}$, known as the Goncharov--Brown coaction, plays a key role in Brown's proof that every period of mixed Tate motives over $\mathbb{Z}$ is generated, over $\mathbb{Q}[(2\pi i)^{-1}]$, by a specific set of motivic MZVs \cite{Brown2012}, \cite{Goncharov2}.


Let $\Set$ denote the category of sets, and $\Qalg$ the category of unital, commutative, and associative $\mathbb{Q}$-algebras.
The extended double shuffle relations are conjectured to generate all $\Q$-linear relations among MZVs.  
From the perspective of the extended double shuffle relations, Racinet (\cite{Racinet}) introduced the affine scheme $\DMR_0 : \Qalg \rightarrow \Set; (R \mapsto \DMR_0(R))$ whose coordinate ring is $\mathcal{Z}^f/\zeta^f(2)\mathcal{Z}^f$. 
Note that we can regard $\DMR_0(R)$ as the set of elements $\Phi\in R\cotimes \ide{h}^{\vee}$ whose coefficients satisfy the extended double shuffle relations.
For instance, let
\[
\Phi_{\shuffle} := \sum_{w \in X^{*}} \zeta^{\shuffle}(w) = 1 + \zeta^{\shuffle}(2) x_0x_1  - \zeta^{\shuffle}(2) x_1x_0 + \zeta^{\shuffle}(3)x_0x_0x_1 +  \zeta^{\shuffle}(1,2)x_1x_0x_1+\cdots
\]
and consider $\overline{\Phi}_{\shuffle}$ as the image of $\Phi_{\shuffle}$ in $(\mathcal{Z}/\zeta(2)\mathcal{Z}) \cotimes \ide{h}^{\vee}$. Then $\overline{\Phi}_{\shuffle} \in \DMR_0(\mathcal{Z}/\zeta(2) \mathcal{Z})$.
A remarkable property of $\DMR_0$ is that it possesses a natural structure of an affine group scheme, with group law given by the Ihara product $\circledast$ (see Section \ref{section:review-on-DMR}).
The group $\DMR_0$ is conjectured to be isomorphic to the motivic Galois group $U^{\rm{dr}}$.
Indeed, the pro-unipotent part $U^{\rm{dr}}$ is also expected to be isomorphic to the Grothendieck-Teichm\"{u}ller group $\mathrm{GRT}$. %Belyi (\cite{Belyi}) 
Combining Brown's theorem \cite{Brown2012} and Ihara's theorem \cite{Ihara} shows the existence of the embedding $U^{\rm{dr}} \hookrightarrow \mathrm{GRT}_1$.
Furusho (\cite{Furusho}) showed $\mathrm{GRT}_1 \hookrightarrow \DMR_0$. Hence $U^{\rm{dr}} \hookrightarrow \DMR_0$.
Moreover, the Kashiwara-Vergne group $\mathrm{KRV}$, which arises from the Kashiwara-Vergne conjecture, is also predicted to be isomorphic to $U^{\rm{dr}}$ (\cite[Question 3.31]{Furusho2}).  

%\subsection*{$\mathcal{Z}/\zeta(2)\mathcal{Z}$ and Kaneko--Zagier conjecture}

\nsubsection{$\mathcal{Z}/\zeta(2)\mathcal{Z}$ and Kaneko-Zagier conjecture}

In connection with the quotient $\mathbb{Q}$-algebra $\mathcal{Z}/\zeta(2) \mathcal{Z}$, the Kaneko--Zagier conjecture has attracted attention since around 2010. 
Let $t$ be an indeterminate.
For $l\in\mathbb{Z}_{>0}$, the \textbf{(generalized) Kaneko--Zagier conjecture} (\cite{Kan1}) predicts the existence of a well-defined $\mathbb{Q}$-algebra isomorphism such that
\begin{align}
\begin{array}{ccc}
\mathcal{Z}^{\mathscr{A}_l} &\longrightarrow &  \mathcal{Z}/\zeta(2) \mathcal{Z} \otimes \mathbb{Q}[t]/(t^l)  \\
\rotatebox{90}{$\in$}&& \rotatebox{90}{$\in$}\\
\zeta_{\mathscr{A}_l} ( \mathbf{k})  & \longmapsto & \zeta_{\S_l}(\mathbf{k})\\
\mathbf{p}& \longmapsto &t
\end{array}
\label{eq:KZ-conj}
\end{align}
Here, given $(k_1,\ldots,k_r)\in\mathbb{Z}_{>0}^r$ and $l \in \mathbb{Z}_{> 0}$, $l$-\textbf{finite multiple zeta values} ($l$-FMZVs) are defined by
\[
\zeta_{\A_l}(k_1,\ldots,k_r) := \left( \sum_{0<m_1<\cdots<m_r<p} \frac{1}{m_1^{k_1}\cdots m_r^{k_r}} \mod p^l \right)_p \in \A_l,
\]
where
\[
\mathscr{A}_l := \prod_{p\text{ : prime}} \mathbb{Z}/p^l \mathbb{Z} \Bigm/ \bigoplus_{p \text{ : prime}} \mathbb{Z}/p^l\mathbb{Z}.
\]
We define $\mathcal{Z}^{\A_l}$ as the $\Q$-algebra generated by $l$-FMZVs and $\mathbf{p}:= (2,3,5,7,\ldots) \in \A$.

The counterparts of FMZVs, in the context of the Kaneko-Zagier conjecture \eqref{eq:KZ-conj}, are $l$-\textbf{symmetric multiple zeta values} ($l$-SMZVs) given by 
\begin{align*}
\zeta_{\S_l}(k_1,\ldots,k_r) := \sum_{0\le l' < l} \Admzv{l'}{k_1,\ldots,k_r} \otimes t^{l'}
\end{align*}
Here, for $l\in\mathbb{Z}_{>0}$, we define \textbf{adjoint multiple zeta values} (AdMZVs), $\Admzv{l}{k_1,\ldots,k_r}$ by
\begin{align*}
\Admzv{l}{k_1,\ldots,k_r} :=& \sum_{i = 0}^r (-1)^{k_{i+1}+\cdots+k_r + l} \zeta^{\shuffle}(k_1,\ldots,k_i)\zeta^{\shuffle}_l(k_r,\ldots,k_{i+1}) \mod \zeta(2),
\end{align*}
where
\[
\zeta_{l}^{\shuffle}(k_r,\ldots,k_{i+1}) = (-1)^l \sum_{\substack{l_{i+1}+\cdots+l_r = l \\ l_{i+1},\ldots,l_r\ge 0}}
\left(\prod_{j = i+1}^r \binom{k_j + l_j+1}{l_j}\right)
\zeta^{\shuffle}(k_r+l_r,\ldots,k_{i+1}+l_{i+1}).
\]
It is known that $\zeta^{\shuffle}_l$ appears in certain coefficients of $\Phi_{\shuffle}$.
If $l = 1$, we simply call $l$-FMZVs (respectively, $l$-SMZVs) as FMZVs (respectively, SMZVs).

These concepts were first introduced by Kaneko and Zagier (\cite{Kan1}) for $l = 1$ and by Rosen (\cite{Rosen}) for $l \ge 2$.
Thus, according to the Kaneko--Zagier conjecture~\eqref{eq:KZ-conj}, it is expected that studying congruences for the truncation of MZVs can contribute to the research on MZVs.
This motivates the study of FMZVs and SMZVs. 
In particular, FMZVs and SMZVs have been extensively studied by many researchers  (\cite{Hirose}, \cite{Rosen}, \cite{BTT}, etc). 
Indeed, Yasuda (\cite{Yasuda}) proved that SMZVs span $\mathcal{Z}$.
If \eqref{eq:KZ-conj} is an well-defined $\Q$-algebra homomorphism, then Yasuda's theorem implies that \eqref{eq:KZ-conj} is surjective.
Besides, while the definition of SMZVs is defined in terms of MZVs, Akagi--Hirose--Yasuda conjectured that FMZVs admit explicit expressions in terms of $p$-adic MZVs; this was proved by Jarossay \cite{Jarossay2020}. Building on Jarossay’s theorem, Rosen \cite{Rosen} constructed a motivic version of multiple harmonic sums.
The ring $\mathscr{A}_1$, in which finite multiple zeta values are defined, is motivated by the Kaneko--Zagier conjecture and is expected to serve as a finite analogue of $\mathbb{R}$.
Indeed, Rosen constructed $\mathscr{A}_1$-analogues of periods and of algebraic numbers \cite{Rosen1}.
Kaneko, Matsusaka, and Seki proposed an $\mathscr{A}_1$-analogue of Euler's constant \cite{Kaneko-Matsusaka-Seki}.
The author has also investigated the structure of $\mathscr{A}_1$. By using the $q$-Fibonacci sequence, he constructed transcendental elements of $\mathscr{A}_1$ under GRH \cite{Anzawa-Funakura}.
Recently, Luca and Zudilin extended the main theorem of \cite{Anzawa-Funakura}.
They removed the GRH assumption and constructed another transcendental element of $\mathscr{A}_1$ from Frobenius traces of elliptic curves (\cite{Luca-Zudilin},\cite{Luca-Zudilin-2}).

\nsubsection{The adjoint double shuffle relations}


Jarossay studied the occurrence of AdMZVs\footnote{In Jarossay's terminology, he called them adjoint MZVs. However, the name ``SMZVs" is spread and similar to many researchers. So we rename them.} in the coefficients of $\overline{\Phi}_{\shuffle}^{-1}x_1\overline{\Phi}_{\shuffle}$.
He conducted extensive research on AdMZVs in several papers (\cite{Jarossay}, \cite{Jarossay2020}) and found some non-trivial conditions.
One of his results was the proposal of the adjoint double shuffle relations.
The conditions dual to the extended double shuffle relations can be phrased as two coproduct conditions for $\overline{\Phi}._{\shuffle}$
Based on this, he derived two properties of $\overline{\Phi}_{\shuffle}^{-1}x_1\overline{\Phi}_{\shuffle}$ from the dual conditions of the extended double shuffle relations.
From these, he deduced two identities of $\overline{\Phi}_{\shuffle}^{-1}x_1\overline{\Phi}_{\shuffle}$, and the found $\mathbb{Q}$-linear relations among AdMZVs are called the adjoint double shuffle relations.
Jarossay also proposes the \textbf{adjoint double shuffle affine scheme}, which arises from the adjoint double shuffle relations, and he posed the questions, \textit{Are the two affine schemes, $\DMR_0$ and the adjoint double shuffle scheme, actually isomorphic ?}
Computational calculation suggests that this question is likely not true. However, imposing a natural assumption, termed the adjoint condition, we establish the conjecture from Jarossay's conjecture. 
Let $\FAd$ denote the algebraic group defined by the adjoint condition.

\begin{conj}\label{conj:Jarossay-1}
There is a natural isomorphism $\Ad(x_1) : \DMR_0 \Longrightarrow \AdDMR_0 \times_{\TM_1} \FAd$ ; $(\Ad(x_1)^R : \DMR_0(R) \rightarrow \AdDMR_0(R) \cap \Fad(R) ; \phi \mapsto \phi^{-1}x_1\phi$.
\end{conj}



In this paper, we report a partial and positive result about this conjecture.


\nsubsection{Results}

To consider Jarossay's question, we focus on a tangent space $\addmr$ of $\AdDMR_0(\Q)$ at the unit because the tangential space of an affine group scheme equips a structure of Lie algebra.
For our main theorems, we utilize the following $\Q$-linear spaces
\begin{align*}
V_{\strprty}:=& \left\{\Psi \in \ide{h}^{\vee} \middle|
\begin{matrix}
 \wt \Psi \ge 2 \\
 \Psi^{11} + \Psi^{10} + \Psi^{01} = 0
\end{matrix}
\right\}.
\end{align*}
The details of this space are in Section \ref{sect:Main Theorem}.
%Related to these two $\Q$-linear spaces, we construct affine group schemes $\exp(\Fad)$ and $\exp(V_{\strprty})$.
Then, we obtain the following main theorem:
\begin{mt}\label{mt:Lie-str}
Let $\Fad$ be a corresponding Lie algebra of $\FAd$. Then the following $\addmr \cap \Fad \cap V_{\strprty}$ forms a Lie algebra with respect to a specific operation.
\end{mt}

\end{comment}


\ifdefined\isMainDocument
\else
    %\bibliographystyle{amsplain}
    %\bibliography{reference-adjoint}
    \end{document}
\fi