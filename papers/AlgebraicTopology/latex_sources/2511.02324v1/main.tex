\documentclass[11pt, a4paper]{article}

\usepackage[utf8]{inputenc}
\usepackage[english]{babel}
\usepackage[T1]{fontenc}

\usepackage{amsmath}
\usepackage{amssymb}
\usepackage{amsthm}
\usepackage{mathabx}
\usepackage{tikz}
\usepackage{tikz-cd}
\usepackage{todonotes}
\usepackage{hyperref}
\usepackage{cleveref}
\usepackage{footmisc}
\usepackage{bbm}
\usepackage{afterpage}
\usepackage{mathrsfs}
\usepackage{sseq}
\usepackage{spectralsequences}
\usepackage{extarrows}
\usepackage{enumitem}
\usepackage{stmaryrd}
\usepackage{pdflscape}
%\usepackage[inline]{showlabels}

%\usepackage{biblatex}
%\addbibresource{textdokument2.bib}

\newcommand{\llpar}{\llparenthesis}
\newcommand{\rrpar}{\rrparenthesis}

%adjectives
\newcommand{\ab}{\mathrm{ab}}
\newcommand{\cn}{\mathrm{cn}}
\newcommand{\Ell}{\mathrm{Ell}}
\newcommand{\cl}{\mathrm{cl}}
\newcommand{\et}{\mathrm{\acute{e}t}}
\newcommand{\ori}{\mathrm{or}}
\newcommand{\ord}{\mathrm{ord}}
\newcommand{\un}{\mathrm{un}}

\DeclareMathOperator{\Spec}{Spec}
\DeclareMathOperator{\MF}{MF}
\DeclareMathOperator{\mf}{mf}

%spectra
\newcommand{\1}{\mathbf{1}}
\newcommand{\BP}{\mathrm{BP}}
\newcommand{\ELL}{{{E}{\ell\ell}}}
\renewcommand{\L}{\mathrm{L}}
\DeclareMathOperator{\KU}{KU}
\DeclareMathOperator{\K}{K}
\DeclareMathOperator{\ku}{ku}
\DeclareMathOperator{\KO}{KO}
\DeclareMathOperator{\ko}{ko}
\DeclareMathOperator{\MMF}{MMF}
\DeclareMathOperator{\mmf}{mmf}
\DeclareMathOperator{\Mmf}{Mmf}
\DeclareMathOperator{\MU}{MU}
\DeclareMathOperator{\MString}{MString}
\newcommand{\Sph}{\mathbf{S}}
\DeclareMathOperator{\Sq}{Sq}
\DeclareMathOperator{\TMF}{TMF}
\DeclareMathOperator{\SMF}{SMF}
\DeclareMathOperator{\tmf}{tmf}
\DeclareMathOperator{\Tmf}{Tmf}

%categories
\DeclareMathOperator{\Ab}{Ab}
\DeclareMathOperator{\Aff}{Aff}
\DeclareMathOperator{\CAlg}{CAlg}
\DeclareMathOperator{\CRing}{CRing}
\DeclareMathOperator{\DM}{DM}
\DeclareMathOperator{\Fun}{Fun}
\DeclareMathOperator{\Glo}{Glo}
\DeclareMathOperator{\Isog}{Isog}
\newcommand{\ISOG}{\mathcal{I}\mathrm{sog}}
\DeclareMathOperator{\Mod}{Mod}
\DeclareMathOperator{\Lie}{Lie}
\newcommand{\Lur}{\mathrm{Lur}}
\newcommand{\OS}{\mathcal{OS}}
\newcommand{\SO}{\mathrm{SO}}
\DeclareMathOperator{\PreAb}{PreAb}
\DeclareMathOperator{\PreBT}{PreBT}
\DeclareMathOperator{\PreEll}{PreEll}
\DeclareMathOperator{\QCoh}{QCoh}
\DeclareMathOperator{\Span}{Span}
\DeclareMathOperator{\Shv}{Shv}
\DeclareMathOperator{\Stk}{Stk}
\DeclareMathOperator{\Sp}{Sp}
\DeclareMathOperator{\SpDM}{SpDM}
\newcommand{\Spc}{\mathcal{S}}
\DeclareMathOperator{\Syn}{Syn}

%categorical stuff
\DeclareMathOperator{\Aut}{Aut}
\DeclareMathOperator{\coker}{coker}
\newcommand{\colim}{\mathrm{colim}\,}
\DeclareMathOperator{\Ext}{Ext}
\DeclareMathOperator{\fib}{fib}
\DeclareMathOperator{\eq}{eq}
\newcommand{\id}{\mathrm{id}}
\DeclareMathOperator{\Map}{Map}
\newcommand{\MAP}{\underline{\mathrm{Map}}}
\DeclareMathOperator{\Hom}{Hom}
\newcommand{\HOM}{\underline{\mathrm{Hom}}}
\newcommand{\op}{\mathrm{op}}

%single letters
\newcommand{\A}{\mathsf{A}}
\newcommand{\B}{\mathfrak{B}}
\newcommand{\bG}{\mathfrak{B}G}
\newcommand{\C}{\mathcal{C}}
\newcommand{\D}{\mathcal{D}}
\newcommand{\E}{\mathbf{E}}
\newcommand{\EE}{\mathscr{E}}
\newcommand{\F}{\mathbf{F}}
\newcommand{\FF}{\mathsf{F}}
\newcommand{\G}{\mathbf{G}}
\newcommand{\h}{\mathrm{h}}
\renewcommand{\H}{\mathcal{H}}
\newcommand{\J}{\mathrm{J}}
\renewcommand{\j}{\mathrm{j}}
\newcommand{\M}{\mathsf{M}}
\newcommand{\N}{\mathbf{N}}
\newcommand{\NN}{\mathcal{N}}
\renewcommand{\O}{\mathbf{O}}
\newcommand{\OO}{\mathbf{O}}
\renewcommand{\P}{\mathbf{P}}
\newcommand{\Q}{\mathsf{Q}}
\newcommand{\QQQ}{\mathfrak{Q}}
\newcommand{\QQ}{\mathcal{Q}}
\newcommand{\R}{\mathbf{R}}
\newcommand{\s}{\mathsf{S}}
\newcommand{\T}{\mathrm{T}}
\renewcommand{\u}{\mathrm{u}}
\newcommand{\x}{\mathsf{X}}
\newcommand{\xx}{\mathsf{X}}
\newcommand{\y}{\mathsf{Y}}
\newcommand{\Z}{\mathbf{Z}}
\newcommand{\z}{\mathsf{Z}}

%greek letters
\newcommand{\al}{\alpha}
\newcommand{\be}{\beta}
\newcommand{\ga}{\gamma}
\newcommand{\Ga}{\Gamma}
\newcommand{\eps}{\varepsilon}

\newcommand{\ueq}{\overset{\circ}{=}}
\newcommand{\uequiv}{\overset{\circ}{\equiv}}
\newcommand{\qeq}{\overset{?}{=}}

%proof triangles and squares
\newenvironment{proofclaim}[1]
{\emph{#1.}}
{\hfill $\triangle$\break}

%ceiling function
\usepackage{mathtools}
\DeclarePairedDelimiter{\ceil}{\lceil}{\rceil}

\theoremstyle{theorem}\numberwithin{equation}{section}
\newtheorem{theorem}[equation]{Theorem}
\crefname{theorem}{{th}.\!\!}{{ths}.\!\!}
\Crefname{theorem}{{Th}.\!\!}{{Ths}.\!\!}
%\newtheorem*{theoremnonum}[equation]{Theorem}
\newtheorem{theoremalph}{Theorem}
\renewcommand*{\thetheoremalph}{\Alph{theoremalph}}
\newtheorem{coralph}[theoremalph]{Corollary}
\renewcommand*{\thecoralph}{\Alph{theoremalph}}
\crefname{theoremalph}{{th}.\!\!}{{ths}.\!\!}
\Crefname{theoremalph}{{Th}.\!\!}{{Ths}.\!\!}
\newtheorem{problem}[equation]{Problem}
\Crefname{problem}{{Prb}.\!\!}{{Prbs}.\!\!}
\newtheorem{prop}[equation]{Proposition}
\Crefname{prop}{{Pr}.\!\!}{{Prs}.\!\!}
\newtheorem{lemma}[equation]{Lemma}
\Crefname{lemma}{{Lm}.\!\!}{{Lms}.\!\!}
\newtheorem{cor}[equation]{Corollary}
\Crefname{cor}{{Cor}.\!\!}{{Cors}.\!\!}
\newtheorem{conjecture}[equation]{Conjecture}
\Crefname{conjecture}{{Conj}.\!\!}{{Conjs}.\!\!}
\newtheorem*{conjecturemaeda}{Maeda's conjecture}

\theoremstyle{definition}\numberwithin{equation}{section}
\newtheorem{mydef}[equation]{Definition}
\Crefname{mydef}{{Df}.\!\!}{{Dfs}.\!\!}
\newtheorem{variant}[equation]{Variant}
\Crefname{variant}{{Var}.\!\!}{{Vars}.\!\!}
\newtheorem{recall}[equation]{Recollection}
\Crefname{recall}{{Rcl}.\!\!}{{Rcls}.\!\!}
\newtheorem{construction}[equation]{Construction}
\Crefname{construction}{{Con}.\!\!}{{Cons}.\!\!}
\newtheorem{ass}[equation]{Assumptions}
\Crefname{ass}{{As}.\!\!}{{As}.\!\!}
\newtheorem{question}[equation]{Question}
\Crefname{question}{{Q}.\!\!}{{Qs}.\!\!}
\newtheorem{notation}[equation]{Notation}
\Crefname{notation}{{Nt}.\!\!}{{Nts}.\!\!}
\newtheorem{situation}[equation]{Situation}
\Crefname{situation}{{St}.\!\!}{{Sts}.\!\!}

\theoremstyle{remark}\numberwithin{equation}{section}
\newtheorem{example}[equation]{Example}
\Crefname{example}{{Ex}.\!\!}{{Exs}.\!\!}
\newtheorem{nonexample}[equation]{Non-example}
\Crefname{nonexample}{{NonEx}.\!\!}{{NonEx}.\!\!}
\newtheorem{claim}[equation]{Claim}
\Crefname{claim}{{Clm}.\!\!}{{Clms}.\!\!}
\newtheorem{remark}[equation]{Remark}
\Crefname{remark}{{Rmk}.\!\!}{{Rmks}.\!\!}
\newtheorem{idea}[equation]{Idea}
\Crefname{idea}{{Id}.\!\!}{{Ids}.\!\!}
\newtheorem{warn}[equation]{Warning}
\Crefname{warn}{{Warn}.\!\!}{{Warns}.\!\!}

\Crefname{figure}{{Fig.}\!\!}{{Figs.}\!\!}
\Crefname{footnote}{{Fn.}\!\!}{{Fn.}\!\!}

\Crefname{part}{{\textsection}\!\!}{{\textsection}\!\!}
\Crefname{chapter}{{\textsection}\!\!}{{\textsection}\!\!}
\Crefname{section}{{\textsection}\!\!}{{\textsection}\!\!}
\Crefname{subsection}{{\textsection}\!\!}{{\textsection}\!\!}
\Crefname{appendix}{{\textsection}\!\!}{{\textsection}\!\!}

%\setlength\parindent{0pt}
\hoffset -16truemm
\oddsidemargin=18truemm
\evensidemargin=18truemm
\textwidth=157truemm

\newcommand{\jdnote}[1]{\todo[color=pink!40,linecolor=blue!40!black,size=\tiny]{#1}}

\newcommand{\ccnote}[1]{\todo[color=green!40,linecolor=green!40!black,size=\tiny]{#1}}

\begin{document}
\title{
Revisiting the $\be_1$-action on the\\ 
    $3$-primary stable homotopy groups of spheres
}
\author{Jack Morgan Davies\footnote{\href{mailto:davies@uni-wuppertal.de}{\texttt{davies@uni-wuppertal.de}}}}
\date{November 4, 2025}
\maketitle

\begin{abstract}
    Let $\be_1$ be the first $3$-torsion class in the stable homotopy groups of spheres in even degree. Toda showed that $\be_1^5 \neq 0$, whilst $\be_1^6 = 0$. Shimomura generalised this to the $144$-periodic family generated by $\be_1$, written as $\{\be_{1+9s}\}_{s\geq 0}$, and showed that any $5$-fold product $\prod_5 \be_{1+9s} \neq 0$, whilst all $6$-fold products $\prod_6 \be_{1+9s} = 0$. In this article, we give a simple proof of these results as well as some generalisations to other $144$-periodic families. Our tools include $\BP$-synthetic spectra, and the well-known Adams--Novikov spectral sequence for the spectrum of topological modular forms at the prime $3$ as well as its Adams operations.
\end{abstract}

\setcounter{tocdepth}{1}
\tableofcontents
%\newpage

%%%%%%%%%%%%%%%%%%%%%%%%%%%%%%%%%%%%%%%%%%%%%%%%%%%%%%%%%%%%%%%
%%%%%%%%%%%%%%%%%%%%%%%%%%%%%%%%%%%%%%%%%%%%%%%%%%%%%%%%%%%%%%%
%%%%%%%%%%%%%%%%%%%%%%%%%%%%%%%%%%%%%%%%%%%%%%%%%%%%%%%%%%%%%%%
\section{Introduction}
One of the cornerstones of algebraic topology is the stable homotopy groups of spheres $\pi_n \Sph = \colim \pi_{n+k} S^k$. These abelian groups assemble into a graded ring $\pi_\ast \Sph = \bigoplus \pi_n \Sph$ by composing and suspending maps between spheres. These rich algebraic structures are very complicated---we only know the order of the abelian groups $\pi_n \Sph$ for roughly $n\leq 90$, for example; see \cite{greenbook,iwxpisph}. In this article, we are interested in a small slice of the multiplicative structure of $\pi_\ast \Sph$. First, we work at the prime $3$, so we implicitly invert all other primes---everything will be implicitly $3$-localised \emph{from now on.} Secondly, we are interested in how the particular element $\be_1$, a generator of $\pi_{10} \Sph \simeq \F_3$ and the smallest even dimensional class in $\pi_\ast\Sph$, acts on $\pi_\ast \Sph$ by multiplication. In short, we show that for a host of known nonzero classes in $\pi_\ast \Sph$, multiplication by $\be_1$ is zero, and for another related collection of nonzero classes, multiplication by $\be_1$ is nonzero. These statements would be interesting even if we knew the precise structure of the groups $\pi_n \Sph$ in those degrees containing these products, but our results are obtained without this information. %To state these results more precisely, we should introduce these other nonzero classes.

The classes we are interested in here arise from \emph{chromatic homotopy theory}; a methodology for filtering each group $\pi_n \Sph$ by \emph{chromatic height}. For example, the only group $\pi_n \Sph$ with nonzero height $0$ information is $\pi_0 \Sph \simeq \Z$, a consequence of Hopf's computation $\pi_n S^n \simeq \Z$ in \cite{hopfpin} and Serre's finiteness theorem \cite{serreclasses}. Adams' work on the $J$-homomorphism and topological $K$-theory \cite{adamsjofx} computes all of the height $1$ information in $\pi_n \Sph$ for all $n$. Although there has been much work on computations at higher heights, perhaps most notably by Miller--Ravenel--Wilson \cite{MRW77} and Goerss--Henn--Mahowald--Rezk \cite{ktwolocalsphere}, height $2$ remains the edge of our knowledge.

The class $\be_1$ can alternatively be defined as the first class of height $2$ in $\pi_\ast \Sph$. The machine of chromatic homotopy theory leads to many other classes of height $2$, the so-called \emph{divided $\beta$-family}. For us, these are classes $\be_{i/j} \in \pi_{16i-4j-2} \Sph$ for various values of $i$ and $j$, and we write $\be_i = \be_{i/1}$. More specifically, works of Behrens--Pemmaraju \cite{marksatya} and Belmont--Shimomura \cite{tmfthree} show the existence of the families of $3$-torsion elements in \Cref{tab:elements} that we focus on here.
\begin{table}[h]
\centering
\begin{tabular}{|l|c|c|c|c|c|c|c|}\hline
Degree mod 144  & 10            & 26            & 37                    & 74            & 81 & 82                & 109                   \\ \hline
Families        & $\be_{1+9s}$  & $\be_{2+9s}$  & $[\al_1\be_{3+9s/3}]$ & $\be_{5+9s}$  & $\langle \al_1, \al_1, \be_{5+9s}\rangle$&$\be_{6+9s/3}$    & $[\al_1\be_{7+9s}]$   \\ \hline
\end{tabular}
\caption{Main characters in the divided $\be$-family used here.}
\label{tab:elements}
\end{table}

The Toda brackets in the $81$-column above are not necessarily a single element, but rather a coset of $\pi_{81+144s} \Sph$. When we say that this coset does not vanish, we mean that it does not contain zero.

We can now state our first theorem concerning the action of any element in the family $\{\be_{1+9s}\}_{s\geq 0}$ on various other elements from \Cref{tab:elements}.

\begin{theoremalph}\label{main:equivalences}
    Let $s_i,t\geq 0$ be a collection of nonnegative integers. Then
    \begin{enumerate}
        \item $\prod_{i=1}^A \be_{1+9s_i} \neq 0$ if and only if $A \leq 5$,
        \item $\be_{2+9t}\prod_{i=1}^B \be_{1+9s_i} \neq 0$ if and only if $B \leq 2$, 
        \item $\al_1\be_{2+9t}\prod_{i=1}^C \be_{1+9s_i} \neq 0$ if and only if $C \leq 2$, and
        \item $[\al_1\be_{3/3}]\prod_{i=1}^D \be_{1+9s_i} \neq 0$ if and only if $D \leq 1$.
    \end{enumerate}
\end{theoremalph}

This result is very much inspired by Shimomura's work \cite{shimbetaoneactionatthree}, which states versions of parts 1 and 2 above. Parts 2 and 3 are related, in that the vanishing statement of 2 implies that of 3, and the nonvanishing of 3 implies that of 2. Each parts of \Cref{main:equivalences} is really two statements, one concerning the vanishing of a large product, and the other concerning the nonvanishing of a smaller product. To prove the vanishing half, we periodify classical differentials in the \emph{Adams--Novikov spectral sequence} (ANSS) for $\Sph$ killing products in low degrees, to obtain differentials killing the periodified products in higher degree. On the other hand, our nonvanishing arguments use Hopkins' spectrum of \emph{topological modular forms} $\TMF$ and operations thereupon to detect this nonvanishing. Both of these halves admit further extensions to the (non)vanishing of other families of products, although the results are not as sharp as \Cref{main:equivalences}; for example, we know that $\al_1 \prod_{i=1}^N \be_{1+9s_i}$ is nonzero for $N\leq 2$ and vanishes for $N\geq 4$, but our techniques do not apply to $N=3$. For this reason, we record some generalised vanishing and nonvanishing statements separately. 

The first we state as a corollary, as it follows immediately by adapting the proof of \Cref{main:equivalences} without any extra computations.

\begin{coralph}\label{main:vanishing}
    Let $s_i,t\geq 0$ be a collection of nonnegative integers. Then the following products of elements in $\pi_\ast\Sph$ all vanish:
    \[\al_1 \prod_{i=1}^4 \be_{1+9s_i}, \qquad \be_{5+9t} \prod_{i=1}^4 \be_{1+9s_i}, \qquad \al_1\be_{6/3} \prod_{i=1}^3 \be_{1+9s_i}\]
\end{coralph}

For our generalised nonvanishing statement, recall that $\TMF$ has an endomorphism $\psi^2$, a kind of \emph{Adams operation}, and that we define $\J^2$ for the equaliser of $\psi^2$ and the identity acting on $\TMF$; a kind of ``height $2$ image-of-$J$ spectrum''. By definition, we have maps of ring spectra $\Sph \to \J^2 \to \TMF$.

\begin{theoremalph}\label{main:nonvanishing}
    Let $s_i,t,w\geq 0$ and $x_i \in \{\be_{1+9s}, \be_{6+9s/3}\}_{s\geq 0} \subseteq \pi_{\ast} \Sph$. Then the classes
    \begin{equation}\label{eq:line1}\prod_{i=1}^5 x_i, \qquad \be_{2+9t}\prod_{i=1}^2 x_i, \qquad \be_{5+9t}\prod_{i=1}^2 x_i,\end{equation}
    \begin{equation}\label{eq:line2}\al_1\be_{2+9t}\prod_{i=1}^2 \be_{1+9s_i}, \quad \al_1\be_{2+9t}\prod_{i=1}^2 \be_{6+9s_i/3}, \quad
    [\al_1\be_{7+9t}]\prod_{i=1}^2 \be_{1+9s_i}, \quad
    [\al_1\be_{7+9t}]\prod_{i=1}^2 \be_{6+9s_i/3},\end{equation}
    \begin{equation}\label{eq:line3}\langle \al_1, \al_1, \be_{5+9t}\rangle \be_{6+9w/9} \prod_{i=1}^3 \be_{1+9s_i}, \qquad \langle \al_1, \al_1, \be_{5+9t}\rangle \be_{1+9w} \prod_{i=1}^3 \be_{6+9s_i/3}\end{equation}
    which all vanish in $\J^{2}$ and $\TMF$, do not vanish in $\pi_\ast\Sph$.
\end{theoremalph}

This last statement complements our results of \cite{heighttwojat3} together with Carrick. Indeed, one of the main ingredients used here is the \emph{Hurewicz image} of $\J^2$, computed in \cite[Th.A]{heighttwojat3}, and we do not include any nonvanishing products already described in \cite[Th.B]{heighttwojat3}.

We obtain all of these results using \emph{$\BP$-synthetic spectra} à la Pstragowski \cite{syntheticspectra}. For our vanishing statements, this is mostly an aesthetic choice. Similar arguments can be made explicitly using the ANSS, although our use of synthetic spectra avoids the use of any subtle geometric boundary theorems (such as \cite[Th.2.3.4]{greenbook}). For the nonvanishing statements, the flexibility granted by synthetic spectra, especially in giving us a \emph{modified ANSS} for $\J^2$, is incredibly useful. Computing this modified ANSS is simple; see \Cref{pr:manssforJ}. The nonvanishing results above are then a corollary of this modified ANSS, the Hurewicz image of $\J^2$ studied in \cite{heighttwojat3}, and some filtration arguments.

%\subsection*{Further directions}
Working with only $\TMF$ and $\J^{2}$, as well as their synthetic versions, has its limitations though. For example, $\be_2^2=0$ in $\J^{2}$, see \Cref{rmk:boundaryissquarezero}, so we cannot recover the fact that $\be_2^2\neq0$ in $\Sph$; see \cite[Tab.A3.4]{greenbook}. The same goes for $\be_5^2$. Moreover, we know that $\al_1\be_1^2\be_2\neq0$ in $\Sph$, again by \cite[Tab.A3.4]{greenbook}, but it is not clear to us if one can deduce this from the modified ANSS for $\J^2$. In particular, $\al_1\be_{2+9t}\prod_{i=1}^3 \be_{1+9s_i}$ is clearly $0$ by \Cref{main:equivalences}, but we do not know if $\al_1 \be_{2+9t} \prod_{i=1}^2 \be_{1+9s_i}$ vanishes or not in general; see \Cref{question:products}.

The $Q(N)$-spectra of Behrens' are a further refinement of $\TMF$ and $\J^{2}$, living even closer to $\Sph$; see \cite{buildingandec}. We hope to come back to a discussion of the obvious synthetic version of $Q(N)$, as in \cite[Rmk.2.19]{heighttwojat3}, in the future, utilising either $Q(2)$ or $Q(7)$ at the prime $3$; the former has been used to great success by Behrens' \cite{ktwospheremark}, and the latter would build upon the computations of $\pi_\ast \TMF_0(7)$ of Meier--Ozornova \cite{meieroroztmfo7}. %The latter might be more useful for synthetic Hurewicz image computations as $\TMF_0(7)$ has $3$-torsion in its homotopy groups, whereas $\TMF_0(2)$ does not.
This story is also interesting at the prime $2$, as shown in \cite{v232families}, and also at primes $p\geq 5$, as shown in \cite{dividedbetafamily}. An optimist might also hope to access information in $\pi_\ast \Sph$ at heights $h\geq 3$ using higher real $K$-theories, perhaps using a model in synthetic spectra via \cite[Th.4.3]{osyn} or \cite{syntheticslicey}, or the \emph{topological automorphic forms} spectra of Behrens--Lawson \cite{taf}. 

We also hope that the results of this article could help us understand the homotopy groups of $\1[\be_1^{-1}]$, the synthetic spectrum $\1$ associated with the ANSS for the sphere $\Sph$ with the class $\be_1$-inverted. Of course, Nishida's nilpotence theorem (or \Cref{main:equivalences}) tells us that classically $\Sph[\be_1^{-1}]=0$, but the synthetic spectrum $\1[\be_1^{-1}]$ is not zero, and captures $\be_1$-periodic behaviour in the ANSS for $\Sph$. This should be the first step in a series of \emph{exotic periodicities}. At the prime $2$, the $\eta$-inverted synthetic sphere $\1[\eta^{-1}]$ has been computed by Andrews--Miller \cite{invertingeta}. Further work of Andrews \cite{andrewsw1}, Gheorghe \cite{exoticktheories}, and recently by Isaksen--Kong--Li--Ruan--Zhu \cite{w1periodicityisaksenetal} interpret $\eta=w_0$ as the first in a sequence of periodicities $w_n$, and explicitly compute with the ``height $1$ exotic periodicity'' $w_1$. Given our current knowledge of this $\be_1$-action on $\1$, a computation of $\pi_{\ast,\ast}\1[\be_1^{-1}]$ may be within reach. %For example, $\1[\be_1^{-1}]$ has a $v_2^9$-periodicity given by $\be_{10}/\be_1 \in \pi_{144,0} \1[\be_1^{-1}]$, and up to the $90$-stem, it seems as though the generators on bigraded homotopy groups are given by
%\[\al_1, \quad \be_2, \quad [\al_1\be_{3/3}], \quad \be_5, \quad x_{75}, \quad x_{81}, \quad \ga_2, \quad \be_{6/3}%, \quad \be_7\pm \be_{9/9}, \quad [\al_1\be_7] \]
%as an $\F_3[\tau, \be_1^{\pm}]/\tau^8$-module. Some of the relations among these generators can be determined by looking at the ANSS and the results of this article.

%%%%%%%%%%%%%%%%%%%%%%%%%%%%%%%%%%%%%%%%%%%%%%%%%%%%%%%%%%%%%%%
\subsection*{Outline}
In \Cref{sec:moorespectra}, we prove the vanishing half of \Cref{main:equivalences} and \Cref{main:vanishing}. To do this, we take differentials in the ANSS and produce $v_2^9$-periodic versions using the language of self maps of synthetic Moore spectra. In \Cref{sec:tmfsection}, we review some basic facts of $3$-local topological modular forms and $\J^{2}$, including a description of a modified ANSS for $\J^2$. In \Cref{sec:detection}, we prove the nonvanishing half of \Cref{main:equivalences} and \Cref{main:nonvanishing} using this modified ANSS.

%%%%%%%%%%%%%%%%%%%%%%%%%%%%%%%%%%%%%%%%%%%%%%%%%%%%%%%%%%%%%%%
\subsection*{Notation}
As a reminder, everything is implicitly localised at the prime $3$.

We use the notation and foundational facts concerning \emph{$\BP$-synthetic spectra} $\Syn$ of \cite{syntheticspectra}; everything we will use is also laid out in \cite{syntheticsven} in the language of filtered spectra. In particular, that $\Syn$ is a stable symmetric monoidal $\infty$-category equipped with a lax monoidal functor $\nu\colon \Sp \to \Syn$ called the \emph{synthetic analogue}, which is a section to the strong monoidal localisation $\tau^{-1}\colon \Syn \to \Sp$ called \emph{$\tau$-inversion} (\cite[\textsection4]{syntheticspectra}). We write $\sigma\colon \Syn \to \Fun(\Z^\op, \Sp)$ for the lax monoidal functor to \emph{filtered spectra} called the \emph{signature} (\cite[Not.2.5]{smfcomputation}), and implicitly use the fact that the signature of $\nu X$ is the ($3$-local) ANSS for $X$ (\cite[Pr.1.25]{osyn}). Following \cite[\textsection2.2]{syntheticj}, we will call $\sigma(X)$ the spectral sequence associated with $X$ and also the (chosen) modified Adams--Novikov spectral sequence for $\tau^{-1}X$. Moreover, we follow the \emph{stem--filtration} grading for $\Syn$, meaning $\pi_{s,f}$ corresponds to a $(s,f)$-location in an AN-chart. Formally, we write $\Sigma^{s,f} \1 = \Sigma^{-f}\nu\Sph^{s+f}$ where $\1$ is the unit of $\Syn$. In particular, the element $\tau$ lives in $\pi_{0,-1} \1$ and the $\infty$-categorical suspension has bidegree $(1,-1)$.

We also implicitly use the fact that the ANSS for $\1$ and $\TMF$ are both concentrated in those bidegrees $(s,f)$ with $4|s+f$; this is the usual sparsity in the ANSS at $p=3$. We will also use explicit computations in these ANSSs, for which we refer the reader to \cite{greenbook} (also see the chart made available by Belmont \href{https://github.com/ebelmont/ANSS_data/raw/master/anss_E2_158.pdf}{here}) and \cite[Fig.A.1]{smfcomputation}, respectively. We also fix synthetic lifts of the divided $\beta$-family elements of \Cref{tab:elements}, except for the family $\be_{1+9s}$ for which we fix a specified synthetic lift in \Cref{sec:moorespectra}. In particular, these synthetic lifts have mod $\tau$ reduction given by Ravenel's Greek letter construction of \cite[Def.1.3.19]{greenbook}.

%%%%%%%%%%%%%%%%%%%%%%%%%%%%%%%%%%%%%%%%%%%%%%%%%%%%%%%%%%%%%%%
\subsection*{Acknowledgements}
Thank you to Adela Zhang and Gabriel Angelini-Knoll for their invitations to visit Copenhagen and Paris, respectively, and for providing the opportunity for these ideas to come together. Thank you also to Christian Carrick for our sustained and enjoyable collaboration which led me in this direction and comments on a draft, and to Gabriel, William Balderrama, Eva Belmont, Lennart Meier, Sven van Nigtevecht, Liz Tatum, and Jana Wehlburg for some helpful conversations. This article was partially written while I was at the Max-Planck Institute for Mathematics in Bonn, and was conducted in the framework of the DFG-funded research training group GRK 2240: Algebro-Geometric Methods in Algebra, Arithmetic and Topology.

%%%%%%%%%%%%%%%%%%%%%%%%%%%%%%%%%%%%%%%%%%%%%%%%%%%%%%%%%%%%%%%
%%%%%%%%%%%%%%%%%%%%%%%%%%%%%%%%%%%%%%%%%%%%%%%%%%%%%%%%%%%%%%%
%%%%%%%%%%%%%%%%%%%%%%%%%%%%%%%%%%%%%%%%%%%%%%%%%%%%%%%%%%%%%%%
\section{Vanishing of products in periodic families}\label{sec:moorespectra}
The goal of this section is to obtain the upper bounds from \Cref{main:equivalences}, so to obtain the vanishing statements from \Cref{main:equivalences}, and to prove \Cref{main:vanishing}. To do this, we show that chosen synthetic lifts of these elements to $\1$ are $\tau$-power torsion, hence are hit by differentials in the ANSS for $\Sph$, and hence vanish in $\pi_\ast \Sph$. We begin by defining the synthetic divided $\beta$-family $\be_{1+9s}$; this is merely bookkeeping. Then we perform some computations in the bigraded homotopy groups of the synthetic Moore spectrum $\1/(3,v_1)$, which will then give us the desired upper bound.

Start with the mod $3$ Moore spectrum $\Sph/3$. This has a $v_1$-self map $v_1\colon \Sph^4/3 \to \Sph/3$ by \cite{adamsjofx}; also see \cite[Th.6.3]{syntheticj}. Behrens--Pemmaraju \cite{marksatya} show that the cofibre of this map $\Sph/(3,v_1)$ have $v_2^9$-self maps of degree $144$. These self maps immediately produces self maps of synthetic Moore spectra by applying the synthetic analogue functor
\[v_1\colon \1^{4,0}/3 \to \1/3, \qquad\qquad v_2^9 \colon \1^{144,0}/(3,v_1) \to \1/(3,v_1);\]
where we have identified $\1/3 \simeq \nu(\Sph/3)$ and $\1/(3,v_1) \simeq \nu(\Sph/(3,v_1))$ courtesy of \cite[Lm.4.23]{syntheticspectra}. Let us also use the following notation for the maps in the cofibre sequences
\[\1 \xrightarrow{3} \1 \xrightarrow{q_0} \1/3 \xrightarrow{\partial_0} \1^{1,-1}, \qquad \1^{4,0}/3 \xrightarrow{v_1} \1/3 \xrightarrow{q_{1}} \1/(3,v_1) \xrightarrow{\partial_1} \1^{5,-1}/3.\]
It is now simple to define a synthetic lift of the divided $\be$-family $\be_{1+9s}$.

\begin{mydef}
    Writing $v_1 \in \pi_{16,0} \1/(3,v_1)$ for a generator, for each $s\geq 0$ we define $\be_{1+9s} \in \pi_{10+144s,2} \1$ as $\partial_0\partial_1(v_2^{9s}\circ v_1)$.
\end{mydef}

Fixing synthetic lifts for the other families of \Cref{tab:elements} is also not hard; one just repeats the constructions of \cite{marksatya,tmfthree} synthetically. As we will not need specific synthetic lifts here, we leave the particular choice of such a lift up to the reader.

Now we can start with our vanishing results. The proof outline for the lemmata that follow is the same. Using \Cref{lm:60} as an example, to show that $\tau^8 \be_1^5 v_2 = 0$ in $\pi_{66,2} \1/(3,v_1)$, we use the fact that $\tau^8 \be_1^6 = 0$ in $\1$, and show that multiplication by $\be_1$ on $\pi_{66,2} \1/(3,v_1)$ is injective, which boils down to the $4$-lemma. These kinds of arguments also can be made directly on the level of spectral sequences, but since we will need the synthetic language for \Cref{sec:tmfsection,sec:detection}, we have chosen to introduce it already.

\begin{lemma}\label{lm:60}
    The class $\tau^8 \be_1^5 v_2 \in \pi_{66,2} \1/(3,v_1)$ vanishes.%\footnote{Notice that $\tau^4 \be_1^5 v_2 \neq 0$. If it were, then $\tau^4\be_1^6 = 0$ in $\1$ which is not true; this fact can also be detected in $\TMF$, see part 3 of \Cref{pr:nonvanishingbeta1beta63}.}
\end{lemma}

\begin{proof}
    As $\tau^8 \be_1^6 = 0$ in $\1$, we have $\tau^8 \be_1^6 v_2 = 0$ in $\pi_{76,4}\1/(3,v_1)$. It then suffices to see that multiplication by $\be_1$ is injective on $\pi_{66,2} \1/(3,v_1)$ to conclude that $\tau^8 \be_1^5 v_2$ vanishes. The cofibre sequence defining $\1/(3,v_1)$ induces a commutative diagram of abelian groups
    \begin{equation}\label{eq:5lemma}\begin{tikzcd}
        {0}\ar[r]	&	{(\pi_{s,f} \1/3))/v_1}\ar[r, "{q_1}"]\ar[d, "{\cdot \be_1}"] &   {\pi_{s,f} \1/(3,v_1)}\ar[r, "{\partial_1}"]\ar[d, "{\cdot \beta_1}"]  &   {(\pi_{s-5,f+1} \1/3)[v_1]}\ar[d, "{\cdot \be_1}"]\ar[r]	&	{0}\\
        {0}\ar[r]	&	{(\pi_{s+10,f+2} \1/3)/v_1}\ar[r, "{q_1}"] &   {\pi_{s+10,f+2} \1/(3,v_1)}\ar[r, "{\partial_1}"]   &   {(\pi_{s+9,f+3})[v_1]}\ar[r]	&	{0}
    \end{tikzcd}\end{equation}
    with short exact rows; the notation on the left-hand side indicates the $v_1$-quotient on homotopy groups and the notation on the right-hand side is the $v_1$-torsion taken on homotopy groups. It suffices to show that the left-hand and the right-hand vertical maps are injective for $(s,f)=(66,2)$. For this pair, the left-hand map is an injection as $(\pi_{66,2} \1/3)/v_1=0$. Indeed, $\pi_{66,2}\1=0$ and we have $\pi_{65,3}\1/3 \simeq \F_3$ generated by a lift $x$ of $\tau^4\al_1\be_1\be_2^2$ through the boundary map in the exact sequence
    \[0=\pi_{66,2}\1 \to \pi_{66,2} \1/3 \xrightarrow{\partial_0} \pi_{65,3} \1 \to 0.\]
	However, by \cite[Lm.2.6]{heighttwojat3}, we have $\partial_0(v_1 \cdot q_0(f)) = \al_1 f$ for any element $f\in \pi_{\ast,\ast} X$ of a synthetic spectrum $X$. In particular, we see that
	\[\partial_0( v_1 \cdot q_0(\tau^4\al_1\be_1\be_2^2)) = \tau^4\al_1\be_1\be_2^2 = \partial_0(x),\]
	which as $\partial_0$ is injective in this degree, shows that this lift $x$ is divisible by $v_1$, hence it vanishes in $(\pi_{66,2} \1/3)/v_1$. In particular, the left-hand vertical map in (\ref{eq:5lemma}) is injective. For the right-hand vertical map of (\ref{eq:5lemma}) with $(s,f) = (66,2)$, we immediately obtain injectivity as $\pi_{61,3} \1/3 = 0$ as $\pi_{61,3}\1$ and $\pi_{60,4}\1$ both vanish.
\end{proof}

The following two lemmata are proven with even simpler instances of the same technique.

\begin{lemma}\label{lm:56}
    The class $\tau^4 \be_2\be_1^2 v_2 \in \pi_{62,2} \1/(3,v_1)$ vanishes.
\end{lemma}

%\begin{proof}
%    Similar to \Cref{lm:60}, it suffices to prove that multiplication by $\be_1$ is injective on $\pi_{62,2} \1/(3,v_1)$, which follows by inspection of (\ref{eq:5lemma}) for $(s,f)=(62,2)$.
%\end{proof}

\begin{lemma}\label{lm:57}
    The class $\tau^4 \be_1 [\al_1\be_{3/3}] v_2 \in \pi_{63,1} \1/(3,v_1)$ vanishes.
\end{lemma}

%\begin{proof}
%    Similar to \Cref{lm:60,lm:56}, we want to show that multiplication by $\be_1$ is injective on $\pi_{63,1} \1/(3,v_1)$. We will not directly be able to use (\ref{eq:5lemma}), but the following simple alteration where we force the long exact rows to be short exact sequences in the usual way:
%    \[\begin{tikzcd}
%        {0}\ar[r]   &   {\coker(v_1\colon \pi_{59,1} \1/3\to \pi_{63,1} \1/3)}\ar[r, "{q_1}"]\ar[d, "{\cdot \be_1}"] &   {\pi_{63,1} \1/(3,v_1)}\ar[r, "{\partial_1}"]\ar[d, "{\cdot \be_1}"]  &   {\pi_{58,2} \1/3}\ar[d, "{\cdot \be_1}"]  \\
%        {0}\ar[r]   &   {\coker(v_1\colon \pi_{62,2} \1/3 \to \pi_{66,2} \1/3)}\ar[r, "{q_1}"] &   {\pi_{66,2} \1/(3,v_1)}\ar[r, "{\partial_1}"]   &   {\pi_{61,3} \1/3}
%    \end{tikzcd}\]
%\end{proof}

\begin{remark}\label{rmk:tautologicalvanishing}
    Inside $\pi_{\ast,\ast}\1$ the products $\tau^4\al_1\be_1^3$, $\tau^4 \be_5 \be_1^3$, and $\tau^4\al_1 \be_{6/3} \be_1^2$ all vanish. We have not been able to show that either $\tau^4 \al_1 \be_1^2 v_2\in \pi_{39,1} \1/(3,v_1)$, $\tau^4 \be_1^2 \be_5 v_2 \in \pi_{110,2} \1/(3,v_1)$, nor $\tau^4\al_1\be_{6/3}\be_1 v_2\in \pi_{111,1}\1/(3,v_1)$ vanish, meaning that we cannot conclude that the associated classical differentials periodify. However, it is tautological from the vanishing of the above products in $\1$, that the associated products in $\1/(3,v_1)$ vanish after multiplication by $\be_1$. This is precisely what gives \Cref{main:vanishing}.
\end{remark}

A simple argument turns the lemmata above into vanishing statements in $\1$ and $\Sph$.

\begin{cor}\label{cor:vanishinghalf}
	For $s_i,t\geq 0$, the products
	\begin{equation}\label{eq:vanishinglineone}\prod_{i=1}^6 \be_{1+9s_i}, \qquad \be_{2+9t}\prod_{i=1}^3 \be_{1+9s_i},
    \qquad [\al_1 \be_{3/3}]\prod_{i=1}^2 \be_{1+9s_i}\end{equation}
    \begin{equation}\label{eq:vanishinglinetwo}\al_1 \prod_{i=1}^4 \be_{1+9s_i}, \qquad \be_{5+9t} \prod_{i=1}^4 \be_{1+9s_i}, \qquad \al_1\be_{6/3} \prod_{i=1}^3 \be_{1+9s_i}\end{equation}
	are all $\tau^4$-torsion, except for the first $6$-fold product, which is $\tau^8$-torsion, in $\pi_{\ast,\ast}\1$. In particular, all of the associated products in $\pi_\ast \Sph$ vanish.
\end{cor}

\begin{proof}
    Let us start with the family of classes $\prod_{i=1}^6 \be_{1+9s_i}$. Consider the synthetic product $\be_1^5 \be_{1+9s} \in \pi_{60+144s,12} \1$ for some $s\geq 0$. This class is defined as $\be_1^5 \partial_0\partial_1(v_2^{9s}\circ v_2)$. Using the $\1$-linearity of $\partial_0\partial_1$ and $v_2^{9s}$, we obtain
    \[\tau^8 \be_1^5 \partial_0\partial_1(v_2^{9s}\circ v_2) = \partial_0\partial_1(v_2^{9s}\circ(\tau^8 \be_1^5 v_2)) = 0,\]
    the last equality courtesy of \Cref{lm:60}. We will see in \Cref{pr:nonvanishingbeta1beta63}, which is proven independently, that these classes $\be_1^5 \be_{1+9s}$ are \textbf{not} $\tau^4$-torsion, so we see that the mod $\tau$-reduction of these classes are hit by $d_9$-differentials in the ANSS for $\Sph$. Using Toda's result that for any $s+t = u+v$ we have
    \[uv\be_s \be_t = st \be_u \be_v \in \pi_{\ast,\ast} \1/\tau,\]
    see \cite[Th.5.3]{todabetaproducts} for the original statement or \cite[Th.5.6.5]{greenbook} for this precise form, we see that
    \[\be_1^5 \be_{1+9s} = \prod_{i=1}^6 \be_{1+9s_i} \in \pi_{\ast,\ast} \1/\tau\]
    as long as $s = \sum_{i=1}^6 s_i$. In particular, we see that every product of the form $\prod_{i=1}^6 \be_{1+9s_i}$ has a mod $\tau$-reduction hit by a $d_9$, hence these products are also $\tau^8$-torsion. Inverting $\tau$ then shows that products of the form $\prod_{i=1}^6 \be_{1+9s_i}$ vanish in $\pi_\ast\Sph$. The proofs of the other two families of (\ref{eq:vanishinglineone}) of products are the same, referring to \Cref{lm:56} and \Cref{lm:57} when necessary. For the families of (\ref{eq:vanishinglinetwo}), we refer to \Cref{rmk:tautologicalvanishing},
\end{proof}

\begin{proof}[Proof of \Cref{main:vanishing}]
    This is just the vanishing of (\ref{eq:vanishinglinetwo}) from \Cref{cor:vanishinghalf}.
\end{proof}

By the nonvanishing results to come, the products in (\ref{eq:vanishinglineone}) do not vanish if we divide them by $\be_1$, however, what follows does not answer this question for those products in (\ref{eq:vanishinglinetwo}).

\begin{question}\label{question:products}
	Do the products of (\ref{eq:vanishinglinetwo}) vanish in $\pi_\ast\Sph$ when one factor of $\be_{1+9s}$ is removed?
\end{question}

%The first product should be approachable, proving that $\tau^4\al_1 \be_1^2 v_2$ in $\pi_{39,1} \1/(3,v_1)$ vanishes, but the second might require an extension of Ravenel's computations from \cite{greenbook}.
%\begin{remark}
%    Shimomura also proves that the first two families of products in \Cref{pr:upperbounds} also vanish using explicit computations in the $L_2$-local ANSS. We cannot quite understand his final step though, where he lifts his differentials from the $L_2$-local spectral sequence to the $3$-local spectral sequence, ie, it is not clear to us that these equalities of differentials also hold modulo the kernel of the $L_2$-localisation map.
%\end{remark}

%%%%%%%%%%%%%%%%%%%%%%%%%%%%%%%%%%%%%%%%%%%%%%%%%%%%%%%%%%%%%%%
%%%%%%%%%%%%%%%%%%%%%%%%%%%%%%%%%%%%%%%%%%%%%%%%%%%%%%%%%%%%%%%
%%%%%%%%%%%%%%%%%%%%%%%%%%%%%%%%%%%%%%%%%%%%%%%%%%%%%%%%%%%%%%%
\section{The modified Adams--Novikov spectral sequence for $\J^2$}\label{sec:tmfsection}
In this section, we give an exposition of the $\E_\infty$-ring $\J^2$ and its associated modified Adams--Novikov spectral sequence.

We start with Hopkins' spectrum $\TMF$ of \emph{topological modular forms}, defined as the global sections of the spectral moduli stack $\M_\Ell^\ori$ of \emph{oriented elliptic curves}; see \cite[\textsection7]{ec2} for this definition and \cite{uniqueotop} for the comparison to the Goerss--Hopkins--Miller model of \cite{tmfbook}. Hopkins--Mahowald computed the homotopy groups of $\TMF$ using its ANSS, see \cite[Cor.E]{smfcomputation} or \cite[\textsection5.8]{lennartlecturenotes} for details. We will then use $\nu\TMF$ as our synthetic lift of $\TMF$, and denote it by $\TMF_\BP$. By \cite[Th.C]{osyn}, this synthetic spectrum can be identified with \emph{synthetic modular forms} $\SMF$, which naturally captures the \emph{descent spectral sequence} for $\TMF$. In fact, this result identifies this spectral sequence with the ANSS for $\TMF$. We will implicitly use this identification in this section by identifying the $E_2$-page of the ANSS for $\TMF$ in terms of the cohomology of the classical stack $\M_\Ell$.

The computation of the ANSS for $\TMF$ can be encoded synthetically as follows: its $E_2$-page is given by the mod $\tau$ homotopy groups of $\TMF_\BP$
\begin{equation}\label{eq:modtauhomotopyoftmf}
\pi_{\ast,\ast} \TMF_\BP/\tau \simeq \frac{\Z[c_4, c_6, \Delta^{\pm}, \al, \be]}{(c_4^3-c_6^2 - 24^3\Delta, 3\al, \al^2, \al c_4, \al c_6, 3\be, \be c_4, \be c_6)}
\end{equation}
generated by elements of bidegree
\[|c_i|=(2i,0),\quad |\Delta|=(24,0),\quad |\al|=(3,1),\quad |\be| = (10,2);\]
the elements $c_4$ and $c_6$ correspond to the normalised Eisenstein series and $\Delta$ to the discriminant modular form. The elements $\al$ and $\be$ also have algebraic descriptions: in the exact sequence
\[H^0(\M_\Ell, \omega^{\otimes 2}) \to H^0(\M_{\Ell,\F_3}, \omega^{\otimes 2}) \to H^1(\M_\Ell, \omega^{\otimes 2}) \simeq \F_3\{\al\},\]
the class $\al$ is the obstruction to lifting the mod $3$ modular form $b_2$ of weight $2$ to an integral modular form, see \cite[Pr.7.2(II)]{delignecohomologyMELL}, and $\be$ is the Massey product $\langle \al, \al, \al\rangle$. There are only $d_5$'s and $d_9$'s in this spectral sequence \cite[\textsection7.1]{smfcomputation}, multiplicatively generated by
\[d_5(\Delta) = \pm \al\be^2,\qquad d_9([\al\Delta]) = \pm \be^5.\]
This then leads to the bigraded homotopy groups of $\TMF_\BP$ having generators
\[\pi_{\ast,\ast} \nu\TMF \simeq \Z[\tau, c_4, c_6, [3\Delta], [c_4\Delta], [c_6\Delta], [3\Delta^2], [c_4\Delta^2], [c_6\Delta^2], \Delta^{\pm 3}, \al, \be, [\al\Delta]] / I,\]
where $I$ is an ideal which is not worth writing down here; it captures the facts that upon inverting $\tau$ we have $\pi_\ast\TMF$ and upon killing $\tau$ we have (\ref{eq:modtauhomotopyoftmf}). Again, we suggest that the reader refers to 
%\[c_4^3-c_6^2 - 8\cdot 24^2 [3\Delta], \quad [3\Delta]^2 - 3[3\Delta^2], \quad [3\Delta][3\Delta^2] - 9\Delta^3, \quad [3\Delta^2]^2 - 3[3\Delta]\Delta^3\]
%\[3\al, \quad \al^2, \quad \al c_4,\quad \al c_6,\quad \al[3\Delta],\quad \al[3\Delta^2]\] (include those for $[c_4\Delta]$ etc?)
%\[c_4 [\al\Delta], \quad c_6[\al\Delta], \quad [3\Delta][\al\Delta], \quad [3\Delta^2][\al\Delta]\]
%\[3\be, \quad \be c_4, \quad\be c_6, \quad\be[3\Delta], \quad\be[3\Delta^2], \quad \tau^4 \al \be^2,\quad \tau^4 \be^2 [\al\Delta],\quad \tau^8 \be^5, \quad\al[\al\Delta] - \tau^4 \be^3;\]
\cite[Fig.A1]{smfcomputation} and invert $\Delta^3$ for a picture.

Now we move onto Adams operations on $\TMF$.

\begin{mydef}\label{def:operationsonTMF}
    For an integer $3\nmid k$, let $\psi^k\colon \TMF \to \TMF$ be the \emph{$k$th Adams operation}, the maps of $\E_\infty$-rings of \cite[Df.2.1]{heckeontmf}. We also write $\psi^k\colon \TMF_\BP \to \TMF_\BP$ for the map of synthetic $\E_\infty$-rings by applying $\nu$ to $\psi^k$.
\end{mydef}

\begin{remark}\label{rmk:homotopymodtaucomputation}
By \cite[Cor.2.12]{heckeontmf}, we can compute these operations on $\TMF_\BP$ modulo $\tau$ as the corresponding algebraic operations. In particular, if $f\in \pi_{2d,0} \TMF_\BP/\tau \simeq \MF_d$ is a modular form of weight $d$, then we have $\psi^k(f) = k^d f$. One can also compute $\psi^k(\al) = \al$ and $\psi^k(\be) = \be$ as in $\TMF$ these classes lie in the image of the unit map $\Sph \to \TMF$. As these classes generate $\pi_{\ast,\ast}\TMF_\BP/\tau$ as a ring, see (\ref{eq:modtauhomotopyoftmf}), this gives us a computations of $\psi^k$ on the bigraded homotopy groups of $\TMF_\BP/\tau$ as $\psi^k$ is multiplicative. 
\end{remark}

These kinds of computation of Adams operations on elements in filtration $0$ is true in general \cite[Pr.6.18]{luriestheorem}, and computations of $\psi^k$ on torsion elements in $\pi_\ast \TMF$ at the prime $2$ is discussed in \cite{adamsontmf}.

In \cite{heighttwojat3}, together with Carrick, we studied the equaliser of $\psi^2$ and the identity acting on $\TMF$, as well as the appropriate synthetic equaliser.

\begin{mydef}
    Let $k$ be an integer. Write $\J^{2}$ for the $\E_\infty$-ring given as equaliser of the two $\E_\infty$-endomorphisms $\psi^2,\id$ of $\TMF$, or equivalently, the $\E_0$-ring given by the fibre of $\psi^2-1$. Write $\J^{2}_\BP$ for the synthetic $\E_\infty$-ring given as the equaliser of the two maps $\nu\psi^2,\id$ on $\TMF_\BP$. In both cases, we write
    \[\partial \colon \Sigma^{-1}\TMF \to \J^{2}, \qquad \partial \colon \Sigma^{-1,1}\TMF_\BP \to \J_\BP^{2}\]
    for the boundary map of their defining fibre sequences.
\end{mydef}

The ``$\J$''-notation hearkens back to Whitehead's \emph{$J$-homomorphism} and its relation to Adams operations on topological $K$-theory discussed in \cite{adamsjofx}.

\begin{figure}[h]\begin{center}
\makebox[\textwidth]{\includegraphics[trim={3.5cm 15.5cm 2.5cm 4.2cm},clip,page = 1, scale = 1]{spectralsequences}}
\caption{\label{ss1} Signature spectral sequence of $\J^{2}_\BP$ for stems $s$ in $0 \leq s \leq 64$; see \Cref{pr:manssforJ}.}
\end{center}\end{figure}

\begin{figure}[h]\begin{center}
\makebox[\textwidth]{\includegraphics[trim={3.5cm 15.5cm 2.5cm 4.2cm},clip,page = 2, scale = 1]{spectralsequences}}
\caption{\label{ss2} Signature spectral sequence of $\J^{2}_\BP$ for stems $70\leq 134$; see \Cref{pr:manssforJ}.}
\end{center}\end{figure}

The Adams operation $\psi^2$ is used for simplicity here; most of this article also works for any $\psi^k$ with $3\nmid k$.

\begin{remark}\label{rmk:boundaryissquarezero}
    The product structure on $\J^{k}$ can be made reasonably explicit. The Bousfield--Kan filtration of $\J^{k}$, placing the source $\TMF$ of $\psi^k$ and $\id$ in filtration $0$ and the target $\TMF$ in filtration $1$, is a multiplicative filtration. In particular, two elements in filtration $1$ multiply to elements of filtration $\geq 2$, which is trivial. Specifically, given $x,y\in \pi_\ast\TMF$, we have $\partial(x)\partial(y)=0$ in $\pi_\ast\J^{2}$. The same goes for $\J^{2}_\BP$ using a Bousfield--Kan spectral sequence in synthetic spectra.
\end{remark}

The following is a nonconnective version of \cite[Pr.2.12]{heighttwojat3}.

\begin{prop}
    The synthetic spectrum $\J^{2}_\BP$ is a $\tau$-complete and synthetic lift of $\J^{2}$.
\end{prop}

\begin{proof}
    As $\TMF$ is $\MU$-nilpotent complete, the $\tau$-completeness follows from \cite[Pr.A.13]{burkhahnseng} as $\tau$-complete synthetic spectra are closed under limits. The fact that $\tau$-inversion is exact and is a retract of $\nu$ shows that these synthetic spectra are lifts of the indicated spectra.
\end{proof}

As $\J^{2}_\BP$ is a synthetic lift of $\J^{2}$, its associated signature spectral sequence converges to the homotopy groups of $\J^{2}$. Moreover, the $E_2$-page of this spectral sequence is readily computable from the $E_2$-page of the DSS for $\TMF$. For degree reasons, there are also not many differentials or extensions; this was essentially used in \cite{heighttwojat3}.

\begin{prop}\label{pr:manssforJ}
    The signature spectral sequence associated with $\J^{2}$ has differentials determined by the fibre sequence
    \[\Sigma^{-1,1}\TMF_\BP \xrightarrow{\partial} \J^{2}_\BP \xrightarrow{p} \TMF_\BP;\]
    see \Cref{ss1,ss2}. There is one exotic $3$-extension from $[\partial(\Delta^{1+3s})]$ to $[\al_1\be_1^2\Delta^{3s}]$ for each $s\geq 0$. In these charts, blue signifies classes lifted from the ANSS for $\TMF$, and red and yellow mean the image of the ANSS for $\TMF$ along the boundary map; the yellow classes follow an image-of-$J$ pattern. Red and blue dots are copies of $\F_3$, the blue rectangle is a $\Z[j]$ for $j=\frac{c_4^3}{\Delta}$, and yellow circles are isomorphic to $\Z[j]/(2^{r}-1)$, where $r=\tfrac{s+1}{2}$ for a yellow dot in the $s$-stem. Lines indicate multiplication by $\al_1$ or $\be_1$; multiplication lines with yellow source are omitted.
\end{prop}

In other words, once we use $\partial$ and $p$ to induce as many differentials as possible, there is no room for any other differentials. This signature spectral sequence also gives the structure of $\pi_{\ast,\ast} \J^{2}_\BP$ by the usual omnibus theorem; see \cite[\textsection A]{burkhahnseng} or \cite[Th.3.26]{syntheticsven}.

\begin{proof}
    For the $E_2$-page, one uses the computation of $\psi^2$ on $\pi_{\ast,\ast}\TMF_\BP/\tau$ described in \Cref{rmk:homotopymodtaucomputation}. The differentials are clear, and once one projects and lifts as many differentials as are in $\TMF_\BP$, there is no more room for any more differentials in $\J^2_\BP$. The exotic $3$-extension comes from the fact that if we compute $\pi_{23+72s}\J^{2}$, then all of the generators can be taken to be simple $9$-torsion, the image of $\pi_{24+72s}\TMF$. This forces these extensions in the spectral sequence.
\end{proof}

The final piece of information we need about $\J^2$ is the image of $\pi_{\ast,\ast}\1$ in its homotopy groups, so the \emph{synthetic Hurewicz image} of $\J^2_\BP$. This follows from \cite{tmfthree,heighttwojat3}.

\begin{prop}\label{pr:synthetihurewicz}
    The chosen synthetic lifts of all of the classes in \Cref{tab:elements}, have nonzero image in $\pi_{\ast,\ast}\J^2_\BP$. In more detail, for $t\geq 0$, the classes
    \[\be_{1+9t}, \quad \be_{2+9t}, \quad [\al_1\be_{3+9s/3}], \quad\be_{5+9t}, \quad\langle \al_1, \al_1, \be_{5+9t}\rangle, \quad \be_{6+9t/3}, \quad [\al_1\be_{7+9t}]\]
    have image in $\J^2_\BP$ given by the classes
    \[[\be\Delta^{6t}],  \partial([\al\Delta]\Delta^{6t}),  \be_1([\al\Delta]\Delta^{6t}), \partial([\al\Delta^3]\Delta^{6t}]),  \partial(\be\Delta^{3+6t}),  [\be\Delta^{3+6t}], \be_1([\al\Delta]\Delta^{3+6t}),\]
    respectively, up to a unit. In particular, all of the products of \Cref{main:equivalences}, \Cref{main:vanishing} and \Cref{main:nonvanishing} have nonzero image in $\J_\BP^{2}$.
\end{prop}

Of course, having nonzero image in $\J_\BP^2$ does not mean that an element is nonzero in $\J_\BP$; it may be hit by a differential in the modified ANSS for $\J^2$.

\begin{proof}
	This detection statements holds after inverting $\tau$, see \cite[Th.6.5]{tmfthree} and \cite[Th.A]{heighttwojat3}. The fact that $\pi_{\ast,\ast}\J^2_\BP$ is $\tau$-torsion free in these degrees, see \Cref{pr:manssforJ} and \Cref{ss1,ss2}, shows that this detection also must hold synthetically.
\end{proof}

%%%%%%%%%%%%%%%%%%%%%%%%%%%%%%%%%%%%%%%%%%%%%%%%%%%%%%%%%%%%%%%
%%%%%%%%%%%%%%%%%%%%%%%%%%%%%%%%%%%%%%%%%%%%%%%%%%%%%%%%%%%%%%%
%%%%%%%%%%%%%%%%%%%%%%%%%%%%%%%%%%%%%%%%%%%%%%%%%%%%%%%%%%%%%%%
\section{Nonvanishing of periodic families}\label{sec:detection}
Now we can use $\J^{2}$ to prove some nonvanishing statements. These nonvanishing statements are simplified versions of some used in \cite{heighttwojat3}, and are a consequence of the following lemma.

\begin{lemma}\label{lm:sourcehaslowfiltration}
    Let $\varphi\colon \1 \to X$ be a map of synthetic spectra and $x\in \pi_{s,f}\1$. If $\varphi(x)$ is nonzero and $\tau^{r}$-torsion for $r\geq f-2$ but not $\tau^{r-1}$-torsion, then $x$ is $\tau$-torsion free.
\end{lemma}

The moral of this statement is that a differential in a synthetic spectrum $X$ with source in filtration $\leq 1$ cannot come from $\1$. 

\begin{proof}
    By assumption, $x$ is either $\tau^{s}$-torsion for some $s\geq r$ or $\tau$-torsion free. As $r\geq f-2$, we see that $x$ cannot $\tau^{s}$-torsion for any $s\geq r$, as then it would be hit by a $d_{s+1}$-differential with source in filtration $\leq 1$, and all such classes are permanent cycles by \cite{anoneline}. In particular, we see $x$ is $\tau$-torsion free.
\end{proof}

When the spectrum $X =\J^2$, there is a further refinement of this lemma.

\begin{lemma}\label{lm:filtrationtwo}
    For all $k\geq 0$, the mod $\tau$ unit map $\1/\tau \to \J^2_\BP/\tau$ is zero in bidegrees $(50+144k,2)$ and $(130+144k,2)$. In particular, any differential in the signature spectral sequence for $\J^2_\BP$ with source is one of these bidegrees does not come from a differential in the ANSS for $\Sph$.
\end{lemma}

The point is that only certain divided $\beta$-family elements could have nonzero image in $\J^2_\BP/\tau$, but as $\J^2_\BP/27$ already has a $v_2^9$-self map, elements like $\be_{9a/b}$ are rarely detected in $\J^2_\BP/\tau$.

\begin{proof}
    To begin with, we note that by \cite[Pr.2.16]{heighttwojat3} that using the connective variant of $\J^2_\BP$, denoted by $\j^2_\BP$, of \cite[Df.2.10]{heighttwojat3}, that the map $\j^2 \to \J^2$ is injective on mod $\tau$ bigraded homotopy groups, so it suffices to show that the unit map $\1 \to \j^2$ is zero in these degrees modulo $\tau$. By \cite[Th.2.6]{MRW77}, the only potential classes in $\pi_{50+144k,2} \1/\tau$ are $\be_{s3^n/j}$ where $3\nmid s$ and $j\equiv 23$ modulo $36$, and then necessarily $n\geq 24$. Similarly, in bidegree $(130+144k,2)$, we have divided $\beta$-family elements of the form $\be_{s3^n/j}$ with $j\equiv 3$ modulo $36$ and $n\geq 4$. By \cite[Lm.3.30]{heighttwojat3}, all such classes in the divided $\beta$-family map to zero in $\j^2_\BP/\tau$.
\end{proof}

With everything now in hand, we immediately obtain the nonvanishing statements advertised in the introduction.

\begin{prop}\label{pr:nonvanishingbeta1beta63}
    Choose elements $x_i \in \{\be_{1+9s}, \be_{6+9s/3}\}_{s\geq 0} \subseteq \pi_{\ast,2} \1$. Then for all $t\geq 0$:
    \begin{enumerate}
%        \item $\prod_{i=1}^4 x_i$ in $\TMF_\BP$ is $\tau$-torsion free.
        \item $\prod_{i=1}^5 x_i$ is $\tau^8$-torsion in $\TMF_\BP$ and $\J^{2}_\BP$, and $\tau$-torsion free in $\1$.
%        \item $\prod_{i=1}^N x_i$ is \textbf{not} $\tau^4$-torsion in $\TMF_\BP$ for any $N\geq 1$.
%        \item $\al_1 [\al_1\be_{3+9t/3}]$ and $\al_1 [\al_1\be_{7+9t}]$ in $\TMF_\BP$ are $\tau$-torsion free.
        \item $[\al_1\be_{3+9t/3}] x_i$ and $[\al_1\be_{7+9t}] x_i$ are $\tau^4$-torsion in $\TMF_\BP$ and $\tau$-torsion free in $\J^{2}_\BP$.
        \item $\be_{2+9t} x_i$ and $\be_{5+9t} x_i$ are $\tau$-torsion free in $\J^{2}_\BP$.
        \item $\be_{2+9t} \prod_{i=1}^2 x_i$ and $\be_{5+9t} \prod_{i=1}^2 x_i$ are $\tau^4$-torsion in $\J^{2}_\BP$ and $\tau$-torsion free in $\1$.
    \end{enumerate}
\end{prop}

\begin{proof}
    %Part 1 is clear as in the DSS for $\TMF$, this product is not hit by any differentials. 
    For part 1, we appeal to \Cref{lm:sourcehaslowfiltration} and the spectral sequence of \Cref{pr:manssforJ}, and observe that the $d_9$-differential hitting this product has source of filtration $1$. %Part 3 follows from DSS for $\TMF$, as this product is always hit by a $d_9$-differential. 
    In part 2, the arguments for the $[\al_1\be_{3+9t/3}]$-family and the $[\al_1\be_{7+9t}]$-family are identical, so we focus on the former. %Part 3 then follows from part 1 and the extension $\al_1[\al_1\be_{3+9t/3}] \ueq \tau^4 \be_1^3 \be_{1+9t}$ in $\TMF_\BP$; this follows from the Toda bracket expression $\be_1 = \langle \al_1, \al_1, \al_1\rangle$, the DSS for $\TMF$ of \cite[Th.7.1]{smfcomputation}, and the juggling formula
    %\[\al_1 [\al_1\be_{3+9t/3}] \ueq \al_1 \be_1[\al_1 \Delta^{1+6t}] \ueq \be_1 \langle \tau^4\be_{1+9t}, \al_1, \al_1 \rangle \al_1 = \tau^4\be_1^2 \be_{1+9t}\neq 0.\]
    We note that in the ANSS for $\TMF$, these products are hit by $d_5$-differentials which do not lift to differentials in $\J^2_\BP$ by \Cref{pr:manssforJ}. Part 3 and the $\tau^4$-torsion statement of part 4 both follow from \Cref{pr:manssforJ}. As the products in part 4 have filtration 6 and are hit by $d_5$-differentials in the spectral sequence associated with $\J^{\psi^2}_\BP$, see \Cref{pr:manssforJ}, then by \Cref{lm:sourcehaslowfiltration} we see that these products are $\tau$-torsion free in $\1$.
\end{proof}

\begin{prop}\label{pr:secondlayerusingsecondfiltration}
    For any collection of integers $s_i,t,w\geq 0$, then the classes of (\ref{eq:line2}) are $\tau^4$-torsion in $\J^2_\BP$, the classes of (\ref{eq:line3}) are $\tau^8$-torsion in $\J^2_\BP$, and all are $\tau$-torsion free in $\1$.
\end{prop}

\begin{proof}
    For the classes in (\ref{eq:line2}), we see by \Cref{pr:synthetihurewicz} that they are nonzero in $\J^2_\BP$ and then use \Cref{pr:manssforJ} to see that they are all $\tau^4$-torsion; the classes $\al_1\be_1^2\be_2$, $\al_1\be_2\be_{6/3}^2$, and their periodic variants are $\tau^4$-divisible by something hit by a $d_9$-differential, hence these classes are $\tau^4$-torsion. In particular, all of these products are either hit by a $d_5$-differential in the ANSS for $\Sph$ or not. If they are killed by a $d_5$, then this would have to be the same $d_5$ as in the modified ANSS of $\J^2_\BP$, which is impossible as the bidegree for the source of this differential is zero by \Cref{lm:filtrationtwo}. Hence the classes of (\ref{eq:line2}) are $\tau$-torsion free in $\1$. The classes in (\ref{eq:line3}) are $\tau^8$-torsion in $\J^2_\BP$ again by inspection of \Cref{pr:manssforJ}, and by the same argument as above, they are $\tau$-torsion free in $\1$.
\end{proof}

\begin{proof}[Proof of \Cref{main:equivalences}]
    Combine \Cref{cor:vanishinghalf} with parts 2 and 6 of \Cref{pr:nonvanishingbeta1beta63}.
\end{proof}

\begin{proof}[Proof of \Cref{main:nonvanishing}]
    Parts 2 and 6 of \Cref{pr:nonvanishingbeta1beta63} give us (\ref{eq:line1}), and \Cref{pr:secondlayerusingsecondfiltration} gives us the rest.
\end{proof}

%%%%%%%%%%%%%%%%%%%%%%%%%%%%%%%%%%%%%%%%%%%%%%%%%%%%%%%%%%%%%%%
%%%%%%%%%%%%%%%%%%%%%%%%%%%%%%%%%%%%%%%%%%%%%%%%%%%%%%%%%%%%%%%
%%%%%%%%%%%%%%%%%%%%%%%%%%%%%%%%%%%%%%%%%%%%%%%%%%%%%%%%%%%%%%%

\addcontentsline{toc}{section}{References}
\scriptsize
\bibliography{references}
\bibliographystyle{alpha}

\end{document}