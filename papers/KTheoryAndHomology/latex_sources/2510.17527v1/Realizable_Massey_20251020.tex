\documentclass{amsart}
\usepackage{mathabx}
\usepackage{amsmath,amscd,amsthm}
\usepackage{amsfonts,amssymb}
\usepackage[all]{xy}
\usepackage{mathtools}
\usepackage{enumerate}

\usepackage{hyperref}

%\usepackage{adjustbox}

%%%%%% added by Gereon 

\usepackage[usenames, dvipsnames]{color}

\definecolor{NTNUblue}{RGB}{0,80,158}
\definecolor{NTNUbluesupport}{RGB}{62,98,138}
\definecolor{NTNUorange}{RGB}{239,129,20}

\newcommand{\Eivind}[1]{{\textcolor{NTNUblue}{\bf Eivind: #1}}}

\newcommand{\Gereon}[1]{{\textcolor{NTNUorange}{\bf Gereon: #1}}}

%\usepackage{color}

%%%%%%%%%

\newcommand{\into}{\hookrightarrow}

\newcommand{\onto}{\twoheadrightarrow}

\newcommand{\homeoto}{\xrightarrow{\approx}}

\newcommand{\isoto}{\xrightarrow{\cong}}

\newcommand{\xto}{\xrightarrow}

%\newcommand{\uoplus}{\underline{\oplus}}

\newcommand{\uoplus}{\sqcap}

\newcommand{\DGA}{\mathbf{DGA}}

\newcommand{\C}{\mathbb{C}}
\newcommand{\CP}{\C\mathrm{P}}
\newcommand{\F}{\mathbb{F}}
%\newcommand{\HP}{\HH\mathrm{P}}
%\newcommand{\OO}{\mathbb{O}}
%\newcommand{\OP}{\OO\mathrm{P}}
\newcommand{\PP}{\mathbb{P}}
\newcommand{\Q}{\mathbb{Q}}
\newcommand{\R}{\mathbb{R}}
\newcommand{\Z}{\mathbb{Z}}

\newcommand{\BBB}{\mathbf{B}}

\newcommand{\DDD}{\mathbf{D}}

\newcommand{\DDDw}{\widetilde{\DDD}}

\newcommand{\DDDwr}{\DDDw^{\re}}

\newcommand{\DDDwl}{\DDDw^{\li}}

\newcommand{\DDDs}{\DDD^*}

\newcommand{\DDDws}{\DDDw^*}

\newcommand{\SSS}{\mathbf{S}}

\newcommand{\TTT}{\mathbf{T}}

\newcommand{\li}{\mathbf{l}}

\newcommand{\re}{\mathbf{r}}

\newcommand{\Ah}{\mathcal{A}}
\newcommand{\Bh}{\mathcal{B}}
\newcommand{\Ch}{\mathcal{C}}

\newcommand{\ET}{\mathcal{ET}}

\newcommand{\Hh}{\mathcal{H}}

\newcommand{\RFG}{\mathcal{RFG}}

\newcommand{\Mh}{\mathcal{M}}

\newcommand{\Nh}{\mathcal{N}}

\newcommand{\Sb}{\mathcal{S}}

\newcommand{\Th}{\mathcal{T}}

\newcommand{\Vh}{\mathcal{V}}

\newcommand{\Wh}{\mathcal{W}}

\newcommand{\Xh}{\mathcal{X}}

\newcommand{\Yh}{\mathcal{Y}}

\newcommand{\Zh}{\mathcal{Z}}

\newcommand{\Ext}{\mathrm{Ext}}

\newcommand{\uExt}{\underline{\Ext}}

\newcommand{\abs}{\mathrm{abs}}

\newcommand{\Hom}{\mathrm{Hom}}

\newcommand{\uHom}{\underline{\Hom}}

\newcommand{\gExt}{\underline{\Ext}}

\newcommand{\gHom}{\underline{\Hom}}

\newcommand{\Map}{\mathrm{Map}}

\newcommand{\Mat}{\mathrm{Mat}}

\newcommand{\oddeven}{\mathrm{oe}}

\newcommand{\oden}{\mathrm{oe}} 

\newcommand{\ooden}{\mathrm{ooe}} 

\newcommand{\odeen}{\mathrm{oee}} 

\newcommand{\enod}{\mathrm{eo}} 

\newcommand{\Sym}{\mathrm{Sym}}

\newcommand{\gr}{\mathrm{gr}}

\newcommand{\op}{\mathrm{op}}

\newcommand{\res}{\mathrm{res}}

\newcommand{\HH}{\mathrm{HH}}

\newcommand{\id}{\mathrm{id}}

\newcommand{\Imm}{\mathrm{Im}\,}

\newcommand{\Ker}{\mathrm{Ker}\,}

\newcommand{\tK}{\tilde{K}}

\newcommand{\bbb}{\bullet}

\newcommand{\ba}{\bar{a}}

\newcommand{\bb}{\bar{b}}

\newcommand{\bc}{\bar{c}}

\newcommand{\Cb}{\Ch^\bbb}

\newcommand{\Hb}{H^\bbb}

\newcommand{\Hba}{\Hb(\Ah)}

\newcommand{\Hbb}{\Hb(\Bh)}

\newcommand{\Hbc}{\Hb(\Ch)}

\newcommand{\fb}{f_\bbb}

\newcommand{\Kbb}{K^{\bbb}_{\bbb}}

\newcommand{\bbf}{\bar{f}}

\newcommand{\bbg}{\bar{g}}

\newcommand{\dee}{\partial}

\newcommand{\dd}{\delta}

\newcommand{\ddA}{\dd_{\Ah}}

\newcommand{\ddB}{\dd_{\Bh}}

\newcommand{\ddC}{\dd_{\Ch}}

\newcommand{\fA}{f^{\Ah}}

\newcommand{\fB}{f^{\Bh}}

\newcommand{\fC}{f^{\Ch}}

\newcommand{\mult}{\cdot}

\newcommand{\One}{\mathbf{1}}

\newcommand{\eps}{\epsilon}

\newcommand{\ma}{m^{\Ah}} 

\newcommand{\mb}{m^{\Bh}} 

\newcommand{\mc}{m^{\Ch}} 

\newcommand{\mh}{m^H} 

\newcommand{\mk}{m^H} 

\newcommand{\mmm}{m} 

\newcommand{\pam}{\sigma} 

\newcommand{\ga}{\gamma_{\Ah}} 

\newcommand{\gb}{\gamma_{\Bh}} 

\newcommand{\gc}{\gamma_{\Ch}} 

\newcommand{\gag}{\gamma_G} 

\newcommand{\tf}{\widetilde{f}} 

\newcommand{\wth}{\widetilde{h}} 

\newcommand{\tk}{\widetilde{k}} 

\newcommand{\wphi}{\widetilde{\varphi}} 

\newcommand{\wpsi}{\widetilde{\psi}} 

\newcommand{\wPsi}{\widetilde{\Psi}} 

\newcommand{\wkappa}{\widetilde{\kappa}} 

\newcommand{\wtheta}{\widetilde{\vartheta}} 

\newcommand{\tmh}{\widetilde{m}^H} 

\newcommand{\tm}{\widetilde{m}} 

\newcommand{\mm}{\mu} 

\newcommand{\tmm}{\widetilde{\mm}} 

\newcommand{\pr}{\mathrm{pr}} 

\newcommand{\ip}{(\iota p)} 

\newcommand{\ot}{\otimes} 

\newcommand{\nn}{\mu} 

\newcommand{\mmp}{m} 

\newcommand{\kk}{k} 

\newcommand{\bU}{\overline{U}} 

\newcommand{\bphi}{\overline{\varphi}} 

\newcommand{\bn}{\overline{n}} 

\newcommand{\rr}{\rho} 

\newcommand{\brr}{\overline{\rho}} 

\newcommand{\wrr}{\widetilde{\rho}} 

\newcommand{\Ep}{\mathbf{E}} 

\newcommand{\tphi}{\widetilde{\varphi}} 

\newcommand{\car}{\theta} 

\newcommand{\vspan}{\mathrm{span}}

\newcommand{\Span}{\mathrm{Span}}


%%%%%%%%%%%%%%%%%%%%%%%%%%%%

%%%%%%%%%%%%

%\numberwithin{equation}{subsection}
\theoremstyle{plain}
\newtheorem{thm}{Theorem}[section]
\newtheorem{theorem}{Theorem}[section]
\newtheorem{question}[theorem]{Question}
\newtheorem{prop}[theorem]{Proposition}
\newtheorem{proposition}[theorem]{Proposition}
\newtheorem{lemma}[theorem]{Lemma}
\newtheorem{cor}[theorem]{Corollary}
\newtheorem{corollary}[theorem]{Corollary}
\newtheorem{conjecture}[theorem]{Conjecture}
\newtheorem{conj}[theorem]{Conjecture}
%
\theoremstyle{definition}
\newtheorem{definition}[theorem]{Definition}
\newtheorem{defn}[theorem]{Definition}
\newtheorem{example}[theorem]{Example}
\newtheorem{examples}[theorem]{Examples}
\newtheorem{notn}[theorem]{Notation}
%\newtheorem{cont}[subsection]{Contents}
%\newtheorem{ackn}[subsection]{Acknowledgement}
%\newtheorem*{cont}{Contents}[section]
%\newtheorem*{ackn}{Acknowledgement}[section]
\theoremstyle{remark}
\newtheorem{rem}[theorem]{Remark}
\newtheorem{remark}[theorem]{Remark}
\newtheorem{rems}[theorem]{Remarks}

\def\noqed{\renewcommand{symbol}{}}
\input cyracc.def
\font\tencyr=wncyr10
\font\eightcyr=wncyr8
\def\cyr{\tencyr\cyracc}
\def\cyri{\eightcyr\cyracc}
\def\Sh{\mathrm{{\cyr Sh}}}
\def\Shi{\mathrm{{\cyri Sh}}}


\title{Non-realizability of a triple Massey product}

\author{Eivind Xu Djurhuus}
\address{Department of Mathematics, UiO, Oslo, Norway.}
\email{eivindxd@uio.no}

\author{Gereon Quick}
\address{Department of Mathematical Sciences, NTNU, Trondheim, Norway}
\email{gereon.quick@ntnu.no}
% \thanks{Both authors were partially supported by RCN Project No.\,313472 {\it Equations in Motivic Homotopy}, 
% and the project \emph{Pure Mathematics in Norway} funded by the Trond Mohn Foundation.}

\begin{document}

\begin{abstract} 
We show that an often used example of a cohomology algebra with non-vanishing triple Massey product is intrinsically $A_3$-formal and therefore, in fact, cannot be realized as the cohomology of a differential graded algebra with non-vanishing triple Massey product. 
We prove this result by computing the graded Hochschild cohomology group which contains the potential obstruction to the vanishing. 
\end{abstract}
\subjclass{55S30, 16E40, 18N40.}

\maketitle

\section{Introduction}


%\begin{definition}\label{def:Massey_product} 
Let $\Cb$ be a differential graded $\F_2$-algebra (DGA) with differential $\delta$ and cohomology algebra $\Hb$. 
Let $a,b,c$ be cohomology classes such that $a \cup b = 0$ and $b \cup c = 0$. 
We recall that the Massey product $\langle a,b,c \rangle$ is defined as follows. 
% Let $\widetilde{a}$, $\widetilde{b}$, $\widetilde{c}$ be cocycles representing $a$, $b$, $c$, respectively, 
% and let $\widetilde{e}_{ab}$ and $\widetilde{e}_{bc}$ be cochains such that $\delta \widetilde{e}_{ab} = \widetilde{a} \cup \widetilde{b}$ and 
% $\delta \widetilde{e}_{bc} = \widetilde{b} \cup \widetilde{c}$. 
%
Let $A$, $B$, $C$ be cocycles representing $a$, $b$, $c$, respectively, 
and let $E_{ab}$ and $E_{bc}$ be cochains such that $\delta E_{ab} = A \cup B$ and 
$\delta E_{bc} = B \cup C$.  
%
The set $M \coloneqq \{A,B,C,E_{ab},E_{bc}\}$ is called a defining system for the triple Massey product of $a$, $b$, and $c$. 
The cochain $A \cup E_{bc} + E_{ab} \cup C$ is a cocycle. 
We write $\langle a,b,c \rangle_M \in \Hb$ for the corresponding  cohomology class. 
%
The {\em triple Massey product} $\langle a,b,c \rangle$ is the set of all cohomology classes $\langle a,b,c \rangle_M$ for all such defining systems $M$. 
%
The class $\langle a,b,c \rangle_M$ depends on the choice of the defining system $M$. 
The image in the quotient $\Hb/(a \cup \Hb + \Hb \cup c)$, however, 
is uniquely determined by $a$, $b$, and $c$. 
We say that $\langle a,b,c \rangle$ {\em vanishes} if its corresponding class in $\Hb/(a \cup \Hb + \Hb \cup c)$ is zero. 
%We may also consider $\langle a,b,c \rangle$ as the unique class in 
%\end{definition} 
%
%%%%
%
Massey products play an important role in the classification of DGAs with a given cohomology ring.  
%since the existence of non-vanishing Massey products distinguishes isomorphism classes of DGAs. 
%In this paper we study the following natural question: 

% \begin{question}\label{question:intro}
% What is the minimal number of generators for a commutative graded $\F_2$-algebra that can be realized as the cohomology of a differential graded algebra with a non-vanishing triple Massey product of degree one classes? 
% \end{question}


To construct the simplest commutative graded algebra which may be realized as the cohomology of a DGA %differential graded algebra 
with a non-vanishing Massey product, 
one may consider $\F_2[a, b, c]/(ab, bc)$ with $a,b,c$ elements in degree one. 
We have $ab=0$ and $bc=0$ by construction, 
and hence $\langle a,b,c \rangle$ is defined. 
We may then expect that it may be possible 
to find a DGA such that $\langle a,b,c \rangle$ does not vanish.  
%
In this note, however, we show that the Massey product $\langle a,b,c \rangle$ always vanishes. 
%
More precisely, our main result is the following. 
Recall that a differential graded algebra is called {\em $A_3$-formal} if the minimal $A_{\infty}$-model of $\Cb$  has a trivial homotopy associator $m_3$ (see e.g.~\cite{Keller}). % and \cite[Section 2]{PQ2}). 

\begin{theorem}\label{thm:main_intro}
% The graded algebra $\F_2[a, b, c]/(ab, bc)$, with $a,b,c$ of degree one, is intrinsically $A_3$-formal. 
% In particular, 
Let $\Cb$ be a differential graded algebra over $\F_2$ with cohomology algebra 
isomorphic to $\F_2[a, b, c]/(ab, bc)$. 
Then $\Cb$ is $A_3$-formal. 
%
Thus, all triple Massey products for $\Cb$ vanish. 
%
In particular, the triple Massey product 
$\langle a, b, c \rangle$ vanishes for $\Cb$.  
\end{theorem}

% Since Theorem \ref{thm:main_intro} shows that the a priori simplest possible cohomology algebra does not allow for a non-trivial triple Massey product, 
% we may ask:  \Gereon{rewrite the above and think about whether we keep the question?}

Our interest in the realizability of triple Massey products grew out of the work of Hopkins--Wickelgren in \cite{HW} on triple Massey products in Galois cohomology. 
The latter has inspired a lot of research in recent years, see for example \cite{MS2} and \cite{MT2}. 

\begin{remark}\label{rem:smaller_algebras}
One may consider other $\F_2$-algebras and ask whether they realize a non-trivial Massey product.  
%
Since the definition of the Massey product does not require distinct elements, 
we may first consider the algebra $\F_2[a]/(a^2)$ with just one generator. 
However, $\F_2[a]/(a^2)$ is zero in degree two, 
and hence $\langle a,a,a \rangle$ must vanish.  
% We then ask whether there is a DGA over $\F_2$ with cohomology algebra $\F_2[a]/(a^2)$ and non-vanishing Massey product $\langle a,a,a \rangle$. 
% %
% However, by \cite[Lemma 6.5]{PQ1}, we have $\HH^{n,2-n}(\F_2[a]/(a^2))=0$ for all $n\ge 3$, 
% and hence $\F_2[a]/(a^2)$ is easily seen to be intrinsically $A_{\infty}$-formal. 
% In particular, 
% every DGA with cohomology algebra $\F_2[a]/(a^2)$ only has vanishing Massey products. 
%
%The next case is an $\F_2$-algebra with two generators.  
%
The algebra $\F_2[a,b]/(ab)$ is a Boolean graded algebra in the sense of \cite[Definition 6.1]{PQ1}. 
More generally, any algebra of the form $\F_2[a_1,\ldots,a_n]/I$ 
where $I$ is the ideal generated by all products $a_ia_j$ for $i \ne j$ 
is a Boolean graded algebra. 
The algebra $\F_2[a,b]/(a^2,ab)$ is a connected sum of a dual and a Boolean graded algebra in the sense of \cite[Section 6]{PQ1}. 
All these algebras are intrinsically $A_{\infty}$-formal by \cite[Theorem 7.13]{PQ1} 
and do not allow for non-vanishing Massey products. 
\end{remark}

We now outline the proof of Theorem \ref{thm:main_intro} and thereby describe the content of the paper. 
%
For the whole manuscript, 
we assume that all algebras and vector spaces are over $\F_2$. 
In Section \ref{sec:HH} we recall the Hochschild cohomology of graded algebras and construct the Hochschild cohomology class $[m_3] \in \HH^{3,-1}(\Hb(\Cb))$ associated to a differential graded algebra $\Cb$. 
We note that $[m_3]$ equals the canonical class of $\Cb$ introduced by Benson--Krause--Schwede in \cite{BKS} as an obstruction for the realizability of modules over Tate cohomology.  
%
We then show that $\Cb$ is $A_3$-formal if and only if $[m_3]$ is zero. 
For the latter, we assume familiarity with some basic theory of $A_{\infty}$-algebras.  
%
In Section \ref{sec:Koszul_algebras} we recall the definition of Koszul algebras and show that $\F_2[a, b, c]/(ab, bc)$ is Koszul. 
Knowing that an algebra is Koszul simplifies the task to compute its Hochschild cohomology significantly. 
%
In Section \ref{sec:main_result_proof} we prove Theorem \ref{thm:computationof_HH} which states that $\HH^{3,-1}(\F_2[a, b, c]/(ab, bc)) = 0$ by computing the image and the kernel of the differential in the Hochschild complex. 
Theorem \ref{thm:computationof_HH} then implies Theorem \ref{thm:main_intro}. 

% We prove Theorem \ref{thm:main_intro} by first showing that $A=\F_2[a, b, c]/(ab, bc)$ is a Koszul algebra. 
% Then we show that the graded Hochschild cohomology group $\HH^{3,-1}$ of the graded algebra $\F_2[a, b, c]/(ab, bc)$ vanishes. 
% Since the homotopy associator $m_3$ of the minimal model of a DGA defines a class in $\HH^{3,-1}$, 
% we can then show that this implies that any $\Cb$ with cohomology $A$ must be $A_3$-formal. 
% That all triple Massey products contain zero 
% then follows from the fact that $m_3(x \ot y \ot z)$ is an element in $\langle x, y, z \rangle$. 
%
% In fact, a simplified version of the proof of \cite[Theorem 4.7]{ST} shows that the graded algebra $\F_2[a, b, c]/(ab, bc)$ is intrinsically $A_3$-formal, 
% i.e., 
% every $\Cb$ with cohomology $\F_2[a, b, c]/(ab, bc)$ is $A_3$-formal, 
% or equivalently, any two dg algebras with cohomology $\F_2[a, b, c]/(ab, bc)$ are quasi-isomorphic as $A_3$-algebras. 
%


%%%%%%%%%

\subsection*{Acknowledgement} 
We thank Mads Hustad Sandøy for helpful conversations. 

%%%%%%%%%%%%%%%%%%%%%%%

\section{Hochschild cohomology and Massey products}\label{sec:HH}

Let $A$ be a graded $\F_2$-algebra. 
%
We recall that the bar resolution
$B(A)$ of $A$ is the non-negative chain complex
of free graded $A$-bimodules
given by $B_n(A) \coloneqq A^{\otimes n+2}$ for $n \geq 0$.
% \[
% \cdots \to A^{\otimes n+1} \to A^{\otimes n} 
% \to A^{\otimes n-1} \to \cdots
% \]
% The left (right) $A$-action is given by
% multiplying on the left (right) of the
% leftmost (rightmost) factor of $A$.
The differential 
$d \colon B_n(A) \to B_{n-1}(A)$
is given by
\[
a_0 \otimes \cdots \otimes a_{n+1} 
\mapsto \sum_{i = 0}^n %(-1)^i
a_0 \otimes \cdots \otimes 
a_i a_{i+1} \otimes \cdots \otimes a_{n+1}.
\]
%
We write $A^e = A \ot A^{\op}$. 
Note that $A^{\otimes n+2} \cong A^e \otimes A^{\otimes n}$
as a graded $A$-bimodule, hence $B(A)$
indeed consists of free modules. 
%Here we are using
% the fact that we are working over a field $k$,
% such that $A^e \otimes V$ is free as an
% $A$-bimodule for any
% graded vector space $V$.

\begin{proposition}\label{bar-acyclic}
The bar resolution $B(A)$ is a free resolution
of $A$ as a graded $A$-bimodule.
\end{proposition}
\begin{proof}
It suffices to show that 
the extended complex $\widetilde{B}(A)$ is acyclic,  
where $\widetilde{B}(A)$  
is extended from $B(A)$ by adjoining 
$\widetilde{B}_{-1}(A) := A$ 
in degree $-1$ via the multiplication 
map $\mu \colon A \otimes A \to A$. 
We claim that the map
$h \colon \widetilde{B}(A) \to \widetilde{B}(A)$ of degree $1$ given by
\[
a_0 \otimes \cdots \otimes a_{n+1}
\mapsto 1 \otimes a_0 \otimes \cdots \otimes a_{n+1}
\]
is a contracting homotopy i.e., 
$dh + hd = 1$. 
Indeed, we compute directly that
\[
dh(a_0 \otimes \cdots \otimes a_{n+1})
= a_0 \otimes \cdots \otimes a_{n+1}
- hd(a_0 \otimes \cdots \otimes a_n). \qedhere
\]
\end{proof}

\begin{definition}
Let $M$ and $N$ be graded left $A$-modules. We define
$\gHom_A(M, N)$ as the graded $\F_2$-vector space
with degree $s$ component given by
$A$-linear graded maps $f : M \to N[s]$, where $N[s]$ is the graded $A$-module given by $N[s]^n = N^{s+n}$.
\end{definition}

\begin{definition}
Let $M$ be a graded $A$-bimodule. 
We define the Hochschild cohomology $\HH^{n,*}(A,M)$
as the $n$th cohomology of the cochain complex
\[
\gHom_{A^e}(B(A),M)
\]
of graded $\F_2$-vector spaces.
% obtained by applying the functor
% $\gHom_{A^e}(-, M)$ to the bar resolution of $A$.
When $M=A$ we will write
$\HH(A) \coloneqq \HH(A, A)$. 
\end{definition}

We note that the groups $\HH^{*,*}(A,M)$ are equipped with a cohomological grading, and an internal grading induced by the grading of $A$ and $M$.
%
We can describe $\HH^{n,s}(A,M)$ more concretely as follows. 
%
Using the natural contracting isomorphism
\[
\gHom_{A^e}(A^e \otimes A^{\otimes n}, M) \cong 
    \gHom_{\F_2}(A^{\otimes n}, M)
\]
%from Lemma~\ref{adjunction-iso}, 
we see that $\HH^{n,*}(A, M)$ is isomorphic to the $n$th cohomology of the complex
\begin{align}
\label{eq:HH_reduced_complex}
\cdots \to \gHom_{\F_2}(A^{\otimes n-1}, M) \xto{\partial}
\gHom_{\F_2}(A^{\otimes n}, M) 
\xto{\partial} \gHom_{\F_2}(A^{\otimes n+1}, M) \to \cdots,
\end{align}
where the differentials are given by
\begin{align*}
\partial(f)(a_1 \otimes \cdots \otimes a_{n+1})
    & = %{(-1)}^{|f||a_1|} 
    a_1f(a_2 \otimes \cdots \otimes a_{n+1})\\
    & + \sum_{i=1}^{n} %(-1)^{i}
        f(a_1 \otimes \cdots \otimes a_i a_{i+1}
        \otimes \cdots \otimes a_{n+1})\\
    & + %{(-1)}^{n+1}
    f(a_1 \otimes \cdots \otimes a_{n})a_{n+1}.
\end{align*}

%%%%%%%%

\begin{remark}\label{rem:HH_and_Ext}
By Proposition~\ref{bar-acyclic},
we see that
$\HH(A, M)$ computes the graded Ext
modules $\gExt_{A^e}(A, M)$. 
In particular,
we can compute $\HH(A, M)$
using any free resolution of $A$ as
a graded $A$-bimodule. 
\end{remark}

%%%%%%%%%

We now assume that the reader is familiar with $A_{\infty}$-algebras. 
For an introduction to the theory of $A_{\infty}$-algebras and references with all details we refer to \cite{Keller}. 
%
Let $(\Cb, \delta, \cup)$ be a differential graded algebra (DGA) over $\F_2$ with cohomology algebra $\Hb$. 
% Recall that an $A_{\infty}$-algebra $(A,\{m_n\}_{n\ge 1})$ is called {\em minimal} if $m_1=0$. 
%
By the work of Kadeishvili \cite{Kadeishvili82, Kadeishvili88} (see also \cite{Kadeishvili23}, \cite{Keller}, and \cite{Merkulov}), 
one can equip $\Hb$ with the structure of an $A_{\infty}$-algebra $(\Hb,\{m_n\}_{n\ge 1})$ such that $m_1=0$ together with a quasi-isomorphism of $A_{\infty}$-algebras 
$(\Hb,\{m_n\}) \xto{\simeq} (\Cb, \delta, \cup)$. 
The $A_{\infty}$-algebra $(\Hb,\{m_n\})$ is called a {\em minimal model} of $(\Cb, \delta, \cup)$.  
%
A DGA is called {\em $A_{\infty}$-formal} if its minimal model can be chosen such that $m_n = 0$ for all $n \ge 3$. 
%
% For any details on $A_{\infty}$-algebras we refer to \cite{Keller}. 
%
We now consider the following weaker notion. 

\begin{definition}\label{def:A3formal}
Let $\Cb$ be a DGA. 
We say that $\Cb$ is {\em $A_3$-formal} if its minimal model can be chosen such that $m_3 = 0$.  
\end{definition}

We recall the following special case from \cite[Theorem C]{BMM}. 
Let $a,b,c \in \Hb$ be cohomology classes such that $a \cup b = 0$ and $b \cup c = 0$. 
Then $m_3(a \ot b \ot c) \in \langle a,b,c \rangle$. 
This implies the following well-known fact. 

\begin{proposition}\label{prop:A3formal_Massey}
Let $\Cb$ be a DGA.  
Assume that $\Cb$ is $A_3$-formal. 
Then the triple Massey product $\langle a,b,c \rangle$ contains zero. \qed
\end{proposition}

We note that $m_3$ is a homomorphism $m_3 \colon (\Hb)^{\ot 3} \to \Hb[-1]$ of graded algebras,  
and we can construct $m_3$ as follows (see for example \cite[Section 5]{BKS}, \cite{Merkulov}).  %, or \cite{PQ2}). 
%
We choose an $\F_2$-linear graded map $f_1 \colon \Hb \to \Ker \delta$ which induces the identity on $\Hb$.   
Since $f_1$ is multiplicative on cohomology, we can find a graded $\F_2$-linear map  
$f_2 \colon \Hb \otimes \Hb \to \Cb$ of degree $-1$ satisfying  %use that $m_1^A=0$   
\begin{align*}%\label{eq:f2_condition_in_minimal_thm}
\delta(f_2(a \ot b)) = f_1(a \cup b) + f_1(a) \cup f_1(b). 
\end{align*}
%for all $x,y \in \Hb$. 
%Since $\mh_2$ is associative, relation ? holds.  
% See Remark \ref{rem:Merkulov_structure} for a formula for $f_2$. 
%
Now we define a graded $\F_2$-linear map $\Phi_3 \colon (\Hb)^{\otimes 3} \to \Cb[-1]$ by % in \eqref{eq:def_of_g3}. 
\begin{align}\label{eq:def_of_Phi3_minimal}
\Phi_3(a \ot b \ot c) \coloneqq 
f_1(a) f_2(b \ot c) 
+ f_2(a \ot b) f_1(c) + f_2((a b) \ot c + a \ot (b c)) 
\end{align}
for all homogeneous elements $a,b,c \in \Hb$ where we write $xy$ for the product $x \cup y$ to shorten the notation.  
%
We check that $\Phi_3$ has image in the cocycles of $\Cb$, 
and hence $\Phi_3$ induces a graded map $[\Phi_3] \colon (\Hb)^{\ot 3} \to \Hb[-1]$. 
We set $m_3 := [\Phi_3]$. 
%
By \cite[Proposition 5.4]{BKS}, $m_3$ is a cocycle in the complex \eqref{eq:HH_reduced_complex}.   
By \cite[Corollary 5.7]{BKS}, 
the corresponding Hochschild cohomology class  $[m_3] \in \HH^{3,-1}(\Hb)$ is independent of the choice of $f_1$ and $f_2$,   
and it is called the {\em canonical class} of $\Cb$ following Benson--Krause--Schwede who studied this class as an obstruction to the realizability of modules over Tate cohomology in \cite{BKS}.   
%
The following result is a modified version of Kadeishvili's theorem \cite{Kadeishvili88} (see also %\cite[Theorem 3.9]{PQ2}, and 
\cite[Theorem 4.7]{ST}). 

\begin{theorem}\label{thm:canonical_A3formal}
Let $\Cb$ be a DGA %over $\F_2$ 
with canonical class $[m_3] \in \HH^{3,-1}(\Hb)$. 
Then $\Cb$ is $A_3$-formal if and only if $[m_3]=0$. % in $\HH^{3,-1}(\Hb)$. 
\end{theorem}
\begin{proof}
If $\Cb$ is $A_3$-formal then $m_3$ is a trivial cocycle, 
and the class of $m_3$ vanishes in $\HH^{3,-1}(\Hb)$. 
%
%Now we assume that $[\mh_3]  \in \HH^{3,-1}(\Hb(A),\Hb(A))$ vanishes. 
Now we assume that $[m_3]=0$  in $\HH^{3,-1}(\Hb)$. 
We may assume that $\Phi_3$ and hence $m_3$ is constructed using maps $f_1,f_2$ as in \eqref{eq:def_of_Phi3_minimal}. 
%
Then there exists an $\F_2$-linear map $\eta \colon (\Hb)^{\otimes 2} \to (\Ker \delta)[-1]$ 
such that $\dee^2 [\eta] = m_3$ as maps 
$(\Hb)^{\otimes 3} \to \Hb[-1]$. 
%
% We will now show that we can use $\eta$ to modify our initial choice of $f_2$ and thereby $\Phi_3$ 
% such that the new map $\widetilde{\Phi}_3$ has values in the image of $\ma_1$. 
%
We set $\tilde{f}_2 = f_2 + \eta$. 
We note that $\tilde{f}_2$ satisfies 
\begin{align*}
\delta\tilde{f}_2(a \ot b) = \delta(f_2(a \ot b) + \eta(a \ot b))  
= f_1(a \cup b) + f_1(a) \cup f_1(b)
\end{align*}
since $\delta \circ \eta=0$. % by the assumption on the image of $\eta$. 
% Thus, $\tilde{f}_2$ is also a cochain homotopy between $f_1\mh_2$ and $\ma_2(f_1 \otimes f_1)$.  
%and we could replace the initial pair $(f_1,f_2)$ with the pair $(f_1,\tilde{f}_2)$.  
%
We then define the map $\widetilde{\Phi}_3$ by replacing $f_2$ with $\tilde{f}_2$, 
i.e., we define 
\begin{align*}%\label{eq:def_of_h3}
\widetilde{\Phi}_3(a \ot b \ot c) \coloneqq 
f_1(a) \tilde{f}_2(b \ot c) 
+ \tilde{f}_2(a \ot b) f_1(c) + \tilde{f}_2((a b) \ot c + a \ot (b c)) 
\end{align*}
for all homogeneous elements $a,b,c \in \Hb$ where we again write $xy$ for $x \cup y$ to shorten the notation.  
%
We then have 
\begin{align*}%\label{eq:g3_minus_tildeg3_is_dee_of_eta}
(\Phi_3 - \widetilde{\Phi}_3)(a \ot b \ot c)
=  f_1(a) \eta(b \ot c) + \eta(a \ot b) f_1(c) + \eta((a b) \ot c + a \ot (b c)). 
\end{align*}
%
By definition of $\dee^2$ and the assumption on $\eta$, 
this implies $\widetilde{\Phi}_3 = \Phi_3 - \dee^2\eta = 0$ as maps $(\Hb)^{\otimes 3} \to \Hb[-1]$. 
% This implies that the image of $\widetilde{\Phi}_3$ is contained in the image of $\ma_1$. 
% As explained in Remark \ref{rem:A3_formality_morphism}, this shows that there is a graded $\F$-linear map $f_3$ which extends $f_1, \tilde{f}_2$ to an $A_3$-algebra morphism which lifts $\id_{\Hba}$. 
\end{proof}

As a direct consequence we get: 

\begin{cor}\label{cor:HH_and_A3_formal}
Let $A$ be a graded algebra with $\HH^{3,-1}(A)=0$. 
Then every DGA $\Cb$ whose cohomology algebra is isomorphic to $A$ is $A_3$-formal. \qed
\end{cor}

By Proposition \ref{prop:A3formal_Massey} and Corollary \ref{cor:HH_and_A3_formal},   
in order to show Theorem \ref{thm:main_intro} it will suffice to show $\HH^{3,-1}(\F_2[a, b, c]/(ab, bc))=0$. 
This is what we now set out to prove. 
% To simplify the computation of the Hochschild cohomology of $A=\F_2[a, b, c]/(ab, bc)$ 
% we will first show in the next section that $A$ has a particularly nice structure. 

%%%%%%%%%%%%%%

% \begin{remark}\label{rem:smaller_algebras}
% One may consider other $\F_2$-algebras and ask whether they realize a non-trivial Massey product.  
% %
% Since the definition of the Massey product does not require distinct elements, 
% we may first consider the algebra $\F_2[a]/(a^2)$ with just one generator. 
% However, $\F_2[a]/(a^2)$ is zero in degree two, 
% and hence $\langle a,a,a \rangle$ must vanish.  
% % We then ask whether there is a DGA over $\F_2$ with cohomology algebra $\F_2[a]/(a^2)$ and non-vanishing Massey product $\langle a,a,a \rangle$. 
% % %
% % However, by \cite[Lemma 6.5]{PQ1}, we have $\HH^{n,2-n}(\F_2[a]/(a^2))=0$ for all $n\ge 3$, 
% % and hence $\F_2[a]/(a^2)$ is easily seen to be intrinsically $A_{\infty}$-formal. 
% % In particular, 
% % every DGA with cohomology algebra $\F_2[a]/(a^2)$ only has vanishing Massey products. 
% %
% %The next case is an $\F_2$-algebra with two generators.  
% %
% The algebra $\F_2[a,b]/(ab)$ is a Boolean graded algebra in the sense of \cite[Definition 6.1]{PQ1}. 
% More generally, any algebra of the form $\F_2[a_1,\ldots,a_n]/I$ 
% where $I$ is the ideal generated by all products $a_ia_j$ for $i \ne j$ 
% is a Boolean graded algebra. 
% The algebra $\F_2[a,b]/(a^2,ab)$ is a connected sum of a dual and a Boolean graded algebra in the sense of \cite[Section 6]{PQ1}. 
% All these algebras are intrinsically $A_{\infty}$-formal by \cite[Theorem 7.13]{PQ1} 
% and do not allow for non-vanishing Massey products. 
% \end{remark}

% \Gereon{work on and maybe move the following remark}

% \begin{remark}\label{rem:smaller_algebras}
% There are potentially smaller algebras than $\F_2[a, b, c]/(ab, bc)$ for which we could ask whether they realize a non-trivial Massey product.  
% %
% Since the definition of the Massey product does not require distinct elements, 
% we may first consider the algebra $\F_2[a]/(a^2)$ with just one generator. 
% However, $\F_2[a]/(a^2)$ is zero in degree two, 
% and hence $\langle a,a,a \rangle$ must vanish.  
% % We then ask whether there is a DGA over $\F_2$ with cohomology algebra $\F_2[a]/(a^2)$ and non-vanishing Massey product $\langle a,a,a \rangle$. 
% % %
% % However, by \cite[Lemma 6.5]{PQ1}, we have $\HH^{n,2-n}(\F_2[a]/(a^2))=0$ for all $n\ge 3$, 
% % and hence $\F_2[a]/(a^2)$ is easily seen to be intrinsically $A_{\infty}$-formal. 
% % In particular, 
% % every DGA with cohomology algebra $\F_2[a]/(a^2)$ only has vanishing Massey products. 
% %
% The next case is an $\F_2$-algebra with two generators.  
% %
% The algebra $\F_2[a,b]/(ab)$ is a Boolean graded algebra in the sense of \cite[Definition 6.1]{PQ1}, 
% and $\F_2[a,b]/(a^2,ab)$ is a connected sum of a dual and a Boolean graded algebra in the sense of \cite[Section 6]{PQ1}. 
% Both algebras are intrinsically $A_{\infty}$-formal by \cite[Theorem 7.13]{PQ1} 
% and again do not allow for non-vanishing Massey products. 
% %
% The last case with two generators is $\F_2[a,b]/(a^2)$. 
% Then $\langle a,a,a \rangle$ is defined.  could be non-trivial. 

% \end{remark}


%%%%%%%%%%%%%%%%%%%%%%%

\section{Koszul algebras}\label{sec:Koszul_algebras}

% We first introduce the main type of graded algebras
% we will work with.
For a vector space $V$, let $T(V)$ denote its graded tensor algebra over $\F_2$.  
We recall that 
a graded $\F_2$-algebra $A = \bigoplus_{n\ge 0} A^n$ is called {\em quadratic} 
if the map $T(A^1) \to A$ is surjective
with kernel generated by elements in
$A^1 \otimes A^1$. 
We see that any
quadratic algebra $A$ is canonically isomorphic 
to $T(V)/(R)$, for a
vector space of generators $V$ and a subspace of relations
${R \subseteq V \otimes V}$.  
% We also observe that
% a quadratic algebra is connected, that is
% $A^0 \cong k$.
%
For any quadratic algebra we can define the following
chain complex of
free graded $A$-bimodules. 
 

\begin{definition}\label{koszulcomplex}
Let $A = T(V)/(R)$ be a quadratic algebra. 
For $n\geq 0$ and $1 \leq i \leq n-1$, let
\begin{align}\label{eq:X_is}
X_i^n = 
V^{\otimes i-1} \otimes R \otimes V^{\otimes n-i-1} \subseteq V^{\otimes n}     
\end{align}
and 
\[
K'_n = \bigcap_{i=1}^{n-1} X_i^n\subseteq V^{\otimes n}.
\]
Here we interpret the empty intersection as the whole space,
i.e., $K'_0 = \F_2$ and $K'_1 = V$.
The Koszul complex $K(A^e, A)$ of $A$ is defined
as the nonnegative chain complex of graded $A$-bimodules with
\[
K_n(A^e, A) = A \otimes K'_n \otimes A,
\]
and differential $d_n$ induced by the one in the bar resolution $B(A)$, 
i.e., 
\[
d_n \colon a \otimes v_1 \otimes \cdots \otimes v_n \otimes b
\mapsto
av_1 \otimes v_2 \otimes \cdots \otimes v_n \otimes b
+ a \otimes v_1 \otimes \cdots \otimes v_n b.
\]
\end{definition}



\begin{definition}
A quadratic algebra $A$ is called {\em Koszul}
if its Koszul complex $K(A^e, A)$ is
a resolution of $A$ as a graded $A$-bimodule,
i.e., if $H_n(K(A^e, A)) = 0$ for $n > 0$
and $H_0(A^e, A) = A$.
\end{definition}

%%%%%%%%%%

% \subsection{Koszulness of \texorpdfstring{$A$}{A}}

We will now show that $A= \F_2 [a, b, c]/(ab, bc)$ is Koszul. 
Consider the $\F_2$-vector spaces 
$V \coloneqq \Span_{\F_2}\{a, b,c\}$ 
and 
\[
R \coloneqq \Span_{\F_2} 
\{a\otimes b, b\otimes a, c \otimes b, b \otimes c, a\otimes c + c \otimes a\}
\subseteq V \otimes V,
\]
chosen such that we can identify $A$ with
$T(V)/(R)$, and in particular we see that $A$ is quadratic.
%
% For $1 \leq i \leq n-1$, 
% we consider the following subspaces of $V^{\otimes n}$:
% \[
% X_i^n = 
% V^{\otimes i-1} \otimes R \otimes V^{\otimes n-i-1}.
% \]
%
Let $e \coloneqq a \otimes c + c \otimes a$. 
In the space $X_i^n$ defined in \eqref{eq:X_is} we  consider the
set $\mathcal{B}^n$ consisting of strings 
$x = x_1 \cdots x_k$
of symbols from the set $\{ a, b, c, e\}$ such that
$|x_1| + \cdots + |x_k| = n$ and
$ca$ does not occur as a substring of $x$. Here
$|x_i|$ denotes the degree of the symbol $x_i$,
so $|a|=|b|=|c|=1$ and $|e|=2$. We will identify
the strings in $\mathcal{B}^n$ with tensors in $V^{\otimes n}$.
As an example we see that
\[\mathcal B^2 = \{a \otimes a, a \otimes b, a \otimes c,
    b \otimes a, b \otimes b, b \otimes c, c \otimes b,
    c \otimes c, e\}.\]
We also introduce the subsets $\mathcal B_i^n$ for
$1 \leq i \leq n-1$ consisting of the strings in $\mathcal B^n$
where
the $(i, i+1)$-part is in $R$, where we count
$e$ with multiplicity $2$. More precisely,
for a string $x \in \mathcal B^n$ and 
an integer $i$, $1 \leq i \leq n$, we 
obtain a symbol $x_{(i)}$ by the following process. We first
modify the string $x$ into a string $x'$ of length $n$
by doubling every occurrence of $e$ in $x$, and
we then set $x_{(i)} = x'_i$. We now define 
\[ \mathcal B_i^n \coloneqq
    \{x \in \mathcal{B}^n \mid
    x_{(i)}x_{(i+1)} \in 
    \{ab, ba, cb, bc,
     eb, be, ee\}\}.
\]
For instance, for the string $x = bebe$ we get that
$x' = beebee$ such that, e.g., 
$x_{(4)} = b$, and we see that
$x \in \mathcal{B}_i^6$ for all $1 \leq i \leq 5$. 

\begin{lemma}\label{lemma:basis_of_Vn}
For each $n$, the set $\Bh^n$ is a basis
for $V^{\otimes n}$. 
\end{lemma}
\begin{proof}
From the basis
$\{a, b, c\}$ of $V$ we obtain a standard
basis of $V^{\otimes n}$ consisting
of the pure tensors in the symbols
$\{a, b, c\}$. 
We obtain $\mathcal B^n$
from this standard basis using only elementary
column operations, hence showing that $\mathcal B^n$
is also a basis. 
Starting from the standard basis,
we can replace each occurrence of $c \otimes a$
with $e$ by adding a suitable linear combination
of standard pure tensors. 
More formally,
we will do this using induction.

For $i \geq 0$, let
$\mathcal B^{n, i}$ be the set of strings $x_1 \cdots x_k$
in the symbols $\{a, b, c, e\}$ satisfying the conditions:
\begin{itemize}
\item $|x_1| + \cdots + |x_k| = n$,
\item there is at most $i$ occurrences of $e$,
\item the substring $ca$ only occurs after all the $e$'s,
\item if there is less than $i$ occurrences of $e$, 
there are no occurrences of $ca$.
\end{itemize}
%
We see that $\mathcal{B}^{n,0}$ is the standard
basis, while $\mathcal{B}^{n, i} = \mathcal B^n$
for large enough $i$ (e.g., for $i \geq n/2$). 
We will show that each $\mathcal{B}^{n,i+1}$
can be obtained from $\mathcal{B}^{n,i}$
using only elementary column operations, where
we have identified the strings with tensors in $V^{\ot n}$.   
%
We obtain $\mathcal B^{n, i+1}$ from
$\mathcal B^{n,i}$ in the following manner.
If $x \in \mathcal B^{n,i}$ have no
occurrence of $ca$, then
we already have $x \in \mathcal{B}^{n,i+1}$,
so we do nothing. 
Otherwise, there are precisely $i$ occurrences of $e$ and
at least one occurrence of $ca$ in $x$.
%
Consider the string $x'$ obtained by
replacing the first occurrence of $ca$
with $ac$. 
We see that $x'\in \mathcal B^{n, i}$
and we add $x'$ to $x$
to obtain the tensor $x' + x \in \mathcal{B}^{n, i+1}$.
%
All tensors in $\mathcal{B}^{n, i+1}$ can be obtained
in precisely one of these two ways, hence
we obtain $\mathcal{B}^{n, i+1}$ from $\mathcal{B}^{n,i}$
as wanted.
\end{proof}

Now we can prove the main result of this section. 

\begin{proposition}\label{prop:A_is_Koszul}
The quadratic algebra $A = \F_2 [a, b, c]/(ab, bc)$
is Koszul.    
\end{proposition}

\begin{proof}
We will show that, for each $n \geq 0$, 
the set $\mathcal B^n$ is
a basis for $V^{\otimes n}$ which distributes 
the subspaces $X_1^n, \ldots, X_{n-1}^n$, 
i.e., 
for each $X^n_i$ the 
subset $\Bh^n_i \subseteq \Bh^n$ 
forms a basis for $X^n_i$.  
%
By \cite[Chapter 2, Theorem 4.1]{ppqa}, 
this implies that $A$ is Koszul.  
%We now show that $\mathcal{B}^n$ distributes 
%the subspaces $X_1^n, \ldots, X_{n-1}^n$.
By definition, 
${X_i^n = V^{\otimes i-1} \otimes R \otimes V^{n-i-1}}$,
hence from the standard basis of $V$ and
the basis $\mathcal R = \{a \otimes b, b\otimes a, 
c \otimes b, b \otimes c, e\}$
of $R$, we obtain a basis of $X_i^n$
consisting of strings $x$ in the symbols
$\{a, b, c, e\}$ with at most one $e$ and
$x_{(i)}x_{(i+1)} \in \mathcal R$,
where $x_{(i)}$ is the notation introduced to
define $\mathcal B_i^n$.
Using a similar induction argument as above, we
replace each occurrence of $ca$ with $e$ in
this basis using
only elementary column operations to obtain a new
basis for $X_i^n$. This
gives precisely the set 
$\mathcal B_i^n \subseteq \mathcal B^n$,
which shows that $\mathcal B^n$ distributes
$X_1^n, \ldots, X_{n-1}^n$. 
\end{proof}

\begin{remark}
As a consequence of the proof of Proposition \ref{prop:A_is_Koszul} 
we see that
$\Bh'_n \coloneqq \bigcap_{i=1}^{n-1} \mathcal B_i^n$
is a basis
for
$K'_n = \bigcap_{i=1}^{n-1} X_i^n$.  
%
This basis $\mathcal B'_n$ can be explicitly
described as the set of strings $x_1 \cdots x_k$
in the symbols
$\{a, b, c, e\}$ satisfying
that
$|x_1|+ \cdots + |x_k|$
and the symbols in the string alternates
between $b$ and one
from the set $\{a, c, e\}$.
For example, we get 
\begin{align}
\label{eq:B'3}
\Bh_3' = \{
    aba, abc,
    cba, cbc,
    bab, bcb,
    eb, be\}
\end{align}
and 
\begin{align}
\label{eq:B'4}
\Bh_4' = \{
    abab, abcb, bcba, bcbc, 
    baba, babc, cbab, cbcb, 
    abe, eba, cbe, ebc, beb\}.
\end{align}
\end{remark}

% Using these bases, we obtain the following nice
% pattern for the dimensions
% of $K_n'$. 
% \begin{corollary}
%     The dimension of $K_n'$ for $n \geq 1$ is the $(n+3)$-th
%     Fibonacci number.
% \end{corollary}

% \begin{proof}
%     Since $|\mathcal B'_1| = 3$ and
%     $|\mathcal B'_2| = 5$, it suffices
%     to show that
%     \[|\mathcal B'_n| = |\mathcal B'_{n-2}| + |\mathcal B'_{n-1}|
%     ,\]
%     for all $n \geq 3$. 
    
%     Let $X_n$ denote the set of
%     strings $x \in \mathcal B'_n$ with $x_1 \in \{a, c, e\}$
%     and let $Y_n$ be the set of $x \in \mathcal{B}'_n$ with
%     $x_1 = b$. For $x \in Y_n$, removing $x_1$ gives
%     a string in $X_{n-1}$ by the alternating property.
%     Similarly, removing $x_1$ from a string $x \in X_n$
%     gives either a string in $Y_{n-1}$ (if $x_1$ is $a$ or $c$),
%     or a string in $Y_{n-2}$ (if $x_1$ is $e$). Hence
%     all strings in $Y_n$ comes from adjoining
%     $b$ to the front of a string in $X_{n-1}$, while
%     all strings in $X_n$ comes from adjoining
%     $a$ or $c$ to the front of a string 
%     in $Y_{n-1}$ or adjoining $e$ to the front of
%     a string in $Y_{n-2}$.
%     We therefore obtain the relations
%     \[|Y_n| = |X_{n-1}|\]
%     and
%     \[|X_n| = 2|Y_{n-1}| + |Y_{n-2}|.\]
%     Since $|\mathcal B'_n| = |X_n| + |Y_n|$ for $n \geq 1$,
%     we deduce that
%     \begin{align*}
%         |\mathcal B'_n| &= 2|Y_{n-1}| + |Y_{n-2}| + |X_{n-1}| \\
%         &= |\mathcal B'_{n-1}| + |Y_{n-1}| + |Y_{n-2}| \\
%         &= |\mathcal B'_{n-1}| + |X_{n-2}| + |Y_{n-2}| \\
%         &= |\mathcal B'_{n-1}| + |\mathcal B'_{n-2}|.
%     \end{align*}
% \end{proof}

%%%%%%%%%

% \subsection{Vanishing of \texorpdfstring{$\HH ^{3,-1}(A)$}{HH3,-1(A)}}

% \section{Computation of \texorpdfstring{$\HH ^{3,-1}(A)$}{HH3,-1(A)}}

\section{Proof of the main result}\label{sec:main_result_proof}

By Proposition \ref{prop:A3formal_Massey} and Corollary \ref{cor:HH_and_A3_formal},   
Theorem \ref{thm:main_intro} will follow from the following: 

\begin{theorem}\label{thm:computationof_HH}
We have $\HH^{3,-1}(\F_2[a, b, c]/(ab, bc))=0$.
\end{theorem}
%
% i.e., we will show that the graded algebra
% \[
% A = \F_2 [a, b, c]/(ab, bc)
% \] 
% is intrinsically $A_3$-formal. 
%
\begin{proof}%[Proof of Theorem \ref{thm:computationof_HH}]
%
% By Proposition \ref{HH-implies-vanishing}, it suffices to
% show that $\HH^{3,-1}(A) = 0$.  
%
Since $A$ is Koszul, 
the natural inclusion $K(A^e, A), A) \into B(A)$ 
is a quasi-isomorphism. 
Hence we can compute
$\HH(A)$ as the cohomology of
the complex $\gHom_{A^e}(K(A^e, A), A)$.
We first observe that we have
\[
K(A^e, A)_n = A \otimes K'_n \otimes A \cong A^e \otimes K'_n,
\]
as graded $A$-bimodules. 
Using the contracting isomorphism
\[
\gHom_{A^e} (A^e \otimes K'_n, A) \cong \gHom_{\F_2}(K'_n, A)
\]
of graded vector spaces 
% from Lemma \ref{adjunction-iso}
we see that $\HH(A)$ can be computed as the cohomology
of the following complex of graded vector spaces:
\[
\cdots \to \gHom_{\F_2}(K'_{n-1}, A) 
\xrightarrow{\partial^{n-1}} \gHom_{\F_2}(K'_n, A)
\xrightarrow{\partial^{n}} \gHom_{\F_2}(K'_{n+1}, A)
\to \cdots 
\]
where the differential $\partial^n$
is given by
\[
\partial^n(f)(v_1 \otimes \cdots \otimes v_{n+1}) 
    = v_1f(v_2 \otimes \cdots \otimes v_{n+1}) + 
    f(v_1 \otimes \cdots \otimes v_n)v_{n+1}.
\]
%
To compute $\HH^{3,-1}(A)=0$,
we need to show that  
\[
\Hom_{\F_2}(K'_2, A^1) 
    \xrightarrow{\partial^2} \Hom_{\F_2}(K'_3, A^2)
    \xrightarrow{\partial^3} \Hom_{\F_2}(K'_4, A^3)
\]
is exact in the middle, 
i.e., $\Imm \partial^2 = \Ker \partial^3$. 

%%%%%

First we will describe $\Ker \partial^3$. 
To do so, we use the following notation for elements 
in the basis $\mathcal B'_4$ of $K'_4$. 
%
Since
$a$ and $c$ play symmetrical roles in $A$,
we will introduce the notation $(a|c)$ to mean
that each of $a$ and $c$ can be used in the expression.
For example, for a map $f \in \Hom_{\F_2}(K'_3, A^2)$, 
the equation $f((a|c) \otimes b) = 0$ would
mean that we have two equations $f(a \otimes b) = 0$ 
and $f(c \otimes b) = 0$. If there are several instances
of $(a|c)$ in the expression, each instance
can be replaced by $a$ or $c$ independently of each other.

\begin{lemma}\label{kernel-rels}
A map $f \in \Hom_{\F_2}(K'_3, A^2)$
lies in $\Ker \partial^3$ if and only if it
satisfies the following relations:
%
\begin{align}
f(b \otimes (a|c) \otimes b) \in \Span_{\F_2}\{ b^2 \}, \label{i} \tag{i}\\
%
f((a|c) \otimes b \otimes (a|c)) \in \Span_{\F_2}\{ a^2, ac, c^2\}, \label{ii} \tag{ii} \\
%
a(f(c \otimes b \otimes a) + f(e \otimes b)) 
    + cf(a \otimes b \otimes a) = 0, \label{iii} \tag{iii} \\
%
c(f(a \otimes b \otimes c) + f(e \otimes b))
    + af(c \otimes b \otimes c) = 0, \label{iv} \tag{iv} \\
%
a(f(a \otimes b \otimes c) + f(b \otimes e))
    + cf(a \otimes b \otimes a) = 0, \label{v} \tag{v}\\
%
c(f(c \otimes b \otimes a) + f(b \otimes e))
    + af(c \otimes b \otimes c) = 0, \label{vi} \tag{vi} \\
%
f(e \otimes b) + f(b\otimes e) \in \Span_{\F_2}\{ a^2, ac, c^2 \}. \label{vii} \tag{vii}
\end{align}
\end{lemma}
\begin{proof}
A map $f \in \Hom_{\F_2}(K'_3, A^2)$ is in the kernel of $\partial^3$ 
if and only if $\partial^3(f)$ vanishes on all 
elements of the basis $\Bh'_4$. 
We now evaluate $\partial^3(f)$ on $\Bh'_4$ as described in \eqref{eq:B'4}. 
% 
First, 
since $(a|c)b = 0$ in $A^2$, 
the equation 
\[
\partial^3(f)((a|c)\ot b \ot (a|c) \ot b) 
= (a|c) f(b \ot (a|c) \ot b) + f((a|c) \ot b \ot (a|c)) b
= 0
\]
implies 
$f(b \otimes (a|c) \otimes b) \in \Span_{\F_2}\{b^2\}$, 
and
$f((a|c) \otimes b \otimes (a|c)) \in \Span_{\F_2}\{a^2, ac, c^2\}$.  
%
The equation $\partial^3(f)(b \otimes (a|c)\otimes b \otimes (a|c))=0$
gives the same relations. 
This shows that \eqref{i} and \eqref{ii} are necessary and sufficient. 
%
Second, 
\eqref{iii} and \eqref{iv} are imposed by 
the equations  
\[
\partial^3(f)(e \ot b \ot a) 
= af(c \ot b \ot a) + cf(a \ot b \ot a) + f(e \ot b)a 
= 0
\]
and  
\[
\partial^3(f)(e \ot b \ot c) 
= af(c \ot b \ot c) + cf(a \ot b \ot c) + f(e \ot b)c
= 0.
\]
% gives the relations
% \begin{equation*}
%     a(f(c \otimes b \otimes a) + f(e \otimes b)) 
%     + cf(a \otimes b \otimes a) = 0,
% \end{equation*}
% and 
% \begin{equation*}
%     c(f(a \otimes b \otimes c) + f(e \otimes b))
%     + af(c \otimes b \otimes c) = 0.
% \end{equation*}
%
Similarly, 
\eqref{v} and \eqref{vi} are imposed by 
the equations  
\[
\partial^3(f)(a \ot b \ot e)
= a f(b \ot e) + f(a \ot b \ot a)c + f(a \ot b \ot c)a 
= 0
\]
and 
\[
\partial^3(f)(c \ot b \ot e)
= c f(b \ot e) + f(c \ot b \ot a)c + f(c \ot b \ot c)a 
= 0. 
\]
% gives the relations
% \begin{equation*}
%     a(f(a \otimes b \otimes c) + f(b \otimes e))
%     + cf(a \otimes b \otimes a) = 0,
% \end{equation*}
% and
% \begin{equation*}
%     c(f(c \otimes b \otimes a) + f(b \otimes e))
%     + af(c \otimes b \otimes c) = 0.
% \end{equation*}
%
Finally, 
the condition
\[
\partial^3(f)(b \ot e \ot b) 
= bf(e \ot b) + f(b \ot e)b
= 0 
\]
gives the relation 
$f(e \ot b) + f(b \ot e) \in \Span_{\F_2}\{ a^2, ac, c^2 \}$ 
which is \eqref{vii}. 
%This finishes the proof of the assertion.
\end{proof}
%

\begin{notn}
For $v \in \mathcal B'_n$ and $x \in A^i$, 
we write
$F_n(v;x)$ for
the map in $\Hom_{\F_2}(K'_n, A^i)$ sending
$v$ to $x$ and other basis vectors in $\mathcal B'_n$ 
to zero. 
\end{notn}

\begin{lemma}\label{lemma:basis_of_Kerd3}
The set of maps 
\begin{align*}
\Sb_3 \coloneqq 
\Big\{
\partial^2(F_2(b \ot a;b)) & = F_3(b \ot a \ot b; b^2), \\
\partial^2(F_2(b \ot c;b)) & = F_3(b \ot c \ot b; b^2), \\
\partial^2(F_2(e; b)) & = F_3(b \ot e; b^2) + F_3(e \ot b; b^2), \\
\partial^2(F_2(c\ot b; a)) & = F_3(e \ot b; a^2) + F_3(c \ot b \ot a; a^2) +  F_3(c \ot b \ot c; ac), \\
\partial^2(F_2(b \ot c; a)) & = F_3(b \ot e; a^2) + F_3(a \ot b \ot c; a^2) +  F_3(c \ot b \ot c; ac), \\
\partial^2(F_2(a \ot b; a)) & = F_3(e \ot b; ac) + F_3(a \ot b \ot c; ac) +  F_3(a \ot b \ot a; a^2), \\
\partial^2(F_2(a \ot b; c)) & = F_3(e \ot b; c^2) + F_3(a \ot b \ot c; c^2) +  F_3(a \ot b \ot a; ac), \\
\partial^2(F_2(b \ot a; c)) & = F_3(b \ot e; c^2) + F_3(c \ot b \ot a; c^2) +  F_3(a \ot b \ot a; ac), \\
\partial^2(F_2(c \ot b; c)) & = F_3(e \ot b; ac) + F_3(c \ot b \ot a; ac) +  F_3(c \ot b \ot c; c^2), \\
\partial^2(F_2(b \ot c; c)) & = F_3(b \ot e; ac) + F_3(a \ot b \ot c; ac) +  F_3(c \ot b \ot c; c^2)
\Big\}
\end{align*}
%in $\Hom_{\F_2}(K'_3,A^2)$.  
is a basis of $\Ker \partial^3$. 
\end{lemma}
\begin{proof}
We consider the set $\Sb'_3$ of $\F_2$-linear maps $K'_3 \to A^2$  defined by 
\begin{align*}
\Sb'_3 \coloneqq 
\Big\{ 
F_3(b \ot a \ot b; b^2), 
F_3(b \ot c \ot b; b^2),  
F_3(b \ot e; b^2),  
F_3(e \ot b; a^2), 
F_3(b \ot e; a^2), \\
F_3(a \ot b \ot a; a^2), 
F_3(e \ot b; c^2), 
F_3(b \ot e; c^2), 
F_3(c \ot b \ot a; ac), 
F_3(b \ot e; ac)
\Big\}. 
\end{align*}
%
We observe that each element of $\Sb'_3$ occurs exactly once as a term in one of the maps in $\Sb_3$. 
Since 
$\Ah_2 \coloneqq \{a^2, b^2, c^2, ac\} \subseteq A^2$ 
is a basis of $A^2$, it follows that 
the set $\Sb_3$ is linearly independent.  
%
It remains to show that $\Sb_3$ generates $\Ker \partial^3$.  
Let $f \in \ker \partial^3$ be an arbitrary element.   
%We need to show that $f$ is an $\F_2$-linear combination of the elements of $\Sb_3$.  
%
For $v \in \Bh'_3$ and 
$x \in \Ah_2$, 
let $\varphi(v;x) \in \F_2$
be the coefficient 
such that, for all $v \in \Bh'_3$, 
\[
f(v) = \sum_{x \in \Ah_2} \varphi(v;x)x. 
\]
% 
Again, since each element of $\Sb'_3$ occurs exactly once as a term in one of the maps in $\Sb_3$, 
we can assume 
by adding a linear combination of the elements of $\Sb_3$ to $f$  
that the coefficients 
\begin{align*}
\varphi(b \ot a \ot b; b^2), 
\varphi(b \ot c \ot b; b^2),  
\varphi(b \ot e; b^2),  
\varphi(e \ot b; a^2), 
\varphi(b \ot e; a^2), \\
\varphi(a \ot b \ot a; a^2), 
\varphi(e \ot b; c^2), 
\varphi(b \ot e; c^2), 
\varphi(c \ot b \ot a; ac), 
\varphi(b \ot e; ac)
        % \varphi_{b^2, b\otimes a \otimes b} = 
        % \varphi_{b^2, b\otimes c \otimes b} =
        % \varphi_{b^2, b \otimes e}\\
        % = \varphi_{a^2, e \otimes b}
        % = \varphi_{a^2, b \otimes e}
        % = \varphi_{a^2, a \otimes b \otimes a}\\
        % = \varphi_{c^2, e \otimes b}
        % = \varphi_{c^2, b \otimes e}
        % = \varphi_{ac, c \otimes b \otimes a} \\
        % = \varphi_{ac, b \otimes e} = 0.
\end{align*}
%corresponding to elements in $\Sb_3$ 
are all zero. 
%
Now it suffices to show that then $f$ must be the zero map. 
%
Since $f \in \ker \partial^3$, 
$f$ must  satisfy
the relations in Lemma~\ref{kernel-rels}.
%
We see that \eqref{i} implies
that $\varphi(b \ot (a|c) \ot b;x) = 0$ 
for $x \neq b^2$. 
%
Hence we have 
$f(b \otimes (a|c) \otimes b) = 0$.  
%    
We also see that \eqref{vii} implies that
$\varphi(b\otimes e; b^2) = \varphi(e \otimes b; b^2) = 0$.
We therefore have
$f(b\otimes e)=0$.  
%
Equation \eqref{v} implies that 
$\varphi(a\otimes b \otimes a; c^2) =0$, 
%
and \eqref{ii} implies that
$\varphi(a\otimes b \ot a; b^2) =0$. 
This shows that
either $f(a \otimes b \otimes a) = ac$ or
$f(a\otimes b \otimes a) = 0$.  
%    
If
$f(a \otimes b \otimes a) = ac$, then
\eqref{v} implies that
$f(a\otimes b \otimes c) = c^2$. 
Now \eqref{iv} forces $f(e \otimes b) = c^2$, 
which contradicts $\varphi(e \ot b; c^2) = 0$. 
%    
We thus have
$f(a \otimes b \otimes a) = 0$.  
%
From \eqref{v} and \eqref{ii}, we then get
$f(a \otimes b \otimes c) = 0$.  
%
From \eqref{iii} we deduce that
$\varphi(e \ot b;ac) = 
    \varphi(c \ot b \ot a; ac) = 0$,
and hence  
$f(e \otimes b) = 0$.  
%
It now follows from \eqref{iii} and \eqref{ii}
that
$f(c \otimes b \otimes a) = 0$.  
%
Finally, from \eqref{vi} and \eqref{ii} we get 
$f(c \otimes b \otimes c) = 0$.  
%
This shows that $f$ vanishes on all elements of the basis $\Bh'_3$, 
and hence $f$ is the zero map.  
\end{proof}
%
Since the elements in the set $\Sb_3$ belong to the subset $\Imm \partial^2$ of $\Ker \partial^3$, 
Lemma \ref{lemma:basis_of_Kerd3} implies $\Imm \partial^2 = \Ker \partial^3$. 
This finishes the proof of Theorem \ref{thm:computationof_HH}. 
\end{proof}

% \begin{lemma}\label{lemma:span_of_Imm_d2}
% The following set of linear maps  
% \begin{align*}
% \Sb'_3 \coloneqq \Big\{ 
% \partial^2(F(b \ot a;b)), 
% \partial^2(F(b \ot c;b)), 
% \partial^2(F(e; b)) 
% \partial^2(F(c\ot b; a)), 
% \partial^2(F(b \ot c; a)), \\
% \partial^2(F(a \ot b; a)), 
% \partial^2(F(a \ot b; c)), 
% \partial^2(F(b \ot a; c)), 
% \partial^2(F(c \ot b; c)), 
% \partial^2(F(b \ot c; c)) 
% \Big\}
% \end{align*}
% forms a basis of the vector space $\Imm \partial^2$. 
% In particular, we have $\Imm \partial^2 = \Ker \partial^3$. 
% \end{lemma}
% \begin{lemma}\label{lemma:Imm_d2=Kerd3}
% We have $\Imm \partial^2 = \Ker \partial^3$. 
% \end{lemma}
% \begin{proof}
% Consider the set of maps 
% \begin{align*}
% \Sb'_3 \coloneqq 
% \Big\{
% \partial^2(F_2(b \ot a;b)) & = F_3(b \ot a \ot b; b^2), \\
% \partial^2(F_2(b \ot c;b)) & = F_3(b \ot c \ot b; b^2), \\
% \partial^2(F_2(e; b)) & = F_3(b \ot e; b^2) + F_3(e \ot b; b^2), \\
% \partial^2(F_2(c\ot b; a)) & = F_3(e \ot b; a^2) + F_3(c \ot b \ot a; a^2) +  F_3(c \ot b \ot c; ac), \\
% \partial^2(F_2(b \ot c; a)) & = F_3(b \ot e; a^2) + F_3(a \ot b \ot c; a^2) +  F_3(c \ot b \ot c; ac), \\
% \partial^2(F_2(a \ot b; a)) & = F_3(e \ot b; ac) + F_3(a \ot b \ot c; ac) +  F_3(a \ot b \ot a; a^2), \\
% \partial^2(F_2(a \ot b; c)) & = F_3(e \ot b; c^2) + F_3(a \ot b \ot c; c^2) +  F_3(a \ot b \ot a; ac), \\
% \partial^2(F_2(b \ot a; c)) & = F_3(b \ot e; c^2) + F_3(c \ot b \ot a; c^2) +  F_3(a \ot b \ot a; ac), \\
% \partial^2(F_2(c \ot b; c)) & = F_3(e \ot b; ac) + F_3(c \ot b \ot a; ac) +  F_3(c \ot b \ot c; c^2), \\
% \partial^2(F_2(b \ot c; c)) & = F_3(b \ot e; ac) + F_3(a \ot b \ot c; ac) +  F_3(c \ot b \ot c; c^2)
% \Big\}
% \end{align*}
% in $\Hom_{\F_2}(K'_3,A^2)$.  
% We observe that each element of $\Sb_3$ occurs exactly once as a term in one of the maps in $\Sb'_3$. 
% Since $\Sb_3$ is linearly independent, 
% this shows that $\Sb'_3$ is linearly independent. 
% %
% Since $\Imm \partial^2 \subseteq \Ker \partial^3$, 
% we see $\Span_{\F_2}\Sb'_3 \subseteq \Imm \partial^2$ is a ten-dimensional subspace of $\Ker \partial^3$. 
% By Lemma \ref{lemma:basis_of_Ker_d3}, we know $\dim_{\F_2}(\Ker \partial^3) = 10$.  
% Thus, we must have $\Imm \partial^2 = \Ker \partial^3$. 
% \end{proof}
%
% This finishes the proof of Theorem \ref{thm:computationof_HH}. 
% \end{proof}

% We now consider the following maps in the image of
% $\partial^2$:
% \begin{itemize}
%     \item $Z_1 := \partial^2(E_{b, b \otimes a}) 
%         = E_{b^2, b\otimes a \otimes b}$
%     \item $Z_2 := \partial^2(E_{b, b\otimes c})
%         = E_{b^2, b \otimes c \otimes b}$
%     \item $ Z_3 := \partial^2(E_{b, e})
%         = E_{b^2, b\otimes e} + E_{b^2, e \otimes b}$
%     \item $ Z_4 := \partial^2(E_{a, c\otimes b})
%         = E_{a^2, e \otimes b}
%         + E_{a^2, c \otimes b \otimes a} 
%         + E_{ac, c\otimes b\otimes c}$
%     \item $ Z_5 := \partial^2(E_{a, b\otimes c})
%         =E_{a^2, b \otimes e} 
%         + E_{a^2, a \otimes b \otimes c}
%         + E_{ac, c \otimes b \otimes c}$
%     \item $Z_6 := \partial^2(E_{a, a\otimes b})
%     = E_{ac, e \otimes b}
%     + E_{ac, a \otimes b \otimes c}
%     + E_{a^2, a \otimes b \otimes a}$
%     \item $Z_7 := \partial^2(E_{c, a \otimes b})
%         = E_{c^2, e \otimes b}
%         + E_{c^2, a \otimes b \otimes c}
%         + E_{ac, a \otimes b \otimes a}$
%     \item $Z_8 := \partial^2(E_{c, b \otimes a})
%         = E_{c^2, b \otimes e}
%         + E_{c^2, c \otimes b \otimes a}
%         + E_{ac, a \otimes b \otimes a}$
%     \item $Z_9 := \partial^2(E_{c, c \otimes b})
%         = E_{ac, e \otimes b}
%         + E_{ac, c \otimes b \otimes a}
%         + E_{c^2, c \otimes b \otimes c}$
%     \item $Z_{10} := \partial^2(E_{c, b\otimes c})
%         = E_{ac, b \otimes e}
%         + E_{ac, a \otimes b \otimes c}
%         + E_{c^2, c \otimes b \otimes c}$
% \end{itemize}

% \begin{lemma}
% The maps $Z_1, \ldots, Z_{10}$ form a 
% basis of $\ker \partial^3$.
% \end{lemma}


% By applying Proposition~\ref{HH-implies-vanishing},
% we finally obtain our main
% result.

% \begin{corollary}\label{final}
% If $C$ is a dg algebra with
% $\Hb(C) \cong \F_2[a, b, c]/(ab, bc)$ as
%     a graded algebra,
%     then the triple Massey product
%     $\langle a, b, c \rangle$ vanishes.
% \end{corollary}

%%%%%%%%%%%%%%%%

% \section{Exploring stuff}

% Now we study the algebra $B \coloneqq \F_2[a,b,c]/(ab)$. 
% It is a quadratic algebra and should be Koszul \Gereon{Koszulity to be proven!} with 
% $V = \Span_{\F_2}\{a,b,c\}$ and 
% $R = \Span_{\F_2}\{a \ot b, b \ot a, a \ot c + c \ot a, 
% b \ot c + c \ot b\}$. 
% %
% And we conjecture that
% \begin{align*}
% (H^1 \ot R) \cap (R \ot H^1) 
% =  
% \Span_{\F_2}\{ & a \ot b \ot a, b \ot a \ot b, \\
% & a \ot b \ot c + a \ot c \ot b + c \ot a \ot b \\
% & b \ot a \ot c + b \ot c \ot a + c \ot b \ot a
% \}
% \end{align*}.
% \Gereon{Though this seems small... but I don't see other elements in the intersection... to be checked by Eivind!}

% We conjecture $\HH^{3,-1}(B)=0$. 

%%%%%%%%%%%%%%%%

\begin{thebibliography}{99}

\bibitem{BKS} D.\,Benson, H.\,Krause and S.\,Schwede, {\it Realizability of modules over Tate cohomology}, Trans. Amer. Math. Soc. \textbf{356} (2004), no. 9, 3621--3668. 

%\bibitem{Berger} R.\,Berger, {\it Weakly confluent quadratic algebras}, Algeb. Represent. Theory \textbf{1} (1998), 189--213.

\bibitem{BMM} U.\,Buijs, J.\,M.\,Moreno-Fern\'andez and A.\,Murillo, {\it $A_{\infty}$-structures and Massey products}, Mediterr. J. Math. \textbf{17}, 31 (2020). 

% \bibitem{Dwyer} W.\,G.\,Dwyer, {\it Homology, Massey products and maps between groups}, J. Pure Appl. Algebra \textbf{6} (1975), 177--190. 

\bibitem{HW} M.\,Hopkins and K.\,Wickelgren, {\it Splitting varieties for triple Massey products}, J. Pure Appl. Algebra \textbf{219} (2015), 1304--1319. 

% \bibitem{Isaksen} D.\,C.\,Isaksen, {\it When is a fourfold Massey product defined?}, Proc. Amer. Math. Soc. \textbf{143} (2015), no. 5, 2235--2239. 

\bibitem{Kadeishvili82} T.\,V.\,Kadeishvili, {\it The algebraic structure in the homology of an $A_{\infty}$-algebra} (Russian. English summary) Soobshch. Akad. Nauk Gruzin. SSR \textbf{108} (1982), no.\,2, 249--252 (1983).

\bibitem{Kadeishvili88} T.\,V.\,Kadeishvili, {\it The structure of the $A_{\infty}$-algebra, and the Hochschild and Harrison cohomologies} (Russian), Trudy Tbiliss. Mat. Inst. Razmadze Akad. Nauk Gruzin. SSR \textbf{91} (1988), 19--27 (an English version is available at arXiv:math/0210331). 

\bibitem{Kadeishvili23} T.\,V.\,Kadeishvili, {\it $A_{\infty}$-algebra Structure in Cohomology and its Applications}, Lecture notes 2023, available at \url{https://doi.org/10.48550/arXiv.2307.10300}. 

\bibitem{Keller} B.\,Keller, {\it Introduction to $A$-infinity algebras and modules},  Homology Homotopy Appl. \textbf{3} (2001), 1--35. 

%\bibitem{Lefevre} K.\,Lef\`evre-Hasegawa,{\it  Sur les $A_{\infty}$-cat\'etgories}, Th\`ese de Doctorat, Universit\'e Paris VII, 2005. Available from \url{http://www.math.jussieu.fr/~keller/lefevre/publ.html}. 

% \bibitem{Luetal0} D.-M.\,Lu, J.\,H.\,Palmieri, Q.-S.\,Wu, J.\,J.\,Zhang, {\it $A_{\infty}$-algebras for ring theorists}, %Proceedings of the International Conference on Algebra
% Algebra Colloq. \textbf{11} (2004), no.1, 91--128.

%\bibitem{Luetal} D.-M.\,Lu, J.\,H.\,Palmieri, Q.-S.\,Wu, J.\,J.\,Zhang, {\it $A_{\infty}$- structure on Ext-algebras}, J. Pure Appl. Algebra \textbf{213} (2009), no.11, 2017--2037. 

\bibitem{Merkulov} S.\,A.\,Merkulov, {\it Strong homotopy algebras of a K\"ahler manifold}, Internat. Math. Res. Notices (3) (1999), 153--164. 

% \bibitem{MS1} A.\,Merkurjev and F.\,Scavia, {\it Degenerate fourfold Massey products over arbitrary fields}, %arXiv:2208.13011 (2022), to appear in 
% J. Eur. Math. Soc., published online. 

\bibitem{MS2} A.\,Merkurjev and F.\,Scavia, {\it The Massey Vanishing Conjecture for fourfold Massey products modulo 2}, arXiv:2301.09290 (2023), 
to appear in Ann.~Sci.~\'Ec.~Norm.~Sup\'er.

% \bibitem{MS2023} A.\,Merkurjev and F.\,Scavia, {\it Non-formality of Galois cohomology modulo all primes}, arXiv:2309.17004 (2023), to appear in Compos.~Math. 

% \bibitem{MT0} J.\,Min\'a\v c and N.\,D.\,T\^an, {\it The kernel unipotent conjecture and the vanishing of Massey products for odd rigid fields}, Adv.~Math.~\textbf{273} (2015), 242--270.

% \bibitem{MT1} J.\,Min\'a\v c and N.\,D.\,T\^an, {\it Triple Massey products vanish over all fields}, J. Lond. Math. Soc. (2) \textbf{94} (2016), no.\,3, 909--932.

\bibitem{MT2} J.\,Min\'a\v c and N.\,D.\,T\^an, {\it Triple Massey products and Galois theory}, J. Eur. Math. Soc. \textbf{19} (2017), 255--284.

% \bibitem{MPQT} J.\,Min\'a\v c, F.\,W.\,Pasini, C.\,Quadrelli and N.\,D.\,T\^an, \emph{Koszul algebras and quadratic duals in Galois cohomology}, Adv. Math. \textbf{380} (2021), Paper No. 107569, 49 pp. 

\bibitem{PQ1} A.\,P\'al and G.\,Quick, {\it Real projective groups are formal}, Math.~Ann. \textbf{392} (2025), 1833--1876. 

%\bibitem{PQ2} A.\,P\'al and G.\,Quick, {\it $A_3$-formality for Demshkin groups at odd primes}, preprint, soon on the arXiv. \Gereon{item to be fixed!} 

%\bibitem{PoVi} L.\,Positselski and A.\,Vishik, {\it Koszul duality and Galois cohomology}, Mathematical Research Letters \textbf{2} (1995), 771--781. 

\bibitem{ppqa} A.\,Polishchuk and L.\,Positselski, {\it Quadratic algebras}, University Lecture Series, 37. American Mathematical Society, Providence, RI, 2005. xii+159 pp.

% \bibitem{PoMoscow} L.\,Positselski, {\it Mixed Artin--Tate motives with finite coefficients}, Moscow Math. J. \textbf{11} (2011), 317--402. 

% %\bibitem{PoGalois} L.\,Positselski, {\it Galois cohomology  of a number field is Koszul}, J. Number Theory \textbf{145} (2014), 126--152.

% \bibitem{Po} L.\,Positselski, {\it Koszulity of cohomology $= K(\pi,1)$-ness + quasi-formality}, J. Algebra \textbf{483} (2017), 188--229.

% \bibitem{Quadrelli} C.\,Quadrelli, {\it Massey products in Galois cohomology and the Elementary Type Conjecture}, J. Number Theory \textbf{258} (2024), 40--65.

\bibitem{ST} P.\,Seidel and R.\,Thomas, {\it Braid group actions on derived categories of coherent sheaves}, Duke Math. J. \textbf{108} (2001), no.\,1, 37--108. 

% \bibitem{vandenbergh} M.\,Van den Bergh, {\it Non-commutative homology of some three-dimensional quantum spaces}, K-Theory \textbf{8} (1994), 213--220.

% \bibitem{Witherspoon} S.\,J.\,Witherspoon, {\it Hochschild cohomology for algebras}, Graduate Studies in Mathematics, \textbf{204}, American Mathematical Society, Providence, RI, 2019.
 
\end{thebibliography}

\end{document}
