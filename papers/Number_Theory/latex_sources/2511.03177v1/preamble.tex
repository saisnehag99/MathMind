\usepackage[margin=20truemm]{geometry}
\usepackage{amsmath,amssymb,amsthm}
\usepackage{float}
\usepackage{shuffle}
\usepackage{multicol}
\usepackage{ascmac}
\usepackage{mathtools}
\usepackage{bm}
\usepackage{amscd}
\usepackage{comment}
\usepackage{mathrsfs}
\usepackage[all]{xy}
\usepackage{color}
\usepackage[dvipsnames]{xcolor}
\usepackage{setspace}
\usepackage{cases}
\usepackage{paralist}
\usepackage{autobreak}
\usepackage{enumitem}
\usepackage{stackrel}

%\definecolor{Mycolor}{cmyk}{ 1.00, 0.00, 1.00, 0.2} %緑基調
%使いたい色を使ってください。
%\definecolor{Mycolor}{cmyk}{ 1.00, 0.50, 0.00, 0.0}  %青基調
%\definecolor{Mycolor}{cmyk}{ 0.1, 0.15, 1.0, 0.1} %黄色基調
%\definecolor{Mycolor}{cmyk}{ 0.20, 1.00, 1.00, 0.} %赤基調
\definecolor{Mycolor}{cmyk}{ 0.90, 0.00, 0.34, 0.35} %碧

% ==== hyperref はエンジンで分岐 ====
\usepackage{iftex}
\ifptex
  % pLaTeX / upLaTeX(DVI→dvipdfmx)のとき
  \usepackage[dvipdfmx,bookmarksnumbered,colorlinks]{hyperref}
  \usepackage{pxjahyper}
\else
  % pdfLaTeX / LuaLaTeX(PDF直出力)のとき
  \usepackage[bookmarksnumbered,colorlinks]{hyperref} % ドライバ指定しない
\fi

\hypersetup{%
 setpagesize=false,%
 bookmarks=true,%
 bookmarksdepth=tocdepth,%
 bookmarksnumbered=true,%
 colorlinks=true,
 linkcolor=Mycolor,   % 内部リンク(式番号・図表番号・目次項目)
 citecolor=blue,   % 文献引用
 %urlcolor=blue,    % URL
 %filecolor=blue    % ファイルリンク(必要なら)
 pdftitle={},%
 pdfsubject={},%
 pdfauthor={Takumi Anzawa},%
 pdfkeywords={SMZVs}}





\title[A Lie algebra associated with AMZVs]{A Lie algebra associated with adjoint multiple zeta values}
\author{Takumi Anzawa}
\address{graduate school of mathematics, Nagoya University, furo-cho, chikusa-ku, Nagoya, 464-8602, japan}
\email{m20001s@math.nagoya-u.ac.jp}


% カスタムカウンタの定義
\newcounter{dfcounter}
\renewcommand{\thedfcounter}{\thesection}

% カウンタの同期
\makeatletter
\@addtoreset{dfcounter}{chapter}    % 章が変わったらリセット
\@addtoreset{dfcounter}{section} % 小節が変わったらリセット
\makeatother

% 定理環境の設定
\newtheorem{df}{Definition}[dfcounter]
\newtheorem{thm}[df]{Theorem}
\newtheorem{cor}[df]{Corollary}
\newtheorem{axi}[df]{\pmb{Axiom}}
\newtheorem{conj}[df]{Conjecture}
\newtheorem{qu}[df]{Question}
\newtheorem*{mt}{Main Theorem}


\theoremstyle{remark}{
\newtheorem{lem}[df]{Lemma}
\newtheorem{prop}[df]{Proposition}
\newtheorem{ex}[df]{Example}
\newtheorem{rmk}[df]{Remark}
}


\DeclareMathOperator{\Hom}{Hom}
\DeclareMathOperator{\Aut}{Aut}
\DeclareMathOperator{\id}{id}
\DeclareMathOperator{\SL}{SL}
\DeclareMathOperator{\Mn}{Mn}
\DeclareMathOperator{\Gal}{Gal}
\DeclareMathOperator{\Span}{Span}
\DeclareMathOperator{\Fil}{Fil}


\DeclareMathOperator{\sgn}{sgn}
\DeclareMathOperator{\ord}{ord}



%\DeclareMathOperator{\Re}{Re}
%\DeclareMathOperator{\Cl}{Cl}
\newcommand{\dtext}[1]{\textbf{\textit{#1}}\;}
\newcommand{\ide}[1]{\mathfrak{#1}}
\newcommand{\pf}{(pf)}
\newcommand{\ca}[1]{\mathscr{#1}}
\newcommand{\cat}[1]{\mathbf{#1}}
\newcommand{\del}{\partial}
\newcommand{\qbinom}[2]{\left[ #1 \atop #2 \right]}
\newcommand{\ibtext}[1]{\textit{\textbf{#1}}}
\newcommand{\btext}[1]{\textbf{#1}}
\newcommand{\A}{\mathscr{A}}

\DeclareMathOperator{\dch}{dch}


\newcommand{\engMonth}{
	\ifcase\month
		\or January 
		\or February 
		\or March 
		\or April
		\or May 
		\or June
		\or July 
		\or August 
		\or September
		\or October 
		\or November 
		\or December\fi
} 
% write date in english(for reports in japanese)
\newcommand{\fullday}{
	\engMonth \, \the\day, \the\year	
} 

% mathmatical template comand
% the set of natural number
\newcommand{\N}{\mathbb{Z}_{>0}}
% the set of integers
\newcommand{\Z}{\mathbb{Z}}
% the set of rational numbers
\newcommand{\Q}{\mathbb{Q}}
% the set of real numbers 
\newcommand{\R}{\mathbb{R}}
% the set of complex numbers
\newcommand{\C}{\mathbb{C}}
% a field
\newcommand{\K}{\mathbb{K}}
% the power of the set
\newcommand{\powerof}{\mathcal{P}}
% left or right hand side. 
\newcommand{\RHS}{
	\left(\textrm{RHS}\right)
}
\newcommand{\LHS}{
	\left(\textrm{LHS}\right)
}

% mapping
% the image of mapping
\let\Im\relax
\newcommand{\Im}{\textrm{Im}}

%double shuffle
\newcommand{\DMR}{\mathrm{DMR}}
\newcommand{\dmr}{\mathfrak{dmr}}
\newcommand{\ls}{\mathfrak{ls}}


%Adjoint side
\newcommand{\AdDMR}{\mathrm{AdDMR}}
\newcommand{\addmr}{\mathfrak{addmr}}
\newcommand{\adls}{\mathfrak{adls}}
\newcommand{\der}{\mathfrak{der}}


%Adjoint maps
\newcommand{\Ad}{\mathrm{Ad}}
\newcommand{\ad}{\mathfrak{ad}}

% ring
\newcommand{\fA}{\mathcal{A}}
\newcommand{\Qi}{\mathcal{C}_{0}^{\A}}
\newcommand{\fQ}{\mathcal{Q}}
\newcommand{\fP}{\mathcal{P}^{0}_{\fA}}
\newcommand{\fa}[2]{\alpha^{\fA}(#1; #2)}
\newcommand{\seq}[2]{s(#1; #2)}
\newcommand{\resi}[2]{\left(\frac{#1}{#2}\right)}
\newcommand{\Prime}{P}
\newcommand{\rep}[1]{\left\langle\left\langle#1\right\rangle\right\rangle}
\newcommand{\qint}[1]{[#1]_{q}}
\newcommand{\floor}[1]{\lfloor #1 \rfloor}
\newcommand{\lsym}[1]{\left(\frac{#1}{5}\right)}
\DeclareMathOperator{\BCH}{BCH}



\newcommand{\TM}{\mathrm{TM}}
\newcommand{\tm}{\mathfrak{tm}}
\newcommand{\Fad}{\mathfrak{F}_2^{\ad(x_1)}}
\newcommand{\FAd}{\mathfrak{F}_2^{\Ad(x_1)}}

\DeclareMathOperator{\inv}{inv}


%Linear operation
\DeclareMathOperator{\GL}{GL}
\newcommand{\gl}{\mathfrak{gl}}
\newcommand{\cotimes}{\widehat{\otimes}}


\newcommand{\AdTM}{\mathrm{AdTM}}
\newcommand{\adtm}{\mathfrak{adtm}}

%EDS
\DeclareMathOperator{\EDS}{EDS}



%MZV
\let\S\relax
\newcommand{\S}{\mathcal{S}}
\let\LS\relax
\newcommand{\LS}{\mathcal{LS}}
\newcommand{\zs}{\zeta_{\mathcal{S}}}
\newcommand{\zsl}[1]{\zeta_{\mathcal{S}_{#1}}}

\newcommand{\slmzv}[2]{\zeta_{\mathcal{S}}(#2;#1)}
\newcommand{\bslmzv}[2]{\zeta_{\mathcal{S}}^{\bullet}(#2;#1)}
\newcommand{\sslmzv}[2]{\zeta_{\mathcal{S}}^{\shuffle}(#2;#1)}
\newcommand{\hslmzv}[2]{\zeta_{\mathcal{S}}^{*}(#2;#1)}
\newcommand{\fslmzv}[2]{\zeta_{\mathcal{S}}^f(#2;#1)}

\newcommand{\Admzv}[2]{\zeta_{\mathrm{Ad}}(#2;#1)}
\newcommand{\bAdmzv}[2]{\zeta_{\mathrm{Ad}}^{\bullet}(#2;#1)}
\newcommand{\sAdmzv}[2]{\zeta_{\mathrm{Ad}}^{\shuffle}(#2;#1)}
\newcommand{\hAdmzv}[2]{\zeta_{\mathrm{Ad}}^{*}(#2;#1)}
\newcommand{\fAdmzv}[2]{\zeta_{\mathrm{Ad}}^f(#2;#1)}

\newcommand{\hzs}{\zeta_{\widehat{\mathcal{S}}}}
\newcommand{\zshift}[1]{\zeta_{\mathrm{sft},{#1}}}
\newcommand{\RS}{\mathcal{RS}}
\DeclareMathOperator{\sft}{sft}

%MZV

\newcommand{\cmzv}[4]{\zeta\left(\begin{matrix} {#1} \hspace{-0.3cm}&,\ldots,&\hspace{-0.3cm} {#2} \\ {#3} \hspace{-0.3cm}&,\ldots,&\hspace{-0.3cm}{#4}\end{matrix} \right)}

\newcommand{\csmzv}[4]{\zeta^{\shuffle}\left(\begin{matrix} {#1} \hspace{-0.3cm}&,\ldots,&\hspace{-0.3cm} {#2} \\ {#3} \hspace{-0.3cm}&,\ldots,&\hspace{-0.3cm}{#4}\end{matrix} \right)}

\newcommand{\chmzv}[4]{\zeta^{*}\left(\begin{matrix} {#1} \hspace{-0.3cm}&,\ldots,&\hspace{-0.3cm} {#2} \\ {#3} \hspace{-0.3cm}&,\ldots,&\hspace{-0.3cm}{#4}\end{matrix} \right)}

\newcommand{\cbmzv}[4]{\zeta^{\bullet}\left(\begin{matrix} {#1} \hspace{-0.3cm}&,\ldots,&\hspace{-0.3cm} {#2} \\ {#3} \hspace{-0.3cm}&,\ldots,&\hspace{-0.3cm}{#4}\end{matrix} \right)}

\newcommand{\csmzvT}[5]{\zeta^{\shuffle}\left(\begin{matrix} {#1} \hspace{-0.3cm}&,\ldots,&\hspace{-0.3cm} {#2} \\ {#3} \hspace{-0.3cm}&,\ldots,&\hspace{-0.3cm}{#4}\end{matrix};{#5} \right)}

\newcommand{\chmzvT}[5]{\zeta^{*}\left(\begin{matrix} {#1} \hspace{-0.3cm}&,\ldots,&\hspace{-0.3cm} {#2} \\ {#3} \hspace{-0.3cm}&,\ldots,&\hspace{-0.3cm}{#4}\end{matrix};{#5} \right)}

\newcommand{\cslmzv}[5]{\zeta^{\shuffle}_{\sft_{#5}}\left(\begin{matrix} {#1}\hspace{-0.3cm}&,\ldots,&\hspace{-0.3cm}{#2} \\ {#3}\hspace{-0.3cm}&,\ldots,&\hspace{-0.3cm} {#4}\end{matrix} \right)}

\newcommand{\chlmzv}[5]{\zeta^{*}_{\sft_{#5}}\left(\begin{matrix} {#1}\hspace{-0.3cm}&,\ldots,&\hspace{-0.3cm}{#2} \\ {#3}\hspace{-0.3cm}&,\ldots,&\hspace{-0.3cm} {#4}\end{matrix} \right)}

%Lifted SMZV

\newcommand{\cslsmzv}[4]{\zeta^{\shuffle}_{\LS.\alpha}\left(\begin{matrix} {#1} \hspace{-0.3cm}&,\ldots,&\hspace{-0.3cm} {#2} \\ {#3} \hspace{-0.3cm}&,\ldots,&\hspace{-0.3cm}{#4}\end{matrix} \right)}

\newcommand{\chlsmzv}[4]{\zeta^{*}_{\LS,\alpha} \left(\begin{matrix} {#1} \hspace{-0.3cm}&,\ldots,&\hspace{-0.3cm} {#2} \\ {#3} \hspace{-0.3cm}&,\ldots,&\hspace{-0.3cm}{#4}\end{matrix} \right)}

\newcommand{\clsmzv}[4]{\zeta_{\LS,\alpha} \left(\begin{matrix} {#1} \hspace{-0.3cm}&,\ldots,&\hspace{-0.3cm} {#2} \\ {#3} \hspace{-0.3cm}&,\ldots,&\hspace{-0.3cm}{#4}\end{matrix} \right)}

\let\cslslmzv\relax
\newcommand{\cslslmzv}[5]{\zeta^{\shuffle}_{\LS_{#5},\alpha}\left(\begin{matrix} {#1} \hspace{-0.3cm}&,\ldots,&\hspace{-0.3cm} {#2} \\ {#3} \hspace{-0.3cm}&,\ldots,&\hspace{-0.3cm}{#4}\end{matrix} \right)}

\let\chlslmzv\relax
\newcommand{\chlslmzv}[5]{\zeta^{*}_{\LS_{#5},\alpha}\left(\begin{matrix} {#1} \hspace{-0.3cm}&,\ldots,&\hspace{-0.3cm} {#2} \\ {#3} \hspace{-0.3cm}&,\ldots,&\hspace{-0.3cm}{#4}\end{matrix} \right)}

%SMZV

\newcommand{\cssmzv}[4]{\zeta^{\shuffle}_{\S(N,\alpha)}\left(\begin{matrix} {#1} \hspace{-0.3cm}&,\ldots,&\hspace{-0.3cm} {#2} \\ {#3} \hspace{-0.3cm}&,\ldots,&\hspace{-0.3cm}{#4}\end{matrix} \right)}

\newcommand{\chsmzv}[4]{\zeta^{*}_{\S(N,\alpha)} \left(\begin{matrix} {#1} \hspace{-0.3cm}&,\ldots,&\hspace{-0.3cm} {#2} \\ {#3} \hspace{-0.3cm}&,\ldots,&\hspace{-0.3cm}{#4}\end{matrix} \right)}

\newcommand{\cbsmzv}[4]{\zeta^{\bullet}_{\S(N,\alpha)} \left(\begin{matrix} {#1} \hspace{-0.3cm}&,\ldots,&\hspace{-0.3cm} {#2} \\ {#3} \hspace{-0.3cm}&,\ldots,&\hspace{-0.3cm}{#4}\end{matrix} \right)}

\newcommand{\csslmzv}[5]{\zeta^{\shuffle}_{\S_{#5},\alpha}\left(\begin{matrix} {#1} \hspace{-0.3cm}&,\ldots,&\hspace{-0.3cm} {#2} \\ {#3} \hspace{-0.3cm}&,\ldots,&\hspace{-0.3cm}{#4}\end{matrix} \right)}

\newcommand{\chslmzv}[5]{\zeta^{*}_{\S_{#5},\alpha}\left(\begin{matrix} {#1} \hspace{-0.3cm}&,\ldots,&\hspace{-0.3cm} {#2} \\ {#3} \hspace{-0.3cm}&,\ldots,&\hspace{-0.3cm}{#4}\end{matrix} \right)}

\newcommand{\cSmzv}[4]{\zeta_{\S(N,\alpha)}\left(\begin{matrix} {#1} \hspace{-0.3cm}&,\ldots,&\hspace{-0.3cm} {#2} \\ {#3} \hspace{-0.3cm}&,\ldots,&\hspace{-0.3cm}{#4}\end{matrix} \right)}

\newcommand{\cSlmzv}[5]{\zeta_{\S_{#5},\alpha}\left(\begin{matrix} {#1} \hspace{-0.3cm}&,\ldots,&\hspace{-0.3cm} {#2} \\ {#3} \hspace{-0.3cm}&,\ldots,&\hspace{-0.3cm}{#4}\end{matrix} \right)}

%Refineed SMZV

\newcommand{\csrsmzv}[4]{\zeta^{\shuffle}_{\RS(N,\alpha)}\left(\begin{matrix} {#1} \hspace{-0.3cm}&,\ldots,&\hspace{-0.3cm} {#2} \\ {#3} \hspace{-0.3cm}&,\ldots,&\hspace{-0.3cm}{#4}\end{matrix} \right)}

\newcommand{\chrsmzv}[4]{\zeta^{*}_{\RS(N,\alpha)} \left(\begin{matrix} {#1} \hspace{-0.3cm}&,\ldots,&\hspace{-0.3cm} {#2} \\ {#3} \hspace{-0.3cm}&,\ldots,&\hspace{-0.3cm}{#4}\end{matrix} \right)}

\newcommand{\crsmzv}[4]{\zeta_{\RS(N,\alpha)} \left(\begin{matrix} {#1} \hspace{-0.3cm}&,\ldots,&\hspace{-0.3cm} {#2} \\ {#3} \hspace{-0.3cm}&,\ldots,&\hspace{-0.3cm}{#4}\end{matrix} \right)}

%FMZV

\newcommand{\cfmzv}[4]{\zeta_{\A(N,\alpha)}\left(\begin{matrix} {#1} \hspace{-0.3cm}&,\ldots,&\hspace{-0.3cm} {#2} \\ {#3} \hspace{-0.3cm}&,\ldots,&\hspace{-0.3cm}{#4}\end{matrix} \right)}

%UMZV

\newcommand{\cUmzv}[4]{\zeta_{U(N,\alpha)}\left(\begin{matrix} {#1} \hspace{-0.3cm}&,\ldots,&\hspace{-0.3cm} {#2} \\ {#3} \hspace{-0.3cm}&,\ldots,&\hspace{-0.3cm}{#4}\end{matrix} \right)}


%mould
\newcommand{\YZ}{\mathcal{Y}_{\mathbb{Z}}}

\let\neg\relax
\DeclareMathOperator{\neg}{neg}
\DeclareMathOperator{\anti}{anti}
\DeclareMathOperator{\mantar}{mantar}
\DeclareMathOperator{\pus}{pus}
\DeclareMathOperator{\push}{push}
\DeclareMathOperator{\sh}{sh}
\DeclareMathOperator{\swap}{swap}
\DeclareMathOperator{\Sh}{Sh}
\DeclareMathOperator{\mi}{mi}
\DeclareMathOperator{\ma}{ma}
\DeclareMathOperator{\vimo}{vimo}

\DeclareMathOperator{\odd}{odd}

\DeclareMathOperator{\pol}{pol}

\let\x\relax
\newcommand{\x}{\mathrm{x}}
\newcommand{\y}{\mathrm{y}}
\newcommand{\z}{\mathrm{z}}
\let\Y\relax
\newcommand{\X}{\mathrm{X}}
\let\Y\relax
\newcommand{\Y}{\mathrm{Y}}
\let\V\relax
\newcommand{\V}{\mathrm{V}}

\DeclareMathOperator{\Conv}{Conv}
\newcommand{\ConvL}{\Conv_L}
\newcommand{\ConvR}{\Conv_R}
\newcommand{\ConvLM}{\Conv_{LM}}
\newcommand{\ConvMR}{\Conv_{MR}}

\let\u\relax
\newcommand{\u}{\mathrm{u}}
\let\v\relax
\newcommand{\v}{\mathrm{v}}
\let\w\relax
\newcommand{\w}{\mathrm{w}}

\newcommand{\XZ}{\mathcal{X}_{\mathbb{Z}}}


%complete generating series
\newcommand{\cUF}{\widehat{U\mathfrak{F}_2}}
\newcommand{\cF}{\widehat{\mathfrak{F}_2}}

\newcommand{\cUFY}{\widehat{U\mathfrak{F}_Y}}
\newcommand{\cFY}{\widehat{\mathfrak{F}_Y}}

\DeclareMathOperator{\gr}{gr}
\DeclareMathOperator{\op}{op}
\newcommand{\dec}{\mathrm{dec}}


%parity
\newcommand{\parity}{\mathrm{parity}}
\newcommand{\uparity}{\underline{\mathrm{parity}}}
\newcommand{\strprty}{\mathrm{str.prty}}
\newcommand{\weakprty}{\mathrm{weak.prty}}

\def\utilde#1{\mathord{\vtop{\ialign{##\crcr
$\hfil\displaystyle{#1}\hfil$\crcr\noalign{\kern1.5pt\nointerlineskip}
$\hfil\tilde{}\hfil$\crcr\noalign{\kern1.5pt}}}}}

% #1 without #2
\newcommand{\wo}[2]{#1^{\hat{#2}}}

% math frak
%\newcommand{\fa}{\mathfrak{a}}
%\newcommand{\fb}{\mathfrak{b}}
%\newcommand{\fn}{\mathfrak{n}}
%\newcommand{\fm}{\mathfrak{m}}
\newcommand{\fp}{\mathfrak{p}}
\newcommand{\fq}{\mathfrak{q}}
%\let\endo\relax
%\newcommand{\endo}{\mathfrak{end}}


%\bibtex{C:/Users/User/Documents/Tex/reference/transcendental}


%category
\newcommand{\Qalg}{\mathbb{Q}\text{-}\mathrm{\mathbf{Alg}}}
\newcommand{\Grp}{\mathrm{\mathbf{Grp}}}
\newcommand{\Set}{\mathrm{\mathbf{Set}}}
\newcommand{\Lalg}{\mathrm{\mathbf{Lie}}\text{-}\mathrm{\mathbf{alg}}}




%%%%%%%ゼータ関連
\newcommand{\za}{\zeta_{\mathscr{A}}}
%\newcommand{\zs}{\zeta_{\mathcal{S}}}
%\newcommand{\zsl}[1]{\zeta_{\mathcal{S}_{#1}}}
\newcommand{\SMZV}{\mathcal{S}\;MZV}
\newcommand{\FMZV}{\mathcal{F}\;MZV}
\newcommand{\KZ}{\mathrm{KZ}}
\DeclareMathOperator{\dep}{dep}
\newcommand{\Li}{\mathrm{Li}}


%%%%%%%空間関連
%\newcommand{\A}{\mathscr{A}}

%%%%%%%operator関連
%\newcommand{\dch}{\mathrm{dch}}
\DeclareMathOperator{\wt}{wt}

\font\sy=cmsy10
\font\fm=eufm10
\font\xm=msam10
\font\ym=msbm10
\allowdisplaybreaks[0]
\renewcommand{\theenumi}{\roman{enumi}}

%\usepackage{lineno}
%\linenumbers

\makeatletter
\newcommand{\nsubsection}[1]{%
  \par\bigskip
  \noindent\textbf{#1}\par
  \medskip
}
\makeatother



%\addtocontents{toc}{\protect\setcounter{tocdepth}{1}}

