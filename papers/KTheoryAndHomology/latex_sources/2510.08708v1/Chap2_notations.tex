\subsection*{General notations}

\begin{itemize}
    \item Let $\QQ \subset \RR \subset \CC$ be the set of rational numbers, real numbers and complex numbers. 
    \item Let $\QQ^{\times}_{+}$ be the set of non-zero positive rational numbers. 
    \item We use $\overline{\QQ}$ to denote the set of algebraic numbers and use $\overline{\QQ}^{\times}$ to denote the set of non-zero algebraic numbers. 
    \item $=_{\overline{\QQ}^{\times}}$: for $A, B \in \CC$, $A =_{\overline{\QQ}^{\times}} B$ means that there exists a $c \in \overline{\QQ}^{\times}$ such that $A = c \cdot B$. 
    \item For a number field $F$, we use $\AAA_{F}$ to denote the ring of adeles and use $\AAA_{F, f}$ to denote the ring of finite adeles. We use $\AAA$ to denote the ring of adeles of $\QQ$ and use $\AAA_{f}$ to denote the ring of finite adeles of $\QQ$.
    \item We use the notation $\hat{\ZZ}$ to denote the ring $\varprojlim \ZZ / N \ZZ$ and use $\hat{\ZZ}^{\times}$ to denote the abelian group $\varprojlim (\ZZ / N \ZZ)^{\times}$, where $N$ runs over positive integers.
    \item We use $i$ to denote the complex number $\sqrt{-1}$.
\end{itemize}

\subsection*{Algebraic geometry}

\begin{itemize}
    \item $\mathcal{O}_{X}$: When $X$ is a scheme (resp., a complex manifold), we denote by $\mathcal{O}_{X}$ the structure sheaf (resp.,  structure sheaf of holomorphic functions) of $X$. 
    \item Let $\mathbb{S} = \mathrm{Res}_{\CC/\RR} \Gm$ be the Deligne torus. Following Deligne and Pink, we use the following convention for the equivalence of categories between algebraic representations of $\mathbb{S}$ in finite-dimensional $\RR$-vector spaces and (semisimple) $\RR$-mixed Hodge structures. Let  $(\rho, V)$ be such a representation of $\mathbb{S}$. The summand $V^{p, q} \subset V$ of Hodge type $(p, q)$ is the summand of $V$ such that $\rho(z_1, z_2)$ acts by $z_1^{-p}z_2^{-q}$ for $(z_1, z_2) \in \mathbb{S}(\CC)$. In particular, any algebraic representation $V$ of $\mathbb{S}$ with central character $c$ corresponds to a pure Hodge structure of weight $-c$. 
    \item For a $\CC$-scheme $X$, we use the notation $X^{an}$ to denote the complex analytic variety associated to $X$ through GAGA. 
    \item For a field extension $E$ of $F$ and a scheme over the field $E$, we use the notation $\mathrm{Res}_{E/F}(X)$ to denote the Weil restriction of $X$ to $F$. To be more precise, for any $F$-scheme $S$, the $S$-points of the scheme $\mathrm{Res}_{E/F}(X)$ are the $S \times_{\Spec (F)} \Spec (E)$-points of $X$. 
    \item When we say Betti cohomology of an algebraic variety over $\QQ$, we always mean Betti cohomology of the $\CC$-points of the algebraic variety. In order to simplify notation, we avoid any notation for taking $\CC$-points.
\end{itemize}

\subsection*{Group Theory}

\begin{itemize}
    \item For a positive integer $n$, we use $S_{n}$ to denote the permuation group of $n$ elements. 
    \item For an affine group scheme $G$ over a ring $k$, sometimes we write $g \in G$ to replace 
          \begin{equation*}
              g \in G(R),
          \end{equation*}
          for a $k$-algebra $R$. 
    \item For a reductive group $G$ over $\QQ$, we denote by $A_{G}$ the connected component of the identity of the real points of the maximal $\QQ$-split center of $G$.  
    \item For a reductive group $G$ over $\QQ$, let $K$ be a maximal compact group of $G(\RR)$ and denote $K_{G} = A_{G} K$. 
    \item For a Lie group $G$, we denote by $G^{+}$ the connected component of the identity in $G$. 
\end{itemize}

\subsection*{Representation theory}

\begin{itemize}
    \item For a representation $(\pi, V)$ of a group $G$, sometimes we use $\pi$ to denote the space $V$ when there is no confusion. 
    \item A representation of a $p$-adic group is always a smooth admissible $\CC$-representation.
    \item For a representation $V$ of a group (algebraic group, real Lie group, $p$-adic group or adelic group), we use $V^{\vee}$ to denote the contragredient representation of $V$. 
    \item We use the notation $\Ind$ to the denote unnormalized induction functor. 
    \item For a representation $\pi$ of a group $G$ over a field $E$, which is an extension of a field $F$, we use the notation $\mathrm{Res}_{E/F} \pi$ to denote the restriction of the representation to the field $F$. 
\end{itemize}

\subsection*{Automorphic Forms}

\begin{itemize}
    \item For a reductive group $G$ over $\QQ$ and a quasi-character $\xi$ (not necessarily unitary) of $A_{G}$,  let $\mathrm{L}^{2}(G(\QQ) \backslash G(\AAA), \xi)$ be the space of functions $\phi \colon G(\QQ) \backslash G(\AAA) \rightarrow \CC$ such that 
    \begin{equation*}
        \phi(zx) = \xi(z) \phi(x), \, \, \, z \in A_{G}, x \in G(\AAA), 
    \end{equation*}
    and which are square integrable modulo $A_{G}$. 
    \item Let $\mathrm{C}^{\infty}(G(\QQ) \backslash G(\AAA), \xi)$, respectively $\mathrm{C}_{c}^{\infty}(G(\QQ) \backslash G(\AAA), \xi)$ be the space of functions $\phi \colon G(\QQ) \backslash G(\AAA) \rightarrow \CC$ such that 
        \begin{itemize}
            \item $\phi(zx) = \xi(z) \phi(x), \, \, \, z \in A_{G}, x \in G(\AAA)$, 
            \item the restriction of $\phi$ to $G(\RR)$ is smooth, respectively, smooth and compactly supported modulo $A_{G}$, 
            \item the restriction of $\phi$ to $G(\AAA_f)$ is locally constant and compactly supported. 
        \end{itemize}
    \item Let $\mathrm{C}_{(2)}^{\infty}(G(\QQ) \backslash G(\AAA), \xi)$ be the space 
          \begin{equation*}
              \mathrm{L}^{2}(G(\QQ) \backslash G(\AAA), \xi) \cap \mathrm{C}^{\infty}(G(\QQ) \backslash G(\AAA), \xi). 
          \end{equation*}
          We have inclusions of  $((\lieg_{\CC}, K_{G}) \times G(\AAA_f))$-modules 
          \begin{equation*}
              \mathrm{C}_{c}^{\infty}(G(\QQ) \backslash G(\AAA), \xi) \subset \mathrm{C}_{(2)}^{\infty}(G(\QQ) \backslash G(\AAA), \xi) \subset \mathrm{C}^{\infty}(G(\QQ) \backslash G(\AAA), \xi). 
          \end{equation*}
    \item Let $\mathrm{C}_{\mathrm{cusp}}^{\infty}(G(\QQ) \backslash G(\AAA), \xi) = \mathrm{L}^{2}_{\mathrm{cusp}}(G(\QQ) \backslash G(\AAA), \xi)$ be the space of cusp forms in 
    \begin{equation*}
        \mathrm{C}^{\infty}(G(\QQ) \backslash G(\AAA), \xi). 
    \end{equation*}
    Smooth truncation to a large compact set modulo $A_{G}$ induces a map of $((\lieg_{\CC}, K_{G}) \times G(\AAA_f))$-modules 
    \begin{equation*}
        \mathrm{C}_{\mathrm{cusp}}^{\infty}(G(\QQ) \backslash G(\AAA), \xi) \rightarrow \mathrm{C}_{c}^{\infty}(G(\QQ) \backslash G(\AAA), \xi). 
    \end{equation*}
    \item An irreducible cuspidal automorphic representation $\pi = \pi_{f} \otimes \pi_{\infty}$ is an irreducible $((\lieg_{\CC}, K_{G}) \times G(\AAA_f))$-submodule of $\mathrm{C}_{\mathrm{cusp}}^{\infty}(G(\QQ) \backslash G(\AAA), \xi)$. 
\end{itemize}
