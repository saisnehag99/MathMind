\newpage

%==================================================================
%  Appendix B.  Check-points of the proof  (updated, corrected)
%==================================================================

\section{Check-points of the proof}
\label{app:B}

Below we collect those places in the main text where computational details and
scaling exponents were re-verified “by hand’’ after the updates in
§§\ref{sec:narrow}, \ref{sec:balance}.  
Each item is marked by the status \texttt{OK} (checked) or by a brief
comment if a further clarification is needed.

\begin{table}[h]
  \centering
  \caption{Summary of verified check-points}
  \label{tab:checkpoints}
  \begin{tabular}{@{}p{0.36\linewidth}p{0.46\linewidth}p{0.12\linewidth}@{}}
    \toprule
    Argument node & Brief verification & Status \\ \midrule
    %------------------------------------------------------------
    Global balance (Table~\ref{tab:global-balance}) &
    \textbf{Without §\ref{sec:tube-pack}}: 
    $\sigma_\lambda=-2557/576\ (\approx-4.44)$, $\sigma_D=-3$ \,(\,matches §\ref{sec:balance}\,). \par
    \textbf{With §\ref{sec:tube-pack} instead of Robust Kakeya}: 
    $\sigma_\lambda=-3277/576\ (\approx-5.69)$, \underline{$\sigma_D=-15/4$} (regimes are \emph{mutually exclusive}). &
      \texttt{OK} \\[4pt]
    %------------------------------------------------------------
    Kernel block (6 in $t$, 6 in $x'$) &
    Raw Schur after $12$ IBP and the physical Jacobian:
    \[
      \lambda^{+3} \cdot \lambda^{-2}D^{-3} \cdot \lambda^{-3} \;=\; \lambda^{-2}D^{-3}.
    \]
    $TT^{*}$ refinement:
    \[
      (\lambda^{-2})\cdot(\lambda\alpha)^{-8}=\lambda^{-5}\ \Rightarrow\ 
      \|T\|_{2\to2}\lesssim \lambda^{-5/2},
    \]
    hence $\|K\|_{2\to2}\lesssim \lambda^{-9/2}D^{-3}$ (see §\ref{sec:kernel}). &
    \texttt{OK} \\[4pt]
    %------------------------------------------------------------
    Separation of regimes Robust\,Kakeya / Tube packing &
      The regimes are used \emph{mutually exclusively}. \par
      \emph{High density} ($>c_*D$): only Robust\,Kakeya is active, contributing $+\tfrac{1}{12}$ in $\lambda$ and $+1$ in $D$. \par
      \emph{Low density} ($\le c_*D$): only §\ref{sec:tube-pack} (tube packing) is active, contributing $-\tfrac{7}{6}$ in $\lambda$ and $+\tfrac{1}{4}$ in $D$. \par
      There is no mixing of contributions; this is precisely why in the variant “with §\ref{sec:tube-pack}’’ the final $D$–sum equals $-15/4$. &
      \texttt{OK} \\[4pt]
    %------------------------------------------------------------
    Estimate of the sum of tube overlaps (formula \eqref{eq:L2-tubes}) &
      Recomputed:
      \[
        S\;\lesssim\; D^{1/2}\lambda^{-7/3},\qquad
        \Bigl\|\sum_{\Theta} \mathbf 1_{\widetilde{\mathcal T}_{\Theta}}\Bigr\|_{L^{2}(Q_{\lambda})}
        \;\lesssim\; \lambda^{-7/6}D^{1/4}.
      \]
      Agrees with §\ref{sec:tube-pack} and is used only in the “non-robust’’ regime. &
      \texttt{OK} \\ \bottomrule
  \end{tabular}
\end{table}

\bigskip
\noindent
\textbf{Remarks.}
\begin{enumerate}[label=\textbf{\arabic*.}, wide, labelwidth=0pt, itemsep=4pt]
  \item In all items marked \texttt{OK},
        the computations reproduce the final exponents
        up to harmless constants; these nodes are considered fully closed.
  \item Additional details
        (the gradient $\partial_t\Phi$, the “six-color’’ partition algorithm,
        the 4D version of the Mark–Graun lemma, etc.)
        are in Appendices~A and C; they do not affect the cumulative exponents.
  \item Logarithmic
        factors $\log^{k}\!\lambda$ do not appear; the estimate remains \(\varepsilon\)–free.
\end{enumerate}
