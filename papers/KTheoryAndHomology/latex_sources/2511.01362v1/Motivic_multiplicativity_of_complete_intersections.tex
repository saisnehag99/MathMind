%%  Motivic multiplicativity of complete intersections
%%  Ze XU


\documentclass[12pt]{amsart}
\usepackage{amsfonts}
\usepackage{amsmath,amssymb,latexsym}
\usepackage{tikz-cd}
\usepackage[all,cmtip]{xy}

\textwidth=16cm
\topmargin=0mm
\oddsidemargin=0mm
\evensidemargin=0mm
\textheight=23cm

\def\ZZ{\mathbb Z}
\def\CC{\mathbb C}
\def\cO{\mathcal O}
\def\cS{\mathcal S}
\def\cC{\mathcal C}
\def\ra{\rightarrow}
\def\lra{\longrightarrow}
\def\PP{\mathbb P}
\def\QQ{\mathbb Q}
\def\op{\oplus}
\def\we{\wedge}
\def\ot{\otimes}
\def\a{\alpha}
\def\b{\beta}
\def\d{\delta}

\newtheorem{theorem}{Theorem}[section]
\newtheorem{proposition}[theorem]{Proposition}
\newtheorem{question}[theorem]{Question}
\newtheorem{definition}[theorem]{Definition}
\newtheorem{conjecture}[theorem]{Conjecture}
\newtheorem{lemma}[theorem]{Lemma}
\newtheorem{corollary}[theorem]{Corollary}
\newtheorem{remark}[theorem]{Remark}
\newtheorem{example}[theorem]{Example}
\newtheorem{${}$}[theorem]{${}$}

\begin{document}

\title{Motivic multiplicativity of complete intersections}

\author{Ze Xu}

\date{November 3, 2025}

\begin{abstract}
For a smooth projective variety equipped with a Chow-Künneth (abbr. CK) decomposition, 
the notions of motivic multiple twist-multiplicativity and multiplicativity defect are introduced to interpret the obstruction to the compatibility of the multiple intersection product with its CK decompositions, 
generalizing the more restrictive notion of multiplicativity introduced in \cite{SV16a}.
We establish the basic properties of these notions.
Then we show that the multiplicativity defects of curves, surfaces and ample subvarieties in varieties with trivial Chow groups have reasonable upper bounds. Furthermore, we determine explicitly the motivic 2-fold multiplicativity defect for any Fano or Calabi-Yau complete intersection in a smooth weighted projective space, strengthening a main result of \cite{F13} in the Calabi-Yau case.
Particularly, any Fano or Calabi-Yau hypersurface admits motivic 0-multiplicativity,
generalizing the case of cubic hypersurfaces proved in \cite{D21} and \cite{FLV21}, and conforming a conjecture of Voisin \cite{Voi15} in the Calabi-Yau case. 
As a consequence, certain relative powers of the corresponding universal families satisfy the Franchetta property.
We also provide several other applications.
\end{abstract}

\maketitle

\section{Introduction}
\bigskip

In the landmark paper \cite{Bei87}, based on Bloch's work \cite{B10}, Beilinson predicted that there should exist a Tannakian category of mixed motives over any field with mild properties and its derived category should contain the category of Chow motives. Consequently, the Chow motive of any smooth projective variety should have a canonical ascending filtration and hence admit many conjugate but different Chow-Künneth decompositions \cite{Ja94}, lifting the homological Künneth decompositions, generalizing Grothendieck's Künneth standard conjecture. 
Notably, the CK decompositions have been constructed for many interesting varieties, 
e.g., curves, surfaces, abelian varieties, certain hyper-Käher varieties, certain Calabi-Yau varieties, ample subvarieties in varieties with trivial Chow groups, their product, their hyperplane sections and so on, although the general conjecture is far from being solved. 
Observe that while the cup product of any Weil cohomology ring of a smooth projective variety is compatible with degree,
the naive analogue of this classical fact for its Chow motive (with the intersection product) fails generally.
Thus a natural problem is how to understand this phenomenon? 

For simplicity, we will work over the field $\mathbb{C}$ of complex numbers.
To measure the obstruction to the compatibility of the multiple intersection product of the Chow motive of any smooth projective variety with its CK decompositions, in Definition \ref{def2.3} of Section 2,
we introduce  the notions of motivic multiple twist-multiplicativity and multiplicativity defect, by taking into account the decomposition properties of the small diagonal classes.
These generalize the more restrictive notion of multiplicative CK decompositions introduced by Shen and Vial in \cite{SV16a}, because only few varieties are expected to satisfy the latter property.
A remarkable emerging phenomenon is that the motivic multiple multiplicativity defects would converge when the multiple grows, 
according to a result of Voisin \cite{Voi15} joint with Murre's conjecture (A) and (B) \cite{Mu93}.
This would allow us to define in Definition \ref{def2.12}, the notion of stable motivic multiplicativity defect 
which leads to vanishing results of the modified diagonal classes.
After establishing a criterion for motivic multiple twist-multiplicativity, 
we examine the behavior of these notions under certain geometric constructions, 
e.g. products, projective bundles and blow-ups.

It is a naturally general question to determine the motivic multiple multiplicativity defect for any variety, 
which would be a challengeable task.
For the first step, it seems plausible to propose the following.
\begin{conjecture}\label{conj1.1}
Let $X$ be a smooth projective variety of dimension $n$ equipped with a CK decomposition.
Then any CK decomposition of $X$ is $m$-fold $mn$-multiplicative for any integer $m\geq 2$.
In particular, the motivic $m$-fold multiplicativity defect $\mathfrak{d}_m(X)\leq mn$. 
\end{conjecture}
In fact, this is a direct consequence of Murre's conjecture (B). 
However, the above upper bound is generally not sharp.
So, to obtain the sharp bounds, more sophisticated investigations of the decomposition properties of the small diagonal classes for specific varieties would be desired.

The main aim of this article is to verify (part of) Conjecture \ref{conj1.1} for curves, surfaces and ample subvarieties in varieties with trivial Chow groups, and to determine explicitly the motivic 2-fold multiplicativity defect for Fano and Calabi-Yau complete intersections in certain ambient varieties.

\begin{theorem}
(i) Conjecture \ref{conj1.1} holds true for curves. For a curve $X$ of genus $g\geq 2$,
if $2\leq m\leq g$, then $\mathfrak{d}_m(X)\leq m-1$; if $m\geq g$, then $\mathfrak{d}_m(X)\leq g-1$ and hence $\mathfrak{sd}(X)\leq g-1$.

(ii) Let $X$ be any smooth connected projective surface. 
Then Murre's conjecture (B) holds true for (a set of CK projectors of) $X^{m+1}$ in codimension $2m$ 
for each integer $m\geq 2$.
As a consequence, $\mathfrak{d}_m(X)\leq 2m$.
If moreover, $X$ is regular, then $\mathfrak{d}_m(X)$ is even and $\mathfrak{d}_m(X)\leq 2m-2$.
Further, if $X$ is swept-out by (irreducible) curves of genus $g$ supporting a 0-cycle rationally equivalent to a fixed 0-cycle class $o_X$ of degree one, 
then one has $\mathfrak{sd}(X)\leq 6g+2$.

(iii) Let $Y$ be a smooth connected projective variety with trivial Chow groups 
and $X\subseteq Y$ be any smooth connected ample subvariety of dimension $n$. 
Then $\mathfrak{d}_m(X)\leq mn$ for any integer $m\geq 2$.
\end{theorem}

Subsequently, for Fano and Calabi-Yau complete intersections in certain varieties with trivial Chow groups,
we explore how their geometry impose restrictions on their motivic 2-fold multiplicativity defect. 
Among other things, we will provide a criterion to detect motivic $0$-multiplicativity for Fano and Calabi-Yau hypersurfaces in certain varieties with trivial Chow groups. 
As an sample, we will show the following theorem, which conforms Voisin's Conjecture 3.5 in \cite{Voi12} for the case of Calabi-Yau hypersurfaces.
\begin{theorem}
Any Fano or Calabi-Yau hypersurface in a smooth weighted projective space admits motivic $0$-multiplicativity.
\end{theorem}
Previously, Voisin \cite{Voi15} proved a decomposition property of the third small diagonal class for Calabi-Yau hypersurfaces by a very different method and leaves a conjecture.
The case of cubic hypersurfaces was shown simultaneously in \cite{D21} and \cite{FLV21} using distinct strategies.
From another aspect, Fano and Calabi-Yau hypersurfaces are interesting, because most of them are not expected to have abelian Grothendieck motives.

For Fano and Calabi-Yau complete intersections, we can determine explicitly their motivic 2-fold multiplicativity defect.
\begin{theorem}
Let $Y:=\mathbb{P}_w^{N}$ be any smooth weighted projective space of dimension $N=n+e$ with $e\geq 1$
and $X\subseteq Y$ be a smooth Fano or Calabi-Yau complete intersection of dimension $n$. 

(i) There is an equality of cycle classes:
\begin{equation*}
\Delta_{123}^X=\delta_{12*}P(D_1, D_2)+\delta_{13*}P(D_1, D_2)+\delta_{23*}P(D_1, D_2)
+Q(D'_1, D'_2, D'_3)
\end{equation*}
in $\text{CH}^{2n}(X^3)$, where $\Delta_{123}^X$ is the third small diagonal of $X$, $\delta_{ij}: X^2\rightarrow X^3$ are the embeddings to the big diagonals for $i<j$, both $P(s_1,s_2)\in\mathbb{Q}[s_1,s_2]$ and $Q(t_1,t_2,t_3)\in\mathbb{Q}[t_1,t_2,t_3]$ are symmetric homogenous polynomials, $D_i=p_i^*D$ and $D'_j=q_j^*D$ with $D$ the hyperplane class, $p_i$ and $q_j$ the natural projections.

(ii) A naturally constructed self-dual CK decomposition of $X$ is multiplicative 
if and only if the cycle class $\delta_{X*}D$ is completely decomposable, where $\delta_X: X\rightarrow X^2$ is the diagonal morphism.

(iii) The motivic 2-fold multiplicativity defect $\mathfrak{d}(X)=2k$ with $1\leq k\leq n-1$ and $k\neq\frac{n}{2}$
if and only if the cycle class $\delta_{X*}D^{k+1}$ is completely decomposable but $\delta_{X*}D^{k}$ is not.
\end{theorem} 
In the case of general Calabi-Yau complete intersections, the statement (i) strengthens a main result of L. Fu \cite{F13} proved by a quite different method. 

Under the Fano or Calabi-Yau condition, a new geometric input is the existence of \emph{isogenous} correspondences and relevant varieties. Our method stresses on establishing the required cycle relations by considering proper intersections of closely relevant varieties.
This highlights the importance of the geometry of isogenous correspondences on these varieties,
which can be regarded as analogues of isogenous homomorphisms of abelian varieties as indicated by Voisin \cite{Voi04}. 
Moreover, we have to point out that the approach evolves from that for K3 surfaces in \cite{Ba19}.

We will provide several applications throughout. 
To finish the introduction, we only mention the following two.
\begin{corollary}
Let $f:  \mathcal{X}\rightarrow B$ be the universal family of smooth Fano or Calabi-Yau hypersurfaces of degree $d$ in a projective space. 
Then for any positive integer $k\leq d+1$, the $k$-th relative power $f_k: \mathcal{X}_{/B}^k\rightarrow B$ of $f$
satisfies the Franchetta property, that is, for a cycle class $\Gamma\in\text{CH}^*(\mathcal{X}_{/B}^k)$,
the restriction $\Gamma|_{\mathcal{X}_b^k}$ vanishes for any closed point $b\in B$, 
if it is homologically trivial for a very general closed point $b\in B$.
\end{corollary}

\begin{corollary}
Let $X$ and $B$ be smooth varieties and $f: X\rightarrow B$ be a smooth projective morphism. 
Assume that a very general fiber of $f$ admits a multiplicative CK decomposition. 
Then after restriction to some dense Zariski open subset $U\subseteq B$, there exists a multiplicative decomposition isomorphism
\begin{equation*}
\mathbf{R}(f|_U)_*\mathbb{Q}\cong\bigoplus_i\mathbf{R}^i(f|_U)_*\mathbb{Q}[-i]
\end{equation*}
in the derived category of sheaves of $\mathbb{Q}$-vector spaces on $U$.

In particular, the universal family of Fano or Calabi-Yau hypersurfaces in a smooth weighted projective space satisfies this property.
\end{corollary}
\bigskip

\textbf{Conventions}. We work over the field $\mathbb{C}$ of complex numbers unless otherwise stated.
The word \emph{variety} will refer to a reduced separated scheme of finite type over $\mathbb{C}$.
Singular cohomology groups, Chow groups, Chow rings and Chow motives are all with rational coefficients.
Denote by $\mathfrak{h}: \mathcal{V}(\mathbb{C})\rightarrow\text{CH}\mathcal{M}$ the contravariant functor from the category of smooth complex projective varieties to the rigid pseudo-abelian tensor category of Chow motives with rational coefficients.
\bigskip

\bigskip

\section{Motivic multiple twist-multiplicativity}
\bigskip

In this section, we introduce the notions of motivic multiple twist-multiplicativity and multiplicativity defect,
study their basic properties, propose general questions and then test the first examples, 
such as curves and surfaces.

Let $X$ be a smooth connected projective variety of dimension $n$. 
Assume that Murre's conjecture (A) holds true for $X$, that is,
$X$ admits a set $\{\pi_i^X\}_{0\leq i\leq 2n}$ of CK projectors, i.e.,
$\pi_i^X$ are orthogonal idempotents lifting the Künneth projectors and $\Delta_X=\sum_{i=0}^{2n}\pi_i^X$ in the correspondence ring $\text{CH}^n(X\times X)$. (For the whole statement of Murre's conjecture, 
refer to Section 1.4 in \cite{Mu93} or Conjecture 5.1 in \cite{Ja94} for more details.)
Thus one can define an ascending filtration on the Chow motive $\mathfrak{h}(X)$ of $X$ as
\begin{equation}\label{1}
\mathfrak{h}^{\leq k}(X):=\bigoplus_{i\leq k}\mathfrak{h}^i(X).
\end{equation}
According to the philosophy of Bloch-Beilinson \cite{Bei87}, the filtration \eqref{1} should be compatible with any ($\geq 2$)-fold intersection product, that is, there is a commutative diagram in the tenor category of Chow motives:
\begin{equation}\label{2}
\begin{tikzcd}
\mathfrak{h}^{\leq i_1}(X)\otimes\cdots\otimes\mathfrak{h}^{\leq i_m}(X) \arrow{r}{\Delta_{I}^X} \arrow{d}{}
& \mathfrak{h}(X) \\
\mathfrak{h}^{\leq \sum_{\ell}i_{\ell}}(X).\arrow[hook]{ru}
\end{tikzcd}
\end{equation}
Therein $m\geq 2$, $I=I_m:=\{1, 2, \cdots, m+1\}$ and $\Delta_{I}^X\in\text{CH}^{mn}(X^{m+1})$ is the $(m+1)$-th small diagonal class of $X$. More concretly, one has the following.

\begin{conjecture}(\cite{Bei87})\label{conj2.1}
Fix an integer $m\geq 2$.
Assume $X$ has a set $\{\pi_i^X\}_{0\leq i\leq 2n}$ of CK projectors.
Then for each integer $k>\sum_{\ell=1}^{m}i_{\ell}$, 
\begin{equation}\label{3}
\pi_k^X\circ\Delta_{I}^X\circ(\pi_{i_1}^X\otimes\cdots\otimes\pi_{i_m}^X)
=(^t\pi_{i_1}^X\otimes\cdots\otimes{^t}\pi_{i_m}^X\otimes\pi_k^X)_*\Delta_{I}^X=0  
\end{equation}
in $\text{CH}^{mn}(X^{(m+1)n})$.
\end{conjecture}

\begin{remark}
(i) If $X$ has a set $\{\pi_i^X\}_{0\leq i\leq 2n}$ of CK projectors, then it always has a self-dual one (i.e., $^t\pi_i^X=\pi_{2n-i}^X$).
Indeed, this follows immediately by replacing $\pi_j^X$ by $^t\pi_{2n-j}^X$ for each integer $j>n$.

(ii) Assuming the CK decomposition is self-dual, 
then Conjecture \ref{conj2.1} amounts to that the cycle class $\Delta_{I}^X$ does not have a component in the graded piece
\begin{equation*}
\text{CH}_s^{mn}(X^{(m+1)n}):=(\pi_{2mn-s}^X)_*\text{CH}^{mn}(X^{(m+1)n})
\end{equation*}
for each integer $s<0$, which is actually a consequence of Murre's conjecture (B) for $X^3$ in codimension $2n$.
Refer to Section 1.4 in \cite{Mu93} or Conjecture 5.1 in \cite{Ja94} for more details.

(iii) Denote $\text{CH}_s^r(X):=(\pi_{2r-s}^X)_*\text{CH}^r(X)$.
Then Conjecture \ref{conj2.1} implies that 
the $m$-fold intersection product induces the map
\begin{equation*}
\text{CH}_{s_1}^{r_1}(X)\otimes\cdots\otimes\text{CH}_{s_m}^{r_m}(X)\stackrel{\bullet^{m}}\longrightarrow
\text{CH}_{\geq\sum_{\ell}s_{\ell}}^{\sum_{\ell}r_{\ell}}(X).
\end{equation*}
Roughly speaking, this means that the lower grading of Chow groups adds stretching only to right under (multiple) intersection product.
\end{remark}

In \cite{SV16a}, Shen and Vial introduced the notion of multiplicative CK decompositions 
for smooth projective varieties.
As testified by several authors, only very special classes of varieties could have this property and this notion is 
rather restrictive.
To extent this notion to all smooth projective varieties, it is worth observing that the decomposition properties of all small diagonal classes underlie the multiple intersection product of the Chow motive of a smooth projective variety. 
It also turns out that their decomposition properties are closely related and simultaneous considerations of all these are necessary to obtain overall information. In virtue of these observations, the following general notions seem appealing and suitable.

\begin{definition}\label{def2.3}
Let $m\geq 2$ be a fixed integer and put $I:=I_m=\{1, 2,\cdots, m+1\}$.
Let $\tau_m\geq 0$ be an integer.

(i) A set $\{\pi_i^X\}_{0\leq i\leq 2n}$ of CK projectors of $X$ is said to be \emph{$m$-fold (twist-)$\tau_m$-multiplicative}, if the $(m+1)$-th small diagonal class of $X$ has a decomposition:
\begin{equation}\label{4}
\Delta_{I}^X=\sum_{k=0}^{2n}\sum_{-\tau_m\leq k-\sum_{\ell}i_{\ell}\leq 0}\pi_k^X\circ\Delta_{I}^X\circ(\pi_{i_1}^X\otimes\pi_{i_2}^X\otimes\cdots\otimes\pi_{i_m}^X)
\end{equation}
in $\text{CH}^{mn}(X^{m+1})$; 
equivalently, for each integer $k>\sum_{\ell=1}^{m}i_{\ell}$ or $k<\sum_{\ell=1}^{m}i_{\ell}-\tau_m$,
\begin{equation}\label{5}
\pi_k^X\circ\Delta_{I}^X\circ(\pi_{i_1}^X\otimes\pi_{i_2}^X\otimes\cdots\otimes\pi_{i_m}^X)
=(^t\pi_{i_1}^X\otimes{^t}\pi_{i_2}^X\otimes\cdots\otimes{^t}\pi_{i_m}^X\otimes\pi_k^X)_*\Delta_{I}^X=0.
\end{equation}

(ii) The variety $X$ is said to be \emph{motivic $m$-fold (twist)-$\tau_m$-multiplicative}, if it admits a $m$-fold $\tau_m$-multiplicative CK decomposition.
For convenience, we will also refer to this as admitting motivic $m$-fold (twist)-$\tau_m$-multiplicativity 
(or motivic (multiple) twist-multiplicativity for simplicity).

(iii) The \emph{motivic $m$-fold multiplicativity defect} $\mathfrak{d}_m(X)$ of $X$ is defined to be the minimal integer such that $X$ is strictly motivic $m$-fold $\mathfrak{d}_m(X)$-multiplicative, that is,
motivic $m$-fold $\mathfrak{d}_m(X)$-multiplicative but not $k$-multiplicative for any integer $k<\mathfrak{d}_m(X)$.
One can also define a similar notion for a chosen CK decomposition, 
which reflects only the property of this CK decomposition.
According to the associative law of intersection product, one has $\mathfrak{d}_m(X)\leq (m-1)\mathfrak{d}_2(X)$.
In particular,
\begin{equation*}
\Delta_{I}\in\bigoplus_{i=0}^{m\mathfrak{d}_2(X)}\text{CH}_{i}^{nm}(X^{m+1}).
\end{equation*}

(iv) A set $\{\pi_i^X\}_{0\leq i\leq 2n}$ of CK projectors of $X$ is $m$-fold \emph{weakly} $\tau$-multiplicative, 
if for any $z_i\in\text{CH}_{s_i}^{r_{i}}(X)$ for $1\leq i\leq m$, one has
\begin{equation*}
z_1\cdots z_m\in\bigoplus_{s=0}^{\tau}\text{CH}_{\sum_{i=1}^{m}s_i+s}^{\sum_{i=1}^{m}r_i}(X).
\end{equation*}
\end{definition}

\begin{remark}
(i) It should be pointed out that for different $m$, the notions are indeed distinct by the criterion provided by Proposition \ref{prop2.15}.
The $m=2$ case is particularly interesting for obvious reasons; for convenience, we will usually omit the word '2-fold' in related notions, unless otherwise stated. In particular, $\mathfrak{d}(X)=\mathfrak{d}_2(X)$.

(ii) By definition, motivic (2-fold) 0-multiplicativity is equivalent to that $X$ admits a multiplicative CK decomposition as defined in \cite{SV16a}.

(iii) Roughly speaking, the motivic $m$-fold multiplicativity defect measures how far does a variety have a $m$-fold multiplicative CK decomposition.
The vanishing of motivic multiplicativity defect means motivic $0$-multiplicativity.

(iv) If $X$ admits a CK decomposition, then obviously $\mathfrak{d}_m(X)\leq 2mn$ and hence $X$ must admit motivic $m$-fold multiplicativity defect.

(v) One can also define similar notions modulo any equivalence relation finer than homological equivalence (e.g. algebraic equivalence). Moreover, the notions can be extended to smooth projective families.

(vi) Assume $X$ admits a set $\{\pi_i^X\}_{0\leq i\leq 2n}$ of CK projectors.
Let 
\begin{equation}\label{6}
\widetilde{\Delta}_{123}^X:=\Delta_{123}^X-\sum_{k=0}^{2n}\sum_{k\neq i+j}\pi_k^X\circ\Delta_{123}^X\circ(\pi_{i}^X\otimes\pi_{j}^X).
\end{equation}

Define the \emph{modified} intersection product: 
\begin{equation*}
\mathfrak{h}(X)\otimes\mathfrak{h}(X)\stackrel{\widetilde{\Delta}_{123}^X}\longrightarrow\mathfrak{h}(X).
\end{equation*}
Then it is clear that
\begin{equation}\label{7}
\widetilde{\Delta}_{123}^X
=\sum_{k=0}^{2n}\sum_{i+j=k}\pi_k^X\circ\widetilde{\Delta}_{123}^X\circ(\pi_{i}^X\otimes\pi_{j}^X).
\end{equation}
This means that the CK decomposition is always \emph{multiplicative} with respect to the modified intersection product.
\end{remark}

It seems natural and interesting to propose the following general question.

\begin{question}\label{ques2.5}
(i) Let $X$ be any smooth connected projective variety. Determine $\mathfrak{d}_m(X)$  for any integer $m\geq 2$, when well-defined.

(ii) Given an integer $\mathfrak{d}\geq 0$, find all smooth (connected) projective varieties with motivic multiplicativity defect $\mathfrak{d}$.
\end{question}

The first step of answering to Question \ref{ques2.5} is to bound the motivic multiple multiplicativity defect.
We will take into account the following prediction.
\begin{conjecture}\label{conj2.6}
Let $X$ be a smooth connected projective variety of dimension $n$ equipped with a CK decomposition.
Then any CK decomposition of $X$ is $m$-fold $mn$-multiplicative for any integer $m\geq 2$.
In particular, $\mathfrak{d}_m(X)\leq mn$. 
\end{conjecture}

This conjecture is actually a direct consequence of Murre's conjecture (B).
Indeed, let $\{\pi_i\}_{0\leq i\leq 2n}$ be a set of CK projectors of $X$.
If $k>\sum_{\ell=1}^{m}i_{\ell}$ or $k<\sum_{\ell=1}^{m}i_{\ell}-mn$, 
then $2mn+k-\sum_{\ell=1}^{m}i_{\ell}>2nm$ or $2mn+k-\sum_{\ell=1}^{m}i_{\ell}<nm$.
Thus, Murre's conjecture (B) for $X^{m+1}$ in codimension $mn$ implies that
\begin{equation*}
\pi_k\circ\Delta_I\circ(\pi_{i_1}\otimes\cdots\otimes\pi_{i_m})
=(\pi_{2n-i_1}\otimes\cdots\otimes\pi_{2n-i_m}\otimes\pi_k)_*\Delta_I=0.
\end{equation*}
\begin{remark}
(i) Even for $m=2$, a variety may have a strictly motivic $2n$-multiplicative CK decomposition.
Indeed, let $X$ be any curve of genus $g\geq 1$ and take any two points $a_1, a_2\in X$ such that they are not rationally equivalent as cycle. Denote $o_i:=[a_i]$. Let $\pi_0:=o_1\times X$, $\pi_2:=X\times o_2$ and $\pi_1:=\Delta_X-\pi_0-\pi_2$.
Then $\{\pi_i\}_{0\leq i\leq 2}$ forms a set of CK projectors of $X$.
Observe that
\begin{equation*}
\pi_0\circ\Delta_{I}\circ(\pi_1\otimes\pi_1)=(\pi_1\otimes\pi_1\otimes\pi_0)_*\Delta_{I}
=(o_1-o_2)\times(o_1-o_2)\times X\neq 0.
\end{equation*}
Therefore, this CK decomposition is strictly 2-multiplicative by Theorem \ref{thm2.24}.
This indicates that to attain $\mathfrak{d}_m(X)$, a more natural CK decomposition should be constructed, presumably at least a self-dual one.

(ii) Clearly, the inequality $\mathfrak{d}_m(X)\leq mn$ is not sharp.
Only for small values of $m$ (depending on the dimension $n$),  it may be almost sharp for all smooth projective varieties of dimension $n\geq 2$. As it turns out, a smaller bound can be obtained for special classes of varieties (e.g. complete intersections). 

(iii) For a fixed variety, we will see that if $m$ is large enough, then $\mathfrak{d}_m$ must converge (to an integer).
\end{remark}

For different $m$, the notions of motivic $m$-fold twist-multiplicativity are closely related to each others.
\begin{proposition}
Assume for some integer $r\geq 3$, the variety $X$ admits a $r$-fold $\tau_{r}$-multiplicative self-dual CK decomposition with $\pi_{0}^X=o_X\times X$ for some 0-cycle class $o_X$ of degree one on $X$.
Then for $2\leq m\leq r$, this CK decomposition is $m$-fold $\tau_{r}$-multiplicative.
\end{proposition}
\begin{proof}
By induction on $r-\ell$, it suffices to consider the case $\ell=r-1$. Set $J:=I_{r-1}=\{1, 2, \cdots, r\}$.
By assumption, for each integer $k>\sum_{j=2}^{r}i_j$ or $k<\sum_{j=2}^{r}i_j-\tau_r$, 
one gets that
\begin{eqnarray*}
% \nonumber to remove numbering (before each equation)
0=\pi_k^X\circ\Delta_{I}^X\circ(\pi_{0}^X\otimes\pi_{i_2}^X\otimes\cdots\otimes\pi_{i_r}^X) &=& ({}^t\pi_{0}^X\otimes{{}^t}\pi_{i_2}^X\otimes\cdots\otimes{{}^t}\pi_{i_r}^X\otimes\pi_k^X)_*\Delta_{I}^X \\
   &=& (\pi_{2n}^X\otimes\pi_{2n-i_2}^X\otimes\cdots\otimes\pi_{2n-i_r}^X\otimes\pi_k^X)_*\Delta_{I}^X \\
   &=& o_X\times (\pi_{2n-i_2}^X\otimes\cdots\otimes\pi_{2n-i_r}^X\otimes\pi_k^X)_*\Delta_{J}^X.
\end{eqnarray*}
Then 
\begin{equation*}
\pi_k^X\circ\Delta_{J}^X\circ(\pi_{i_2}^X\otimes\cdots\otimes\pi_{i_r}^X)
=(\pi_{2n-i_2}^X\otimes\cdots\otimes\pi_{2n-i_r}^X\otimes\pi_k^X)_*\Delta_{J}^X=0.
\end{equation*}
This means that the CK decomposition is $(r-1)$-fold $\tau_r$-multiplicative.
\end{proof}

\begin{remark}
As a consequence, the motivic $m$-fold multiplicativity defect $\mathfrak{d}_m(X)$ is a monotonically increasing function in $m$.
\end{remark}

A remarkable phenomenon is that Murre's conjecture (A) and (B) joint with a result of Voisin \cite{Voi15} implies that the motivic $m$-fold multiplicativity defect $\mathfrak{d}_m(X)$ converges when $m$ grows.
\begin{proposition}\label{prop2.10}
Assume $X$ has a set $\{\pi_i^X\}_{0\leq i\leq 2n}$ of self-dual CK projectors with $\pi_{0}^X=o_X\times X$ for some 0-cycle class $o_X$ of degree one.
Suppose that Murre's conjecture (B) holds true for the product CK projectors of $X^{l+1}$ 
in codimension $ln$ for any integer $l\geq 2$.
Then there exist integers $m\geq 2$ and $\tau\geq 0$ such that for any integer $k\geq m$, the CK decomposition is $k$-fold $\tau$-multiplicative. So, one has $\mathfrak{d}_k(X)\leq\tau$.
\end{proposition}
\begin{proof}
Observe  that for integers $i\geq 2$ and $j\geq 1$, one has
\begin{equation*}
\Delta_{I_{i-1}}\times\underbrace{o_X\times\cdots\times o_X}_j
=(\Delta_{X^{i}}\otimes\underbrace{\pi_{2n}^X\otimes\cdots\otimes\pi_{2n}^X}_j)_*\Delta_{I_{i+j-1}}
\in\text{CH}^{n(i+j-1)}(X^{i+j}).
\end{equation*}
Now if $\Delta_{I_{i-1}}\in\bigoplus_{s=0}^{\ell}\text{CH}_s^{n(i-1)}(X^{i})$ for some integer $\ell\geq 0$, then 
\begin{equation*}
\Delta_{I_{i-1}}\times\underbrace{o_X\times\cdots\times o_X}_j\in\bigoplus_{s=0}^{\ell}\text{CH}_s^{n(i+j-1)}(X^{i+j}).
\end{equation*}
By symmetry, a similar statement also holds for the permutations of this cycle class.

Denote the $k$-th modified small diagonal class of $(X, o_X)$ by
\begin{equation*}
\Gamma^k(X, o_X):=\sum_{J\subseteq\{1,\ldots,k\}, |J|=j<k}(-1)^jp_{J}^*(o_X^{*j})\cdot p_{J'}^*\Delta_{I_{m-j-1}}\in\text{CH}_n(X^k),
\end{equation*}
where $\{1,\ldots,k\}$ is the disjoint union of $J$ and $J'$, $p_J: X^k\rightarrow X^j$ and $p_{J'}: X^k\rightarrow X^{k-j}$ are natural projections to the products of factors indexed by $J$ and $J'$ respectively, and $o_X^{*j}=p_1^*o_X\cdots p_j^*o_X$.
Invoking Voisin's result Corollary 1.6 in \cite{Voi15}, 
there exists an integer $m\geq 2$ such that $\Gamma^k(X, o_X)$ vanishes for any integer $k>m$. 
Murre's conjecture (B) for $X^m$ and the cycle class $\Delta_{I_{m-1}}$ implies that 
\begin{equation*}
\Delta_{I_{m-1}}\in\bigoplus_{s=0}^{\tau}\text{CH}_s^{nm}(X^{m})
\end{equation*}
for some integer $\tau\in [0, nm]$. 
Therefore,
\begin{equation*}
\Delta_{I_{k-1}}\in\bigoplus_{s=0}^{\tau}\text{CH}_s^{n(k-1)}(X^{k}).
\end{equation*}
So, the CK decomposition is $k$-fold $\tau$-multiplicative for any $k\geq m$.
\end{proof}
\begin{remark}\label{rem2.11}
By Theorem 1.7 in \cite{Voi15}, if $X$ is swept-out by (irreducible) curves of genus $g$ supporting a 0-cycle rationally equivalent to $o_X$, then $\Gamma^k(X, o_X)=0$ for any integer $k\geq (n+1)(g+1)$.
By the above argument, one can take $m\leq (n+1)(g+1)-1$ in Proposition \ref{prop2.10}.
Hence, $\mathfrak{d}_m\leq n[(n+1)(g+1)-2]$, assuming Conjecture \ref{conj2.6}. 
However, this upper bound should generally not sharp, except few special cases.
\end{remark}

Now it makes sense to introduce the following notion.
\begin{definition}\label{def2.12}
Assume $X$ admits a CK decomposition.
The \emph{stable motivic multiplicativity defect} of $X$ is defined to be
\begin{equation*}
\mathfrak{s}\mathfrak{d}(X):=\text{max}\{\mathfrak{d}_m(X)|m\geq 2\}.
\end{equation*}
\end{definition}
\begin{remark}\label{rem2.13}
(i) If well-defined, then $\mathfrak{d}(X)\leq\mathfrak{sd}(X)$. 
Moreover, assuming Murre's conjecture (B) for $X^{k+1}$ in codimension $kn$ for any integer $k\geq 2$,
one has 
$$
\mathfrak{sd}(X)\leq n[(n+1)(g+1)-2]
$$ 
by Remark \ref{rem2.11}.

(ii) If $\mathfrak{d}(X)=0$, then $\mathfrak{sd}(X)=0$.

(iii) One can also define a similar notion for a chosen CK decomposition.

(iv) It seems unclear how to determine (a sharp upper bound of) $\mathfrak{sd}(X)$ for $X$ any smooth connected projective variety, when well-defined.
\end{remark}

For some classes of varieties, one can bound their stable motivic multiplicativity defect.
The following statement is a consequence of Voisin's result in \cite{Voi15}.
\begin{proposition}\label{prop2.14}
Let $X$ be a smooth projective rationally connected variety with a CK decomposition. 
Then $\mathfrak{sd}(X)\leq 2n(n-1)$.
If in addition, Murre's conjecture (B) holds true for $X^n$, then $\mathfrak{sd}(X)\leq n(n-1)$.
\end{proposition}
\begin{proof}
Since $\text{CH}^n(X)=\mathbb{Q}$, then one can take any point on $X$ to represent $o_X$.
By Corollary 3.2 in \cite{Voi15}, the modified diagonal class $\Gamma^{k}(X, o_X)=0$ for any integer $k>n$.
Since $\Delta_{I_{n-1}}\in\text{CH}^{n(n-1)}(X^n)$, then $\mathfrak{sd}(X)\leq 2n(n-1)$.
Moreover, if Murre's conjecture (B) holds true for $X^n$, then $\Delta_{I_{n-1}}\in\bigoplus_{s=0}^{n(n-1)}\text{CH}_s^{n(n-1)}(X^n)$
and hence $\mathfrak{sd}(X)\leq n(n-1)$.
\end{proof}

Now we provide a simple criterion to detect motivic multiple twist-multiplicativity, generalizing Proposition 8.4 in \cite{SV16a}.
\begin{proposition}\label{prop2.15}
Let $m\geq 2$ be a fixed integer and put $I:=I_m=\{1, 2,\cdots, m+1\}$.
Let $\tau_m\geq 0$ be an integer.
Assume $X$ admits a set $\{\pi_i^X\}_{0\leq i\leq 2n}$ of self-dual CK projectors.
For integers $\ell, r, s$, set 
\begin{equation*}
\pi_{\ell}^{X^{m+1}}:=\sum_{i_1+\cdots+i_{m+1}=\ell}\pi_{i_1}^X\otimes\cdots\otimes\pi_{i_{m+1}}^X,\quad
\text{CH}_s^{r}(X^{m+1}):=(\pi_{2r-s}^{X^{m+1}})_*\text{CH}^r(X^{m+1}).
\end{equation*}
Write uniquely
\begin{equation}\label{8}
\Delta_{I}^X=\sum_{s=-2n}^{2mn}\Delta_{I}^s, \quad \Delta_{I}^s:=(\pi_{2mn-s}^{X^{m+1}})_*\Delta_{I}^X\in\text{CH}_s^{mn}(X^{m+1}).
\end{equation}
Then the following three statements are equivalent:

(i) the corresponding CK decomposition of $X$ is $m$-fold $\tau_m$-multiplicative;

(ii) the cycle class $\Delta_{I}^s$ vanishes for each integer $s<0$ or $s>\tau_m$;

(iii) there is an equality of cycle classes: 
$$
\Delta_{I}=\sum_{s=0}^{\tau_m}\Delta_{I}^s.
$$
\end{proposition}
\begin{proof}
Since $^t\pi_i^X=\pi_{2n-i}^X$, then 
\begin{equation*}
\pi_k^X\circ\Delta_{I}^X\circ(\pi_{i_1}^X\otimes\cdots\otimes\pi_{i_m}^X)
=(\pi_{2n-i_1}^X\otimes\cdots\otimes\pi_{2n-i_m}^X\otimes\pi_k^X)_*\Delta_{I}^X
\in\text{CH}_{\sum_{\ell}i_{\ell}-k}^{mn}(X^{m+1}).
\end{equation*}
It suffices to notice that $k>\sum_{\ell}i_{\ell}$ or $k<\sum_{\ell}i_{\ell}-\tau_m$ is equivalent to that $\sum_{\ell}i_{\ell}-k<0$ or $\sum_{\ell}i_{\ell}-k>\tau_m$.
\end{proof}
\begin{remark}
In many situations (e.g., complete intersections), one could easily verify that motivic multiple twist-multiplicativity is preserved under specialization.
\end{remark}

Next, we establish fundamental properties of motivic multiple twist-multiplicativity related to several basic geometric constructions, 
including products, projective bundles and blow-ups.

\begin{proposition}\label{prop2.17}
Let $X$ (resp. $X'$) be a smooth connected projective variety of dimension $n$ (resp. $n'$) admitting an $m$-fold $\tau_m$-multiplicative  (resp. $\tau'_m$-multiplicative) CK decomposition $\{\pi_i^X\}_{0\leq i\leq 2n}$ (resp. $\{\pi_j^{X'}\}_{0\leq j\leq 2n'}$). 
Then the product CK decomposition of $X\times X'$ is $m$-fold $(\tau_m+\tau'_m)$-multiplicative.

In particular, the inequality of Conjecture \ref{conj2.6} is stable under product.
\end{proposition}
\begin{proof}
Denote by $p_X : (X\times X')^{m+1}\rightarrow X^{m+1}$ the natural projection and 
by $p_{X'}$ the projection to the other factors.
Then one has immediately the following equality
\begin{eqnarray*}
% \nonumber to remove numbering (before each equation)
   &  & \pi_k^{X\times X'}\circ\Delta_{I}^{X\times X'}\circ (\pi_{i_1}^{X\times X'}\otimes\cdots\otimes\pi_{i_m}^{X\times X'}) \\
   &=& \sum_{\substack{i_{11}+i_{12}=i_1\\ \cdots\cdots\\ i_{m1}+i_{m2}=i_m\\ k_1+k_2=k}} p_X^*\left(\pi_{k_1}^{X}\circ\Delta_{I}^{X}\circ (\pi_{i_{11}}^{X}\otimes\cdots\otimes\pi_{i_{m1}}^{X})\right)\cdot
p_{X'}^*\left(\pi_{k_2}^{X'}\circ\Delta_{I}^{X'}\circ (\pi_{i_{12}}^{X'}\otimes\cdots\otimes\pi_{i_{m2}}^{X'})\right).
\end{eqnarray*}
Now if $k>\sum_{\ell=1}^mi_{\ell}$, then either $k_1>\sum_{\ell=1}^mi_{\ell 1}$ or $k_2>\sum_{\ell=1}^mi_{\ell 2}$; if $k<\sum_{\ell=1}^{m}i_{\ell}-(\tau_m+\tau'_m)$, then either $k_1<\sum_{\ell=1}^mi_{\ell 1}-\tau_m$ or $k_2<\sum_{\ell=1}^mi_{\ell 2}-\tau'_m$.
By assumption, if $k_1>\sum_{\ell=1}^mi_{\ell 1}$ or $k_1<\sum_{\ell=1}^mi_{\ell 1}-\tau_m$, 
$k_2>\sum_{\ell=1}^mi_{\ell 2}$ or $k_2<\sum_{\ell=1}^mi_{\ell 2}-\tau'_m$,
then both the cycle classes $\pi_{k_1}^{X}\circ\Delta_{I}^{X}\circ (\pi_{i_{11}}^{X}\otimes\cdots\otimes\pi_{i_{m1}}^{X})$ and $\pi_{k_2}^{X'}\circ\Delta_{I}^{X'}\circ (\pi_{i_{12}}^{X'}\otimes\cdots\otimes\pi_{i_{m2}}^{X'})$ vanish.
Thus, 
\begin{equation*}
\pi_k^{X\times X'}\circ\Delta_{I}^{X\times X'}\circ (\pi_{i_1}^{X\times X'}\otimes\cdots\otimes\pi_{i_m}^{X\times X'})=0,
\end{equation*}
if $k>\sum_{\ell=1}^mi_{\ell}$ or $k<\sum_{\ell=1}^mi_{\ell}-(\tau_m+\tau'_m)$.
\end{proof}

The behavior of the small diagonal embeddings plays a fundamental role for the study of motivic multiple twist-multiplicativity.
The following is a generalization of Proposition 8.7(iii) in \cite{SV16a}.

\begin{lemma}\label{lem2.18}
Fix an integer $k\geq 2$.
Assume $X$ admits a $k$-fold $\tau$-multiplicative self-dual CK decomposition $\{\pi_i\}_{0\leq i\leq 2n}$.
Denote by $\delta_k: X\rightarrow X^k$ the $k$-th small diagonal embedding.
Then for integers $r$, $s$, $p$ and $t$, one has
\begin{equation}\label{9}
\delta_{k*}\text{CH}_s^r(X)\subseteq\bigoplus_{i=0}^{\tau}\text{CH}_{s+i}^{r+n(k-1)}(X^k),\quad
\delta_k^*\text{CH}_t^p(X^k)\subseteq\bigoplus_{i=0}^{\tau}\text{CH}_{t+i}^p(X).
\end{equation}
\end{lemma}
\begin{proof}
If $\alpha\in\text{CH}_s^r(X)$, then
\begin{eqnarray*}
% \nonumber to remove numbering (before each equation)
\delta_{k*}\alpha &=& \Delta_{I_k}\circ\pi_{2r-s}^X\circ\alpha \\
   &=& [(\pi_{2n-2r+s}^X\otimes\Delta_{X^k})_*\Delta_{k+1}]\circ\alpha \\
   &=& [(\pi_{2n-2r+s}^X\otimes\Delta_{X^k})_*\Delta_{k+1}]_*\alpha\\
   &=& \sum_{-\tau\leq\sum_{\ell}i_{\ell}-[2n(k-1)+2r-s]\leq 0}[(\pi_{2n-2r+s}^X\otimes\pi_{i_1}^X\otimes\cdots\otimes\pi_{i_k}^X)_*\Delta_{k+1}]_*\alpha\\
   &=& \sum_{-\tau\leq\sum_{\ell}i_{\ell}-[2n(k-1)+2r-s]\leq 0}(\pi_{i_1}^X\otimes\cdots\otimes\pi_{i_k}^X)_*(\delta_{k*}\alpha).
\end{eqnarray*}
The second-to-last equality makes use of the assumption on motivic $k$-fold $\tau$-multiplicativity.
Hence, one gets $\delta_{k*}\alpha\in\bigoplus_{i=0}^{\tau}\text{CH}_{s+i}^{r+n(k-1)}(X^k)$.

If $\beta\in\text{CH}_t^p(X^k)$, then by definition, one has
\begin{equation*}
\delta_{k}^*\beta\in\bigoplus_{i=0}^{\tau}\text{CH}_{t+i}^{p}(X).
\end{equation*}
\end{proof}
\begin{remark}
(i) Assume instead that $X$ admits a 2-fold $\tau$-multiplicative self-dual CK decomposition.
Then 
\begin{equation}\label{10}
\delta_{k*}\text{CH}_s^r(X)\subseteq\bigoplus_{i=0}^{(k-1)\tau}\text{CH}_{s+i}^{r+n(k-1)}(X^k).
\end{equation}

(ii) If $\tau\neq 0$, a new phenomenon is that the decomposition property of the $(k+1)$-th small diagonal class affects the push-forward induced by the $k$-th small diagonal embedding.
\end{remark}

Now we examine the behavior of motivic multiple twist-multiplicativity under projective bundles, 
generalizing Proposition 3.3 in \cite{SV16b} by taking $\tau=0$.
\begin{proposition}\label{prop2.20}
Let $m\geq 2$ be a fixed integer.
Let $\mathcal{E}$ be a locally free sheave of rank $e+1\geq 2$ on $X$ and $\mathbb{P}(\mathcal{E})$ be the projective bundle associated to $\mathcal{E}$ with $\pi: \mathbb{P}(\mathcal{E})\rightarrow X$ the natural projection.
Assume that 

(i) $X$ admits a $(m+1)$-fold $\tau$-multiplicative self-dual CK decomposition;

(ii) for each $i$, the Chern class $c_i(\mathcal{E})\in\text{CH}_0^i(X)$ and intersecting with $c_i(\mathcal{E})$ preserves the grading, that is, for any $\alpha\in\text{CH}_s^r(X)$, then $c_i(\mathcal{E})\cdot\alpha\in\text{CH}_s^{r+i}(X)$.

Then $\mathbb{P}(\mathcal{E})$ admits an $m$-fold $\tau$-multiplicative self-dual CK decomposition
such that
\begin{equation}\label{11}
\text{CH}_s^r(\mathbb{P}(\mathcal{E}))=\bigoplus_{i=0}^e\xi^i\cdot\pi^*\text{CH}_s^{r-i}(X),
\end{equation}
where $\xi\in\text{CH}^1(\mathbb{P}(\mathcal{E}))$ is the first Chern class of the canonical bundle $\mathcal{O}_{\mathbb{P}(\mathcal{E})}(1)$.
If the Chern classes of $X$ are in the graded-0 part, then so are the Chern classes of $\mathbb{P}(\mathcal{E})$.
Moreover, the correspondence $\Gamma_{\pi}$ is of pure grade 0, that is, $\Gamma_{\pi}\in\text{CH}_0^{n}(\mathbb{P}(\mathcal{E})\times X)$.
\end{proposition}
\begin{proof}
Under the assumption (i), $\mathbb{P}(\mathcal{E})$ admits two CK decompositions:
one is given by formula (6) in \cite{SV16b}; the other one is self-dual by formula (8) in \cite{SV16b}, 
whose construction is compatible with product.
The assumption (ii) implies that they have the same graded pieces (\ref{11}) of the Chow groups by the proof of Proposition 3.3 in \cite{SV16b}. 
It suffices to show that the aforementioned self-dual CK decomposition of $\mathbb{P}(\mathcal{E})$ is $m$-fold $\tau$-multiplicative.

By the projective bundle formula for Chow groups, one has
\begin{equation}\label{12}
\Delta_{I}^{\mathbb{P}(\mathcal{E})}
=\sum_{0\leq i_1, \cdots, i_{m+1}\leq e}((\pi^{\times (m+1)})^*(\delta_{m+1}^X)_*\alpha_{i_1\cdots i_{m+1}})\cdot\xi_1^{i_1}\cdots\xi_{m+1}^{i_{m+1}},
\end{equation}
where each $\alpha_{i_1\cdots i_{m+1}}\in\text{CH}^{me-\sum_{\ell=1}^{m+1}i_{\ell}}(X)$ is a polynomial of the Chern classes of $\mathcal{E}$ and $\xi_i=p_i^*\xi\in\text{CH}^1(\mathbb{P}(\mathcal{E})^{m+1})$ is the pull-back of $\xi$ from the $i$-th factor.
By the assumption (ii), one knows $\alpha_{i_1\cdots i_{m+1}}\in\text{CH}_0^{me-\sum_{\ell=1}^{m+1}i_{\ell}}(X)$.
By the assumption (i) and Lemma \ref{lem2.18}, one gets that
\begin{equation*}
(\delta_{m+1}^X)_*\alpha_{i_1\cdots i_{m+1}}\in\bigoplus_{s=0}^{\tau}\text{CH}_s^{m(n+e)-\sum_{\ell=1}^{m+1}i_{\ell}}(X^{m+1}).
\end{equation*}
Thus by the formula \eqref{12}, one has
\begin{equation*}
\Delta_{I}^{\mathbb{P}(\mathcal{E})}\in\bigoplus_{s=0}^{\tau}\text{CH}_s^{m(n+e)}(\mathbb{P}(\mathcal{E})^{m+1}).
\end{equation*}
Hence, one concludes by Proposition \ref{prop2.15}.

The statements about the Chern classes of $\mathbb{P}(\mathcal{E})$ and the correspondence $\Gamma_{\pi}$ are clear,
as in the argument of Proposition 3.3 in \cite{SV16b}.
\end{proof}
\begin{remark}
Without the assumption (ii), the conclusion would probably fail
and the intersection behavior of the Chern classes of $\mathcal{E}$ would affect dramatically the motivic multiple multiplicativity defect of $\mathbb{P}(\mathcal{E})$;
even if $X$ admits motivic 0-multiplicativity, there may exist many vector bundles on $X$ such that the associated projective bundles admit non-vanishing motivic multiplicativity defect.  
\end{remark}

For smooth blow-ups, the behavior of motivic multiple twist-multiplicativity can be demonstrated as follows, 
generalizing Proposition 3.4 in \cite{SV16b} by taking $\tau=0$. 
\begin{proposition}\label{prop2.22}
Let $m\geq 2$ be a fixed integer.
Let $Y$ be a smooth connected subvariety of $X$ with codimension $e+1\geq 2$.
Let $\rho: \widetilde{X}\rightarrow X$ be the blow-up of $X$ along $Y$. Consider the following Cartesian square
\begin{equation}\label{13}
\begin{tikzcd}
E \arrow{r}{\gamma} \arrow{d}{\pi}
&\widetilde{X} \arrow{d}{\rho}\\
Y \arrow{r}{\iota} & X
\end{tikzcd}
\end{equation}
Denote by $\mathcal{N}:=\mathcal{N}_{\iota}$ the normal bundle of $Y$ in $X$.
Assume that

(i) $X$ (resp. $Y$) has an $m$-fold (resp. $(m+1)$-fold) $\tau$-multiplicative self-dual CK decomposition;

(ii) for each $i$, the Chern class $c_i(\mathcal{N})\in\text{CH}_0^i(Y)$ and intersecting with $c_i(\mathcal{N})$ preserves the grading.

(iii) the correspondence $\Gamma_{\iota}$ is of pure grade 0.

Then $\widetilde{X}$ admits an $m$-fold $\tau$-multiplicative self-dual CK decomposition such that
\begin{equation}\label{14}
\text{CH}_s^r(\widetilde{X})=\rho^*\text{CH}_s^r(X)\oplus\bigoplus_{i=0}^{e-1}
\gamma_*(\xi^i\cdot\pi^*\text{CH}_s^{r-i-1}(Y)),
\end{equation}
where $\xi$ is the first Chern class of the line bundle $\mathcal{O}_{\widetilde{X}}(-E)|_E$.
If the Chern classes of $X$ are in the graded-0 part, then so are the Chern classes of $\widetilde{X}$.
Moreover, the correspondences $\Gamma_{\gamma}$ and $\Gamma_{\rho}$ are of pure grade 0.
\end{proposition}
\begin{proof}
Under the assumption (i), $\widetilde{X}$ admits two CK decompositions given by the formula (11) and the formula (13) in \cite{SV16b}, which induce the same associated graded pieces \eqref{14} of the Chow groups by the proof of Proposition 3.4 in \cite{SV16b} using the assumption (ii).
The latter is self-dual and it remains to show that it is $m$-fold $\tau$-multiplicative.

Note that by the diagram \eqref{13}, there is the following commutative diagram with the square Cartesian:
\begin{equation}\label{15}
\begin{tikzcd}
E_{/Y}^{m+1} \arrow{r}{\phi} \arrow{d}{\pi_Y^{\times (m+1)}}
& E^{m+1} \arrow{d}{\pi^{\times (m+1)}}\arrow{r}{\gamma^{\times (m+1)}}&\widetilde{X}^{m+1}\\
Y \arrow{r}{\delta_{m+1}^Y} & Y^{m+1}.
\end{tikzcd}
\end{equation}
By the assumption (iii), $\Gamma_{\iota}$ is of pure grade 0. 
Then so is $\Gamma_{\gamma^{\times (m+1)}}$ for each $m\geq 0$.
By the argument of Proposition 3.4 in \cite{SV16b}, one has
\begin{eqnarray*}
\text{CH}_s^*(\widetilde{X}^{m+1})
&\supseteq& (\rho^{\times (m+1)})^*\text{CH}_s^*(X^{m+1})\\
&  & +\sum_{i_1=0}^{e-1}\cdots\sum_{i_{m+1}=0}^{e-1}(\gamma^{\times (m+1)})_*(\xi_1^{i_1}\cdots\xi_{m+1}^{i_{m+1}}\cdot (\pi^{\times (m+1)})^*\text{CH}_s^*(Y^{m+1})),
\end{eqnarray*}
where $\xi_i\in\text{CH}^1(E^{m+1})$ is the pull-back of $\xi$ via the projection to the $i$-th factor.
By the assumption (ii) and Lemma 3.5 in \cite{SV16b}, one can write
\begin{equation}\label{16}
\Delta_{I}^{\widetilde{X}}=(\rho^{\times (m+1)})^*\Delta_{I}^X+(\gamma^{\times (m+1)})_*\phi_*\alpha,
\end{equation}
where $\alpha=P(\phi^*\xi_i,c((\pi^{\times (m+1)})^*\mathcal{N}))\in\text{CH}^{me-1}(E_{/Y}^{m+1})$ with $P$ a polynomial.
By the projection formula, one gets that
\begin{eqnarray*}
% \nonumber to remove numbering (before each equation)
  \phi_*\alpha &=& P(\xi_i, \phi_*(\pi_Y^{\times (m+1)})^*c(\mathcal{N})) \\
   &=& P(\xi_i, (\pi^{\times (m+1)})^*(\delta_{m+1}^Y)_*c(\mathcal{N})) \\
   &=& \sum_{i_1=0}^{e-1}\cdots\sum_{i_{m+1}=0}^{e-1}(\pi^{\times (m+1)})^*((\delta_{m+1}^Y)_*\alpha_{i_1\cdots i_{m+1}})\cdot\xi_1^{i_1}\cdots\xi_{m+1}^{i_{m+1}},
\end{eqnarray*}
where each $\alpha_{i_1\cdots i_{m+1}}\in\text{CH}^{me-1-\sum_{\ell=1}^{m+1}i_{\ell}}(Y)$ is a polynomial of the Chern classes of $\mathcal{N}$.
By the assumption (ii), one knows that $\alpha_{i_1\cdots i_{m+1}}\in\text{CH}_0^{me-1-\sum_{\ell=1}^{m+1}i_{\ell}}(Y)$.
By the assumption (i) on $Y$ and Lemma \ref{lem2.18}, one has
\begin{equation*}
(\delta_{m+1}^Y)_*\alpha_{i_1\cdots i_{m+1}}\in\bigoplus_{s=0}^{\tau}\text{CH}_s^{mn-m-1-\sum_{\ell=1}^{m+1}i_{\ell}}(Y^{m+1}).
\end{equation*}
By the assumption (i) on $X$, one knows $\Delta_{I}^X\in\bigoplus_{s=0}^{\tau}\text{CH}_s^{mn}(X^{m+1})$.
By the formula \eqref{16}, one gets 
\begin{equation*}
\Delta_{I}^{\widetilde{X}}\in\bigoplus_{s=0}^{\tau}\text{CH}_s^{mn}(\widetilde{X}^{m+1}).
\end{equation*}
So, the self-dual CK decomposition of $\widetilde{X}$ is $m$-fold $\tau$-multiplicative.

The statements about the Chern classes of $\widetilde{X}$ and $\Gamma_{\rho}$ are clear,
as in the argument of Proposition 3.4 in \cite{SV16b}.
\end{proof}

In the sequel, we start to investigate motivic multiple twist-multiplicativity of concrete varieties and verify part of Conjecture \ref{conj2.6} for first examples of varieties. Before that, we need the following preparatory lemma.

\begin{lemma}\label{lem2.23}
Let $X$ be a smooth connected projective variety satisfying the nilpotence conjecture.
If for a set $\{\pi_i\}_{0\leq i\leq 2n}$ of CK projectors of $X$ and a fixed integer $r\geq 1$,
Murre's conjecture (B) holds true in codimension $r$,
then so does for any set of CK projectors of $X$.
\end{lemma}
\begin{proof}
Let $\{\widetilde{\pi}_i\}_{0\leq i\leq 2n}$ be another set of CK projectors of $X$.
The nilpotence conjecture for $X$ implies that the ideal $\text{CH}^n(X\times X)_{\text{hom}}$ of the correspondence ring $\text{CH}^n(X\times X)$ is nilpotent. By Lemma 5.4 in \cite{Ja94}, there exists an element $\eta\in\text{CH}^n(X\times X)_{\text{hom}}$ such that
$\pi_i=(1+\eta)\circ\pi'_i\circ(1+\eta)^{-1}$. Then $\mathfrak{h}^i(X):=(X, \pi_i)\cong (X,\pi'_i)=:\widetilde{\mathfrak{h}}^i(X)$.
By assumption, one has
\begin{equation*}
0=\pi_{i*}\text{CH}^r(X)=\text{Hom}(\mathbb{L}^r, \mathfrak{h}^i(X))\cong\text{Hom}(\mathbb{L}^r, \widetilde{\mathfrak{h}}^i(X))=\widetilde{\pi}_{i*}\text{CH}^r(X)
\end{equation*}
for each $i<r$ or $i>2r$.
\end{proof}

As the first test, we consider the case of curves.
\begin{theorem}\label{thm2.24}
Conjecture \ref{conj2.6} holds true for any smooth connected projective curve.
Moreover, let $X$ be a curve of genus $g\geq 2$.

(i) If $2\leq m\leq g$, then $\mathfrak{d}_m(X)\leq m-1$.

(ii) For any integer $m\geq g$, one has $\mathfrak{d}_m(X)\leq g-1$ and hence $\mathfrak{sd}(X)\leq g-1$.
\end{theorem}
\begin{proof}
Since the Chow motive of $X^{m+1}$ is finite-dimensional, then the nilpotence conjecture holds for $X^{m+1}$. 
By Lemma \ref{lem2.23}, it suffices to show Murre's conjecture (B) for a chosen set of CK projectors of $X^{m+1}$ in codimension $m$. 

Take any closed point $a\in X$ and let $o:=[a]\in\text{CH}^1(X)$. 
Denote 
\begin{equation*}
\pi_0=o\times X,\quad \pi_2=X\times o,\quad \pi_1=\Delta_{X}-\pi_0-\pi_2\in\text{CH}^1(X\times X).
\end{equation*}
Then $\{\pi_i\}_{0\leq i\leq 2}$ forms a set of self-dual CK projectors of $X$.
Equip $X^{m+1}$ with the product CK projectors 
$\{\Pi_k=\sum_{\sum_{\ell=0}^{m}i_{\ell}=k}\pi_{i_0}\otimes\cdots\otimes\pi_{i_{m}}\}_{0\leq k\leq 2(m+1)}$.
We will verify Murre's conjecture (B) for these projectors in codimension $mn$, i.e.,
\begin{equation}\label{17}
\Pi_{k*}\text{CH}^{m}(X^{m+1})
=\text{Hom}(\mathbb{L}^{m}, \mathfrak{h}^{k}(X^{m+1})=0
\end{equation}
for $k<m$ or $k>2m$. This would imply Conjecture \ref{conj2.6} for $X$.

(a) Case: $k>2m$.
Since $k-(m+1)>m-1$, then $i_{\ell}=2$ for some $\ell$.
By symmetry, one may assume $i_{0}=2$.
Thus,
\begin{equation*}
\text{Hom}(\mathbb{L}^{m}, \mathfrak{h}^{2}(X)\otimes\mathfrak{h}^{i_{1}}(X)\otimes\cdots\otimes\mathfrak{h}^{i_{m}}(X))
=\text{Hom}(\mathbb{L}^{m-1}, \mathfrak{h}^{i_{1}}(X)\otimes\cdots\otimes\mathfrak{h}^{i_{m}}(X)),
\end{equation*}
because $\mathfrak{h}^{2}(X)\cong\mathbb{L}$.
Note that $\sum_{\ell1}^{m}i_{\ell}=k-2>2(m-1)$.
By induction on $m$, one is reduced to show that
\begin{equation*}
\text{Hom}(\mathbb{L}, \mathfrak{h}^{i_{m-1}}(X)\otimes\mathfrak{h}^{i_{m}}(X))=0.
\end{equation*}
This is clear, since $i_{m-1}+i_m>2$.

(b) Case: $k<m$.
If $i_{\ell}=2$ for some $\ell$, the one may assume $i_0=2$ by symmetry.
Thus,
\begin{equation*}
\text{Hom}(\mathbb{L}^{m}, \mathfrak{h}^{2}(X)\otimes\mathfrak{h}^{i_1}(X)\otimes\cdots\otimes\mathfrak{h}^{i_{m}}(X))
\cong\text{Hom}(\mathbb{L}^{m-1}, \mathfrak{h}^{i_1}(X)\otimes\cdots\otimes\mathfrak{h}^{i_{m}}(X)).
\end{equation*}
By induction, one is reduced to the case that each $i_{\ell}\leq 1$.
Since $k<m$, then $i_{\ell}=0$ for some $\ell$.
One may assume $i_{0}=0$.
It is enough to show that
\begin{equation*}
\text{Hom}(\mathbb{L}^{m}, \mathfrak{h}^{i_0}(X)\otimes\cdots\otimes\mathfrak{h}^{i_{m}}(X))
\cong\text{Hom}(\mathbb{L}^{m}, \mathfrak{h}^{i_1}(X)\otimes\cdots\otimes\mathfrak{h}^{i_{m}}(X))=0.
\end{equation*}
Note that the group $\text{CH}^{m}(X^{m})$ is generated by 0-cycle classes of the form $z_1\times\cdots\times z_{m}$ with $z_i\in\text{CH}^1(X)$. Since $k<m$, then $i_{\ell}=0$ for some $1\leq\ell\leq m$.
Then the equality \eqref{17} follows from that $\pi_{0*}z_{\ell}=0$.

Now let $X$ be a curve of genus $g\geq 2$. We will show the assertions (i) and (ii).

Case: $m=2$. 
Denote $\Delta_{123}:=\Delta_I$, $\Delta_{ij}:=p_{ij}^*\Delta_X\cdot p_k^*o_X$ 
and $\Delta_i:=p_i^*[X]\cdot p_j^*o_X\cdot p_k^*o_X$ for $\{i,j,k\}=\{1,2,3\}$.
Straightforward computations give that
\begin{equation}\label{18}
\pi_0\circ\Delta_{123}\circ (\pi_0\otimes\pi_0)
=(\pi_0\otimes\pi_2\otimes\pi_2)_*\Delta_{123}
=\Delta_1,
\end{equation}
\begin{equation}\label{19}
\Delta_{123}\circ (\pi_2\otimes\pi_0)
=(\Delta_X\otimes\pi_0\otimes\pi_2)_*\Delta_{123}=\Delta_2,
\end{equation}
\begin{equation}\label{20}
\Delta_{123}\circ (\pi_0\otimes\pi_2)
=(\Delta_X\otimes\pi_2\otimes\pi_0)_*\Delta_{123}=\Delta_3.
\end{equation}
Then it follows that for $k\neq i+j$ and $(i,j,k)\neq (1,1,1)$, one has
\begin{equation}\label{21}
\pi_k\circ\Delta_{123}\circ(\pi_i\otimes\pi_j)=0.
\end{equation}
Denote the third modified diagonal class of $(X, o_X)$ by
\begin{equation*}
\Gamma^3(X,o_X):=\Delta_{123}-\Delta_{12}-\Delta_{13}-\Delta_{23}+\Delta_1-\Delta_2+\Delta_3\in\text{CH}^2(X^3).
\end{equation*}
Then by the formula \eqref{21}, the above CK decomposition is 1-multiplicative.
Moreover, by the equalities \eqref{18}-\eqref{21}, one has
\begin{equation*}
\Gamma^3(X,o_X)=\pi_1\circ\Delta_{123}\circ(\pi_1\otimes\pi_1)
=(\pi_1\otimes\pi_1\otimes\pi_1)_*\Delta_{123}\in\text{CH}_1^2(X^3).
\end{equation*}

Case: $m\geq 3$. By definition, $\mathfrak{d}_m(X)\leq (m-1)\mathfrak{d}(X)\leq m-1$.
We will show that if $k\neq\sum_{\ell=1}^{m}i_{\ell}$ with $k=0$ or some $i_{\ell'}=2$,
then one has
\begin{equation*}
\pi_k\circ\Delta_{I}\circ(\pi_{i_1}\otimes\cdots\otimes\pi_{i_m})=0.
\end{equation*}
Indeed, if some $i_{\ell'}=2$, one may assume $i_{1}=2$ by symmetry.
Then
\begin{eqnarray*}
% \nonumber to remove numbering (before each equation)
 \pi_k\circ\Delta_{I}\circ(\pi_{i_1}\otimes\cdots\otimes\pi_{i_m}) &=& (\pi_{0}\otimes\pi_{2-i_2}\otimes\cdots\otimes\pi_{2-i_m}\otimes\pi_k)_*\Delta_I \\
    &=& X\times(\pi_{2-i_2})_*o_X\times\cdots(\pi_{2-i_m})_*o_X\times\pi_{k*}o_X\\
    &=& 0.
\end{eqnarray*}
The last equality follows from the fact that $\pi_{0*}o_X=0=\pi_{1*}o_X$ and that if $i_{\ell}=0$ for all $\ell\geq 2$, 
then $k\neq 2$. If $k=0$, the argument is similar.

Hence, the above CK decomposition is $m$-fold $(m-1)$-multiplicative.
Moreover, one gets that 
\begin{equation}\label{22}
\Delta_I-\sum_{k=0}^{4}\sum_{\sum_{\ell=1}^{m}i_{\ell}=k}\pi_k\circ\Delta_I\circ(\pi_{i_1}\otimes\cdots\otimes\pi_{i_m})
=\sum_{s=0}^{m-1}\Gamma_s,
\end{equation}
where 
\begin{equation}\label{23}
\Gamma_s=\overbrace{(\pi_1\otimes\cdots\otimes\pi_1}^{s+2})_*\Delta_{I_{s+1}}\times 
\overbrace{o_X\times\cdots\times o_X}^{m-s-1}+(\text{permutations})\in\text{CH}_{s}^{m}(X^{m+1}).
\end{equation}
Hence,
\begin{equation}\label{24}
\Gamma^{m+1}(X,o_X)
=(\overbrace{\pi_1\otimes\cdots\otimes\pi_{1}}^{m+1})_*\Delta_{I}\in\text{CH}_{m-1}^m(X^{m+1}).
\end{equation}

It follows from \cite{MY16} that $\Gamma^{g+2}(X,o_X)=0$. 
Now since the $(g+1)$-th small diagonal class $\Delta_{I_g}\in\bigoplus_{s=0}^{g-1}\text{CH}_{s}^g(X^{g+1})$, then by the proof of Proposition \ref{prop2.10}, the CK decomposition of $X$ is $m$-fold $(g-1)$-multiplicative 
for any $m\geq g$.
\end{proof}

Now we can determine the motivic multiple multiplicativity defect of a very general curve by non-vanishing results of its modified diagonal classes proven in \cite{Y13}.
\begin{corollary}
Let $X$ be a very general curve of genus $g\geq 3$. 

(i) $\mathfrak{d}_m(X)=m-1$ for $2\leq m\leq g-1$. In particular, $\mathfrak{d}(X)=1$.

(ii) $\mathfrak{d}_m(X)=g-1$ for any integer $m\geq g$.
In particular, $\mathfrak{sd}(X)=g-1$.
\end{corollary}
\begin{proof}
By the argument of Theorem \ref{thm2.24}, to determine the stable motivic multiplicativity defect is equivalent to the non-vanishing of some modified diagonal class.
Now according to \cite{BV04}\cite{Y13}, for a very general curve $X$ of genus $g\geq 3$, 
one has $\Gamma^{m+1}(X,o_X)\neq 0$ for any $m\in\{2, 3\cdots, g\}$ and any 0-cycle class $o_X$ of degree one. 
Therefore, $\mathfrak{d}_m(X)=m-1$ for any $m\in\{2, 3\cdots, g-1\}$
and $\mathfrak{d}_m(X)=g-1$ for any $m\geq g$.
\end{proof}

\begin{remark}
(i) For a special non-hyperelliptic curve of genus $\geq 3$, its motivic multiplicativity defect may specialize to 0.
It is interesting to find all such curves.

(ii) Modulo algebraic equivalence, it is expected that the stable motivic multiplicativity defect $\mathfrak{sd}_a(X)=\text{gon}(X)-2=\lfloor\frac{g-1}{2}\rfloor$, where $\text{gon}(X)=\lfloor\frac{g+3}{2}\rfloor$ is the gonality of $X$. Refer to 4.4 in \cite{MY16} for details. For evidences, Beauville \cite{Bea23} has found a non-hyperelliptic curve of genus $3$ with motivic multiplicativity defect 0 modulo algebraic equivalence. There are also other non-hyperelliptic curves expected with this property. Refer to \cite{BLLS23}.

(iii) Presumably, another natural method of detecting the non-vanishing of the modified diagonal classes is to apply the higher Abel-Jacobi maps for Chow groups as defined in \cite{As00}\cite{S01}.

(iv) Let $X_i(1\leq i\leq k)$ be very general curves of genus $\geq 3$.
Then it is not hard to see that $\mathfrak{d}(\prod_{i=1}^{k}C_i)=k$. 
\end{remark}

Next, we proceed to consider the case of surfaces.

\begin{theorem}\label{thm2.27}
Let $X$ be any smooth connected projective surface. 
Then for each integer $m\geq 2$, Murre's conjecture (B) holds true for a set of CK projectors of $X^{m+1}$ in codimension $2m$.
As a consequence, $\mathfrak{d}_m(X)\leq 2m$.
Further, if $X$ is swept-out by (irreducible) curves of genus $g$ supporting a 0-cycle rationally equivalent to $o_X$,
one has $\mathfrak{sd}(X)\leq 6g+2$.
If moreover, $X$ is regular, then $\mathfrak{d}_m(X)$ is even and $\mathfrak{d}_m(X)\leq 2m-2$.
\end{theorem}
\begin{proof}
According to \cite{Mu90}\cite{Sch94}, $X$ admits a self-dual CK decomposition:
$\Delta_X=\sum_{i=0}^{4}\pi_i$, where $\pi_0=o_X\times X$ with $o_X$ a 0-cycle class of degree one.
For each $m\geq 2$, the $(m+1)$-th power $X^{m+1}$ of $X$ is equipped with the product CK projectors.
We will verify Murre's conjecture (B) for these CK projectors in codimension $2m$, that is,
\begin{equation}\label{25}
(\pi_{k_0}\otimes\cdots\otimes\pi_{k_m})_*\text{CH}^{2m}(X^{m+1})
=\text{Hom}(\mathbb{L}^{2m}, \mathfrak{h}^{k_0}(X)\otimes\cdots\otimes\mathfrak{h}^{k_{m}}(X))=0,
\end{equation}
if $\sum_{\ell=0}^{m}k_{\ell}>4m$ or $\sum_{\ell=0}^{m}k_{\ell}<2m$.

(i) Case: $\sum_{\ell=0}^{m}k_{\ell}>4m$.
If $k_{\ell}=4$ for some $\ell$, then one may assume that $\ell=0$ and $k_0=4$ by symmetry.
Thus,
\begin{equation*}
\text{Hom}(\mathbb{L}^{2m}, \mathfrak{h}^{4}(X)\otimes\mathfrak{h}^{k_1}(X)\otimes\cdots\otimes\mathfrak{h}^{k_{m}}(X))
=\text{Hom}(\mathbb{L}^{2m-2}, \mathfrak{h}^{k_1}(X)\otimes\cdots\otimes\mathfrak{h}^{k_{m}}(X)),
\end{equation*}
since $\mathfrak{h}^{4}(X)\cong\mathfrak{h}^0(X)\otimes\mathbb{L}^2\cong\mathbb{L}^2$.
By induction, one is reduced to the case $m=1$ which follows by results in \cite{Mu93}, or to the case that each $k_{\ell}\leq 3$.
For the latter case, one has $4m+1\leq\sum_{\ell=0}^{m}k_{\ell}\leq 3m+3$, which forces that $m=2$ and each $k_{\ell}=3$.
Since $\mathfrak{h}^3(X)\cong\mathfrak{h}^1(X)\otimes\mathbb{L}^1$ by Theorem 4.4(ii) in \cite{Sch94}, 
then one gets that
\begin{eqnarray*}
% \nonumber to remove numbering (before each equation)
   \text{Hom}(\mathbb{L}^4, \mathfrak{h}^3(X)\otimes\mathfrak{h}^3(X)\otimes\mathfrak{h}^3(X))
   &\cong& \text{Hom}(\mathbb{L}^4, \mathfrak{h}^1(X)\otimes\mathfrak{h}^1(X)\otimes\mathfrak{h}^1(X)
   \otimes\mathbb{L}^3) \\
   &=& \text{Hom}(\mathbb{L}, \mathfrak{h}^1(X)\otimes\mathfrak{h}^1(X)\otimes\mathfrak{h}^1(X))\\
   &=& (\pi_1\otimes\pi_1\otimes\pi_1)_*\text{CH}^1(X^3)\\
   &=& 0.
\end{eqnarray*}

(ii) Case: $\sum_{\ell=0}^{m}k_{\ell}<2m$.
One has $k_{\ell}\leq 1$ for some $\ell$, because $2m+2>2m-1$.
Then one may assume $k_0\leq 1$ by symmetry.
If $k_0=0$, then
$$
\text{Hom}(\mathbb{L}^{2m}, \mathfrak{h}^{0}(X)\otimes\mathfrak{h}^{k_1}(X)\otimes\cdots\otimes\mathfrak{h}^{k_{m}}(X))
\cong\text{Hom}(\mathbb{L}^{2m}, \mathfrak{h}^{k_1}(X)\otimes\cdots\otimes\mathfrak{h}^{k_{m}}(X)).
$$
Now since $\sum_{\ell=1}^{m}k_{\ell}<2m$, then one is reduced to the case that each $k_{\ell}\geq 1$ by induction.
So, one may assume $k_0=1$ and it suffices to show that
\begin{eqnarray*}
% \nonumber to remove numbering (before each equation)
   & &\text{Hom}(\mathbb{L}^{2m}, \mathfrak{h}^{1}(X)\otimes\mathfrak{h}^{k_1}(X)\otimes\cdots\otimes\mathfrak{h}^{k_{m}}(X))\\
   &\cong& \text{Hom}(\mathbb{L}^{2m}, \mathfrak{h}^{1}(A)\otimes\mathfrak{h}^{k_1}(X)\otimes\cdots\otimes\mathfrak{h}^{k_{m}}(X)) \\
   &=& 0,
\end{eqnarray*}
since $\mathfrak{h}^1(X)\cong\mathfrak{h}^1(A)$, where $A$ is the Picard variety of $X$.
It is well-known that there is a surjective morphism $C^n\rightarrow A$ for some smooth connected projective curve $C$ and some integer $n\geq 1$ and hence $\mathfrak{h}^1(A)$ is a direct summand of $\mathfrak{h}^1(C^n)$.
It suffices to show that 
\begin{equation}\label{26}
\text{Hom}(\mathbb{L}^{2m}, \mathfrak{h}^{1}(C^n)\otimes\mathfrak{h}^{k_1}(X)\otimes\cdots\otimes\mathfrak{h}^{k_{m}}(X))=0.
\end{equation}
This is easily reduced to show that
\begin{equation}\label{27}
\text{Hom}(\mathbb{L}^{2m}, \mathfrak{h}^{1}(C)\otimes\mathfrak{h}^{k_1}(X)\otimes\cdots\otimes\mathfrak{h}^{k_{m}}(X))=0.
\end{equation}
According to the argument of Theorem 2.4 in \cite{XX13}, it is enough to show that
\begin{equation}\label{28}
\text{Hom}(\mathbb{L}^{2m}, \mathfrak{h}^{k_1}(X)\otimes\cdots\otimes\mathfrak{h}^{k_{m}}(X))=0,
\end{equation}
\begin{equation}\label{29}
\text{Hom}(\mathbb{L}^{2m-1}, \mathfrak{h}^{k_1}(X)\otimes\cdots\otimes\mathfrak{h}^{k_{m}}(X))=0.
\end{equation}
Repeating this procedure, one is reduced to show that
\begin{equation}\label{30}
\text{Hom}(\mathbb{L}^{j}, \mathfrak{h}^{1}(X)\otimes\mathfrak{h}^{2}(X)^{\otimes r})=0
\end{equation}
for some integer $r\geq 0$ and each integer $j\geq 2r+2$.
If $j=2r+2$, then the group $\text{CH}^{2r+2}(X^{r+1})$ is generated by 0-cycle classes of the form $z_1\times\cdots\times z_{r+1}$ with $z_i\in\text{CH}^2(X)$.
Then the equality \eqref{30} follows from the fact that 
\begin{equation*}
  (\pi_1\otimes\overbrace{\pi_2\otimes\cdots\otimes\pi_2}^r)_*(z_1\times\cdots\times z_{r+1})
=\pi_{1*}z_1\times\pi_{2*}z_2\cdots\times\pi_{2*}z_{r+1}=0.
\end{equation*}
If $j>2r+2$, the equality \eqref{30} is trivial for dimension reason.
Given these, one concludes that $\mathfrak{d}_m(X)\leq 2m$. 
By Remark \ref{rem2.13}(i), one knows that $\mathfrak{sd}(X)\leq 6g+2$.

Now suppose that $X$ is regular.
Then $\pi_1=0=\pi_3$ and hence $\mathfrak{d}_m(X)$ is even.
The same argument just as in the case of curves implies that 
\begin{equation}\label{31}
\Gamma^{m+1}(X,o_X)
=(\overbrace{\pi_2\otimes\cdots\otimes\pi_2}^{m+1})_*\Delta_{I}\in\text{CH}_{2(m-1)}^{2m}(X^{m+1}).
\end{equation}
Then 
\begin{equation}\label{32}
\Delta_I-\sum_{k=0}^{4}\sum_{\sum_{\ell=1}^{m}i_{\ell}=k}\pi_k\circ\Delta_I\circ(\pi_{i_1}\otimes\cdots\otimes\pi_{i_m})
=\sum_{s=0}^{m-1}\Gamma_s,
\end{equation}
where 
\begin{equation*}
\Gamma_s=\overbrace{(\pi_2\otimes\cdots\otimes\pi_2}^{s+2})_*\Delta_{I_{s+1}}\times 
\overbrace{o_X\times\cdots\times o_X}^{m-s-1}+(\text{permutations})\in\text{CH}_{2s}^{2m}(X^{m+1}).
\end{equation*}
Thus one has $\mathfrak{d}_m(X)\leq 2(m-1)$.
\end{proof}
\begin{remark}
(i) It is not clear how to determine the sharp bound of the stable motivic multiplicativity defect of a very general surface, unlike the case of curves.

(ii) If a regular surface is not motivic 2-multiplicative, then it must admit motivic 0-multiplicativity.

(iii) Unfortunately, the argument do not work for general varieties of dimension $\geq 3$, 
because we know little about algebraic cycles on powers of such varieties.
\end{remark}

Now the following existence result is in order.
\begin{proposition}
Let $n\geq 1$ be a fixed integer. Given any nonnegative integer $\tau\leq n$, 
there exists a smooth connected projective variety of dimension $n$ with motivic multiplicativity defect $\tau$.
\end{proposition}
\begin{proof}
For each nonnegative integer $i\leq\tau$, take $C_i$ to be a very general curve of genus $\geq 3$.
Let $X:=\prod_{i=1}^{\tau}C_i\times\mathbb{P}^{n-\tau}$.
Then $X$ is motivic $\tau$-multiplicative by Proposition \ref{prop2.17}.
Since $\mathfrak{d}(\prod_{i=1}^{\tau}C_i)=\tau$,
then $\mathfrak{d}(X)=\tau$ by Proposition \ref{prop2.20}.
\end{proof}

As it turns out immediately, stable motivic multiplicativity leads to vanishing results of the modified diagonal classes.
In this perspective, stable motivic multiplicativity could be thought of as a suitable generalization of motivic 0-multiplicativity.
The following vanishing result is a generalization of Proposition 8.12 in \cite{SV16a}.
\begin{proposition}\label{prop2.30}
Assume $X$ admits a set of self-dual CK projectors $\{\pi_i^X\}_{0\leq i\leq 2n}$ with $\pi_{0}^X=o_X\times X$, for which one attains $\mathfrak{sd}(X)=\tau$.
Then the modified diagonal class $\Gamma^{k}(X, o_X)$ vanishes for any integer $k>2n+\tau$.
Moreover, if $\pi_1^X=0$, then $\Gamma^{k}(X, o_X)$ vanishes for any integer $k>n+\frac{\tau}{2}$.
\end{proposition}
\begin{proof}
By assumption, for any $k\geq 2$, the CK decomposition is $k$-fold $\tau$-multiplicative.
So, the small diagonal class $\Delta_{I_{k-1}}\in\bigoplus_{s=0}^{\tau}\text{CH}_s^{n(k-1)}(X^k)$.
Then
\begin{equation*}
(\pi_{i_1}\otimes\cdots\otimes\pi_{i_k})_*\Delta_k=0
\end{equation*}
for $\sum_{\ell=1}^{k}i_{\ell}<2n(k-1)-\tau$ or $\sum_{\ell=1}^{k}i_{\ell}>2n(k-1)$.

Now if $k>2n+\tau$, then the condition 
\begin{equation*}
2n(k-1)-\tau\leq\sum_{\ell=1}^{k}i_{\ell}\leq 2n(k-1)
\end{equation*}
implies that at least one index $i_{\ell}$ is $2n$, because otherwise, 
\begin{equation*}
\sum_{\ell=1}^{k}i_{\ell}\leq k(2n-1)<2n(k-1)-\tau,
\end{equation*}
a contradiction. Hence, $\Gamma^k(X, o_X)=0$ for any integer $k>2n+\tau$.

Suppose that $\pi_1^X=0$. If $k>n+\frac{\tau}{2}$, then the same condition 
\begin{equation*}
2n(k-1)-\tau\leq\sum_{\ell=1}^{k}i_{\ell}\leq 2n(k-1)
\end{equation*}
also implies that $i_{\ell}=2n$ for some index $\ell$.
So, $\Gamma^k(X, o_X)=0$ for any $k>n+\frac{\tau}{2}$.
\end{proof}

\begin{remark}
(i) Voisin \cite{Voi15} shows that if $X$ is swept-out by irreducible curves of genus $g$ supporting a 0-cycle rationally equivalent to $o_X$
then $\Gamma^m(X, o_X)=0$ for any integer $m\geq (n+1)(g+1)$.
Except rationally connected varieties among others, the aforementioned curves are usually taken to be a family of complete intersection curves in $X$
and their genus $g$ is very large.
So, the lower bound of vanishing of the modified diagonal classes cannot be obtained from this result.
To approach the lower bound, it seems necessary to determine the stable motivic multiplicativity defect of any variety.

(ii) It is reasonable to expect the result is sharp for many kinds of varieties with required conditions. 
It seems interesting to prove the optimal vanishing result for any variety and determine the lower bound.

(iii) It should be pointed out that even if a variety satisfies the sharp modified diagonal property, its stable motivic multiplicativity defect may be large, just as witnessed by the case of rationally connected varieties.
\end{remark}

\begin{remark}
In contrast, the lower bound given by Voisin's method in \cite{Voi15} is exactly
\begin{equation}\label{33}
n+(n+1)\left(\sum_{m=1}^{r}(-1)^{m+1}\sum_{1\leq i_1<\cdots<i_m\leq r}\binom{r+1-\sum_{k=1}^md_{i_k}}{r+1}\right).
\end{equation}
\end{remark}

For regular varieties, motivic $1$-multiplicativity leads to vanishing of certain modified diagonal classes.
The following result is a slight generalization of Proposition 8.12 in \cite{SV16a}.

\begin{proposition}\label{prop2.33}
Assume that $X$ admits a $1$-multiplicative self-dual CK decomposition with $b_1(X)=0$ and $\pi_{2n}=X\times o_X$. Then for any integer $k>2n-2$, the modified diagonal class $\Gamma^k(X, o_X)$ vanishes.
\end{proposition}
\begin{proof}
By assumption, one has $\mathfrak{d}_{k-1}(X)\leq k-2$ for each integer $k\geq 3$. 
By Proposition \ref{prop2.15}, one has 
\begin{equation*}
\Delta_{I_{k-1}}=\delta_{k*}[X]\in\bigoplus_{s=0}^{k-2}\text{CH}_{s}^{n(k-1)}(X^k).
\end{equation*}
Then
\begin{equation*}
(\pi_{i_1}\otimes\cdots\otimes\pi_{i_k})_*\Delta_{I_{k-1}}=0
\end{equation*}
for $\sum_{\ell=1}^{k}i_{\ell}<2n(k-1)-(k-2)$ or $\sum_{\ell=1}^{k}i_{\ell}>2n(k-1)$.

Now suppose that $k>2n-2$, then the condition 
\begin{equation*}
2n(k-1)-(k-2)\leq\sum_{\ell=1}^{k}i_{\ell}\leq 2n(k-1)
\end{equation*}
implies that at least one index $i_{\ell}$ is $2n$.
Hence, $\Gamma^k(X, o_X)=0$ for any $k>2n-2$.
\end{proof}
\bigskip

\section{Isogenous correspondences and cycle relations}
\bigskip

In this preparatory section, we provide constructions of isogenous correspondences for Fano and Calabi-Yau complete intersections in certain ambient varieties.
Then we establish relevant cycle relations, which will be important for determining the motivic multiple multiplicativity defect for such varieties.
Remarkably, the notion of $K$-correspondences was first introduced by Voisin to study a problem of Kobayashi and some $K$-correspondences have already been constructed for certain Calabi-Yau varieties \cite{Voi04}. 
In the subsequent Calabi-Yau case, the idea of construction is essentially due to Voisin.
Nevertheless, under our situation, we require a slightly more flexible notion. 
\begin{definition}
Let $X$ be any smooth connected projective variety of dimension $n$.
A correspondence $\Sigma\in\text{CH}^n(X\times X)$ is \emph{isogenous}, if for some representing cycle $\sum_i m_i\Sigma_i$,
each component $\Sigma_i$ of its support dominates the two factors via the natural projections.
It is a $K$-autocorrespondence if moreover, for each $i$ and any desingularization $\tau:\widetilde{\Sigma}_i\rightarrow\Sigma_i$, the two composite morphisms $f_k=p_k\circ\tau: \widetilde{\Sigma}_i\rightarrow X$ ($k=1, 2$) have the same ramification divisor.
\end{definition}

Now let us fix the setting. 
Given a smooth connected projetive variety $Y$ of dimension $n+1\geq 3$, 
we will mainly consider the following two situations:

Case (a):
$Y$ is rationally connected and $X\subseteq Y$ is a smooth anti-canonical divisor. Then $X$ is a Calabi-Yau variety of dimension $n$.

Case (b):
$Y$ is a Fano variety of index $i_Y\geq 2$.
Then there is a fundamental ample divisor $H$ on $Y$ such that $-K_Y=i_Y\cdot H$ in the Picard group $\text{Pic}(Y)$.
Now let $X\in |kH|$ be a smooth member for some positive integer $k<i_Y$.
Then by the adjunction formula, $X$ is a Fano variety of index $i_X=i_Y-k\geq 1$.

Denote by $\iota: X\rightarrow Y$ the inclusion. Notably, the pullback 
$\iota^*: H^n(Y, \mathbb{Q})\rightarrow H^n(X, \mathbb{Q})$ of cohomology groups is not an isomorphism in Case (a).
For our purpose, one may assume furthermore this is also satisfied in Case (b).

Choose an arbitrary smooth rational curve $R\subseteq Y$ of $(-K_Y)$-degree $d:=-\int_Y K_Y\cdot R\geq 2$ in Case (a) and of $H$-degree $d_0:=\int_Y H\cdot R\geq 2$ in Case (b).
Fix an ordered pair of positive integers $\alpha:=(m_1,m_2)$ such that $m_1+m_2=d$.

For a smooth variety $W$, let $W^{[d]}$ be the Hilbert scheme of closed subschemes of length $d$ on $W$ and $W_0^{[d]}$ be its principal component.
Denote by $\pi_0^W=[\cdot]: W_0^{[d]}\rightarrow W^{(d)}$ the Hilbert-Chow morphism.
Let $\text{Hilb}_{R}(Y)$ be the irreducible component of the Hilbert scheme parameterizing rational curves in $Y$ containing $R$. Note that its general member is smooth.

For our purpose, let us construct several relevant varieties by specifying intersection conditions. 
Define three reduced closed subvarieties of relevant varieties:
\begin{equation*}
\Theta_{\alpha}:=\{(x,y, C)\in X\times X\times\text{Hilb}_{R}(Y)|[X\cap C]=m_1x+m_2y,\ \text{or}\ C\subseteq X\},
\end{equation*}
\begin{equation*}
\Lambda_{\alpha}:=\{C\in\text{Hilb}_{R}(Y)|[X\cap C]=m_1x+m_2y,\ \text{or}\ C\subseteq X\},
\end{equation*}
\begin{equation*}
\Sigma_{\alpha}:=\{(x,y)\in X\times X|[X\cap C]=m_1x+m_2y,\ \text{or}\ C\subseteq X\}.
\end{equation*}
Let $q_1: \Theta_{\alpha}\rightarrow X$, $q_2: \Theta_{\alpha}\rightarrow X$, $q_{12}: \Theta_{\alpha}\rightarrow X\times X$ and $q_3: \Theta_{\alpha}\rightarrow\text{Hilb}_{R}(Y)$ be the natural projections.
By construction, it is easy to see that
\begin{equation*}
\Lambda_{\alpha}=q_3(\Theta_{\alpha}), \ \Sigma_{\alpha}=q_{12}(\Theta_{\alpha}).
\end{equation*}
Moreover, there is a dominant rational map
\begin{equation*}
\varphi_{\alpha}: \Lambda_{\alpha}\dashrightarrow\Sigma_{\alpha}
\end{equation*}
defined by $\varphi_{\alpha}(C)=(x,y)$ with $[X\cap C]=m_1x+m_2y$. Thus
\begin{equation*}
\text{dim}\Theta_{\alpha}\geq\text{dim}\Lambda_{\alpha}\geq\text{dim}\Sigma_{\alpha}.
\end{equation*}
Or rather, the corresponding irreducible components of these three varieties satisfy the same inequality.

\begin{proposition}\label{prop3.2}
(i) Assume that $X$ and $Y$ are as in Case (a).
The varieties $\Theta_{\alpha}$, $\Lambda_{\alpha}$ and $\Sigma_{\alpha}$ all have pure dimension $n$.
Moreover, any irreducible component of $\Sigma_{\alpha}$ is not equal to $\Delta_X$ and $\Sigma_{\alpha}$ represents an isogenous autocorrespondence of $X$.

(ii) Assume that $X$ and $Y$ are as in Case (b).
The varieties $\Theta_{\alpha}$, $\Lambda_{\alpha}$ and $\Sigma_{\alpha}$ all have pure dimension $n+i_X\cdot d_0$.
\end{proposition}
\begin{proof}
Overall, this is a typical argument in deformation theory.

(i) Let $X_{0,\alpha}^{[d]}$ (resp. $Y_{0,\alpha}^{[d]}$) be the reduced closed subvariety of $X_0^{[d]}$ (resp. $Y_0^{[d]}$) parameterizing smoothable subschemes of length $d$ with cycles of the form $[Z]=m_1x+m_2y$.  
Notably, $X_{0,\alpha}^{[d]}$ and $Y_{0,\alpha}^{[d]}$ may not be irreducible if $d\geq 4$.
Define the reduced subvariety of $Y_{0,\alpha}^{[d]}\times\text{Hilb}_{R}(Y)$:
\begin{equation*}
\Phi_{\alpha}:=\{(Z,C)\in Y_{0,\alpha}^{[d]}\times\text{Hilb}_{R}(Y)|Z\subseteq C\}.
\end{equation*}
Obviously, the second projection $\phi_{\alpha,2}: \Phi_{\alpha}\rightarrow\text{Hilb}_{R}(Y)$ is surjective and the general fiber $F_C\cong C^{(\alpha)}$ is irreducible of dimension 2,
where $C^{(\alpha)}$ is the stratum of $C^{(d)}$ consisting of effective cycles of the form $m_1x+m_2y$.
Since $\text{Hilb}_{R}(Y)$ is irreducible of dimension $\widetilde{n}:=d+n-2$,
then $\Phi_{\alpha}$ is irreducible and has dimension $\widetilde{n}+2=d+n$. 
Next, define the reduced subvariety of $X_{0,\alpha}^{[d]}\times\text{Hilb}_{R}(Y)$:
\begin{equation*}
\Psi_{\alpha}:=\{(Z,C)\in X_{0,\alpha}^{[d]}\times\text{Hilb}_{R}(Y)|X\cap C\supseteq Z\}.
\end{equation*}
Note that $X_0^{[d]}\times\text{Hilb}_{R}(Y)$ is an integral closed subvariety of $Y_0^{[l]}\times\text{Hilb}_{R}(Y)$ of codimension $d$.
Then at least as sets,
$$
\Psi_{\alpha}=\Phi_{\alpha}\cap (X_{0}^{[d]}\times\text{Hilb}_{R}(Y))
$$
in $Y_0^{[d]}\times\text{Hilb}_{R}(Y)$. According to the Projective Dimension Theorem, $\Psi_{\alpha}$ is nonempty and
each irreducible component of $\Psi_{\alpha}$ has dimension $\geq n$.
Observe that an infinitesimal computation implies that the conditions defining $\Psi_{\alpha}$ in $Y_0^{[d]}\times\text{Hilb}_{R}(Y)$ are infinitesimally independent on a general point. So, each irreducible component of $\Psi_{\alpha}$ has dimension $n$.
However, at this moment, we do not known whether $\Psi_{\alpha}$ is irreducible or not.

Clearly, there is a surjective morphism $\psi_{\alpha}: \Psi_{\alpha}\rightarrow\Theta_{\alpha}$ defined by $\psi_{\alpha}(Z,C)=(x,y, C)$.
Then $\text{dim}\Psi_{\alpha}\geq\text{dim}\Theta_{\alpha}$.
Let $p_{\alpha,i}:\Sigma_{\alpha}\rightarrow X$ be the natural projections.
The principal argument proceeds to show that the composite morphism $\widetilde{p}_{\alpha,i}:=p_{\alpha,i}\circ q_{12}\circ\psi_{\alpha}:\Psi_{\alpha}\rightarrow X$ is surjective on each of its components for each $i$.  By symmetry, it suffices to show that the general fiber of $\widetilde{p}_{\alpha,1}$ is nonempty on each component of $\Psi_{\alpha}$.

Let $Y_{\alpha,c}^{[d]}$ be the reduced variety obtained by taking the Zariski closure in $Y_{0}^{[d]}$ of the subset 
\begin{equation*}
\{Z_1\sqcup Z_2\in Y_{0}^{[d]}|Z_i\ \text{curvilinear of length}\ m_i\ \text{and supported on some point}\};
\end{equation*}
that is, $Y_{\alpha,c}^{[d]}$ parameterizes curvilinear closed subschemes with cycles of the form $m_1y_1+m_2y_2$ and their deformations in $Y$.
Note that the closed subvariety $Y_{\alpha}^{(d)}$ of $Y^{(d)}$ consisting of 0-cycles of the form $m_1y_1+m_2y_2$ is irreducible of dimension $2(n+1)$.
Then $Y_{\alpha,c}^{[d]}$ is the irreducible component of $(\pi_0^Y)^{-1}(Y_{\alpha}^{(d)})$ containing curvilinear closed subschemes, where $\pi_0^Y: Y_0^{[d]}\rightarrow Y^{(d)}$ is the Hilbert-Chow morphism.
Denote by $\widetilde{\pi}_{\alpha}^Y: Y_{\alpha,c}^{[d]}\rightarrow Y_{\alpha}^{(d)}$ the restriction of $\pi_0^Y$.

It is well-known that a closed subscheme of finite length of a smooth variety is smoothable if and only if its connected components are all smoothable. (But it should be notable that an arbitrary closed subscheme of a smoothable subscheme needs not be smoothable.)
Then for a general member $m_1y_1+m_2y_2\in Y_{\alpha}^{(d)}$,
the fiber
\begin{equation*}
(\widetilde{\pi}_{\alpha}^Y)^{-1}(m_1y_1+m_2y_2)=\{Z\sqcup Z_2\in Y_{\alpha,c}^{[d]}|Z_i\in Y_{(m_i),c}^{[m_i]}\}
\end{equation*}
is irreducible of dimension $(m_1-1)n+(m_2-1)n=(d-2)n$.
Therefore, $Y_{\alpha,c}^{[d]}$ is irreducible of dimension $dn+2$ and 
hence its codimension in $Y_{0}^{[d]}$ is exactly $d-2$.

On the other hand, for a fixed closed point $x\in X$, denote by $X_{\alpha,x}^{(d)}$ the reduced closed subvariety of $X^{(d)}$ consisting of 0-cycles of the form $m_1x+m_2x'$, 
which is irreducible of dimension $n$.
Let $X_{\alpha,c,x}^{[d]}$ be the reduced variety obtained by taking the Zariski closure in $X_{\alpha,c}^{[d]}$ of the subset
\begin{equation*}
\{Z_1\sqcup Z_2\in X_{\alpha,c}^{[d]}|Z_1\ \text{supported on}\ x\}.
\end{equation*}
Then $X_{\alpha,c,x}^{[d]}$ is the irreducible component of $(\pi_0^X)^{-1}(X_{\alpha,x}^{(d)})$ containing curvilinear closed subschemes.
Let $\widetilde{\pi}_{\alpha,x}^X: X_{\alpha,c,x}^{[d]}\rightarrow X_{\alpha,x}^{(d)}$ be the restriction of $\pi_0^X$.
Then its general fiber
\begin{equation*}
(\widetilde{\pi}_{\alpha,x}^X)^{-1}(m_1x+m_2x')=\{Z_1\sqcup Z_2\in X_{\alpha,c,x}^{[d]}|Z_2\ \text{supported on}\ x'\}
\end{equation*}
is irreducible and has dimension $(d-2)(n-1)$.
Thus $X_{\alpha,c,x}^{[d]}$ is irreducible of dimension $dn+2-d-n$ and so its codimension in $Y_{\alpha,c}^{[d]}$ is exactly $d+n$.

Denote by $\Psi_{\alpha}^k$ the irreducible components of $\Psi_{\alpha}$ and $\widetilde{p}_{\alpha,1}^k: \Psi_{\alpha}^k\rightarrow X$ the restriction of $\widetilde{p}_{\alpha,1}$. For each $k$ and any closed point $x\in X$, the fiber of the morphism $\widetilde{p}_{\alpha,1}^k: \Psi_{\alpha}^k\rightarrow X$ over $x$ is (as sets)
\begin{equation*}
(\widetilde{p}_{\alpha,1}^k)^{-1}(x)=\Phi_{\alpha}^k\cap (X_{\alpha,c,x}^{[d]}\times\text{Hilb}_{R}(Y))
\end{equation*}
with the intersection taken in $Y_{\alpha,c}^{[d]}\times\text{Hilb}_{R}(Y)$.
It has dimension $\geq 0$, by the Projective Dimension Theorem again. 
In particular, $(\widetilde{p}_{\alpha,1}^k)^{-1}(x)$ is nonempty.

Then one concludes that the morphism $\widetilde{p}_{\alpha,i}: \Psi_{\alpha}\rightarrow X$ is surjective on each irreducible component and so are the morphisms $q_i: \Theta_{\alpha}\rightarrow X$ and $p_{\alpha,i}:\Sigma_{\alpha}\rightarrow X$.
Therefore, the three varieties $\Theta_{\alpha}$, $\Lambda_{\alpha}$ and $\Sigma_{\alpha}$ are all of pure dimension $n$. Note that a general member of $\Lambda_{\alpha}$ is a smooth rational curve.
Moreover, $\Sigma_{\alpha}$ represents an isogenous autocorrespondence of $X$.
Since the cycle of any $Z\in\Psi_{\alpha}$ has the form $[Z]=m_1x+m_2y$ with $x$ and $y$ independent when deforming, then $\Sigma_{\alpha}=q_{12}\psi_{\alpha}(\Psi_{\alpha})$ does not contain $\Delta_X$ as a component.

(ii) In this case, $\text{Hilb}_{R}(Y)$ is irreducible of dimension $n+i_Y\cdot d_0-2$.
A similar argument as in (i) with minor modifications implies that the three varieties $\Theta_{\alpha}$, $\Lambda_{\alpha}$ and $\Sigma_{\alpha}$ all have pure dimension $n+i_X\cdot d_0$.
\end{proof}

\begin{remark}
(i) It is possible to verify that $\Sigma_{\alpha}$ represents a $K$-autocorrespondence in Case (a).
However, we do not need this slightly stronger property throughout this paper and so we skip this point.

(ii) If $m_1=m_2$, then $\Sigma_{\alpha}$ is symmetric, i.e., $\Sigma_{\alpha}=\Sigma_{\alpha}^t$.
\end{remark}

Now we can finish the required constructions as follows.

Case (a): take an arbitrary irreducible component $\widetilde{\Theta}_{\alpha}$ of $\Theta_{\alpha}$.
Let $\widetilde{\Lambda}_{\alpha}:=q_3(\widetilde{\Theta}_{\alpha})$ and $\widetilde{\Sigma}_{\alpha}:=q_{12}(\widetilde{\Theta}_{\alpha})$.

Case (b): denote by $\eta$ the generic point of $X$.
Then both $p_{\alpha,1}^{-1}(\eta)=p_{\alpha,2}^{-1}(\eta)$ and $q_{1}^{-1}(\eta)=q_{2}^{-1}(\eta)$
have pure dimension $i_X\cdot d_0$.
The induced morphism $q_{1}^{-1}(\eta)\rightarrow p_{\alpha,1}^{-1}(\eta)$ is generically finite and surjective.
Let $\widetilde{\Theta}_{\alpha}$ be the Zariski closure of any closed point $\xi$ of $q_{1}^{-1}(\eta)$ in $\Theta_{\alpha}$, $\widetilde{\Lambda}_{\alpha}$ be the Zariski closure of $q_3(\xi)$ in $\text{Hilb}_R(Y)$ and $\widetilde{\Sigma}_{\alpha}$ be the Zariski closure of $q_{12}(\xi)$ in $\Sigma_{\alpha}$ such that 
$\widetilde{\Sigma}_{\alpha}\neq \Delta_X$.

Then both in Case (a) and Case (b), the three varieties $\widetilde{\Theta}_{\alpha}$, $\widetilde{\Lambda}_{\alpha}$ and $\widetilde{\Sigma}_{\alpha}$ are integral of dimension $n$. Moreover, $\widetilde{\Sigma}_{\alpha}(\neq\Delta_X)$ is an isogenous autocorrespondence of $X$.
Now define two reduced varieties
\begin{equation*}
\Pi_{\alpha}:=\{(x, y)\in Y^2|x, y\in C,\ [C]\in\widetilde{\Lambda}_{\alpha}\},
\end{equation*}
\begin{equation*}
\Xi_{\alpha}:=\{(x, y, z)\in Y^3|x, y, z\in C,\ [C]\in\widetilde{\Lambda}_{\alpha}\}.
\end{equation*}
Then $\Pi_{\alpha}$ and $\Xi_{\alpha}$ are integral of dimension $n+2$ and $n+3$ respectively.

It turns out immediately that these varieties find their roles in certain cycle relations modulo rational equivalence.
\begin{proposition}\label{prop3.4}
Let $X$ and $Y$ be as in Case (a) or Case (b).
Denote by $\delta_{12}: X^2\rightarrow X^3$ the morphism defined by $(x,y)\mapsto (x, x, y)$ and similarly for $\delta_{13}$ and $\delta_{23}$.

(i) There is an equality of cycle classes:
\begin{equation}\label{34}
(\iota\times\iota)^*\Pi_{\alpha}=a_1\Delta_X+a_2(\widetilde{\Sigma}_{\alpha}+\widetilde{\Sigma}_{\alpha}^t)
\end{equation}
in $\text{CH}^n(X^2)$ with $a_i\in\mathbb{Z}_{+}$.

(ii) There exists an equality of cycle classes:
\begin{equation}\label{35}
(\iota\times\iota\times\iota)^*\Xi_{\alpha}
=b_1\Delta_{123}+b_2[\delta_{12*}(\widetilde{\Sigma}_{\alpha}+\widetilde{\Sigma}_{\alpha}^t)
+\delta_{13*}(\widetilde{\Sigma}_{\alpha}+\widetilde{\Sigma}_{\alpha}^t)+\delta_{23*}(\widetilde{\Sigma}_{\alpha}
+\widetilde{\Sigma}_{\alpha}^t)]
\end{equation}
in $\text{CH}^{2n}(X^3)$ with $b_i\in\mathbb{Z}_{+}$.

(iii) Assume that the cycle class $[\Xi_{\alpha}]\in A^{2n}(Y^3):=\text{CH}^{2n}(Y^3)/\sim_{\text{hom}}$ is a $\mathbb{Q}$-linear combination of intersections of the big diagonal classes and completely decomposable cycle classes. (E.g., $Y$ has trivial Chow groups).
Then 
\begin{equation}\label{36}
\frac{2a_1}{a_2}=\frac{b_1}{b_2}+d_1+d_2,
\end{equation}
where $d_i$ is the degree of the natural projection of $\widetilde{\Sigma}_{\alpha}$ to the $i$-th factor.
\end{proposition}
\begin{proof}
(i) Observe that as sets, one has
\begin{equation*}
\Pi_{\alpha}\cap X^2=\Delta_X\cup\widetilde{\Sigma}_{\alpha}\cup\widetilde{\Sigma}_{\alpha}^t.
\end{equation*}
Thus the two varieties $\Pi_{\alpha}$ and $X^2$ intersect properly in $Y^2$.
By intersection theory, one has 
\begin{equation*}
(\iota\times\iota)^*\Pi_{\alpha}=a_1\Delta_X+a_2\widetilde{\Sigma}_{\alpha}+a'_2\widetilde{\Sigma}_{\alpha}^t
\end{equation*}
in $\text{CH}^n(X^2)$ with $a_1, a_2, a'_2\in\mathbb{Z}_{+}$.
Since $\Pi_{\alpha}$ is symmetric by construction, one must have $a_2=a'_2$.
So, we get the formula \eqref{34}.

(ii) By inspection, the two varieties $\Xi_{\alpha}$ and $X^3$ intersect properly in $Y^3$. 
Moreover, one has as sets
\begin{equation*}
\Xi_{\alpha}\cap X^3
=\Delta_{123}^X\cup(\cup_{i<j}\delta_{ij*}(\widetilde{\Sigma}_{\alpha}))
\cup(\cup_{i<j}\delta_{ij*}(\widetilde{\Sigma}_{\alpha}^t)).
\end{equation*}
Hence, one gets that
\begin{equation*}
(\iota\times\iota\times\iota)^*\Xi_{\alpha}
=b_1\Delta_{123}^X+b_2\delta_{12*}(\widetilde{\Sigma}_{\alpha}+\widetilde{\Sigma}_{\alpha}^t)
+b'_2\delta_{13*}(\widetilde{\Sigma}_{\alpha}+\widetilde{\Sigma}_{\alpha}^t)
+b''_2\delta_{23*}(\widetilde{\Sigma}_{\alpha}+\widetilde{\Sigma}_{\alpha}^t)
\end{equation*}
in $\text{CH}^{2n}(X^3)$ with $b_1, b_2, b'_2,b''_2\in\mathbb{Z}_{+}$.
Now since $\Xi_{\alpha}$ is symmetric for each factor by construction, 
then one knows that $b_2=b'_2=b''_2$. Then one has the formula \eqref{35}.

(iii) Applying the push-forward $p_{12*}$ to the formula in (ii), one gets that
\begin{equation}\label{37}
p_{12*}(\iota\times\iota\times\iota)^*\Xi_{\alpha}
=(b_1+b_2(d_1+d_2))\Delta_X +2b_2(\widetilde{\Sigma}_{\alpha}+\widetilde{\Sigma}_{\alpha}^t).
\end{equation}
Modulo homological equivalence for the equalities \eqref{34} and \eqref{37}, one gets the following two equalities in $A^n(X^2)$:
\begin{equation}\label{38}
a_1[\Delta_X]+a_2([\widetilde{\Sigma}_{\alpha}]+[\widetilde{\Sigma}_{\alpha}^t])=[(\iota\times\iota)^*\Pi_{\alpha}],
\end{equation}
\begin{equation}\label{39}
(b_1+b_2(d_1+d_2))[\Delta_X]+2b_2([\widetilde{\Sigma}_{\alpha}]+[\widetilde{\Sigma}_{\alpha}^t])
=[p_{12*}(\iota\times\iota\times\iota)^*\Xi_{\alpha}].
\end{equation}
Substituting the equality \eqref{38} into the equality \eqref{39}, one has
\begin{equation}\label{40}
\left(b_1+b_2(d_1+d_2)-\frac{2a_1b_2}{a_2}\right)[\Delta_X]
=-\frac{1}{a_2}[(\iota\times\iota)^*\Pi_{\alpha}]+[p_{12*}(\iota\times\iota\times\iota)^*\Xi_{\alpha}].
\end{equation}

Since the cycle class $[\Xi_{\alpha}]$ is symmetric, by assumption, 
one may write it as a $\mathbb{Q}$-linear combination of cycle classes of the form
\begin{equation*}
z:=p_{12}^*\delta_{Y*}z_1\cdot p_3^*z_2+p_{13}^*\delta_{Y*}z_1\cdot p_2^*z_2+p_{23}^*\delta_{Y*}z_1\cdot p_1^*z_2
\end{equation*}
and completely decomposable cycle classes.
Since $p_{12*}z$ is completely decomposable, then so is the cycle class $[p_{12*}(\iota\times\iota\times\iota)^*\Xi_{\alpha}]$.
Denote by $\widetilde{\pi}_n^{\text{van}}$ the projector corresponding to the vanishing cohomology $H^n(X,\mathbb{Q})_{\text{van}}\subseteq H^n(X,\mathbb{Q})$.
Applying $\widetilde{\pi}_n^{\text{van}}\otimes\widetilde{\pi}_n^{\text{van}}$ to the equality \eqref{40},
one has
\begin{equation}\label{41}
\left(b_1+b_2(d_1+d_2)-\frac{2a_1b_2}{a_2}\right)[\Delta_X^{\text{van}}]=0
\end{equation}
with $[\Delta_X^{\text{van}}]:=(\widetilde{\pi}_n^{\text{van}}\otimes\widetilde{\pi}_n^{\text{van}})_*[\Delta_X]$.
By assumption, $\iota^*: H^n(X,\mathbb{Q})\rightarrow H^n(Y,\mathbb{Q})$ is not isomorphic, 
then $[\Delta_X]$ cannot be the restriction of a cycle class on $Y^2$. 
So, $[\Delta_X^{\text{van}}]\neq 0$.
By the equality \eqref{41}, this forces that
$$
b_1+b_2(d_1+d_2)-\frac{2a_1b_2}{a_2}=0.
$$
Equivalently, 
$$
\frac{2a_1}{a_2}=\frac{b_1}{b_2}+d_1+d_2.
$$
\end{proof}
\begin{remark}
It should be noted that the method was first introduced in \cite{Ba19} for K3 surfaces.
\end{remark}
\begin{corollary}\label{cor3.6}
Assume one of the following two conditions.

(i) Both $\Pi_{\alpha}\in\text{CH}^n(Y^2)$ and $\Xi_{\alpha}\in\text{CH}^{2n}(Y^3)$ are completely decomposable.
(E.g., $Y$ has trivial Chow groups.)

(ii) $Y$ has Picard number one, $\Pi_{\alpha}\in\text{CH}^n(Y^2)$ is completely decomposable and $\Xi_{\alpha}\in\text{CH}^{2n}(Y^3)$ is a $\mathbb{Q}$-linear combination of intersections of the big diagonal classes and divisor classes. (E.g., $Y$ is a Fano complete intersection. Refer to Theorem \ref{thm4.19}.)

Then there is an equality of cycle classes:
\begin{equation}\label{42}
\Delta_{123}=\Delta_{12}+\Delta_{13}+\Delta_{23}
+\delta_{12*}z+\delta_{13*}z+\delta_{23*}z+(\iota\times\iota\times\iota)^*w
\end{equation}
with $z\in\text{CH}^n(X^2)$ completely decomposable and $w\in\text{CH}^{2n}(Y^3)$.
\end{corollary}
\begin{proof}
By assumption, one can write $(\iota\times\iota)^*\Pi_{\alpha}=c'(X\times o_X)+z'$ such that $z'$ a $\mathbb{Q}$-linear combination of cycle classes of the form $z_1\times z_2$, where $z_1\in\text{CH}^i(X)$ and $z_2\in\text{CH}^{n-i}(X)$ with $i>0$.
According to the equalities \eqref{34} and \eqref{35} of Proposition \ref{prop3.4}(i)(ii), it is immediate to get the equality
\begin{equation}\label{43}
(b_1-\frac{3a_1b_2}{a_2})\Delta_{123}
-c(\Delta_{12}+\Delta_{13}+\Delta_{23})
=\frac{b_2}{a_2}\left(\delta_{12*}z'+\delta_{13*}z'
+\delta_{23*}z'\right)-(\iota\times\iota\times\iota)^*\Xi_{\alpha}
\end{equation}
with $c=\frac{b_2c'}{a_2}$.
Note that $b_1-\frac{3a_1b_2}{a_2}=-\left(\frac{a_1b_2}{a_2}+b_2(d_1+d_2)\right)\neq 0$ 
by the equality \eqref{36} of Proposition \ref{prop3.4} (iii).
Note that if $\text{CH}^1(Y)=\mathbb{Q}\cdot H$, then
$\Delta_X\cdot\iota^*H\times X=\frac{1}{d}(\iota\times\iota)^*\Delta_Y$, where $[X]=dH$.
Applying the push-forward $p_{12*}$ to the equality \eqref{43}, under the condition (i) or (ii), 
the cycle class in the right hand side becomes completely decomposable.
Then one gets that 
$$
c=b_1-\frac{3a_1b_2}{a_2}.
$$
Dividing by the constant $c$, one achieves that
$$
\Delta_{123}
=\Delta_{12}+\Delta_{13}+\Delta_{23}+\delta_{12*}z+\delta_{13*}z+\delta_{23*}z+(\iota\times\iota\times\iota)^*w
$$
with both $z\in\text{CH}^n(X^2)$ and $w\in\text{CH}^{2n}(Y^3)$ completely decomposable.
\end{proof}
\bigskip

\bigskip

\section{Complete intersections}
\bigskip

In this section, we mainly focus on complete intersections in certain ambient varieties.
The main results are Theorem \ref{thm4.3}, Theorem \ref{thm4.11} and Theorem \ref{thm4.19}.

The general setting is as follows.
Let $Y$ be a smooth connected projective variety with trivial Chow groups and $X\subseteq Y$ be a smooth connected \emph{ample} subvariety, e.g., complete intersections.
(Refer to \cite{Ot12} for definition of ample subvarieties.)
Denote $N=\text{dim}Y$, $n=\text{dim}X$ and $e:=N-n\geq 1$. Denote by $\iota: X\rightarrow Y$ the inclusion.
According to \cite{Ot12}, the Gysin homomorphism $\iota^*: H^{k}(Y,\mathbb{Q})\rightarrow H^{k}(X,\mathbb{Q})$ 
is isomorphic for each integer $k<n$, injective for $k=n$ and
the push-forward $\iota_*: H^{k}(X,\mathbb{Q})\rightarrow H^{k+2e}(Y,\mathbb{Q})$ is isomorphic for each $k>n$.
Denote by $b_i$ the $i$-th Betti number of $Y$.
For each integer $i$, choose a Poincar\'{e} dual basis $\{\beta_{ij}\}_{1\leq j\leq b_{2i}}$ of the Chow group $\text{CH}^i(Y)\cong H^{2i}(Y,\mathbb{Q})$, i.e., $\int_Y \beta_{ij}\cdot \beta_{N-i,j'}=\delta_{jj'}$.
Then $\beta_{01}=[Y]$ and $o_Y:=\beta_{N,1}$ is a 0-cycle class of degree one. Now fix an ample divisor class $L\in\text{CH}^1(Y)$.
By the hard Lefschetz theorem for singular cohomology,
there exist cycle classes $w_{ij}\in L^{2i-n}\cdot\text{CH}^{n-i}(Y)\subseteq\text{CH}^{i}(Y)$ such that $\iota_*\iota^*w_{ij}=\beta_{i+e,j}$ for each integer $i\geq\frac{n}{2}$ and each $j$,
where $\iota_*:\text{CH}_*(X)\rightarrow\text{CH}_*(Y)$ (resp. $\iota^*:\text{CH}^*(Y)\rightarrow\text{CH}^*(X)$)
is the push-forward (resp. pullback) homomorphism.
For each $i\neq\frac{n}{2}$, define the cycle classes $\alpha_{ij}\in\text{CH}^i(X)$ to be
$$
\alpha_{ij}:=\left\{
\begin{array}{ccc} \iota^*\beta_{ij}, && \text{if}\ \ i<\frac{n}{2};\\
\iota^*w_{ij}, && \text{if}\ \ i>\frac{n}{2}.
\end{array}
\right.
$$
Then one has $\int_X \alpha_{ij}\cdot \alpha_{n-i,j'}=\delta_{jj'}$ for each $i\neq \frac{n}{2}$.
Moreover, $\{cl_X(\alpha_{ij})\}$ forms a basis of the cohomology group $H^{2i}(X,\mathbb{Q})$ for each $i\neq \frac{n}{2}$, where $cl_X: \text{CH}^i(X)\rightarrow H^{2i}(X,\mathbb{Q})$ is the cycle class map.
Then $\alpha_{01}=[X]$ and $o_X:=\alpha_{n1}=\frac{1}{\int_X\iota^*L^n}\iota^*L^n$ is a 0-cycle class of degree one.
For each nonnegative integer $k\leq 2n$, define
\begin{equation*}
\pi_k^X:=\left\{
\begin{array}{ccc} \sum_{j=1}^{b_{k}}\alpha_{n-i,j}\times\alpha_{ij}, && \text{for even}\ k=2i<n;\\
\sum_{j=1}^{b_{k+2e}}\alpha_{n-i,j}\times\alpha_{ij}, && \text{for even}\ k=2i>n;\\
0, && \text{for odd}\ \ k\neq n;\\
\Delta_X-\sum_{\ell\neq n}\pi_{\ell}^X, && \text{for}\ \ k=n.
\end{array}
\right.
\end{equation*}
For $n=2i$ even, take a basis $\{\alpha_{ij}=\iota^*\gamma_{ij}\}_{1\leq j\leq b_n}$ of $\iota^*\text{CH}^i(Y)\subseteq\text{CH}^i(X)$
such that $\{cl_X(\alpha_{ij})\}_{1\leq j\leq b_n}$ forms an orthogonal basis of the primitive part $\iota^*H^n(Y,\mathbb{Q})\subseteq H^n(X,\mathbb{Q})$.
Then define $\pi_n^{\text{a}}:=\sum_{j=1}^{b_n}\alpha_{ij}\times\alpha_{ij}$ and $\pi_n^{\text{van}}:=\pi_n-\pi_n^{\text{a}}$.
Note that $\pi_0^X=o_X\times X$ and $\pi_{2n}^X=X\times o_X$.
It is easily seen that $\{\pi_k^X\}_{0\leq k\leq 2n}$ forms a set of self-dual CK projectors of $X$.
In the following, we will always equip $X$ with such a \emph{natural} self-dual CK decomposition.

For motivic 0-multiplicativity, it is convenient to reinterpret it as a cycle relation.
\begin{lemma}\label{lem4.1}
The following two conditions are equivalent.

(i) The natural CK decomposition of $X$ is multiplicative.

(ii) There is an equality of cycles classes:
\begin{equation}\label{44}
\Delta_{123}-(\Delta_{12}+\Delta_{13}+\Delta_{23})=\Omega_X
\end{equation}
in $\text{CH}^{2n}(X^3)$, 
where $\Delta_{ij}:=p_{ij}^*\Delta_X\cdot p_k^*o_X$,
$\Delta_i:=p_i^*[X]\cdot p_{j}^*o_X\cdot p_k^*o_X$ for $\{i,j,k\}=\{1, 2, 3\}$
and
\begin{eqnarray*}
% \nonumber to remove numbering (before each equation)
 \Omega_X:  &=& \sum_{0<i,j\neq\frac{n}{2}, i+j\neq n,\frac{3n}{2}}
\sum_{l_1, l_2, l_3}a_{ij}^{l_1l_2l_3}(\alpha_{n-i, l_1}\times\alpha_{n-j, l_2}\times\alpha_{i+j, l_3}) \\
   &+& \sum_{i\neq\frac{n}{2},n}\sum_{l_1, l_2}(\alpha_{n-i, l_1}\cdot\alpha_{i-\frac{n}{2},l_2})\times\alpha_{il_1}\times\alpha_{\frac{3n}{2}-i,l_2}
   +\alpha_{il_1}\times(\alpha_{n-i, l_1}\cdot\alpha_{i-\frac{n}{2},l_2})\times\alpha_{\frac{3n}{2}-i,l_2} \\
   &+& \alpha_{il_1}\times\alpha_{\frac{3n}{2}-i,l_2}\times(\alpha_{n-i, l_1}\cdot\alpha_{i-\frac{n}{2},l_2}) \\
   &+&\sum_{l_1, l_2}\sum_{i\neq\frac{n}{2}}2a_{n-i}^{l_1l_2}(o_X\times\alpha_{i,l_1}\times\alpha_{n-i,l_2} +\alpha_{i,l_1}\times o_X\times\alpha_{n-i,l_2}+\alpha_{i,l_1}\times\alpha_{n-i,l_2}\times o_X)
\end{eqnarray*}
with $a_{i}^{l_1l_2}:=\int_X\alpha_{il_1}\cdot\alpha_{n-i,l_2}\in\mathbb{Q}$ and $a_{ij}^{l_1l_2l_3}:=\int_X\alpha_{il_1}\cdot\alpha_{jl_2}\cdot\alpha_{n-i-j,l_3}\in\mathbb{Q}$.
\end{lemma}
\begin{proof}
Note that for $i,j\neq\frac{n}{2}$ with $i+j\geq n$ and $i+j\neq\frac{3n}{2}$, one has
$$
(\pi_{2i}\otimes\pi_{2j}\otimes\pi_{4n-2i-2j})_*\Delta_{123}
=\sum_{l_1, l_2, l_3}a_{n-i,n-j}^{l_1l_2l_3}(\alpha_{il_1}\times\alpha_{jl_2}\times\alpha_{2n-i-j,l_3}),
$$
$$
(\pi_{2i}\otimes\pi_{2j}\otimes\pi_n)_*\Delta_{123}
=\sum_{l_1, l_2}\alpha_{il_1}\times\alpha_{jl_2}\times(\alpha_{n-i, l_1}\cdot\alpha_{n-j,l_2})
$$
for $i,j\neq\frac{n}{2}$ with $i+j=\frac{3n}{2}$
and
$$
(\pi_n\otimes\pi_n\otimes\pi_{2n})_*\Delta_{123}
=\Delta_{12}+\sum_{l_1, l_2}\sum_{i\neq\frac{n}{2}}a_{n-i}^{l_1l_2}(\alpha_{i,l_1}\times\alpha_{n-i,l_2}\times o_X).
$$
Then the result follows immediately by expanding out the equality 
$$
\Delta_{123}=\sum_{k=0}^{2n}\sum_{i+j=k}(\pi_{2n-i}\otimes\pi_{2n-j}\otimes\pi_k)_*\Delta_{123}.
$$
\end{proof}
\begin{remark}
As a consequence, motivic 0-multiplicativity is stable under specialization for these kind of varieties,
because the cycle relation is generically defined.
\end{remark}

The first main result of this section is as follows.
\begin{theorem}\label{thm4.3}
Let $Y$ be a smooth connected projective variety with trivial Chow groups 
and $X\subseteq Y$ be any smooth connected ample subvariety. 
Then $\mathfrak{d}_m(X)\leq mn$ for any integer $m\geq 2$.
\end{theorem}
\begin{proof}
To warm up, we first consider the case: $m=2$.
By straightforward calculations, for three integers $i, j, k\neq\frac{n}{2}$, one gets that
\begin{eqnarray*}
% \nonumber to remove numbering (before each equation)
&  & \pi_{2k}\circ\Delta_{123}\circ (\pi_{2i}\otimes\pi_{2j})\\
&=& \sum_{l}\sum_{l'}\sum_{l''}\left(\int_X\alpha_{il}\cdot\alpha_{jl'}\cdot\alpha_{n-k,l''}\right)
(\alpha_{n-i, l}\times\alpha_{n-j, l'}\times\alpha_{kl''}).
\end{eqnarray*}
\begin{equation*}
\pi_{2k}\circ\Delta_{123}\circ (\Delta_X\otimes\pi_{2j})
=\sum_{l}\sum_{l'}(\alpha_{jl}\cdot\alpha_{n-k,l'})\times\alpha_{n-j,l}\times\alpha_{kl'},
\end{equation*}
\begin{equation*}
\pi_{2k}\circ\Delta_{123}\circ (\pi_{2i}\otimes\Delta_X)
=\sum_{l}\sum_{l'}\alpha_{n-i,l}\times(\alpha_{il}\cdot\alpha_{n-k,l'})\times\alpha_{kl'},
\end{equation*}
\begin{equation*}
\Delta_{123}\circ (\pi_{2i}\otimes\pi_{2j})
=\sum_{l}\sum_{l'}\alpha_{n-i,l}\times\alpha_{n-j,l'}\times(\alpha_{i,l}\cdot\alpha_{j,l'}),
\end{equation*}
\begin{equation*}
\pi_{2k}\circ\Delta_{123}
=(\Delta_X\otimes\Delta_X\otimes\pi_{2k})_*\Delta_{123}
=\sum_{l}\delta_{X*}\alpha_{n-k,l}\times\alpha_{kl},
\end{equation*}
\begin{equation*}
\Delta_{123}\circ (\Delta_X\otimes\pi_{2j})
=\sum_{l}p_{13}^*(\delta_{X*}\alpha_{jl})\cdot p_2^*\alpha_{n-j,l},
\end{equation*}
\begin{equation*}
\Delta_{123}\circ (\pi_{2i}\otimes\Delta_X)
=\sum_{l}\alpha_{n-i,l}\times\delta_{X*}\alpha_{il},
\end{equation*}
\begin{equation*}
\pi_0\circ\Delta_{123}\circ(\pi_n\otimes\pi_n)=(\pi_n\otimes\pi_n)_*(\delta_{X*}o_X)\times X=0,
\end{equation*}
\begin{equation*}
\pi_{n}\circ\Delta_{123}\circ(\pi_n\otimes\pi_{2n})=0,
\end{equation*}
\begin{equation*}
\pi_{n}\circ\Delta_{123}\circ(\pi_{2n}\otimes\pi_{n})=0.
\end{equation*}
If each $i, j, k\neq\frac{n}{2}$ and $k\neq i+j$, then $\int_X\alpha_{il}\cdot\alpha_{jl'}\cdot\alpha_{n-k,l''}=0$ for dimension reason and thus
\begin{equation*}
\pi_{2k}\circ\Delta_{123}\circ (\pi_{2i}\otimes\pi_{2j})=0.
\end{equation*}
For $i\neq\frac{n}{2}$ and each $l$, one has
\begin{eqnarray*}
% \nonumber to remove numbering (before each equation)
\pi_{n*}\alpha_{il}
&=& (\Delta_X-\sum_{j\neq\frac{n}{2}}\sum_{l'}\alpha_{n-j,l'}\times\alpha_{jl'})_*\alpha_{il}\\
&=& \alpha_{il}-\sum_{j\neq\frac{n}{2}}\sum_{l'}\left(\int_X\alpha_{n-j,l'}\cdot\alpha_{il}\right)\alpha_{jl'}\\
&=& 0.
\end{eqnarray*}
Thus, if $k>\frac{n}{2}+i$, then $n+i-k<\frac{n}{2}$ and hence the cycle class $\alpha_{il}\cdot\alpha_{n-k,l'}$ is a $\mathbb{Q}$-linear combination of $\alpha_{n+i-k,l''}$, according to Lemma \ref{lem4.8}.
If $k<i$, then $n+i-k>n$ and hence $\alpha_{il}\cdot\alpha_{n-k,l'}=0$ for dimension reason.
So, for both cases one gets that
\begin{equation*}
\pi_{2k}\circ\Delta_{123}\circ (\pi_n\otimes\pi_{2i})
=\sum_{l}\sum_{l'}\pi_{n*}(\alpha_{il}\cdot\alpha_{n-k,l'})\times\alpha_{n-i,l}\times\alpha_{kl'}=0,
\end{equation*}
\begin{equation*}
\pi_{2k}\circ\Delta_{123}\circ (\pi_{2i}\otimes\pi_n)
=\sum_{l}\sum_{l'}\alpha_{n-i,l}\times\pi_{n*}(\alpha_{il}\cdot\alpha_{n-k,l'})\times\alpha_{kl'}=0.
\end{equation*}

If $i+j<\frac{n}{2}$ or $i+j>n$, then
\begin{eqnarray*}
% \nonumber to remove numbering (before each equation)
\pi_n\circ\Delta_{123}\circ (\pi_{2i}\otimes\pi_{2j})
&=& (\pi_{2n-2i}\otimes\pi_{2n-2j}\otimes\pi_n)_*\Delta_{123}\\
&=& \sum_{l}\sum_{l'}\alpha_{n-i,l}\times\alpha_{n-j,l'}\times \pi_{n*}(\alpha_{il}\cdot\alpha_{jl'}))\\
&=& 0.
\end{eqnarray*}
Therefore, we obtain that $\mathfrak{d}(X)\leq 2n-1$.

Now we deal with the case: $m\geq 3$. If $k>\sum_{\ell=1}^{m}i_{\ell}$, then $i_{\ell'}\neq n$ for some $\ell'$; 
otherwise, $k>mn\geq 3n$, a contradiction.
By symmetry, one may assume that $\ell'=1$ and $i_1=2j$ with $j\neq\frac{n}{2}$ and $j\leq n$. Then one has
\begin{eqnarray*}
% \nonumber to remove numbering (before each equation)
   \pi_k\circ\Delta_I\circ(\pi_{i_1}\otimes\cdots\otimes\pi_{i_m})
   &=& (\pi_{2n-i_1}\otimes\cdots\otimes\pi_{2n-i_m}\otimes\pi_k)_*\Delta_I \\
   &=&
   \sum_{l}(\Delta_X\otimes\pi_{2n-i_2}\otimes\cdots\otimes\pi_{2n-i_m}\otimes\pi_k)_*
   (\alpha_{n-j,l}\times\delta_{m*}\alpha_{jl})\\ 
   &=& \sum_{l}\alpha_{n-j,l}\times(\pi_{2n-i_2}\otimes\cdots\otimes\pi_{2n-i_m}\otimes\pi_k)_*(\delta_{m*}\alpha_{jl}).
\end{eqnarray*}
It suffices to show that for each $m\geq 1$, each $l$ and each $j\leq n$ such that $j\neq\frac{n}{2}$ and $\sum_{\ell=1}^{m}k_{\ell}>2(m-1)n+2j$, one has
\begin{equation}\label{45}
(\pi_{k_1}\otimes\cdots\otimes\pi_{k_m})_*(\delta_{m*}\beta)=0
\end{equation}
for any cycle class $\beta\in\text{CH}^j(X)$.

We proceed by induction on $m\geq 1$. If $m=1$, this is obvious.
Assume that $m\geq 2$. Then there must be some $k_{\ell}\neq n$. One may assume that $k_1\neq n$ and $k_1=2n-2j'$ with $j'\neq\frac{n}{2}$.
By induction, one has
\begin{equation*}
(\pi_{k_1}\otimes\cdots\otimes\pi_{k_m})_*(\delta_{m*}\beta)
=\sum_{l'}\alpha_{n-j',l'}\times (\pi_{k_2}\otimes\cdots\otimes\pi_{k_m})_*((\delta_{m-1})_*(\beta\cdot\alpha_{j'l'}))
=0,
\end{equation*}
since $\sum_{\ell=2}^{m}k_{\ell}>2(m-2)n+2j+2j'$ and $\beta\cdot\alpha_{j'l'}\in\text{CH}^{j+j'}(X)$.

Suppose that $k<\sum_{\ell=1}^{m}i_{\ell}-mn$.
Then $i_{\ell'}>n$ for some $\ell'$.
One may assume that $i_1=2j>n$ by symmetry.
Then
\begin{equation*}
\pi_k\circ\Delta_I\circ(\pi_{i_1}\otimes\cdots\otimes\pi_{i_m})
=\sum_{l}\alpha_{n-j,l}\times(\pi_{2n-i_2}\otimes\cdots\otimes\pi_{2n-i_m}\otimes\pi_k)_*(\delta_{m*}\alpha_{jl}).
\end{equation*}
So it suffices to show that 
\begin{equation}\label{46}
(\pi_{2n-i_2}\otimes\cdots\otimes\pi_{2n-i_m}\otimes\pi_k)_*(\delta_{m*}\beta)=0
\end{equation}
for any cycle class $\beta\in\text{CH}^j(X)$.

We show by induction again on $m\geq 1$.
If $m=1$, then $\pi_{k*}\beta=0$, since $k<j$.
Now assume that $m\geq 2$.
Note that $\sum_{\ell=2}^{m}2n-i_{\ell}+k<(m-1)n+j$.
Then $i_{\ell'}\neq n$ for some $\ell'$ or $k\neq n$.
If $i_{\ell'}\neq n$ for some $\ell'$, then one may assume that $i_2=2j'<n$.
By induction,
\begin{eqnarray*}
% \nonumber to remove numbering (before each equation)
   & & (\pi_{2n-i_2}\otimes\cdots\otimes\pi_{2n-i_m}\otimes\pi_k)_*(\delta_{m*}\beta) \\
   &=& \sum_{l}\alpha_{n-j',l}\times(\pi_{2n-i_3}\otimes\cdots\otimes\pi_{2n-i_m}\otimes\pi_k)_*
   ((\delta_{m-1})_*(\beta\cdot\alpha_{j'l})) \\
   &=& 0,
\end{eqnarray*}
since $\sum_{\ell=3}^{m}2n-i_{\ell}+k<(m-2)n+j+j'$.
If $k=2k'<n$, then $j+n-k'>n$ and hence
\begin{eqnarray*}
% \nonumber to remove numbering (before each equation)
   & &  (\pi_{2n-i_2}\otimes\cdots\otimes\pi_{2n-i_m}\otimes\pi_k)_*(\delta_{m*}\beta) \\
   &=& \sum_{l}(\pi_{2n-i_2}\otimes\cdots\otimes\pi_{2n-i_m})_*((\delta_{m-1})_*(\beta\cdot\alpha_{n-k',l}))\times\alpha_{k'l} \\
   &=& 0.
\end{eqnarray*}
This completes the proof.
\end{proof}

\begin{corollary}
The inequality $\mathfrak{d}_m(X)\leq mn$ holds true for arbitrary products of curves, surfaces and smooth ample subvarieties in ambient varieties with trivial Chow groups.
\end{corollary}

For complete intersections in projective spaces, a slightly smaller bound of motivic multiple twist-multiplicativity defect can be attained.
\begin{proposition}
Let $Y=\mathbb{P}^{n+e}$ and $X\subseteq Y$ be a smooth complete intersection of dimension $n\geq 2$.
Then $\mathfrak{d}_m(X)\leq mn-1$ for any integer $m\geq 2$.
If $e\leq n$, then one has 
\begin{equation}\label{47}
\mathfrak{d}_m(X)\leq \text{max}\{(m-1)n, (m-2)n+2e-1\}.
\end{equation}
\end{proposition}
\begin{proof}
Denote by $D\in\text{CH}^1(X)$ the hyperplane section class. 
Let $d:=\int_XD^n$ be the degree of $X$. Then $o_X=\frac{1}{d}D^n$.
For $0\leq i\leq n$ and $i\neq\frac{n}{2}$, one has
\begin{equation*}
\pi_{2i}:=\frac{1}{d}D^{n-i}\times D^i,\quad
\pi_n:=\Delta_X-\sum_{i\neq\frac{n}{2}}\pi_{2i}.
\end{equation*}

(i) Case: $m=2$.
If $e\leq n$, then it is easily seen that 
\begin{equation*}
\delta_{X*}D^e=\Delta_X\cdot (D^e\times X)
=\frac{1}{d}(\iota\times\iota)^*\Delta_{\mathbb{P}^{n+e}}=\frac{1}{d}\sum_{j=e}^{n+e}D^{n+e-j}\times D^j.
\end{equation*}
Accordingly, for each $i\geq e$, it follows that
\begin{equation*}
\delta_{X*}D^i=\Delta_X\cdot (D^i\times X)=\frac{1}{d}\sum_{j=i}^{n+e}D^{n+i-j}\times D^j.
\end{equation*}
For $m\geq 2$ and $i\geq e$, one gets that
\begin{eqnarray}
% \nonumber to remove numbering (before each equation)
\delta_{m*}D^i
&=& \Delta^{m}\cdot (D^i\times D^{m-1})\label{48}\\
&=& (\Delta_X\times X^{m-2})\cdot (D^i\times X^{m-1})\cdot (X\times\Delta_X^{m-1})\nonumber \\
&=& \frac{1}{d}\sum_{j=i}^{n+e}D^{n+i-j}\times\delta_{m-1*}D^j.\nonumber
\end{eqnarray}

If $k\leq n-e$, then one obtains that
\begin{eqnarray*}
% \nonumber to remove numbering (before each equation)
\pi_{2k}\circ\Delta_{123}\circ(\pi_n\otimes\pi_n)
&=& (\pi_n\otimes\pi_n\otimes\pi_{2k})_*(\Delta_{123}) \\
&=& \frac{1}{d}(\pi_n\otimes\pi_n)_*(\delta_{X*}D^{n-k})\times D^k \\
&=& \frac{1}{d^2}\sum_{j=n-k}^{n+e}\pi_{n*}D^{2n-k-j}\times \pi_{n*}D^j\times D^k\\
&=& 0.
\end{eqnarray*}
For each $i\geq e$, one gets that
\begin{eqnarray*}
% \nonumber to remove numbering (before each equation)
  \pi_{n}\circ\Delta_{123}\circ(\pi_n\otimes\pi_{2i}) &=& (\pi_n\otimes\pi_{2n-2i}\otimes\pi_n)_*\Delta_{123} \\
   &=& (\pi_n\otimes\Delta_X\otimes\pi_n)_*(\frac{1}{d}p_{13}^*\delta_{X*}D^i\cdot p_2^*D^{n-i})\\
   &=& \frac{1}{d^2}\sum_{j=i}^{n+e}\pi_{n*}D^{n+i-j}\times D^{n-i}\times\pi_{n*}D^j\\
   &=& 0.
\end{eqnarray*}
By symmetry, one has $\pi_{n}\circ\Delta_{123}\circ(\pi_{2i}\otimes\pi_n)=0$ for $i\geq e$.

Then, we conclude that
\begin{eqnarray}
% \nonumber to remove numbering (before each equation)
   &  &\Delta_{123}-\sum_{i}\sum_j\pi_{i+j}\circ\Delta_{123}\circ(\pi_i\otimes\pi_j)\label{49}\\
   &=& \sum_{i=1}^{e-1}[\pi_{n}\circ\Delta_{123}\circ(\pi_{2i}\otimes\pi_n)
   +\pi_{n}\circ\Delta_{123}\circ(\pi_{n}\otimes\pi_{2i})]
   +\sum_{k=n-e+1}^{n-1}\pi_{2k}\circ\Delta_{123}\circ(\pi_n\otimes\pi_n)\nonumber\\
   &  & +\pi_n\circ\Delta_{123}\circ(\pi_n\otimes\pi_n).\nonumber
\end{eqnarray}
So, for $e\leq n$, one has $\mathfrak{d}(X)\leq\text{max}\{n, 2e-1\}$.

(ii) Case: $m\geq 3$.
Suppose that $k<\sum_{\ell=1}^{m}i_{\ell}-(mn-1)$.
If each $i_{\ell}=n$, then $k=0$.
Hence,
\begin{eqnarray*}
% \nonumber to remove numbering (before each equation)
   \pi_0\circ\Delta_I\circ(\pi_{n}\otimes\cdots\otimes\pi_{n})
   &=& (\pi_{n}\otimes\cdots\otimes\pi_{n}\otimes\pi_0)_*\Delta_I \\
   &=& (\pi_{n}\otimes\cdots\otimes\pi_{n})_*(\delta_{m*}o_X)\times X\\ 
   &=& 0.
\end{eqnarray*}
One may assume that $i_1=2j>n$ by symmetry.
Then
\begin{equation*}
\pi_k\circ\Delta_I\circ(\pi_{i_1}\otimes\cdots\otimes\pi_{i_m})
=\frac{1}{d}D^{n-j}\times(\pi_{2n-i_2}\otimes\cdots\otimes\pi_{2n-i_m}\otimes\pi_k)_*(\delta_{m*}D^j).
\end{equation*}
So, it suffices to show that for $m\geq 1$, one has
\begin{equation}\label{50}
(\pi_{2n-i_2}\otimes\cdots\otimes\pi_{2n-i_m}\otimes\pi_k)_*(\delta_{m*}D^j)=0.
\end{equation}

We proceed by induction on $m$.
If $m=1$, then $\pi_{k*}h^j=0$, since $k\leq j$.
Now assume that $m\geq 2$.
Note that $\sum_{\ell=2}^{m}2n-i_{\ell}+k<(m-1)n+j+1$.
If $i_{\ell}=n$ for each $\ell\geq 2$ and $k=n$, then $j=n$.
In this case, $(\pi_n\otimes\cdots\otimes\pi_n)_*((\delta_m)_*o_X)=0$.
If $i_{\ell'}\neq n$ for some $\ell'$, then one may assume that $i_2=2j'<n$,
since $j+j'>n$ and $D^{j+j'}=0$ for $2j'>n$.
By induction,
\begin{eqnarray*}
% \nonumber to remove numbering (before each equation)
   & & (\pi_{2n-i_2}\otimes\cdots\otimes\pi_{2n-i_m}\otimes\pi_k)_*(\delta_{m*}D^j) \\
   &=& \frac{1}{d}D^{n-j'}\times(\pi_{2n-i_3}\otimes\cdots\otimes\pi_{2n-i_m}\otimes\pi_k)_*((\delta_{m-1})_*D^{j+j'}) \\
   &=& 0,
\end{eqnarray*}
since $\sum_{\ell=3}^{m}2n-i_{\ell}+k<(m-2)n+j+j'+1$.
If $k=2k'<n$, then
\begin{eqnarray*}
% \nonumber to remove numbering (before each equation)
   & &  (\pi_{2n-i_2}\otimes\cdots\otimes\pi_{2n-i_m}\otimes\pi_k)_*(\delta_{m*}D^j) \\
   &=& (\pi_{2n-i_2}\otimes\cdots\otimes\pi_{2n-i_m})_*((\delta_{m-1})_*D^{j+n-k'}) \\
   &=& 0,
\end{eqnarray*}
since $j+n-k'>n$.

Assume that $e\leq n$ and $k<\sum_{\ell=1}^{m}i_{\ell}-[(m-2)n+2e-1]$.
If each $i_{\ell}=n$, then for $k=2k'<mn-[(m-2)n+2e-1]=2n-2e+1$, one has
\begin{eqnarray*}
% \nonumber to remove numbering (before each equation)
   & &\pi_k\circ\Delta_I\circ(\overbrace{\pi_{n}\otimes\cdots\otimes\pi_{n}}^m)\\
   &=& (\pi_{n}\otimes\cdots\otimes\pi_{n}\otimes\pi_k)_*\Delta_I \\
   &=& \frac{1}{d}(\pi_{n}\otimes\cdots\otimes\pi_{n})_*(\delta_{m*}D^{n-k'})\times D^{k'}\\ 
   &=& \frac{1}{d^2}\sum_{i=n-k'}^{n+e}\pi_{n*}D^{2n-k'-i}\times
   (\overbrace{\pi_n\otimes\cdots\otimes\pi_n}^{m-1})_*((\delta_{m-1})_*D^{i})\times D^{k'}\\
   &=& \frac{1}{d^3}\sum_{i=n-k'}^{n+e}\sum_{j=i}^{n+e}\pi_{n*}D^{2n-k'-i}\times\pi_{n*}D^{n+i-j}\times
   (\overbrace{\pi_n\otimes\cdots\otimes\pi_n}^{m-2})
   ((\delta_{m-2})_*D^j)\times D^{k'}\\
   &=& 0.
\end{eqnarray*}
The last equality follows from the fact that if $2n-k'-i=\frac{n}{2}$ and $n+i-j=\frac{n}{2}$, 
then $j=n+e>n$ and hence $h^j=0$.
If $i_{\ell'}\neq n$ for some $\ell'$, then one may assume that $\ell'=1$ by symmetry and $i_1=2j_1$.
Then
\begin{eqnarray*}
% \nonumber to remove numbering (before each equation)
   & &\pi_k\circ\Delta_I\circ(\pi_{i_1}\otimes\cdots\otimes\pi_{i_m})\\
   &=& (\pi_{2n-2j_1}\otimes\cdots\otimes\pi_{2n-i_m}\otimes\pi_k)_*\Delta_I \\
   &=& \frac{1}{d}D^{n-j_1}\times (\pi_{2n-i_2}\otimes\cdots\otimes\pi_{2n-i_m}\otimes\pi_{k})_*(\delta_{m*}D^{j_1}).
\end{eqnarray*}
It suffices to show that $(\pi_{2n-i_2}\otimes\cdots\otimes\pi_{2n-i_m}\otimes\pi_{k})_*(\delta_{m*}D^{j_1})=0$.
One may proceed by induction on $m\geq 2$.
If $m=2$, this follows by (i).
Assume that $m\geq 3$.
If $i_{\ell}=n$ for each $\ell\geq 2$ and $k=n$, 
then $j_1\geq e$. Thus, the result follows from the previous argument.
So, one may assume that $i_{2}=2j_2\neq n$.
Since $\sum_{\ell=3}^{m}2n-i_{\ell}+k<(m-2)n+2(j_1+j_2)-2e+1$,
then by induction, one gets that
\begin{eqnarray*}
% \nonumber to remove numbering (before each equation)
   & &(\pi_{2n-i_2}\otimes\cdots\otimes\pi_{2n-i_m}\otimes\pi_{k})_*(\delta_{m*}D^{j_1})\\
   &=& \frac{1}{d}D^{n-j_2}\times (\pi_{2n-i_3}\otimes\cdots\otimes\pi_{2n-i_m}\otimes\pi_{k})_*((\delta_{m-1})_*D^{j_1+j_2}) \\
   &=& 0.
\end{eqnarray*}
\end{proof}

\begin{remark}
(i) For $e\leq n$, the result relies on the equality \eqref{48} for each $i\geq e$. 
However, this may fail for $i<e$, when $X$ is of general type.
Indeed, if $X$ is a very general complete intersection curve of genus $g\geq 3$, 
then the divisor class $o_X$ cannot be represented by a point and thus
$\delta_{X*}o_X\neq o_X\times o_X$.

(ii) If $e=1$, i.e., $X$ is a hypersurface, then the above argument implies that the obstruction to the multiplicativity of the natural CK decomposition is exactly the cycle class 
\begin{equation*}
\pi_n\circ\Delta_{123}\circ(\pi_n\otimes\pi_n)=(\pi_n\otimes\pi_n\otimes\pi_n)_*\Delta_{123}\in\text{CH}_n^{2n}(X^3).
\end{equation*}
So, if $X$ is not motivic 0-multiplicative, then its motivic multiplicativity defect must be exactly $n$. 
An interesting phenomenon here is that there is \emph{no} in-between state.

(iii) By the same argument, the theorem still holds true for complete intersection with at most quotient singularities.

(iv) Presumably, the components of the cycle class $\Gamma^3(X, o_X)$ may be detected by its Mumford infinitesimal invariants or rather the higher Abel-Jacobi maps defined in \cite{As00}\cite{S01}. 
\end{remark}

\begin{definition}
For each integer $r\geq 0$ and $k\geq 1$, 
the \emph{tautological} subgroup $R^r(X^k)$ of $\text{CH}^r(X^k)$ (with respect to the closed embedding 
$\iota^{\times k}: X^k\rightarrow Y^k$) is defined to be 
\begin{equation*}
R^r(X^k):=\text{Image}((\iota^{\times k})^*: \text{CH}^r(Y^k)\rightarrow\text{CH}^r(X^k)).
\end{equation*}
The tautological subring of the Chow ring $\text{CH}^*(X^k)$ is defined as $R^*(X^k):=\oplus_{r}R^r(X^k)$.

Let $T^*(X^k)\subseteq\text{CH}^*(X^k)$ be the subring generated by $R^*(X^k)$ and all the big diagonal classes $p_{ij}^*\Delta_X$ with $i\neq j$, where $p_{ij}: X^k\rightarrow X^2$ is the natural projection to the $i$-th and $j$-th factors.
\end{definition}
\begin{lemma}\label{lem4.8}
Let $Y$ be a smooth connected projective variety with trivial Chow groups and $X$ be a smooth connected ample subvariety of $Y$. Then for each integer $r\leq\frac{n}{2}$, the restricted cycle class map
\begin{equation*}
cl_X: R^r(X)\rightarrow H^{2r}(X,\mathbb{Q})
\end{equation*}
is injective.
\end{lemma}
\begin{proof}
The Gysin homomorphism $\iota^*: H^{2r}(Y,{\mathbb{Q}})\rightarrow H^{2r}(X,{\mathbb{Q}})$ for cohomology groups 
is injective for each $r\leq\frac{n}{2}$, according to the generalized Lefschetz hyperplane theorem (Corollary 5.2 in \cite{Ot12}).
Since $Y$ has trivial Chow groups, then the Gysin homomorphism $\iota^*: \text{CH}^r(Y)\rightarrow\text{CH}^r(X)$ for Chow groups is also injective.
Accordingly, the restricted cycle class map $cl_X: R^r(X)\rightarrow H^{2r}(X,\mathbb{Q})$ is injective.
\end{proof}

It seems natural to propose the following.
\begin{question}\label{ques4.9}
Let $Y$ be a smooth connected projective variety with trivial Chow groups and $X\subseteq$ be a smooth connected ample subvariety. Is it true that the restricted cycle class map
\begin{equation*}
cl_X: R^*(X)\longrightarrow H^{*}(X,\mathbb{Q})
\end{equation*}
is injective?
\end{question}
For $X$ a complete intersection, this equivalently asks whether the Franchetta property holds true for the corresponding universal family. Obviously, if $Y$ is a (smooth) weighted projective space, then Question \ref{ques4.9} has an affirmative answer.

Motivic $0$-multiplicativity is equivalent to a property of certain tautological subring.
\begin{proposition}\label{prop4.10}
Let $Y$ be a smooth connected projective variety with trivial Chow groups and $X\subseteq$ be a smooth connected ample subvariety.
Then the natural CK decomposition of $X$ is multiplicative if and only if the restricted cycle class map
\begin{equation*}
cl_{X^3}: T^*(X^3)\longrightarrow H^*(X^3,\mathbb{Q})
\end{equation*}
is injective.
\end{proposition}
\begin{proof}
By Lemma \ref{lem4.1}, for each $k\neq i+j$, the cycle class $\pi_k\circ\Delta_{123}\circ(\pi_i\otimes\pi_j)$ is homologically trivial and lies in $T^*(X^3)$. If the map $cl_{X^3}: T^*(X^3)\longrightarrow H^*(X^3,\mathbb{Q})$
is injective, then one gets $\pi_k\circ\Delta_{123}\circ(\pi_i\otimes\pi_j)=0$.
By definition, the natural CK decomposition of $X$ is multiplicative.

For the converse, the assumption implies that the correspondence $\Gamma_{\iota}$ is of pure grade 0,
that is, $\Gamma_{\iota}=(\iota\times 1_Y)^*\Delta_Y\in\text{CH}_0^{n+1}(X\times Y)$.
Then for each integer $i$, the composite of morphisms of Chow motives $\mathfrak{h}^i(Y)\hookrightarrow\mathfrak{h}(Y)\stackrel{\iota^*}\longrightarrow\mathfrak{h}(X)$
factors through $\mathfrak{h}^i(X)$.
So, for any integer $k\geq 1$ and each integer $j$, the composite of morphisms of Chow motives $\mathfrak{h}^j(Y^k)\hookrightarrow\mathfrak{h}(Y^k)\stackrel{(\iota^{\times k})^*}\longrightarrow\mathfrak{h}(X^k)$
factors through $\mathfrak{h}^j(X^k)$, since $(\iota^{\times k})^*=(\iota^*)^{\otimes k}$.
Hence, the restricted cycle class map $cl_{X^k}: R^*(X^k)\rightarrow H^*(X^k,\mathbb{Q})$ is injective for any $k\geq 1$.

On the other hand, by assumption, for each integer $0<i<\frac{n}{2}$, one has
\begin{equation}\label{51}
0=(\pi_n\otimes\pi_n\otimes\pi_{2n-2i})_*\Delta_{123}
=\sum_{j}\left(\delta_{X*}\alpha_{ij}-(\iota\times\iota)^*\omega\right)\times\alpha_{n-i,j}
\end{equation}
where
\begin{eqnarray*}
% \nonumber to remove numbering (before each equation)
  \omega &:=& \sum_{k<\frac{n}{2}}\sum_{l=1}^{b_{2k}}w_{n-k, l}\times (\beta_{kl}\cdot\beta_{ij})+
\sum_{l=1}^{b_n}\gamma_{\frac{n}{2}l}\times(\gamma_{\frac{n}{2}l}\cdot\beta_{ij})+
\sum_{k>\frac{n}{2}}\sum_{l=1}^{b_{2k+2e}}\beta_{n-k, l}\times(w_{kl}\cdot\beta_{ij}).
\end{eqnarray*}
For each $0<i<\frac{n}{2}$ and each $j$, intersecting both sides of the equality \eqref{51} with the cycle class $X^2\times\alpha_{ij}$ and then applying the push-forward $p_{12*}$, one obtains that
\begin{equation*}
\Delta_X\cdot (X\times\alpha_{ij})=\delta_{X*}\alpha_{ij}=(\iota\times\iota)^*\omega.
\end{equation*}
From this, one immediately deduces that $R^r(X^2)=T^r(X^2)$ for each $r\neq n$.

Now observe that the remaining cycle relation in $T^*(X^3)$ is exactly that of multiplicativity of the natural CK decomposition of $X$. Therefore, the restricted cycle class map
$cl_{X^3}: T^*(X^3)\longrightarrow H^*(X^3,\mathbb{Q})$ is injective.
\end{proof}

Based on the previous preprarations, we can state the second main result of this section,
providing a criterion for motivic 0-multiplicativity.
\begin{theorem}\label{thm4.11}
Let $Y$ and $X$ be as in Section 3. Assume $Y$ has trivial Chow groups.
Then the natural CK decomposition of $X$ is multiplicative if and only if the following two conditions holds:

(i) Question \ref{ques4.9} has an affirmative answer for $X$;

(ii) $R^r(X^2)=T^r(X^2)$ for each integer $r\neq n$.

In particular, any Fano or Calabi-Yau hypersurface in a smooth weighted projective space
admits motivic 0-multiplicativity.
\end{theorem}
\begin{proof}
By Proposition \ref{prop4.10}, the conditions (i) and (ii) are necessary.

For the converse, assume the conditions (i) and (ii) hold.
According to Corollary \ref{cor3.6}, one knows immediately that the homologically trivial cycle class
\begin{equation*}
\Delta_{123}-\sum_{i}\sum_{j}\pi_{i+j}\circ\Delta_{123}\circ(\pi_i\otimes\pi_j)
\end{equation*}
must belong to the tautological subgroup $R^{2n}(X^3)$.
Then it follows from the condition (i) that
\begin{equation*}
\Delta_{123}=\sum_{i}\sum_{j}\pi_{i+j}\circ\Delta_{123}\circ(\pi_i\otimes\pi_j).
\end{equation*}
Hence, the natural CK decomposition of $X$ is multiplicative.
\end{proof}

\begin{remark}\label{rem4.12}
(i) Minor modifications of the argument infers that the result still holds for any Fano or Calabi-Yau hypersurface with at most quotient singularities in any quotient of a projective space by a finite group.
For example, a general Eisenbud-Popescu-Walter (abbr. EPW) sextic fourfold admits motivic 0-multiplicativity.

(ii) If $Y$ is a quadric, then $Y$ is trivial Chow groups and $X$ satisfies the two conditions of Theorem \ref{thm4.11}. 
So, $X$ admits motivic 0-multiplicativity.

(iii) The condition (ii) of Theorem \ref{thm4.11} can be deduced by the Franchetta property 
for the second relative power of the corresponding universal family.
It seems not easy to check the condition (ii) for $Y$ with non-Lefschetz cycle classes, 
unless they can be represented by varieties whose desingularizations have trivial Chow groups.
\end{remark}

As direct applications of Theorem \ref{thm4.11}, we test several interesting Fano varieties.
\begin{example}
Let $X$ be a prime Fano threefold of genus $g\leq 5$. Then $X$ admits motivic 0-multiplicativity.
Indeed, $X$ is a complete intersection and $X$ is an ample divisor in a quadric hypersurface.
Then the result follows by Theorem \ref{rem4.12}(ii).

In a concomitant work \cite{X25a}, we will test the case of all Fano threefolds.
\end{example}

\begin{example}
Let $X$ be a smooth Gushel-Mukai (GM) variety of dimension $n\in\{3, 4, 5\}$.
Then $X$ admits motivic 0-multiplicativity.
Indeed, a generically defined self-dual CK decomposition of $X$ has been explicitly provided in \cite{FM24}.
We will show that it is multiplicative.
It is sufficient to consider ordinary GM varieties by specialization.
Then $X$ can be realized as a smooth dimensionally transverse intersection $X=G(2,5)\cap\mathbb{P}(W)\cap Q\subseteq\mathbb{P}^9$, where $\mathbb{P}(W)$ is a linear subspace of dimension $n+4$ and $Q$ is a quadric hypersurface. Let $Y:=G(2,5)\cap\mathbb{P}(W)$. Then $X$ is a smooth ample divisor in $Y$.
Notably, $Y$ is Fano and has trivial Chow groups.
It is easily seen that the restricted cycle class map $cl: R^*(X)\rightarrow H^*(X,\mathbb{Q})$ is injective.
It remains to verify that $R^r(X^2)=T^r(X^2)$ for each $r\neq n$.
Note that $\text{CH}^1(X)=\mathbb{Q}\cdot H$ for each $n$ and the cycle class $\delta_{X*}H$ is completely decomposable.
Then the result follows immediately for $n=3$ or $4$.

For $n=5$, one has $X=G(2,5)\cap Q$ and $Y=G(2,5)$. This case has been proven in \cite{La21}.

The $n=6$ case will be dealt with in another concomitant work \cite{X25b}.
\end{example}

\begin{example}
Let $Y$ be a 5-fold with trivial Chow groups and $X\subseteq Y$ be a Fano hypersurface of cohomological K3 type.
Suppose $Y$ has Picard number one. Then $X$ admits motivic 0-multiplicativity.
Note that the Néron-Severi Lie algebra $\mathfrak{g}_{\text{NS}}(X)\cong\mathfrak{sl}_2(\mathbb{Q})$ by assumption.
Since $\delta_{X*}D$ is completely decomposable, then it is easily seen that the action of $\mathfrak{g}_{\text{NS}}(X)$ can be lifted to the Chow motive $\mathfrak{h}(X)$ and $\text{CH}^*(X)$.
Then $R^*(X)$ is an irreducible representation of $\mathfrak{g}_{\text{NS}}(X)$.
Since the map $\text{cl}_X: R^*(X)\rightarrow H^*(X,\mathbb{Q})$ is a homomorphism of $\mathfrak{g}_{\text{NS}}(X)$-modules, then it is injective.
Moreover, it is clear that $R^r(X^2)=T^r(X^2)$ for each $r\neq n$.
Then one concludes by Theorem \ref{thm4.11}.
\end{example}

Subsequently, we determine the possible values of the motivic 2-fold multiplicativity defect for the cases of Fano and Calabi-Yau complete intersections in smooth weighted projective spaces.
Notably, the conclusion cannot be drawn directly from the cases of generalized hypersurfaces,
because the apparent ambient varieties may have nontrivial Chow groups, which seems a difficult point.
To overcome this, a key observation is that one can carry out necessary constructions for relevant families and 
then use the fact that the Chow rings of related total spaces are simple enough for our purpose.

Let $Z=\mathbb{P}_w^N$ be a smooth weighted projective space of dimension $N=n+e$ with $e\geq 2$ and $X$ be a smooth Fano or Calabi-Yau complete intersection in $Z$ of multi-degrees $(d_1, d_2,\cdots, d_e)$ such that $d_{i+1}\geq d_i\geq 2$.
Since $X$ is Fano or Calabi-Yau, then $\sum_{i=1}^{e}d_i\leq i_{\mathbb{P}_w^N}$, the Fano index of $Z$.
Denote by $\mathbb{P}:=\mathbb{P}(\oplus_{i=1}^eH^0(\mathbb{P}_w^{N},\mathcal{O}_{\mathbb{P}_w^N}(d_i)))$ 
(resp. $\mathbb{P}':=\mathbb{P}(\oplus_{j=1}^{e-1}H^0(\mathbb{P}_w^{N},\mathcal{O}_{\mathbb{P}_w^N}(d_j)))$) the parameter space of all complete intersections of multi-degrees $(d_1, d_2,\cdots, d_e)$ 
(resp. of multi-degrees $(d_1, d_2,\cdots, d_{e-1})$).
By specialization, one may assume that $X$ is general and write $X=V(\sigma_1,\cdots,\sigma_e)$.
Fixing $\sigma_e$, one can identify $\mathbb{P}'$ with a linear section of $\mathbb{P}$.
Let $B\subseteq\mathbb{P}$ be the locus of smooth members and $B':=B\cap\mathbb{P}'$. 
Then $B'$ is a nonempty open subset of the locus of smooth complete intersections of multi-degrees $(d_1, d_2,\cdots, d_{e-1})$. Denote by $f: \mathcal{X}\rightarrow B$ and $g: \mathcal{Y}\rightarrow B'$ the corresponding universal families of smooth complete intersections respectively.
Let $f': \mathcal{X}'\rightarrow B'$ be the base change of $f$. Thus $X=\mathcal{X}'_{b'_0}$ for some closed point $b'_0\in B'$.
Clearly, there are closed embeddings $\mathcal{X}'\hookrightarrow\mathcal{Y}\hookrightarrow\mathbb{P}_{B'}^N$ over $B'$.

Now keep the notations in Section 3. Denote $i_X:=n+e+1-\sum_{i=1}^ed_i$.
Note that $i_X=0$ for $X$ Calabi-Yau and $i_X>0$ for $X$ Fano.
Let $F(\mathcal{Y}/B')$ be the relative Fano scheme of relative lines of $\mathcal{Y}/B'$.
Then $F(\mathcal{Y}/B')$ is smooth projective over $B'$ of dimension $n+i_X+\text{dim}B'$.
Define three varieties:
\begin{equation*}
\Theta_{\alpha,B'}=\{(x_{b'}, y_{b'}, L_{b'})\in\mathcal{X}_{/B'}^{'2}\times_{B'}F(\mathcal{Y}/B')|[\mathcal{X}'_{b'}\cap L_{b'}]=m_1x_{b'}+m_2y_{b'}, \ \text{or}\ L_{b'}\subseteq\mathcal{X}'_{b'}, \forall b'\in B'\},
\end{equation*}
\begin{equation*}
\Lambda_{\alpha, B'}:=\{L_{B'}\in F(\mathcal{Y}/B')|[\mathcal{X}'_{b'}\cap L_{b'}]=m_1x_{b'}+m_2y_{b'}, \ \text{or}\ L_{b'}\subseteq\mathcal{X}'_{b'}\ \forall b'\in B'\},
\end{equation*}
\begin{equation*}
\Sigma_{\alpha,B'}=\{(x_{b'}, y_{b'})\in\mathcal{X}'\times_{B'}\mathcal{X}'|[\mathcal{X}'_{b'}\cap L_{b'}]=m_1x_{b'}+m_2y_{b'}, \ \text{or}\ L_{b'}\subseteq\mathcal{X}'_{b'}, \forall b'\in B'\}.
\end{equation*}

Denote by $q_{i,B'}:\Theta_{\alpha,B'}\rightarrow\mathcal{X}'$, $q_{12,B'}:\Theta_{\alpha,B'}\rightarrow\mathcal{X}'\times_{B'}\mathcal{X}'$, 
$q_{3,B'}:\Theta_{\alpha,B'}\rightarrow F(\mathcal{Y}/B')$ and $p_{\alpha,i,B'}:\Sigma_{\alpha, B'}\rightarrow\mathcal{X}'$ the natural projections.

\begin{lemma}
The varieties $\Theta_{\alpha,B'}$, $\Lambda_{\alpha, B'}$ and $\Sigma_{\alpha,B'}$ all have pure dimension $n+i_X+\text{dim}B'$.
\end{lemma}
\begin{proof}
The natural projection $\Theta_{\alpha, B'}\rightarrow B'$ is surjective with fiber $\Theta_{\alpha, b'}$ over any closed point $b'$. By Proposition \ref{prop3.2}, $\Theta_{\alpha, b'}$ has pure dimension $n+i_X$.
Thus $\Theta_{\alpha, B'}$ is of pure dimension $n+i_X+\text{dim}B'$.
The argument is similar for $\Lambda_{\alpha, B'}$ and $\Sigma_{\alpha,B'}$.
\end{proof}

If $X$ is Calabi-Yau, let $\widetilde{\Theta}_{\alpha, B'}$ be any irreducible component of $\Theta_{\alpha, B'}$.
If $X$ is Fano, then both $p_{\alpha,1,B'}^{-1}(\eta')=p_{\alpha,2,B'}^{-1}(\eta')$ and $q_{1,B'}^{-1}(\eta')=q_{2,B'}^{-1}(\eta')$ have pure dimension $i_X$, where $\eta'$ is the generic point of $\mathcal{X}'$.
Note that the natural projection $q_{1,B'}^{-1}(\eta')\rightarrow p_{\alpha,1,B'}^{-1}(\eta')$ is generically finite and surjective.
Let $\widetilde{\Theta}_{\alpha, B'}$ be the Zariski closure of any closed point $\xi'\in q_{1,B'}^{-1}(\eta')$,
$\widetilde{\Lambda}_{\alpha,B'}$ be the Zariski closure of $q_{3,B'}(\xi')$ in $F(\mathcal{Y}/B')$ and $\widetilde{\Sigma}_{\alpha,B'}$ be the Zariski closure of $q_{12,B'}(\xi')$ in $\Sigma_{\alpha,B'}$ such that 
$\widetilde{\Sigma}_{\alpha,B'}\neq \Delta_{\mathcal{X'/B'}}$.
Then the three varieties $\widetilde{\Theta}_{\alpha, B'}$, $\widetilde{\Lambda}_{\alpha,B'}$ and $\widetilde{\Sigma}_{\alpha,B'}$
are integral of dimension $n+\text{dim}B'$.
Define two more varieties:
\begin{equation*}
\widetilde{\Pi}_{\alpha, B'}:=\{(x_{b'}, y_{b'})\in\mathcal{Y}\times_{B'}\mathcal{Y}|x_{b'},y_{b'}\in L_{B'}, L_{B'}\in\widetilde{\Lambda}_{\alpha, B'}\}.
\end{equation*}
\begin{equation*}
\widetilde{\Xi}_{\alpha, B'}:=\{(x_{b'}, y_{b'}, z_{b'})\in\mathcal{Y}\times_{B'}\mathcal{Y}\times_{B'}\mathcal{Y}|x_{b'},y_{b'},z_{b'}\in L_{B'}, L_{B'}\in\widetilde{\Lambda}_{\alpha, B'}\}.
\end{equation*}
These indeed relativize the constructions given in Section 3 for the cases of Fano or Calabi-Yau complete intersections.

It is immediate to know the following.
\begin{lemma}
The varieties $\widetilde{\Pi}_{\alpha, B'}$ and $\widetilde{\Xi}_{\alpha, B'}$ are integral of dimension $n+2+\text{dim}B'$ and $n+3+\text{dim}B'$ respectively.
\end{lemma}

A key lemma is the following. 
\begin{lemma}\label{lem4.18}
For any closed point $b'\in B'$, both the cycle classes $\widetilde{\Pi}_{\alpha,b'}$ and $\widetilde{\Xi}_{\alpha,b'}$ are generically defined, that is,
\begin{equation*}
\widetilde{\Pi}_{\alpha,b'}\in T^n(\mathcal{Y}_{b'}^2),\quad
\widetilde{\Xi}_{\alpha,b'}\in T^{2n}(\mathcal{Y}_{b'}^3).
\end{equation*}
\end{lemma}
\begin{proof}
As sets, one has 
\begin{equation*}
\widetilde{\Pi}_{\alpha, B'}\cap\mathcal{Y}_{b'}^2=\widetilde{\Pi}_{\alpha,b'},\quad
\widetilde{\Xi}_{\alpha, B'}\cap\mathcal{Y}_{b'}^3=\widetilde{\Xi}_{\alpha,b'}.
\end{equation*}
By intersection theory, one gets that
\begin{equation*}
\widetilde{\Pi}_{\alpha, B'}|_{\mathcal{Y}_{b'}^2}=l_1\widetilde{\Pi}_{\alpha,b'},\quad
\widetilde{\Xi}_{\alpha, B'}|_{\mathcal{Y}_{b'}^3}=l_2\widetilde{\Xi}_{\alpha,b'}
\end{equation*}
with $l_1,l_2\in\mathbb{Z}_{+}$.
Notably, the Chow groups of a natural compactification of $\mathcal{Y}$ are simple enough.
Then the same argument as in Proposition 4.1 and Proposition 5.3 in \cite{FLV19} implies that
\begin{equation*}
\widetilde{\Pi}_{\alpha,b'}\in T^n(\mathcal{Y}_{b'}^2),\quad
\widetilde{\Xi}_{\alpha,b'}\in T^{2n}(\mathcal{Y}_{b'}^3).
\end{equation*}
\end{proof}

Now we can state the third main result of this section.
\begin{theorem}\label{thm4.19}
Let $Y:=\mathbb{P}_w^{N}$ be any smooth weighted projective space of dimension $N=n+e$ with $e\geq 1$
and $X\subseteq Y$ be a smooth Fano or Calabi-Yau complete intersection of dimension $n$. 

(i) There is an equality:
\begin{equation}\label{52}
\Delta_{123}^X=\delta_{12*}P(D_1, D_2)+\delta_{13*}P(D_1, D_2)+\delta_{23*}P(D_1, D_2)
+Q(D'_1, D'_2, D'_3)
\end{equation}
in $\text{CH}^{2n}(X^3)$, where both $P(s_1,s_2)\in\mathbb{Q}[s_1,s_2]$ and $Q(t_1,t_2,t_3)\in\mathbb{Q}[t_1,t_2,t_3]$ are symmetric homogenous polynomials, $D_i=p_i^*D$ and $D'_j=q_j^*D$ with $p_i, q_j$ the natural projections.

(ii) The natural CK decomposition of $X$ is multiplicative 
if and only if the cycle class $\delta_{X*}D=\frac{1}{d_e}(\iota\times\iota)^*\Delta_Y$ is completely decomposable;
equivalently, the correspondence $\Gamma_{\iota}$ is of pure grade 0.

(iii) $\mathfrak{d}(X)=2k$ with $1\leq k\leq n-1$ and $k\neq\frac{n}{2}$
if and only if the cycle class $\delta_{X*}D^{k+1}$ is completely decomposable but $\delta_{X*}D^{k}$ is not.
\end{theorem}
\begin{proof}
(i) We will prove the equality \eqref{52} by induction on $e\geq 1$.
The $e=1$ case is a consequence of Theorem \ref{thm4.11}.
Assume $e\geq 2$. Note that $\widetilde{\Pi}_{\alpha}=\widetilde{\Pi}_{\alpha,b'}$ for some closed point $b'\in B'$.

Denote by $H\in\text{CH}^1(\mathcal{Y}_{b'})$ the hyperplane class and $D=\iota^*H$.
Since $\widetilde{\Pi}_{\alpha}$ and $\widetilde{\Xi}_{\alpha}$ are symmetric, 
by Lemma \ref{lem4.18}, one gets that
\begin{equation}\label{53}
\widetilde{\Pi}_{\alpha}=\sum_{i}\sum_ja_iH^{n-i}\times H^i\in T^n(\mathcal{Y}_{b'}^2),
\end{equation}
\begin{equation}\label{54}
\widetilde{\Xi}_{\alpha}=a(p_{12}^*\Delta_Y\cdot p_{3}^*H^{n-1}+p_{13}^*\Delta_Y\cdot p_{2}^*H^{n-1}+p_{23}^*\Delta_Y\cdot p_{1}^*H^{n-1})
+\sum_{i}\sum_jb_{ij}H^{2n-i-j}\times H^i\times H^j
\end{equation}
for some $a, a_i, b_{ij}\in\mathbb{Q}$.
Resorting to Corollary \ref{cor3.6}, one obtains that
\begin{equation*}
\Delta_{123}=\delta_{12*}P(D_1,D_2)+\delta_{13*}P(D_1,D_2)+\delta_{23*}P(D_1,D_2)+Q(D'_1,D'_2,D'_3)
\end{equation*}
with both $P(s_1,s_2)\in\mathbb{Q}[s_1,s_2]$ and $Q(t_1,t_2,t_3)\in\mathbb{Q}[t_1,t_2,t_3]$ symmetric homogenous.

(ii) This follows immediately from Theorem \ref{thm4.11}.

(iii) By (i), one has $\mathfrak{d}(X)\neq n$ and hence it is even.
Since $X$ is Fano or Calabi-Yau, then $o_X$ can be represented by any point on any rational curve. Hence, $\delta_{X*}o_X=o_X\times o_X$.
If the cycle class $\delta_{X*}D^k$ is completely decomposable, then so is $\delta_{X*}D^i$ for any $i\geq k$.
Therefore, $\mathfrak{d}(X)=2k$ with $1\leq k\leq n-1$ and $k\neq\frac{n}{2}$ if and only if
\begin{equation*}
0\neq (\pi_n\otimes\pi_n\otimes\pi_{2n-2k})_*\Delta_{123}
=\left(\delta_{X*}D^k-\frac{1}{d}\sum_{j=k}^{n+e}D^{n+k-j}\times D^j\right)\times D^{n-k}
\end{equation*}
and
\begin{equation*}
0=(\pi_n\otimes\pi_n\otimes\pi_{2n-2(k+1)})_*\Delta_{123}
=\left(\delta_{X*}D^{k+1}-\frac{1}{d}\sum_{j=k+1}^{n+e}D^{n+k+1-j}\times D^j\right)\times D^{n-k-1}
\end{equation*}
if and only if
\begin{equation*}
\delta_{X*}D^k\neq\frac{1}{d}\sum_{j=k}^{n+e}D^{n+k-j}\times D^j,\quad
\delta_{X*}D^{k+1}=\frac{1}{d}\sum_{j=k+1}^{n+e}D^{n+k+1-j}\times D^j,
\end{equation*}
that is, the cycle class $\delta_{X*}D^k$ is not completely decomposable but $\delta_{X*}D^{k+1}$ is not.
\end{proof}

\begin{remark}
(i) In the case of general Calabi-Yau complete intersections, the statement (i) implies that
the cycle class $\Gamma:=\bigcup_{t\in F(X)}\mathbb{P}_t^1\times\mathbb{P}_t^1\times\mathbb{P}_t^1$ with $F(X)$
the Fano variety of lines of $X$, is the restriction of a cycle class on $\mathbb{P}^N\times\mathbb{P}^N\times\mathbb{P}^N$,
strengthening Theorem 0.7 in \cite{F13}.

(ii) The result still holds true for Fano or Calabi-Yau complete intersections with at most quotient singularities in any quotient of a projective space by a finite group.

(iii) Unfortunately, if each $Y_k:=V(\sigma_k)$ has nontrivial Chow groups, 
then the cycle class $\delta_{X*}D$ may fail to be completely decomposable even if $e=2$.
Let us explain why. For simplicity, take $Y=Y_k$ and write as cycle:
\begin{equation}\label{55}
\Delta_{X}-\frac{1}{d_1^{n+1}d_2}\sum_{i=1}^{n+1}X_{n+1-i}\times X_i=\sum_{j}\text{div}(f_j)
\end{equation}
in the free group of cycles $Z^*(Y\times Y)$, where $X_{i}$ are $i$-dimensional linear sections of $X$ and $f_j\in\mathbb{C}(W_j)^*$ with $W_j\subseteq Y\times Y$ integral of dimension $n+1$.
Since the left hand side of \eqref{55} is symmetry, then $W_j$ and $f_j$ should be taken to be symmetric.
(Actually, one could write explicitly $f_j$ and $W_j$ by chasing routes of rational equivalence by intersection theory.)
Since $Y$ has nontrivial Chow groups, then $W_j$ should dominate $Y$ by the natural projections.
Thus, the equality \eqref{55} may fail to hold on $X\times Y$, that is, the equality $\Gamma_{\iota}=\sum_{i=0}^{n}D^i\times H^{n+1-i}$ may fail in $\text{CH}^{n+1}(X\times Y)$ (or rather, $\Gamma_{\iota}$ may be not of pure grade 0). 
So, the cycle class $\delta_{X*}D$ may be not completely decomposable.
\end{remark}
\bigskip

\section{Applications}
\bigskip

In this section, we provides several applications of the main results in the previous sections.

\begin{proposition}
Let $X$ be a smooth Fano or Calabi-Yau hypersurface of dimension $n$ in a smooth weighted projective space.
Then $X$ satisfies the sharp modified diagonal property, that is, 
the modified diagonal class $\Gamma^{k}(X, o_X)$ vanishes in $\text{CH}_n(X^k)$ for any integer $k>n$.
\end{proposition}
\begin{proof}
This is an immediate consequence of Proposition \ref{prop2.33} and Theorem \ref{thm4.11}.
\end{proof}

\begin{proposition}\label{prop5.2}
Let $X$ be a smooth Fano or Calabi-Yau hypersurface of dimension $n$ in a smooth weighted projective space.
Denote by $b_{n}^{\text{tr}}(X)$ the dimension of the transcendental part of $H^n(X,\mathbb{Q})$.
Then for each positive integer $k\leq 2b_{n}^{\text{tr}}(X)+1$, the restricted cycle class map
\begin{equation*}
\text{cl}_{X^k}: T^*(X^k)\rightarrow H^*(X^k,\mathbb{Q})
\end{equation*}
is injective.
Furthermore, this holds for any integer $k\geq 1$ if and only if the Chow motive of $X$ is finite-dimensional.
\end{proposition}
\begin{proof}
The first part follows from Theorem \ref{thm4.11}.
The argument of the second part is essentially similar to that of Proposition 4.5 in \cite{FLV21}.
\end{proof}

\begin{remark}
For each $k\leq 2b_{n}^{\text{tr}}(X)+1$, it is possible to lift the action of the Néron-Severi Lie algebra $\mathfrak{g}_{\text{NS}}(X^k)$ on cohomology groups to the Chow motive $\mathfrak{h}(X^k)$ 
and then to the Chow group $\text{CH}^*(X^k)$.
Thus the tautological subring $R^*(X^k)$ would be realized as an irreducible representation of $\mathfrak{g}_{\text{NS}}(X^k)$, while $T^*(X^k)$ would be an irreducible representation of a larger (but not necessarily semisimple) Lie algebra over $\mathbb{Q}$.
\end{remark}

\begin{corollary}
Let $f:  \mathcal{X}\rightarrow B$ be the universal family of smooth Fano or Calabi-Yau hypersurfaces of degree $d$ in a projective space.
Then for any $k\leq d+1$, the $k$-th relative power $f_k: \mathcal{X}_{/B}^k\rightarrow B$ of $f$
satisfies the Franchetta property.
\end{corollary}
\begin{proof}
Note that 
\begin{equation*}
\text{Im}(\iota_k^*: \text{CH}^*(\mathcal{X}_{/B}^k)\rightarrow\text{CH}^*(\mathcal{X}_b^k))\subseteq R^*(\mathcal{X}_b^k).
\end{equation*}
Then the result follows from Proposition \ref{prop5.2}.
\end{proof}

For general consideration, we propose the following conjecture of Franchetta type, among other things, generalizing the case of hyper-Kähler varieties.

\begin{conjecture}
Let $X$ and $B$ be smooth connected varieties and $f: X\rightarrow B$ be a smooth projective morphism
such that a very general fiber of $f$ admits motivic 0-multiplicativity.
Assume that the monodromy-invariant sub-Hodge structure $H^*(X_b,\mathbb{Q})^{\pi_1(B,b)}$ consists of Hodge classes for a very general point $b\in B$. 
Then the Franchetta property holds true for any relative power of $f$, that is,
for each integer $N\geq 1$ and any cycle class $\Gamma\in\text{CH}^*(X_{/B}^N)$,
the restriction $\Gamma|_{X_b^N}$ vanishes for any closed point $b\in B$,
if it is homologically trivial for a very general point $b\in B$.
\end{conjecture}

As the last application, we demonstrate that motivic 0-multiplicativity of a very general fiber of a smooth projective family leads to multiplicative decomposition isomorphism in the corresponding derived category.
\begin{proposition}
Let $X$ and $B$ be smooth varieties and $f: X\rightarrow B$ be a smooth projective morphism. 
Assume that a very general fiber of $f$ admits a multiplicative CK decomposition. 
Then after restriction to some dense Zariski open subset $U\subseteq B$, there exists a multiplicative decomposition isomorphism
\begin{equation}\label{56}
\mathbf{R}(f|_U)_*\mathbb{Q}\cong\bigoplus_i\mathbf{R}^i(f|_U)_*\mathbb{Q}[-i]
\end{equation}
in the derived category of sheaves of $\mathbb{Q}$-vector spaces on $U$.

In particular, the universal family of Fano or Calabi-Yau hypersurfaces in a smooth weighted projective space satisfies this property.
\end{proposition}
\begin{proof}
Denote by $\eta$ the generic point of $B$. Then $\overline{\eta}:=\overline{\mathbb{C}(\eta)}\cong\mathbb{C}$ as fields.
By definition, there is an isomorphism of Chow groups $\text{CH}^*(X_{\overline{\eta}}^2)\cong\text{CH}^*(X_{b}^2)$ for a very general closed point $b\in B$.
Since by assumption, a very general fiber of $f$ admits a multiplicative CK decomposition,
then so does the generic fiber $X_{\eta}$ of $f$ by Galois descent.

Note that the pullbacks of the restriction morphisms to the generic fibers commute with push-forward, intersection product and composition of correspondences, because they are flat.
By Lemma (1A.1) in \cite{B10}, there exists a dense Zariski open subset $U\subseteq B$ such that $f|_U: f^{-1}(U)\rightarrow U$ admits a multiplicative relative CK decomposition over $U$. 
Denote by $\text{CH}\mathcal{M}(U)$ the category of (relative) Chow motives over $U$,
by $\mathrm{D}(U; \mathbb{Q})$ the derived category of sheaves of $\mathbb{Q}$-vector spaces on $U$
and by $r_U: \text{CH}\mathcal{M}(U)\rightarrow \mathrm{D}(U; \mathbb{Q})$ the realization functor. 
Applying the functor $r_U$ to the previous multiplicative relative CK decomposition,
one gets a multiplicative decomposition isomorphism \eqref{56}.
\end{proof}
\bigskip

\bigskip

{\noindent\textbf{Funding}}
\bigskip

This work was supported by the Fundamental Research Funds of Shandong University
No.11140075614046, by the Future Program for Young Scholars of Shandong University No.11140089964236 
and by the National Natural Science Foundation of China No.11601276 and No.11771426.
\bigskip

\bigskip

{\noindent\textbf{Acknowledgments}}
\bigskip

The author would like to thank Professor Baohua Fu and Professor Kejian Xu gratefully for discussions, constant encouragements and supports. 

\bigskip

\bigskip

\begin{thebibliography}{------}
\bigskip

\bibitem[1]{As00}
Asakura, M., Motives and algebraic de Rham cohomology. CRM Proc. Lecture Notes, \textbf{24}, American Mathematical Society, Providence, RI, 2000, 133-154.

\bibitem[2]{Ba19}
Bazhov, I., On the decomposition of the small diagonal of a K3 surface. Adv. Geom. \textbf{19} (2019), no. 3, 353-358.

\bibitem[3]{Bea23}
Beauville, A., Schoen, C., A non-hyperelliptic curve with torsion Ceresa cycle modulo algebraic equivalence. Int. Math. Res. Not., Vol. \textbf{2023}, No. 5, pp. 3671-3675.

\bibitem[4]{BV04}
Beauville, A. and Voisin, C., On the Chow ring of a K3 surface. J. Alg. Geom. \textbf{13} (2004), 417-426.

\bibitem[5]{Bei87}
Beilinson, A., Height pairing between algebraic cycles. In: \emph{$K$-Theory, Arithmetic and Geometry (Y. Manin, ed.)}, 
Lecture Notes in Math., vol. \textbf{1289}, Springer-Verlag, New York, 1987, pp. 1-26.

\bibitem[6]{BLLS23}
Bisogno, D., Li, W., Litt, D., Srinivasan, P., Group-theoretic Johnson classes and non-hyperelliptic curves with torsion Ceresa class. Épijournal Géom. Algébrique \textbf{7} (2023), Art. 8, 19 pp.

\bibitem[7]{B10}
Bloch, S., Lectures on algebraic cycles. New Math. Monogr. \textbf{16}, Cambridge University Press, Cambridge, 2010, xxiv+130 pp.

\bibitem[8]{BL24}
Bolognesi, M., Laterveer, R., Some motivic properties of Gushel-Mukai sixfolds. Math. Nachr. \textbf{297} (2024), no. 1, 246-265.

\bibitem[9]{CC83}
Ceresa, G., Collino, A., Some remarks on algebraic equivalence of cycles. Pacific Journal of Mathematics, Vol. \textbf{105}, No. 2, 1983.

\bibitem[10]{D21}
Diaz, H. A.,  The Chow ring of a cubic hypersurface. Int. Math. Res. Not. \textbf{2021}, no. 22, 17071-17090.

\bibitem[11]{F13}
Fu, L., Decomposition of small diagonals and Chow rings of hypersurfaces and Calabi-Yau complete intersections. Adv. Math. \textbf{244} (2013), 894-924.

\bibitem[12]{FLV19}
Fu, L., Laterveer, R., Vial, C., The generalized Franchetta conjecture for some hyper-Kähler varieties. J. Math. Pures Appl. (9) \textbf{130} (2019), 1-35.

\bibitem[13]{FLV21}
Fu, L., Laterveer, R., Vial, C., Multiplicative Chow-Künneth decompositions and varieties of cohomological K3 type.
Ann. Mat. Pura Appl. (4) \textbf{200} (2021), no. 5, 2085-2126.

\bibitem[14]{FM24}
Fu, L.; Moonen, B., Algebraic cycles on Gushel-Mukai varieties. Épijournal Géom. Algébrique. Special volume in honour of C. Voisin, Article No. 17 (2024).

\bibitem[15]{Fu98}
Fulton, W., Intersection theory. Second edition. Ergebnisse der Mathematik und ihrer Grenzgebiete; Folge 3, Vol. \textbf{2}, Springer-Verlag, Berlin, 1998.

\bibitem[16]{Ja94}
Jannsen, U., Motivic sheaves and filtrations on Chow groups. Proc. Sympos. Pure Math., \textbf{55}, Part 1. American Mathematical Society, Providence, RI, 1994, 245-302.

\bibitem[17]{Ji23}
Jiang, Q., On the Chow theory of projectivizations. J. Inst. Math. Jussieu \textbf{22} (2023), no. 3, 1465-1508.

\bibitem[18]{KMP07}
Kahn, B., Murre, J., Claudio P., On the transcendental part of the motive of a surface. In: \emph{Algebraic cycles and motives}. Vol. 2, London Math. Soc. Lecture Note Ser., vol. \textbf{344}, Cambridge Univ. Press, Cambridge, 2007, pp. 143-202. 

\bibitem[19]{K94}
Kleiman, S. L., The standard conjectures. Proc. Sympos. Pure Math. \textbf{55}, Part 1. American Mathematical Society, Providence, RI, 1994, 3-20.

\bibitem[20]{La19}
Laterveer, R., On the Chow ring of certain hypersurfaces in a Grassmannian. Matematiche (Catania) \textbf{74} (2019), no. 1, 95-108.

\bibitem[21]{La21}
Laterveer, R., Algebraic cycles and Gushel-Mukai fivefolds. J. Pure Appl. Algebra \textbf{225} (2021), no. 5, Paper No. 106582, 18 pp.

\bibitem[22]{La24}
Laterveer, R., Some more Fano threefolds with a multiplicative Chow-Künneth decomposition. Publ. Res. Inst. Math. Sci. \textbf{60} (2024), no. 3, 561-581.

\bibitem[23]{LL97}
Looijenga E., Lunts V.A., A Lie algebra attached to a projective variety. Invent. Math. \textbf{129} (1997), no. 2, 361-412.

\bibitem[24]{MY16}
Moonen, B., Yin, Q., Some remarks on modified diagonals. Commun. Contemp. Math. \textbf{18} (2016), no. 1, 1550009, 16 pp.

\bibitem[25]{Mu90}
Murre, J. P., On the motive of an algebraic surface. J. reine angew. Math. \textbf{409} (1990), 190-204.

\bibitem[26]{Mu93}
Murre, J., On a conjectural filtration on the Chow groups of an algebraic variety I, II. Indag. Math. New Series Vol. \textbf{4} (1993), 177-201.

\bibitem[27]{O'G14}
O'Grady, K., Computations with modified diagonals. Rend. Lincei Mat. Appl. \textbf{25} (2014), 249-274.

\bibitem[28]{Ot12}
Ottem, J. C., Ample subvarieties and $q$-ample divisors. Adv. Math. \textbf{229} (2012), no. 5, 2868-2887.

\bibitem[29]{S01}
Saito, M., Arithmetic mixed sheaves. Invent. Math. \textbf{144} (2001), no. 3, 533-569.

\bibitem[30]{Sch94}
Scholl, A. J., Classical motives. Proc. Sympos. Pure Math. \textbf{55} (1994), 163-187.

\bibitem[31]{SV16a}
Shen, M., Vial, C., The Fourier transform for certain hyper-Kähler fourfolds. Mem. Amer. Math. Soc. \textbf{240} (2016), no. 1139, vii+163 pp.

\bibitem[32]{SV16b}
Shen, M.; Vial, C., The motive of the Hilbert cube $X^{[3]}$. Forum Math. Sigma \textbf{4} (2016), Paper No. e30, 55 pp.

\bibitem[33]{Voi04}
Voisin, C., Intrinsic pseudo-volume forms and K-correspondences. In: \emph{The Fano Conference}, Univ. Torino, Turin, 2004, pp. 761-792.

\bibitem[34]{Voi12}
Voisin, C., Chow rings and decomposition theorems for families of K3 surfaces and Calabi-Yau hypersurfaces. Geom. Topol. \textbf{16} (2012), no. 1, 433-473.

\bibitem[35]{Voi15}
Voisin, C., Some new results on modified diagonals. Geom. Topol. \textbf{19} (2015), no. 6, 3307-3343.

\bibitem[36]{XX13}
Xu, K., Xu, Z., Remarks on Murre's conjecture on Chow groups. J. K-Theory \textbf{12} (2013), no. 1, 3-14.

\bibitem[37]{X25a}
Xu, Z., On motivic multiplicativity of Fano threefolds, in preparation, 2025.

\bibitem[38]{X25b}
Xu, Z., On motivic multiplicativity of cyclic coverings, in preparation, 2025.

\bibitem[39]{Y13}
Yin, Q., Tautological cycles on curves and Jacobians. Thesis, Radboud University Nijmegen (2013). 
http: //www.math.ethz.ch/$\sim$yinqi.
\end{thebibliography}
\vspace{1cm}

{\small

\noindent
{\bf Ze Xu}\\ School of Mathematics, Shandong University, 27 South Shanda Road, Jinan, Shandong 250100, P. R. China\\
{\bf Email: xuze@sdu.edu.cn}

\end{document}
