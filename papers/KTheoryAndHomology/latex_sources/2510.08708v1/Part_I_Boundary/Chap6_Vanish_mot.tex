\subsection{Weight structures}
\label{SS: weight structures}

\subsubsection{Abstract weight structure}
First we collect some definitions and facts about weight structrue on an abstract triangulated category. We mainly follow \cite[\S 1]{Wild_09}. 

\begin{definition}[Weight structure]
\label{def: weight structure}
    Let $\mathcal{C}$ be a triangulated category. A \textit{weight structure} on $\mathcal{C}$ is a pair $w = (\mathcal{C}_{w \le 0}, \mathcal{C}_{w \ge 0})$ of full-subcategories of $\mathcal{C}$ such that if we put 
    \begin{equation*}
        \mathcal{C}_{w \le n} := \mathcal{C}_{w \le 0}[n], \, \, \, \mathcal{C}_{w \ge n}:= \mathcal{C}_{w \ge 0}[n]
    \end{equation*}
    for $n \in \ZZ$, the following conditions are satisfied. 
    \begin{enumerate}
        \item The categories $\mathcal{C}_{w \le 0}$ and $\mathcal{C}_{w \ge 0}$ are Karoubi-closed: for any object $M$ of $\mathcal{C}_{w \le 0}$ (resp., $\mathcal{C}_{w \ge 0}$), any direct sum of $M$ formed in $\mathcal{C}$ is an object of $\mathcal{C}_{w \le 0}$ (resp., $\mathcal{C}_{w \ge 0}$). 
        \item (Semi-invariance with respect to shifts) \  We have the following inclusions: 
        \begin{equation*}
            \mathcal{C}_{w \le 0} \subset \mathcal{C}_{w \le 1},  \, \, \, \mathcal{C}_{w \ge 0} \supset \mathcal{C}_{w \ge 1}
        \end{equation*}
        of full sub-categories of $\mathcal{C}$. 
        \item (Orthogonality) \ For any pair of objects $M \in \mathcal{C}_{w \le 0}$ and $N \in \mathcal{C}_{w \ge 1}$, we have 
        \begin{equation*}
            \Hom_{\mathcal{C}}(M, N) = 0. 
        \end{equation*}
        \item (Weight filtration) \ For any object $M \in \mathcal{C}$, there exists an exact triangle 
        \begin{equation*}
            A \rightarrow M \rightarrow B \rightarrow A[1]
        \end{equation*}
        in $\mathcal{C}$ such that $A \in \mathcal{C}_{w \le 0}$ and $B \in \mathcal{C}_{w \ge 1}$. We shall refer to any such exact triangle as a weight filtration of $M$. 
    \end{enumerate}
\end{definition}

\begin{remark}
    \begin{itemize}
        \item By (2), we have 
        \begin{equation*}
            \mathcal{C}_{w \le n} \subset \mathcal{C}_{w \le 0}
        \end{equation*}
        for any negative integer $n$, and 
        \begin{equation*}
            \mathcal{C}_{w \ge n} \subset \mathcal{C}_{w \ge 0}
        \end{equation*}
        for any postive integer $n$. There are analogues of the other conditins for all the categories $\mathcal{C}_{w \le n}$ and $\mathcal{C}_{w \ge n}$. In particular, they are all Karoubi-closed, and for any object $M \in \mathcal{C}$, there is an exact triangle
        \begin{equation*}
            A \rightarrow M \rightarrow B \rightarrow A[1]
        \end{equation*}
        in $\mathcal{C}$ such that $A \in \mathcal{C}_{w \le n}$ and $B \in \mathcal{C}_{w \ge n + 1}$. 
        \item In (4), ``the" weight filtration is not assumed to be unique. 
        \item Recall that on a triangulated category $\mathcal{C}$, there is a notion of \textit{t-structure} \cite[D\'{e}f. 1.3.1]{BBD}. It consists of a pair $t = (\mathcal{C}^{t \le 0}, \mathcal{C}^{t \ge 0})$ of full sub-categories satisfying formal analogues of conditions (2)-(4), but putting 
        \begin{equation*}
            \mathcal{C}^{t \le n} := \mathcal{C}^{t \le 0}[-n], \, \, \, \mathcal{C}^{t \ge n} := \mathcal{C}^{t \ge 0}[-n],
        \end{equation*}
        for any $n \in \ZZ$. Note that in the context of $t$-structures, the analogues of (4) is unique up to isomorphism, and the analogue of (1) is implied by the other conditions. 
    \end{itemize}
\end{remark}

\begin{definition}[Heart] \cite[Definition 1.2.1]{Bondarko10}
    Let $w = (\mathcal{C}_{w \le 0}, \mathcal{C}_{w \ge 0})$ be a weight structure on $\mathcal{C}$. The \textit{heart} of $w$ is the full additive subcategory $\mathcal{C}_{w = 0}$ of $\mathcal{C}$ whose objects lie both in $\mathcal{C}_{w \le 0}$ and in $\mathcal{C}_{w \ge 0}$. 
\end{definition}

\begin{definition}[Weight exact] \cite[Definition 4.4.4]{Bondarko10}
    Let $F \colon \mathcal{C} \rightarrow \mathcal{D}$ be a functor between two triangulated categories with weight structures $(\mathcal{C}_{w \le 0}, \mathcal{C}_{w \ge 0})$ and $(\mathcal{D}_{w \le 0}, \mathcal{D}_{w \ge 0})$. The functor $F$ is called weight exact if $F$ maps $\mathcal{C}_{w \le 0}$
    to $\mathcal{D}_{w \le 0}$ and maps $\mathcal{C}_{w \ge 0}$ to $\mathcal{D}_{w \ge 0}$. 
\end{definition}

\begin{proposition}
    Let $w = (\mathcal{C}_{w \le 0}, \mathcal{C}_{w \ge 0})$ be a weight structure on $\mathcal{C}$. There is an exact triangle 
    \begin{equation*}
        L \rightarrow M \rightarrow N \rightarrow L[1]
    \end{equation*}
    in $\mathcal{C}$.
    \begin{enumerate}
        \item If both $L$ and $N$ belong to $\mathcal{C}_{w \le 0}$, then so does $M$. 
        \item If both $L$ and $N$ belong to $\mathcal{C}_{w \ge 0}$, then so does $M$. 
    \end{enumerate}
\end{proposition}
\begin{proof}
    See \cite[Proposition 1.3.3.3]{Bondarko10}. 
\end{proof}

From now on, we consider a fixed weight structure $w$ on a triangulated category $\mathcal{C}$. 
\begin{definition}
    Let $M \in \mathcal{C}$ and $m \le n$ be two integers. A \textit{weight filtration of $M$ avoiding weights $m, m + 1, \ldots, n - 1, n$} is an 
    exact triangle 
    \begin{equation*}
        M_{w \le m - 1} \rightarrow M \rightarrow M_{\ge n + 1} \rightarrow M_{w \le m - 1}[1]
    \end{equation*}
    in $\mathcal{C}$,  with $M_{w \le m - 1} \in \mathcal{C}_{w \le m - 1}$ and $M_{\ge n + 1} \in \mathcal{C}_{\ge n + 1}$. 
\end{definition}

\begin{proposition}
    Assume that $m \le n$, and that $M, N \in \mathcal{C}$ admit weight filtrations
    \[
        \begin{tikzcd}
             M_{w \le m - 1} \ar[r, "{x_{-}}"] & M \ar[r, "{x_{+}}"] & M_{\ge n + 1} \ar[r] & {M_{w \le m - 1}[1]}
        \end{tikzcd}
    \]
    and 
    \[
        \begin{tikzcd}
             N_{w \le m - 1} \ar[r, "{y_{-}}"] & N \ar[r, "{y_{+}}"] & N_{\ge n + 1} \ar[r] & {N_{w \le m - 1}[1]}
        \end{tikzcd}
    \]
    avoiding weights $m, \ldots, n$. Then any morphism $M \rightarrow N$ in $\mathcal{C}$ extends uniquely to a morphism of exact triangles: 
    \[
        \begin{tikzcd}
             M_{w \le m - 1} \ar[r, "{x_{-}}"] \ar[d] & M \ar[r, "{x_{+}}"] \ar[d] & M_{\ge n + 1} \ar[r] \ar[d] & {M_{w \le m - 1}[1]} \ar[d] \\
             N_{w \le m - 1} \ar[r, "{y_{-}}"] & N \ar[r, "{y_{+}}"] & N_{\ge n + 1} \ar[r] & {N_{w \le m - 1}[1]}
        \end{tikzcd}.
    \]
\end{proposition}

\begin{corollary}
    Assume that $m \le n$. If $M \in \mathcal{C}$ admits a weight filtration avoiding weights $m, \ldots, n$, then it is unique up to a unique isomorphism. 
\end{corollary}

\begin{definition}
    Assume that $m \le n$. We say that \textit{$M \in \mathcal{C}$ does not have weights $m, \cdots n$} or \textit{$M$ is without weights $m \ldots n$}, if it admits a weight filtration avoiding $m, \ldots, n$. 
\end{definition}

\subsubsection{Motivic weight structure}

We recall H\'{e}bert's construction of motivic weight structure by applying the abstract machine to the triangulated category of constructible Beilinson motives. 

\begin{theorem}[{\cite[Th\'{e}or\`{e}me 3.3]{Hebert10}}]
    Let $S \in \mathrm{Sch}(\QQ)$ be an excellent scheme. 
    Set
    \begin{equation*}
        \mathrm{DM}_{B, c}(S) \supset \mathcal{H}_{S}:= \{ f_{*} 1_{X}(n)[2n] | n \in \ZZ, f \colon X \rightarrow S \, \text{is proper with regular source $X$} \}.
    \end{equation*}
    There exists a unique weight structure $w = (\mathrm{DM}_{B, c}(S)_{w \le 0}, \mathrm{DM}_{B, c}(S)_{w \ge 0})$ on $\mathrm{DM}_{B, c}(S)$ such that 
    \begin{equation*}
        \mathcal{H}_{S} \subset \mathrm{DM}_{B, c}(S)_{w = 0}. 
    \end{equation*}
\end{theorem}

\begin{remark}
    \begin{itemize}
        \item If $S = \Spec(\QQ)$, the weight structure 
        \begin{equation*}
            w = (\mathrm{DM}_{gm}(\QQ)_{w \le 0}, \mathrm{DM}_{gm}(\QQ)_{w \ge 0}) 
        \end{equation*}
        is constructed by Bondarko in \cite[Proposition 6.5.3]{Bondarko10}. 
        \item The existence of this weight structure was also proved independently by Bondarko in \cite{Bondarko17}. 
        \item It can be seen from the definition that $\mathrm{CHM}(S) \subset \mathcal{H}_{S}$. 
    \end{itemize}
\end{remark}

\subsection{Intersection motives, interior motives and boundary motives}
\label{Sec: interior and boundary motive}
In this subsection, we recall the definition of intersection motives, interior motives and boundary motives of Picard modular surfaces and some of their properties. 

\begin{notation}
    We use the same notation as in section \ref{Sec: vanishing of the boudary abs Hodge}:
    for each of the two Shimura varieties $X = \{M, S\}$, we denote by $X^{*}$ its Baily{\textendash}Borel compactification and let $\partial{X} = X^{*} - X$ be the cusps of the compactification. Then we have the following commutative diagram: 
    \begin{equation}
    \label{eq: diagram of compactification}
            \begin{tikzcd}
                M \ar[r, "j^{\prime}"] \ar[d, "{\iota}"] & M^{*} \ar[d, "p"] & \partial{M} \ar[l, "i^{\prime}" above] \ar[d, "q"] \\
                S \ar[r, "j"] & S^{*} & \partial{S} \ar[l, "i" above]  
            \end{tikzcd},
    \end{equation}
    where $j$ and $j^{\prime}$ are open immersions, $i$ and $i^{\prime}$ are closed immersions and $p$ and $q$ are closed immersions induced by the closed immersion $\iota$. We also have 
    \begin{equation*}
        \dim \partial S = 0, \, \dim \partial M = 0. 
    \end{equation*}
\end{notation}

\subsubsection{Boundary motives}

 Let $g \colon \partial S \rightarrow \Spec (\QQ)$ be the structure morphism of $\partial S$.
\begin{definition}[Boundary motives]
    For $V(2) \in \mathrm{Rep}_{\QQ}(G)$, the boundary motive $\partial M_{gm}(V(2))$ is defined as 
    \begin{equation*}
        \partial M_{gm}(V(2)) := (g_{*}i^{*}j_{*}V(2))^{*} \in \mathrm{DM}_{gm}(\QQ), 
    \end{equation*}
    where the notation $(\cdot)^{*}$ denotes taking dual. 
\end{definition}

\begin{remark}
    It can be seen from the proof of \cite[Corollary 3.10]{Wild_ChowII_20} that the definition of the boundary motive is the same as the one given in \cite[Definition 2.1]{Wild_boundary05}. 
\end{remark}

\begin{notation}
    Denote by $\mathrm{DM}_{B, c}(S)_{w = 0, \partial w \neq 0, 1}$ the full sub-category of $\mathrm{DM}_{B, c}(S)_{w = 0}$ of objects $N$ such that $i^{*}j_{*}N$ is without weights $0$ and $1$. 
\end{notation}

\begin{proposition}
\label{Prop: boundary weight}
For $V = V^{a, b}\{r, s\}$ and $a > 0$ and $b > 0$, we have 
\begin{align*}
    V(2) \in \mathrm{DM}_{B, c}(S)_{w = 0, \partial w \neq 0, 1}. 
\end{align*}
\end{proposition}

\begin{proof}
    First, the $k = \min(k_1 - k_2, k_2)$ in \cite[Page 21]{Wild_09} is $\min(a, b)$ in our convention, and the condition $k_1 > k_2 > 0$ in \cite{Wild_09} is equivalent to our condition $a > 0$ and $b > 0$. Second, it follows from \cite[Theorem 3.8]{Wild_15} that the boundary motive $\partial M_{gm}(V(2))$ is without weights $-1$ and $0$. Finally, it can be seen from the fact that $g_{*}i^{*}j_{*}V(2) = g_{!}i^{*}j_{*}V(2) = \partial M_{gm}(V(2))^{*}$ and the fact that $g_{!}$ is weight exact \cite[Th\'{e}or\`{e}me 3.3(i)]{Hebert10} that $i^{*}j_{*}V(2)$ is without weights $0$ and $1$. 
\end{proof}

\subsubsection{Intersection motives and interior motives}
Let $a \colon S^{*} \rightarrow \Spec(\QQ)$ be the structure morphism of $S^{*}$.
\par Recall that in \cite[Definition 2.4]{Wild_ChowII_20}, the author defines the motivic intermediate extension functor 
\begin{equation*}
    j_{!*} \colon \mathrm{DM}_{B, c}(S)_{w = 0, \partial w \neq 0, 1} \rightarrow \mathrm{DM}_{B, c}(S^{*})_{w = 0}. 
\end{equation*}

\begin{definition}
\label{def: intersection motive and interior motive}
    \begin{enumerate}
        \item (\textnormal{Intersection motives}) For $V = V^{a, b}\{r, s\}$ and $a > 0$ and $b > 0$, the \textit{intersection motive} of $S$ relative to $S^{*}$ is defined as 
        \begin{equation*}
            a_{*}j_{!*}V(2) \in \mathrm{DM}_{B, c}(\Spec(\QQ))_{w = 0}. 
        \end{equation*}
        \item (\textnormal{Interior motives})  For $V = V^{a, b}\{r, s\}$ and $a > 0$ and $b > 0$, the \textit{interior motive} $\mathrm{Gr_{0}}M_{gm}(V(2))$ of $S$ is defined as 
        \begin{equation*}
            \mathrm{Gr_{0}}M_{gm}(V(2)) := (a_{*}j_{!*}V(2))^{*} \in \mathrm{DM}_{gm}(\QQ)_{w = 0},
        \end{equation*}
        where the notation $(\cdot)^{*}$ denotes taking dual. 
    \end{enumerate}
    
\end{definition}

\begin{remark}
    \begin{itemize}
        \item 
        It follows from \cite[Corollary 3.10(b)]{Wild_ChowII_20} that the definition of interior motives is the same as the one defined in \cite[Corollary 3.9]{Wild_15}
        \item By \cite[Remark 3.13(a)]{Wild_ChowII_20}, the interior cohomology $\mathrm{H}^{2}_{B, !}(S, V(2))$ is the Hodge realization of the interior motive $\mathrm{Gr_{0}}M_{gm}(V(2))$. 
    \end{itemize}
\end{remark}


Now we recall a few properties of intersection motives. 
\begin{proposition}
Let $k = \mathrm{min}(a, b)$, which is greater than $0$. 
\begin{enumerate}
    \item There are canonical exact triangles in $\mathrm{DM}_{B, c}(\Spec(\QQ))$
         \begin{equation}
          \label{eq: exact triangle: intersection motive and j_{*}} 
            \begin{tikzcd}
                 a_{*}j_{!*}V(2) \ar[r, "\iota_{1}"] & a_{*}j_{*}V(2) \ar[r] & C_{\ge (k + 1)} \ar[r] & a_{*}j_{!*}V(2)[1]
            \end{tikzcd}
         \end{equation}
         and 
         \begin{equation}
         \label{eq: exact triangle: intersection motive and j_{!}}
             \begin{tikzcd}
                 a_{*}j_{!}V(2) \ar[r, "\iota_{2}"] & a_{*}j_{!*}V(2) \ar[r] & C_{\le {-k}} \ar[r] & a_{*}j_{!}V(2)[1]
             \end{tikzcd}, 
         \end{equation}
         where $C_{\ge (k + 1)} \in \mathrm{DM}_{B, c}(\Spec(\QQ))_{w \ge k + 1}$ and $C_{\le {-k}} \in \mathrm{DM}_{B, c}(\Spec(\QQ))_{w \le {-k}}$. 
    \item We have the following commutative diagram in $\mathrm{DM}_{B, c}(\Spec(\QQ))$: 
          \begin{equation}
           \label{eq: commutative diagram factor through j_{!} to j_{*}}
             \begin{tikzcd}
                 a_{*}j_{!}V(2) \ar[rr, "m"] \ar[rd, "\iota_{2}"]& & a_{*}j_{*}V(2) \\
                 &  a_{*}j_{!*}V(2) \ar[ru, "\iota_{1}"] & 
             \end{tikzcd}, 
          \end{equation}
          where $m$ is the map induced by the first arrow in the exact triangle of Proposition \ref{Prop: motivic localization}, and $\iota_{1}$ and $\iota_{2}$ are the corresponding maps in (1). 
\end{enumerate}
    
\end{proposition}

\begin{proof}
    The two exact triangles follow from \cite[Corollary 3.9]{Wild_15} and Definition \ref{def: intersection motive and interior motive}, and the commutative diagram of (2) follows from \cite[Theorem 2.4(c)]{Wild_09} and Definition \ref{def: intersection motive and interior motive}. 
\end{proof}

\subsection{Vanishing on the boundary for motivic cohomology}
\label{SS: the proof of the vanishing}
In this subsection, we state the theorem of vanishing on the boundary for motivic cohomology. 

\begin{theorem}
\label{Thm: mot vanish on the boudary}
Under the conditions
\begin{enumerate}
     \item $0 \le -r \le a$ and  $0 \le -s \le b$, 
     \item $a > 0$ and $b > 0$,
     \item $r \neq 0$ or $s \neq 0$, 
\end{enumerate}
the map $\mathcal{E}is_{M}^n \colon \mathcal{B}_{n} \rightarrow \mathrm{H}^{3}_{M}(S, V(2))$ factors through the inclusion 
\[
    \mathrm{H}_{M}^{3}(\mathrm{Gr_{0}}M_{gm}(V(2)), \QQ(0)) \hookrightarrow \mathrm{H}^{3}_{M}(S, V(2)). 
\] 
\end{theorem}

\begin{remark}
    Using the Beilinson regulator $r_{H}$, $\mathrm{H}_{M}^{3}(\mathrm{Gr_{0}}M_{gm}(V(2)), \QQ(0))$ has image in 
    \begin{equation*}
        \Ext^{1}_{\mathrm{MHS}_{\RR}^{+}}(\mathbf{1}, \mathrm{H}^{2}_{B,!}(S, V(2))).
    \end{equation*}
    Hence, Theorem \ref{Thm: mot vanish on the boudary} can be viewed as a ``motivic lifting'' of Theorem \ref{Thm: Hdg vanish on the boudary}. 
\end{remark}
 
\subsection{The proof of motivic vanishing}
\label{Sec: motivic vanishing}
In this subsection, we prove the theorem of vanishing on the boundary for motivic cohomology. 

\subsubsection{Conversativity and weight convervativity} Before going into the steps of the proof, we first recall the conservativity and weight conservativity theorem of the Hodge realization functor over motives of Abelian type. 


\begin{theorem}[Conservativity, {\cite[Theorem 1.12.]{Wild_15}}]
\label{Theorem: conservativity}
    For $M \in \mathrm{DM}_{B, c}^{Ab}(\Spec(\QQ))$, if 
    \begin{equation*}
        H^{n} R_{H}(M) = 0
    \end{equation*}
    for all $n \in \ZZ$, then $M = 0$. 
\end{theorem}

A direct corollary of the conservativity theorem is the following. 
\begin{corollary}
\label{corllaray: conservativity}
For all $M, N \in \mathrm{DM}_{B, c}^{Ab}(\Spec(\QQ))$, if 
\begin{equation*}
    \Hom_{\mathrm{D}^{b}(\mathrm{MHS}_{\RR}^{+})}(R_{H}(M), R_{H}(N)) = 0, 
\end{equation*}
then 
\begin{equation*}
    \Hom_{\mathrm{DM}_{B, c}(\Spec(\QQ))}(M, N) = 0. 
\end{equation*}
\end{corollary}

\begin{theorem}[Weight conservativity, {\cite[Theorem 1.13]{Wild_15}}]
\label{Theorem: weight conservativity}
    Let $M \in \mathrm{DM}_{B, c}^{Ab}(\Spec(\QQ))$ and $\alpha \le \beta$ be two integers. We have the following statements: 
    \begin{enumerate}
        \item $M$ lies in $\mathrm{DM}_{B, c}^{Ab}(\Spec(\QQ))_{w = 0}$ if and only if $H^{n}R_{H}(M)$ is pure of weight $n$, for all $n \in \ZZ$,
        \item $M$ lies in $\mathrm{DM}_{B, c}^{Ab}(\Spec(\QQ))_{w \le \alpha}$ if and only if $H^{n}R_{H}(M)$ is of weights $\le n + \alpha$, for all $n \in \ZZ$, 
        \item $M$ lies in $\mathrm{DM}_{B,c }^{Ab}(\Spec(\QQ))_{w \ge \beta}$ if and only if $H^{n}R_{H}(M)$ is of weights $\ge n + \beta$, for all $n \in \ZZ$,
        \item $M$ is without weights $\alpha, \alpha + 1, \ldots, \beta$ if and only if $H^{n}R_{H}(M)$ is without weights $n + \alpha, n + (\alpha + 1), \ldots, n + \beta$, for all $n \in \ZZ$. 
    \end{enumerate}
\end{theorem}


\subsubsection{The proof}
The proof is parallel to the Hodge case, so it is also divided into three steps. 
Recall that we let $a \colon S^{*} \rightarrow \Spec(\QQ)$ be the structure morphism of $S^{*}$. \\

\noindent \textbf{Step I.} In the first step, the goal is to prove the exactness of the sequence in Proposition \ref{Prop: mot exact sequence}.

\begin{lemma}
\label{Lemma: mot vanshing}
We have the following vanishing of motivic cohomology: 
\begin{equation*}
   \Hom_{\mathrm{DM}_{B, c}(\Spec(\QQ))}(1_{\Spec(\QQ)}, C_{\ge(k + 1)}[2]) = 0, 
\end{equation*}
where $C_{\ge (k + 1)} \in \mathrm{DM}_{B, c}(\Spec(\QQ))_{w \ge k + 1}$ is the one in the exact triangle (\ref{eq: exact triangle: intersection motive and j_{*}}). 
    
\end{lemma}

\begin{proof}
    We first apply the Hodge realization functor $R_{H}$ to the above $\Hom$ space and get 
    \begin{align*}
        \Hom_{\mathrm{D^{b}}(\mathrm{MHS}_{\RR}^{+})}(\RR(0), N[2]), 
    \end{align*} 
    where $N = R_{H}(C_{\ge(k + 1)})$. 
    By the fact that $\mathrm{MHS}_{\RR}^{+}$ has cohomological dimension $1$,  we have the short exact sequence 
    \begin{equation*}
        \begin{tikzcd}[column sep = 1em]
            0 \ar[r] &  \Ext_{\mathrm{MHS}_{\RR}^{+}}^{1}(\RR(0), \mathcal{H}^{1}(N)) \ar[r] & \Hom_{\mathrm{D^{b}}(\mathrm{MHS}_{\RR}^{+})}(\RR(0), N[2]) \ar[r] & \Hom_{\mathrm{MHS}_{\RR}^{+}}(\RR(0), \mathcal{H}^{2}(N)) \ar[r] & 0. 
        \end{tikzcd}
    \end{equation*}
    It follows from Theorem \ref{Theorem: weight conservativity} that $\mathcal{H}^{1}(N)$ is of weights $\ge k + 2$ and $\mathcal{H}^{2}(N)$ is of weights $\ge k + 3$. Hence, we have $\Hom_{\mathrm{MHS}_{\RR}^{+}}(\RR(0), \mathcal{H}^{2}(N)) = 0$ for weight reasons and $\Ext_{\mathrm{MHS}_{\RR}^{+}}^{1}(\RR(0), \mathcal{H}^{1}(N)) = 0$ by \cite[Theorem A.2.10]{HW98}. Therefore, the following identity holds: 
    \begin{equation*}
        \Hom_{\mathrm{D}^{b}(\mathrm{MHS}_{\RR}^{+})}(\RR(0), N[2]) = 0.
    \end{equation*}
    Finally, the vanishing 
    \begin{equation*}
         \Hom_{\mathrm{DM}_{B, c}(\Spec(\QQ))}(1_{\Spec(\QQ)}, C_{\ge(k + 1)}[2]) = 0
    \end{equation*}
    follows from Corollary \ref{corllaray: conservativity}. 
\end{proof}

\begin{proposition}
We have the following exact sequence 
\[  
    \begin{tikzcd}
        0 \ar[r] & \mathrm{H}_{M}^{3}(S^{*}, j_{!*}V(2)) \ar[r] & \mathrm{H}^{3}_{M}(S,V(2)) \ar[r, "{\partial_{S}^{M}}"] & \mathrm{H}^{3}_{M}(\partial{S}, i^{*}j_{*}V(2)). 
    \end{tikzcd}
\]
\end{proposition}

\begin{proof}
First, if we let $N = j_{*}V(2)$ in the exact triangle of Proposition \ref{Prop: motivic localization} and apply the functor $a_{*}$ to it, then we get the following exact triangle in $\mathrm{DM}_{B, c}(\Spec(\QQ))$: 
\begin{equation*}
    a_{*}j_{!}V(2) \rightarrow a_{*}j_{*}V(2) \rightarrow a_{*}i_{*}i^{*}j_{*}V(2) \rightarrow a_{*}j_{!}V(2)[1].
\end{equation*}
We then apply the functor $\Hom_{\mathrm{DM}_{B, c}(\Spec(\QQ))}(1_{\Spec(\QQ)}, -[3])$ to it and get the long exact sequence
\begin{equation*}
    \cdots \rightarrow \mathrm{H}^{3}_{M}(S^{*},j_{!}V(2))  \rightarrow \mathrm{H}^{3}_{M}(S,V(2)) \rightarrow \mathrm{H}^{3}_{M}(\partial{S}, i^{*}j_{*}V(2)) \rightarrow \cdots. 
\end{equation*}
\par Second, if we apply the functor $\Hom_{\mathrm{DM}_{B, c}(\Spec(\QQ))}(1_{\Spec(\QQ)}, -[3])$ to the commutative diagram (\ref{eq: commutative diagram factor through j_{!} to j_{*}}), then we have the following commutative diagram: 
          \begin{equation}
             \begin{tikzcd}
                 \mathrm{H}^{3}_{M}(S^{*},j_{!}V(2)) \ar[rr, "m_{*}"] \ar[rd, "\iota_{2, *}"]& & \mathrm{H}^{3}_{M}(S,V(2)) \\
                 & \mathrm{H}_{M}^{3}(S^{*}, j_{!*}V(2))  \ar[ru, "\iota_{1, *}"] & 
             \end{tikzcd}, 
          \end{equation}
where the map $\iota_{1, *}$ is induced from the map $\iota_{1}$ in the commutative diagram (\ref{eq: commutative diagram factor through j_{!} to j_{*}}). 
\par Finally, after applying the functor 
\begin{equation*}
    \Hom_{\mathrm{DM}_{B, c}(\Spec(\QQ))}(1_{\Spec(\QQ)}, -[3])
\end{equation*}
to the exact triangle (\ref{eq: exact triangle: intersection motive and j_{!}}) we get the exact sequence
\begin{equation*}
    \begin{tikzcd}
       \Hom_{\mathrm{DM}_{B, c}(\Spec(\QQ))}(1_{\Spec(\QQ)}, C_{\ge(k + 1)}[2]) \ar[r] & \mathrm{H}_{M}^{3}(S^{*}, j_{!*}V(2)) \ar[r, "\iota_{1, *}"] & \mathrm{H}^{3}_{M}(S,V(2)). 
    \end{tikzcd}
\end{equation*}
It follows from Lemma \ref{Lemma: mot vanshing} that $\iota_{1, *}$ is injective. Therefore, we get the following exact sequence: 
\begin{equation*}
     0 \rightarrow  \mathrm{H}_{M}^{3}(S^{*}, j_{!*}V(2)) \rightarrow \mathrm{H}^{3}_{M}(S,V(2)) \rightarrow \mathrm{H}^{3}_{M}(\partial{S}, i^{*}j_{*}V(2)).
\end{equation*}
\end{proof}

\begin{definition}
\label{def: motivic cohomology of interior motives}
    For the interior motive $\mathrm{Gr_{0}}M_{gm}(V(2)) \in \mathrm{DM}_{gm}(\QQ)$, we define its motivic cohomology $\mathrm{H}_{M}^{3}(\mathrm{Gr_{0}}M_{gm}(V(2)), \QQ(0))$ as 
    \begin{equation*}
        \mathrm{H}_{M}^{3}(\mathrm{Gr_{0}}M_{gm}(V(2)), \QQ(0)) := \Hom_{\mathrm{DM}_{gm}(\QQ)}(\mathrm{Gr_{0}}M_{gm}(V(2)), \QQ(0)), 
    \end{equation*}
    which is isomorphic to 
    \begin{equation*}
        \mathrm{H}_{M}^{3}(S^{*}, j_{!*}V(2)) = \Hom_{\mathrm{DM}_{B, c}(\Spec(\QQ))}(1_{\Spec(\QQ)}, a_{*}j_{!*}V(2)[3]). 
    \end{equation*}
\end{definition}

Hence, we get the following proposition. 
\begin{proposition}
\label{Prop: mot exact sequence}
\label{}
    We have the exact sequence 
    \[  
        \begin{tikzcd}
            0 \ar[r] &   \mathrm{H}_{M}^{3}(\mathrm{Gr_{0}}M_{gm}(V(2)), \QQ(0)) \ar[r] & \mathrm{H}^{3}_{M}(S,V(2)) \ar[r, "{\partial_{S}^{M}}"] & \mathrm{H}^{3}_{M}(\partial{S}, i^{*}j_{*}V(2)). 
        \end{tikzcd}
    \]
\end{proposition}

\begin{remark}
    The exact sequence in Proposition \ref{Prop: mot exact sequence} is the ``motivic lifting" of the exact sequence in Proposition \ref{Prop: Hodge exact sequence}. 
\end{remark}

\vspace{10pt}

\noindent \textbf{Step II.} In the second step, the goal is to prove the diagram in Proposition \ref{Prop: mot commutative diagram} commutes. 

\begin{proposition}
\label{Prop: mot commutative diagram}
We have the following commutative diagram:
\[
    \begin{tikzcd}
         \mathcal{B}_{n} \ar[d, "p^{*}\circ Eis_{M}^{n}"] & \\
        \mathrm{H}^{1}_{M}(M,W(1)) \ar[r, "\partial^{M}_{M}"] \ar[d, "{\iota_{*}^{M}}"] & \mathrm{H}^{1}_{M}(\partial{M}, i^{\prime, *}j^{\prime}_{*}W(1))) \ar[d, "{\theta^{M}}"] \\
        \mathrm{H}^{3}_{M}(S, V(2)) \ar[r, "\partial_{S}^{M}"] & \mathrm{H}^{3}_{M}(\partial{S}, i^{*}j_{*}V(2))
    \end{tikzcd}, 
\]
where $\iota_{*}^{M}$ is the Gysin morphism defined in Proposition \ref{Prop: Gysin for motivic cohomology}, $\partial_{M}^{M}$ and $\partial_{S}^{M}$ are the corresponding boundary maps and the map $\theta^{M}$ is induced from the map, 
\begin{equation}
\label{eq: Hodge theta in derived category.}
    (\partial \iota_{*}^{M} \colon q_{*}i^{\prime*}j^{\prime}_{*}W(1)[1] \rightarrow i^{*}j_{*}V(2)[3]) \in \mathrm{DM}_{B,c}(\partial S),
\end{equation}
which is degenerated from $\iota^{M}_{*}$.
\end{proposition}

\begin{proof}
The proof is parallel to the proof of Proposition \ref{Prop: Hodge commutative diagram}. First, recall that we have the following diagram:
\[
    \xymatrix{
        M \ar[r]^-{j^{'}} \ar[d]_{\iota} & M^{*} \ar[d]_-{p} & \partial{M} \ar[l]_-{i^{'}} \ar[d]_-{q} \\
        S \ar[r]^-{j} & S^{*} & \partial{S} \ar[l]_-{i}\\
    }. 
\]
It follows from Proposition \ref{Prop: Gysin for motivic cohomology} that the Gysin morphism $\iota_{*}: \mathrm{H}^{1}_{M}(M,W(1)) \rightarrow \mathrm{H}^{3}_{M}(S, V(2))$ is induced by 
\begin{equation*}
    (\iota_{*}W(1)[1] \rightarrow V(2)[3]) \in \mathrm{DM}_{B,c}(S). 
\end{equation*}
\par Second,  after applying the functor $j_{*}$ to the morphism and by the fact that $j_{*}\iota_{*} = p_{*}j^{\prime}_{*}$,
we have the morphism 
\begin{equation}
\label{Map: in motivic commutative1}
    (p_{*}j^{\prime}_{*}W(1)[1] \rightarrow j_{*}V(2)[3]) \in \mathrm{DM}_{B,c}(S^{*}). 
\end{equation}
The morphism 
\begin{equation*}
    (i_{*}i^{*}p_{*}j^{\prime}_{*}W(1)[1] \rightarrow i_{*}i^{*}j_{*}V(2)[3]) \in \mathrm{DM}_{B,c}(S^{*})
\end{equation*}
follows by applying the functor $i_{*}i^{*}$ to the morphism in (\ref{Map: in motivic commutative1}). It follows from the proper base change theorem for Beilinson motives \cite[Theorem 2.2.14.(4)]{Cisinski_Deglise} and the fact that $p$ is proper that $i^{*}p_{*} = q_{*}i^{\prime*}$. From it,  we get 
\begin{equation*}
    (i_{*}q_{*}i^{\prime*}j^{\prime}_{*}W(1)[1] \rightarrow i_{*}i^{*}j_{*}V(2)[3]) \in \mathrm{DM}_{B,c}(S^{*}). 
\end{equation*}
The natural transformation $\mathrm{id} \rightarrow i_{*}i^{*}$ from adjunction gives us the following commutative diagram in $\mathrm{DM}_{B,c}(S^{*})$: 
\[
    \begin{tikzcd}
        p_{*}j^{\prime}_{*}W(1)[1] \ar[r] \ar[d] & i_{*}q_{*}i^{\prime*}j^{\prime}_{*}W(1)[1] \ar[d] \\
        j_{*}V(2)[3] \ar[r] & i_{*}i^{*}j_{*}V(2)[3]. 
    \end{tikzcd}
\]
\par Finally, we apply the functor $\Hom_{\mathrm{DM}_{B,c}(S^{*})}(1_{S^{*}}, - )$ to the above  commutative diagram and get the commutative diagram: 
\[
    \begin{tikzcd}
        \mathrm{H}^{1}_{M}(M, W(1)) \ar[r] \ar[d] & \mathrm{H}^{1}_{M}(\partial M, i^{\prime*}j^{\prime}_{*}W(1))  \ar[d] \\
        \mathrm{H}^{3}_{M}(S, V(2)) \ar[r] & \mathrm{H}^{3}_{M}(\partial S, i^{*}j_{*}V(2)). 
    \end{tikzcd}
\]
\end{proof}

\begin{remark}
    \begin{itemize}
        \item  From the construction, we can see the boundary map $\partial_{S}^{M}$ in Proposition \ref{Prop: mot commutative diagram} is the same as the $\partial_{S}^{H}$ in Proposition \ref{Prop: mot exact sequence}. Hence, we use the same notation for them. 
        \item  The commutative diagram in Proposition \ref{Prop: mot commutative diagram} is the ``motivic lifting" of the commutative diagram in Proposition \ref{Prop: Hodge commutative diagram}. 
    \end{itemize}
\end{remark}

\vspace{10pt}

\noindent \textbf{Step III.} In the third step, the goal is to prove the map $\theta^{M}$ in Proposition \ref{Prop: mot commutative diagram} is zero, which is stated in Proposition \ref{Prop: mot theta is zero}. \\

\begin{proposition}
\label{Prop: mot theta is zero}
    The map $\theta^{M}$ in Proposition \ref{Prop: mot commutative diagram} is zero. 
\end{proposition}
\begin{proof}
    First, it follows from Proposition \ref{Prop: mot commutative diagram} that in order to prove $\theta^{M} = 0$, it suffices to prove the following statement: 
    for any 
    \begin{equation*}
        f \in \mathrm{H}^{1}_{M}(\partial M, i^{\prime*}j^{\prime }_{*}W(1)) = \Hom_{\mathrm{DM}_{B,c}(S^{*})}(1_{S^{*}}, i_{*}q_{*}i^{\prime*}j^{\prime}_{*}W(1)[1]), 
    \end{equation*}
    the element 
    \begin{equation*}
        i_{*} (\partial \iota_{*}^{M}) \circ f \in \Hom_{\mathrm{DM}_{B,c}(S^{*})}(1_{S^{*}}, i_{*}i^{*}j_{*}V(2)[3]) = \mathrm{H}^{3}_{M}(\partial S, i^{*}j_{*}V(2))
    \end{equation*}
    is $0$. 
    \par Second, it follows from the identity $R_{H}(i_{*} (\partial \iota_{*}^{M}) \circ f) = i_{*} (\partial \iota_{*}^{H}) \circ R_{H}(f)$ in
    \begin{equation*}
        \Hom_{\mathrm{D}^{b}(\mathrm{MHM}_{\RR}(S^{*}/\RR))}(\RR(0)_{S^{*}}, i_{*}i^{*}j_{*}V(2)[3]) = \mathrm{H}^{1}_{H}(\partial S, i^{*}j_{*}V(2))
    \end{equation*}
    where 
    \begin{equation*}
        R_{H}(f) \in \mathrm{H}^{0}_{H}(\partial M, i^{\prime*}j^{\prime }_{*}W(1)) = \Hom_{\mathrm{D}^{\mathrm{b}}(\mathrm{MHM}_{\RR}(S^{*}/\RR))}(\RR(0)_{S^{*}}, i_{*}q_{*}i^{\prime*}j^{\prime}_{*}W(1)[2])
    \end{equation*}
    and Corollary \ref{Corollary: Hodge theta is zero} that $R_{H}(i_{*} (\partial \iota_{*}^{M}) \circ f) = 0$. 
    \par Finally, by Theorem \ref{Theorem: conservativity}, we have $i_{*} (\partial \iota_{*}^{M}) \circ f = 0$. 
\end{proof}

Finally, by combining Proposition \ref{Prop: mot exact sequence}, Proposition \ref{Prop: mot commutative diagram} and Proposition \ref{Prop: mot theta is zero}, we finish the proof of Theorem \ref{Thm: mot vanish on the boudary}.