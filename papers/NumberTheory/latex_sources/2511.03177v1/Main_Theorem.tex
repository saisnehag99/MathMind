\ifdefined\isMainDocument
\else
    \documentclass[10pt,oneside,reqno]{amsart}
    \usepackage{xr}
    \externaldocument{../main} % Main の .aux を参照
    \usepackage{latexsym}
\usepackage{amscd}
\usepackage{amsmath}
\usepackage{amssymb}
\usepackage[mathscr]{euscript}
\usepackage{graphicx}
\usepackage{geometry}
\usepackage{mathtools}
\usepackage[all]{xy}
\usepackage[dvipsnames]{xcolor}
\usepackage[colorlinks,final,hyperindex]{hyperref}
\usepackage{tikz}
\usepackage{tikz-cd}
\usetikzlibrary{decorations.pathmorphing,decorations.markings,arrows,calc,shapes.geometric,arrows.meta,positioning}
\usepackage{enumerate}
\usepackage{multirow}

\usepackage[bb=dsserif]{mathalpha}
\usepackage{bm}

\newcommand{\QQ}{\ensuremath{\mathbb{Q}}}
\newcommand{\DD}{\ensuremath{\mathbb{D}}}
\newcommand{\CC}{\ensuremath{\mathbb{C}}}
\newcommand{\RR}{\ensuremath{\mathbb{R}}}
\newcommand{\ZZ}{\ensuremath{\mathbb{Z}}}
\newcommand{\NN}{\ensuremath{\mathbb{N}}}

\newcommand{\g}{\ensuremath{\mathfrak{g}}}
\renewcommand{\SS}{\ensuremath{\mathbb{S}}}

\newcommand{\id}{\ensuremath{\mathrm{id}}}

\makeatletter

\newcommand{\bbone}{\mathbb{1}}

\newcommand{\calA}{\mathcal{A}}
\newcommand{\calB}{\mathcal{B}}
\newcommand{\calC}{\mathcal{C}}
\newcommand{\calD}{\mathcal{D}}
\newcommand{\calE}{\mathcal{E}}
\newcommand{\calF}{\mathcal{F}}
\newcommand{\calG}{\mathcal{G}}
\newcommand{\calH}{\mathcal{H}}
\newcommand{\calI}{\mathcal{I}}
\newcommand{\calJ}{\mathcal{J}}
\newcommand{\calK}{\mathcal{K}}
\newcommand{\calL}{\mathcal{L}}
\newcommand{\calM}{\mathcal{M}}
\newcommand{\calN}{\mathcal{N}}
\newcommand{\calO}{\mathcal{O}}
\newcommand{\calP}{\mathcal{P}}
\newcommand{\calQ}{\mathcal{Q}}
\newcommand{\calR}{\mathcal{R}}
\newcommand{\calS}{\mathcal{S}}
\newcommand{\calT}{\mathcal{T}}
\newcommand{\calU}{\mathcal{U}}
\newcommand{\calV}{\mathcal{V}}
\newcommand{\calW}{\mathcal{W}}
\newcommand{\calX}{\mathcal{X}}
\newcommand{\calY}{\mathcal{Y}}
\newcommand{\calZ}{\mathcal{Z}}
\newcommand{\kk}{\Bbbk}

\newcommand{\scrY}{\mathscr{Y}}
\newcommand{\scrV}{\mathscr{V}}
\newcommand{\scrP}{\mathscr{P}}

\newcommand{\Rep}{\mathop{\mathrm{Rep}}\nolimits}
\newcommand{\alg}{\mathop{\mathrm{alg}}\nolimits}
\newcommand{\Span}{\mathop{\mathrm{Span}}\nolimits}
\newcommand{\proj}{\mathop{\mathrm{proj}}\nolimits}
\newcommand{\Hom}{\mathop{\mathrm{Hom}}\nolimits}
\newcommand{\PHom}{\mathop{\mathrm{PHom}}\nolimits}
\newcommand{\Tw}{\mathop{\mathrm{Tw}}\nolimits}
\newcommand{\Dim}{\mathop{\mathrm{Dim}}\nolimits}
\newcommand{\Imm}{\mathop{\mathrm{Im}}\nolimits}
\newcommand{\Aus}{\mathop{\mathrm{Aus}}\nolimits}
\newcommand{\Ann}{\mathop{\mathrm{Ann}}\nolimits}
\newcommand{\APC}{\mathop{\mathrm{APC}}\nolimits}
\newcommand{\Barr}{\mathop{\mathrm{Bar}}\nolimits}
\newcommand{\Cone}{\mathop{\mathrm{Cone}}\nolimits}
\newcommand{\modu}{\mathop{\mathrm{mod}}\nolimits}
\newcommand{\dgMod}{\mathop{\mathrm{dgMod}}\nolimits}
\newcommand{\soc}{\mathop{\mathrm{soc}}\nolimits}
\newcommand{\dg}{\mathop{\mathrm{dg}}\nolimits}
\newcommand{\red}{\mathop{\mathrm{red}}\nolimits}
\newcommand{\Map}{\mathop{\mathrm{Map}}\nolimits}
\newcommand{\inj}{\mathop{\mathrm{inj}}\nolimits}
\newcommand{\op}{\mathop{\mathrm{op}}\nolimits}
\newcommand{\Ker}{\mathop{\mathrm{Ker}}\nolimits}
\newcommand{\Tor}{\mathop{\mathrm{Tor}}\nolimits}
\newcommand{\Tot}{\mathop{\mathrm{Tot}}\nolimits}
\newcommand{\Ext}{\mathop{\mathrm{Ext}}\nolimits}
\newcommand{\sg}{\mathop{\mathrm{sg}}\nolimits}
\newcommand{\Sl}{\mathop{\mathfrak{sl}}\nolimits}
\newcommand{\nc}{\mathop{\mathrm{nc}}\nolimits}
\newcommand{\Tr}{\mathop{\mathrm{Tr}}\nolimits}
\newcommand{\Irr}{\mathop{\mathrm{Irr}}\nolimits}
\newcommand{\HC}{\mathop{\mathrm{HC}}\nolimits}
\newcommand{\HH}{\mathop{\mathrm{HH}}\nolimits}
\newcommand{\THH}{\mathop{\mathrm{TH}}\nolimits}
\newcommand{\Perf}{\mathop{\mathrm{Perf}}\nolimits}
\newcommand{\del}{\partial}
\newcommand{\ev}{\mathop{\mathrm{ev}}\nolimits}
\newcommand{\co}{\mathop{\mathrm{co}}\nolimits}
\newcommand{\pt}{\mathop{\mathrm{pt}}\nolimits}
\newcommand{\wt}{\mathop{\mathrm{wt}}\nolimits}
\newcommand{\pCY}{\mathop{\mathrm{pCY}}\nolimits}
\newcommand{\gr}{\mathop{\mathrm{gr}}\nolimits}
\newcommand{\coker}{\mathop{\mathrm{coker}}\nolimits}


\newcommand{\Top}{\mathop{\mathrm{Top}}\nolimits}
\newcommand{\Set}{\mathop{\mathrm{Set}}\nolimits}
\newcommand{\Vect}{\mathop{\mathrm{Vect}}\nolimits}
\newcommand{\Mod}{\mathop{\mathrm{Mod}}\nolimits}
\newcommand{\Ob}{\mathop{\mathrm{Ob}}\nolimits}
\newcommand{\Ind}{\mathop{\mathrm{Ind}}\nolimits}
\newcommand{\Sym}{\mathop{\mathrm{Sym}}\nolimits}
\newcommand{\Alt}{\mathop{\mathrm{Alt}}\nolimits}
\newcommand{\End}{\mathop{\mathrm{End}}\nolimits}
\newcommand{\Ch}{\mathop{\mathrm{Ch}}\nolimits}


\newcommand{\Ass}{\mathop{\mathrm{Ass}}\nolimits}
\newcommand{\Com}{\mathop{\mathrm{Com}}\nolimits}
\newcommand{\Lie}{\mathop{\mathrm{Lie}}\nolimits}
\newcommand{\colim}{\mathop{\mathrm{colim}}\nolimits}

\newcommand{\Op}{\mathop{\mathsf{Operads}}\nolimits}
\newcommand{\Diop}{\mathop{\mathsf{Dioperads}}\nolimits}
\newcommand{\PDiop}{\mathop{\mathsf{PDioperads}}\nolimits}
\newcommand{\Coop}{\mathop{\mathsf{Cooperads}}\nolimits}
\newcommand{\Codiop}{\mathop{\mathsf{Codioperads}}\nolimits}
\newcommand{\PCodiop}{\mathop{\mathsf{PCodioperads}}\nolimits}

\newcommand{\antish}{\ensuremath{\mbox{!`}}}


\tikzset{->-/.style={decoration={
			markings,
			mark=at position #1 with {\arrow{>}}},postaction={decorate}}}
\tikzset{-w-/.style={decoration={
			markings,
			mark=at position #1 with {\arrow{Stealth[fill=white,scale=1.4]}}},postaction={decorate}}}
\tikzset{->-/.default=0.65}
\tikzset{-w-/.default=0.65}
\tikzstyle{bullet}=[circle,fill=black,inner sep=0.5mm]
\tikzstyle{circ}=[circle,draw=black,fill=white,inner sep=0.5mm]
\tikzstyle{vertex}=[circle,draw=black,thick,inner sep=0.5mm]
\tikzstyle{dot}=[draw,circle,fill=black,minimum size=0.5mm,inner sep = 0mm, outer sep = 0mm]
\tikzset{darrow/.style={double distance = 4pt,>={Implies},->},
	darrowthin/.style={double equal sign distance,>={Implies},->},
	tarrow/.style={-,preaction={draw,darrow}},
	qarrow/.style={preaction={draw,darrow,shorten >=0pt},shorten >=1pt,-,double,double
		distance=0.2pt}}
\newcommand\tikcirc[1][2.5]{\tikz[baseline=-#1]{\draw[thick](0,0)circle[radius=#1mm];}}
\newcommand{\tikzfig}[1]{\begin{tikzpicture}[auto,baseline={([yshift=-.5ex]current bounding box.center)}]#1\end{tikzpicture}}
\usetikzlibrary{decorations.pathreplacing}

\usepackage{amsthm}
\usepackage{thmtools}
\usepackage[noabbrev,capitalize]{cleveref}
\newtheorem{theorem}{Theorem}
\newtheorem{lemma}[theorem]{Lemma}
\newtheorem{proposition}[theorem]{Proposition}
\newtheorem{corollary}[theorem]{Corollary}
\newtheorem{conjecture}[theorem]{Conjecture}
\newtheorem{Claim}[theorem]{Claim}

\newtheorem*{theorem*}{Theorem}

\theoremstyle{definition}
\newtheorem{definition}{Definition}
\newtheorem{example}{Example}
\newtheorem{cexample}{Counterexample}
\newtheorem{exercise}{Exercise}
\newtheorem{question}{Question}
\newtheorem{remark}{Remark}

\addtolength{\textheight}{1.75cm}
\addtolength{\voffset}{-0.5cm}
\addtolength{\footskip}{+0.6cm}
\geometry{margin=1.2in}


\usepackage[backend=biber, maxbibnames=99, bibencoding=utf8, doi=false, isbn=false, url=false, style=alphabetic]{biblatex}
\addbibresource{refs.bib}

    \begin{document}
\fi

%\setcounter{section}{4}
\section[The main Theorem]{The Main Theorem}\label{sect:Main Theorem}
% and its bigraded version  $\adls$}



\nsubsection{Parity results}

Let us define $\mathcal{Z}_{>0} := \Span_{\mathbb{Q}} \{\zeta(\mathbf{k}) \mid \mathbf{k} \neq \emptyset\}$.
In this subsection, we put  $R = \mathcal{Z} / (\mathcal{Z}_{>0}^2 +\Q \pi^2)$.
In our main theorem, we utilize the parity results, which state that $\zeta(k_1,\ldots,k_r)$ with $k_1+\dots + k_r + r \equiv 1 \mod 2$ lies in $\Q[\pi^2]$-span of MZVs of depth less than $r$ (The depth of MZV means the number of entries in an argument of MZV).
These parity results are first proved analytically in \cite{Tsumura} and later algebraically proved in \cite{Ihara-Kaneko-Zagier}.
In \cite{Hirose-parity}, Hirose provided an explicit formula of the parity results from the point of view of the multitangent functions.
The significant point of his proof is the following formula, which follows from the functional equation of multitangent functions
\begin{thm}[\cite{Hirose-parity}]
For $(k_1,\ldots,k_r)\in\mathbb{Z}_{>0}^r$, we have
\begin{align}
\begin{aligned}
&\sum_{j = 0}^r (-1)^{k_{j+1}+\cdots + k_r} \zeta^*(k_{j},\ldots,k_1)\zeta^*(k_{j+1},\ldots,k_r) \\
=&\delta^{k_1,\ldots,k_r} + \sum_{j = 1}^r \sum_{\substack{a+2m+b = k_j \\ a,b,m\ge 0}} (-1)^{b+k_{j+1}+\cdots +k_r + m + 1}\\
&\times \frac{(2\pi )^{2m}}{(2m)!}B_{2m}\zeta_a^*(k_{j-1},\ldots,k_1)\zeta_b^*(k_{j+1},\ldots,k_d).
\end{aligned}
\label{eq:parity}
\end{align}
\end{thm}
We omit the definition of $\delta^{k_1,\ldots,k_r}$ but only note that $\delta^{k_1,\ldots,k_d} \in \Q[\pi^2]$.
The left-hand side of \eqref{eq:parity} is no longer AdMZVs, but we must note that AdMZVs appear after tanking modulo some sums of a product of MZVs and $\pi^2$.
From this perspective, we determine the $\Q$-linear relations of AdMZVs.

By definition of AdMZVs, we have
\begin{align*}
\Admzv{0}{k_1,\ldots,k_r} &\equiv \zeta(k_1,\ldots,k_r) + (-1)^{k_1+\cdots + k_r}\zeta(k_r,\ldots,k_1) \\
\Admzv{l}{k_1,\ldots,k_r} &\equiv (-1)^{k_1+\cdots + k_r + l} \zeta_l(k_r,\ldots,k_1)
\end{align*}
modulo $\mathcal{Z}_{>0}^2 +\Q \pi^2$ for arbitrary $k_1,\ldots,k_r\in\mathbb{Z}_{>0}$ and $l\in\mathbb{Z}_{ \ge 1 }$.

Considering \eqref{eq:parity} as the image of $R$, we obtain
\begin{align*}
&(-1)^{k_1+\cdots + k_r}(\zeta^*(k_1,\ldots,k_r) + (-1)^{k_1+\cdots + k_r} \zeta^*(k_r,\ldots,k_1))\\
\equiv &- \zeta_{k_r}^*(k_{r-1},\ldots,k_1) - (-1)^{k_1+\cdots+k_r} \zeta_{k_1}(k_2,\ldots,k_r)
\end{align*}
as an equality of $R$.
Hence, the parity result of AdMZVs can be written as
\begin{align}
\zeta_{\mathrm{Ad}}(k_1,\ldots,k_r;0)
\equiv&- \zeta_{\mathrm{Ad}}(k_1,\ldots,k_{r-1};k_r) - (-1)^{k_1+\cdots + k_r} \zeta_{\mathrm{Ad}}(k_r,\ldots,k_{2};k_1)  \label{eq:parity-s}
\end{align}
modulo $\mathcal{Z}_{>0}^2 +\Q \pi^2$.
%This is the precise $\Q$-linear relations among AdMZVs that we seek.

Next, we determine the relation for the commutative generating function derived from \eqref{eq:parity-s}.
From \eqref{eq:parity-s}, we get the following equality in $R\cotimes \ide{h}^{\vee}$:

\begin{prop}
Let $\Phi := \Phi_{\shuffle}^{-1}x_1\Phi_{\shuffle}$. Then we have
\begin{align*}
x_1\Phi^{11} x_1 = -x_1\Phi^{01}x_1 + S_X^{\vee}(x_1\Phi^{01}x_1). 
\end{align*}
Equivalently, 
\[
\Phi^{11} + \Phi^{01} - S_X^{\vee}(\Phi^{01}) = 0
\]
holds.
\end{prop}

\begin{proof}
Since
\[
\langle \Phi \mid x_0^lx_1x_0^{k_r}x_1\cdots x_0^{k_1}x_1 \rangle
= \zeta_{\mathrm{Ad}}(k_1+1.\ldots,k_r+1;l)
\]
holds, \eqref{eq:parity-s} turns out to be
\begin{align*}
& \langle \Phi \mid x_1x_0^{k_r}x_1\cdots x_0^{k_1}x_1 \rangle\\
=&-  \langle \Phi \mid x_0^{k_r+1}x_1x_0^{k_{r-1}}\cdots x_0^{k_1}x_1 \rangle - (-1)^{k_1+\cdots +k_r + r} \langle \Phi \mid x_0^{k_1+1}x_1x_0^{k_{2}}\cdots x_0^{k_r}x_1 \rangle.
\end{align*}

Therefore, we have
\begin{align*}
x_1\Phi x_1 =& \sum_{k_1.\ldots,k_r\ge 0}  \langle \Phi \mid x_1x_0^{k_r}x_1\cdots x_0^{k_1}x_1 \rangle x_1x_0^{k_r}x_1\cdots x_0^{k_1}x_1 \\
&
\begin{aligned}
\hspace{-0.3cm}=&\sum_{k_1.\ldots,k_r\ge 0} (-  \langle \Phi \mid x_0^{k_r+1}x_1x_0^{k_{r-1}}\cdots x_0^{k_1}x_1 \rangle\\
&\qquad - (-1)^{k_1+\cdots +k_r + r} \langle \Phi \mid x_0^{k_1+1}x_1x_0^{k_{2}}\cdots x_0^{k_r}x_1 \rangle ) x_1x_0^{k_r}x_1\cdots x_0^{k_1}x_1
\end{aligned}\\
&
\begin{aligned}
\hspace{-0.3cm}=&-\sum_{k_1.\ldots,k_r\ge 0}   \langle x_0\Phi^{01} x_1 \mid x_0^{k_r+1}x_1x_0^{k_{r-1}}\cdots x_0^{k_1}x_1 \rangle _1x_0^{k_r}x_1\cdots x_0^{k_1}x_1 \\
&\qquad + \sum_{k_1.\ldots,k_r\ge 0} \langle \Phi \mid x_0^{k_1+1}x_1x_0^{k_{2}}\cdots x_0^{k_r}x_1 \rangle S_X^{\vee} (x_1x_0^{k_1}x_1\cdots x_0^{k_r}x_1)\\
\end{aligned}\\
=& - x_1\Phi^{01}x_1 + S_X^{\vee}(x_1\Phi^{01}x_1).
\end{align*}

\end{proof}

From this perspective, we say $\Psi \in \ide{h}^{\vee}$ with $\wt \Psi\ge 2$ satisfies  the \textbf{strong parity result} if $\Psi$ satisfies
\begin{align}
\Psi^{11} + \Psi^{01} - S_{X}^{\vee}(\Psi^{01}) = 0. \label{eq:str-parity-pri}
\end{align}

Assume that $\Psi \in \ide{F}_2$ satisfies Equality \eqref{eq:str-parity-pri}.
Then by Equality \eqref{eq:antipode-X}, we have $S_{X}^{\vee}(\Psi^{10}) = -\Psi^{01}$.
Therefore, the strong parity result turns out to be
\begin{align}
\Psi^{11} + \Psi^{10} + \Psi^{01} = 0. \label{eq:str-prty-3cycle}
\end{align}
 
To conclude, we define two $\Q$-linear subspaces of $\ide{h}^{\vee}$ as follows:
\begin{align*}
V_{\strprty} =& \{\Psi \in  \ide{h}^{\vee} \mid \Psi^{11} + \Psi^{10} + \Psi^{01} = 0\}.
\end{align*}

\begin{prop}\label{prop:deriv-strprty}
We have a Lie subalgebra $V_{\strprty} \cap \Fad \subset \tm_1$.
\end{prop}

This proposition follows from the following lemma.

\begin{lem}
For $\Psi_1$, $\Psi_2 \in  \Fad \cap V_{\strprty}$, we have
\begin{align*}
&d_{\Psi_1}(\Psi_1)^{11} + d_{\Psi_1}(\Psi_2)^{01} + d_{\Psi_1}(\Psi_2)^{10}\\
&\begin{aligned}
=&(\Psi_{1}^{11}x_1\Psi_2^{11} + \Psi_{2}^{11}x_1\Psi_1^{11}) - ( \Psi_1^{01}x_1\Psi_2^{11} + \Psi_2^{01}x_1\Psi_1^{11}) \\
&\quad-(\Psi_1^{11}x_1\Psi_2^{10} + \Psi_2^{11}x_1\Psi_1^{10}) + (\Psi_{1}^{10}x_0\Psi_2^{01}  +  \Psi_{2}^{10}x_0\Psi_1^{01} ).
\end{aligned}
\end{align*}
\end{lem}

\begin{proof}

Since 
\begin{align*}
x_1d_{\Psi_1}(x_1\Psi_2^{11}x_1)^{11}x_1=x_1\Psi_1^{11}x_1 \Psi_2^{11}x_1 + x_1\Psi_1^{10}x_0 \Psi_2^{11}x_1 + x_1\Psi_2^{11}x_1\Psi_1^{11}x_1 + x_1\Psi_2^{11}x_0\Psi_1^{01}x_1 + x_1d_{\Psi_1}(\Psi_2^{11})x_1
\end{align*}
holds, we have
\begin{align*}
d_{\Psi_1}(\Psi_2)^{11} = \Psi_1^{11}x_1\Psi_2^{11} + \Psi_2^{11}x_1\Psi_1^{11} + \Psi_1^{10}x_0\Psi_2^{11} + \Psi_2^{11}x_0\Psi_1^{01} + d_{\Psi_1}(\Psi_2^{11}).
\end{align*}

In a similar manner, we have
\begin{align*}
&d_{\Psi_1}(x_0\Psi_2^{01}x_1 + x_1\Psi_2^{11}x_1)\\
=&x_0\Psi_2^{01}x_1\Psi_1^{11}x_1 + x_0\Psi_{2}^{01}x_0\Psi_1^{01}x_1 + x_0\Psi_{1}^{01}x_1\Psi_2^{11}x_1 + x_0\Psi_{1}^{00}x_0\Psi_2^{11}x_1 + x_0d_{\Psi_1}(\Psi_2^{01})x_1
\end{align*}
and
\begin{align*}
&d_{\Psi_1}(x_1\Psi_2^{10}x_0 + x_1\Psi_2^{11}x_1)\\
=&x_1\Psi_1^{11}x_1\Psi_2^{10}x_0 + x_1\Psi_{1}^{10}x_0\Psi_2^{10}x_0 + x_1\Psi_{2}^{11}x_1\Psi_2^{10}x_0 + x_1\Psi_{2}^{11}x_0\Psi_1^{00}x_0 + x_1d_{\Psi_1}(\Psi_2^{10})x_0.
\end{align*}
Thus we have

\begin{align*}
d_{\Psi_1}(\Psi_2)^{01} =& \Psi_2^{01}x_1\Psi_1^{11} + \Psi_{2}^{01}x_0\Psi_1^{01} + \Psi_{1}^{01}x_1\Psi_2^{11} + \Psi_{1}^{00}x_0\Psi_2^{11} + d_{\Psi_1}(\Psi_2^{01})
\end{align*}

and

\begin{align*}
d_{\Psi_1}(\Psi_2)^{10} =&\Psi_1^{11}x_1\Psi_2^{10} + \Psi_{1}^{10}x_0\Psi_2^{10} + \Psi_{2}^{11}x_1\Psi_2^{10} + \Psi_{2}^{11}x_0\Psi_1^{00} + d_{\Psi_1}(\Psi_2^{10}).
\end{align*}

Therefore, we get

\begin{align*}
&d_{\Psi_1}(\Psi_1)^{11} + d_{\Psi_1}(\Psi_2)^{01} + d_{\Psi_1}(\Psi_2)^{10}\\
&\begin{aligned}
=&(\Psi_{1}^{11}x_1\Psi_2^{11} + \Psi_{2}^{11}x_1\Psi_1^{11}) + ( \Psi_1^{01}x_1\Psi_2^{11} + \Psi_2^{01}x_1\Psi_1^{11}) \\
&\quad+(\Psi_1^{11}x_1\Psi_2^{10} + \Psi_2^{11}x_1\Psi_1^{10}) + (\Psi_1^{00}x_0\Psi_2^{11} + \Psi_2^{11}x_0\Psi_1^{00})\\
&\qquad + (\Psi_{1}^{10}x_0\Psi_2^{11} +  \Psi_{2}^{11}x_0\Psi_1^{01} + \Psi_1^{10}x_0\Psi_2^{10} + \Psi_2^{01}x_0 \Psi_1^{01})\\
&\qquad +(d_{\Psi_1}(\Psi_2)^{11} + d_{\Psi_1}(\Psi_2^{01}) + d_{\Psi_1}(\Psi_2^{10})).
\end{aligned}
\end{align*}

Since $\Psi_2 \in V_{\strprty}$, we have

\begin{align*}
 &\Psi_{1}^{10}x_0\Psi_2^{11} +  \Psi_{2}^{11}x_0\Psi_1^{01} + \Psi_1^{10}x_0\Psi_2^{10} + \Psi_2^{01}x_0 \Psi_1^{01}\\
\stackrel{\eqref{eq:str-prty-3cycle}}{=}&-\Psi_{1}^{10}x_0\Psi_2^{10} - \Psi_{1}^{10}x_0\Psi_2^{01}  -  \Psi_{2}^{10}x_0\Psi_1^{01} -  \Psi_{2}^{01}x_0\Psi_1^{01} + \Psi_1^{10}x_0\Psi_2^{10} + \Psi_2^{01}x_0 \Psi_1^{01}\\
=&-\Psi_{1}^{10}x_0\Psi_2^{01}  -  \Psi_{2}^{10}x_0\Psi_1^{01} 
\end{align*}

and 
\begin{align*}
&d_{\Psi_1}(\Psi_2)^{11} + d_{\Psi_1}(\Psi_2^{01})  d_{\Psi_1}(\Psi_2^{10}) \\
=&d_{\Psi_1}(\Psi_2^{11} + \Psi_2^{10} + \Psi_2^{01})\\
\stackrel{\eqref{eq:str-prty-3cycle}}{=}& 0.
\end{align*}

Since $\Psi \in \Fad$ satisfies $\Psi^{00} = 0$, we have
\[
\Psi_1^{00}x_0\Psi_2^{11} + \Psi_2^{11}x_0\Psi_1^{00} = 0.
\]


Therefore, it follows that
\begin{align*}
&d_{\Psi_1}(\Psi_1)^{11} - d_{\Psi_1}(\Psi_2)^{01} - d_{\Psi_1}(\Psi_2)^{10}\\
&\begin{aligned}
=&(\Psi_{1}^{11}x_1\Psi_2^{11} + \Psi_{2}^{11}x_1\Psi_1^{11}) + ( \Psi_1^{01}x_1\Psi_2^{11} + \Psi_2^{01}x_1\Psi_1^{11}) \\
&\quad +(\Psi_1^{11}x_1\Psi_2^{10} + \Psi_2^{11}x_1\Psi_1^{10}) - (\Psi_{1}^{10}x_0\Psi_2^{01}  +  \Psi_{2}^{10}x_0\Psi_1^{01} ).
\end{aligned}
\end{align*}

\end{proof}






\nsubsection{Main Theorems}

In this paper, we provide the following:

\begin{mt}\label{mt:addmr-Lie}
The $\Q$-linear space $\addmr \cap \Fad \cap V_{\strprty}$ is a Lie subalgebra of $\adtm_1$.
Equivalently, the fiber product
\[
\addmr_{\ide{a}} \times_{\tm_{1,\ide{a}}}
\ide{F}_{2,\ide{a}}^{\ad(x_1)} \times_{\tm_{1,\ide{a}}}
V_{\strprty,\ide{a}}
\]
is a closed affine subscheme of $\tm_{1,\ide{a}}$ whose codomain is $\Lalg$.
\end{mt}

\begin{comment}
As a corollary, we get the following as our main theorem.

\begin{mt}\label{mt:AdDMR-Grp}
For $\Q$-algebra $R$, $R$-module $\AdDMR(R) \cap \FAd(R) \cap V_{\strprty}(R)$ is the subgroup of $\AdTM_1(R)$.
Equivalently, we define a functor $\AdDMR \cap \FAd \cap V_{\strprty} : \Qalg \rightarrow \Grp$ by $R \mapsto  \AdDMR(R) \cap \FAd(R) \cap V_{\strprty}(R)$. Then, $\AdDMR \cap \FAd \cap V_{\strprty} $ is the closed affine group subscheme of $\TM_1$.
\end{mt}
\end{comment}




\ifdefined\isMainDocument
\else
    %\bibliographystyle{amsplain}
    %\bibliography{reference-adjoint}
    \end{document}
\fi