\newpage

%------------------------------------------------------------------
\section{Algebraic “shell”}\label{sec:alg-skin}
%------------------------------------------------------------------
In the “shell” block we cut off a narrow layer near the zero set
of low-rank polynomials, which could otherwise worsen the cumulative exponent in~$D$.
The idea goes back to~\cite{Guth_Huang_Inflation}; here it suffices to adapt it to the three-dimensional paraboloid
with an application of Wongkew’s tubular estimate.

%------------------------------------------------------------------
\subsection{Problem setup}\label{subsec:skin-setup}
%------------------------------------------------------------------
Let $P(z)$ be a polynomial of total degree $d:=\deg P\le D^{1/4}$ and
let $Z(P)=\{z:\,P(z)=0\}$ denote its zero set.
Introduce the anisotropic scaling
\[
   S_\lambda(t,x) := (\lambda^{3/2}t,\ \lambda^{1/2}x)
\]
and the induced metric
\[
   \operatorname{dist}_\lambda(z,Z) := 
   \operatorname{dist}\bigl(S_\lambda z,\ S_\lambda Z\bigr).
\]
For a small absolute constant $c>0$ set
\[
   \beta := \frac{c\,D^{-1}}{d}, \qquad
   \mathcal N_{\beta}(P) :=
   \bigl\{(t,x)\in Q_\lambda:\ 
      \operatorname{dist}_\lambda\bigl((t,x),Z(P)\bigr)
      < \beta \bigr\}.
\]
Thus $\mathcal N_{\beta}(P)$ is the tubular $\beta$–neighborhood of $Z(P)$
in the anisotropic metric (we keep the notation $\mathcal N_{\beta}$ for consistency).
The goal is to estimate the measure of $\bigl|Q_{\lambda}\cap\mathcal N_{\beta}(P)\bigr|$ uniformly in $P$.

%------------------------------------------------------------------
\subsection{Measure estimate for the shell}\label{subsec:skin-measure}
%------------------------------------------------------------------
\begin{lemma}\label{lem:skin-measure}
There exists an absolute constant $C>0$ such that for all
polynomials $P$ of degree $d\le D^{1/4}$ one has
\[
   \bigl|Q_{\lambda}\cap\mathcal N_{\beta}(P)\bigr|
   \;\le\;
   C\,D^{-1}\,\bigl|Q_{\lambda}\bigr|.
\]
\end{lemma}

\begin{proof}
Consider the image of $Q_\lambda$ under $S_\lambda$; this is a domain
of unit scale (comparable volume), and the relative measure
is preserved by the definition of \(\operatorname{dist}_\lambda\).
Let $P_\lambda := P\circ S_\lambda^{-1}$; then $\deg P_\lambda = d$,
and $S_\lambda\!\bigl(\mathcal N_\beta(P)\bigr)$ is the usual
$r$–tube around $Z(P_\lambda)$ of thickness $r=\beta$ in the Euclidean metric.
By Wongkew’s theorem (\cite{Wongkew2003})
in $\mathbb{R}^4$ the volume of an $r$–neighborhood of an algebraic hypersurface
of degree $d$ in a unit-sized region is $\lesssim d\,r$.
Substituting $r=\beta=cD^{-1}/d$ and returning to the original variables, we obtain
\[
   \bigl|Q_{\lambda}\cap\mathcal N_{\beta}(P)\bigr|
   \lesssim \beta\,|Q_\lambda|
   \lesssim D^{-1}\,|Q_\lambda|.
\]
\end{proof}

%------------------------------------------------------------------
\subsection{Contribution to the balance of exponents}\label{subsec:skin-balance}
%------------------------------------------------------------------
Lemma~\ref{lem:skin-measure} states that the set 
$\mathcal N_{\beta}(P)$ occupies at most $C\,D^{-1}$ fraction of the volume of $Q_\lambda$. 
To correctly transfer this smallness to the $L^6$ level, we localize the functional 
by anisotropic blocks and separate the \emph{transversal} contribution (which yields the $D^{-1}$ penalty)
from the \emph{tangential} one (which is redirected to the narrow cascade §\,\ref{sec:narrow}).

\paragraph{Block localization and trans/tan splitting.}
Cover $Q_\lambda$ by anisotropic blocks $\{B\}$ of scale
$\lambda^{-3/2} \times (\lambda^{-1/2})^3$ with bounded overlap.
Let $\mathcal J(B)\ge 0$ be the local functional collecting the contribution of block $B$ 
in the $TT^\ast$+broad estimate, so that
\[
\|F\|_{L^6(Q_\lambda)}^6 \;\lesssim\; \sum_{B\subset Q_\lambda} \mathcal J(B).
\]
For each $B\subset Q_\lambda$ split $\mathcal J(B)$ as
\[
\mathcal J(B)\ =\ \mathcal J^{\mathrm{tr}}(B)\ +\ \mathcal J^{\mathrm{tan}}(B),
\]
where $\mathcal J^{\mathrm{tr}}(B)$ collects contributions of those wave packets $T_\Theta$ 
which on $B\cap\mathcal N_\beta(P)$ are \emph{transversal} to $Z(P)$
(the angle between the axis of $T_\Theta$ and the tangent plane $T_zZ(P)$ is at least $c\,\alpha$),
while $\mathcal J^{\mathrm{tan}}(B)$ collects the \emph{tangential} contributions
(angle $\lesssim \alpha$). This splitting is standard and consistent with the transverse dichotomy of §\,\ref{subsec:gradients}.

\medskip
\noindent\emph{Transversal contribution.}
For transversal tubes the “geometry of intersection with the wall’’ holds:
each $T_\Theta$ intersects $\mathcal N_\beta(P)$ along a relative length
$\lesssim \beta\,d \sim D^{-1}$ (in the scale of $Q_\lambda$), and the block overlap
is bounded by a constant. Hence
\begin{equation}\label{eq:skin-trans}
\sum_{B\subset \mathcal N_\beta} \mathcal J^{\mathrm{tr}}(B)
\ \le\ C\,D^{-1}\!\sum_{B\subset Q_\lambda} \mathcal J(B).
\end{equation}
The proof is direct: the tubular estimate of Lemma~\ref{lem:skin-measure} in the anisotropic
metric and the estimate of the relative time a transversal tube spends in $\mathcal N_\beta(P)$
(thickness $\beta/d$ for degree $\deg P\le d$) yield a fraction $\lesssim D^{-1}$ at the level of volume, 
and bounded overlap of blocks transfers this to the localized sum.

\medskip
\noindent\emph{Tangential contribution.}
For $\mathcal J^{\mathrm{tan}}(B)$ we do not attempt to estimate by the measure of $\mathcal N_\beta(P)$:
by definition they correspond to configurations where packets “lie along’’ $Z(P)$
(angles $\lesssim\alpha$). Their contribution is redirected to the narrow regime §\,\ref{sec:narrow},
which is activated precisely in the presence of such clusters; thus
\begin{equation}\label{eq:skin-tangent}
\sum_{B\subset \mathcal N_\beta} \mathcal J^{\mathrm{tan}}(B)
\ \text{is estimated by the narrow cascade block (§\,\ref{sec:narrow}) and does not carry the $D$ penalty.}
\end{equation}

\paragraph{Conclusion for the balance.}
Combining \eqref{eq:skin-trans}–\eqref{eq:skin-tangent} with the $TT^\ast$ reconstruction from the localized sum,
we see that the “algebraic shell’’ block contributes to the cumulative balance \emph{through the transversal part}
\[
   \boxed{-\tfrac{1}{12}\ \text{in }\lambda,\qquad -1\ \text{in }D},
\]
while the tangential part is accounted for in the “narrow’’ block (§\,\ref{sec:narrow}) and does not
contribute to the $D$ penalty.

\begin{remark}[Why mere smallness of measure is insufficient]
A naive bound
\[
\|F_\Theta\|_{L^6(Q_\lambda)} \le \|F_\Theta\|_{L^6(Q_\lambda\setminus \mathcal N_\beta)} 
+ O(D^{-1})\,\|F_\Theta\|_{L^6(Q_\lambda)}
\]
does not follow from $|\mathcal N_\beta(P)|\lesssim D^{-1}|Q_\lambda|$, since the energy could concentrate
on $\mathcal N_\beta$. This is why we (i) localize the functional by blocks, 
(ii) \emph{separate} the contribution into transversal/tangential parts, and 
(iii) use the time-of-stay argument for transversal tubes in $\mathcal N_\beta(P)$
for \eqref{eq:skin-trans}, while the tangential contribution \eqref{eq:skin-tangent} is redirected to §\,\ref{sec:narrow}.
In this way the same $D^{-1}$ penalty is realized, but in a strictly controlled part.
\end{remark}

%------------------------------------------------------------------
\subsection{Remark on the constant}\label{subsec:skin-remark}
%------------------------------------------------------------------
The coefficient $-\tfrac{1}{12}$ (in~$\lambda$) corresponds to the choice
of thickness $\beta/d$ of the tubular layer with $\beta=c\,D^{-1}$ and the restriction
$\deg P\le D^{1/4}$. By Lemma~\ref{lem:skin-measure} the fraction
$|Q_\lambda\cap\mathcal N_\beta(P)|$ does not exceed $C\,D^{-1}$.
In the transversal part this directly yields a $-1$ in $D$
(equivalently $-\tfrac{1}{12}$ in $\lambda$, since $D=\lambda^{1/12}$).
The tangential part, corresponding to angles $\lesssim\alpha$, 
is handled by the narrow cascade (§\,\ref{sec:narrow}) and does not affect the $D$ exponent.
No refinements are required here; possible improvements would be related to a more delicate choice
of thickness/degree and additional structural assumptions, and lie beyond the scope of this paper.
