
\documentclass{amsart}

\usepackage{txfonts}
\usepackage{graphicx}%
\usepackage{multirow}%
\usepackage{amsmath,amssymb,amsfonts}%
\usepackage{amsthm}%
\usepackage{mathrsfs}%
\usepackage[title]{appendix}%
\usepackage{xcolor}%
\usepackage{textcomp}%
\usepackage{manyfoot}%
\usepackage{booktabs}%
\usepackage{algorithm}%
\usepackage{algorithmicx}%
\usepackage{algpseudocode}%
\usepackage{listings}%

%\usepackage{refcheck}


\usepackage{pdfsync}
\usepackage{enumitem}
\usepackage{hyperref}
\hypersetup{
    colorlinks=true,       % false: boxed links; true: colored links
    linkcolor=blue,          % color of internal links
    citecolor=blue,        % color of links to bibliography
    filecolor=blue,      % color of file links
    urlcolor=blue           % color of external links
}




%%%%
\newtheorem{theo}{Theorem}
\newtheorem{prop}{Proposition}
\newtheorem{lemm}{Lemma}

\theoremstyle{thmstyletwo}
\newtheorem{coro}{Corollary}
\newtheorem{question}{Question}


\theoremstyle{remark}
\newtheorem{rema}{\bf Remark}
\newtheorem{example}{\bf Example}


\raggedbottom


\begin{document}

\title{${\mathbb Z}_{p}^{m}$-actions of type $(d;p,n)$}


\author{Rub\'en A. Hidalgo} 
\address{Departamento de Matem\'atica y Estad\'istica, Universidad de la Frontera, Temuco, 4811230, Chile}
\email{ruben.hidalgo@ufrontera.cl}

\author{Maximiliano Leyton-\'Alvarez}
\address{Instituto de Matem\'atica y F\'isica, Universidad de Talca, Talca, 3460000, Chile}
\email{leyton@inst-mat.utalca.cl}





%%==================================%%
%% Sample for unstructured abstract %%
%%==================================%%

\begin{abstract}
A ${\mathbb Z}_{p}^{m}$-action of type $(d;p,n)$, where $2 \leq d \leq m \leq n$  are integers, is a pair $(S,N)$ where $S$ is a $d$-dimensional compact complex manifold, $N \cong {\mathbb Z}_{p}^{m}$ is a group of holomorphic automorphisms of $S$ such that the quotient orbifold $S/N$ is the $d$-dimensional projective space ${\mathbb P}^{d}$ whose branch locus consists of $n+1$ hyperplanes in general position, each one of branch order $p$.

If $(d;p,n) \notin \{(2;2,5),(2;4,3)\}$ and $d+1 \leq n$,  then we prove that: (i)  $N$ is a normal subgroup of ${\rm Aut}(S)$ and (ii) if $(S,M)$ is a ${\mathbb Z}_{\hat{p}}^{\hat{m}}$-action of type $(d;\hat{p},\hat{n})$, then $M=N$. If, moreover, $d+1 \leq n \leq 2d-1$, then we observe that $S$ is not algebraically hyperbolic. 
\end{abstract}


\keywords{Algebraic variety, Automorphisms}

\subjclass[2020]{14J50; 32Q40; 53C15.}

%%%%%%%%%%%%%%%%%%%%%%


\maketitle

%%%%%%%%%%%%%%%%%%%%%%%%
%%%%%%%%%%%%%%%%%%%%%%%%
\section{Introduction}
Let $S$ be a compact complex manifold of dimension $d \geq 1$.
Its group ${\rm Aut}(S)$ of holomorphic automorphisms is known to be a complex Lie group \cite{BM} and 
there is a natural short exact sequence $1 \to {\rm Aut}^{0}(S) \to {\rm Aut}(S) \to {\rm Aut}(S)/{\rm Aut}^{0}(S)$, where ${\rm Aut}^{0}(S)$ denotes the connected component of the identity. Let $N$ be a subgroup of ${\rm Aut}(S)$ which acts properly discontinuously on $S$; so, we have associated the quotient orbifold $S/N$.
 We are interested in the following two natural questions: 
 \begin{enumerate}
 \item[(1)] May we decide, in terms of the structure of the quotient orbifold $S/N$, if $N$ is a normal subgroup of ${\rm Aut}(S)$? 
 \item[(2)] Let $M$ be another properly discontinuous subgroup of ${\rm Aut}(S)$, which is isomorphic as an abstract group to $N$ and such that the quotient orbifolds $S/N$ and $S/M$ are homeomorphic. May we decide, in terms of the structure of the quotient orbifold, if $N=M$?. 
\end{enumerate}
 
\medskip
 
In this paper, we investigate the above questions in a very particular class of manifolds. More precisely, we consider those pairs $(S,N)$, where 
$N \cong {\mathbb Z}_{p}^{m}$, $m \geq 1$ and $p \geq 2$ are integers, and the quotient orbifold $S/N$ is the $d$-dimensional projective space ${\mathbb P}^{d}$ whose branch locus consists of $n+1$ hyperplanes in general position, each one of branch order $p$. Let us recall that the hyperplanes are in general position if: (i) the intersection of every subcollection of $1 \leq k \leq d$ hyperplanes has dimension $d-k$, and (ii) every subcollection of $k \geq d+1$ hyperplanes has empty intersection.
 In this situation, we will say that $(S,N)$ is a  ${\mathbb Z}_{p}^{m}$-action of type $(d;p,n)$.
Necessarily, $ d \leq m \leq n$, and  $S$ is known to be projective, i.e., it may be holomorphically embedded in some projective space (and ${\rm Aut}(S)$ is a group of biregular automorphisms). If $n=d$, then $S$ is isomorphic to ${\mathbb P}^{d}$. If $n=m=d+1$, then $S$ is isomorphic to the Fermat hypersurface of degree $p$.
 

\begin{theo}
Let $(S,N)$ is a  ${\mathbb Z}_{p}^{m}$-action of type $(d;p,n) \notin \{(2;2,5),(2;4,3)\}$ and $3 \leq d+1 \leq n$. Then
(i) ${\rm Aut}(S)$ is finite, (ii) $N$ is a normal subgroup of ${\rm Aut}(S)$, and  (iii) if $(S,M)$ is a ${\mathbb Z}_{q}^{r}$-action, then $M=N$.
\end{theo}

We should note that the facts (ii) and (iii), in the previous result, are not generally true for the case of curves (i.e, $d=1$). 
 
\medskip

Examples of compact complex manifolds, for which the group of holomorphic automorphisms is finite, are provided by the so-called algebraically hyperbolic manifolds \cite{BKV}. In \cite{Demailly},  Demailly observed that every compact complex Kobayashi hyperbolic manifold is algebraically hyperbolic. In the same paper, he conjectured the converse. 
 
Now, if $(S,N)$ is a ${\mathbb Z}_{p}^{m}$-action of type $(d;p,n) \notin \{(2;2,5),(2;4,3)\}$, where $3 \leq d+1 \leq n$, then ${\rm Aut}(S)$ is finite. It seems natural to ask if $S$ is algebraically hyperbolic. 
The next result is a negative answer in some cases.

\begin{theo}
Let $(S,N)$ be a ${\mathbb Z}_{p}^{m}$-action of type $(d;p,n) \notin \{(2;2,5),(2;4,3)\}$, where $3 \leq d+1 \leq n$.
If either (i) $n \leq 2d-1$, or (ii) $n=2d$ and $p \in \{2,3\}$,  or (iii) $n=2d+1$ and $p=2$, then $S$ is not algebraically hyperbolic, in particular, not Kobayashi hyperbolic. 
\end{theo}

A natural question is whether the exceptional cases provided in the above result are the only ones for which $S$ is not algebraically hyperbolic.
 
\medskip
 
%Our next result concerns the number of topologically actions. We say that two ${\mathbb Z}_{p}^{m}$-actions $(S_{1},N_{1})$ and $(S_{2},N_{2})$, both of type $(d;p,n)$, are topologically equivalent if there is an orientation-preserving homeomorphism $F:S_{1} \to S_{2}$ such that $F N_{1} F^{-1}=N_{2}$.  We are concerned with the following question: Given a triple $(d;p,n)$ and $d \leq m \leq n$: How many topologically equivalent ${\mathbb Z}_{p}^{m}$-actions are there? In \cite{HR}, this has been studied for the case $d=1$, i.e., when $S$ is a closed Riemann surface. Next, we assume the case  $d \geq 2$ (so $m \geq 2$). Let $H=\langle \varphi_{1},\ldots,\varphi_{n} \rangle \cong {\mathbb Z}_{p}^{n}$ and set $\varphi_{n+1}=(\varphi_{1}\cdots\varphi_{n})^{-1}$, where each $\varphi_{j}$ has order $p$. Let us denote by ${\mathcal F}(d;p,n,m)$ the collection of those subgroups $K \cong {\mathbb Z}_{p}^{m}$ of $H$ such that $\varphi_{i_{1}}^{l_{1}} \varphi_{i_{2}}^{l_{2}} \cdots \varphi_{i_{j}}^{l_{j}} \notin K$, where $1 \leq j \leq d$, $l_{j} \in \{1,\ldots, p-1\}$ and $1 \leq i_{1}<\cdots < i_{j} \leq n+1$. Let ${\rm Aut}_{g}(H) \cong {\mathfrak S}_{n+1}$ be the group of automorphisms of $H$ which correspond to permutations of the set $\{\varphi_{1},\ldots,\varphi_{n+1}\}$. Note that ${\rm Aut}_{g}(H)$ preserves the collection ${\mathcal F}(d;p,n,m)$. 
%\begin{theo}
%There is a bijection between ${\mathcal F}(d;p,n,m)/{\rm Aut}_{g}(H)$ and the set of topologically equivalent ${\mathbb Z}_{p}^{m}$-actions of type $(d;p,n)$. 
%\end{theo}


{\bf Notations:}
Suppose $Y\subset {\mathbb P}^{k}$ is a smooth irreducible projective complex algebraic variety of dimension $d$. In that case, we will denote by $\rm{Aut}(Y)$ its group of all holomorphic automorphisms and by $\rm{Lin}(Y)$ its group of linear automorphisms (that is, its automorphisms obtained as the restriction of a projective linear transformation of ${\mathbb P}^{k}$).




%%%%%%%%%%%%%%%%%%%%
%%%%%%%%%%%%%%%%%%%%%
\section{Generalized Fermat varieties}
As noticed above, the maximal value of $m$, in the definition of ${\mathbb Z}_{p}^{m}$-action of type $(d;p,n)$, is $m=n$. 
Also, as observed in \cite{HHL23}, $n \geq d$. 


%%%%%%%%%%%%%%
\subsection{The group $H$}
Let $n \geq 1, p \geq 2$ be integers. Set $\omega_{p}=e^{2 \pi i/p}$. 
Let us consider the linear automorphisms 
$\varphi_{1},\ldots,\varphi_{n+1} \in {\rm PGL}_{n+1}({\mathbb C})$  of ${\mathbb P}^{n}$, defined by 
$$\varphi_j([x_1:\cdots:x_j:\cdots:x_{n+1}]):=[x_1:\cdots:\omega_{p}x_j:\cdots:x_{n+1}].$$
Then $\varphi_{1} \circ \cdots \circ \varphi_{n+1}=1$ and
$H:=\langle\varphi_1,\cdots,\varphi_{n}\rangle \cong {\mathbb Z}_{p}^{n}$.
We say that $\{\varphi_{1},\ldots, \varphi_{n+1}\}$ is a set of canonical generators of $H$.

Let us denote by ${\rm Aut}_{g}(H)$ the group of automorphisms of $H \cong {\mathbb Z}_{p}^{n}$ which correspond to permutations of the set of canonical generators  $\{\varphi_{1},\ldots,\varphi_{n+1}\}$. 
Note that ${\rm Aut}_{g}(H)=\langle \Psi_{1}, \Psi_{2}\rangle \cong {\mathfrak S}_{n+1}$, where
$$\Psi_{1}:(\varphi_{1}, \ldots, \varphi_{n+1}) \mapsto (\varphi_{2},\varphi_{1},\varphi_{3},\ldots,\varphi_{n+1}),\;
\Psi_{2}:(\varphi_{1}, \ldots, \varphi_{n+1}) \mapsto (\varphi_{n+1},\varphi_{1},\varphi_{2},\ldots,\varphi_{n}).$$





%%%%%%%%%%%%
\subsection{Generalized Fermat pairs}
A generalized Fermat pair of type $(d;k,n)$ is a 
 ${\mathbb Z}_{p}^{n}$-action $(X,H_{X})$ of type $(d;p,n)$. We also say that $X$ 
 is a generalized Fermat variety of type $(d;p,n)$, and that $H_{X}$ is a generalized Fermat group of type $(d;p,n)$. 
 
 If $d=1$, then $X$ is a closed Riemann surface uniformized by the derived subgroup of a Fuchsian group of signature $(0;p,\stackrel{n+1}{\ldots},p)$); we also say that $X$ is 
 a generalized Fermat curve of type $(p,n)$. 


%%%%%%%%%%%%%%
\subsection{Case $n=d$}
In this case, we may assume (up to biholomorphisms) that $X={\mathbb P}^{d}$. The group $H$ is a generalized Fermat group of type $(d;p,d)$. This is not the unique generalized Fermat group of such type, but any other is ${\rm PGL}_{d+1}({\mathbb C})$-conjugated to $H$. 



%%%%%%%%%%%%%%%%
\subsection{Case $n=d+1$}
In this case, (up to biholomorphisms) we may assume that $X=F_{p}=\{x_{1}^{p}+ \cdots+x_{d+2}^{p}=0\} \subset {\mathbb P}^{d+1}$, the Fermat hypersurface of degree $p$.
The group $H$ is a generalized Fermat group of type $(d;p,d+1)$.
If (i) $d \geq 2$ and $(d,p) \neq (2,4)$, or (ii) $d=1$ and $p>3$, then $H$ is the unique generalized Fermat group of type $(d;p,d+1)$, and ${\rm Aut}(X)=H \rtimes {\mathfrak S}_{d+2}$, where ${\mathfrak S}_{d+2}$ is the subgroup of ${\rm PGL}_{d+2}({\mathbb C})$ given by permutations of the coordinates.


%%%%%%%%%%%%%%%
\subsection{Case $n \geq d+2$}
Next, we recall the algebraic models of $(X,H_{X})$ and the uniqueness results for generalized Fermat groups.

%%%%%%%%%%%%
\subsubsection{\bf The parameter space $\Omega_{n,d}$}
Assume $d \geq 1$, and $n \geq d + 2$ are integers.
If $\Lambda=(\lambda_{i,j}) \in {\rm M}_{(n-d-1) \times d}({\mathbb C})$, then we may consider the collection ${\mathcal B}(\Lambda)$ consisting of the following $(n+1)$ hyperplane in ${\mathbb P}^{d}$:
$$\Sigma_{j}=\{[y_{1}:\cdots:y_{d+1}] \in {\mathbb P}^{d}: y_{j}=0\}, \; j=1,\ldots,d+1,$$
$$\Sigma_{d+2}=\{[y_{1}:\cdots:y_{d+1}] \in {\mathbb P}^{d}: y_{1}+\cdots+y_{d+1}=0\},$$
$$\Sigma_{d+2+j}(\Lambda)=\{[y_{1}:\cdots:y_{d+1}] \in {\mathbb P}^{d}: \lambda_{j,1}y_{1}+\cdots+\lambda_{j,d} y_{d}+y_{d+j}=0\}, \; j=1,\ldots,n-d-1.$$


Let us denote by $\Omega_{n,d} \subset {\rm M}_{(n-d-1) \times d}({\mathbb C})$ the subset consisting of those $\Lambda$ such that the above collection is in general position. This space is a connected, open, and dense subset of ${\rm M}_{(n-d-1) \times d}({\mathbb C}) \cong {\mathbb C}^{(n-d-1)d}$. 


%%%%%%%%%%%%
\subsubsection{\bf A family of algebraic varieties parametrized by $\Omega_{n,d}$}\label{Sec:algebra}
If $\Lambda=(\lambda_{i,j})\in \Omega_{n,d}$, then we may consider the following algebraic variety

\begin{equation}\label{Equation Algebraic Model}
X_{n}^{p}(\Lambda):=\left\{\begin{matrix}
x_{1}^{p}+\cdots+x_{d}^{p}+x_{d+1}^{p}+x_{d+2}^{p}&=&\ 0\\
\lambda_{1,1}x_{1}^{p}+\cdots+\lambda_{1,d}x_{d}^{p}+x_{d+1}^{p}+x_{d+3}^{p}&=&\ 0\\
\vdots&\vdots&\ \vdots\\
\lambda_{n-d-1,1}x_{1}^{p}+\cdots+\lambda_{n-d-1,d}x_{d}^{p}+x_{d+1}^{p}+x_{n+1}^{p}&=&\ 0
\end{matrix}\right\}\subset{\mathbb P}^{n}.
\end{equation}

\begin{rema} \label{rem:sc}
The variety $X^{p}_{n}(\Lambda)$ is an irreducible nonsingular complete intersection projective variety of dimension $d$. So, 
if $d \geq 2$, then $X^{d}_{n}(\Lambda)$ is simply connected (this result is attributed to Lefschetz; see \cite{Har74}). 
\end{rema}

The following facts can be deduced from the above algebraic model of $X_{n}^{p}(\Lambda)$ and the form of the elements $\varphi_{i}$.
\begin{enumerate}
\item[(I)] ${\mathbb Z}_{p}^{n}\cong H<{\rm Aut}(X^{p}_{n}(\Lambda))<{\rm PGL}_{n+1}({\mathbb C})$.
\item[(II)] $\varphi_{1} \varphi_{2} \cdots \varphi_{n+1}=1$.
\item[(III)] The only non-trivial elements of $H$ with fixed set points being of maximal dimension $d-1$ are the non-trivial powers
of the generators $\varphi_{1}, \ldots, \varphi_{n+1}$. Moreover, for $d \geq 2$, ${\rm Fix}(\varphi_{j}):=\{x_{j}=0\} \cap X^{p}_{n}(\Lambda)$
is isomorphic to a generalized Fermat variety of type $(d-1;k,n-1)$.
\item[(IV)]  $\pi:X^{p}_{n}(\Lambda) \to {\mathbb P}^{d}: [x_{1}:\cdots:x_{n+1}] \mapsto [x_{1}^{p}: \cdots: x_{d+1}^{p}]$ is a Galois branched cover with deck group $H$, whose branch locus is the collection ${\mathcal B}(\Lambda)$. In particular, $(X_{n}^{p}(\Lambda),H)$ is a generalized Fermat pair of type $(d;p,n)$
\end{enumerate}


\begin{rema}
As a consequence of Randell's isotopy theorem \cite{Randell}, for $\Lambda_{1}, \Lambda_{2} \in \Omega_{n,d}$, there is an orientation-preserving homeomorphism 
 $f:{\mathbb P}^{d} \to {\mathbb P}^{d}$ carrying ${\mathcal B}(\Lambda_{1})$ onto ${\mathcal B}(\Lambda_{2})$. This homeomorphism lifts to 
 an orientation-preserving homeomorphism 
 $h:X_{n}^{p}(\Lambda_{1}) \to X_{n}^{p}(\Lambda_{2})$ such that $h H h^{-1}=H$.
\end{rema}


The following fact was obtained in \cite{HHL23}, as a consequence of the results in \cite{Kon02,MaMo64}.


\begin{theo}[\cite{HHL23}]\label{maintheo2}
(1) The linear group ${\rm Lin}(X^{p}_{n}(\Lambda))$ consists of matrices such that only an element in each row and column is non-zero.
(2) If $(d;p,n) \notin \{(2;2,5),(2;4,3)\}$, then ${\rm Aut}(X^{p}_{n}(\Lambda))={\rm Lin}(X^{p}_{n}(\Lambda))$.
\end{theo}





%%%%%%%%%
\subsubsection{\bf Algebraic equations of all generalized Fermat varieties}
Let $(X,H_{X})$ be a generalized Fermat pair of type $(d;p,n)$ and let $\pi:X \to {\mathbb P}^{d}$ be a 
Galois branched cover, with deck group $H_{X}$, and whose branch locus consists of $(n+1)$ hyperplanes $B_{1},\ldots,B_{n+1}$ which are in general poistion. Let us consider any permutation $\sigma \in {\mathfrak S}_{n+1}$. There is a unique $T \in {\rm PGL}_{d+1}({\mathbb C})$ such that $T(B_{\sigma^{-1}(i)})=\Sigma_{i}$, for $i=1,\ldots,d+2$.
 As the $T$-image of these $(n+1)$ hyperplanes are in general position, there is a unique $\Lambda=\Lambda_{\sigma} \in \Omega_{n,d}$ such that
$T(B_{\sigma^{-1}(d+2+j)})=\Sigma_{d+1+j}(\Lambda)$, for $j=1,\ldots,n-1-d$.

\begin{rema}
The above construction of $T_{\sigma} \in {\rm PGL}_{d+1}({\mathbb C})$, for each $\sigma \in {\mathfrak S}_{n+1}$, induces a one-to-one homomorphism $\Theta:{\mathfrak S}_{n+1} \to  {\rm Aut}(\Omega_{n,d})$. We set ${\mathbb G}_{n,d}=\Theta({\mathfrak S}_{n+1}) \cong {\mathfrak S}_{n+1}$.
\end{rema}


\begin{theo}[\cite{GHL09}, \cite{HHL23}]
If $n \geq d+2$ and $(X,H_{X})$ is a generalized Fermat pair of type $(d;p,n)$, then there is some $\Lambda \in \Omega_{n,d}$ and a biholomorphism $\phi:X \to X_{n}^{p}(\Lambda)$ such that $\phi H_{X} \phi^{-1}=H$. Moreover, $\Lambda_{1}, \Lambda_{2} \in \Omega_{n,d}$ produce isomorphic pairs if and only if they belong to the same ${\mathbb G}_{n,d}$-orbit.
\end{theo}

\begin{rema}
The above result, for $d \geq 2$, may be seen as a consequence of Pardini's classification of abelian branched covers \cite{Par91}, and that of maximal branched abelian covers \cite{AlPa13}. The proof of the case $d=1$ in \cite{GHL09} was obtained from Fuchsian group theory.
\end{rema}



%%%%%%%%%%%%%%%
\subsection{A simple remark on the cohomological information of generalized Fermat varieties}\label{re:cohom}
The fact that $X:=X^{p}_{n}(\Lambda)$ is a complete intersection variety allows us to compute the cohomology groups of the twisting sheaf $\mathcal{O}_{X}(r)$ in a relatively direct way, and in particular, to obtain the following.

\begin{prop}\label{cohomologia}
Let $d \geq 2$, $\Lambda \in \Omega_{n,d}$, $n \geq d+1$, and $X:=X^{p}_{n}(\Lambda)$. 
Set $r_1=(n-d)p-n-1$. Then
\begin{enumerate}[leftmargin=*,align=left]
\item The plurigenera $P_{m}(X)$ of $X$ satisfies
{\small
$$P_{m}(X)= \frac{p^{n-d}((n-d)p-n-1)^{d}}{d!}m^{d}+O(m^{d-1}).$$
}

\item The arithmetic genus $p_{a}(X)$ and the geometric genus $p_{g}(X)$ are given by
{\small
$$p_a(X)=p_g(X)= \left \{ \begin{array}{ccc}
0 & \mbox{if} & r_1<0\\
            \binom{r_1+n}{n} & \mbox{if} & 0\leq r_1<p\\
            \sum_{j\in \Delta_{r_1}}\binom{r_1-\overline{j}+d}{d} & \mbox{if} & r_1\geq p\\
           \end{array} \right .$$
           }

\item If $(n-d)p-n-1=0$, then $X$ is a Calabi-Yau variety.
\item If  $d=2$, then $X$ is a general type surface except for the rational varieties cases $(p,n)\in \{(2,3), (3,3), (2,4)\}$ and the $K3$ varieties $(p,n)\in \{(4,3), (2,5)\}$.

\end{enumerate}
\end{prop}
\begin{proof}
Let $\mathbb{C}[x_1,...,x_{m}]_l$ be  the  homogeneous polynomials of degree $l$.




\begin{enumerate} [label=(\alph*),leftmargin=*,align=left]
\item We first proceed to describe the cohomology groups of the twisting sheaf   $\mathcal{O}_{X}(r), r\in \mathbb{Z}.$
\begin{enumerate}[leftmargin=*,align=left]
 \item[(a1)]  Let  $\Delta_r:=\{(j_1,...,j_{n-d})\in \mathbb{Z}^{n-d}:\; 0\leq j_i\leq p-1, 0\leq i\leq n-d, \;\mbox{and}\; \overline{j}:=j_1+j_2+\cdots j_{n-d}\leq r \}$. Then
{\small
 $$\displaystyle H^{0}(X, \mathcal{O}_{X}(r)):=\left \{  \begin{array}{ccc}
                                             0 & \mbox{if} & r<0\\
                                           \mathbb{C}[x_1,...,x_{n+1}]_r &\mbox{if} & 0\leq r<p \\
                                           \bigoplus_{j\in \Delta_r}Q_j&\mbox{if}  & r\geq p
                                          \end{array} \right .$$
                                          }
where  $ Q_j:=\mathbb{C}[x_1,...,x_{d+1}]_{(r-\overline{j})}x_{d+2}^{j_1} x_{d+3}^{j_2}\cdots x_{n+1}^{j_{n-d}}$, $j:=(j_1,....j_{n-d})$.
\item[(a2)] By Grothendieck's vanishing theorem, 
 $H^{i}(X, \mathcal{O}_{X}(r))=0$  for $i>d$, and $r\in \mathbb{Z}$,



\item[(a3)] and, as $X$ is a complete intersection variety, 
  $H^{i}(X, \mathcal{O}_{X}(r))=0$ for $0<i<d$, and $r\in \mathbb{Z}$
 (see page 231 of \cite{Har77}).
 
 \item[(a4)]  Finally, using the Serre duality, 
     $H^{d}(X, \mathcal{O}_{X}(r))\cong H^{0}(X, \mathcal{O}_{X}(r_1-r))$.

Remember that $\omega_X\cong  \mathcal{O}_X(r_1)$ (see page 188 of \cite{Har77}).

\end{enumerate}


\item With the former, we can calculate the plurigenus of $X$
{\small
$$P_m(X)=\dim_{\mathbb{C}} H^{0}(X,\omega_X^{\otimes m})=\dim_{\mathbb{C}} H^{0}(X,\mathcal{O}_X(r_m))$$
}
where $r_m:=mr_1=m((n-d)p-n-1)$.
\begin{enumerate}
 \item[(b1)]  If  $ (n-d)p-n-1<0$, we obtain that $P_{m}(X)=0$. This implies that the Kodaira dimension of $X$ is  $\kappa(X)=-\infty$.
 \item[(b2)] If  $(n-d)p-n-1=0$, we obtain that $P_{m}(X)=1$. This implies that the Kodaira dimension of $X$ is   $\kappa(X)=0$.
 \item[(b3)] If $(n-d)p-n-1>0$, the canonical sheaf in very ample and
{\small
 \begin{center}
   $P_m(X)=\left \{ \begin{array}{ccc}

            \binom{r_m+n}{n} & \mbox{if} &0\leq  r_m<p\\
             & &\\
             \sum_{j\in \Delta_{r_m}}\binom{r_m-\overline{j}+d}{d} & \mbox{if} & r_m\geq p\\\
           \end{array} \right .
$
 \end{center}
 }
In particular, if $r_m\geq \max\{ p, (n-d)(p-1)\}$, we obtain the assertion (1).
\end{enumerate}
This implies that the Kodaira dimension of  $X$ is  $\kappa(X)=d$.


\item The former also permits us to determine the arithmetic genus and geometric genus of  $X$. As seen from the above,
$p_a(X)=p_g(X)=\dim_{\mathbb{C}}H^{d}(X,\mathcal{O}_X)=\dim_{\mathbb{C}}H^{0}(X,\mathcal{O}_X(r_1)),$
so, we obtain assertion (2).
 \end{enumerate}


\end{proof}







%%%%%%%%%%%%%%%%%%%%%
\subsubsection{\bf Uniqueness of generalized Fermat groups}
If $n=d$, then the generalized Fermat group is not unique (but it is unique up to conjugation). 


\begin{theo}[\cite{HKLP17}]
If $d=1$ and $(n-1)(p-1)>2$, then a generalized Fermat curve of type $(p,n)$ has a unique generalized Fermat group.
\end{theo}

\begin{theo}[\cite{HHL23}]\label{teo2}
Let $d \geq 2$ and $(X,H_{X})$ be a generalized Fermat pair of type $(d;p,n) \notin \{(2;2,5),(2;4,3)\}$. If $\hat{H}$ is a generalized Fermat group of $X$ of some type $(d;\hat{p},\hat{n})$, then $\hat{H}=H_{X}$.
\end{theo}
\begin{proof}
We may assume $X=X_{n}^{p}(\Lambda)$, for some $\Lambda \in \Omega_{n,d}$ and $H_{X}=H$. 

Let $\psi \in \hat{H}$ be an element whose fixed point locus has dimension $d-1$ (i.e., a canonical generator for $\hat{H}$). By Theorem \ref{maintheo2}, $\psi \in {\rm Lin}(X)$ corresponds to a matrix such that only an element in each row and column is non-zero. If such a matrix is not diagonal, then its locus of fixed points in ${\mathbb P}^{n}$ is a linear subspace of codimension at least two; so ${\rm Fix}(\psi) \cap X$ cannot have dimension $d-1$, a contradiction. So, 
$$\psi([x_{1}:\cdots:x_{n+1}])=[\alpha_{1} x_{1}: \cdots : \alpha_{n+1} x_{n+1}].$$

If $[x_{1}: \cdots:x_{n+1}] \in X$, then as $\psi \in {\rm Aut}(X)$, it follows that
\begin{equation}
\left\{\begin{matrix}
\alpha_{1}^{p}x_{1}^{p}+\cdots+\alpha_{d}^{p}x_{d}^{p}+\alpha_{d+1}^{p}x_{d+1}^{p}+\alpha_{d+2}^{p}x_{d+2}^{p}&=&\ 0\\
\lambda_{1,1}\alpha_{1}^{p}x_{1}^{p}+\cdots+\lambda_{1,d}\alpha_{d}^{p}x_{d}^{p}+\alpha_{d+1}^{p}x_{d+1}^{p}+\alpha_{d+3}^{p}x_{d+3}^{p}&=&\ 0\\
\vdots&\vdots&\ \vdots\\
\lambda_{n-d-1,1}\alpha_{1}^{p}x_{1}^{p}+\cdots+\lambda_{n-d-1,d}\alpha_{d}^{p}x_{d}^{p}+\alpha_{d+1}^{p}x_{d+1}^{p}+\alpha_{n+1}^{p}x_{n+1}^{p}&=&\ 0
\end{matrix}\right\}\subset{\mathbb P}^{n}.
\end{equation}

Since 
$x_{1}^{p}+\cdots+x_{d}^{p}+x_{d+1}^{p}+x_{d+2}^{p}= 0,$
we may observe that 
$\alpha_{1}^{p}=\cdots=\alpha_{d+1}^{p}=\alpha_{d+2}^{p}.$

Since, for $i=1,\ldots,n-d-1$,
$\lambda_{i,1}x_{1}^{p}+\cdots+\lambda_{i,d}x_{d}^{p}+x_{d+1}^{p}+x_{d+2+i}^{p}=0,$
we also observe that 
$\alpha_{1}^{p}=\cdots=\alpha_{d+1}^{p}=\alpha_{d+2+i}^{p}.$

All of the above asserts that $\psi \in H$ and that it has a $(d-1)$-dimensional locus of fixed points. So, $\psi$ is a non-trivial power of one of the canonical generators of $H$. 

The above asserts that $\hat{H} \leq H$.
Now, by interchanging the roles of $\hat{H}$ and $H$ in the above, we also obtain that $H \leq \hat{H}$.
\end{proof}





\begin{rema}
The two exceptional cases $(d;p,n) \in \{(2;2,5),(2;4,3)\}$ correspond to the only K3-surfaces among generalized Fermat surfaces. They have infinite group of holomorphic automorphisms, the corresponding linear subgroup has infinite index and it is non-normal. Anyway, inside the linear subgroup of automorphisms there is a unique generalized Fermat group.
\end{rema}

%%%%%%%%%%%%%%%%%%%
%%%%%%%%%%%%%%%%%%%
\subsection{Automorphisms of generalized Fermat varieties}\label{Ssec:Aut}


As a consequence of Theorem \ref{teo2}, is the following fact, which together with Theorem \ref{maintheo2} below, might be used to explicitly compute the full group of automorphismsm of a generalized Fermat variety.


\begin{coro}\label{coro-unico}
Let $d \geq 2$, $p \geq 2$, $n \geq d+1$ be integers and $(d;p,n) \notin \{(2;2,5), (2;4,3)\}$. Let $(X,H)$ be a generalized Fermat pair of type $(d;p,n)$. If $G_{0}$ is the ${\rm PGL}_{d+1}({\mathbb C})$-stabilizer of the $n+1$ branch hyperplanes of $X/H={\mathbb P}^{d}$, then $|{\rm Aut}(X)|= |G_0|p^{n}$ and, if the order of $G_{0}$ is relatively prime with $p$, then 
${\rm Aut}(X) \cong H \rtimes G_{0}$.
\end{coro}
\begin{proof}
We know that $X$ admits a unique generalized Fermat group $H$ of type $(d;p,n)$. 
Let $\pi:X \to {\mathbb P}^{d}$ be a Galois branched covering, with $H$ as its desk group, and let $\{L_{1},\ldots, L_{n+1}\}$ be its set of branch hyperplanes. Let $G_{0}$ be the ${\rm PGL}_{d+1}({\mathbb C})$-stabilizer of these $n+1$ branch hyperplanes.
As $H$ is a normal subgroup of ${\rm Aut}(X)$, it follows the existence of a homomorphism $\theta:{\rm Aut}(X) \to G_{0}$, with kernel $H$.
As $X$ is a universal branched cover, every element $Q$ of $G_{0}$ lifts to a holomorphic automorphism $\widehat{Q}$ of $X$.
Then there is a short exact sequence
$1 \rightarrow H \rightarrow {\rm Aut}(X) \stackrel{\rho}{\rightarrow} G_0 \rightarrow 1.$
In particular,  $|{\rm Aut}(X)|= |G_0|p^{n}$. Also, by the Schur-Zassenhaus theorem \cite{Dummit}, in the case that the order of $G_{0}$ is relatively prime with $p$, then 
${\rm Aut}(X) \cong H \rtimes G_{0}$.
\end{proof}




\begin{coro}
Let $d \geq 2$ and $p \geq 2$ be integers. If
$G_{0}$ be a finite subgroup of ${\rm PGL}_{d+1}({\mathbb C})$, then there exists a generalized Fermat pair $(X,H)$ of type $(d;p,n)$, for some
$n \geq d+1$, such that ${\rm Aut}(X/H) \cong G_{0}$. In fact, for $|G_{0}| \leq d+1$ we may assume $n=d+1$ and, for $|G_{0}| \geq d+2$, we may assume $n=|G_{0}|-1$.
\end{coro}
\begin{proof}
If $|G_{0}| \leq d+1$, then take $n=d+1$  and note that for the classical Fermat hypersurface $F_{p} \subset {\mathbb P}^{n}$ of degree $p$ one has that ${\rm Aut}(F_{p})/H$ contains the permutation group of $d+1$ letters.
Let us assume $|G_{0}| \geq d+2$.
The linear group $G_{0}$ induces a linear action on the space ${\mathbb P}^{d}_{hyper}$ of hyperplanes of ${\mathbb P}^{d}$. As $G_{0}$ is finite, we may find (generically) a point $q\in {\mathbb P}^{d}_{hyper}$ whose $G_{0}$-orbit is a generic set of points. Such an orbit determines a collection of $|G_{0}|$ lines in general position in ${\mathbb P}^{d}$. Let us observe that, by the generic choice, we may even assume the above set of points to have ${\rm PGL}_{d+1}({\mathbb C})$-stabilizer exactly $G_{0}$, so the same situation for our collection of hyperplanes.
Now, the results follow from Corollary \ref{coro-unico}.
\end{proof}







%%%%%%%%%%%%%%%%%%%%%%%%%
\subsection{\bf Fixed points of elements of $H$}
Let us consider a generalized Fermat pair $(X_{p}^{n}(\Lambda),H)$ of type $(d;p,n)$, where $d \geq 2$, and let $\pi:X_{n}^{p}(\Lambda) \to {\mathbb P}^{d}$ be as previously defined in Section \ref{Sec:algebra}.
The branch locus of $\pi$ is the collection ${\mathcal B}(\Lambda)$, the union of the following $n+1$ hyperplanes (in general position)
$$\Sigma_{1},\ldots,\Sigma_{d+2},\Sigma_{d+3}=\Sigma_{d+3}(\Lambda),\ldots,\Sigma_{n+1}=\Sigma_{n+1}(\Lambda).$$

Next, we describe those elements of $H$ acting with fixed points on $X_{n}^{p}(\Lambda)$.

\begin{prop}\label{puntosfijos}
Let $\varphi \in H$ be different from the identity. Then $\varphi$ has fixed points on $X_{n}^{p}(\Lambda)$ if and only if there exist 
$1 \leq j \leq d$, $1 \leq i_{1}<\ldots <i_{j} \leq n+1$, and $1 \leq m_{i_{1}},\ldots, m_{i_{j}} \leq p-1$, such that
$\varphi:=\varphi_{i_{1}}^{m_{_{1}}} \circ \cdots \circ \varphi_{i_{j}}^{m_{i_{j}}}$.
\end{prop}
\begin{proof}
Let $p \in X_{n}^{p}(\Lambda)$ be a fixed point of $\varphi$. Then $\pi(p) \in {\mathcal B}(\Lambda)$. Let $1 \leq i_{1}<\ldots <i_{j} \leq n+1$ a maximal collection of indices so that
$p \in \Sigma_{i_{1}} \cap \cdots \cap \Sigma_{i_{j}}$. As the hyperplanes $\Sigma_{j}$ are in general position, necessarily $j \leq d$. Now, the previous asserts that 
$p \in {\rm Fix}(\varphi_{i_{1}}) \cap \cdots \cap {\rm Fix}(\varphi_{i_{j}})$, so $\varphi \in \langle \varphi_{i_{1}},\ldots,\varphi_{i_{j}}\rangle$. The converse is clear.
\end{proof}


\begin{rema}\label{observafijos}
Let $d \geq 2$, $n \geq d+1$, $p \geq 2$, $\Lambda \in \Omega_{n,d}$, $X^{p}_{n}(\Lambda)$.  
Let us consider an element $\varphi \in H$, different from the identity, acting with fixed points on $X^{p}_{n}(\Lambda)$. 
As seen above, we can write
$\varphi:=\varphi_{1}^{m_{1}} \circ \cdots \circ \varphi_{n+1}^{m_{n+1}} \in H$, where there are $1 \leq j \leq d$ and 
$1 \leq i_{1}<\ldots <i_{j} \leq n+1$ such that (i) $m_{i}=0$ if and only if $i \notin \{i_{1},\ldots,i_{j}\}$ and 
(ii) $m_{i_{1}},\ldots,m_{i_{j}} \in \{1,\ldots,p-1\}$. 
For each $l \in \{0,1,\ldots,p-1\}$, set
$$L_{l}(\varphi):=\{j \in \{1,\ldots, n+1\}: m_{j}=l\},$$
and the (possibly empty) algebraic sets
$$\widetilde{F}_{l}(\varphi)=\{[x_{1}:\cdots:x_{n+1}] \in {\mathbb P}^{n} : x_{i}=0, \;  \forall i \notin L_{l}(\varphi)\}, \;
F_{l}(\varphi):=\widetilde{F}_{l}(\varphi)\cap X^{p}_{n}(\Lambda).$$

The locus of fixed points of $\varphi$ in ${\mathbb P}^{n}$ is the disjoint union of the algebraic sets
$\widetilde{F}_{l}(\varphi)$.

Note that each $\widetilde{F}_{l}(\varphi)$ is:
(i) just a point if $\# L_{l}(\varphi)= 1$, and (ii) a projective linear space of dimension $\# L_{l}(\varphi) -1$ if $\# L_{l}(\varphi)> 1$.
The locus of fixed points of $\varphi$ on $X^{p}_{n}(\Lambda)$ is then given as the disjoint union of the sets $F_{l}(\varphi)=\widetilde{F}_{l}(\varphi) \cap X^{p}_{n}(\Lambda)$.
But on $X^{p}_{n}(\Lambda)$ we cannot have points $[x_{1}:\cdots:x_{n+1}]$ with at least $d+1$ coordinates equal to zero. This fact asserts that for $\# L_{l}(\varphi)\leq n-d$ one has that  $F_{l}(\varphi)=\emptyset$.
Also, for
$\# L_{l}(\varphi) \geq n+1-d$, we obtain that $F_{l}(\varphi) \neq \emptyset$ is a generalized Fermat variety of dimension $\# L_{l}(\varphi)+d-n-1$.

In particular, its number of (non-empty) connected components (if non-empty) equals the number of exponents $l$ appearing in $\varphi$ at least $n+1-d$ times.
\end{rema}



\begin{example}
Let $d \geq 2$, $n \geq d+1$, $p \geq 2$, $\Lambda \in \Omega_{n,d}$, $X:=X^{p}_{n}(\Lambda)$.  
\begin{enumerate}[leftmargin=*,align=left]
\item If $p=2$, and $\varphi \in H \cong {\mathbb Z}_{2}^{n}$, different from the identity.
In this case, we have only two sets to consider, say $\#L_{0}(\varphi)$ and $\#L_{1}(\varphi)$, satisfying that $\#L_{0}(\varphi)+\#L_{1}(\varphi)=n+1$.
By Proposition \ref{observafijos}, $\varphi$ has no fixed points on $X^{2}_{n}(\Lambda)$ if and only if
$$\#L_{0}(\varphi),\#L_{1}(\varphi) \leq n-d.$$

Since, $n+1=\#L_{0}(\varphi)+\#L_{1}(\varphi)\leq (n-d)+(n-d)$, necessarily $n \geq 2d+1$. In other words, if $n \leq 2d$, then $H$ does not have non-trivial elements acting freely.

\item
If $d=2$, and $\varphi \in H$, different from the identity. By Proposition \ref{puntosfijos}, ${\rm Fix}(\varphi) \neq \emptyset$ if and only if there exists some $l \in \{0,1,\ldots,p-1\}$ such that 
$\# L_{l}(\varphi) \geq n-1$. In other words, if and only if $\varphi$ is one of the following elements: $\varphi_{i}^{l}$ or $\varphi_{j}^{s} \circ \varphi_{k}^{r}$, where $l,r,s \in \{1,\ldots,p-1\}$, and $i,j,k \in \{1,\ldots,n+1\}$ with $j \neq k$.



\item
Let us assume $p \geq 2$ is a prime integer. Let $K \cong {\mathbb Z}_{p}^{n-r}$ be a subgroup of $H$ acting freely on $X$. Let $F_{j} \subset X$, $j=1,\ldots, n+1$, be the locus of fixed points of the canonical generator $\varphi_{j}$. As $H$ is an abelian group, each $F_{j}$ is invariant under $K$ and acts freely on it.  Let $S=X/K$ (which is a compact complex manifold of dimension $d$) and $X_{j}=F_{j}/K$ (a connected complex submanifold of $S$). The $(n+1)$ connected sets $X_{j}$ are the locus of fixed points of the induced holomorphic automorphism by $\varphi_{j}$. As each two different $F_{i}$ and $F_{j}$ always intersect transversely, it follows that the same happens for $X_{i}$ and $X_{j}$. As the locus of fixed points of (finite) holomorphic automorphisms is smooth, it follows that different $X_{i}$ and $X_{j}$ are the fixed points of different cyclic groups of $N=H/K \cong {\mathbb Z}_{p}^{r}$. This in particular asserts that $n+1 \leq (p^{r}-1)/(p-1)$. So, for instance, the cases (i) $r=1$ and (ii) $r=2$ and $p=2$, are impossible (note that this is in contrast to the case $p=2$ and $d=1$, where these subgroups exist and are related to hyperelliptic Riemann surfaces).

\item Let $n=p=3$ and $d=2$. In this case, $X$ is just the Fermat hypersurface $\{x_{1}^{3}+x_{2}^{3}+x_{3}^{3}+x_{4}^{3}=0\} \subset {\mathbb P}^{3}$. If
$\varphi=\varphi_{1}\varphi_{2}\varphi_{3}^{2}$, then
$(m_{1},m_{2},m_{3},m_{4})=(1,1,2,0)$ and
$L_{0}(\varphi)=\{4\}, \; L_{1}(\varphi)=\{1,2\}, \; L_{2}(\varphi)=\{3\}$.
The locus of fixed points (in ${\mathbb P}^{3}$) of $\varphi$ is given by
{\small
$$\widetilde{F}_{0}(\varphi) \cup \widetilde{F}_{1}(\varphi) \cup \widetilde{F}_{2}(\varphi)=$$
$$\{[0:0:0:1]\} \cup \{[x_{1}:x_{2}:0:0] \in {\mathbb P}^{3}\} \cup \{[0:0:1:0]\}.$$
}
As the cardinalities of $L_{0}(\varphi)$ and $L_{2}(\varphi)$ are at most equal to $n-d$, these two do not introduce fixed points of $\varphi$ on $X$ (this can be seen also directly). The set $L_{1}(\varphi)$ has cardinality $2 \geq n-d+1$, so it produces a zero-dimensional set of fixed points consisting of the three points $[1:-1:0]$, $[1:\omega_{6}:0]$ and $[1:\omega_{6}^{-1}:0]$, where $\omega_{6}=e^{\pi i/3}$.


\item Let us consider the case $n=d+1$, that is, $X$ is the Fermat hypersurface of degree $p$. Let us consider an element $\varphi \in H$, different from the identity. Let us write
$$\varphi=\varphi_{1}^{m_{1}}\circ \cdots \circ \varphi_{d+1}^{m_{d+1}}, \; 0 \leq m_{i} \leq p-1.$$

By Proposition \ref{puntosfijos}, for $\phi$ to act freely on $X$, necessarilly $1 \leq m_{i} \leq p-1$. Since $\varphi_{1} \circ \cdots \circ \varphi_{d+2}=1$, we also have that, for every $i \in \{1,\ldots,d+1\}$, 
$$\varphi=\varphi_{1}^{m_{1}-m_{i}}\circ \cdots \circ \varphi_{i-1}^{m_{i-1}-m_{i}} \circ \varphi_{i+1}^{m_{i+1}-m_{i}}   \cdots \circ \varphi_{d+1}^{m_{d+1}-m_{i}} \circ \varphi_{d+2}^{-m_{i}}.$$
So, for $\varphi$ to acts freely, we must also have that $m_{j}-m_{i} \nequiv 0 \mod(p)$, for every $i \neq j$.

These conditions ensure that the existence of such $\varphi$ obligates for $p \geq d+2$. Now, if $p \geq d+2$, then we may consider $m_{i}=i$, for $i=1,\ldots,d+1$, and set $K=\langle \varphi \rangle \cong {\mathbb Z}_{p}$. 
Then, $(S=X/K,N=H/K)$ is a ${\mathbb Z}_{p}^{d}$-action of type $(d;p;d+1)$.

\end{enumerate}
\end{example}



%%%%%%%%%%%%%%%%%%%%
%%%%%%%%%%%%%%%%%%%%
\section{${\mathbb Z}_{p}^{m}$-actions of type $(d;p,n)$, $d \geq 2$}
In this section, we assume $d \geq 2$.

%%%%%%%%%%%
\subsection{${\mathbb Z}_{p}^{m}$-actions as quotients of generalized Fermat varieties}
Let us consider a ${\mathbb Z}_{p}^{m}$-action $(S,N)$ of type $(d;p,n)$, and let $A={\rm Aut}(S)$ be the group of holomorphic automorphisms of $S$. 

Let us consider a Galois branched cover $\pi_{N}:S \to {\mathbb P}^{d}$ with deck group $N \cong {\mathbb Z}_{p}^{m}$ and whose branch locus consists of $(n+1)$ hyperplanes in general position. Up to postcomposition with a suitable element of ${\rm PGL}_{d+1}({\mathbb C})$, we may assume this $(n+1)$ hyperplanes to be given by the collection ${\mathcal B}(\Lambda)$, for a suitable $\Lambda \in \Omega_{n,d}$. 

As generalized Fermat varieties of type $(d;p,n)$ are universal (branched) covers of orbifolds with underlying space ${\mathbb P}^{d}$ and branch locus consisting of $(n+1)$ hyperplanes in general position (each one of cone order $p$), we may observe the following fact.

\begin{theo}
There is a subgroup ${\mathbb Z}_{p}^{n-1} \cong K \lhd H$, acting freely on $X_{n}^{p}(\Lambda)$, and a biholomorphism $\phi:S \to X_{n}^{p}(\Lambda)/K$ such that $\phi N \phi^{-1}=H/K$. In particular, (i) $m \leq n$, and (ii) if $m=n$, then $K=\{1\}$.
\end{theo}

As a consequence of the above, we will assume (and this will be in what follows) that $m \leq n-1$.


Let us denote by $\pi_{K}:X_{p}^{n}(\Lambda) \to S$ a Galois covering with deck group $K$.
The fact that $X_{p}^{n}(\Lambda)$ is simply connected ensures that $A$ lifts, under $\pi_{K}$, to a group $Q$ of biholomorphisms of $X_{p}^{n}(\Lambda)$, i.e., there is a short exact sequence
\begin{equation}\label{short1}
1 \to K \to Q \stackrel{\rho}{\to} A \to 1,
\end{equation}
where $\pi_{K} \circ \psi =\rho(\psi) \circ \pi_{K}$.

As $H/K=N \leq A$, it follows that $H \leq Q$. So, if $(d;p,n) \notin \{(2;2,5),(2;4,3)\}$, then the uniqueness of $H$ ensures that $H \lhd Q$, i.e., $N \lhd A$. In particular, the above short exact sequence determines (i) a short exact sequence
\begin{equation}\label{short3}
1 \to N \to A \stackrel{\theta}{\to} L \to 1,
\end{equation}
where $\pi_{N} \circ \psi =\theta(\psi) \circ \pi_{N}$, $L=A/N=Q/H$ is a subgroup of the ${\rm PGL}_{d+1}$-stabilizer of the configuration ${\mathcal B}(\Lambda)$, and (ii) a short exact sequence
\begin{equation}\label{short2}
1 \to H \to Q \stackrel{\eta}{\to} L \to 1,
\end{equation}
where $\pi \circ \psi =\eta(\psi) \circ \pi$. 

In particular, if $(p,|L|)=1$, then (by the Schur-Zassenhaus theorem), $Q \cong H \rtimes L$ and $A \cong K \rtimes L$.



We have proved the following.

\begin{theo}
Let $(S,N)$ be a ${\mathbb Z}_{p}^{m}$-action $(S,N)$ of type $(d;p,n) \notin \{(2;2,5),(2;4,3)\}$ and $d \geq 2$. Then
\begin{enumerate}
\item $N \lhd {\rm Aut}(S)$.
\item Let $\pi:S \to {\mathbb P}^{d}$ be a Galois branched cover with deck group $N$ and with branch locus ${\mathcal B}$ being a collection of $n+1$ hyperplanes in general position.Then, there is a short exact sequence
\begin{equation}
1 \to N \to {\rm Aut}(S) \stackrel{\theta}{\to} L \to 1,
\end{equation}
where $\pi \circ \psi =\theta(\psi) \circ \pi$, and $L$ is a subgroup of the ${\rm PGL}_{d+1}$-stabilizer of ${\mathcal B}$.
\end{enumerate}
\end{theo}



%%%%%%%%%%%%%%%
\subsection{Uniqueness}
As already noticed, a generalized Fermat variety of type $(d;p,n) \notin \{(2;2,5),(2;4,3)\}$ admits a unique generalized Fermat group. The following result states a similar uniqueness result for 
${\mathbb Z}_{p}^{m}$-action $(S,N)$ of type $(d;p,n) \notin \{(2;2,5),(2;4,3)\}$ and $d \geq 2$.

\begin{theo} Let $d \geq 2$ and 
$(S,N)$ be a ${\mathbb Z}_{p}^{m}$-action $(S,N)$ of type $(d;p,n) \notin \{(2;2,5),(2;4,3)\}$. If $(S,M)$ is a ${\mathbb Z}_{q}^{r}$-action of type $(d;q,s)$, then $M=N$.
\end{theo}
\begin{proof}
Assume $S=X_{n}^{p}(\Lambda)/K$. Let $\hat{\psi} \in M \cong {\mathbb Z}_{q}^{r}$ be such that its locus of fixed points has dimension $d-1$. Let us consider a lifting $\psi \in {\rm Aut}(X_{n}^{p}(\Lambda))$ of $\hat{\psi}$. We may take $\psi$ so that its locus of fixed points has dimension $d-1$, so $\psi \in H$ is a non-trivial power of some canonical generator. So, $M \leq N$. Now, by looking at the equations for $H$ and $X_{n}^{p}$, we may observe that the only subgroup $L$ of $N$, for which $(S,L)$ is a ${\mathbb Z}_{p}^{r}$-action, is for $L=N$.
\end{proof}




%%%%%%%%%%%%%%%%%%%
%%%%%%%%%%%%%%%%%%%%%
\section{Freely acting subgroups of $H$}
As previously seen, if $(S,N)$ is a ${\mathbb Z}_{p}^{m}$-action of type $(d;p,n)$, then $(S,N)$ is biholomorphically equivalent to $(X_{n}^{p}(\Lambda)/K,H/K)$, where $\Lambda \in \Omega_{n,d}$ and $K$ is a subgroup of $H$ acting freely on $X_{n}^{p}(\Lambda)$ such that  $H/K \cong {\mathbb Z}_{p}^{m}$. The freely acting condition for $K$ is, by Proposition \ref{puntosfijos}, independent of the choice of $\Lambda$. 




Let us denote by ${\mathcal F}(d;p,n,m)$ the collection of the subgroups $K$ of $H$ such that:
\begin{enumerate}
\item  $H/K \cong {\mathbb Z}_{p}^{m}$, and 
\item $K$ does not contain those  
$\varphi_{i_{1}}^{l_{1}} \varphi_{i_{2}}^{l_{2}} \cdots \varphi_{i_{j}}^{l_{j}}$, where $1 \leq j \leq d$, $l_{j} \in \{1,\ldots, p-1\}$ and $1 \leq i_{1}<\cdots < i_{j} \leq n+1$.
\end{enumerate}

Observe that this collection is invariant under the action of 
${\rm Aut}_{g}(H)$.



\begin{lemm}\label{emes}
If $d \geq 2$ and ${\mathcal F}(d;p,n,m) \neq \emptyset$, then $d \leq m$. Moreover, 
if $m=d=2$, then $p \geq 4$.
\end{lemm}
\begin{proof}
Let $\theta:H \to {\mathbb Z}_{p}^{m}$ be a surjective homomorphism such that $\ker(\theta)=K \in {\mathcal F}(d;p,n,m)$. Let us set $\theta(\varphi_{j})=\phi_{j}$.
As ${\rm Aut}_{g}(H)$ keeps invariant ${\mathcal F}(d;p,n,m)$, up to precomposition of $\theta$ by a suitable element of ${\rm Aut}_{g}(H)$, we may assume that $\theta(H)=\langle \phi_{1},\ldots,\phi_{m}\rangle$. 

As $\varphi_{1} \circ \cdots \circ \varphi_{n+1}=1$, we may observe that 
$$K=\langle \varphi_{1}^{l_{m+1,1}} \circ \cdots \circ \varphi_{m}^{l_{m+1,m}} \varphi_{m+1}^{-1}, \ldots,
\varphi_{1}^{l_{n,1}} \circ \cdots \circ \varphi_{m}^{l_{n,m}} \varphi_{n}^{-1} \rangle.$$ 
So, if $m<d$, then $K$ has elements of $H$ acting with fixed points, a contradiction.

Let us now assume $m=d=2$, $p \in \{2,3\}$, and that there is a surjective homomorphism $\theta:H \to {\mathbb Z}_{p}^{2}$ such that $\varphi_{k}, \varphi_{i}\varphi_{j}^{l} \notin K=\ker(\theta)$, for $l \in \{1,\ldots,p-1\}$. In particular, $\langle \theta(\varphi_{1})=\phi_{1}, \theta(\varphi_{2})=\phi_{2}\rangle = {\mathbb Z}_{p}^{2}$. 
For $j=3,\ldots,n+1$, $\theta(\varphi_{j})=\phi_{1}^{r_{j}}\phi_{2}^{s_{j}}$, where $r_{j},s_{j} \in \{0,\ldots,p-1\}$. Since $\varphi_{j}, \varphi_{1}\varphi_{j},\varphi_{2}\varphi_{j}, \varphi_{1}\varphi_{j}^{p-1},\varphi_{2}\varphi_{j}^{p-1} \notin K$, then $r_{j}=s_{j} \in \{1,2\}$. But, in this situation $\varphi_{3}\varphi_{4}$ or $\varphi_{3}\varphi_{4}^{2} \in K$, a contradiction.
\end{proof}


\subsubsection{\bf Description of elements of ${\mathcal F}(2;p,n,m)$}
Let $K \in {\mathcal F}(2;p,n,m)$. By the definition of ${\mathcal F}(2;p,n,m)$, 
$K$ does not contain those non-trivial elements of the form 
$\varphi_{k}, \varphi_{i} \varphi_{j}^{l}$, where $1 \leq k \leq n+1$, $1 \leq i < j \leq n+1$, and $l \in \{1,\ldots, p-1\}$. 


Let us consider a surjective homomorphism $\theta_{1}:H \to {\mathbb Z}_{p}^{m}$ whose kernel is $K$. 
There is a subset (not unique) of indices $1=i_{1}<i_{2}<\cdots<i_{m} \leq n+1$ such that $\langle \phi_{1}=\theta_{1}(\varphi_{i_{1}}),\ldots,\phi_{m}=\theta_{1}(\varphi_{i_{m}})\rangle ={\mathbb Z}_{p}^{m}$. Let $\Phi \in {\rm Aut}_{g}(H)$ be such that $\Phi^{-1}(\varphi_{j})=\varphi_{i_{j}}$, for $j=1,\ldots,m$. Then $\Phi(K) \in {\mathcal F}(2;p,n,m)$ is the kernel of the surjective homomorphism $\theta=\theta_{1} \circ \Phi^{-1}:H \to {\mathbb Z}_{p}^{m}$. Note that 
$$\theta(\varphi_{j})=\phi_{j}, \; j=1,\ldots,m,$$
$$\theta(\varphi_{i})=\phi_{1}^{r_{i,1}} \cdots \phi_{m}^{r_{i,m}}, \; i=m+1,\ldots, n+1,$$
where the tuples $(r_{i,1},\ldots,r_{i,m}) \in \{0,1,\ldots,p-1\}^{m}$ satisfy the following properties.
\begin{enumerate}
\item ($\varphi_{1} \cdots \varphi_{n+1}=1$)
$$1+r_{m+1,i}+ r_{m+2,i} +\cdots+r_{n+1,i} \equiv 0 \mod(p), \; i=1,\ldots,m.$$

\item ($\varphi_{i}\notin K$, for $i=m+1,\ldots,n+1$)
$$(r_{i,1},\ldots,r_{i,m})  \neq (0,\ldots,0), \; i=m+1,\ldots,n+1.$$

\item ($\varphi_{k} \varphi_{i}^{l} \notin K$, for $k=1,\ldots,m$, $i=m+1,\ldots,n+1$, and $l=1,\ldots,p-1$)

$(r_{i,1},\ldots,r_{i,m})$ cannot have $(m-1)$ of its coordinates equal to zero, for $i=m+1,\ldots,n+1.$


\item ($\varphi_{i} \varphi_{j}^{l} \notin K$, for $m+1 \leq i < j \leq n+1$, and $l=1,\ldots,p-1$)
$$(r_{i,1}+lr_{j,1},\ldots,r_{i,m}+lr_{j,m})  \nequiv (0,\ldots,0) \mod(p), \; m+1 \leq i < j \leq n+1, \; l=1,\ldots,p-1.$$
\end{enumerate}

In this case, 
$$\Phi(K)=\langle \varphi_{1}^{r_{m+1,1}}\cdots \varphi_{m}^{r_{m+1,m}}\varphi_{m+1}^{-1}, \ldots,  \varphi_{1}^{r_{n,1}}\cdots \varphi_{m}^{r_{n,m}}\varphi_{n}^{-1} \rangle.$$

Summarizing the above is the following.

\begin{theo}
Up to ${\rm Aut}_{g}(H)$, the elements of ${\mathcal F}(2;p,n,m)$ are given by the following normalized ones
$$K=\langle \varphi_{1}^{r_{m+1,1}}\cdots \varphi_{m}^{r_{m+1,m}}\varphi_{m+1}^{-1}, \ldots,  \varphi_{1}^{r_{n,1}}\cdots \varphi_{m}^{r_{n,m}}\varphi_{n}^{-1} \rangle,$$
where the exponents $r_{i,j} \in \{0,1,\ldots,p-1\}$ satisfy the conditions (1)-(4) as described above.
\end{theo}

%%%%%%%%%%%%%%%%%%
\subsubsection{The case $d=p=2$}
As already noticed in Lemma \ref{emes}, in this case $m \geq 3$. In the following, we observe that, for $m=3$, necessarily $n=6$.

\begin{prop}\label{propo3}\mbox{}
\begin{enumerate}
\item ${\mathcal F}(2;2,n,3) \neq \emptyset$ if and only if $n=6$.
Moreover, ${\mathcal F}(2;2,6,3)/{\rm Aut}_{g}(H)$ has exactly one element, this one represented by the group $K=\langle \varphi_{1}\varphi_{2}\varphi_{4}, \varphi_{1}\varphi_{3}\varphi_{5},\varphi_{2}\varphi_{3}\varphi_{6}\rangle$.
\item ${\mathcal F}(2;2,n,n-1) \neq \emptyset$, for $n \geq 5$.
\item ${\mathcal F}(2;2,n,n-2) \neq \emptyset$, for $n \geq 6$.
\item ${\mathcal F}(2;2,(m-1)(m+2)/2,m) \neq \emptyset$, for $m \geq 4$ even.
\item ${\mathcal F}(2;2,m(m+1)/2,m) \neq \emptyset$, for $m \geq 3$ odd.
\end{enumerate}
\end{prop}
\begin{proof}
Part (1): we may check by direct inspection that
${\mathcal F}(2;2,4,3)={\mathcal F}(2;2,5,3)=\emptyset$.
Assume ${\mathcal F}(2;2,n,3) \neq \emptyset$, where $n \geq 6$. Up to ${\rm Aut}_{g}(H)$, there is a surjective homomorphism $\theta:H \to {\mathbb Z}_{2}^{3}=\langle \phi_{1},\phi_{2},\phi_{3}\rangle$,
where $\phi_{j}=\theta(\varphi_{j})$, for $j=1,2,3$, and 
$\varphi_{k}, \varphi_{i} \varphi_{j} \notin K=\ker(\theta)$, where $1 \leq k \leq n+1$, and $1 \leq i < j \leq n+1$. 
Let us write, for $j=4,\ldots,n+1$,  $\theta(\varphi_{j})=\phi_{1}^{r_{j}}\phi_{2}^{s_{j}}\phi_{3}^{t_{j}}$, where $r_{j},s_{j},t_{j} \in \{0,1\}$. The condition that $\varphi_{j} \notin K$ is equivalent to have that $(r_{j},s_{j},t_{j}) \neq (0,0,0)$. The condition that $\varphi_{i}\varphi_{j} \notin K$, for $i \in \{1,2,3\}$ and $j\in\{4,\ldots,n+1\}$, is equivalent to have that $(r_{j},s_{j},t_{j}) \neq (1,0,0), (0,1,0),(0,0,1)$. In particular, $(r_{j},s_{j},t_{j}) \in \{(1,1,1), (1,1,0),(0,1,1),(1,0,1)\}$. The condition that $\varphi_{i}\varphi_{j} \notin K$, for $4 \leq i <j \leq n+1$ is equivalent to have that for different indices $4 \leq i<j \leq n+1$, $(r_{i},s_{i},t_{i}) \neq (r_{j},s_{j},t_{j})$. This ensures that $n=6$ and that, up to ${\rm Aut}_{g}(H)$, we may choose $(r_{4},s_{4},t_{4})=(1,1,0)$, $(r_{5},s_{5},t_{5})=(1,0,1)$, $(r_{6},s_{6},t_{6})=(0,1,1)$, and  $(r_{7},s_{7},t_{7})=(1,1,1)$.

Part (2): just consider the surjective homomorphism
$\theta:H \to {\mathbb Z}_{2}^{n-1}=\langle \phi_{1},\ldots,\phi_{n-1}\rangle$, defined by 
$\theta(\varphi_{k})=\phi_{k}, \; k=1,\ldots n-1,$
$\theta(\varphi_{n})=\phi_{i_{1}}\cdots \phi_{i_{l_{1}}},$ and 
$\theta(\varphi_{n+1})=\phi_{j_{1}}\cdots \phi_{j_{l_{2}}},$
where $\{i_{1},\ldots,i_{l_{1}}\}$ and $\{j_{1},\ldots,j_{l_{2}}\}$ is a disjoint partition of $\{1,\ldots,n-1\}$, with $l_{1},l_{2} \geq 2$.

Part (3): just consider the surjective homomorphism
$\theta:H \to {\mathbb Z}_{2}^{n-2}=\langle \phi_{1},\ldots,\phi_{n-2}\rangle$, defined by 
$\theta(\varphi_{k})=\phi_{k}, \; k=1,\ldots n-2,$
$\theta(\varphi_{n-1})=\phi_{i_{1}}\cdots \phi_{i_{l_{1}}},$
$\theta(\varphi_{n})=\phi_{j_{1}}\cdots \phi_{j_{l_{2}}}$ and
$\theta(\varphi_{n+1})=\phi_{k_{1}}\cdots \phi_{k_{l_{3}}},$
where $\{i_{1},\ldots,i_{l_{1}}\}$, $\{j_{1},\ldots,j_{l_{2}}\}$,
and $\{j_{1},\ldots,j_{l_{3}}\}$ is a disjoint partition of $\{1,\ldots,n-2\}$, with $l_{j} \geq 2$.


Part (4): just consider the surjective homomorphism
$\theta:H \to {\mathbb Z}_{2}^{m}=\langle \phi_{1},\ldots,\phi_{m}\rangle$, defined by 
$\theta(\varphi_{k})=\phi_{k}, \; k=1,\ldots m,$
and $\{a_{m+1},\ldots,n+1\}$ are sent to $\{\phi_{1}\phi_{2},\ldots,\phi_{m-1}\phi_{m}\}$ bijectively.

Part (5): just consider the surjective homomorphism
$\theta:H \to {\mathbb Z}_{2}^{m}=\langle \phi_{1},\ldots,\phi_{m}\rangle$, defined by 
$\theta(\varphi_{k})=\phi_{k}, \; k=1,\ldots m,$
and $\{a_{m+1},\ldots,n\}$ are sent to $\{\phi_{1}\phi_{2},\ldots,\phi_{m-1}\phi_{m}\}$ bijectively, and $\theta(\varphi_{n+1})=\phi_{1}\cdots\phi_{m}$.
\end{proof}




\begin{example}
By Proposition \ref{propo3}, for the type ${\mathcal F}(2;2,6,3)/{\rm Aut}_{g}(H)$ has cardinality one. A representative is 
$$K=\langle \varphi_{1}\varphi_{2}\varphi_{4}, \varphi_{1}\varphi_{3}\varphi_{5},\varphi_{2}\varphi_{3}\varphi_{6}\rangle.$$

This provides the $6$-dimensional family 
$$\left\{\left(S_{\Lambda}=X_{6}^{2}(\Lambda)/K, N_{\Lambda}=H/K\right): \Lambda \in \Omega_{6,2}\right\}$$
of ${\mathbb Z}_{2}^{3}$-actions of type $(2;2,6,3)$, all of them topologically conjugated.
Below, we proceed to compute algebraic equations for these pairs $(S_{\Lambda},N_{\Lambda})$.

Let us first consider the affine model $X(\Lambda) \subset {\mathbb C}^{6}$ of $X_{6}^{2}(\Lambda)$ by taking $x_{7}=1$. In this affine model, $K$ is generated by the linear transformations
$$\eta_{1}(x_{1},\ldots,x_{6})=(-x_{1},-x_{2},x_{3},-x_{4},x_{5},x_{6}),$$
$$\eta_{2}(x_{1},\ldots,x_{6})=(-x_{1},x_{2},-x_{3},x_{4},-x_{5},x_{6}),$$
$$\eta_{3}(x_{1},\ldots,x_{6})=(x_{1},-x_{2},-x_{3},x_{4},x_{5},-x_{6}).$$

A set of generators for the invariants ${\mathbb C}[x_{1},\ldots,x_{6}]^{K}$ is
$$u_{1}=x_{1}^{2}, u_{2}=x_{2}^{2}, u_{3}=x_{3}^{2}, u_{4}=x_{4}^{2}, u_{5}=x_{5}^{2}, u_{6}=x_{6}^{2}, u_{7}=x_{1}x_{2}x_{3}, u_{8}=x_{1}x_{4}x_{5},$$
$$u_{9}=x_{2}x_{4}x_{6}, u_{10}=x_{3}x_{5}x_{6}, u_{11}=x_{1}x_{2}x_{5}x_{6}, u_{12}=x_{1}x_{3}x_{4}x_{6}, u_{13}=x_{2}x_{3}x_{4}x_{5}.$$

So, if we consider the map $\Phi:{\mathbb C}^{6} \to {\mathbb C}^{13}$, defined by $\Phi(x_{1},\ldots,x_{6})=(u_{1},\ldots,u_{13})$, then $\Phi(X(\Lambda))$ is isomorphic to the affine model of $S_{\Lambda}$. The image (affine) surface $\Phi(X(\Lambda))$ is defined by the following equalities

$$u_{6}u_{13} =u_{9}u_{10},
u_{5}u_{12} = u_{8}u_{10},
u_{1}u_{2}u_{3} = u_{7}^{2},
u_{5}u_{6}u_{7} = u_{10}u_{11},
u_{4}u_{11} = u_{8}u_{9},
u_{1}u_{2}u_{5}u_{6} = u_{11}^{2},
$$
$$
u_{4}u_{6}u_{7 } =u_{9}u_{12},
u_{1}u_{2}u_{10} = u_{7}u_{11},
u_{4}u_{5}u_{7 }= u_{8}u_{13},
u_{3}u_{11} = u_{7}u_{10},
u_{1}u_{3}u_{4}u_{6} = u_{12}^{2},
u_{3}u_{6}u_{8} = u_{10}u_{12},
$$
$$
u_{3}u_{5}u_{9} = u_{10}u_{13},
u_{3}u_{5}u_{6} = u_{10}^{2},
u_{3}u_{8}u_{9} = u_{12}u_{13},
u_{2}u_{12} = u_{7}u_{9},
u_{1}u_{3}u_{9} = u_{7}u_{12},
u_{2}u_{6}u_{8} = u_{9}u_{11},
$$
$$
u_{2}u_{8}u_{10} = u_{11}u_{13},
u_{2}u_{4}u_{10} = u_{9}u_{13,}
u_{2}u_{4}u_{6} = u_{9}^{2},
u_{1}u_{4}u_{5} = u_{8}^{2},
u_{2}u_{3}u_{8} = u_{7}u_{13},
u_{2}u_{3}u_{4}u_{5} = u_{13}^{2},
$$
$$
u_{1}u_{13} = u_{7}u_{8},
u_{1}u_{4}u_{10} = u_{8}u_{12},
u_{1}u_{5}u_{9} = u_{8}u_{11},
u_{1}u_{9}u_{10} = u_{11}u_{12}
$$
$$
u_{4}=-u_{1}-u_{2}-u_{3},
u_{5}=-\lambda_{1,1}u_{1}-\lambda_{1,2}u_{2}-u_{3},
u_{6}=-\lambda_{2,1}u_{1}-\lambda_{2,2}u_{2}-u_{3},
u_{3}=-\lambda_{3,1}u_{1}-\lambda_{2,2}u_{2}-1.
$$

In this model, the group $N=\langle \phi_{1},\phi_{2},\phi_{3}\rangle$ is given by:
$$\phi_{1}: \left\{\begin{array}{ll}
u_{i} \mapsto -u_{i}, & i=7,8,11,12\\
u_{j} \mapsto u_{j}, & \mbox{otherwise}
\end{array}\right.
$$
$$\phi_{2}: \left\{\begin{array}{ll}
u_{i} \mapsto -u_{i}, & i=7,9,11,13\\
u_{j} \mapsto u_{j}, & \mbox{otherwise}
\end{array}\right.
$$
$$\phi_{3}: \left\{\begin{array}{ll}
u_{i} \mapsto -u_{i}, & i=7,10,12,13\\
u_{j} \mapsto u_{j}, & \mbox{otherwise}
\end{array}\right.
$$

\end{example}



%%%%%%%%%%%%%%%%
\subsection{On topologically equivalence}
Two ${\mathbb Z}_{p}^{m}$-actions $(S_{1},N_{1})$ and $(S_{2},N_{2})$, both of type $(d;p,n)$, are topologically equivalent if there is an orientation-preserving homeomorphism $F:S_{1} \to S_{2}$ such that $F N_{1} F^{-1}=N_{2}$.  
Assume that
$S_{j}=X_{n}^{p}(\Lambda_{j})/K_{j}$, and $N_{j}=H/K_{j}$, where $\Lambda_{j} \in \Omega_{n,d}$ and $K_{j} \in {\mathcal F}(d;p,n,m)$. Then, as $X_{n}^{p}(\Lambda_{j})$ are universal covers, $F$ lifts to an orientation-preserving homeomorphism $\widetilde{F}:X_{n}^{p}(\Lambda_{1}) \to X_{n}^{p}(\Lambda_{2})$ such that $\widetilde{F} K_{1} \widetilde{F}^{-1}=K_{2}$. The homomorhism $\widetilde{F}$ induces, by the conjugation action, an element $\Phi \in {\rm Aut}_{g}(H)$, which satisfies that $\Phi(K_{1})=K_{2}$. We have obtained the following fact.

\begin{prop}
If $K_{1}, K_{2} \in {\mathcal F}(d;p,n,m)$ determine topologically equivalent ${\mathbb Z}_{p}^{m}$-actions of type $(d;p,n)$, then there exists some $\Phi \in {\rm Aut}_{g}(H)$ such that
$K_{2}=\Phi(K_{1})$.
\end{prop}

Now, assume that we have 
$K_{1}, K_{2} \in {\mathcal F}(d;p,n,m)$ such that there is some $\Phi \in {\rm Aut}_{g}(H)$ satisfying 
$K_{2}=\Phi(K_{1})$. Is such $\Phi$ induced by an orientation-preserving homeomorphism? If this is the case, then the above result will state that the number of 
topologically equivalent ${\mathbb Z}_{p}^{m}$-actions of type $(d;p,n)$ is equal to the cardinality of ${\mathcal F}(d;p,n,m)/{\rm Aut}_{g}(H)$.
This is true for $d=1$ \cite{HR}, but it is not clear for $d \geq 2$.





%\begin{theo}
%There is a bijection between ${\mathcal F}(d;p,n,m)/{\rm Aut}_{g}(H)$ and the set of topologically equivalent ${\mathbb Z}_{p}^{m}$-actions of type $(d;p,n)$. 
%\end{theo}







%%%%%%%%%%%%%%%%%
%%%%%%%%%%%%%%%%%%%
\section{On hyperbolicity of ${\mathbb Z}_{p}^{m}$-actions}
Let $S$ be a compact complex manifold of dimension $d \geq 2$.
The manifold $S$ is Kobayashi hyperbolic if its Kobayashi pseudometric is non-degenerate. In \cite{Brody}, Brody observed that $S$ is Kobayashi hyperbolic if and only if 
there is no non-constant holomorphic map $f:{\mathbb C} \to S$.

Assume that $S$ is a projective variety. In \cite{Demailly}, Demailly introduced an algebraic analogue for hyperbolicity. More precisely, $S$ is called 
algebraically hyperbolic if there exists a positive constant $A$ such that the degree of any curve of genus $g$ on $S$ is bounded from above by $A(g-1)$. 
In the same paper, Demailly proved that Kobayashi hyperbolicity implies algebraically hyperbolicity.  By the definition, an algebraically hyperbolic manifold does not contain genus $g \in \{0,1\}$ curves.


In \cite{BKV}, Bogomolov, Kamenova, and Verbitsky proved that, if $S$ is algebraically hyperbolic, then  ${\rm Aut}(S)$ is finite (for the Kobayashi hyperbolic case, this was proved by Kobayashi in \cite{Ko}). 




\medskip

Let us consider a ${\mathbb Z}_{p}^{m}$-action $(S,N)$ of type $(d;p,n)$, where $n \geq d+1$.

\subsection{Case $m=n$ and $(d;p,n) \in \{(2;4,3),(2;2,5)\}$}
If $(d;k,n)=(2;4,3)$, then $S$ corresponds to the classical Fermat hypersurface of degree $4$ in ${\mathbb P}^{3}$ for which ${\rm Lin}(S) \cong {\mathbb Z}_{4}^{3} \rtimes {\mathfrak S}_{4}$ and ${\rm Aut}(S)$ infinite; so $S$ is not algebraically hyperbolic.
If $(d;k,n)=(2;2,5)$, then ${\rm Lin}(S)$ is a finite extension of ${\mathbb Z}_{2}^{5}$ (generically a trivial extension) and ${\rm Aut}(S)$ is infinite by results due to Shioda and Inose in \cite[Thm 5]{Shioda} (in \cite{Vinberg} Vinberg computed it for a particular case). So, again, these surfaces are not algebraically hyperbolic.



\subsection{Case $m=n$ and $(d;p,n) \notin \{(2;4,3),(2;2,5)\}$}
Let us now asume that $(d;p,n) \notin \{(2;4,3),(2;2,5)\}$, where $n \geq d+1$. In this case, we know that $S$ is a compact projective complex manifold of dimension $d$ with ${\rm Aut}(S)$ finite. 
We wonder if, in these cases, $S$ is or is not algebraically hyperbolic.

\subsection{Case $d+1 \leq m \leq n \leq 2d-1$}
In the next result, we observe that, for $n\leq 2d-1$, $S$ cannot be algebraically hyperbolic.



\begin{theo}
If  $(S,N)$ is a ${\mathbb Z}_{p}^{m}$-action of type $(d;p,n)$, where $3 \leq d+1 \leq n$. Then, in the following situations, $S$ is not algebraically hyperbolic.
\begin{enumerate}
\item $n \leq 2d-1$.
\item $n=2d$ and $p \in \{2,3\}$.
\item $n=2d+1$ and $p=2$.
\end{enumerate}
\end{theo}
\begin{proof}
Let $\pi_{N}:S \to {\mathbb P}^{d}$ be a Galois branched covering with deck group $N$, whose branch locus is given by the collection 
${\mathcal B}$, consisiting of the $n+1$ hyperplanes 
$\Sigma_{1}, \ldots,\Sigma_{n+1}$, that are in general position.
By the general position condition, the intersection of the planes $\Sigma_{1},\ldots,\Sigma_{d}$ consists of a unique point 
$\alpha$. 

(1) Let us first consider the case $n \leq 2d-1$.
Now, let us consider the intersection of the $n+1-d$ hyperplanes $\Sigma_{d+1},\ldots,\Sigma_{n+1}$, which is non-empty since $n+1-d \leq d$. Again, by the general position condition, we can find a point $\beta$ in that intersection that does not belong to $\Sigma_{j}$, for $j=1,\ldots,d$.
 Let $L \subset {\mathbb P}^{d}$ the line connecting $\alpha$ with $\beta$. We observe that $L \cap {\mathcal B}(\Lambda)=\{\alpha,\beta\}$. Set $L^{*}=L\setminus\{\alpha,\beta\} \cong {\mathbb C}\setminus\{0\}$. Let $\hat{L}$ be any connected component of $\pi_{N}^{-1}(L^{*})$, which is a Riemann surface that finitely covers $L^{*}$. In this way, inside $S$ we have a genus zero curve (by adding the two missing points to $\hat{L}$), so $S$ cannot be algebraically hyperbolic.
 
(2) Let us now assume that $n=2d$. We proceed similarly as in the previous case, but in this case, we consider the intersection of the $d$ hyperplanes $\Sigma_{d+1},\ldots,\Sigma_{2d}$; which is a point $\beta$. We consider the line $L \subset {\mathbb P}^{d}$ connecting $\alpha$ and $\beta$. In this case, $L$ intersects $\Sigma_{2d+1}$ in a third point $\gamma$. 
Set $L^{*}=L\setminus\{\alpha,\beta,\gamma\} \cong {\mathbb C}\setminus\{0,1\}$. Let $\hat{L}$ be any connected component of $\pi_{N}^{-1}(L^{*})$, which is a punctured Riemann surface. Moreover, $\pi_{N}:\hat{L} \to L^{*}$ is a finite abelian cover of degree $p^{2}$. By adding the missing punctures to $\hat{L}$, we obtain a closed Riemann surface $W$ such that $\pi_{N}:W \to L$ is an abelian covering, with three branch values, each of order $p$. By the Riemann-Hurwitz formula, if $p \in \{2,3\}$, then $W$ has genus $0$ or $1$.
 So, $S$ cannot be algebraically hyperbolic. 
 
 (3) The argument is similar to that in case (2), except that in this case $L$ intersects the branch locus of $\pi_{N}$ in four points. So, we will have an abelian covering $W \to L$, branched at four points, each of order $2$. This again ensures that $W$ has genus one.
\end{proof}



\begin{example}
Let us consider a generalized Fermat variety $X=X_{4}^{2}(\Lambda)$ of type $(2;2,4)$; so $n=2d$ and we are in case (2) of the previous result. In this case, the locus of fixed points $F_{1} \subset X$ of $\varphi_{1}$ has genus one, in particular, $X$ is not algebraically hyperbolic.
\end{example}


\begin{question}
Let $(S,N)$ be a ${\mathbb Z}_{p}^{m}$-action of type $(d;p,n)$, where $d \geq 2$, $n \geq 2d$ and, if $n=2d$, then $p \geq 4$, and if $n=2d+1$, then $p \geq 3$. When is $S$ algebraically hyperbolic?
\end{question}




%%%%%%%%%%%%%%%%%%%%
%%%%%%%%%%%%%%%%%%%%
\section*{\bf Acknowledgements}
The first author would like to thank the {\it Instituto de Matem\'atica} at Universidad de Talca for providing both a challenging and a motivating environment during a visiting position from September to November 2025, where this project started. In particular, to thank Max Leyton and Alvaro Liendo for the fruitful conversations concerning this (and other) mathematical ideas.
%The authors would like to thank the referees for their effort in reviewing this paper and for the provided suggestions, comments, and corrections. 

%\section*{Declarations}

%\begin{itemize}
%\item \textbf{Funding.} This work was partially supported by \textit{Fondo Nacional de Desarrollo Cient\'ifico y Tecnol\'ogico (FONDECYT)}: projects ANID: 1230001, 1220261 and 1221535.\\

%\item \textbf{Conflict of interest.} The authors declare no competing interests.

%\end{itemize}


%%%%%%%%%%%%%%%%%%%%
%%%%%%%%%%%%%%%%%%%%

\begin{thebibliography}{99}
\bibitem{AlPa13}
Alexeev, V. and Pardini, R.
On the existence of ramified abelian covers.
{\it Rend. Semin. Mat. Univ. Politec. Torino} {\bf 71} (2013), 307--315.


%\bibitem{Be84}
%Berry, T. G.
%Infinitesimal deformations of cyclic covers.
%{\it Acta Cient\'{\i}fica Venezolana. Asociaci\'{o}n Venezolana para el Avance de la Ciencia}
%{\bf 35} (1984), 177--183.

\bibitem{BM}
Bochner, S. and Montgomery, D.
Groups on analytic manifolds. 
{\it Ann. of Math.} {\bf 48} (1947), 659--669.

\bibitem{BKV}
Bogomolov, F., Kamenova, L., and Verbitsky, M.
Algebraically hyperbolic manifolds have finite automorphism groups.
{\it Communications in Contemporary Mathematics} {\bf 22}, 1950003 (2020).
 
 
\bibitem{Brody}
Brody, R.
Complex manifolds in hyperbolicity.
{\it Trans. Amer. Math. Soc.} {\bf 235} (1978), 213--219.

\bibitem{Demailly}
Demailly, J.-P. 
Algebraic criteria for Kobayashi hyperbolic projective varieties and jet differentials. 
{\it Proc. Symp. Pure Math.} {\bf 62} (1997), 285--360.
 
\bibitem{Dummit}
Dummit, D. S., and R. M. Foote, R. M.
{\it Abstract Algebra}. Wiley (2003). ISBN 978-0-471-43334-7.

 
 
 
% \bibitem{Eis95}
% Eisenbud, D.
% {\it Commutative algebra}.
% Graduate Texts in Mathematics {\bf 150}, Springer-Verlag, New York, 1995.
 

%\bibitem{GLMV23}
%Gonz\'alez-Aguilera, V., Liendo, A., Montero, P. and Villaflor, R.
%On a Torelli Principle for automorphisms of Klein hypersurfaces. 
%{\it Transactions of the American Mathematical Society} {\bf 377} (2024), 5483--5511.


\bibitem{GHL09}
Gonz\'{a}lez-Diez, G., Hidalgo, R. A. and Leyton, M.
Generalized {F}ermat curves.
{\it Journal of Algebra} {\bf 321} (2009), 1643--1660.


\bibitem{Har74}
R. Hartshorne.
Varieties of small codimension in projective space.
{\it Bull. Amer. Math. Soc.}, 80:1017--1032, 1974.
     
\bibitem{Har77}
Hartshorne, R.
{\it Algebraic geometry}.
Graduate Texts in Mathematics {\bf 52}, Springer-Verlag, New York, 1977.


\bibitem{HHL23}
Hidalgo, R. A., Hughes, H. F. and Leyton-Alvarez, M.
Uniqueness of Generalized Fermat Groups in Positive Characteristic.
{\it Transformation Groups} (2025). https://doi.org/10.1007/s00031-025-09910-6

\bibitem{HKLP17}
Hidalgo, R. A., Kontogeorgis, A., Leyton-\'Alvarez, M. and Paramantzoglou, P.
Automorphisms of generalized {F}ermat curves.
{\it Journal of Pure and Applied Algebra} {\bf 221} (2017), 2312--2337.

\bibitem{HR}
Hidalgo, R. A., and Reyes-Carocca, S.
${\mathbb Z}_{k}^{m}$-actions of signature $(0;k,\stackrel{n+1}{\dots},k)$. In preparation.


\bibitem{Ko}
Kobayashi, S.
Intrinsic distances, measures and geometric function theory. 
{\it Bull. Amer. Math. Soc.} {\bf 82} (1976), 357--416.

%\bibitem{Koizumi}
%Koizumi, S.
%Fields of moduli for polarized Abelian varieties and for curves.
%{\it Nagoya Math. J.} {\bf 48} (1972), 37-55.

\bibitem{Kon02}
Kontogeorgis, A.
Automorphisms of {F}ermat-like varieties.
{\it Manuscripta Mathematica} {\bf 107} (2002), 187--205.

\bibitem{MaMo64}
Matsumura, H. and Monsky, P.
On the automorphisms of hypersurfaces.
{\it Journal of Mathematics of Kyoto University} {\bf 3} (1963/1964), 347--361.


%\bibitem{Matsusaka}
%Matsusaka, T.
%Polarized varieties, field of moduli and generalized Kummer varieties of polarized abelian varieties.
%{\it Amer. J. Math.} {\bf 80} (1958), 45-82.

%\bibitem{Mum74}
%Mumford, D.
%{\it Abelian Varieties}.
%Tata Institute of Fundamental Research Studies in Mathematics {\bf 5}, Oxford University Press, London, 1974.

%\bibitem{OY19}
%Oguiso, K. and Yu, X.
%Automorphism groups of smooth quintic threefolds
%{\it Asian J. Math.} {\bf 23} (2019), 201--256.


\bibitem{Par91}
Pardini, R.
Abelian covers of algebraic varieties.
{\it Journal f\"{u}r die Reine und Angewandte Mathematik} {\bf 417} (1991), 191--213.


%\bibitem{Poo05}
%Poonen, B.
%Varieties without extra automorphisms. {III}. {H}ypersurfaces.{\it Finite Fields Appl.} {\bf 11} (2005), 230--268.


\bibitem{Randell}
Randell, R.
Lattice-isotopic arrangements are topologically isomorphic.
{\it Proc. Amer. Math. Soc.} {\bf 107} (1989), 555--559.



%\bibitem{Se44}
%Segre, B.
%On the quartic surface {$x_1^4+x_2^4+x_3^4+x_4^4=0$}.
%{\it Proceedings of the Cambridge Philosophical Society} {\bf 40} (1944), 121--145.
 
 
\bibitem{Shioda}
T. Shioda and H. Inose.
On singular K3 surfaces.
{\it In Complex analysis and algebraic geometry}, pages 119--136. Iwanami Shoten, Tokyo, 1977.

 
 
 
%\bibitem{Shimura}
%G. Shimura.
%On the field of rationality for an abelian variety. 
%{\it Nagoya Math. J.} {\bf 45} (1972), 167-178. 



%\bibitem{ShiIno}
%Shioda, T. and Inose, H.
%On singular {$K3$} surfaces. In 
%{\it Complex analysis and algebraic geometry}, 119--136. Iwanami Shoten, Tokyo, 1977.

%\bibitem{Te88}
%Terasoma, T.
%Complete intersections of hypersurfaces---the {F}ermat case and the quadric case.
%{\it Japanese Journal of Mathematics. New Series} {\bf 14} (1988), 309--384.

\bibitem{Vinberg}
Vinberg, \`E. B.
The two most algebraic $K3$ surfaces.
{\it Mathematische Annalen} {\bf 265} (1983), 1--21.

\end{thebibliography}

\end{document}



%%%%%%%%%%%%%
\subsection{Topological equivalence}


%%%%%%%%%%%%%%
\subsubsection{Topological equivalence}


\begin{lemm}\label{lema3}
Let $\Lambda_{1}, \Lambda_{2} \in \Omega_{n,d}$.
If $K \in {\mathcal F}(d;p,n,m)$ and $\Phi \in {\rm Aut}_{g}(H)$, then there is an orientation-preserving homeomorphism $F:X_{n}^{p}(\Lambda_{1}) \to X_{n}^{p}(\Lambda_{2})$ such that
$F K F^{-1}=\Phi(K)$.
\end{lemm}
\begin{proof}
The element $\Phi \in {\rm Aut}_{g}(H)$ induces a permutation $\sigma \in {\mathfrak S}_{n+1}$ (given by the permutation of the indices of the elements $\varphi_{i}$'s). Consider an orientation-preserving homeomorphism $F_{1}:{\mathbb P}^{d} \to {\mathbb P}^{d}$ that sends the configuration ${\mathcal B}(\Lambda_{1})$ onto ${\mathcal B}(\Lambda_{2})$ and sends $\Sigma_{j}(\Lambda_{1})$ to $\Sigma_{\sigma(j)}(\Lambda_{2})$, for $j=1,\ldots,n+1$ (in here, for $1 \leq j \leq d+1$, $\Sigma_{j}(\Lambda_{i})=\Sigma_{j}$). Now, lifts $F_{1}$ (under $\pi$) to obtain the desired $F$.
\end{proof}

As a consequence of the above lemma, we observe the following.

\begin{theo}\label{teorema5}
There is a bijection between ${\mathcal F}(d;p,n,m)/{\rm Aut}_{g}(H)$ and the set of topologically equivalent ${\mathbb Z}_{p}^{m}$-actions of type $(d;p,n)$.
\end{theo}


Let us recall that if $(S,N)$ be a ${\mathbb Z}_{p}^{m}$-action of type $(d;p,n)$ and $A={\rm Aut}(S)$, then there is the short exact sequence \eqref{short1}. The associated group $Q$ induces a subgroup $Q^{*} \leq {\rm Aut}_{g}(H)$. Let us denote by $N_{Q}$ the normalizer of $Q^{*}$ in ${\rm Aut}_{g}(H)$. Inside the collection ${\mathcal F}(d;p,n,m)$ we consider the subcollection
${\mathcal F}(d;p,n,m)(Q)$ consisting of those $K$'s which are $Q^{*}$-invariant. The collection ${\mathcal F}(d;p,n,m)(Q)$ is $N_{Q}$-invariant.
A similar argument as for the proof of Lemma \ref{lema3} permits us to observe the following.

\begin{theo} Assume there is some ${\mathbb Z}_{p}^{m}$-action $(S,N)$ of type $(d;p,n)$, and let $A={\rm Aut}(S)$. Then 
there is a bijection between ${\mathcal F}(d;p,n,m)(Q)/N_{Q}$ and the set of topologically equivalent triples $(\hat{S},\hat{N},\hat{A})$, where $(\hat{S},\hat{N})$ runs over all ${\mathbb Z}_{p}^{m}$-actions of type $(d;p,n)$, where $\hat{N} \lhd \hat{A} \leq {\rm Hom}^{+}(\hat{S})$ and $\hat{A}/\hat{N}$ induces the same permutation action in ${\mathfrak S}_{n+1}$ as $A/N$.
\end{theo}




