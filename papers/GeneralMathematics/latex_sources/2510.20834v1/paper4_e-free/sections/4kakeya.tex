\newpage

%=====================================================================
% 4. Robust Kakeya (revised version)
%=====================================================================
\section{Robust~Kakeya}\label{sec:kakeya}

Throughout this section we fix
\[
   r=\lambda^{-2/3},\qquad
   D=\lambda^{1/12},\qquad
   \alpha:=c_{0}\,r\,D^{1/2}=c_{0}\,\lambda^{-5/8},
   \qquad c_{0}>0,\;\lambda\ge 2 .
\]
As before, the \emph{Robust\,Kakeya} block is activated only in the \emph{high angular density regime}
(see below), contributing \(+\tfrac{1}{12}\) in~\(\lambda\) and \(+1\) in~\(D\) to the balance (see \S\ref{subsec:kakeya-exponents}).

%--------------------------------------------------------------------
\subsection{High-density condition}\label{subsec:kakeya-density}
%--------------------------------------------------------------------
Let \(G\subset\{\Theta\}\) be a family of caps of radius \(r\) with centers \(|\xi_\Theta|\sim\lambda\).
Assume the \emph{uniform} density condition
\begin{equation}\label{eq:kakeya-density}
   \min_{\Theta\in G}
   \#\Bigl\{
      \Theta'\in G:\;
      \angle\bigl(\xi_{\Theta},\xi_{\Theta'}\bigr)\le\alpha
   \Bigr\}
   \;>\; c_{*}D .
\end{equation}
In other words, for \emph{each} \(\Theta\in G\) its \(\alpha\)–cap contains \(>c_*D\) other caps (the threshold is attainable,
since \(\alpha/r=D^{1/2}\)). This strengthening relative to the “\(\max\)” form is needed to apply
the Robust–Kakeya block and derive the estimate in \S\ref{subsec:kakeya-robust}: it guarantees high multiplicity
\(M(t,x)\gtrsim D\) on the entire union \(\bigcup_{\Theta\in G}T_\Theta\) up to a thin exceptional set; see §\ref{subsec:kakeya-tubes}–§\ref{subsec:kakeya-robust}. 
See also Appendix~\eqref{app:det}: the partition into \(O(D)\) classes (Lemma~\eqref{lem:AD-colouring})
and packing in angular annuli (Appendix~\eqref{app:A8}).

%--------------------------------------------------------------------
\subsection{Tubes and \emph{average} multiplicity}\label{subsec:kakeya-tubes}
%--------------------------------------------------------------------
Associate to each cap \(\Theta\) the \emph{parabolic tube}
\[
   \mathcal T_{\Theta}
   :=
   \Bigl\{(t,x)\in Q_{\lambda}:\ |x-2t\,\xi_{\Theta}|\le\rho,\ \ |t|\le\tfrac12\lambda^{-3/2}\Bigr\},
   \qquad
   \rho=\lambda^{-1/2},
\]
where the cylinder \(Q_{\lambda}\) is given in \eqref{eq:def-cylinder}. Define the overlap multiplicity by
\[
   M(t,x):=\sum_{\Theta\in G}\mathbf 1_{\mathcal T_{\Theta}}(t,x).
\]

\begin{lemma}[Averaged Robust Kakeya]\label{lem:kakeya-avg}
Assume the high-density condition \emph{\eqref{eq:kakeya-density}}. Then there exists an exceptional set
\(E\subset Q_\lambda\) such that
\[
   |E|\ \lesssim\ \lambda^{-5/8}\sum_{\Theta\in G}\!|\mathcal T_\Theta|,
\]
and for all \((t,x)\in \Bigl(\bigcup_{\Theta\in G}\mathcal T_\Theta\Bigr)\setminus E\) one has
\(M(t,x)\ \ge\ c\,D\) with an absolute constant \(c>0\) (depending only on \(c_0,c_\ast\)).
In particular,
\begin{equation}\label{eq:union-measure}
   \Bigl|\bigcup_{\Theta\in G}\mathcal T_\Theta\Bigr|
   \ \lesssim\ D^{-1}\!\sum_{\Theta\in G}\!|\mathcal T_\Theta|.
\end{equation}
\end{lemma}

\begin{proof}[Idea of the proof]
For fixed \(t\), the centers of the disks \(B\!\bigl(2t\,\xi_\Theta,\rho\bigr)\) for \(\Theta\) within one
\(\alpha\)–cap lie in a cluster of radius \(\lesssim 2|t|\,\lambda\alpha\).
Since \(2|t|\,\lambda\alpha\lesssim \lambda^{-9/8}\ll \rho=\lambda^{-1/2}\), these disks have a large
common \emph{core} in the slice \(t=\mathrm{const}\), through each point of which there pass \(\gtrsim D\) tubes
(i.e. \(M\gtrsim D\)). The boundary layer of the cluster has transverse thickness \(\lesssim 2|t|\,\lambda\alpha\),
and its relative measure \(\lesssim \lambda^{-5/8}\) is uniform for all admissible \(t\).
Integrating in \(t\) yields the required bounds for \(E\) and \eqref{eq:union-measure}.
\end{proof}

\paragraph{Remark on the sign of the phase.}
Throughout §4 we fix the phase \(x\!\cdot\!\xi - t|\xi|^2\) (a “minus” sign in front of \(t|\xi|^2\)),
so that the stationary set \(\nabla_\xi(x\!\cdot\!\xi - t|\xi|^2)=0\) coincides with
\(|x-2t\,\xi|\lesssim\rho\) in the tube definition. With the opposite sign one should read
\(|x+2t\,\xi|\lesssim\rho\) — all estimates transfer verbatim.



%--------------------------------------------------------------------
\subsection{Robust--Kakeya: $L^2$ estimate}\label{subsec:kakeya-robust}
%--------------------------------------------------------------------
Let $G\subset\{\Theta\}$ be a family of caps of radius $r=\lambda^{-2/3}$,
\emph{lying within one fixed $\alpha$–cap} (the dense one selected in Lemma~\ref{lem:kakeya-avg}). 
Then $\#G\sim (\alpha/r)^2\sim D$, and through each point $(t,x)$ there can pass at most one tube from each cap, so
\begin{equation}\label{eq:M-sup}
   \sup_{(t,x)\in Q_\lambda} M(t,x) \ \lesssim\ D,
\end{equation}
where
\[
   g(t,x) := \sum_{\Theta\in G}F_{\Theta}(t,x)\,\mathbf 1_{\mathcal T_\Theta}(t,x),
   \qquad
   M(t,x) := \sum_{\Theta\in G}\mathbf 1_{\mathcal T_\Theta}(t,x).
\]
By convention, set $M^{-1}(t,x):=0$ when $M(t,x)=0$.

For any complex numbers $a_1,\dots,a_m$ one has
$\big|\sum_{j=1}^{m} a_j\big|^{2}\le m\sum_{j=1}^{m}|a_j|^2$.
Applying this at a fixed point $(t,x)$ with $m=M(t,x)$ and
$a_j=F_{\Theta_j}(t,x)\,\mathbf 1_{\mathcal T_{\Theta_j}}(t,x)$, we obtain the pointwise estimate
\begin{equation}\label{eq:pointwise}
   \frac{\mathbf 1_{\{M\ge cD\}}(t,x)}{M(t,x)}\,|g(t,x)|^2
   \ \le\
   \sum_{\Theta\in G}|F_\Theta(t,x)|^2 \,\mathbf 1_{\mathcal T_\Theta}(t,x),
\end{equation}
where $c>0$ is the constant from Lemma~\ref{lem:kakeya-avg}.

\medskip
\noindent\textit{From pointwise to integral estimate.}
Multiply \eqref{eq:pointwise} by $M(t,x)$ and integrate over $(t,x)\in Q_\lambda$:
\[
   \int_{Q_\lambda} \mathbf 1_{\{M\ge cD\}}\,|g|^2
   \ \le\ 
   \sum_{\Theta\in G} \int_{Q_\lambda} M\,|F_\Theta|^2 \mathbf 1_{\mathcal T_\Theta}.
\]
From \eqref{eq:M-sup} we have $M(t,x)\lesssim D$ on all of $Q_\lambda$, hence
\begin{equation}\label{eq:kakeya-factor}
   \|g\cdot \mathbf 1_{\{M\ge cD\}}\|_{L^{2}(Q_{\lambda})}^{2}
   \ \lesssim\
   D\sum_{\Theta\in G}\|F_{\Theta}\,\mathbf 1_{\mathcal T_\Theta}\|_{L^{2}(Q_\lambda)}^{2}
   \ \le\
   D\sum_{\Theta\in G}\|F_{\Theta}\|_{L^{2}(Q_\lambda)}^{2}.
\end{equation}

\paragraph{Remark.}
The left-hand side of \eqref{eq:kakeya-factor} already contains $\mathbf 1_{\{M\ge cD\}}$,
so the contribution of points with $M<cD$ vanishes, and no special accounting of the exceptional set from §\,\ref{subsec:kakeya-tubes} is needed. 
The passage from \eqref{eq:pointwise} to \eqref{eq:kakeya-factor} uses only the bound $\sup M\lesssim D$ for the family inside a single $\alpha$–cap; summation over different $\alpha$–caps in the full configuration is with bounded multiplicity (see Lemmas~\ref{lem:AD-colouring}–\ref{lem:A8}), which does not introduce additional powers of~$\lambda$ or~$D$.

%--------------------------------------------------------------------
\subsection{Exponents in
\texorpdfstring{$\lambda$}{lambda} and \texorpdfstring{$D$}{D}}
\label{subsec:kakeya-exponents}
%--------------------------------------------------------------------
Since \(D=\lambda^{1/12}\), estimate \eqref{eq:kakeya-factor} contributes to the global balance
\[
   +\frac{1}{12}\quad\text{in }\lambda,
   \qquad
   +1\quad\text{in }D.
\]
These exponents are recorded in the table in \S~\ref{tab:global-balance} and in the Introduction.

\begin{remark}[Separation of regimes]
The \emph{robust} Kakeya block applies only under the high angular
density condition \eqref{eq:kakeya-density}. In the “broad” regime (density \(\le c_{*}D\)) this block
is not activated; in those places where the tube-packing estimate
(\S\ref{sec:tube-pack}) is used, the contribution of \emph{Robust\,Kakeya} is omitted — the regimes are
mutually exclusive according to the principle of “maximum gain.”
\end{remark}
