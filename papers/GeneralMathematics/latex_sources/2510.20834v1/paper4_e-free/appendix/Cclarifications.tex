\newpage

%==================================================================
%  Appendix C.  Clarifications and auxiliary material  (updated)
%==================================================================

\section{Clarifications and auxiliary material}%
\label{app:clarifications}

Below we collect explanations on technical details for which, after the updates in
\ref{sec:narrow} and \ref{sec:balance},
only secondary factors changed.
The main body of the proofs is unaffected.

\subsection{$x$–dependence of the amplitude}
\label{app:C1}
When integrating by parts in $x'=(x_2,x_3)$ the amplitude may be of the form
\[
a(t,x,\xi)=\omega(t)\vartheta(\xi)\chi(x/\lambda^{1/2}).
\]
If one of the operators $L_{x'}$ falls on $\chi$,
an additional factor $\lambda^{-1/2}$ appears on top of the baseline gain $(\lambda\alpha)^{-1}=\lambda^{-3/8}$
(see \ref{app:damp-IBP}), i.e. $\|L_{x'}(\chi)\|_\infty\lesssim \lambda^{-5/6}D^{-1/2}$. Therefore the $x$–dependence of the amplitude
does \emph{not worsen} the kernel estimate; in fact, the cumulative exponent in $\lambda$ becomes more negative by $1/2$
under one such hit, and the balance in $D$ remains unchanged.

%------------------------------------------------------------------
\subsection{Partition into $O(D)$ classes}%
\label{app:C2}

The adjacency graph is constructed as before, but now a single cap
may have up to $\asymp D$ neighbors
(since $\alpha/r=D^{1/2}$).
Therefore, instead of a constant six we need
$O(D)$ “colors’’.
This suffices: within each class the normals satisfy
the condition of Lemma \ref{lem:A4},
and the additional $D$ factor is already accounted for
in the \textit{Robust--Kakeya} block; no new losses appear.

%------------------------------------------------------------------
\subsection{Estimate of the operator \texorpdfstring{$L_{x'}$}{Lx'} in the narrow regime}%
\label{app:C3}

In the \emph{narrow} regime (all $\xi_{m}$ lie in one $\alpha$–cluster) 
we do not use a universal lower bound for $|\nabla_{x'}\Phi_6|$, 
since under near-total transverse cancellation of the three differences 
in $x'$ this quantity can be arbitrarily small. 
Instead, all the gain in the narrow regime is extracted 
from time integrations by parts and globalized via the $TT^{*}$ procedure 
(see \S\ref{sec:narrow}).

If $|\nabla_{x'}\Phi_6|$ is small, then by the \emph{transverse dichotomy} 
of \S\ref{subsec:gradients} the corresponding contribution is redirected 
to the \emph{Robust\,Kakeya} block (\S\ref{subsec:kakeya-robust}) 
or to the narrow cascade (\S\ref{sec:narrow}); 
these regimes are used \emph{mutually exclusively} with the kernel block (\S\ref{sec:kernel}). 

Thus in the narrow regime the operator $L_{x'}$ 
is not applied in the kernel block calculations.

%------------------------------------------------------------------
\subsection{Boundary layer
           \texorpdfstring{$|t|\ll\lambda^{-3/2}$}{|t| ≪ λ^{-3/2}}}%
\label{app:C4}

On the layer $|t|\le\lambda^{-3/2}/16$ we do not need to assume
non-overlap of the tubes
\(
  \{(t,x):\,|x-2t\xi_{\Theta}|\le\rho\}
\).
Indeed, for fixed $t$ all $x$–sections
lie inside the ball
\(
  \{|x|\le \rho+2|t|\}\subset\{|x|\le 2\rho\}
\)
(since $2|t|\le \lambda^{-3/2}/8\ll \rho=\lambda^{-1/2}$),
hence the measure of the boundary layer is crudely estimated as
\[
\bigl|\{|t|\le\lambda^{-3/2}/16\}\cap Q_\lambda\bigr|
\;\lesssim\;
\Bigl(\tfrac{\lambda^{-3/2}}{8}\Bigr)\cdot (2\rho)^3
\;\lesssim\; \lambda^{-3}
\;\simeq\; |Q_\lambda|.
\]
In other words, the contribution of this layer is controlled by a constant and
does not affect the exponents in $\lambda$ and $D$.
\smallskip

%------------------------------------------------------------------
\subsection{Tubular estimate in \texorpdfstring{$\mathbb{R}^{4}$}{R$^4$} at the scale \texorpdfstring{$Q_\lambda$}{Q\_\lambda}}%
\label{app:C5}

\begin{lemma}[Version at the scale \texorpdfstring{$Q_\lambda$}{Q\_\lambda}]\label{lem:MG4D}
Let $P(t,x)$ be a polynomial of degree $d \ge 1$, $d \le D^{1/4}$, and
denote $Z(P) := \{(t,x) \in \mathbb{R}^4 : P(t,x) = 0\}$.
With the notation of \S\ref{subsec:skin-setup} introduce the anisotropic mapping
\[
S_\lambda(t,x) := (\lambda^{3/2} t,\ \lambda^{1/2} x), \qquad
\operatorname{dist}_\lambda\bigl((t,x),Z\bigr)
:= \operatorname{dist}\bigl(S_\lambda(t,x),\ S_\lambda(Z)\bigr),
\]
and set
\[
\beta := \frac{c\,D^{-1}}{d}, \qquad
\mathcal{N}_\beta(P)
:= \bigl\{(t,x) \in Q_\lambda : \operatorname{dist}_\lambda\bigl((t,x),Z(P)\bigr) < \beta \bigr\}.
\]
Then there exists an absolute constant $C > 0$ such that
\[
   \bigl|Q_\lambda \cap \mathcal{N}_\beta(P)\bigr|
   \;\le\; C\,D^{-1}\,|Q_\lambda|.
\]
\end{lemma}

\begin{proof}
Let $P_\lambda := P \circ S_\lambda^{-1}$. Then
$S_\lambda\!\bigl(\mathcal{N}_\beta(P)\bigr)$ is a Euclidean
$r$–tube around $Z(P_\lambda)$ of thickness $r = \beta$
in a unit–scale region. By Wongkew’s theorem~\cite{Wongkew2003}
in $\mathbb{R}^4$ the volume of such a tube is
\(\lesssim d\,r\). For $r = c D^{-1} / d$ we get
\[
\bigl| S_\lambda\!\bigl(\mathcal{N}_\beta(P)\bigr) \bigr| \ \lesssim\ D^{-1}.
\]
Returning to the original variables and noting that $S_\lambda$
preserves the volume of $Q_\lambda$ up to an absolute constant,
we obtain the required bound
\(\bigl|Q_\lambda \cap \mathcal{N}_\beta(P)\bigr| \lesssim D^{-1} |Q_\lambda|\).
\end{proof}

\begin{remark}
Previously a sublevel–set formulation of the form $\{|P| < \beta\}$ was used.
To obtain the factor $D^{-1}$ at the scale $Q_\lambda$
it is more convenient to work with the tubular neighborhood of the zero set
in the anisotropic metric and apply Wongkew’s estimate for the volume
of an $r$–tube: $\mathrm{vol}\bigl(\mathcal{N}_r(Z(P))\bigr) \lesssim d\,r$.
Equivalence with the formulation of \S\ref{subsec:skin-setup}
follows after introducing the mapping $S_\lambda$ and choosing
$\beta = c D^{-1} / d$.
\end{remark}

%------------------------------------------------------------------
\subsection{Cumulative effect on the balance}\label{app:C6}

Clarifications \ref{app:C1}–\ref{app:C5} either improve constants or do not change the scales.
\[
  \sigma_\lambda=-\tfrac{2557}{576}\approx -4.44,\qquad \sigma_D=-3
\]
remain unchanged in the statement of Theorem~\ref{thm:main}; the $\varepsilon$–free conclusion persists.
