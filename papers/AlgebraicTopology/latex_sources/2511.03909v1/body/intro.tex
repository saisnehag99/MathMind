\section{Introduction} \label{sec:intro}

The field of topological data analysis (TDA) utilizes invariants from topology
to study the shape of data.
Among the invariants employed are homology and the original
topological invariant: the Euler characteristic \cite{euler1758elementa}.
In practice, a single invariant is not typically robust enough to capture the
shape of data.
For this reason, filtrations (i.e., nested sequences of shapes) are often
constructed from data and topological invariants are then applied.
This allows for significantly more information to be retained, leading to
constructions such as persistence diagrams \cite{edel2002topological}
(resulting from applying homology to filtrations) and Euler characteristic
functions \cite{Serra1982} (resulting from applying the Euler characteristic
to~filtrations).

Topological transforms \cite{Turner2014PHT} further extend the idea of applying
topological invariants to filtrations by constructing a parameterized family of
filtrations and then applying topological invariants.
The starting point for topological transforms is typically a geometric simplicial or cubical
complex~$K$ in $\R^{n}$.
For each direction vector $\vect{s} \in \sph^{n-1}$, a height filtration of~$K$
is constructed and then a topological invariant is applied to each
height filtration.
Applying the Euler characteristic~(EC) to each height filtration results in the
Euler characteristic transform~(ECT) \cite{Turner2014PHT,munch2023invitation},
the focus of this paper.
The ECT is a \emph{complete invariant}: a geometric simplicial complex $K$
in $\R^{n}$ can be reconstructed solely from
its~ECT~\cite{curry2022many,Ghrist2018}.
In other words, no information is lost when passing from the simplicial
complex~$K$ to its ECT.

A useful variant of the ECT is the weighted Euler characteristic transform
(WECT)~\cite{Jiang2020TheWE}.
The WECT extends the ECT to the setting of weighted geometric complexes
in~$\R^{n}$.
Much like the ECT, the WECT is a complete invariant for weighted geometric
complexes~\cite{Jiang2020TheWE}.
Moreover,
the WECT has been employed in a broad range of applications.
In the field of oncology, the WECT has been used to predict survival rates of
patients with glioblastoma multiforme brain tumors~\cite{crawford2020predicting}.
In plant biology, the WECT has been used to classify seed phenotypes
\cite{amezquita2022measuring}.
The WECT has also proven particularly adept at image classification
tasks~\cite{cisewski2023weighted, Jiang2020TheWE}.

Numerous software packages already exist for computing the WECT, including
\demeter~\cite{amezquita2022measuring}, \sinatra
\cite{crawford2020predicting}, \eucalc
\cite{lebovici2024efficientcomputationtopologicalintegral},
and \fasttopology \cite{laky2024}.
%
The packages \demeter and \sinatra processes arbitrary
geometric simplicial and cubical complexes in Euclidean
spaces,
but are not optimized for speed.
%
The packages \fasttopology and \eucalc are optimized for speed,
but only apply to axis-aligned cubical complexes (i.e., grids) in $\R^2$
or~$\R^3$.
%
Two other packages, \cite{roell2023differentiable} and
\cite{saxena2025scalable}, provide vectorized implementations optimal for
Graphics Processing Unit (GPU) hardware.
%
However, these packages compute an approximation of the ECT
known as the \emph{differentiable Euler characteristic transform} (DECT).
%
Furthermore, \cite{saxena2025scalable} is an optimization for the specific
case of unweighted cubical complexes in $\R^2$ or~$\R^3$.
%
As outlined in \cite{roell2023differentiable}, the DECT provides several
advantages for computation in neural network applications.
%
However, our method is focused on exact computation of the WECT in full
generality.

\paragraph{Contributions}

In this work, we provide a fully vectorized framework for computing WECFs and
WECTs.
Our implementation applies in full generality to arbitrary geometric simplicial
and cubical complexes in Euclidean spaces, and supports automatic
gradient computations.
The vectorized operations are highly optimized for fast computation on
modern GPU hardware.
Moreover, our implementation significantly outperforms existing software
packages, including the optimized packages
\eucalc and \fasttopology.
