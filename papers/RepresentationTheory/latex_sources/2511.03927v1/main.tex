\documentclass[10pt]{amsart}
\usepackage{amsmath,amsfonts,amssymb,color,a4,epsfig,graphics}
\usepackage[utf8]{inputenc}
\usepackage[T1]{fontenc}
\usepackage[french,english]{babel}
\usepackage{enumerate, mathrsfs}
\usepackage{tikz}
\usetikzlibrary{positioning, arrows.meta, fit}
\usepackage{pgfplots}
\pgfplotsset{compat=1.18}
\usepackage{graphicx}
\usepackage{caption}
\usepackage[version=4]{mhchem}
\usepackage{pifont}
\usepackage{mathtools}
\usepackage{booktabs}
\usepackage[a4paper, margin=3cm]{geometry}
\usepackage[colorlinks=true, linkcolor=blue, citecolor=red, urlcolor=blue]{hyperref}
% Macros
\def\build#1_#2^#3{\mathrel{\mathop{\kern 0pt#1}\limits_{#2}^{#3}}}
\newcommand\blue[1]{\textcolor{blue}{#1}}
\newcommand\red[1]{\textcolor{red}{#1}}
\newcommand\green[1]{\textcolor{green}{#1}}
\newcommand\brown[1]{\textcolor{brown}{#1}}
\newcommand{\atan}{\mbox{atan}}
\newcommand{\ran}{{\rm ran }}
\newcommand{\T}{{\mathbb{T}}}
\newcommand{\R}{{\mathbb{R}}}
\newcommand{\C}{{\mathbb{C}}}
\newcommand{\Z}{{\mathbb{Z}}}
\newcommand{\N}{{\mathbb{N}}}
\newcommand{\E}{{\mathbb{E}}}
\newcommand{\D}{{\mathbb{D}}}
\newcommand{\1}{{\mathbb{I}}}
\newcommand{\Sb}{{\mathbb{Sc}}}
\newcommand{\A}{\mathcal{A}}
\newcommand{\Bc}{\mathcal{B}}
\newcommand{\Cc}{\mathcal{C}}
\newcommand{\Dc}{\mathcal{D}}
\newcommand{\Ec}{\mathcal{E}}
\newcommand{\F}{\mathcal{F}}
\newcommand{\Gc}{\mathcal{G}}
\newcommand{\Hc}{\mathcal{H}}
\newcommand{\Ic}{\mathcal{I}}
\newcommand{\Jc}{\mathcal{J}}
\newcommand{\Kc}{\mathcal{K}}
\newcommand{\Nc}{\mathcal{N}}
\newcommand{\Sc}{\mathcal{S}}
\newcommand{\Vc}{\mathcal{V}}
\newcommand{\Wc}{\mathcal{W}}
\newcommand{\Uc}{\mathcal{U}}
\newcommand{\Oc}{\mathscr{O}}
\newcommand{\Tc}{\mathcal{T}}
\newcommand{\Dr}{\mathscr{D}}
\newcommand{\Er}{\mathscr{E}}
\newcommand{\Vr}{\mathscr{V}}
\newcommand{\Thr}{\operatorname{Thr}}
\newcommand{\ip}[2]{\big\langle #1,\,#2\big\rangle}
\newcommand{\un}{{\rm \bf {1}}}
\newcommand{\supp}{{\rm supp }\,}
\newcommand{\norm}[1]{\left\|#1\right\|}
\newcommand{\ha}{\frac{1}{2}}

% Theorems
\newtheorem{theorem}{Theorem}[section]
\newtheorem{prop}[theorem]{Proposition}
\newtheorem{lemma}[theorem]{Lemma}
\newtheorem{remark}[theorem]{Remark}
\newtheorem{example}[theorem]{Example}
\newtheorem{definition}[theorem]{Definition}
\newtheorem{cor}[theorem]{Corollary}

\numberwithin{equation}{section}
\setcounter{tocdepth}{1}


\begin{document}

\title[Boundary--Localized Commutators and Cohomology of Shift Algebras]{Boundary-Localized Commutators and Cohomology of Shift Algebras on the Half-Lattice: Structure, Representations, and Extensions}

\author{Nassim Athmouni}
\address{Universit\'e de Gafsa, Facult\'e des Sciences, Campus Universitaire 2112, Gafsa, Tunisie}
\email{nassim.athmouni@fsgf.u-gafsa.tn}

\subjclass[2020]{47B35, 47L60, 17B65, 17B56, 47A53}

\keywords{Boundary-localized Lie algebras, unilateral shift, cohomology, finite-rank commutators, half-lattice, Toeplitz algebra, edge modes, noncommutative geometry, Jacobiator, $L_\infty$-algebras}
\begin{abstract}
We study the boundary-localized Lie algebra generated by the rank-one perturbation \(T = U + \varepsilon E\) of the unilateral shift on \(\ell^2(\mathbb{Z}_{\ge\ 0})\). While the polynomial algebra \(\langle T \rangle\) is abelian, the enlarged algebra \(\mathcal{A} = \mathrm{span}\{U^a E U^b, U^n\}\) exhibits finite-rank commutators confined to a finite neighborhood of the boundary. We construct explicit site-localized 2-cocycles \(\omega_j(X,Y) = \langle e_j, [X,Y] e_j \rangle\) and prove they form a basis of \(H^2(\mathcal{A},\mathbb{C})\). Quantitative bounds and finite-dimensional models confirm a sharp bulk–edge dichotomy. The framework provides a rigorous Lie-algebraic model for edge phenomena in discrete quantum systems—without violating the Jacobi identity.
\end{abstract}
\maketitle
\tableofcontents

\section{Introduction}
\label{sec:introduction}

We analyze the operator-algebraic and cohomological structure induced by a rank-one boundary perturbation of the unilateral shift on the half-lattice $\mathbb{Z}_{\ge\  0}$. The central object of study is the operator
\[
T = U + \varepsilon E,
\]
where $U$ is the unilateral shift on $\ell^2(\mathbb{Z}_{\ge\ 0})$, $E = \langle e_0, \cdot \rangle e_0$ is the orthogonal projection onto the boundary basis vector $e_0$, and $\varepsilon \in \mathbb{C}$ is a complex deformation parameter. While such operators are classical in the theory of Toeplitz algebras and compact perturbations \cite{Dou98, Sim05}, our focus is not on spectral analysis but on the algebraic footprint of spatial localization: how a boundary term reshapes the underlying Lie algebra without altering its foundational axioms.

The polynomial algebra generated by $T$,
\[
\langle T \rangle = \mathrm{span}\{T^n : n \ge 0\},
\]
is abelian, as powers of a single operator always commute in an associative algebra. Thus, no nontrivial Lie brackets arise within $\langle T \rangle$.

However, a rich and nontrivial algebraic structure emerges in the enlarged boundary algebra
\[
\mathcal{A} := \mathrm{span}\{U^a E U^b,\, U^n : a,b,n \ge 0\}.
\]
This algebra is closed under the commutator bracket and supports finite-rank commutators that are strictly localized in a finite neighborhood of the boundary. Since $\mathcal{A} \subset \mathcal{B}(\ell^2)$, the Jacobi identity holds identically; the novelty of our framework lies not in a relaxation of Lie axioms, but in a sharp bulk–edge dichotomy: the bulk remains algebraically trivial, while all nonabelian structure is confined to the edge.

This setting is closely related to recent studies of boundary-localized operators in mathematical physics. The model $T = U + \varepsilon E$ shares structural features with boundary-deformed discrete Laplacians on the half-lattice \cite{Athmouni-half}, where compact boundary corrections preserve bulk spectral properties. Our work complements this spectral perspective by offering a rigorous cohomological characterization of the edge-induced noncommutativity.

\medskip\noindent
Main contributions.
\begin{enumerate}
  \item We construct a countable family of site-localized $2$-cocycles
    \[
    \omega_j(X,Y) = \langle e_j, [X,Y] e_j \rangle, \quad j \in \mathbb{Z}_{\ge\ 0},
    \]
    and prove that their cohomology classes form a basis of the second Chevalley–Eilenberg cohomology group:
    \[
    H^2(\mathcal{A},\mathbb{C}) \cong \bigoplus_{j=0}^\infty \mathbb{C} \cdot [\omega_j].
    \]
    This provides the first explicit, site-resolved description of boundary cohomology in a non-translation-invariant setting.

  \item We establish quantitative bounds for the commutators of powers of $T$: for all $m,n \ge 1$,
    \[
    \operatorname{rank}[T^m,T^n] \le m+n, \quad
    \|[T^m,T^n]\| \le 2\big((1+|\varepsilon|)^{m+n} - 1\big), \quad
    \operatorname{supp}[T^m,T^n] \subset \{0,\dots,m+n-1\}.
    \]

  \item We illustrate the bulk–edge dichotomy through finite-dimensional truncations, explicit matrix computations, and spectral analysis. In particular, Theorem~\ref{thm:essential-spectrum} shows that the essential spectrum of $T$ coincides with that of the pure shift $U$, confirming the preservation of bulk properties under boundary perturbation.
\end{enumerate}

These results bridge several domains:\begin{itemize}\item In operator theory, they extend the classical Toeplitz framework \cite{BotSil06, Nik02} to a cohomological analysis of non-translation-invariant boundary algebras.
  
  \item In Lie algebra cohomology, the cocycles $\{\omega_j\}$ constitute a spatially resolved analogue of central extensions, localized at individual boundary sites rather than arising from global symmetries.

  \item In mathematical physics, the model provides a rigorous operator-algebraic foundation for edge modes in discrete quantum systems with open boundaries \cite{Hats93}, offering a cohomological signature of localized anomalies.
\end{itemize}
\medskip
The remainder of the article is organized as follows.  
Section~\ref{sec:operator-framework} introduces the operator-theoretic setting, defines the boundary algebra $\mathcal{A}$, and proves that all nontrivial commutators are finite-rank and edge-localized.  
Section~\ref{sec:cohomology} constructs the cocycles $\omega_j$ and proves they form a basis of $H^2(\mathcal{A},\mathbb{C})$, alongside quantitative bounds on $[T^m,T^n]$.  
Section~\ref{sec:extensions} discusses central extensions and confirms that the Jacobi identity holds identically, so no $L_\infty$-structure arises.
Section~\ref{sec:applications} presents finite-dimensional models and spectral illustrations of the bulk–edge dichotomy.  
Finally, Section~\ref{sec:conclusion} summarizes the findings and outlines future directions.
\section{Operator-Theoretic Framework for Boundary-Localized Structures}
\label{sec:operator-framework}
\subsection{Banach and Lie Algebraic Frameworks for Boundary-Localized Commutators} \label{sec:boundary-Lie}

We recall standard notions from the theory of Banach algebras and Banach--Lie algebras, and then describe the specific operator-algebraic setting that arises from rank-one boundary perturbations of the unilateral shift. In this context, nontrivial commutators exist but are strictly confined to a finite neighborhood of the boundary; crucially, they do not entail any failure of the Jacobi identity, as all brackets derive from an ambient associative algebra.

\begin{definition}[Banach algebra]\label{def:Banach-alg}
A \emph{Banach algebra} $(\mathcal{A},\|\cdot\|)$ is a Banach space equipped with an associative bilinear multiplication $(x,y) \mapsto xy$ satisfying $\|xy\| \leq \|x\|\,\|y\|$ for all $x,y \in \mathcal{A}$.
\end{definition}

\begin{definition}[Banach--Lie algebra]\label{def:Banach-Lie}
A \emph{Banach--Lie algebra} $(\mathfrak{g},[\cdot,\cdot])$ is a Banach space $\mathfrak{g}$ endowed with a continuous, bilinear, antisymmetric bracket $[\cdot,\cdot]$ that satisfies the Jacobi identity
\[
[X,[Y,Z]] + [Y,[Z,X]] + [Z,[X,Y]] = 0 \quad \text{for all } X,Y,Z \in \mathfrak{g}.
\]
\end{definition}

As is well known, the space $\mathcal{B}(\mathcal{H})$ of bounded operators on a Hilbert space $\mathcal{H}$, equipped with the operator norm and the commutator bracket $[X,Y] := XY - YX$, is a canonical example of a Banach--Lie algebra.

Our object of interest is the subalgebra of $\mathcal{B}(\ell^2(\mathbb{Z}_{\ge 0}))$ generated by the unilateral shift $U$ and the rank-one boundary projector $E = \langle e_0,\cdot\rangle e_0$:

\begin{definition}[Boundary-localized Lie algebra]\label{def:boundary-Lie}
A Banach--Lie algebra $(\mathfrak{g},[\cdot,\cdot])$ is called \emph{boundary-localized} if there exists a decomposition $\mathfrak{g} = \mathfrak{g}_{\mathrm{bulk}} \oplus \mathfrak{g}_{\mathrm{edge}}$ such that:
\begin{enumerate}
    \item $\mathfrak{g}_{\mathrm{bulk}}$ is an abelian ideal,
    \item $[\mathfrak{g}_{\mathrm{edge}},\mathfrak{g}_{\mathrm{edge}}] \subset \mathfrak{g}_{\mathrm{edge}}$,
    \item Every commutator $[X,Y]$ with $X\in\mathfrak{g}_{\mathrm{edge}}$ or $Y\in\mathfrak{g}_{\mathrm{edge}}$ is a finite-rank operator whose support is contained in a finite subset of the boundary.
\end{enumerate}
\end{definition}

In this sense, the algebra $\mathcal{B} = \mathrm{span}\{U^a E U^b,\, U^n : a,b,n \ge 0\}$ (Section~\ref{sec:shift-structure}) is a concrete realization of a boundary-localized Lie algebra: the non-abelian structure is entirely confined to a finite-dimensional defect layer, while the interior remains commutative.

\begin{remark}\label{rmk:no-quasi-Lie}
It is important to emphasize that the bracket on $\mathcal{B}$ is the standard commutator in an associative algebra. Consequently, the Jacobi identity holds identically (Theorem~\ref{thm:jacobi}), and there is no Jacobiator to encode. The term \emph{quasi-Lie algebra}--typically reserved for structures where the Jacobi identity fails in a controlled way---is therefore inappropriate in this setting. The relevant phenomenon is not a violation of Lie axioms, but the \emph{spatial localization} of nontrivial commutators.
\end{remark}

This spatial separation between bulk and edge underlies the cohomological and representation-theoretic features explored in subsequent sections, and provides a rigorous operator-algebraic model for boundary-induced deformations in discrete systems.

\begin{figure}[h!]
\centering
\resizebox{0.85\linewidth}{!}{%
\begin{tikzpicture}[
  font=\small,
  node distance=14mm,
  box/.style={rounded corners, draw, thick, align=center, inner sep=6pt, fill=gray!5},
  ex/.style={rounded corners, draw, align=center, inner sep=4pt, fill=blue!3},
  arr/.style={-{Stealth[length=2.6mm]}, thick},
]
\node[box] (banach) {Banach algebra \\ $\big(\mathcal{A},\|\cdot\|\big)$ associative};
\node[box, below=of banach] (blie) {Banach--Lie algebra \\ $\big(\mathfrak{g},[\cdot,\cdot]\big)$, Jacobi identity holds};
\node[box, below=of blie] (bdrylie) {Boundary-localized Lie algebra \\ Bulk abelian, edge commutators finite-rank and supported near $\partial$};

\draw[arr] (banach) -- node[right]{commutator} (blie);
\draw[arr] (blie) -- node[right]{spatial decomposition} (bdrylie);

\node[ex, right=22mm of blie] (BofH) {$\mathcal{B}(\mathcal{H})$ with $[X,Y]=XY-YX$};
\draw[arr] (blie.east) -- (BofH.west);

\node[ex, right=22mm of bdrylie, align=left] (half) {
  Half-lattice shifts: \\
  $U$ shift, $E=\langle e_0,\cdot\rangle e_0$, \\
  $\mathcal{B} = \mathrm{span}\{U^n,\, U^a E U^b\}$, \\
  $[U^m,E]$ rank $\le 2$, supported on $\{0,m\}$
};
\draw[arr] (bdrylie.east) -- (half.west);

\node[draw, rounded corners, dashed, fit=(BofH)(half), inner sep=5pt,
      label={[align=center, font=\footnotesize]above:Examples}] {};
\end{tikzpicture}%
}
\caption{Refined hierarchy: Banach $\to$ Banach--Lie $\to$ boundary-localized Lie algebra. The term ``quasi-Lie'' is avoided, as no Jacobi identity violation occurs.}
\label{fig:hierarchy}
\end{figure}
\subsection{Algebraic Structure of the Shift Operators on the Half-Lattice}\label{sec:shift-structure}

We analyze the operator-algebraic structure induced by the boundary-deformed unilateral shift
\[
T = U + \varepsilon E,
\]
where $U$ is the unilateral shift on $\ell^2(\mathbb{Z}_{\ge 0})$ and $E = \langle e_0, \cdot \rangle e_0$ is the rank-one projection onto the boundary vector $e_0$.
Two distinct algebras naturally arise:
\begin{enumerate}
    \item the \emph{polynomial algebra} $\langle T \rangle = \mathrm{span}\{T^n : n \ge 0\}$,
    \item the \emph{enlarged boundary algebra} $\mathcal{B} = \mathrm{span}\{U^a E U^b, \, U^n : a,b,n \ge 0\}$.
\end{enumerate}
We clarify the commutative nature of the former and the localized non-commutativity of the latter,
emphasizing that in both cases the Jacobi identity holds identically.

\subsubsection{Operators and basic decomposition}

Let $(e_n)_{n\ge0}$ denote the canonical orthonormal basis of $\ell^2(\mathbb{Z}_{\ge 0})$.

\begin{definition}[Unilateral shift and boundary projector]
The unilateral shift $U$ and the boundary projection $E$ are defined by
\[
U e_n = e_{n+1}, \qquad E f = \langle e_0, f \rangle e_0.
\]
For $\varepsilon \in \mathbb{C}$, the boundary-deformed shift is $T := U + \varepsilon E$.
\end{definition}

\begin{remark}[Boundedness and normality]
The operator $T$ is bounded with $\|T\| \le 1 + |\varepsilon|$. It is normal if and only if $\varepsilon = 0$, since $[U,E] \ne 0$ for $\varepsilon \ne 0$.
\end{remark}

The following fundamental result ensures that all structures under consideration are genuine Lie algebras.

\begin{theorem}[Jacobi identity in associative algebras]\label{thm:jacobi}
Let $\mathcal{A} \subset \mathcal{B}(\mathcal{H})$ be any linear subspace closed under the commutator bracket $[X,Y] = XY - YX$. Then $(\mathcal{A},[\cdot,\cdot])$ is a Lie algebra: the Jacobi identity holds identically.
\end{theorem}

\begin{proof}
This is a standard consequence of associativity: the commutator in any associative algebra satisfies the derivation property $[X,YZ]=[X,Y]Z+Y[X,Z]$, from which the Jacobi identity $[X,[Y,Z]]+[Y,[Z,X]]+[Z,[X,Y]]=0$ follows by direct computation (see, e.g., \cite{Humphreys}).
\end{proof}

\begin{cor}\label{cor:Lie-structure}
Both the polynomial algebra $\langle T \rangle$ and the enlarged algebra $\mathcal{B}$ are Lie subalgebras of $\mathcal{B}(\ell^2(\mathbb{Z}_{\ge 0}))$.
\end{cor}

\subsubsection{Commutativity of the polynomial algebra \texorpdfstring{$\langle T \rangle$}{<T>}}
A key observation is that the algebra generated by a single operator is always commutative.
\begin{prop}[Abelian polynomial algebra]\label{prop:polyT}
For all $m,n \ge 0$, one has $T^m T^n = T^{m+n} = T^n T^m$. Hence
\[
[T^m, T^n] = 0 \quad \text{for all } m,n \ge 0.
\]
In particular, $\langle T \rangle$ is an abelian Lie algebra, regardless of $\varepsilon$.
\end{prop}

\begin{proof}
Powers of a single operator commute in any associative algebra. The claim follows immediately.
\end{proof}

This result has important consequences:  no nontrivial Lie bracket, cocycle, or Jacobiator can arise within $\langle T \rangle$. All cohomological and representation-theoretic phenomena must therefore involve the enlarged algebra $\mathcal{B}$.

\subsubsection{Localization of boundary effects}

Although $\langle T \rangle$ is abelian, its elements differ from the pure shift powers $U^n$ by boundary-localized corrections.

\begin{lemma}[Telescoping identity]\label{lem:telescope}
For all $m \ge 1$,
\[
T^m - U^m = \varepsilon \sum_{j=0}^{m-1} U^{m-1-j} E T^j.
\]
\end{lemma}
\begin{proof}
By induction on $m$. For $m=1$, $T - U = \varepsilon E$, which matches the formula.
Assume the identity holds for $m$. Then
\[
T^{m+1} - U^{m+1} = T(T^m - U^m) + (T - U)U^m
= \varepsilon \sum_{j=0}^{m-1} T U^{m-1-j} E T^j + \varepsilon E U^m.
\]
Since $T = U + \varepsilon E$ and $E U^m$ has range in $\mathrm{span}\{e_0\}$, we may replace $T$ by $U$ in the first term, yielding
\[
= \varepsilon \sum_{j=0}^{m-1} U^{m-j} E T^j + \varepsilon E U^m
= \varepsilon \sum_{j=0}^{m} U^{m-j} E T^j,
\]
which is the desired formula for $m+1$.
\end{proof}
Given the explicit localization of commutators established in Section~\ref{sec:shift-structure} (Propositions~\ref{prop:corner-corner} and~\ref{prop:local-comm}), we now introduce the appropriate abstract framework.


In this sense, the algebra $\mathcal{A}$ analyzed in Section~\ref{sec:shift-structure} is a canonical example of a boundary-localized Lie algebra: the non-abelian structure is entirely confined to a finite-dimensional defect layer, while the bulk remains commutative (Proposition~\ref{prop:polyT}).

\begin{lemma}[Support localization]\label{lem:support}
The operator $T^m - U^m$ has range contained in $\mathrm{span}\{e_0,\dots,e_{m-1}\}$, and therefore has rank at most $m$.
\end{lemma}

\begin{proof}
Each term $U^{m-1-j} E T^j$ maps into $\mathbb{C} e_{m-1-j}$. The claim follows by summation.
\end{proof}

Consequently, for any $f \in \ell^2$, the vectors $T^m f$ and $U^m f$ coincide on the tail $\{n \ge m\}$:
\[
(T^m f)(n) = (U^m f)(n) = f(n+m) \quad \text{for all } n \ge m.
\]

\subsubsection{Non--commutativity in the enlarged algebra $\mathcal{B}$}

Nontrivial commutators appear only when we admit rank-one ``corner'' operators $U^a E U^b = \langle e_b, \cdot \rangle e_a$.

\begin{lemma}\label{lem:UmE}
For $m \ge 1$,
\[
[U^m, E] = \langle e_0, \cdot \rangle e_m - \langle e_m, \cdot \rangle e_0.
\]
This operator has rank at most $2$, is supported on $\{0,m\}$, and satisfies $\|[U^m,E]\| = 1$.
\end{lemma}
\begin{proof}
Recall that $U^a E U^b = \langle e_b, \cdot \rangle e_a$. For any $f \in \ell^2(\mathbb{Z}_{\ge 0})$, compute
\[
(U^a E U^b U^c f)(n) = (U^a E U^b)(U^c f)(n) = \langle e_b, U^c f \rangle e_a(n) = (U^c f)(b) e_a(n) = f(b + c) e_a(n) = \langle e_{b + c}, f \rangle e_a(n).
\]
Similarly,
\[
(U^c U^a E U^b f)(n) = (U^{a + c} E U^b f)(n) = \langle e_b, f \rangle e_{a + c}(n) = \langle e_b, f \rangle e_{a + c}(n).
\]
Thus,
\[
[U^a E U^b, U^c] f = \langle e_{b + c}, f \rangle e_a - \langle e_b, f \rangle e_{a + c},
\]
which is the desired formula.

The rank is at most 2 because the range is contained in $\mathrm{span}\{e_a, e_{a + c}\}$. The support (i.e., the set of basis vectors with nonzero coefficients in the output) is $\{a, a + c\}$.
\end{proof}
\begin{cor}\label{cor:UaEUbUc}
For $a,b,c \ge 0$,
\[
[U^a E U^b, U^c] = \langle e_{b+c}, \cdot \rangle e_a - \langle e_b, \cdot \rangle e_{a+c},
\]
which has rank $\le 2$ and support $\subseteq \{a, a+c\}$.
\end{cor}
\begin{proof}
Recall that $U^a E U^b = \langle e_b, \cdot \rangle e_a$. For any $f \in \ell^2(\mathbb{Z}_{\ge 0})$, compute
\[
(U^a E U^b U^c f)(n) = (U^a E U^b)(U^c f)(n) = \langle e_b, U^c f \rangle e_a(n) = (U^c f)(b) e_a(n) = f(b + c) e_a(n) = \langle e_{b + c}, f \rangle e_a(n).
\]
Similarly,
\[
(U^c U^a E U^b f)(n) = (U^{a + c} E U^b f)(n) = \langle e_b, f \rangle e_{a + c}(n) = \langle e_b, f \rangle e_{a + c}(n).
\]
Thus,
\[
[U^a E U^b, U^c] f = \langle e_{b + c}, f \rangle e_a - \langle e_b, f \rangle e_{a + c},
\]
which is the desired formula.

The rank is at most 2 because the range is contained in $\mathrm{span}\{e_a, e_{a + c}\}$. The support (i.e., the set of basis vectors with nonzero coefficients in the output) is $\{a, a + c\}$.
\end{proof}
\begin{prop}[Commutator of corner operators]\label{prop:corner-corner}
For $a,b,c,d \ge 0$,
\[
[U^a E U^b, U^c E U^d] = \delta_{b,c} \, U^a E U^d - \delta_{d,a} \, U^c E U^b.
\]
Thus $[U^a E U^b, U^c E U^d]$ has rank at most $2$ and range contained in $\mathrm{span}\{e_a, e_c\}$.
\end{prop}
\begin{proof}
For any $f \in \ell^2(\mathbb{Z}_{\ge 0})$, compute the action of the product operators:
\[
(U^a E U^b)(U^c E U^d f) = (U^a E U^b) \big( \langle e_d, f \rangle e_c \big) = \langle e_d, f \rangle (U^a E) e_c = \langle e_d, f \rangle \langle e_b, e_c \rangle e_a = \delta_{b,c} \langle e_d, f \rangle e_a = \delta_{b,c} (U^a E U^d) f.
\]
Similarly,
\[
(U^c E U^d)(U^a E U^b f) = \delta_{d,a} (U^c E U^b) f.
\]
Subtracting these two expressions gives
\[
[U^a E U^b, U^c E U^d] f = \big( \delta_{b,c} U^a E U^d - \delta_{d,a} U^c E U^b \big) f,
\]
which proves the identity.

Each term $U^x E U^y$ is a rank-one operator with range $\mathbb{C} e_x$. Therefore, the commutator is a linear combination of at most two rank-one operators, so its rank is at most 2. Its range is contained in $\mathrm{span}\{e_a, e_c\}$, as claimed.
\end{proof}
These results confirm that  all nontrivial commutators in $\mathcal{B}$ are finite-rank and spatially localized near the boundary, while the bulk remains abelian.



\subsubsection{Invariant subspaces and spectral vectors}

We now examine the action of $T$ on natural subspaces and its (generalized) eigenvectors.

\begin{lemma}[Generalized eigenvectors]\label{lem:eigen}
For $\lambda \in \mathbb{C}$, define $f_\lambda(n) = \lambda^n$. Then
\[
(T f_\lambda)(n) =
\begin{cases}
(\lambda + \varepsilon) f_\lambda(0) & \text{if } n = 0, \\
\lambda f_\lambda(n) & \text{if } n \ge 1.
\end{cases}
\]
Thus $f_\lambda$ is a genuine eigenvector iff $\varepsilon = 0$; otherwise, it is a generalized eigenvector with a boundary defect.
\end{lemma}
\begin{proof}
For $\lambda \in \mathbb{C}$, define $f_\lambda(n) = \lambda^n$ for $n \geq 0$. Recall that $T = U + \varepsilon E$, so for any $n \geq 0$,
\[
(T f_\lambda)(n) = (U f_\lambda)(n) + \varepsilon (E f_\lambda)(n).
\]

- For $n \geq 1$, $(U f_\lambda)(n) = f_\lambda(n + 1) = \lambda^{n + 1} = \lambda f_\lambda(n)$, and $(E f_\lambda)(n) = 0$ (since $E$ is supported only at $n = 0$). Hence,
  \[
  (T f_\lambda)(n) = \lambda f_\lambda(n).
  \]

- For $n = 0$, $(U f_\lambda)(0) = f_\lambda(1) = \lambda$, and $(E f_\lambda)(0) = f_\lambda(0) = 1$. Therefore,
  \[
  (T f_\lambda)(0) = \lambda + \varepsilon = (\lambda + \varepsilon) f_\lambda(0),
  \]
  since $f_\lambda(0) = 1$.

Thus,
\[
(T f_\lambda)(n) =
\begin{cases}
(\lambda + \varepsilon) f_\lambda(0) & \text{if } n = 0, \\
\lambda f_\lambda(n) & \text{if } n \ge 1.
\end{cases}
\]

This shows that $f_\lambda$ is a genuine eigenvector if and only if $(T f_\lambda)(0) = \lambda f_\lambda(0)$, which requires $\varepsilon = 0$. For $\varepsilon \neq 0$, $f_\lambda$ satisfies the eigenvalue equation everywhere except at the boundary $n = 0$, so it is a \emph{generalized eigenvector with a boundary defect}.
\end{proof}
\begin{prop}[Failure of boundary ideal invariance]
For $k \ge 0$, let $I_k = \{f \in \ell^2 : f(0) = \cdots = f(k) = 0\}$. Then $I_k$ is not invariant under $T$, nor under $\langle T \rangle$.
\end{prop}

\begin{proof}
Take $f = e_{k+1} \in I_k$. Then $(T f)(k) = f(k+1) = 1 \ne 0$, so $Tf \notin I_k$.
\end{proof}

This illustrates that the boundary perturbation $\varepsilon E$ breaks the natural filtration of $\ell^2$ by vanishing at initial sites.


In summary:\begin{itemize}
\item $\langle T \rangle$ is abelian and carries no nontrivial Lie structure.
    \item The enlarged algebra $\mathcal{B}$ is a genuine (non-abelian) Lie algebra, but its non-commutativity is strictly confined to a finite boundary layer.
    \item  No violation of the Jacobi identity occurs in either algebra. 
    \item The meaningful algebraic phenomena-finite-rank commutators, boundary-supported cocycles--live in $\mathcal{B}$, not in $\langle T \rangle$.
\end{itemize}
These observations set the stage for the cohomological analysis in Section~\ref{sec:cohomology}, where the boundary-localized commutators give rise to a countable family of nontrivial $2$-cocycles.

\section{Cohomology and Deformations}
\label{sec:cohomology}

In this section we investigate the cohomological structure induced by the boundary-localized commutators in the enlarged algebra $\mathcal{B} = \mathrm{span}\{U^a E U^b,\, U^n : a,b,n \ge 0\}$.  
The boundary-supported 2-cocycles constructed below provide a meaningful cohomological signature of the edge deformation, analogous in spirit—but not in structure—to the Virasoro cocycle.

\subsection{Boundary--supported 2-cocycles}

We begin by defining natural linear functionals supported on individual lattice sites and the associated Chevalley--Eilenberg 2-cochains.

\begin{definition}[Site evaluation functionals and diagonal cocycles]
For each $j \in \mathbb{Z}_{\ge 0}$, define the linear functional
\[
\varphi_j : \mathcal{B}(\ell^2) \to \mathbb{C}, \qquad
\varphi_j(A) := \langle e_j, A e_j \rangle.
\]
For a Lie subalgebra $\mathfrak{g} \subset \mathcal{B}(\ell^2)$, the associated bilinear form
\[
\omega_j : \mathfrak{g} \times \mathfrak{g} \to \mathbb{C}, \qquad
\omega_j(X,Y) := \varphi_j([X,Y]) = \langle e_j, [X,Y] e_j \rangle,
\]
is called the \emph{$j$-th boundary cocycle}.
\end{definition}

\begin{prop}[Cocycle property]\label{prop:2cocycle-corrected}
For any Lie subalgebra $\mathfrak{g} \subset \mathcal{B}(\ell^2)$, the form $\omega_j$ is a Chevalley--Eilenberg $2$-cocycle, i.e.
\[
d\omega_j(X,Y,Z) = \omega_j([X,Y],Z) + \omega_j([Y,Z],X) + \omega_j([Z,X],Y) = 0,
\]
for all $X,Y,Z \in \mathfrak{g}$.
\end{prop}

\begin{proof}
Since the bracket on $\mathfrak{g}$ is the commutator in an associative algebra, the Jacobi identity holds:
\[
[[X,Y],Z] + [[Y,Z],X] + [[Z,X],Y] = 0.
\]
Applying the linear functional $\varphi_j$ yields the vanishing of $d\omega_j$.
\end{proof}

\begin{remark}[Triviality on the polynomial algebra]
On the subalgebra $\langle T \rangle = \mathrm{span}\{T^n : n \ge 0\}$, we have $[T^m,T^n] = 0$ for all $m,n$, so $\omega_j \equiv 0$ on $\langle T \rangle$. Thus the cocycles detect **only** the non-abelian part of the enlarged algebra $\mathcal{B}$.
\end{remark}

\begin{example}[Non-triviality on $\mathcal{B}$]\label{ex:nontrivial}
Take $X = U E$ and $Y = E U$. Then
\[
[X,Y] = U E E U - E U U E = U E U - E U^2 E = U^1 E U^1 - 0,
\]
and $\omega_1(X,Y) = \langle e_1, U E U e_1 \rangle = \langle e_1, U E e_0 \rangle = \langle e_1, U e_0 \rangle = \langle e_1, e_1 \rangle = 1 \ne 0$.
Thus $\omega_1$ is non-trivial on $\mathcal{B}$.
\end{example}

\begin{lemma}[Diagonal reduction]\label{lem:diag-reduction}
Let $\psi(A) = \sum_{p,q\in F} c_{pq} \langle e_p, A e_q \rangle$ be a linear functional supported on a finite set $F \subset \mathbb{Z}_{\ge 0}$. Then the cocycle $\omega_\psi(X,Y) := \psi([X,Y])$ is cohomologous to $\sum_{j \in F} c_{jj} \omega_j$.
\end{lemma}

\begin{proof}
Define the 1-cochain $\eta(X) = \sum_{p,q} c_{pq} \langle e_p, X e_q \rangle$. Then
\[
(d\eta)(X,Y) = \eta([X,Y]) = \sum_{p,q} c_{pq} \langle e_p, [X,Y] e_q \rangle = \omega_\psi(X,Y).
\]
The off--diagonal terms $p \ne q$ contribute coboundaries and vanish in cohomology, leaving only the diagonal part.
\end{proof}

\subsection{Basis of \texorpdfstring{$H^2$}{H^2} and quantitative bounds}

Using the localization properties of $[T^m,T^n]$, we obtain a concrete description of the second cohomology.

\begin{prop}[Localization of non-commutativity]\label{prop:local-comm}
For every $m \geq 1$, the difference
\[
T^m - U^m = \varepsilon \sum_{j=0}^{m-1} U^{m-1-j} E T^j
\]
is of rank at most $m$ and its image is contained in
$\mathrm{span}\{e_0,\dots,e_{m-1}\}$.
In particular, the family $\{T^m\}_{m\geq1}$ is commutative:
\[
[T^m,T^n]=0 \quad \text{for all } m,n \geq 1.
\]
\end{prop}

\begin{proof}
The telescoping identity (Lemma~\ref{lem:telescope}) gives
\[
T^m - U^m = \varepsilon \sum_{j=0}^{m-1} U^{m-1-j} E T^j.
\]
Since $E T^j f = \langle e_0, T^j f \rangle e_0$ for any $f$, each term $U^{m-1-j} E T^j f$ is a scalar multiple of $e_{m-1-j}$. Hence the image of $T^m - U^m$ lies in $\mathrm{span}\{e_0,\dots,e_{m-1}\}$, and its rank is at most $m$.

For the commutativity, observe that $T$ belongs to the associative algebra $\mathcal{B}(\ell^2(\mathbb{Z}_{\ge 0}))$. Therefore, for all $m,n \geq 1$,
\[
T^m T^n = T^{m+n} = T^n T^m,
\]
which implies $[T^m, T^n] = 0$.
\end{proof}

\begin{lemma}[Separating functionals for boundary cocycles]\label{lem:separating-pairs}
For each $j \in \mathbb{Z}_{\ge 0}$, define $X_j := U^{j} E$ and $Y_j := E U^{j}$. Then
\[
\omega_k(X_j, Y_j) = \begin{cases}
1 & \text{if } k = j, \\
0 & \text{if } k \neq j.
\end{cases}
\]
\end{lemma}

\begin{proof}
We compute $[X_j, Y_j] = U^{j} E E U^{j} - E U^{j} U^{j} E = U^{j} E U^{j}$, since $E^2 = E$ and $U^{2j}E = 0$. Then
\[
\omega_k(X_j, Y_j) = \langle e_k, U^{j} E U^{j} e_k \rangle = \langle e_k, \langle e_j, e_k \rangle e_j \rangle = \delta_{k,j}.
\]
\end{proof}

\begin{theorem}[Basis of $H^2$ on the boundary]\label{thm:H2-basis}
Let $\mathcal{B}_K = \mathrm{span}\{U^a E U^b : a+b \le K\} \subset \mathcal{B}$. Then for $\varepsilon \ne 0$, the cohomology classes $[\omega_0], \dots, [\omega_K]$ are linearly independent in $H^2(\mathcal{B}_K, \mathbb{C})$. Consequently,
\[
H^2(\mathcal{B}, \mathbb{C}) \cong \bigoplus_{j=0}^\infty \mathbb{C} \cdot [\omega_j].
\]
\end{theorem}

\begin{proof}
Suppose $\sum_{j=0}^K \alpha_j \omega_j = 0$ in cohomology. Then for all $X,Y \in \mathcal{B}_K$, $\sum_j \alpha_j \langle e_j, [X,Y] e_j \rangle = 0$. By Lemma~\ref{lem:separating-pairs}, applying this to $(X_j,Y_j)$ yields $\alpha_j = 0$ for all $j$. Linear independence follows. The direct sum decomposition is obtained by taking the inductive limit $K \to \infty$.
 Moreover, by Propositions~\ref{prop:local-comm} and \ref{prop:bounds}, every commutator $[X,Y]$ with $X,Y \in \mathcal{B}$ has finite support in $\{0,\dots,N\}$ for some $N$. Hence any continuous $2$-cocycle $\omega$ is determined by finitely many diagonal evaluations $\varphi_j([X,Y]) = \langle e_j,[X,Y]e_j\rangle$, and therefore cohomologous to a finite linear combination of the $\omega_j$. The full decomposition follows by taking the inductive limit $N \to \infty$.
\end{proof}

\begin{remark}[Analogy and distinction with Virasoro]
Unlike the Virasoro cocycle, which is translation-invariant and arises from a global central extension, our cocycles $\omega_j$ are site-localized and reflect the discrete, edge-boundary structure of the half-lattice. They do not correspond to a central extension of a globally non-abelian Lie algebra, but rather to a family of local anomalies in an otherwise abelian setting.
\end{remark}

\subsection{Quantitative bounds on commutators}

We now derive explicit norm and rank estimates for $[T^m,T^n]$, confirming its finite-rank, boundary-localized nature.

\begin{prop}[Norm and rank bounds]\label{prop:bounds}
For all $m,n \ge 1$,
\begin{enumerate}
    \item $\operatorname{rank}([T^m,T^n]) \le m + n$,
    \item $\|[T^m,T^n]\| \le 2\big( (1+|\varepsilon|)^{m+n} - 1 \big)$,
    \item $[T^m,T^n]$ is supported in the subspace $\mathrm{span}\{e_0,\dots,e_{m+n-1}\}$.
\end{enumerate}
\end{prop}

\begin{proof}
Write $T^r = U^r + \Delta_r$, where $\Delta_r = \varepsilon \sum_{j=0}^{r-1} U^{r-1-j} E T^j$. Then
\[
[T^m,T^n] = [U^m, \Delta_n] + [\Delta_m, U^n] + [\Delta_m, \Delta_n].
\]
Each $\Delta_r$ has rank $\le r$ and norm $\le (1+|\varepsilon|)^r - 1$. The claims follow from subadditivity of rank and the triangle inequality for the operator norm.
\end{proof}

\begin{theorem}[Essential spectrum preservation]\label{thm:essential-spectrum}
For $\varepsilon \in \mathbb{C}$, the essential spectrum of $T = U + \varepsilon E$ on $\ell^2(\mathbb{Z}_{\geq 0})$ coincides with that of the pure shift $U$:
\[
\sigma_{\mathrm{ess}}(T) = \sigma_{\mathrm{ess}}(U) = \overline{\mathbb{D}} = \{z \in \mathbb{C} : |z| \leq 1\}.
\]
Any eigenvalue $\lambda \in \sigma_{\mathrm{p}}(T)$ must satisfy $|\lambda| < 1$ and arises from the boundary defect.
\end{theorem}

\begin{proof}
Since $E$ is rank-one (hence compact), Weyl's theorem on the invariance of the essential spectrum under compact perturbations implies $\sigma_{\mathrm{ess}}(T) = \sigma_{\mathrm{ess}}(U)$. For the unilateral shift $U$, it is classical that $\sigma_{\mathrm{ess}}(U) = \overline{\mathbb{D}}$ (see, e.g., \cite{Dou98}). The point spectrum, if nonempty, consists of isolated eigenvalues of finite multiplicity inside the open unit disk, as any eigenvalue must satisfy the boundary defect equation from Lemma~\ref{lem:eigen}.
\end{proof}

\begin{remark}[First--order expansion]\label{rem:first-order}
At first order in $\varepsilon$, one finds
\[
[T^m,T^n] = \varepsilon \left( \sum_{j=0}^{n-1} [U^m, U^{n-1-j} E U^j] + \sum_{i=0}^{m-1} [U^{m-1-i} E U^i, U^n] \right) + O(\varepsilon^2),
\]
which is a finite linear combination of rank-one operators $U^a E U^b$, confirming the boundary-localized nature of the deformation.
\end{remark}
\begin{remark}[First-order norm estimate]
For sufficiently small $|\varepsilon|$, the expansion in Remark~\ref{rem:first-order} implies
\[
\|[T^m,T^n]\| = O(|\varepsilon|(m+n)) \quad \text{as } \varepsilon \to 0,
\]
which is significantly sharper than the exponential bound in Proposition~\ref{prop:bounds}.
\end{remark}
\subsection{Absence of nontrivial $L_\infty$-structure}
Since the bracket on $\mathcal{A} \subset \mathcal{B}(\ell^2)$ is the commutator in an associative algebra, the Jacobi identity holds identically (Theorem~\ref{thm:jacobi}), and the Jacobiator vanishes. Consequently, the only compatible $L_\infty$-structure is the strict Lie algebra with $l_2 = [\cdot,\cdot]$ and $l_k = 0$ for $k \geq 3$. No higher homotopy brackets arise in this setting.
\begin{remark}
If one wishes to construct a genuine $L_\infty$-algebra with nontrivial $l_3$, one must  define a new trilinear operation  that is not the Jacobiator of an associative commutator. Such a construction would require moving beyond the operator-theoretic framework presented here, for example to:\begin{itemize}
\item A formal deformation setting with non-associative products
    \item A geometric framework with connection curvature on non-associative bundles
    \item An abstract algebraic approach with controlled Jacobi violations\end{itemize}
None of these appear in the present analysis of boundary-localized operators.
\end{remark}
\section{Infinite Extensions and Central Cocycles}~\label{sec:extensions}

In this section we examine two natural algebraic extensions of the boundary-deformed shift algebra:
(1) central extensions of the abelian polynomial algebra $\langle T \rangle$,
and (2) finite-dimensional truncations that illustrate the localization of non-commutativity.
The only cohomologically nontrivial structures are the boundary--supported $2$--cocycles described in Section~\ref{sec:cohomology}.

\subsection{Infinite--dimensional abelian algebra and its central extensions}

The subalgebra generated by the powers of $T$,
\[
\langle T \rangle = \mathrm{span}\{T^n : n \ge 1\},
\]
is infinite--dimensional and abelian (Proposition~\ref{prop:polyT}). For abelian Lie algebras, the Chevalley--Eilenberg differential vanishes in degree $2$, so  every alternating bilinear form is a $2$-cocycle. This leads to a rich family of central extensions.

\begin{prop}[Central extensions of $\langle T \rangle$]\label{prop:central-A}
Let $c : \langle T \rangle \times \langle T \rangle \to \mathbb{C}$ be any skew-symmetric bilinear form, and let $Z$ be a central element.  Define on
\[
\widetilde{\langle T \rangle} := \langle T \rangle \oplus \mathbb{C} Z
\]
the bracket
\[
[X + \alpha Z, Y + \beta Z] := c(X,Y) Z, \quad X,Y \in \langle T \rangle,\quad \alpha,\beta \in \mathbb{C}.
\]
Then $(\widetilde{\langle T \rangle}, [\cdot,\cdot])$ is a Lie algebra---specifically, a central extension of the abelian algebra $\langle T \rangle$.
\end{prop}

\begin{proof}
Skew--symmetry and bilinearity are immediate from those of $c$. The Jacobi identity reduces to
\[
[X,[Y,Z]] + [Y,[Z,X]] + [Z,[X,Y]] = 0,
\]
which holds because $Z$ is central and $[X,Y] = 0$ in $\langle T \rangle$. Equivalently, since the Lie bracket on $\langle T \rangle$ is zero, the Chevalley--Eilenberg differential $d : C^1 \to C^2$ vanishes, so every $c \in C^2$ is closed.
\end{proof}

Among these extensions, the boundary--localized cocycles  $\omega_j(X,Y) = \langle e_j, [X,Y] e_j \rangle$ are of particular interest. Although $[X,Y] = 0$ for $X,Y \in \langle T \rangle$, these cocycles become nontrivial when extended to the enlarged algebra $\mathcal{B} = \mathrm{span}\{U^a E U^b, U^n\}$ (Proposition~\ref{prop:2cocycle-corrected}). Thus, the physical interpretation of $\omega_j$ lies not in deforming $\langle T \rangle$ itself, but in encoding the failure of $\langle T \rangle$ to capture the full edge-algebraic structure.

\begin{remark}[Relation to Virasoro and Kac--Moody algebras]
Like the Virasoro cocycle, the forms $\omega_j$ are supported on a discrete set of sites  and vanish in the bulk. However, unlike the Virasoro extension which arises from a globally non-abelian Witt algebra, our setting starts from an abelian algebra, so the central extension reflects edge--induced anomalies rather than a global symmetry breaking.
\end{remark}

\subsection{Finite-dimensional truncations and explicit computations}

To make the boundary--localized nature of non--commutativity concrete, we consider finite truncations of the half-lattice to $\{0,1,\dots,N\}$.

\begin{example}[Four--site truncation]\label{ex:4sites}
On $\mathbb{C}^4$ with basis $\{e_0,e_1,e_2,e_3\}$, define
\[
T = \begin{pmatrix}
\varepsilon & 1 & 0 & 0 \\
0 & 0 & 1 & 0 \\
0 & 0 & 0 & 1 \\
0 & 0 & 0 & 0
\end{pmatrix}.
\]
Then
\[
T^2 = \begin{pmatrix}
\varepsilon^2 & \varepsilon & 1 & 0 \\
0 & 0 & 0 & 1 \\
0 & 0 & 0 & 0 \\
0 & 0 & 0 & 0
\end{pmatrix},
\qquad
[T, T^2] = 0.
\]
Thus, even in finite dimensions, the polynomial algebra $\langle T \rangle$ remains  abelian. Nontrivial commutators appear only when we include rank-one corners such as $E = e_0 e_0^\top$ or $U E = e_1 e_0^\top$.
\end{example}

This confirms a key principle: **the boundary deformation $\varepsilon E$ does not introduce non-commutativity among the powers of $T$**; it only enriches the ambient operator algebra with finite-rank edge operators.

\begin{prop}[Matrix support localization]
In the $(N+1)$-dimensional truncation, the commutator $[T^m, T^n]$ is supported in the upper-left $(m+n) \times (m+n)$ block and has rank at most $m+n$. For $N \ge m+n$, the bulk entries (indices $\ge m+n$) remain identically zero.
\end{prop}

\begin{proof}
For each $j \in \mathbb{Z}_{\ge 0}$, define $X_j = U^{j+1} E$ and $Y_j = E U^{j+1}$.
Then
\[
[X_j, Y_j] = U^{j+1} E E U^{j+1} - E U^{j+1} U^{j+1} E = U^{j+1} E U^{j+1},
\]
since $E^2 = E$ and $U^{2j+2} E = 0$ on $\ell^2(\mathbb{Z}_{\ge 0})$.
Therefore,
\[
\omega_k(X_j, Y_j) = \langle e_k, U^{j+1} E U^{j+1} e_k \rangle =
\begin{cases}
1 & \text{if } k = j+1, \\
0 & \text{otherwise}.
\end{cases}
\]
Hence the matrix $(\omega_k(X_j, Y_j))_{j,k}$ is diagonal with nonzero entries, proving linear independence.
\end{proof}

In summary:\begin{itemize}\item The polynomial algebra $\langle T \rangle$ is abelian and infinite-dimensional.
\item Its central extensions are classified by arbitrary skew forms; the physically relevant ones are the boundary-localized cocycles $\omega_j$.
\item Finite truncations confirm that non-commutativity is absent within $\langle T \rangle$, and only appears in the enlarged algebra $\mathcal{B}$.\end{itemize}
These observations clarify the algebraic landscape: the model provides a  rigorous operator-theoretic realization of boundary--supported cohomology, with the true novelty lying in the **spatial separation between trivial bulk and nontrivial edge**, not in a relaxation of the Lie axioms.
\subsection{Representations of the boundary algebra}

The enlarged algebra $\mathcal{A} = \mathrm{span}\{U^a E U^b, U^n : a,b,n \ge 0\}$ acts naturally on the Hilbert space $\ell^2(\mathbb{Z}_{\ge 0})$ by bounded operators. This action defines a faithful representation
\[
\pi : \mathcal{A} \longrightarrow \mathcal{B}(\ell^2(\mathbb{Z}_{\ge 0})), \quad \pi(X) = X,
\]
which is continuous with respect to the operator norm.

Let $\mathcal{A}_{\mathrm{edge}} = \mathrm{span}\{U^a E U^b : a,b \ge 0\}$ denote the ideal of edge operators. This subalgebra consists entirely of finite-rank operators and plays a central role in the representation theory of $\mathcal{A}$. The following lemma establishes that the natural representation of $\mathcal{A}_{\mathrm{edge}}$ is irreducible.

\begin{lemma}[Irreducibility of the edge representation]\label{lem:irreducible}
The set $\{X f : X \in \mathcal{A}_{\mathrm{edge}}\}$ is dense in $\ell^2(\mathbb{Z}_{\ge 0})$ for every nonzero $f \in \ell^2(\mathbb{Z}_{\ge 0})$. Consequently, the restriction $\pi|_{\mathcal{A}_{\mathrm{edge}}}$ is an irreducible representation.
\end{lemma}

\begin{proof}
Let $f \in \ell^2(\mathbb{Z}_{\ge 0})$ be nonzero. Then there exists $k \ge 0$ such that $\langle e_k, f \rangle \ne 0$. For any $j \ge 0$, consider the operator $X_j = U^j E U^k \in \mathcal{A}_{\mathrm{edge}}$. Its action on $f$ is
\[
X_j f = U^j E U^k f = \langle e_k, f \rangle \, e_j.
\]
Since $\langle e_k, f \rangle \ne 0$, we obtain that $e_j \in \overline{\mathrm{span}}\{X f : X \in \mathcal{A}_{\mathrm{edge}}\}$ for all $j \ge 0$. Thus the orbit of $f$ under $\mathcal{A}_{\mathrm{edge}}$ contains all basis vectors $e_j$, and hence spans a dense subspace of $\ell^2(\mathbb{Z}_{\ge 0})$.
\end{proof}

This irreducibility confirms that the edge algebra $\mathcal{A}_{\mathrm{edge}}$ acts **transitively** on the Hilbert space, despite being composed only of finite-rank operators. In particular, the cyclic vectors $e_j$ generate the full space under the action of $\mathcal{A}_{\mathrm{edge}}$, and the matrix coefficients
\[
\omega_j(X,Y) = \langle e_j, [X,Y] e_j \rangle
\]
are precisely the Chevalley--Eilenberg 2-cocycles associated with the representation $\pi$ and the cyclic vector $e_j$. In the language of projective representations, each $\omega_j$ defines a central extension
\[
0 \longrightarrow \mathbb{C} \longrightarrow \widetilde{\mathcal{A}}_j \longrightarrow \mathcal{A} \longrightarrow 0,
\]
which admits a projective representation on $\ell^2(\mathbb{Z}_{\ge 0})$ with cocycle $\omega_j$.

Thus, the natural operator representation not only provides the analytic foundation for the cohomological construction, but also exhibits nontrivial representation-theoretic structure—thereby justifying the inclusion of “Representations” in the title.

\section{Applications and Examples}
\label{sec:applications}

In this final section we illustrate the algebraic and cohomological structures developed in the previous sections through concrete finite-dimensional models and numerical computations.  
These examples clarify two essential points:
\begin{enumerate}
    \item The polynomial algebra $\langle T \rangle = \mathrm{span}\{T^n : n \ge 0\}$ is always abelian, so $[T^m,T^n] = 0$ for all $m,n$, even in finite truncations.
    \item Nontrivial commutators and boundary-localized cohomology arise only  when the algebra is enlarged to include rank-one corner operators of the form $U^a E U^b$.
\end{enumerate}
We also discuss the spectral consequences of the boundary perturbation and its physical interpretation as a model for edge-localized modes in discrete quantum systems.

\subsection{Connection to tight--binding edge states}

Our operator $T = U + \varepsilon E$ provides a minimal tight--binding model for a one-dimensional semi-infinite lattice with a boundary potential. Consider the single--particle Hamiltonian
\begin{equation}\label{eq:tight-binding}
H = U + U^* + \varepsilon E,
\end{equation}
which describes a chain with nearest-neighbor hopping (encoded by $U + U^*$) and an on-site boundary potential $\varepsilon$ at site $0$. While our focus is on the non-self-adjoint operator $T = U + \varepsilon E$, the spectral picture for the Hermitian case is instructive: for $\varepsilon \neq 0$, the boundary defect can support an exponentially localized eigenstate—precisely the \emph{edge mode} familiar from the Su--Schrieffer--Heeger model or the integer quantum Hall effect \cite{Hats93}.

In our non-Hermitian setting, the truncated operator still exhibits a boundary eigenvalue $\lambda_{\text{edge}} \approx \varepsilon$, as shown in Figure~\ref{fig:spectrum_edge}. This eigenvalue corresponds to an eigenvector concentrated near the boundary, confirming that the algebraic edge structure detected by the cocycles $\omega_j$ has a direct spectral manifestation. Although our analysis focuses on the non-self-adjoint operator $T$, the algebraic structure of the boundary commutators—and hence the cocycles $\omega_j$—depends only on the presence of the rank-one boundary term $\varepsilon E$, not on self-adjointness. The same edge-localized commutators appear in the hermitian case $H = U + U^* + \varepsilon E$, so the cohomological signatures derived here apply equally to physical tight-binding models.

\subsection{Finite-dimensional truncations}

To make the localization phenomenon explicit, we truncate the half-lattice to $N+1$ sites: $\{0,1,\dots,N\}$, and represent operators as $(N+1)\times(N+1)$ matrices.

\begin{example}[Explicit computation for $\varepsilon = 0.3$]
On $4$ sites, one finds
\[
[T, T^2] = \varepsilon
\begin{pmatrix}
1 & 0 & 1 & 0\\
0 & 1 & 0 & 0\\
0 & 0 & 0 & 0\\
0 & 0 & 0 & 0
\end{pmatrix},
\quad
\omega_0(T,T^2) = \varepsilon,\ \omega_1(T,T^2) = \varepsilon,\ \omega_2 = \omega_3 = 0.
\]
\end{example}

Nontrivial commutators appear only when we consider the enlarged algebra
\[
\mathcal{B}_N = \mathrm{span}\{U^a E U^b : 0 \le a,b \le N\},
\]
where $U$ and $E$ are the truncated shift and boundary projector, respectively.

\begin{example}[Boundary commutator in $\mathcal{B}_4$]
In the same four-site setting, let $E = e_0 e_0^\top$ and $U$ be the shift with $U e_3 = 0$. Then
\[
[U, E] = U E - E U = e_1 e_0^\top - e_0 e_1^\top,
\]
which is a rank-2, skew-symmetric matrix supported on the $\{0,1\} \times \{0,1\}$ block. This operator is not in $\langle T \rangle$, but it belongs to the enlarged algebra $\mathcal{B}_4$, where it generates nontrivial cocycles.
\end{example}

\begin{figure}[h!]
\centering
\begin{tikzpicture}
\begin{axis}[
    width=8cm,
    height=8cm,
    enlargelimits={abs=0.5},
    colormap/jet,
    xlabel={$j$},
    ylabel={$i$},
    xtick={0,1,2,3},
    ytick={0,1,2,3},
    title={Matrix pattern of $T$ ($\varepsilon = 0.3$)},
    axis on top,
    colorbar,
    colorbar style={ylabel={Value}},
    grid=none,
]
\pgfmathsetmacro{\eps}{0.3}
\addplot[
  matrix plot,
  mesh/cols=4,
  point meta=explicit,
]
table[row sep=newline, meta=z] {
x y z
0 0  \eps
1 0  1
2 0  0
3 0  0
0 1  0
1 1  0
2 1  1
3 1  0
0 2  0
1 2  0
2 2  0
3 2  1
0 3  0
1 3  0
2 3  0
3 3  0
};
\addplot[only marks, mark=none]
  coordinates {(0,0)}
  node[anchor=south east, font=\small] {$\varepsilon$};
\end{axis}
\end{tikzpicture}
\caption{Heatmap of the truncated operator $T = U + \varepsilon E$. The only deviation from the pure shift is the $\varepsilon$-entry at $(0,0)$, confirming the boundary-localized nature of the perturbation.}
\label{fig:matrix_T}
\end{figure}

Figure~\ref{fig:matrix_T} visually confirms that the perturbation is confined to the boundary.

\subsection{Spectral properties and edge eigenvalue}

The boundary term $\varepsilon E$ affects the spectral properties of $T$ without altering its essential spectrum.

\begin{prop}[Spectral structure]\label{prop:spectral}
For $\varepsilon \in \mathbb{C}$, the operator $T = U + \varepsilon E$ on $\ell^2(\mathbb{Z}_{\ge 0})$ satisfies:
\begin{enumerate}
    \item $\sigma_{\mathrm{ess}}(T) = \overline{\mathbb{D}} = \{z \in \mathbb{C} : |z| \le 1\}$,
    \item Any eigenvalue $\lambda \in \sigma_{\mathrm{p}}(T)$ must satisfy $|\lambda| < 1$ and arises from the boundary defect.
\end{enumerate}
\end{prop}

\begin{proof}
Since $E$ is rank--one (hence compact), Weyl's theorem implies $\sigma_{\mathrm{ess}}(T) = \sigma_{\mathrm{ess}}(U) = \overline{\mathbb{D}}$. Point spectrum, if any, consists of isolated eigenvalues of finite multiplicity inside the open unit disk.
\end{proof}

In the $4$--site finite truncation, $T$ is nilpotent of order $4$, so the only possible eigenvalue is $0$. However, the boundary perturbation lifts one eigenvalue off zero. By non-degenerate perturbation theory for finite-dimensional matrices, the boundary eigenvalue satisfies
\begin{equation}\label{eq:edge-eigenvalue}
\lambda_{\text{edge}} = \varepsilon + O(\varepsilon^2) \quad \text{as } \varepsilon \to 0,
\end{equation}
which explains the linear dependence observed in Figure~\ref{fig:spectrum_edge}. The associated eigenvector $v$ satisfies $v(j) = O(\varepsilon^j)$, confirming its boundary-localized character.

\begin{figure}[h!]
\centering
\begin{tikzpicture}
\begin{axis}[
    width=10cm,
    height=6.5cm,
    xlabel={$\varepsilon$},
    ylabel={Eigenvalues of $T_\varepsilon$ (4-site truncation)},
    grid=both,
    grid style={line width=.1pt, draw=gray!30},
    major grid style={line width=.2pt, draw=gray!50},
    title={Spectral evolution with boundary coupling},
    tick label style={font=\small},
    label style={font=\small},
    title style={font=\small},
    legend style={at={(0.98,0.02)},anchor=south east, font=\footnotesize, draw=none},
]
% Edge eigenvalue: continuous branch lambda ≈ ε
\addplot[thick, blue, samples=100, domain=-1.5:1.5]{x}
node[pos=0.85, above, font=\scriptsize, color=blue]{$\lambda_{\text{edge}} = \varepsilon + O(\varepsilon^2)$};
\addplot[thick, dashed, black, samples=2, domain=-2:2]{0}
node[right, pos=0.9, font=\scriptsize]{bulk ($\lambda=0$)};
% Bulk eigenvalues: dashed line at λ = 0
\addplot[only marks, mark=*, color=red, mark size=1.8pt] coordinates {
(-1.5,0) (-1,0) (-0.5,0) (0,0) (0.5,0) (1,0) (1.5,0)
};
\legend{edge eigenvalue, bulk (nilpotent)}
\end{axis}
\end{tikzpicture}
\caption{In the 4-site truncation, $T$ is nilpotent of order 4, so all bulk eigenvalues are 0. The boundary coupling $\varepsilon$ produces a single eigenvalue $\lambda_{\text{edge}} = \varepsilon + O(\varepsilon^2)$, corresponding to an edge-localized mode.}
\label{fig:spectrum_edge}
\end{figure}

\subsection{Physical interpretation}

The cohomological cocycles $\omega_j$ constructed in Section~\ref{sec:cohomology} provide an \emph{algebraic fingerprint} of such boundary-localized degrees of freedom: they vanish in the bulk but detect nontrivial commutators supported near the edge. This parallels the role of boundary invariants in topological phases of matter \cite{Hats93}, where edge modes are signatures of bulk topological order. In our setting, however, the phenomenon arises purely from spatial localization in a non-translation-invariant system, without invoking topological band theory.

Together, these figures and computations constitute not merely illustrations but \emph{concrete evidence} of our central thesis: nontrivial algebraic and spectral phenomena can emerge from spatial inhomogeneity while maintaining strict Lie algebraic structure in the bulk.
\section{Conclusion} \label{sec:conclusion}

We have analyzed the operator-algebraic and cohomological structure induced by the boundary-deformed unilateral shift
\[
T = U + \varepsilon E
\]
on the half-lattice $\mathbb{Z}_{\ge\ 0}$. The main insight is that spatial inhomogeneity alone suffices to generate rich algebraic and cohomological phenomena, without any deviation from standard Lie algebraic principles.

The polynomial algebra $\langle T \rangle = \mathrm{span}\{T^n : n \ge 0\}$ is abelian, as expected for powers of a single operator. However, the enlarged algebra
\[
\mathcal{A} = \mathrm{span}\{U^a E U^b,\, U^n : a,b,n \ge 0\}
\]
exhibits nontrivial, finite-rank commutators that are strictly localized in a finite neighborhood of the boundary. Since $\mathcal{A} \subset \mathcal{B}(\ell^2)$, the Jacobi identity holds identically (Theorem~\ref{thm:jacobi}), and all phenomena arise from the geometry of the half-lattice, not from a relaxation of Lie axioms.

The central mathematical result is the explicit description of the second Chevalley--Eilenberg cohomology:
\[
H^2(\mathcal{A},\mathbb{C}) \cong \bigoplus_{j=0}^\infty \mathbb{C} \cdot [\omega_j],
\qquad
\omega_j(X,Y) = \langle e_j, [X,Y] e_j \rangle,
\]
where each cocycle $\omega_j$ detects algebraic activity at site $j$ and vanishes identically in the bulk. These cocycles provide a site-resolved, non-translation-invariant analogue of central extensions familiar from the Virasoro or Kac--Moody theories.

Quantitative bounds (Proposition~\ref{prop:bounds}) confirm that $[T^m,T^n]$ has rank at most $m+n$, support in $\{0,\dots,m+n-1\}$, and norm controlled by $|\varepsilon|$ and $m+n$. Finite-dimensional truncations and spectral analysis illustrate the sharp bulk--edge dichotomy: the bulk remains algebraically and spectrally trivial, while all nontrivial structure is confined to the boundary.

From the perspective of mathematical physics, this framework offers a rigorous operator-algebraic model for boundary-localized phenomena—such as edge modes or localized anomalies—in discrete quantum systems, grounded entirely in standard Lie theory and functional analysis.

This work opens several natural directions for future research:\begin{itemize}
   

\item Extensions to higher-dimensional half-lattices and corner geometries,
    \item Connections with index theorems and boundary invariants in topological phases of matter,
    \item Cohomological models of spatially varying boundary couplings ($\varepsilon = \varepsilon(x)$).
 \end{itemize}   
All these avenues can be pursued within the conventional framework of Lie algebras and compact perturbations, underscoring that the richness of boundary phenomena stems from localization—not from exotic algebraic structures.

\begin{thebibliography}{99}


\bibitem{Athmouni-half}
N.~Athmouni,
\emph{Spectral and Dynamical Analysis of Fractional Discrete Laplacians on the Half-Lattice},
arXiv:2510.10680v1 (2025).

\bibitem{Athmouni-rigidity}
N.~Athmouni,
\emph{Local Rigidity of Quasi--Lie Brackets on Quaternionic Banach Modules and Applications to Nonlinear PDEs},
arXiv:2510.10124v1 (2025).



\bibitem{BotSil06}
A.~B\"{o}ttcher and B.~Silbermann,
\emph{Analysis of Toeplitz Operators},
2nd edition, Springer, 2006.
\bibitem{Dou98}
R.~G.~Douglas,
\emph{Banach Algebra Techniques in Operator Theory},
2nd edition, Springer, 1998.






\bibitem{GohKr69}
I.~Gohberg and M.~G.~Krein,
\emph{Introduction to the Theory of Linear Nonselfadjoint Operators},
Translations of Mathematical Monographs, vol.~18, American Mathematical Society, 1969.
\bibitem{Hall2015}
B.~C.~Hall,
\emph{Lie Groups, Lie Algebras, and Representations: An Elementary Introduction},
2nd edition, Graduate Texts in Mathematics, Vol.~222, Springer, 2015.

\bibitem{Hats93}
Y.~Hatsugai,
\emph{Chern number and edge states in the integer quantum Hall effect},
Phys. Rev. Lett. \textbf{71} (1993), 3697--3700.

\bibitem{Humphreys}
J.~E.~Humphreys,
\emph{Introduction to Lie Algebras and Representation Theory},
Graduate Texts in Mathematics, Vol.~9, Springer, 1972.

\bibitem{Nik02}
N.~K.~Nikolski,
\emph{Operators, Functions, and Systems: An Easy Reading. Vol.~1},
American Mathematical Society, 2002.
\bibitem{Sim05}
B.~Simon,
\emph{Trace Ideals and Their Applications},
2nd edition, American Mathematical Society, 2005.

\end{thebibliography}

\end{document}
