\section{Actions of the repetitive automorphism groups} \label{sec:actions}

Using the equivalences with coherent Auslander-Reiten diagrams of the previous section, we prove that the groups of automorphisms of $\ZQ$ and $\ZZQ$ act on $\mathcal{C}^Q$ for any $\mathcal{C}$ stable $\infty$-category.

\begin{defi}
Let $G$ be a group and $\mathcal{C}$ an $\infty$-category. We write $\mathsf{B}G$ for the one-object groupoid associated to $G$, as well as its nerve. A left (resp. right) \emph{$\infty$-action} of $G$ on $\mathcal{C}$, denoted $G \ \rotatebox[origin=c]{-90}{$\circlearrowright$}\ \mathcal{C}$, is a functor $\mathsf{B}G \to \infCAT$ (resp. $\mathsf{B}G^\mathrm{op} \to \infCAT$) sending the only object $ \ast \in \mathsf{B}G$ to $\mathcal{C}$.

We call a \emph{strict action} of $G$ on $\mathcal{C}$ a functor $\mathsf{B}G \to \SSET$ mapping $\ast \mapsto\mathcal{C}$, which is the same as a group homomorphism $G \to \mathrm{Aut}_{\SSET}(\mathcal{C})$. Composing with the localization functor $N(\SSET) \to \infCAT$, this automatically gives also an $\infty$-action $\mathsf{B}G \to \infCAT$.
\end{defi}

\begin{rema}
Let $\eta: \mathsf{B}G \to \infCAT$ be an $\infty$-action of $G$ on $\mathcal{C}$. On mapping spaces this functor induces a morphism of $\infty$-groupoids $\Omega (\mathsf{B}G) \to \mathsf{Aut}(\mathcal{C})^{\simeq}$, where we denote $\mathsf{Aut}(-) \subset \Fun(-,-)$ the full subcategory of autoequivalences and $(-)^\simeq$ the maximal $\infty$-groupoid. Applying $\pi_0$, we get a homomorphism of groups 
$$G \xrightarrow{\quad} \pi_0(\mathsf{Aut}(\mathcal{C})^{\simeq}), \quad g \longmapsto \eta_g$$
from $G$ to the group of autoequivalences of $\mathcal{C}$ (up to natural equivalence).
\end{rema}

\begin{example} \label{example:action-Z}
Let $\mathcal{C}$ be an $\infty$-category and $f:\mathcal{C} \xrightarrow{\simeq} \mathcal{C}$ an autoequivalence. The cyclic free group $\mathbb{Z}$ acts on $\mathcal{C}$ by powers of $f$. That is, there is a functor $\mathsf{B}\mathbb{Z} \to \infCAT$ sending $\ast \mapsto \mathcal{C}$ and $n \mapsto f^n$.
\end{example}

\begin{example} \label{example:action-aut}
Let $K$ be a small $\infty$-category and $\mathcal{C}$ an $\infty$-category. Denoting $\mathrm{Aut}(K) \subset \Hom{\sSet}(K,K)$, there is a natural right strict action of $\mathrm{Aut}(K)$ on $\mathcal{C}^K$. Indeed, the group $\mathrm{Aut}(K)$ can be seen as the subcategory of $\sSet$ with one object $K$ and its automorphisms. Then the action is given by the functor 
$$\mathsf{B}\mathrm{Aut}(K)^\mathrm{op} \xhookrightarrow{\quad} \sSet^\mathrm{op} \xrightarrow{\ \Fun(-,\mathcal{C})\ } \SSET.$$
One can also see this action as a right multiplication $\mathcal{C}^K \times \mathsf{B}\mathrm{Aut}(K) \to \mathcal{C}^K$ given by the mapping $(X,\sigma) \mapsto X\sigma = X_{\sigma(-)}$.
\end{example}

\begin{nota}
Given any translation quiver $(\Gamma,\tau)$ with a fixed polarization, we write $\mathrm{Aut}_\mathsf{tr}(\Gamma)$ for the group of translation automorphisms. For $\ZZQ$, we denote by $\mathrm{Aut}_{\mathsf{tr},\sigma}(\ZZQ)$ the subgroup of translation automorphisms commuting with $\sigma$.
\end{nota}

\begin{rema}
It is easy to see that $\mathrm{Aut}_{\mathsf{tr},\sigma}(\ZZQ) \cong \mathbb{Z}\times\mathrm{Aut}_{\mathsf{tr}}(\ZQ)$, where $\mathbb{Z}$ identifies with the subgroup generated by $\sigma$ and $\mathrm{Aut}_{\mathsf{tr}}(\ZQ)$ with the automorphisms fixing every component of $\ZZQ$.
\end{rema}

\begin{prop} \label{prop:strict-action}
There is a natural right strict action of $\mathrm{Aut}_{\mathsf{tr}}(\ZQ)$ on $\mathcal{C}^\ZQmesh$ for any $\mathcal{C}$ stable $\infty$-category. 
\end{prop}
\begin{proof}
By (the proof of) \Cref{coro:equiv-repet}, any translation automorphism $f:\ZQ \cong \ZQ$ induces an automorphism $\tilde{f^*}: \mathcal{C}^\ZQmesh \cong \mathcal{C}^\ZQmesh$, and by construction, $\tilde{(gf)^*} = \tilde{f^*}\tilde{g^*}$. Hence we get a group homomorphism
\begin{equation*}
\mathrm{Aut}_{\mathsf{tr}}(\ZQ)^\mathrm{op} \to \mathrm{Aut}_{\SSET}(\mathcal{C}^\ZQmesh), \quad f \mapsto \tilde{f^*}. \qedhere  
\end{equation*}
\end{proof}

\begin{theo} \label{theo:actionZQ}
Let $Q$ be a finite acyclic quiver. There is a natural right $\infty$-action 
$$\mathrm{Aut}_\mathsf{tr}(\ZQ) \ \rotatebox[origin=c]{-90}{$\circlearrowright$}\ \mathcal{C}^Q,$$ for any $\mathcal{C}$ stable $\infty$-category.
\end{theo}

We need the following easy but subtle lemma in the proof:

\begin{lemm} \label{lemm:equiv-functors}
Let $u: X \to A$ be a morphism of simplicial sets with $A$ an $\infty$-category. Take $x \in X$ and $\alpha: u(x) \to a$ an equivalence in $A$. Then there exists $u': X \to A$ with $u'(x) = a$ and a natural equivalence $\eta: u \to u'$ in $\Fun(X,A)$.
\end{lemm}
\begin{proof}
Because $A$ is an $\infty$-category and $x:\Delta^0 \to X$ is a monomorphism, the evaluation map $x^*:\Fun(X,A) \to A$ is an isofibration (\Cref{prop:fib-mono}). Hence, one can find a lift in the following diagram.
$$\begin{tikzcd}
\{0\} \ar[r,"u"] \ar[d,hookrightarrow] & \Fun(X,A) \ar[d,"x^*"] \\
J \ar[r,"\alpha"] \ar[ru,"\exists\, \eta", dashed] & A
\end{tikzcd}$$
It only remains to define $u' := \eta(1): X \to A$.
\end{proof}

\begin{proof}[Proof {\normalfont{(of \Cref{theo:actionZQ})}}]
We have a right $\infty$-action $\mathrm{Aut}(\ZQ) \ \rotatebox[origin=c]{-90}{$\circlearrowright$}\ \mathcal{C}^\ZQmesh$ by \Cref{prop:strict-action}. Then using \Cref{lemm:equiv-functors} and the equivalence $\mathcal{C}^\ZQmesh \simeq \mathcal{C}^Q$ of \Cref{theo:ZQ-mesh} we can turn this into an equivalent right $\infty$-action $\mathrm{Aut}_\mathsf{tr}(\ZQ) \ \rotatebox[origin=c]{-90}{$\circlearrowright$}\ \mathcal{C}^Q$, i.e. a functor $\mathsf{B}\mathrm{Aut}_\mathsf{tr}(\ZQ)^\mathrm{op} \to \infCAT$ sending $\ast \mapsto \mathcal{C}^Q$ which is naturally equivalent to the one giving the original action.
\end{proof}

\begin{prop} \label{prop:homomorphismZZQ}
Let $Q$ be a finite acyclic quiver. There is a natural homomorphism
\begin{equation}
\mathbb{Z} \times \mathrm{Aut}_\mathsf{tr}(\ZQ)^\mathrm{op} \xrightarrow{ \ \ } \pi_0(\mathsf{Aut}(\mathcal{C}^Q)^{\simeq}),\quad (n,f) \xmapsto{\ \ } \Sigma^n\tilde{f^*},   
\end{equation}
for any $\mathcal{C}$ stable $\infty$-category.      
\end{prop}
\begin{proof}
Observe that the direct product of groups $G\times H$ can be obtained as a quotient of the free product (coproduct) $G\ast H$ by the normal subgroup generated by elements $ghg^{-1}h^{-1}$ with $g\in G$ and $h\in H$. Hence a pair of homomorphisms $\alpha: G \to K$ and $\beta: H \to K$ with the property that $\alpha(g)\beta(h) = \beta(h)\alpha(g)$ induces a unique homomorphism $G \times H \to K$ sending $(g,h) \mapsto \alpha(g)\beta(h)$.

Since equivalences are exact, $\Sigma^n\tilde{f^*} = \tilde{f^*}\Sigma^n$ in $\pi_0(\mathsf{Aut}(\mathcal{C}^Q)^{\simeq})$. Thus, we can apply the above discussion to the homomorphisms $\mathbb{Z} \to \pi_0(\mathsf{Aut}(\mathcal{C}^Q)^{\simeq})$ (\Cref{example:action-Z}) and $\mathrm{Aut}_\mathsf{tr}(\ZQ)^\mathrm{op} \to \pi_0(\mathsf{Aut}(\mathcal{C}^Q)^{\simeq})$ (\Cref{theo:actionZQ}) to get the desired homomorphism.
\end{proof}

\begin{rema}[Naturality] \label{rema:naturality-action}
The equivalences produced in \Cref{prop:homomorphismZZQ} are stable equivalences, i.e. they are natural with respect to exact functors, in the sense that, given $(n,f)\in \mathbb{Z} \times \mathrm{Aut}_\mathsf{tr}(\ZQ)^\mathrm{op}$ and an exact functor $F:\mathcal{C} \to \mathcal{D}$, the square 
$$\begin{tikzcd}
\mathcal{C}^Q \ar[r,"\Sigma^n\tilde{f^*}"] \ar[d,"F_*"'] & \mathcal{C}^{Q} \ar[d,"F_*"] \\
\mathcal{D}^Q \ar[r,"\Sigma^n\tilde{f^*}"] & \mathcal{D}^{Q}
\end{tikzcd}$$
commutes in $\infCAT$. This is because suspension commutes with exact functors and the naturality of the action of $\mathrm{Aut}_\mathsf{tr}(\ZQ)$ (see \Cref{coro:equiv-repet}).
\end{rema}

Combining \Cref{theo:actionZQ,prop:homomorphismZZQ}, we obtain a general method to produce autoequivalences of $\mathcal{C}^Q$ from symmetries of the irregular Auslander-Reiten quiver $\Gamma_Q^\mathrm{irr}$. By \Cref{lemm:sigma} below, if $Q$ is Dynkin, the automorphism $\sigma$ of $\Gamma_Q^\mathrm{irr}$ corresponding to the suspension of $\Dd{b}{kQ}$ commutes with every other, and so $\mathrm{Aut}_{\mathsf{tr},\sigma}(\Gamma_{Q}^\mathrm{irr})$ is nothing but $\mathrm{Aut}_\mathsf{tr}(\ZQ)$ in the Dynkin case and $\mathrm{Aut}_{\mathsf{tr},\sigma}(\ZZQ)$ in the non-Dynkin.

\begin{lemm} \label{lemm:sigma}
If $Q$ is Dynkin, then $\sigma$ is in the center of the group $\mathrm{Aut}_{\mathsf{tr}}(\Gamma_{Q}^\mathrm{irr})$.
\end{lemm}
\begin{proof}
See \cite[Lemma 3.4]{MiyYek01}.
\end{proof}

\begin{coro} \label{coro:action}
Let $Q$ be a finite acyclic quiver. There is a natural homomorphism
\begin{equation} \label{eq:action}
\mathrm{Aut}_{\mathsf{tr},\sigma}(\Gamma_{Q}^\mathrm{irr}) \xrightarrow{ \ \ } \pi_0(\mathsf{Aut}(\mathcal{C}^Q)^{\simeq}),   
\end{equation}
for any $\mathcal{C}$ stable $\infty$-category.  
\end{coro}

Starting with automorphisms of (a significant piece of) the Auslander-Reiten quiver of $\Dd{b}{kQ}$, we produce autoequivalences of stable $\infty$-categories of representations $\mathcal{C}^Q$.  
A safety check that the autoequivalences obtained are meaningful is that they have the expected action on the AR quiver $\Gamma_{Q}^\mathrm{irr}$ when $\mathcal{C} = \Dd{b}{k}$ for a field $k$. 

We note, however, that functors on $\Dd{b}{kQ}$ do not induce well defined morphisms of quivers on $\Gamma_{Q}^\mathrm{irr}$, as arrows of $\Gamma_{Q}^\mathrm{irr}$ depend on a particular choice of a basis of irreducible morphisms. Instead, we turn our attention to certain permutations of the vertices.

\begin{nota}
For a quiver $\Gamma$, we write $\mathrm{Aut}^0(\Gamma)$ for the group of permutations of $\Gamma_0$ preserving arrow-multiplicity. There is an obvious epimorphism $\mathrm{Aut}(\Gamma) \to \mathrm{Aut}^0(\Gamma)$ which forgets the action of a quiver automorphism on arrows. This epimorphism is split, as any ordering of the arrows between every two vertices defines a section, and it is an isomorphism precisely when $\Gamma$ has no multiple arrows. 

In our context, we denote $\mathrm{Aut}^0_{\mathsf{tr},\sigma}(\Gamma_{Q}^\mathrm{irr}) \subset \mathrm{Aut}^0(\Gamma_{Q}^\mathrm{irr})$ the subgroup of those permutations which commute with $\tau$ and $\sigma$, and we consider fixed a section of the split epimorphism $\mathrm{Aut}_{\mathsf{tr},\sigma}(\Gamma_{Q}^\mathrm{irr}) \to \mathrm{Aut}^0_{\mathsf{tr},\sigma}(\Gamma_{Q}^\mathrm{irr})$. In particular, this produces a natural homomorphism 
\begin{equation} \label{eq:action-0}
\mathrm{Aut}^0_{\mathsf{tr},\sigma}(\Gamma_{Q}^\mathrm{irr}) \xrightarrow{ \ \ } \pi_0(\mathsf{Aut}(\mathcal{C}^Q)^{\simeq})
\end{equation}
for any $\mathcal{C}$ stable $\infty$-category.
\end{nota}

Let $k$ be a field and consider the special case $\mathcal{C} = \Dd{b}{k}$ in the group homomorphism \eqref{eq:action-0}, that we write  
$$\varphi: \mathrm{Aut}^0_{\mathsf{tr},\sigma}(\Gamma_{Q}^\mathrm{irr}) \xrightarrow{ \ \ } \pi_0(\mathsf{Aut}(\Dd{b}{kQ})^{\simeq}), \quad f \mapsto \varphi_f.$$

\begin{lemm} \label{lemm:auts-field}
For all $f \in \mathrm{Aut}^0_{\mathsf{tr},\sigma}(\Gamma_{Q}^\mathrm{irr})$, the functor $\varphi_f$ coincides with $f$ on iso-classes of indecomposable objects of $\Gamma_{Q}^\mathrm{irr}$.
\end{lemm}
\begin{proof}
Let $f \in \mathrm{Aut}^0_{\mathsf{tr},\sigma}(\Gamma_{Q}^\mathrm{irr})$ fixing every component, so that we can identify $f$ with an automorphism in $\mathrm{Aut}^0_\mathsf{tr}(\ZQ)$. By definition of the action and \Cref{rema:Rhom}, there is an equivalence $\mathbb{R}\iHom(f(-),-) \simeq \mathbb{R}\iHom(-,\varphi^{-1}(-))$ of functors $\Dd{b}{kQ} \to \Dd{b}{k}^\ZQ$
(more precisely, of functors $\ZQ \times \Dd{b}{kQ} \to \Dd{b}{k}$). 
If we fix an indecomposable $M \in \ZQ$ and evaluate, we get an equivalence
$$\mathbb{R}\iHom(f(M),-) \simeq \mathbb{R}\iHom(M,\varphi_f^{-1}(-)) \simeq \mathbb{R}\iHom(\varphi_f(M),-)$$
of functors $\Dd{b}{kQ} \to \Dd{b}{k}$. Applying cohomology $H^0$ and Yoneda on the homotopy category, it follows that $f(M) \simeq \varphi_f(M)$ in $\Dd{b}{kQ}$.

If $Q$ is Dynkin, then $\Gamma_{Q}^\mathrm{irr}$ has only one component and we are finished. Otherwise, any $f \in \mathrm{Aut}^0_{\mathsf{tr},\sigma}(\Gamma_{Q}^\mathrm{irr})$ is $\sigma^ng$ with $g \in \mathrm{Aut}^0_\mathsf{tr}(\ZQ)$. The previous argument shows that $\tilde{g^*}(M) \simeq g(M)$, and hence $\varphi_f(M) = \Sigma^n\tilde{g^*}(M) \simeq \sigma^ng(M)$ for all $M \in \Gamma_{Q}^\mathrm{irr}$.
\end{proof}

Therefore, \Cref{coro:action} provides versions of relevant functors in representation theory (e.g. the Auslander-Reiten translation, or the Serre functor) for coefficients in abstract stable homotopy theories, which specialized to coefficients in (the derived category of) a field recover their classical counterparts. A more precise statement of this will appear in \Cref{example:pic-field}.