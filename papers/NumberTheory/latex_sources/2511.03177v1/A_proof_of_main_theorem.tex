\ifdefined\isMainDocument
\else
    \documentclass[10pt,oneside,reqno]{amsart}
    \usepackage{xr}
    \externaldocument{../main} % Main の .aux を参照
    \usepackage{latexsym}
\usepackage{amscd}
\usepackage{amsmath}
\usepackage{amssymb}
\usepackage[mathscr]{euscript}
\usepackage{graphicx}
\usepackage{geometry}
\usepackage{mathtools}
\usepackage[all]{xy}
\usepackage[dvipsnames]{xcolor}
\usepackage[colorlinks,final,hyperindex]{hyperref}
\usepackage{tikz}
\usepackage{tikz-cd}
\usetikzlibrary{decorations.pathmorphing,decorations.markings,arrows,calc,shapes.geometric,arrows.meta,positioning}
\usepackage{enumerate}
\usepackage{multirow}

\usepackage[bb=dsserif]{mathalpha}
\usepackage{bm}

\newcommand{\QQ}{\ensuremath{\mathbb{Q}}}
\newcommand{\DD}{\ensuremath{\mathbb{D}}}
\newcommand{\CC}{\ensuremath{\mathbb{C}}}
\newcommand{\RR}{\ensuremath{\mathbb{R}}}
\newcommand{\ZZ}{\ensuremath{\mathbb{Z}}}
\newcommand{\NN}{\ensuremath{\mathbb{N}}}

\newcommand{\g}{\ensuremath{\mathfrak{g}}}
\renewcommand{\SS}{\ensuremath{\mathbb{S}}}

\newcommand{\id}{\ensuremath{\mathrm{id}}}

\makeatletter

\newcommand{\bbone}{\mathbb{1}}

\newcommand{\calA}{\mathcal{A}}
\newcommand{\calB}{\mathcal{B}}
\newcommand{\calC}{\mathcal{C}}
\newcommand{\calD}{\mathcal{D}}
\newcommand{\calE}{\mathcal{E}}
\newcommand{\calF}{\mathcal{F}}
\newcommand{\calG}{\mathcal{G}}
\newcommand{\calH}{\mathcal{H}}
\newcommand{\calI}{\mathcal{I}}
\newcommand{\calJ}{\mathcal{J}}
\newcommand{\calK}{\mathcal{K}}
\newcommand{\calL}{\mathcal{L}}
\newcommand{\calM}{\mathcal{M}}
\newcommand{\calN}{\mathcal{N}}
\newcommand{\calO}{\mathcal{O}}
\newcommand{\calP}{\mathcal{P}}
\newcommand{\calQ}{\mathcal{Q}}
\newcommand{\calR}{\mathcal{R}}
\newcommand{\calS}{\mathcal{S}}
\newcommand{\calT}{\mathcal{T}}
\newcommand{\calU}{\mathcal{U}}
\newcommand{\calV}{\mathcal{V}}
\newcommand{\calW}{\mathcal{W}}
\newcommand{\calX}{\mathcal{X}}
\newcommand{\calY}{\mathcal{Y}}
\newcommand{\calZ}{\mathcal{Z}}
\newcommand{\kk}{\Bbbk}

\newcommand{\scrY}{\mathscr{Y}}
\newcommand{\scrV}{\mathscr{V}}
\newcommand{\scrP}{\mathscr{P}}

\newcommand{\Rep}{\mathop{\mathrm{Rep}}\nolimits}
\newcommand{\alg}{\mathop{\mathrm{alg}}\nolimits}
\newcommand{\Span}{\mathop{\mathrm{Span}}\nolimits}
\newcommand{\proj}{\mathop{\mathrm{proj}}\nolimits}
\newcommand{\Hom}{\mathop{\mathrm{Hom}}\nolimits}
\newcommand{\PHom}{\mathop{\mathrm{PHom}}\nolimits}
\newcommand{\Tw}{\mathop{\mathrm{Tw}}\nolimits}
\newcommand{\Dim}{\mathop{\mathrm{Dim}}\nolimits}
\newcommand{\Imm}{\mathop{\mathrm{Im}}\nolimits}
\newcommand{\Aus}{\mathop{\mathrm{Aus}}\nolimits}
\newcommand{\Ann}{\mathop{\mathrm{Ann}}\nolimits}
\newcommand{\APC}{\mathop{\mathrm{APC}}\nolimits}
\newcommand{\Barr}{\mathop{\mathrm{Bar}}\nolimits}
\newcommand{\Cone}{\mathop{\mathrm{Cone}}\nolimits}
\newcommand{\modu}{\mathop{\mathrm{mod}}\nolimits}
\newcommand{\dgMod}{\mathop{\mathrm{dgMod}}\nolimits}
\newcommand{\soc}{\mathop{\mathrm{soc}}\nolimits}
\newcommand{\dg}{\mathop{\mathrm{dg}}\nolimits}
\newcommand{\red}{\mathop{\mathrm{red}}\nolimits}
\newcommand{\Map}{\mathop{\mathrm{Map}}\nolimits}
\newcommand{\inj}{\mathop{\mathrm{inj}}\nolimits}
\newcommand{\op}{\mathop{\mathrm{op}}\nolimits}
\newcommand{\Ker}{\mathop{\mathrm{Ker}}\nolimits}
\newcommand{\Tor}{\mathop{\mathrm{Tor}}\nolimits}
\newcommand{\Tot}{\mathop{\mathrm{Tot}}\nolimits}
\newcommand{\Ext}{\mathop{\mathrm{Ext}}\nolimits}
\newcommand{\sg}{\mathop{\mathrm{sg}}\nolimits}
\newcommand{\Sl}{\mathop{\mathfrak{sl}}\nolimits}
\newcommand{\nc}{\mathop{\mathrm{nc}}\nolimits}
\newcommand{\Tr}{\mathop{\mathrm{Tr}}\nolimits}
\newcommand{\Irr}{\mathop{\mathrm{Irr}}\nolimits}
\newcommand{\HC}{\mathop{\mathrm{HC}}\nolimits}
\newcommand{\HH}{\mathop{\mathrm{HH}}\nolimits}
\newcommand{\THH}{\mathop{\mathrm{TH}}\nolimits}
\newcommand{\Perf}{\mathop{\mathrm{Perf}}\nolimits}
\newcommand{\del}{\partial}
\newcommand{\ev}{\mathop{\mathrm{ev}}\nolimits}
\newcommand{\co}{\mathop{\mathrm{co}}\nolimits}
\newcommand{\pt}{\mathop{\mathrm{pt}}\nolimits}
\newcommand{\wt}{\mathop{\mathrm{wt}}\nolimits}
\newcommand{\pCY}{\mathop{\mathrm{pCY}}\nolimits}
\newcommand{\gr}{\mathop{\mathrm{gr}}\nolimits}
\newcommand{\coker}{\mathop{\mathrm{coker}}\nolimits}


\newcommand{\Top}{\mathop{\mathrm{Top}}\nolimits}
\newcommand{\Set}{\mathop{\mathrm{Set}}\nolimits}
\newcommand{\Vect}{\mathop{\mathrm{Vect}}\nolimits}
\newcommand{\Mod}{\mathop{\mathrm{Mod}}\nolimits}
\newcommand{\Ob}{\mathop{\mathrm{Ob}}\nolimits}
\newcommand{\Ind}{\mathop{\mathrm{Ind}}\nolimits}
\newcommand{\Sym}{\mathop{\mathrm{Sym}}\nolimits}
\newcommand{\Alt}{\mathop{\mathrm{Alt}}\nolimits}
\newcommand{\End}{\mathop{\mathrm{End}}\nolimits}
\newcommand{\Ch}{\mathop{\mathrm{Ch}}\nolimits}


\newcommand{\Ass}{\mathop{\mathrm{Ass}}\nolimits}
\newcommand{\Com}{\mathop{\mathrm{Com}}\nolimits}
\newcommand{\Lie}{\mathop{\mathrm{Lie}}\nolimits}
\newcommand{\colim}{\mathop{\mathrm{colim}}\nolimits}

\newcommand{\Op}{\mathop{\mathsf{Operads}}\nolimits}
\newcommand{\Diop}{\mathop{\mathsf{Dioperads}}\nolimits}
\newcommand{\PDiop}{\mathop{\mathsf{PDioperads}}\nolimits}
\newcommand{\Coop}{\mathop{\mathsf{Cooperads}}\nolimits}
\newcommand{\Codiop}{\mathop{\mathsf{Codioperads}}\nolimits}
\newcommand{\PCodiop}{\mathop{\mathsf{PCodioperads}}\nolimits}

\newcommand{\antish}{\ensuremath{\mbox{!`}}}


\tikzset{->-/.style={decoration={
			markings,
			mark=at position #1 with {\arrow{>}}},postaction={decorate}}}
\tikzset{-w-/.style={decoration={
			markings,
			mark=at position #1 with {\arrow{Stealth[fill=white,scale=1.4]}}},postaction={decorate}}}
\tikzset{->-/.default=0.65}
\tikzset{-w-/.default=0.65}
\tikzstyle{bullet}=[circle,fill=black,inner sep=0.5mm]
\tikzstyle{circ}=[circle,draw=black,fill=white,inner sep=0.5mm]
\tikzstyle{vertex}=[circle,draw=black,thick,inner sep=0.5mm]
\tikzstyle{dot}=[draw,circle,fill=black,minimum size=0.5mm,inner sep = 0mm, outer sep = 0mm]
\tikzset{darrow/.style={double distance = 4pt,>={Implies},->},
	darrowthin/.style={double equal sign distance,>={Implies},->},
	tarrow/.style={-,preaction={draw,darrow}},
	qarrow/.style={preaction={draw,darrow,shorten >=0pt},shorten >=1pt,-,double,double
		distance=0.2pt}}
\newcommand\tikcirc[1][2.5]{\tikz[baseline=-#1]{\draw[thick](0,0)circle[radius=#1mm];}}
\newcommand{\tikzfig}[1]{\begin{tikzpicture}[auto,baseline={([yshift=-.5ex]current bounding box.center)}]#1\end{tikzpicture}}
\usetikzlibrary{decorations.pathreplacing}

\usepackage{amsthm}
\usepackage{thmtools}
\usepackage[noabbrev,capitalize]{cleveref}
\newtheorem{theorem}{Theorem}
\newtheorem{lemma}[theorem]{Lemma}
\newtheorem{proposition}[theorem]{Proposition}
\newtheorem{corollary}[theorem]{Corollary}
\newtheorem{conjecture}[theorem]{Conjecture}
\newtheorem{Claim}[theorem]{Claim}

\newtheorem*{theorem*}{Theorem}

\theoremstyle{definition}
\newtheorem{definition}{Definition}
\newtheorem{example}{Example}
\newtheorem{cexample}{Counterexample}
\newtheorem{exercise}{Exercise}
\newtheorem{question}{Question}
\newtheorem{remark}{Remark}

\addtolength{\textheight}{1.75cm}
\addtolength{\voffset}{-0.5cm}
\addtolength{\footskip}{+0.6cm}
\geometry{margin=1.2in}


\usepackage[backend=biber, maxbibnames=99, bibencoding=utf8, doi=false, isbn=false, url=false, style=alphabetic]{biblatex}
\addbibresource{refs.bib}

    \begin{document}
\fi

%\setcounter{section}{4}
\section{Proof of the main theorem}

In this section, we prove the Main Theorem.

\nsubsection{A lemma}
The following lemma plays an essential role in the proof of our main theorem and its proof is given in Appendix \ref{section:proof of lemma}.

\begin{lem}\label{lem:essential}
For $\Psi_1$, $\Psi_2 \in \addmr \cap \Fad \cap V_{\strprty}$, we have
\begin{align}
\begin{aligned}
\Delta_*(d_{\Psi_1}(\Psi_2)_{\#}) =&
d_{\Psi_1}(\Psi_2)_{\#} \otimes 1 + 1\otimes d_{\Psi_1}(\Psi_2)_{\#} + \Psi_{1,\#} \otimes  \Psi_{2,\#}  + \Psi_{2,\#} \otimes  \Psi_{1,\#}.  
\end{aligned}  \label{eq:essential}
\end{align}
\end{lem}

\begin{comment}
\begin{align}
\begin{aligned}
\Delta_*(d_{\Psi_1}^n(\Psi_2)) =&
d_{\Psi_1}^n(\Psi_2) \otimes 1 + 1\otimes d_{\Psi_1}^n(\Psi_2)\\
&\quad+ \sum_{n_1 + n_2 = n-1} \binom{n-1}{n_1} d_{\Psi_1}^{n_1}(\Psi_1) \otimes d_{\Psi_1}^{n_2}(\Psi_2) + \binom{n-1}{n_1} d_{\Psi_1}^{n_1}(\Psi_2) \otimes d_{\Psi_1}^{n_2}(\Psi_1)
\end{aligned}  \label{eq:essential}
\end{align}


\begin{cor}\label{cor:essential}
For a positive integer $n$, and $\Psi_1$ and $\Psi_2 \in \addmr(R)$, 
\begin{align*}
\Delta_*(d_{\Psi_1}^n(x_1)) =& \sum_{n_1 + n_2 = n} \binom{n}{n_1} d_{\Psi_1}^{n_1}(x_1) \otimes d_{\Psi_1}^{n_2}(x_1). 
\end{align*}
\end{cor}

\begin{proof}

\end{proof}

\end{comment}


\nsubsection{Proof of the Main Theorem}

\begin{proof}[Proof of Proposition \ref{mt:addmr-Lie}]

Since, by Corollary \ref{cor:tm-freeLie} and Proposition \ref{prop:deriv-strprty}, $ (\ide{F}_2^{\ge 4}\cap \Fad \cap  V_{\strprty}) \subset \tm_1$ as a Lie algebra, it suffices to show that $\Delta_*(\{\Phi,\Psi\}_{1,\#}) = \{\Phi,\Psi\}_{1,\#} \otimes 1 + 1\otimes \{\Phi,\Psi\}_{1,\#}$.
By \eqref{eq:essential}, we have
\[
\Delta_*(d_{\Psi_1}(\Psi_2)_{\#})  =d_{\Psi_1}(\Psi_2) \otimes 1 + 1\otimes d_{\Psi_1}(\Psi_2) +  \Psi_{1,\#} \otimes \Psi_{2,\#} + \Psi_{2,\#} \otimes \Psi_{1,\#}.
\]
Therefore, we have
\begin{align*}
\Delta_*(\{\Psi_1,\Psi_2\}_{1,\#}) &= \Delta_*(d_{\Psi_1}(\Psi_2)_{\#} - d_{\Psi_2}(\Psi_1)_{\#} ) \\
&\begin{aligned}
&=d_{\Psi_1}(\Psi_2) \otimes 1 + 1\otimes d_{\Psi_1}(\Psi_2) +  \Psi_{1,\#} \otimes \Psi_{2,\#} + \Psi_{2,\#} \otimes \Psi_{1,\#}\\
&\qquad-d_{\Psi_1}(\Psi_2) \otimes 1 - 1\otimes d_{\Psi_1}(\Psi_2)  -  \Psi_{2,\#} \otimes \Psi_{1,\#} + \Psi_{1,\#} \otimes \Psi_{2,\#}\\
\end{aligned}\\
&= d_{\Psi_1}(\Psi_2) \otimes 1 + 1\otimes d_{\Psi_1}(\Psi_2) - d_{\Psi_1}(\Psi_2) \otimes 1 - 1\otimes d_{\Psi_1}(\Psi_2)\\
&=\{\Psi_1,\Psi_2\}_1 \otimes1 + 1\otimes \{\Psi_1,\Psi_2\}_1,
\end{align*}
as claimed.
\end{proof}

\begin{comment}

\nsubsection{A proof of Main Theorem \ref{mt:AdDMR-Grp}}


To prove Main Theorem \ref{mt:AdDMR-Grp}, we consider the affine group scheme $\exp^{\circledast_1}(\addmr(R) \cap \Fad(R) \cap V_{\strprty}(R)))$. Since $\addmr \cap \Fad \cap V_{\strprty}(R) \subset \tm_1(R)$,  $\exp^{\circledast_1}(\addmr(R) \cap \Fad(R) \cap V_{\strprty}(R))$ is a subgroup of $\TM_1(R)$ by Proposition \ref{prop:unipotent}. (\ref{enumi:unipotent-2}).

\begin{lem}
The group $\exp^{\circledast_1}(\addmr(R) \cap \Fad(R) \cap V_{\strprty}(R)$ acts on $\AdDMR_0(R)$.
\end{lem}
\begin{proof}

\end{proof}

Let $\ide{h}^{(n)} := \Q + \Q\cdot X + \cdots + \Q\cdot X^{n-1}$ and we define 
\[
\AdDMR_0^{(n)}(R) := \left\{\Phi \in \ide{h}^{(n)} \middle| 
\begin{matrix}
\wt \Phi - x_1 \ge 4\\
\Delta_{\shuffle}(\Phi) = \Phi \otimes 1 + 1\otimes \Phi\\
\Delta_*(\Phi_{\#}) = \Phi_{\#} \otimes\Phi_{\#}.
\end{matrix}
\right\}.
\]
For example, $\AdDMR_0^{(n)} = 0$ for $n = 0$, $\AdDMR_0^{(n)}(R) = \{x_1\}$ for $n = 2$, $3$.
For $n_1$, $n_2 \in \mathbb{Z}_{\ge 0}$ with $n_1 > n_2$, let us define $ \pi_{n_1,n_2} : \AdDMR_0^{(n_1)}(R) \rightarrow \AdDMR_0^{(n_2)}$ as modded $\Q\cdot X^{n_2}$.
Then, a pair of classes of sets and maps $\left(\{\AdDMR_0^{(n)}\}_{n \in \mathbb{Z}_{\ge 0}},\{\pi_{n_1,n_2}\}_{\substack{n_1> n_2  \\n_1,n_2\in\mathbb{Z}_{\ge 0}}}\right)$ is the projective system and it is easily seen that a projective limit $\displaystyle \lim_{\longleftarrow } \AdDMR_0^{(n)}(R)$ is equal to $\AdDMR_0$.
Let $\pi_n : \AdDMR_0(R) \rightarrow \AdDMR_0^{(n)}(R)$ be a natural projection.

\begin{lem}\label{lem:difference}
For each $n \in \mathbb{Z}_{\ge 0}$, and $\Phi_1$, $\Phi_2 \in \AdDMR_0^{(n+1)}(R)$ with $\pi_{n+1,n}(\Phi_1) = \pi_{n+1,n}(\Phi_2)$,
we have $\Phi_1 - \Phi_2 \in \addmr(R)$ with weight homogeneous degree $n$.
\end{lem}

\begin{proof}
Let us consider the following decomposition $\Phi_i := x_1 + \Phi_i^{(2)} + \cdots + \Phi_i^{(n)}$ for $i = 0$. $1$.
Then, for each $2\le j \le n$, $x_1 + \Phi_i^{(2)} + \cdots \Phi_i^{(j)} \in \AdDMR_0^{(j+1)}(R)$. 
Therefore, 
\[
\langle \Phi_1 - \Phi_2 \mid x_1 \rangle = 0,
\]
\begin{align*}
&\Delta_{\shuffle}(\Phi_1 - \Phi_2)\\
=&\Phi_1 \otimes 1 + 1 \otimes \Phi_1 - \Phi_2 \otimes 1 + 1 \otimes \Phi_2\\
=&\left(\sum_{1 \le j \le n-1} \Phi_1^{j}\right) \otimes 1 + 1\otimes \left(\sum_{1 \le j \le n-1} \Phi_1^{j}\right)  +\Phi_1^{n} \otimes 1 + 1\otimes  \Phi_1^{n} \\
&-\left(\sum_{1 \le j \le n-1} \Phi_1^{j} \right) \otimes 1 - 1\otimes \left( \sum_{1 \le j \le n-1} \Phi_1^{j}\right) - \Phi_2^{n} \otimes 1 - 1\otimes  \Phi_2^{n}\\
=&(\Phi_1^n - \Phi_2^n) \otimes 1 - 1 \otimes (\Phi_1 - \Phi_2),
\end{align*}

and

\begin{align*}
&\Delta_{*}(\Phi_{1,\#} - \Phi_{2,\#})\\
=&\sum_{k + l = n}\left( \Phi_{1,\#}^{(k)} \otimes \Phi_{1,\#}^{(l)} \right)+ \Phi_{1,\#}^{(n)} \otimes 1 + 1\otimes  \Phi_{1,\#}^{(n)}\\
&-\sum_{k + l = n}\left( \Phi_{2,\#}^{(k)} \otimes \Phi_{2,\#}^{(l)} \right)+ \Phi_{2,\#}^{(n)} \otimes 1 + 1\otimes  \Phi_{2,\#}^{(n)}\\
=&(\Phi^{(n)}_{1,\#} -  \Phi^{(n)}_{2,\#})\otimes 1 + 1\otimes  (\Phi^{(n)}_{1,\#} - \Phi^{(n)}_{2,\#}).
\end{align*}
holds.
\end{proof}

\begin{lem}\label{lem:transitivity}
For each $n \in \mathbb{Z}_{\ge 0}$, the action of $\exp(\addmr(R) \cap \Fad(R) \cap V_{\strprty}(R))$ to $\AdDMR_0^{(n)}(R)$ and $\AdDMR_0(R)$ is transitive.
\end{lem}

\begin{proof}
We prove by  induction on $n \in \mathbb{Z}_{\ge 0}$.
The case where $n = 1$, $2$, and $3$ is trivial because $\AdDMR_0^{(n)}(R) = \{x_1\}$.
Assume that the action of $\exp(\addmr(R) \cap \Fad(R) \cap V_{\strprty}(R))$ to $\AdDMR_0^{(n)}(R)$ is transitive.
Then, for $\Phi_1$, $\Phi_2 \in \AdDMR_0^{n+1}(R)$, there exists $\psi \in \addmr(R)$ such that
\[
\pi_{n+1,n}(\exp(d_{\psi_n})(\Phi_1)) = \pi_{n+1,n}(\Phi_2).
\]
Thus by Lemma \ref{lem:difference}, 
\[
\psi':=\pi_{n+1}(\exp(d_{\psi_n})(\Phi_1)) - \Phi_2 \in \addmr(R).
\]
and $\wt \psi = n$. 

Let us consider $\pi_{n}\circ \exp(d_{\psi'})$. Since, for any $x \in \ide{h}^{(n)}$ with $\langle x \mid x_1 \rangle = 1$, we have
\[
\langle \exp(d_{\psi'})(x) = x + \psi' \mod \Q \cdot X^{n+1},
\]
Therefore, 
\begin{align*}
&\pi_{n+1}\circ \exp(d_{\psi'}) \circ \exp(d_{\psi_n})(\Phi_1)\\
=&\pi_{n+1}(\psi' + \exp(d_{\psi_n})(\Phi_1))\\
=&\Phi_2.
\end{align*}
Put $\psi_{n+1} := \BCH(\psi_n,\psi')$.
Here $\BCH$ is the BaKer-Cambell-Hausdorff series.
Since $\psi_n$, $\psi' \in \addmr(R)\cap \Fad(R) \cap V_{\strprty}(R)$, $\BCH(\psi_n.\psi')  \in \addmr(R)\cap \Fad(R) \cap V_{\strprty}(R)$.
Addtionally, since $\addmr(R) \cap \Fad(R) \cap V_{\strprty}(R)\ni \psi \mapsto d_{\psi} \in \ide{gl}(\ide{h}^{\vee})$ is a Lie algebra homomorphism, $d_{\BCH(\psi_n,\psi')} = \BCH(d_{\psi_n}, d_{\psi'})$ holds.
Therefore, 
\begin{align*}
\pi_{n+1}(\exp(d_{\psi_{n+1}})(\Phi_1)) =& \pi_{n+1}(\exp(d_{\BCH(\psi_n,\psi')})(\Phi_1))\\
=&\pi_{n+1}(\exp(\BCH(d_{\psi_n},d_{\psi'}))(\Phi_1))\\
=&\pi_{n+1}(\exp(d_{\psi_n}) \circ \exp(d_{\psi'})(\Phi_1))\\
=&\Phi_2.
\end{align*}
In the second equality above, we use the property of Backer-Cambell-Hausdorff series. 
The transitivity of the action of $\exp(\addmr(R) \cap \Fad(R) \cap V_{\strprty}(R))$ to $\AdDMR_0(R)$ is follows from taking projective limit of the action $\exp(\addmr(R) \cap \Fad(R) \cap V_{\strprty}(R))$ to $\AdDMR_0^{(n)}(R)$.
\end{proof}

\begin{proof}[A proof of Main Theorem \ref{mt:AdDMR-Grp}]\verb| |

By lemma \ref{lem:transitivity}, 

\[
\exp^{\circledast_1}(\addmr(R) \cap \Fad(R) \cap V_{\strprty}(R) ) \circledast_1 \Phi = \AdDMR_0(R).
\]
for any $\Phi \in \AdDMR_0(R)$. Especially, if $\Phi = x_1$, we have
\[
\exp^{\circledast_1}(\addmr(R) \cap \Fad(R) \cap V_{\strprty}(R) ) = \exp^{\circledast_1}(\addmr(R) \cap \Fad(R) \cap V_{\strprty}(R) ) \circledast_1 x_1 = \AdDMR_0(R).
\]

\end{proof}
\end{comment}


\begin{comment}
\begin{proof}[A proof of Main Theorem \ref{mt:AdDMR-Grp}]\verb| |

Since $\addmr(R) \cap \Fad(R) \cap V_{\strprty}(R)$ is a Lie subalgebra of $\tm_1(R)$, $\exp^{\circledast_1}(\addmr(R) \cap \Fad(R) \cap V_{\strprty}(R)$ is a subgroup of $\TM_1(R)$.
We now show that $\AdDMR(R)\cap \FAd(R) \cap \exp(V_{\strprty}(R)) = \exp^{\circledast_1}(\addmr(R) \cap \Fad(R) \cap V_{\strprty}(R))$.

First we show $\AdDMR(R)\cap \FAd(R) \cap \exp(V_{\strprty}(R)) \supset \exp^{\circledast_1}(\addmr(R) \cap \Fad(R) \cap V_{\strprty}(R))$. 
Let $\psi \in  \exp^{\circledast_1}(\addmr(R) \cap \Fad(R) \cap V_{\strprty}(R))$. 
It suffices to show $\Delta_*(\exp^{\circledast_1}(\psi)) = \exp^{\circledast_1}(\psi) \otimes \exp^{\circledast_1}(\psi)$.
By Corollary \ref{cor:essential} to the third equality below, we have
\begin{align*}
\Delta_*(\exp^{\circledast_1}(\psi)) =& \Delta_*\left(\left(\sum_{n\ge 0} \frac{d_{\psi}^n}{n!}(x_1)\right)_{\#}\right) \displaybreak[3]\\
=&\sum_{n\ge 0} \frac{(\Delta_* \circ d_{\psi}^n(x_1))_{\#}}{n!} \displaybreak[3]\\
=&\sum_{n\ge 0} \sum_{n_1 + n_2 = n} \binom{n}{n_1} \frac{1}{n!} (d_{\psi}^{n_1}(x_1) \otimes d_{\psi}^{n_2}(x_1)) \displaybreak[3]\\
=&\sum_{n\ge 0} \sum_{n_1 + n_2 = n} \left(\frac{1}{n_1!} d_{\psi}^{n_1}(x_1)\right) \otimes \left( \frac{1}{n_2!} d_{\psi}^{n_2}(x_1)\right) \displaybreak[3]\\
=&\left(\sum_{n_1 \ge 0}\frac{d_{\psi}^{n_1}(x_1)}{n_1!}\right) \otimes \left(\sum_{n_2 \ge 0}\frac{d_{\psi}^{n_2}(x_1)}{n_2!}\right) \\
=&\exp^{\circledast_1}(\psi) \otimes \exp^{\circledast_1}(\psi)
\end{align*}
as claimed.

Next we show the opposite inclusion.
\end{proof}
\end{comment}


\ifdefined\isMainDocument
\else
    %\bibliographystyle{amsplain}
    %\bibliography{reference-adjoint}
    \end{document}
\fi