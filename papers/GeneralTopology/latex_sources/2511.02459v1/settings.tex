%%% general packages
\usepackage{amsmath, amsthm, amssymb}

%%% typesetting packages
\usepackage{microtype}
\usepackage[hyphens]{url}
\def\UrlBreaks{\do\/\do-}
\usepackage{colonequals}
\usepackage{caption}
\usepackage[position=b]{subcaption}
\usepackage[section]{placeins}
\usepackage{mathtools}
\usepackage{appendix}
\usepackage{listofitems}
\usepackage[dvipsnames]{xcolor}


\definecolor{codegreen}{rgb}{0,0.6,0}
\definecolor{codepurple}{rgb}{0.58,0,0.82}

\usepackage{listings}
\lstset{%
  language=Python,
  commentstyle=\color{codegreen},
  keywordstyle=\color{magenta},
  stringstyle=\color{codepurple},
  basicstyle=\fontfamily{pcr}\selectfont\scriptsize,  % footnotesize?
  commentstyle=\bfseries,
  numbers=left,
}

%%% Numbering and environments
%%% see http://tex.stackexchange.com/questions/41299/how-to-identify-the-counter-of-equation-theorem-and-section                                                    

\makeatletter
\let\c@equation\c@subsection
\let\c@figure\c@equation
\let\c@table\c@equation
\makeatother

\numberwithin{equation}{section}
\numberwithin{figure}{section}
\numberwithin{table}{section}

\theoremstyle{plain}
\newtheorem{theorem}[equation]{Theorem}
\newtheorem{corollary}[equation]{Corollary}
\newtheorem{lemma}[equation]{Lemma}
\newtheorem{conjecture}[equation]{Conjecture}
\newtheorem{proposition}[equation]{Proposition}

\theoremstyle{definition}
\newtheorem{definition/}[equation]{Definition}
\newtheorem{example/}[equation]{Example}
\newtheorem{question/}[equation]{Question}
\newtheorem{algorithm/}[equation]{Algorithm}
\newtheorem*{keywords}{keywords}

\theoremstyle{remark}
\newtheorem{remark/}[equation]{Remark}
\newtheorem{remarks}[equation]{Remarks}

\newtheoremstyle{dotless}{}{}{}{}{\bfseries}{}{ }{}
\theoremstyle{dotless}
\newtheorem*{ccode}{Mathematics Subject Classification 2010:}

%%% https://tex.stackexchange.com/questions/291346/marking-the-end-of-a-definition
%%% Using egreg's solution                                                      
\newenvironment{definition}
  {\renewcommand{\qedsymbol}{$\Diamond$}%
   \pushQED{\qed}\begin{definition/}}
  {\popQED\end{definition/}}

\newenvironment{example}
  {\renewcommand{\qedsymbol}{$\Diamond$}%
   \pushQED{\qed}\begin{example/}}
  {\popQED\end{example/}}

\newenvironment{question}
  {\renewcommand{\qedsymbol}{$\Diamond$}%
   \pushQED{\qed}\begin{question/}}
  {\popQED\end{question/}}

\newenvironment{remark}
  {\renewcommand{\qedsymbol}{$\Diamond$}%
   \pushQED{\qed}\begin{remark/}}
  {\popQED\end{remark/}}

\usepackage[outline]{contour}
\contourlength{0.1em}
\newcommand{\outline}[1]{\contour*{white}{#1}}

\usepackage{tikz}
\usetikzlibrary{calc}
\usetikzlibrary{patterns}
\usetikzlibrary{fadings}
\usetikzlibrary{shapes.misc}
\usetikzlibrary{shadows.blur}
\usetikzlibrary{decorations.pathreplacing}
\usetikzlibrary{decorations.pathmorphing}
\usetikzlibrary{decorations.markings}
\usetikzlibrary{shapes.geometric}
\tikzset{dot/.style={draw,shape=circle,fill=black,scale=0.4}}
% Make the bounded boxes just a little bit bigger:
\tikzset{
every picture/.style={
  execute at end picture={
    \path (current bounding box.south west) +(-0.05, -0.05) (current bounding box.north east) +(0.05, 0.05);
    }
  }
}

\newcommand{\sqdiamond}{\tikz [x=1.2ex,y=1.85ex,line width=.1ex,line
join=round, yshift=-0.285ex] \draw  (0,0.5) -- (0.75,1) -- (1.5,0.5) -- (0.75,0) -- cycle;}
\renewcommand{\Diamond}{$\sqdiamond$} %%% fixes diamond

\newcommand{\bdy}{\partial}
\newcommand{\from}{\colon}
\newcommand{\cross}{\times}
\newcommand{\homeo}{\cong}
\newcommand{\defeq}{\colonequals}

\DeclareMathOperator{\MCG}{MCG}
\DeclareMathOperator{\Mod}{Mod}
\DeclareMathOperator{\Torelli}{Torelli}
\DeclareMathOperator{\GL}{GL}
\DeclareMathOperator{\intersection}{\iota}
\DeclareMathOperator{\move}{\textsc{Mov}}
\DeclareMathOperator{\update}{\textsc{UMov}}
\DeclareMathOperator{\ind}{index} %%% index
\DeclareMathOperator{\shift}{shift} %%% shift
\DeclareMathOperator{\trunc}{trunc} %%% truncate
\DeclareMathOperator{\DCoord}{\textsc{DeltaCoordinate}}
\DeclareMathOperator{\poly}{poly}

\newcommand{\tightness}[1]{\# #1}
\newcommand{\complexity}[1]{||#1||}
\newcommand{\complexitybnd}[1]{\lceil \! \lceil #1 \rceil \! \rceil}
\newcommand{\sharedsize}[1]{||#1||^\cap}
\newcommand{\sharedsizebnd}[1]{\lceil \! \lceil #1 \rceil \! \rceil^\cap}
\newcommand{\uncertainty}[1]{\epsilon( #1 )}
\newcommand{\pair}[1]{\langle #1 \rangle}
\newcommand{\length}[1]{\ell(#1)}

\newcommand{\closure}[1]{\overline{#1}}

\newcommand{\fdiv}{%
  \mathbin{%
    \mkern-6mu
    \begin{tikzpicture}[baseline = -0.5ex]
      \node[inner sep=0pt,outer sep=0pt,rotate=-110] at (0,0) {$\xrightarrow{}$};
    \end{tikzpicture}
    \mkern-3mu
  }%
}%
\newcommand{\cdiv}{%
  \mathbin{%
    \mkern-6mu
    \begin{tikzpicture}[baseline = -0.5ex]
      \node[inner sep=0pt,outer sep=0pt,rotate=-110] at (0,0) {$\xleftarrow{}$};
    \end{tikzpicture}
    \mkern-3mu
  }%
}%

% \newcommand{\calA}{\mathcal{A}}
\newcommand{\calC}{\mathcal{C}}
\newcommand{\Tau}{\boldsymbol{\tau}}
\newcommand{\Id}{\textnormal{Id}}
\newcommand{\II}{\mathbb{I}}
\newcommand{\ZZ}{\mathbb{Z}}

% Uniform constants.
\newcommand{\bee}{\mathcal{B}}  % An upper bound on the number of branches.
\newcommand{\cee}{\mathcal{C}}
\newcommand{\dee}{\mathcal{D}}
\newcommand{\eee}{\mathcal{E}}
\newcommand{\fee}{\mathcal{F}}

\newcommand{\inlineand}{\quad \textrm{and} \quad}
\newcommand{\inlineQED}{\pushQED{\qed} \qedhere \popQED}
\newcommand{\naive}{na\"{i}ve}

\newcommand{\blfootnote}[1]{%
  \begingroup
  \renewcommand\thefootnote{}\footnote{#1}%
  \addtocounter{footnote}{-1}%
  \endgroup
}

\usepackage[pdfborderstyle={},pdfborder={0 0 0},pagebackref,pdftex]{hyperref}
\renewcommand{\backreftwosep}{\backrefsep}
\renewcommand{\backreflastsep}{\backrefsep}
\renewcommand*{\backref}[1]{}
\renewcommand*{\backrefalt}[4]{
  \ifcase #1 %
   [No citations.]%
  \or
   [#2]%
  \else
   [#2]%
  \fi
}

%%% Pseudocode %%%
\usepackage{algorithm}
\usepackage{algpseudocode}
\renewcommand{\algorithmicrequire}{\textbf{Input:}}
\renewcommand{\algorithmicensure}{\textbf{Output:}}
\algnewcommand{\IfThen}[2]{\State \algorithmicif\ #1\ \algorithmicthen\ #2}

% Clever references.
% https://mirror.ox.ac.uk/sites/ctan.org/macros/latex/contrib/cleveref/cleveref.pdf
\usepackage[capitalise,noabbrev]{cleveref}
\newcommand{\crefpairconjunction}{~and~}
\crefname{conclusion}{Conclusion}{Conclusions}

%%% Appendix tables %%%
\usepackage{booktabs}
\usepackage{longtable}
\newcounter{rown}
\newcommand{\row}{\stepcounter{rown}\arabic{rown}}

% For preparation (remove once done).
\usepackage[textsize=tiny]{todonotes}