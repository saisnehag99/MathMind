\documentclass[12pt]{amsart}
\renewcommand{\subjclassname}{\textup{2020} Mathematics Subject Classification}
\usepackage[a4paper,margin=1.1in]{geometry}
\geometry{
    a4paper,
    left=2.5cm,
    right=2.5cm,
    top=2.5cm,
    bottom=2.5cm
}



\usepackage{amsmath,amsthm,amssymb}
\usepackage{hyperref}
%\usepackage[nameinlink,capitalize]{cleveref}

\usepackage{tikz-cd}

\newtheorem{theorem}{Theorem}[section]
\newtheorem{lemma}[theorem]{Lemma}
\newtheorem{proposition}[theorem]{Proposition}
\newtheorem{corollary}[theorem]{Corollary}
\theoremstyle{definition}
\newtheorem{definition}[theorem]{Definition}
\newtheorem{remark}[theorem]{Remark}
\newtheorem{example}[theorem]{Example}

\numberwithin{equation}{section}

\def\DD{D\kern-.7em\raise0.4ex\hbox{\char '55}\kern.33em}

\title[Jensen's Functional Equation on Involution-Generated Groups]
{Jensen's Functional Equation\\ on Involution-Generated Groups: An ($\mathrm{SR}_2$) Criterion\\ and Applications}



\author{\DD\d{\u a}ng V\~o Ph\'uc$^{*}$}
\address{Department of AI, FPT University, Quy Nhon AI Campus,\\
An Phu Thinh New Urban Area, Quy Nhon City, Binh Dinh, Vietnam}
\email{dangphuc150488@gmail.com}
\thanks{$^{*}$ORCID: \url{https://orcid.org/0000-0002-6885-3996}}

\keywords{Jensen's functional equation, groups generated by involutions,  group homomorphisms, symmetric groups, dihedral groups.}

\subjclass[2020]{Primary 39B52; Secondary 20F55.}


\begin{document}
\maketitle

\begin{abstract}
We study the Jensen functional equations on a group \(G\) with values in an abelian group \(H\):
\begin{align}
\tag{J1}\label{eq:J1}
f(xy)+f(xy^{-1})&=2f(x)\qquad(\forall\,x,y\in G),\\
\tag{J2}\label{eq:J2}
f(xy)+f(x^{-1}y)&=2f(y)\qquad(\forall\,x,y\in G),
\end{align}
with the normalization \(f(e)=0\).
Building on techniques for the symmetric groups \(S_n\), we isolate a structural criterion on \(G\)---phrased purely in terms of involutions and square roots---under which every solution to \eqref{eq:J1} must also satisfy \eqref{eq:J2} and is automatically a group homomorphism.
Our new criterion, denoted \((\mathrm{SR}_2)\), implies that $S_1(G,H) = S_{1,2}(G,H) = \mathrm{Hom}(G,H)$, applies to many reflection--generated groups and, in particular, recovers the full solution on \(S_n\).
Furthermore, we give a transparent description of the solution space in terms of the abelianization \(G/[G,G]\), and we treat dihedral groups \(D_m\) in detail, separating the cases \(m\) odd vs.\ even.
The approach is independent of division by \(2\) in \(H\) and complements the classical complex-valued theory that reduces \eqref{eq:J1} to functions on \(G/[G,[G,G]]\).
\end{abstract}


\section{Introduction}

On the real line, Jensen's equation captures convexity. On noncommutative groups, two canonical Jensen--type equations emerge:
\begin{align}
\tag{J1}
f(xy)+f(xy^{-1})&=2f(x)\\
\tag{J2}
f(xy)+f(x^{-1}y)&=2f(y).
\end{align}
A systematic study on groups was initiated by C.\,T.~Ng \cite{Ng1990, Ng2001, Ng2005}, who developed reduction formulas and solved Pexiderized variants on important classes (free groups, linear groups, semidirect products). Remarkably, in \cite{Stetkaer2003}, Stetk\ae r showed that for complex-valued solutions of \eqref{eq:J1}, every solution factors through the \(2\)-step derived quotient
\[
G\longrightarrow G/[G,[G,G]],
\]
and provided explicit solution formulas; moreover, the odd-solution quotient by homomorphisms is canonically isomorphic to \(\mathrm{Hom}([G,G]/[G,[G,G]],\mathbb{C})\).

In parallel, Jensen-type equations on (possibly noncommutative) semigroups with endomorphisms have recently been solved under 2-torsion-free hypotheses on the codomain, dropping involutivity of endomorphisms and expressing solutions via additive maps \cite{Akkaoui2023}; see also \cite{Aissi2024} relating Jensen-type and quadratic equations with endomorphisms. These developments largely rely either on divisibility by \(2\) in the codomain or on endomorphism structure in the domain.

\subsection*{Motivation and contribution}

The present work is motivated by the specific case of the symmetric group~$S_n$. The result that all Jensen solutions on~$S_n$ are homomorphisms was previously asserted by C.T.~Ng in his foundational work~\cite{Ng2001, Ng2005}, though a detailed proof was not published. Addressing this, Trinh and Hieu~\cite{Trinh2011} later provided the first direct and elementary proof of this important fact. The key observation in~\cite{Trinh2011} is that the product of any two transpositions in~$S_n$ is always a square. 
This property leads to two interesting consequences for any Jensen solution~$f$: first that~$2f\equiv 0$, and second that~$f$ is invariant under reordering of factors, from which additivity ($f(xy)=f(x)+f(y)$) follows.

In this paper, we propose a new structural criterion, which we call "square-root for products of involutions"~$(\mathrm{SR}_2)$. We prove that this criterion has a beautiful consequence: \textit{for any group satisfying $(\mathrm{SR}_2)$, every solution $f$ to the first Jensen equation \eqref{eq:J1} (with $f(e)=0$) must also satisfy the second \eqref{eq:J2} and be a group homomorphism}. This result is highly general, as our proof avoids the common technical assumptions on the codomain group~$H$ (such as divisibility by 2). The criterion's power is demonstrated by its broad applicability, providing a unified explanation for the symmetric group case and extending to a class of "reflection-generated" groups---groups of symmetries constructed from geometric flips. This class includes well-known examples such as certain dihedral groups and their important generalizations, the Coxeter groups.

Building on the arrived results, we then identify the solution spaces $S_1(G,H)$ and $S_{1,2}(G,H)$ as being identical to the group $\mathrm{Hom}(G/[G,G],H)$, providing an immediate combinatorial description whenever $G/[G,G]$ is a 2-group. Finally, we give a sharp dichotomy for the dihedral groups~$D_m$, showing that the mechanism works exactly when $m$ is odd.


\subsection*{Relation to prior work.}

Unlike the complex-valued theory in \cite{Stetkaer2003}, our approach does not divide by \(2\) in \(H\); unlike recent semigroup results \cite{Akkaoui2023,Aissi2024}, our criterion $(\mathrm{SR}_2)$ is a purely group-structural hypothesis on \(G\) (no endomorphisms required) and targets the full equality $S_1(G,H) = S_{1,2}(G,H)=\mathrm{Hom}(G,H)$, showing this follows from \eqref{eq:J1} alone.

\subsection*{Organization of the paper}
The paper is organized as follows. In Section \ref{s2}, we introduce our main structural condition, the $(\mathrm{SR}_2)$ property, and develop the necessary technical lemmas (Lemmas~\ref{lem:three-var-no-halves}--\ref{lem:reorder}). We then prove the main result (Theorem~\ref{thm:hom}), which states that for any group satisfying $(\mathrm{SR}_2)$, every solution to \eqref{eq:J1} is a group homomorphism. In Section \ref{s3}, we apply this theorem to key examples: we recover the known results for the symmetric groups $S_n$ (Theorem~\ref{thm:Sn}) and provide a detailed analysis of the dihedral groups $D_m$, establishing a sharp dichotomy between the $m$ odd and $m$ even cases (Theorem~\ref{thm:dihedral}). This section includes an explicit counterexample (Example~\ref{ex:D2k-counter}) for the even case. Finally, in Section \ref{s4}, we compare our approach to the classical complex-valued theory and recent results on semigroups.

\subsection*{Notation}

Throughout this paper, the group $G$ is written multiplicatively, while the abelian group $H$ is written additively. We also normalize our function by assuming that $f(e) = 0.$

For a group \(G\), write \(S_1(G,H)\) (resp.\ \(S_2(G,H)\)) for the solution sets of \eqref{eq:J1} (resp.\ \eqref{eq:J2}). Then, $S_{1,2}(G,H) := S_1(G,H)\cap S_2(G,H).$

We write
\[
G_{\mathrm{ab}}:=G/[G,G]
\]
for the abelianization of \(G\), and denote by \(\pi:G\to G_{\mathrm{ab}}\) the canonical projection (any homomorphism from \(G\) to an abelian group factors uniquely through \(\pi\)).


\section{A structural square-root criterion}\label{s2}

\begin{definition}[The \((\mathrm{SR}_2)\) property]\label{def:SR2}
Let \(G\) be a group and \(\mathcal I\subseteq G\) a set of involutions (\(i^2=e\)). We say that \(G\) satisfies \(\mathrm{SR}_2(\mathcal I)\) if:
\begin{enumerate}
    \item \(G=\langle\mathcal I\rangle\) (i.e.\ \(G\) is generated by involutions from \(\mathcal I\));
    \item for every \(a,b\in\mathcal I\) there exists \(t\in G\) with \(t^2=ab\).
\end{enumerate}
We write simply \((\mathrm{SR}_2)\) when \(\mathcal I\) is understood.
\end{definition}

\begin{remark}
For \(G=S_n\) with \(\mathcal I\) the set of transpositions, one checks directly that the product of any two transpositions is a square in the subgroup they generate; this is the combinatorial heart of the proof on \(S_n\): see \cite{Trinh2011}.
\end{remark}


\begin{lemma}[Basic identities without dividing by $2$]\label{lem:three-var-no-halves}
Let $G$ be a group and $H$ an abelian group. Suppose $f:G\to H$ satisfies the two Jensen identity \eqref{eq:J1} with $f(e)=0$. Then for all $x,y,z\in G$:
\begin{enumerate}
\item[(I)] Oddness and square rule:
      \[
      f(x^{-1})=-f(x),\qquad f(x^2)=2f(x).
      \]
\item[(II)] Three--variable switching formulas (explicit, no $1/2$):
      \begin{align}
      \label{eq:switchJ1}
      f(xyz)&=2f(x)-f\!\big(xz^{-1}y^{-1}\big),\\
      \label{eq:switchJ1b}
      f(xzy)&=2f(x)-f\!\big(xy^{-1}z^{-1}\big).
      \end{align}
      Consequently,
      \begin{equation}\label{eq:pair-difference}
      f(xyz)-f(xzy)=f\!\big(xy^{-1}z^{-1}\big)-f\!\big(xz^{-1}y^{-1}\big).
      \end{equation}
\end{enumerate}
\end{lemma}

\begin{proof}
Setting $x=e$ in \eqref{eq:J1} gives $f(y)+f(y^{-1})=2f(e)=0$, i.e. $f(y^{-1})=-f(y)$. With $y=x$ in \eqref{eq:J1} we get $f(x^2)+f(e)=2f(x)$, hence $f(x^2)=2f(x)$ since $f(e)=0$. This proves (I).

For (II), apply \eqref{eq:J1} with $(x,\,y)\mapsto \big(x,\,yz\big)$ and with $(x,\,y)\mapsto \big(x,\,zy\big)$ to obtain \eqref{eq:switchJ1}--\eqref{eq:switchJ1b}. Subtracting \eqref{eq:switchJ1b} from \eqref{eq:switchJ1} yields \eqref{eq:pair-difference}.  \qedhere
\end{proof}


\begin{remark}
Formulas \eqref{eq:switchJ1} and \eqref{eq:switchJ1b} are the precise "three--variable manipulations" we use later to permute adjacent factors at the level of $f$--values, and they never invoke division by $2$.
\end{remark}

We now present the main results of this work along with their proofs.

\begin{theorem}[Two involutions: torsion bounds and square roots]\label{lem:two-inv}
Let $G$ be a group, $H$ an abelian group, and $f:G\to H$ satisfy \textup{\eqref{eq:J1}} with $f(e)=0$.
Let $a,b\in G$ be involutions ($a^2=b^2=e$). Then:
\begin{enumerate}
\item[(I)] \label{item:inv-2torsion} $f(a^{-1})=-f(a)$ and $2f(a)=0$; likewise $f(b^{-1})=-f(b)$ and $2f(b)=0$. 
\item[(II)] \label{item:ab-2torsion} $2f(ab)=0$.
\item[(III)] \label{item:sr2-imp} If, in addition, \textup{(SR2)} holds for $a,b$ (so there is $t\in G$ with $t^2=ab$), then
\[
f(ab)=f(t^2)=2f(t)\quad\text{and hence}\quad 4f(t)=0.
\]
In particular, if one further knows that $f(ab)=0$ (e.g. from a later lemma establishing pair--vanishing, or when $H$ is $2$-torsion-free), then $2f(t)=0$.
\end{enumerate}
\end{theorem}

\begin{proof}
(I) Apply Lemma~\ref{lem:three-var-no-halves}(I): from \eqref{eq:J1} with $x=e$ we get $f(y^{-1})=-f(y)$, and with $y=x$ we obtain $f(x^2)=2f(x)$. For an involution $a=a^{-1}$ this gives $2f(a)=f(a^2)=f(e)=0$; similarly for $b$.

(II) Using \eqref{eq:J1} at $(x,y)=(a,b)$ and the fact $b^{-1}=b$,
\[
f(ab)+f(ab^{-1})=2f(a)\quad\Rightarrow\quad 2f(ab)=2f(a)=0
\]
by part (I).

(III) If \textup{(SR2)} provides $t$ with $t^2=ab$, then  Lemma~\ref{lem:three-var-no-halves}(I) yields
\[
f(ab)=f(t^2)=2f(t).
\]
Combining with part (II) gives $4f(t)=0$. If moreover $f(ab)=0$ (for instance because a later "pair--vanishing" lemma shows $f$ vanishes on any product of two involutions, or because $H$ is $2$-torsion-free so from $2f(ab)=0$ we get $f(ab)=0$), then necessarily $2f(t)=0$.
\end{proof}

%\begin{remark}[On the claim $f(a)+f(b)=0$]\label{rem:no-sum}
%From \eqref{eq:J1}--\eqref{eq:J2} one can only conclude $2(f(a)-f(b))=0$ (indeed, $2f(a)=2f(ab)=2f(b)$), but \emph{in general} there is no justification that $f(a)+f(b)=0$ without extra assumptions.
%For example, if $H$ is $2$-torsion-free and $f(ab)=0$, then \eqref{eq:J1}--\eqref{eq:J2} force $f(a)=f(b)=0$ (hence $f(a)+f(b)=0$), but this uses the additional structural condition on $H$.
%Theorem~\ref{lem:two-inv} therefore states only the universally valid $2$-torsion consequences and separates any stronger vanishing into explicit hypotheses.
%\end{remark}

\begin{lemma}\label{lem:twof-on-words}
Let $G$ be a group, $H$ an abelian group, and $f:G\to H$ satisfy \eqref{eq:J1} with $f(e)=0$.
If $g\in G$ can be written as a product of involutions, then
\[
2\,f(g)=0.
\]
In particular, if $G$ is generated by involutions, then $2f\equiv 0$ on $G$.
\end{lemma}

\begin{proof}
Write $g=i_1i_2\cdots i_r$ with each $i_k^2=e$. We prove by induction on $r$ that $2f(i_1\cdots i_r)=0$. For $r=0$ (i.e. $g=e$), $2f(g)=0$ since $f(e)=0$.  
Assume the claim holds for $<r$ and set $X=i_1\cdots i_{r-1}$ and $a=i_r$ (so $a^2=e$).
Applying \eqref{eq:J1} with $(x,y)=(X,a)$ gives
\[
f(Xa)+f(Xa^{-1})=2f(X).
\]
Because $a=a^{-1}$, the left-hand side is $2f(Xa)$. Hence
\[
2f(i_1\cdots i_r)=2f(Xa)=2f(X)=0
\]
by the induction hypothesis. This completes the induction.
\end{proof}

\begin{theorem}[Reordering invariance under $2f\equiv 0$]\label{lem:reorder}
Let $G$ be a group generated by involutions and $H$ an abelian group.
Assume $f:G\to H$ satisfies \textup{\eqref{eq:J1}} with $f(e)=0$.
If $2f\equiv 0$ on $G,$ then:
\begin{enumerate}
\item[(I)] For all $x,y,z\in G$, one has $f(xyz)=f(xzy)$.
\item[(II)] Consequently, for any word $g=i_1\cdots i_r$ in involutions, swapping any adjacent pair
$i_k,i_{k+1}$ does not change the value $f(g)$.
\end{enumerate}
\end{theorem}

\begin{proof}
(I) By Lemma~\ref{lem:three-var-no-halves}(II), we have the explicit switch identities:
\[
f(xyz)=2f(x)-f\!\big(xz^{-1}y^{-1}\big),\qquad
f(xzy)=2f(x)-f\!\big(xy^{-1}z^{-1}\big).
\]
Subtracting the second identity from the first yields the consequence:
\[
f(xyz)-f(xzy)=f(xy^{-1}z^{-1})-f(xz^{-1}y^{-1}).
\]
Applying \eqref{eq:J1} to $(xy^{-1},z)$ and $(xz^{-1},y)$ and using the hypothesis $2f\equiv0$ shows that $f(xy^{-1}z^{-1})=-f(xy^{-1}z)$ and $f(xz^{-1}y^{-1})=-f(xz^{-1}y)$.
Hence the difference becomes
\[
f(xyz)-f(xzy) = -f(xy^{-1}z)+f(xz^{-1}y).
\]
The switching identity \eqref{eq:switchJ1}, $f(uvw)=2f(u)-f(uw^{-1}v^{-1})$, combined with $2f\equiv0$ yields the rule $f(uvw)=-f(uw^{-1}v^{-1})$. Applying this to the term $f(xy^{-1}z)$ gives
\[
f(xy^{-1}z) = -f(x z^{-1} (y^{-1})^{-1}) = -f(xz^{-1}y).
\]
Therefore, the expression for the difference simplifies to
\[
f(xz^{-1}y) + f(xz^{-1}y) = 2f(xz^{-1}y),
\]
which is $0$ by Lemma~\ref{lem:twof-on-words}. Thus $f(xyz)=f(xzy)$, as claimed.

(II) The statement for adjacent swaps in a word follows by taking $x$ to be the prefix before the pair,
$y=i_k$, $z=i_{k+1}$.
\end{proof}

\begin{theorem}\label{thm:hom}
Let $G$ be a group generated by involutions that satisfies the \textup{(SR2)} condition, and let $H$ be an arbitrary abelian group.
If $f:G \to H$ satisfies \textup{\eqref{eq:J1}} and $f(e)=0$, then $f$ is a group homomorphism and also satisfies \textup{\eqref{eq:J2}}.

Consequently, under these conditions on $G$, the solution spaces are identical:
\[
S_1(G,H) = S_{1,2}(G,H) = \mathrm{Hom}(G,H)\cong \mathrm{Hom}(G_{\mathrm{ab}},H).
\]
\end{theorem}

\begin{proof}

Fix a generating set $\mathcal I$ of involutions of $G$. For $g\in G$ define the
\emph{involution-word length}
\[
\ell(g):=\min\{\,r\ge 0:\ g=i_1\cdots i_r\text{ with }i_k\in\mathcal I\,\}.
\]
(Existence follows since $G$ is generated by involutions.)

We will prove additivity $f(xy)=f(x)+f(y)$ by induction on $q:=\ell(y)$, keeping $x$ fixed. 

\smallskip
\emph{Case $q=0$.}\quad Trivial since $y=e$ and $f(e)=0$.

\smallskip

\emph{Case $q=1$.}\quad Let $y=j$ be an involution and define $c(j;x):=f(xj)-f(x)$.

\emph{Step A: $2$--torsion.} From \eqref{eq:J1} with $(x,y)=(x,j)$ and $j^{-1}=j$ we have
$2f(xj)=2f(x)$, hence
\begin{equation}\label{eq:c-2torsion}
2\,c(j;x)=0\qquad(\forall\,x\in G).
\end{equation}

\emph{Step B: reordering and absorption tools.}
By Theorem~\ref{lem:reorder}, $f(Zt^2j)=f(Zjt^2)$ for all $Z,t$.
Applying \eqref{eq:J1} to $(x,y)=(Zt,t)$ and using Lemma~\ref{lem:twof-on-words} ($2f\equiv0$ on words of involutions) yields the absorption identity
\begin{equation}\label{eq:absorb}
f(Zt^2)=-\,f(Z)\quad\forall\,Z,\, t\in G.
\end{equation}

\emph{Step C: two invariances for $c(j;\cdot)$.}
\begin{itemize}
\item[(i)] \textbf{Right--multiplying by $j$ preserves $c$.} Indeed,
\[
c(j;xj)=f(xjj)-f(xj)=f(x)-f(xj)= -\,c(j;x).
\]
By \eqref{eq:c-2torsion} we have $-c(j;x)=c(j;x)$, hence $c(j;xj)=c(j;x)$.

\item[(ii)] \textbf{Right--multiplying by a square $t^2$ preserves $c$.} Using reordering and equations \eqref{eq:c-2torsion}, \eqref{eq:absorb}, we get
\[
\begin{aligned}
c(j;xt^2)
&=f(xt^2j)-f(xt^2)
 =f(xjt^2)-f(xt^2)\\
&=-\,f(xj)-\big(-\,f(x)\big)
 =-\,c(j;x)
 =c(j;x).
\end{aligned}
\]
\end{itemize}

\emph{Step D: use (SR$_2$) to pass from $j$ and $t^2$ to any involution.}
Let $a$ be any involution. By (SR2) applied to the involutions $(j,a)$ there is $t$ with $t^2=ja$, hence left--multiplying by $j$ (using $j^2=e$) gives $a=jt^2$. Therefore, for every $x$,
\[
c(j;xa)=c(j;xjt^2)\stackrel{(ii)}{=}c(j;xj)\stackrel{(i)}{=}c(j;x).
\]
As $G$ is generated by involutions, this shows $c(j;x)$ is independent of $x$.
Evaluating at $x=e$ gives $c(j;e)=f(j)$; hence
\[
f(xj)-f(x)=c(j;x)=f(j)\quad\text{for all }x\in G.
\]
That is, $f(xj)=f(x)+f(j)$, completing the case $q=1$.

\emph{Inductive step for $q\ge 2$.}\quad We now prove the additivity $f(xy)=f(x)+f(y)$ by induction on $q=\ell(y)$. The base case $q=1$ (i.e., $f(xj)=f(x)+f(j)$ for $j\in\mathcal{I}$) was completed above.

Assume the statement holds for all elements $Y\in G$ with $\ell(Y)=q-1$. That is, we assume $f(xY)=f(x)+f(Y)$ for any $x\in G$.
Let $y\in G$ be an element with $\ell(y)=q \ge 2$. We can write $y$ as $y = Yj$, where $j\in\mathcal{I}$ and $\ell(Y)=q-1$.
We then compute as follows:
\begin{align*}
f(xy) &= f(x(Yj)) = f((xY)j) \\
      &= f(xY) + f(j) && \text{(by applying the base case $q=1$ to $X=xY$)} \\
      &= \big(f(x) + f(Y)\big) + f(j) && \text{(by the induction hypothesis on $Y$)} \\
      &= f(x) + \big(f(Y) + f(j)\big) && \text{(since $H$ is abelian)} \\
      &= f(x) + f(Yj) && \text{(by applying the base case $q=1$ again, with $x=Y$)} \\
      &= f(x) + f(y)
\end{align*}
This completes the inductive step. By the principle of induction, the additivity $f(xy)=f(x)+f(y)$ holds for all $x,y\in G$. 

\smallskip
Finally, we verify that $f$, now proven to be a homomorphism, also satisfies \eqref{eq:J2}. This follows from the homomorphism property and Lemma~\ref{lem:three-var-no-halves}(I) (which itself was derived from \eqref{eq:J1}):
\[
f(xy)+f(x^{-1}y) = \big(f(x)+f(y)\big) + \big(f(x^{-1})+f(y)\big) = \big(f(x)+f(x^{-1})\big) + 2f(y) = 2f(y).
\]
Thus, $f$ is a homomorphism that satisfies both Jensen equations. The proof of the theorem is complete.
\end{proof}

\begin{corollary}[Parity normal form under (SR$_2$)]\label{cor:parity}
Assume $G$ satisfies (SR$_2$) and let $f\in S_1(G,H)$ with $f(e)=0$. Then $f$ is a group homomorphism and:
\[
\exists\,u\in H[2]\ \text{such that}\quad f(i)=u\ \ \text{for every involution }i,
\]
and for any involution word $g=i_1\cdots i_r$,
\[
f(g)=
\begin{cases}
0,& r\ \text{even},\\
u,& r\ \text{odd}.
\end{cases}
\]
Here, $H[2]:=\{h\in H:\ 2h=0\}$ denotes the $2$-torsion subgroup of $H$.
\end{corollary}

\begin{proof}
Let $a,b$ be involutions. By \textup{(SR2)}, pick $t\in G$ with $t^2=ab$.
Since $G$ is generated by involutions, $t$ is a product of involutions; hence,
by Lemma~\ref{lem:twof-on-words}, $2f(t)=0$. Using Lemma~\ref{lem:three-var-no-halves}(I),
we obtain
\[f(ab)\;=\;f(t^2)\;=\;2f(t)\;=\;0.\]
Since $f$ is a homomorphism, $0=f(ab)=f(a)+f(b)$, hence $f(b)=-f(a)=f(a)$ because $2f(a)=0$ (Lemma~\ref{lem:three-var-no-halves}(I)). Thus all involutions have a common value $u\in H[2]$.

For any involution word $g=i_1\cdots i_r$, by additivity from the proof of Theorem \ref{thm:hom},
\(
f(g)=\sum_{k=1}^r f(i_k)=r\cdot u,
\)
which equals $0$ if $r$ is even and $u$ if $r$ is odd. The corollary follows.
\end{proof}


\section{Applications}\label{s3}

In this section, we demonstrate how the $(\mathrm{SR}_2)$ criterion applies to several important classes of groups such as the symmetric and dihedral groups. We note that the  results for these groups are already known (see \cite{Ng2001}). Our contribution here is to show that these results can be recovered in a unified and direct manner as immediate consequences of our general $(\mathrm{SR}_2)$ criterion. This highlights the structural insight provided by our approach.

\subsection{Symmetric groups}

We now show how our general criterion recovers the known result for the symmetric group. The fact that all Jensen solutions on~$S_n$ are homomorphisms, and therefore have the parity form stated below, was first established by C.T.~Ng~\cite{Ng2001} using properties of the quotient group~$G/(\text{squares})$. A direct, combinatorial proof was later given by Trinh and Hieu~\cite{Trinh2011}. Our contribution is to demonstrate that this result also follows as an immediate consequence of the~$(\mathrm{SR}_2)$ property.

\begin{theorem}[Symmetric group]\label{thm:Sn}
Let \(n\ge 2\) and let \(\mathcal I\) be the set of transpositions in \(S_n\).
Then \((\mathrm{SR}_2)\) holds for \(S_n\) with respect to \(\mathcal I\).
Consequently, by Theorem~\ref{thm:hom},
\[
S_1(S_n,H) = S_{1,2}(S_n,H)\;\cong\;\mathrm{Hom}(S_n,H).
\]
Moreover, every solution \(f\in S_1(S_n,H)\) with \(f(e)=0\) is determined by a choice of \(u\in H[2]\) and satisfies
\[
f(\sigma)=
\begin{cases}
0,& \text{if $\sigma$ even},\\
u,& \text{if $\sigma$ odd.}
\end{cases}
\]
\end{theorem}

\begin{proof}
Let $\tau_1,\tau_2$ be transpositions in $S_n$ with $n\ge 2$. We show that there exists $t\in S_n$ with $t^2=\tau_1\tau_2$. There are three possibilities for the pair $(\tau_1, \tau_2)$.

\medskip
\noindent\textbf{Case 1: $\tau_1 = \tau_2$.}\quad
In this case, their product is $\tau_1\tau_2 = \tau_1^2 = e$, the identity element. The condition \textup{(SR2)} requires finding $t\in S_n$ such that $t^2=e$. This is trivially satisfied by choosing $t=e$.

\medskip
\noindent\textbf{Case 2: $\tau_1, \tau_2$ are distinct and intersect in a single element.}
Write $\tau_1=(ab)$ and $\tau_2=(bc)$ with $a,b,c$ pairwise distinct. Then
\[
\tau_1\tau_2=(ab)(bc)=(abc),
\]
a $3$-cycle. Let $t:=(acb)$; since the square of a $3$-cycle is its inverse,
\[
t^2=(acb)^2=(abc)=\tau_1\tau_2.
\]

\medskip
\noindent\textbf{Case 3: $\tau_1,\tau_2$ are disjoint.}
Write $\tau_1=(ab)$ and $\tau_2=(cd)$ with $a,b,c,d$ pairwise distinct. Then
\[
\tau_1\tau_2=(ab)(cd).
\]
Let $t:=(acbd)$ be the $4$-cycle $a\mapsto c\mapsto b\mapsto d\mapsto a$. The square of a $4$-cycle is the product of the two opposite transpositions, hence
\[
t^2=(acbd)^2=(ab)(cd)=\tau_1\tau_2.
\]

\medskip
In all three cases there exists $t\in S_n$ with $t^2=\tau_1\tau_2$, so $(\mathrm{SR}_2)$ holds for $S_n$ with respect to the set of transpositions.
By Theorem~\ref{thm:hom} we obtain \(S_{1,2}(S_n,H)\cong\mathrm{Hom}(S_n,H)\).
Finally, since $(S_n)_{\mathrm{ab}}\cong C_2$ (the commutator subgroup is $A_n$) and $S_n$ is generated by involutions, Corollary~\ref{cor:parity} gives the stated parity formula with parameter \(u\in H[2]\).
\end{proof}


\subsection{Dihedral groups}

Our criterion ($\mathrm{SR}_2$) also provides a particularly insightful perspective on the dihedral groups. While the general result that all Jensen solutions on $D_m$ are homomorphisms is known \cite{Ng2001}, the $(\mathrm{SR}_2)$ property itself reveals a sharp dichotomy: it holds if and only if $m$ is odd. This provides a new structural explanation for why the odd-order case is particularly well-behaved under our approach.

\begin{theorem}[Dihedral dichotomy]\label{thm:dihedral}
Let $D_m=\langle r,s\mid r^m=e,\ s^2=e,\ srs=r^{-1}\rangle$ be the dihedral group, and let $\mathcal{I}=\{sr^k\mid 0\le k<m\}$ be its set of reflections.
\begin{enumerate}
    \item[(I)] If $m$ is odd, then $D_m$ satisfies $(\mathrm{SR}_2)$ with respect to $\mathcal{I}$. Consequently, $S_1(D_m,H) = S_{1,2}(D_m,H)=\mathrm{Hom}(D_m,H).$
    \item[(II)] If $m$ is even, $(\mathrm{SR}_2)$ may fail in general.
\end{enumerate}
\end{theorem}

\begin{proof}
(I) For reflections \(sr^i,sr^j\) we have \((sr^i)(sr^j)=r^{\,j-i}\).
When \(m\) is odd, \(2\) is invertible mod \(m\): choose \(u\) with \(2u\equiv 1\pmod m\).
Then for every \(k\) we have \(r^k=(r^{ku})^2\), so products of two reflections are squares.
Thus \((\mathrm{SR}_2)\) holds and Theorem~\ref{thm:hom} applies.

(II) For \(m\) even, not every rotation is a square. For example in \(D_4\),
\(s\cdot (sr)=s^{2}r = r\), while the set of squares is \(\{e,r^2\}\) (indeed, \((r^a)^2=r^{2a}\) and \((sr^a)^2=e\)).
Hence \(r\) is not a square and \((\mathrm{SR}_2)\) fails for the pair \((s,sr)\).
\end{proof}

\begin{remark}
Theorem~\ref{thm:dihedral}(II) established that the \textup{(SR2)} criterion fails when $m$ is even, as not all rotations are squares. This raises the question of whether the \emph{conclusion} of Theorem~\ref{thm:hom} (namely, $S_1(G,H) = \mathrm{Hom}(G,H)$) also fails. The following example confirms this by constructing an explicit solution to \eqref{eq:J1} that is not a group homomorphism.

\begin{example}[A non-homomorphic Jensen solution on $D_{2k}$ when $m=2k$ is even]\label{ex:D2k-counter}
Let $D_m$ be as in Theorem \ref{thm:dihedral} with $m=2k$ even, and let $(H,+)$ be an abelian group.
Fix $u,c\in H$ with $2u=0$ and $2c=0$ (i.e. $u,c\in H[2]$). Define $f:D_m\to H$ by
\[
f(r^{2t})=0,\qquad f(r^{2t+1})=u,\qquad f(sr^j)=c\quad (t\in\mathbb Z,\ j\in\mathbb Z/m\mathbb Z).
\]
Then $f$ satisfies the Jensen equation \eqref{eq:J1} for all $x,y\in D_m$, but $f$ is not a homomorphism provided $u\neq 0$.
\end{example}

\begin{proof}
\underline{Well-definedness.}
Because $m$ is even, the parity of the exponent $j$ is well-defined modulo $m$; hence $f(r^j)$ is unambiguous. Also $f(e)=f(r^0)=0$.

\smallskip
\underline{Verification of \eqref{eq:J1}.}
Since $f$ takes values in $H[2]$, we have $2f(x)=0$ for all $x$. Thus \eqref{eq:J1} is equivalent to
\[
f(xy)=f(xy^{-1})\qquad(\forall x,y\in D_m).
\]
We check this by cases, writing $y=r^j$ or $y=sr^j$ and $x=r^p$ or $x=sr^p$.

\begin{itemize}
\item[(i)] $x=r^p$, $y=r^j$. Then $xy=r^{p+j}$ and $xy^{-1}=r^{p-j}$. Since $p+j\equiv p-j\pmod 2$, both exponents have the same parity; hence $f(xy)=f(xy^{-1})$.

\item[(ii)] $x=r^p$, $y=sr^j$. Then $xy=r^psr^j$ and $xy^{-1}=r^pr^{-j}s=r^{p-j}s$. Both products are reflections, so $f(xy)=f(xy^{-1})=c$.

\item[(iii)] $x=sr^p$, $y=r^j$. Then $xy=sr^{p+j}$ and $xy^{-1}=sr^{p-j}$, again both reflections; hence $f(xy)=f(xy^{-1})=c$.

\item[(iv)] $x=sr^p$, $y=sr^j$. Using $sr^a s=r^{-a}$, we get
\[
xy=sr^p sr^j=r^{-p}r^j=r^{j-p},\qquad
xy^{-1}=sr^p(r^{-j}s)=sr^{p-j}s=r^{-(p-j)}=r^{j-p}.
\]
Thus $xy$ and $xy^{-1}$ are the same rotation (in particular have the same parity), and $f(xy)=f(xy^{-1})$.
\end{itemize}
In all cases $f(xy)=f(xy^{-1})$, so \eqref{eq:J1} holds.

\smallskip
\underline{$f$ is not a homomorphism when $u\neq 0$.}
We have $f(r)=u$ and $f(s)=c$, while $sr$ is a reflection, so $f(sr)=c$.
If $f$ were a homomorphism, we would need $f(sr)=f(s)+f(r)=c+u$, forcing $u=0$ (since $H$ is abelian). Hence for any $u\in H[2]\setminus\{0\}$ the map $f$ satisfies \eqref{eq:J1} but is not additive.
\end{proof}
\end{remark}


We conclude this section with two examples that generalize the preceding analysis and clarify the structure of the solution space identified in Theorem~\ref{thm:hom}.

\subsection{Consequences and examples}
\begin{example}[Coxeter-type situations]
Let \(W\) be a Coxeter group generated by reflections.
If for each rank--$2$ parabolic \(W_{ij}\cong D_{m_{ij}}\) the order \(m_{ij}\) is odd, then
the argument of Theorem~\ref{thm:dihedral}(1) applies in each rank--$2$ factor,
so products of two reflections are locally squares and \((\mathrm{SR}_2)\) can be verified by reduction to rank two.
A full Coxeter-theoretic characterization of \((\mathrm{SR}_2)\) is an interesting direction for future work.
\end{example}

\begin{example}[Explicit solutions under \((\mathrm{SR}_2)\)]
Fix \(H\) and choose any homomorphism \(\phi:G_{\mathrm{ab}}\to H\).
Then \(f=\phi\circ\pi\) (with \(\pi:G\to G_{\mathrm{ab}}\) the canonical projection) solves \eqref{eq:J1} and \eqref{eq:J2}.
Conversely, by Theorem~\ref{thm:hom} every solution arises this way.
\end{example}

\section{Final remark: Comparison with complex-valued theory\\ and semigroup variants}\label{s4}

When \(H=\mathbb{C}\), Stetk\ae r \cite{Stetkaer2003} proves
\[
f(xy)=f(x)+f(y)+\tfrac12 f([x,y])
\]
on \(G/[G,[G,G]]\), and characterizes the odd-solution quotient by homomorphisms.
Our criterion \((\mathrm{SR}_2)\) can be seen as a group-structural hypothesis ensuring the commutator correction term vanishes without dividing by two in the codomain.

On semigroups with endomorphisms, recent results \cite{Akkaoui2023,Aissi2024}
solve generalized Jensen equations assuming \(H\) is $2$-torsion-free and at least one endomorphism is surjective; our results are orthogonal, as we require no endomorphisms but impose \((\mathrm{SR}_2)\) on the source group.




\begin{thebibliography}{99}

\bibitem{Aissi2024}
Y. Aissia, D. Zeglamia, and A. Mouzoun, \emph{On a Pexider--Drygas functional equation on semigroups with an endomorphism}, Filomat \textbf{38} (2024), 11159--11169. DOI: \href{https://doi.org/10.2298/FIL2431159A}{10.2298/FIL2431159A}.

\bibitem{Akkaoui2023}
A.~Akkaoui, \emph{Jensen's functional equation on semigroups}, Acta Math.\ Hungar.\ \textbf{170} (2023), 261--268.
DOI: \href{https://doi.org/10.1007/s10474-023-01341-7}{10.1007/s10474-023-01341-7}.

\bibitem{Ng1990}
C.\,T.~Ng, \emph{Jensen's functional equation on groups}, Aequationes Math.\ \textbf{39} (1990), 85--99.
Available via EuDML: \url{https://eudml.org/doc/137340}.

\bibitem{Ng2001}
C.\,T.~Ng, \emph{Jensen's functional equation on groups, III}. Aequationes Math.\ \textbf{62} (2001), 143--159. 
DOI: \href{https://doi.org/10.1007/PL00000135}{10.1007/PL00000135}.

\bibitem{Ng2005}
C.\,T.~Ng, \emph{A Pexider--Jensen functional equation on groups}, Aequationes Math.\ \textbf{70} (2005), 131--153.
DOI: \href{https://doi.org/10.1007/s00010-005-2785-7}{10.1007/s00010-005-2785-7}.

\bibitem{Stetkaer2003}
H.~Stetk\ae r, \emph{On Jensen's functional equation on groups}, Aequationes Math.\ \textbf{66} (2003), 100--118.
DOI: \href{https://doi.org/10.1007/s00010-003-2679-5}{10.1007/s00010-003-2679-5}.

\bibitem{Trinh2011}
L.C.~Trinh, and T.T.~Hieu, \emph{Jensen's functional equation on the symmetric group \(S_n\)}, Aequationes Math.\ \textbf{82} (2011), 269--276. DOI: \href{https://doi.org/10.1007/s00010-011-0089-7}{10.1007/s00010-011-0089-7}.



\end{thebibliography}
\end{document}


