\ifdefined\isMainDocument
\else
    \documentclass[10pt,oneside,reqno]{amsart}
    \usepackage{xr}
    \externaldocument{../main} % Main の .aux を参照
    \usepackage{latexsym}
\usepackage{amscd}
\usepackage{amsmath}
\usepackage{amssymb}
\usepackage[mathscr]{euscript}
\usepackage{graphicx}
\usepackage{geometry}
\usepackage{mathtools}
\usepackage[all]{xy}
\usepackage[dvipsnames]{xcolor}
\usepackage[colorlinks,final,hyperindex]{hyperref}
\usepackage{tikz}
\usepackage{tikz-cd}
\usetikzlibrary{decorations.pathmorphing,decorations.markings,arrows,calc,shapes.geometric,arrows.meta,positioning}
\usepackage{enumerate}
\usepackage{multirow}

\usepackage[bb=dsserif]{mathalpha}
\usepackage{bm}

\newcommand{\QQ}{\ensuremath{\mathbb{Q}}}
\newcommand{\DD}{\ensuremath{\mathbb{D}}}
\newcommand{\CC}{\ensuremath{\mathbb{C}}}
\newcommand{\RR}{\ensuremath{\mathbb{R}}}
\newcommand{\ZZ}{\ensuremath{\mathbb{Z}}}
\newcommand{\NN}{\ensuremath{\mathbb{N}}}

\newcommand{\g}{\ensuremath{\mathfrak{g}}}
\renewcommand{\SS}{\ensuremath{\mathbb{S}}}

\newcommand{\id}{\ensuremath{\mathrm{id}}}

\makeatletter

\newcommand{\bbone}{\mathbb{1}}

\newcommand{\calA}{\mathcal{A}}
\newcommand{\calB}{\mathcal{B}}
\newcommand{\calC}{\mathcal{C}}
\newcommand{\calD}{\mathcal{D}}
\newcommand{\calE}{\mathcal{E}}
\newcommand{\calF}{\mathcal{F}}
\newcommand{\calG}{\mathcal{G}}
\newcommand{\calH}{\mathcal{H}}
\newcommand{\calI}{\mathcal{I}}
\newcommand{\calJ}{\mathcal{J}}
\newcommand{\calK}{\mathcal{K}}
\newcommand{\calL}{\mathcal{L}}
\newcommand{\calM}{\mathcal{M}}
\newcommand{\calN}{\mathcal{N}}
\newcommand{\calO}{\mathcal{O}}
\newcommand{\calP}{\mathcal{P}}
\newcommand{\calQ}{\mathcal{Q}}
\newcommand{\calR}{\mathcal{R}}
\newcommand{\calS}{\mathcal{S}}
\newcommand{\calT}{\mathcal{T}}
\newcommand{\calU}{\mathcal{U}}
\newcommand{\calV}{\mathcal{V}}
\newcommand{\calW}{\mathcal{W}}
\newcommand{\calX}{\mathcal{X}}
\newcommand{\calY}{\mathcal{Y}}
\newcommand{\calZ}{\mathcal{Z}}
\newcommand{\kk}{\Bbbk}

\newcommand{\scrY}{\mathscr{Y}}
\newcommand{\scrV}{\mathscr{V}}
\newcommand{\scrP}{\mathscr{P}}

\newcommand{\Rep}{\mathop{\mathrm{Rep}}\nolimits}
\newcommand{\alg}{\mathop{\mathrm{alg}}\nolimits}
\newcommand{\Span}{\mathop{\mathrm{Span}}\nolimits}
\newcommand{\proj}{\mathop{\mathrm{proj}}\nolimits}
\newcommand{\Hom}{\mathop{\mathrm{Hom}}\nolimits}
\newcommand{\PHom}{\mathop{\mathrm{PHom}}\nolimits}
\newcommand{\Tw}{\mathop{\mathrm{Tw}}\nolimits}
\newcommand{\Dim}{\mathop{\mathrm{Dim}}\nolimits}
\newcommand{\Imm}{\mathop{\mathrm{Im}}\nolimits}
\newcommand{\Aus}{\mathop{\mathrm{Aus}}\nolimits}
\newcommand{\Ann}{\mathop{\mathrm{Ann}}\nolimits}
\newcommand{\APC}{\mathop{\mathrm{APC}}\nolimits}
\newcommand{\Barr}{\mathop{\mathrm{Bar}}\nolimits}
\newcommand{\Cone}{\mathop{\mathrm{Cone}}\nolimits}
\newcommand{\modu}{\mathop{\mathrm{mod}}\nolimits}
\newcommand{\dgMod}{\mathop{\mathrm{dgMod}}\nolimits}
\newcommand{\soc}{\mathop{\mathrm{soc}}\nolimits}
\newcommand{\dg}{\mathop{\mathrm{dg}}\nolimits}
\newcommand{\red}{\mathop{\mathrm{red}}\nolimits}
\newcommand{\Map}{\mathop{\mathrm{Map}}\nolimits}
\newcommand{\inj}{\mathop{\mathrm{inj}}\nolimits}
\newcommand{\op}{\mathop{\mathrm{op}}\nolimits}
\newcommand{\Ker}{\mathop{\mathrm{Ker}}\nolimits}
\newcommand{\Tor}{\mathop{\mathrm{Tor}}\nolimits}
\newcommand{\Tot}{\mathop{\mathrm{Tot}}\nolimits}
\newcommand{\Ext}{\mathop{\mathrm{Ext}}\nolimits}
\newcommand{\sg}{\mathop{\mathrm{sg}}\nolimits}
\newcommand{\Sl}{\mathop{\mathfrak{sl}}\nolimits}
\newcommand{\nc}{\mathop{\mathrm{nc}}\nolimits}
\newcommand{\Tr}{\mathop{\mathrm{Tr}}\nolimits}
\newcommand{\Irr}{\mathop{\mathrm{Irr}}\nolimits}
\newcommand{\HC}{\mathop{\mathrm{HC}}\nolimits}
\newcommand{\HH}{\mathop{\mathrm{HH}}\nolimits}
\newcommand{\THH}{\mathop{\mathrm{TH}}\nolimits}
\newcommand{\Perf}{\mathop{\mathrm{Perf}}\nolimits}
\newcommand{\del}{\partial}
\newcommand{\ev}{\mathop{\mathrm{ev}}\nolimits}
\newcommand{\co}{\mathop{\mathrm{co}}\nolimits}
\newcommand{\pt}{\mathop{\mathrm{pt}}\nolimits}
\newcommand{\wt}{\mathop{\mathrm{wt}}\nolimits}
\newcommand{\pCY}{\mathop{\mathrm{pCY}}\nolimits}
\newcommand{\gr}{\mathop{\mathrm{gr}}\nolimits}
\newcommand{\coker}{\mathop{\mathrm{coker}}\nolimits}


\newcommand{\Top}{\mathop{\mathrm{Top}}\nolimits}
\newcommand{\Set}{\mathop{\mathrm{Set}}\nolimits}
\newcommand{\Vect}{\mathop{\mathrm{Vect}}\nolimits}
\newcommand{\Mod}{\mathop{\mathrm{Mod}}\nolimits}
\newcommand{\Ob}{\mathop{\mathrm{Ob}}\nolimits}
\newcommand{\Ind}{\mathop{\mathrm{Ind}}\nolimits}
\newcommand{\Sym}{\mathop{\mathrm{Sym}}\nolimits}
\newcommand{\Alt}{\mathop{\mathrm{Alt}}\nolimits}
\newcommand{\End}{\mathop{\mathrm{End}}\nolimits}
\newcommand{\Ch}{\mathop{\mathrm{Ch}}\nolimits}


\newcommand{\Ass}{\mathop{\mathrm{Ass}}\nolimits}
\newcommand{\Com}{\mathop{\mathrm{Com}}\nolimits}
\newcommand{\Lie}{\mathop{\mathrm{Lie}}\nolimits}
\newcommand{\colim}{\mathop{\mathrm{colim}}\nolimits}

\newcommand{\Op}{\mathop{\mathsf{Operads}}\nolimits}
\newcommand{\Diop}{\mathop{\mathsf{Dioperads}}\nolimits}
\newcommand{\PDiop}{\mathop{\mathsf{PDioperads}}\nolimits}
\newcommand{\Coop}{\mathop{\mathsf{Cooperads}}\nolimits}
\newcommand{\Codiop}{\mathop{\mathsf{Codioperads}}\nolimits}
\newcommand{\PCodiop}{\mathop{\mathsf{PCodioperads}}\nolimits}

\newcommand{\antish}{\ensuremath{\mbox{!`}}}


\tikzset{->-/.style={decoration={
			markings,
			mark=at position #1 with {\arrow{>}}},postaction={decorate}}}
\tikzset{-w-/.style={decoration={
			markings,
			mark=at position #1 with {\arrow{Stealth[fill=white,scale=1.4]}}},postaction={decorate}}}
\tikzset{->-/.default=0.65}
\tikzset{-w-/.default=0.65}
\tikzstyle{bullet}=[circle,fill=black,inner sep=0.5mm]
\tikzstyle{circ}=[circle,draw=black,fill=white,inner sep=0.5mm]
\tikzstyle{vertex}=[circle,draw=black,thick,inner sep=0.5mm]
\tikzstyle{dot}=[draw,circle,fill=black,minimum size=0.5mm,inner sep = 0mm, outer sep = 0mm]
\tikzset{darrow/.style={double distance = 4pt,>={Implies},->},
	darrowthin/.style={double equal sign distance,>={Implies},->},
	tarrow/.style={-,preaction={draw,darrow}},
	qarrow/.style={preaction={draw,darrow,shorten >=0pt},shorten >=1pt,-,double,double
		distance=0.2pt}}
\newcommand\tikcirc[1][2.5]{\tikz[baseline=-#1]{\draw[thick](0,0)circle[radius=#1mm];}}
\newcommand{\tikzfig}[1]{\begin{tikzpicture}[auto,baseline={([yshift=-.5ex]current bounding box.center)}]#1\end{tikzpicture}}
\usetikzlibrary{decorations.pathreplacing}

\usepackage{amsthm}
\usepackage{thmtools}
\usepackage[noabbrev,capitalize]{cleveref}
\newtheorem{theorem}{Theorem}
\newtheorem{lemma}[theorem]{Lemma}
\newtheorem{proposition}[theorem]{Proposition}
\newtheorem{corollary}[theorem]{Corollary}
\newtheorem{conjecture}[theorem]{Conjecture}
\newtheorem{Claim}[theorem]{Claim}

\newtheorem*{theorem*}{Theorem}

\theoremstyle{definition}
\newtheorem{definition}{Definition}
\newtheorem{example}{Example}
\newtheorem{cexample}{Counterexample}
\newtheorem{exercise}{Exercise}
\newtheorem{question}{Question}
\newtheorem{remark}{Remark}

\addtolength{\textheight}{1.75cm}
\addtolength{\voffset}{-0.5cm}
\addtolength{\footskip}{+0.6cm}
\geometry{margin=1.2in}


\usepackage[backend=biber, maxbibnames=99, bibencoding=utf8, doi=false, isbn=false, url=false, style=alphabetic]{biblatex}
\addbibresource{refs.bib}

    \begin{document}
\fi
\setcounter{section}{2}

\section{Extended double shuffle relations and $\DMR_0$}\label{section:review-on-DMR}

\nsubsection{Extended double shuffle relations}



MZVs are known to satisfy two types of the product to sum relations among MZVs arising from their integral and series expressions.
Combining these two types of the product to sum relations, one can derive the $\Q$-linear relations among MZVs, termed as double shuffle relations. 
However, 
%due to technical difficulties related to the convergence problem of MZVs, 
the double shuffle relation does not suffice all $\Q$-linear relations among MZVs.
 In \cite{Ihara-Kaneko-Zagier}, the authors introduced the extended double shuffle relation as a refinement of the double shuffle relation, and it is conjectured that the extended double shuffle relations derives all $\Q$-linear relations among MZVs. In this subsection, we briefly discuss this framework in more general settings.


For a $\mathbb{Q}$-algebra homomorphism $Z_R : \ide{h}^{0,\shuffle} \rightarrow R$, we say $Z_R$ satisfies the double shuffle conditions if $Z_R \circ \mathbf{p} : \Q\langle Y \rangle^{0,*} \rightarrow R$ is the $\Q$-algebra homomorphism.

Let $Z_R : \ide{h}^{0,\shuffle} \rightarrow R$ be the $\Q$-algebra homomorphism satisfying the double shuffle conditions.
Since $\ide{h}^{1,\shuffle}=\ide{h}^{0,\shuffle}[x_1]$ and
$\Q\langle Y\rangle^{*}=\Q\langle Y\rangle^{0,*}[y_1]$, there exist two unique $\Q$-algebra homomorphisms which extend $Z_R$.
Namely,
\[
\widetilde{Z}^{\shuffle}_R:\ \ide{h}^{1,\shuffle}\longrightarrow R[T],\qquad
\widetilde{Z}^{\shuffle}_R\!\mid_{\ide{h}^{0,\shuffle}}=Z_R,\ \ \widetilde{Z}^{\shuffle}_R(x_1)=T,
\]
and
\[
\widetilde{Z}^{*}_R:\ \Q\langle Y\rangle^{*}\longrightarrow R[T],\qquad
\widetilde{Z}^{*}_R\!\mid_{\Q\langle Y\rangle^{0,*}}=Z_R\circ\mathbf{p},\ \ \widetilde{Z}^{*}_R(y_1)=T,
\]
where $T$ is an indeterminate.

The main theorem of \cite{Ihara-Kaneko-Zagier} is given as follows.

\begin{prop}[{\cite[Theorem 2]{Ihara-Kaneko-Zagier}}]\label{prop:IKZ-EDS}
Let $T$ be a variable and $(R, Z_R)$ a pair of a $\Q$-algebra $R$ and an element $Z_R$ of $\Hom_{\Qalg}(\ide{h}^{0,\shuffle},R)$ with double shuffle conditions. Then the following is equivalent:
\begin{enumerate}
\item 
The following equality holds in $\Hom_{\text{$\Q$-lin}} (\ide{h}^{1}, R)$:
\[
\widetilde{Z}_R = \rho_R \circ \widetilde{Z}^*_R.
\]
Here, we define the $R$-module map $\rho_R : R[T] \rightarrow R[T]$ by the identity below in $\Q[[u]]$, where $u$ is an indeterminate:
\[
\rho_R (e^{Tu}) = \exp \left(\sum_{n\ge 2} \frac{(-1)^n}{n} Z_R(x_0^{n-1}x_1)u^n \right).
\]
\item For $w_1\in \ide{h}^{1,\shuffle}$ and $w_0 \in \ide{h}^{0,\shuffle}$, $Z_R^{\shuffle}(w_1\shuffle w_0 - \mathbf{p}(\mathbf{q}(w_1) * \mathbf{q}(w_0))) = 0$ holds.
\end{enumerate}
\end{prop}


\begin{df}\label{df:eds}
Let $Z_{R} : \ide{h}^{0,\shuffle} \rightarrow R$ be a $\mathbb{Q}$-algebra homomorphism satisfying the double shuffle conditions and Proposition \ref{prop:IKZ-EDS}. The \textbf{extended double shuffle relations} are the $\Q$-linear relations derived from
\[
(\widetilde{Z}_{R} - \rho_{R} \circ \widetilde{Z}^*_{R})(w) |_{T = 0} = 0
\]
for all $w\in X^*$.
\end{df}


\begin{ex}\label{ex:reg-MZV}
The most fundamental examples of Proposition \ref{prop:IKZ-EDS} are MZVs.
Let 
\[
Z_{\mathbb{R}} : \ide{h}^{0,\shuffle} \rightarrow \mathbb{R} ; x_1x_0^{k_1-1}\cdots x_1x_0^{k_r-1} \mapsto \zeta(k_1,\ldots,k_r) \text{\hspace{1cm}($k_1,\ldots,k_r\in\mathbb{Z}_{>0}$ with $k_r>1$}).
\]
Since MZVs have the iterated integral expression, $Z_{\mathbb{R}}$ is the $\mathbb{Q}$-algebra homomorphism.
Moreover, $Z_{\mathbb{R}}$ satisfies the double shuffle condition due to its multiple series expression. 
Therefore, there exist two $\mathbb{Q}$-algebra homomorphisms $\widetilde{Z}_{\mathbb{R}} : \ide{h}^{1,\shuffle} \rightarrow \mathbb{R}$ and $\widetilde{Z}^*_{\mathbb{R}} : \Q\langle Y \rangle^{*} \rightarrow \mathbb{R}$. 
Then, $\widetilde{Z}_{\mathbb{R}}$ and $\widetilde{Z}^*_{\mathbb{R}}$ satisfy the conditions of Proposition \ref{prop:IKZ-EDS}.
In relation to these two maps, we define certain regularizations of MZVs. For $(k_1,\ldots,k_r)\in\mathbb{Z}_{>0}^r$, we respectively define shuffle regularized MZVs $\zeta^{\shuffle}$ and harmonic regularized MZVs $\zeta^*$ by
\begin{align*}
\zeta^{\shuffle}(k_1,\ldots,k_r) =& \widetilde{Z}_{\mathbb{R}}(x_1x_0^{k_1-1}\cdots x_1x_0^{k_r-1})(0)\\
\zeta^{*}(k_1,\ldots,k_r) =& \widetilde{Z}^*_{\mathbb{R}}(y_{k_1}\cdots y_{k_r})(0).
\end{align*}
\end{ex}

Let $I_{\EDS}$ be a two-sided ideal of $\ide{h}^{\shuffle}$ generated by $x_0$, $x_1$, and $w_1\shuffle w_0 - \mathbf{p}(\mathbf{q}(w_1) * \mathbf{q}(w_0))$ for $w_1\in \ide{h}^{1,\shuffle}$ and $w_0 \in \ide{h}^{0,\shuffle}$, and we define
\[
\mathcal{Z}^f := \ide{h}^{\shuffle}/I_{\EDS}.
\]
For $(k_1,\ldots,k_r)\in \mathbb{Z}_{>0}^r$ with $r\ge1$, we write $\zeta^f(k_1,\ldots,k_r)$ for the image of the word $x_1x_0^{k_1-1}\cdots x_1x_0^{k_r-1}$ in $\mathcal{Z}^f$, and we put $\zeta^f(\emptyset)=1$.
In particular, the classes $\zeta^f(k_1,\ldots,k_r)$ generate $\mathcal{Z}^f$ as a $\Q$-vector space.
In this framework, the authors of \cite{Ihara-Kaneko-Zagier} proposed the following conjecture.

\begin{conj}\label{conj:EDS}
The following $\Q$-algebra homomorphism
\[
\mathcal{Z}^f \rightarrow \mathcal{Z} : \zeta^f(k_1,\ldots,k_r)\mapsto  \zeta(k_1,\ldots,k_r)
\]
is a $\Q$-algebra isomorphism.
Namely, all $\Q$-linear relations among MZVs are obtained from the extended double shuffle relations.
\end{conj}
Define the affine scheme $\EDS : \Qalg\rightarrow \cat{Set}; R\mapsto \Hom_{\Qalg}(\mathcal{Z}^f,R)$ whose coordinate ring is $\mathcal{Z}^f$.

\nsubsection{The double shuffle group $\DMR_0$}

Within Hoffman's framework, the double shuffle relations are equivalent to group-likeness for the coproducts $\Delta_{\shuffle}$ and $\Delta_{*}$.
From this observation, Racinet \cite{Racinet} developed another approach to the regularization theorem (Proposition \ref{prop:IKZ-EDS}) and proposed a specific affine group scheme $\DMR_0$ called the \textbf{double shuffle group}. It is well-known that the coordinate ring $\mathcal{O}(\DMR_0)$ of $\DMR_0$ is $\mathcal{Z}^f/\zeta^f(2)\mathcal{Z}^f$.
The group $\DMR_0$ is expected to satisfy certain specific conditions.
For instance, $\DMR_0$ is conjecturally isomorphic to the motivic Galois group over the $\mathbb{Z}$, Grothendieck-Teichm\"{u}ller group, and the Kashiwara-Vergne group.
\begin{comment}
\textcolor{red}{
In this section, we treat two affine group schemes, $\TM$ and $\DMR_0$.
The group law of $\DMR_0$ is called the Ihara product $\circledast$ (see the sentence before Proposition \ref{prop:TM-Grp}).
To establish the property of affine group schemes in $\DMR_0$, Racinet proposed the affine group scheme $\TM$ called the \textbf{twisted Magnus group}. 
As a result, Racinet proved that $\DMR_0$ is the closed affine group subscheme of $\TM$.
We present the review of $\TM$ and $\DMR_0$. Additionally, we also review its corresponding Lie algebras $\tm$ and $\dmr$.
At the end of this section, we treat a bigraded Lie algebra $\ls$.
The bigraded Lie algebra $\ls$ is related to the main theorems in this paper.
}
\end{comment}





%, and its bigraded Lie algebra $\ls$}



As mentioned in the previous subsection, the authors of \cite{Ihara-Kaneko-Zagier} introduced the $\shuffle$-regularized MZVs and the $*$-regularized MZVs to show the connection between the shuffle product relations and the harmonic product relations among MZVs.
Independently, Racinet established the regularized theorem by using the generating function of MZVs. 

Let
\begin{align*}
\Phi_{\shuffle} :=& \sum_{w\in X} \zeta^{\shuffle}(w) \overset{\leftarrow}{w}, \text{ and }\\
\Phi_{*} :=& \sum_{w\in Y} \zeta^{*}(w) \overset{\leftarrow}{w},
\end{align*}
where, $\overset{\leftarrow}{w}$ is the reversal word of $w$.
Since $\widetilde{Z}_{\mathbb{R}}$ (respectively, $\widetilde{Z}_{\mathbb{R}}^*$) is the $\Q$-algebra homomorphism with respect to $\shuffle$ (respectively, $*$), and since a $\Q$-linear map $\mathbb{R}[T] \rightarrow \mathbb{R} ; aT + b \mapsto b$ for $a$, $b\in \mathbb{R}$ is also a $\Q$-algebra homomorphism, we have 
\[
\langle \Phi_{\shuffle} \mid u \shuffle v \rangle = \Phi(u)\Phi(v)
\]
for $u$, $v \in X$.
\[
 \text{(respectively, }\langle \Phi_* \mid u * v \rangle = \langle \Phi_* \mid u \rangle \langle \Phi_* \mid  v \rangle
\]
for $u$, $v\in Y$).
Consequently, we have 
\begin{align*}
\Delta_{\shuffle}(\Phi_{\shuffle}) =& \Phi_{\shuffle} \otimes \Phi_{\shuffle}\\
\left(\text{respectively, } \right. \Delta_*(\Phi_*) =&\left. \Phi_* \otimes \Phi_* \right).
\end{align*}
Related to these two generating functions, Racinet showed the following:
\begin{thm}[{\cite[Corollary 2.24]{Racinet}}]\label{thm:regularized-Racinet}
Let
\[
\Gamma_{\Phi_{\shuffle}} :=\exp \left( \sum_{n\ge 2} \frac{(-1)^{n-1}}{n} \langle \Phi_{\shuffle} \mid x_0^{n-1}x_1 \rangle y_1^n \right).
\]
Then, we have
\[
\Phi_* = \Gamma_{\Phi_{\shuffle}} \mathbf{q}^{\vee}(\Phi_{\shuffle}).
\]
\end{thm}

From Theorem \ref{thm:regularized-Racinet}, Racinet constructed the certain subset of $R\cotimes \ide{h}^{\vee}$. 
\begin{df}[{\cite[D\'{e}finition 3.2.1]{Racinet}}]\label{def:DMR}
Let $R$ be a $\mathbb{Q}$-algebra.
The double shuffle set $\DMR(R)$ consists of those $\Phi \in R \widehat{\otimes} \ide{h}^{\vee}$:
\begin{enumerate}
\item\label{def:DMR-1} $\langle \Phi \mid 1\rangle = 1$,
\item\label{def:DMR-2} $\langle \Phi \mid x_0\rangle = \langle \Phi \mid x_1\rangle = 0$,
\item\label{def:DMR-3} $\Delta_{\shuffle}(\Phi) = \Phi\otimes \Phi$, and
\item\label{def:DMR-4} $\Delta_{*}(\Phi_{\star}) = \Phi_{\star}\otimes \Phi_{\star}$.
\end{enumerate}
Here, $\Phi_{\star} = \Gamma_{\Phi}\mathbf{q}^{\vee}(\Phi)$ and $\Gamma_{\Phi}$ is defined by
\[
\Gamma_{\Phi} = \exp \left(\sum_{n\ge 2}\frac{(-1)^{n}}{n}\langle \Phi \mid x_0^{k-1}x_1 \rangle y_1^n\right).
\]
Let $\lambda\in R$. By $\DMR_{\lambda}(R)$, we denote the subset of $\DMR(R)$ satisfying the additional condition:
\begin{enumerate}
\setcounter{enumi}{4}
\item\label{def:DMR-5} $\langle \Phi \mid x_0x_1\rangle =- \frac{ \lambda^2 }{24}$.
\end{enumerate} 
\end{df}
We define functors $\DMR : \Qalg\rightarrow \cat{Set}; R\mapsto \DMR(R)$ and $\DMR_0 : \Qalg\rightarrow \cat{Set}; R\mapsto \DMR_0(R)$.

\begin{rmk}
Given $\Phi\in\DMR(R)$, the generating series determines a $\Q$-algebra homomorphism $Z_{R,\Phi} : \mathcal{Z}^f \rightarrow R ; \zeta^f(k_1,\ldots,k_r) \mapsto \langle \Phi \mid x_0^{k_r-1}x_1\cdots x_0^{k_1-1}x_1 \rangle$ for $\Phi \in \DMR(R)$. It can be verified that $Z_{R,\Phi} \in \EDS(R)$ induces a natural isomorphism $\DMR \Longrightarrow \EDS$. Further details can be found in \cite{Bachmann-Yaddaden}.
Their result indicates that the works of Ihara, Kaneko, and Zagier (\cite{Ihara-Kaneko-Zagier}) and Racinet (\cite{Racinet}) are essentially equivalent.
\end{rmk}

For $\phi$ and $\psi\in \ide{h}^{\vee}$ with $\langle \phi \mid 1\rangle = \langle \psi \mid 1\rangle = 1$, we define 
\[
\phi \circledast \psi := \phi \cdot \kappa_{\phi^{-1}x_1\phi}(\psi).
\]
Here, $\kappa_{f}$ is an endomorphism of $\ide{h}^{\vee}$ as a $\mathbb{Q}$-algebra, associated with fixed $f\in \ide{h}^{\vee}$ such that $\langle f \mid 1\rangle = 0$, defined by
\[
x_0\mapsto x_0\hspace{1.0ex},\hspace{1.0ex} x_1\mapsto f.
\]

The operation $\circledast$ is referred to as the \textbf{Ihara product}.

\begin{thm}[{\cite[Th\'{e}or\`{e}m I]{Racinet}}]
Let $\DMR_0 : \Qalg\rightarrow \cat{Set} ; R \mapsto \DMR_0(R)$. Then, $\DMR_0$ is the affine group scheme, i.e. for $\mathbb{Q}$-algebra $R$, $\DMR_0(R)$ is a group with respect to $\circledast$.
\end{thm}


\ifdefined\isMainDocument
\else
    %\bibliographystyle{amsplain}
    %\bibliography{reference-adjoint}
    \end{document}
\fi