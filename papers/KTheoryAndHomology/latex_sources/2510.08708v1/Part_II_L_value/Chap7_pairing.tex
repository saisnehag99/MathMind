\begin{convention}
     In this section, we let $V = V^{a,b}\{r, s\} \in \mathrm{Rep}_{\QQ}(V(2))$ and $n = a + b + r + s$, and Betti cohomology, compactly supported cohomology and interior cohomology all have $\QQ$-coefficients, which is different from subsection \ref{SS: the proof of the vanishing}.
\end{convention}

\subsection{The use of Poincar\'{e} duality}
\label{SS: Poincare duality}
In this subsection, we will explain how the Poincar\'{e} duality pairing can be used to compute the Beilinson regulator. 
The idea is due to Beilinson \cite{Beilinson_Modular_Curve} (see also \cite{Kings98}). 

\par Let us first recall a general result.
\begin{lemma}\textnormal{\cite[Lemma 4.11]{LemmaII17}}
\label{Lemma: def of Ext^{1}}
    Let $E$ be a number field and $M$ be an object of $\mathrm{MHS}_{\RR, E}^{+}$ with pure weight $w < 0$. Let $M_{dR}$ be the $E\otimes_{\QQ}{\RR}$-submodule of $M_{\CC}$ where the de Rham involution acts trivially and let $M^{-}$ be the submodule of $M$ where the infinitesimal Frobenius $F_{\infty}$ acts by multiplication by $-1$. We write $M^{-}(-1) = \frac{1}{2 \pi i} M^{-}$. Then there is an short exact sequence of $E \otimes_{\QQ}{\RR}$-modules
    \begin{equation*}
        0 \rightarrow F^{0}M_{dR} \rightarrow M^{-}(-1) \rightarrow \Ext^{1}_{\mathrm{MHS_{\RR}^{+}}}(\RR(0), M) \rightarrow 0,  
    \end{equation*}
    where the first map is the composite of the natural inclusions 
    \begin{equation*}
        F^{0}M_{dR} \rightarrow M_{dR} \rightarrow M_{\CC}
    \end{equation*}
    and of the projections 
    \begin{equation*}
        M_{\CC} \rightarrow M(-1) \rightarrow M^{-}(-1)
    \end{equation*}
    defined by $v \mapsto \frac{1}{2}(v - \bar{v})$ and $\frac{1}{2}(v - F_{\infty}(v))$, respectively. 
\end{lemma}

We apply Lemma \ref{Lemma: def of Ext^{1}} to the situation that we are interested in to get the following lemma.

\begin{lemma}
\label{Lemma: exact seq to def Ext^{1} in our situation}
     Let $\pi_{f}$ be the non-archimedean part of an irreducible cuspidal automorphic representation $\pi$ of $G$ (see Definition \ref{defn:G_H}) whose archimedean component $\pi_{\infty}$ belongs to the discrete series $L$-packet $P(V_{\CC}(2))$ and denote by $\pi_{f}$ its rational model over $E(\pi_f)$ (see Theorem \ref{Thm: rational field}). Then there is a short exact sequence:
     \begin{equation} \label{exact seq: Ext^1}
          0 \rightarrow F^{0}M_{dR}(\pi_f, V(2))_{\RR} \rightarrow M_{B}(\pi_f, V(2))^{-}_{\RR}(-1) \rightarrow \Ext^{1}_{\mathrm{MHS_{\RR}^{+}}}(\RR(0), M_{B}(\pi_f, V(2))_{\RR}) \rightarrow 0,
     \end{equation}
     where the second map is defined in Lemma \ref{Lemma: def of Ext^{1}}. 
\end{lemma}

Let $F^{0}M_{dR}(\pi_f, V(2))^{*}$ be the dual of $F^{0}M_{dR}(\pi_f, V(2))$. It follows from Lemma \ref{Lemma: exact seq to def Ext^{1} in our situation} that the one dimensional $E(\pi_f)$-vector space 
\begin{equation*}
    \mathcal{B}(\pi_f, V(2)) = \mathrm{det}_{E(\pi_f)} F^{0}M_{dR}(\pi_f, V(2))^{*} \otimes_{E(\pi_f)} \mathrm{det}_{E(\pi_f)} M_{B}(\pi_f, V(2))^{-}(-1) 
\end{equation*}
is an $E(\pi_f)$-structure of the $E(\pi_f)\otimes \RR$-module 
\begin{equation*}
    \mathrm{det}_{E(\pi_f)\otimes_{\QQ}\RR}\Ext^{1}_{\mathrm{MHS_{\RR}^{+}}}(\RR(0), M_{B}(\pi_f, V(2))_{\RR}).
\end{equation*}

Here $\mathcal{B}(\pi_f, V(2))$ is the Beilinson $E(\pi_f)$-structure defined in \cite[\S6.1]{Nekovar94}. We now define the Deligne $E(\pi_f)$-structure using the Beilinson $E(\pi_f)$-structure.

\begin{definition}[Deligne $E(\pi_f)$-structure]
\label{def: Delign-rational structure}
    Let $\delta(\pi_f, V(2)) \in (E(\pi_f) \otimes_{\QQ} \CC)^{\times}$ be the determinant of the isomorphism $I_{\infty} \colon M_{B}(\pi_f, V(2))_{\CC} \rightarrow M_{dR}(\pi_f, V(2))_{\CC}$ computed using the basis defined over $E(\pi_f)$ on both sides. Then the Deligne $E(\pi_f)$-structure of 
    \begin{equation*}
            \mathrm{det}_{E(\pi_f)\otimes_{\QQ}\RR}\Ext^{1}_{\mathrm{MHS_{\RR}^{+}}}(\RR(0), M_{B}(\pi_f, V(2))_{\RR})
    \end{equation*}
    is 
    \begin{equation*}
        \mathcal{D}(\pi_f, V(2)) = (2\pi i)^{\dim_{E(\pi_f)} M_{B}(\pi_f, W)^{-}} \delta(\pi_f, W)^{-1} \mathcal{B}(\pi_f, V(2)). 
    \end{equation*}
\end{definition}

\begin{remark}
    This definition does not depend on the choice of the bases. 
\end{remark}

\begin{lemma}
    It can be seen that $\Ext^{1}_{\mathrm{MHS_{\RR}^{+}}}(\RR(0), M_{B}(\pi_f, V(2))_{\RR})$ is a rank one $E(\pi_f) \otimes_{\QQ} \RR$-module. 
\end{lemma}

\begin{proof}
    It can be seen that 
    \begin{equation*}
        \mathrm{rank}_{E(\pi_f) \otimes_{\QQ} \RR} F^{0}M_{dR}(\pi_f, V(2))_{\RR} = 2
    \end{equation*}
    from the Hodge decomposition of $M(\pi_f, V(2))$ in Corollary \ref{Corollary: Hodge decomp for motives} and the condition of coefficients in equation (\ref{eq: Cond of coefficients}). By the exact sequence (\ref{exact seq: Ext^1}) and 
    \begin{equation*}
        \mathrm{rank}_{E(\pi_f) \otimes_{\QQ} \RR} M_{B}(\pi_f, V(2))^{-}_{\RR}(-1) = 3, 
    \end{equation*}
    we conclude that $\Ext^{1}_{\mathrm{MHS_{\RR}^{+}}}(\RR(0), M_{B}(\pi_f, V(2))_{\RR})$ is a rank one $E(\pi_f) \otimes_{\QQ} \RR$-module. 
\end{proof}

Recall that by Theorem \ref{Thm: Hdg vanish on the boudary}, the map $\mathcal{E}is_{H}^n \colon \mathcal{B}_{n,\RR} \rightarrow \mathrm{H}^{3}_{H}(S, V(2))$ factors through the inclusion 
\[
    \Ext^{1}_{\mathrm{MHS}_{\RR}^{+}}(\mathbf{1}, \mathrm{H}^{2}_{B,!}(S, V(2))_{\RR}) \hookrightarrow \mathrm{H}^{3}_{H}(S, V(2)).
\]Let $\mathcal{K}(V(2))$ be the sub $\QQ[G(\mathbb{A}_{f})]$-module of 
\begin{equation*}
    \Ext^{1}_{\mathrm{MHS_{\RR}^{+}}}(\RR(0), \mathrm{H}^{2}_{B, !}(S, V(2))_{\RR}) = \varinjlim_{L}  \Ext^{1}_{\mathrm{MHS_{\RR}^{+}}}(\RR(0), \mathrm{H}^{2}_{B, !}(S(L), V(2))_{\RR}) 
\end{equation*}
generated by the image of $\mathcal{E}is_{H}^{n}$ with domain $\mathcal{B}_{n}$ and let $\mathcal{K}(\pi_{f}, V(2))$ be defined by
\begin{equation}
\label{eq: K(pi, V)}
    \mathcal{K}(\pi_{f}, V(2)) := \Hom_{\QQ[G(\mathbb{A}_f)]}(\Res_{E(\pi_f)/\QQ}\pi_f, \mathcal{K}(V(2)). 
\end{equation}
This is an $E(\pi_f)$-submodule of $\Ext^{1}_{\mathrm{MHS_{\RR}^{+}}}(\RR(0), M_{B}(\pi_f, V(2))_{\RR})$. Therefore, now we have two $E(\pi_f)$-submodules of $\Ext^{1}_{\mathrm{MHS_{\RR}^{+}}}(\RR(0), M_{B}(\pi_f, V(2))_{\RR})$. The first one, which is called $\mathcal{D}(\pi_f, V(2))$, is defined using the comparison between Betti and De Rham cohomology, which is more elementary. The second one, called $\mathcal{K}(\pi_f, V(2))$, is defined using the Beilinson regulator $r_{H}$, which is more sophisticated. By definition, we know that 
$\mathcal{D}(\pi_f, V(2))$ is non-zero, so we hope to know whether $\mathcal{K}(\pi_f, V(2))$ is zero or not by a comparision between $\mathcal{K}(\pi_f, V(2))$ and $\mathcal{D}(\pi_f, V(2))$. 

\begin{definition}
    If $\mu$ and $\mu^{\prime}$ are two elements of $E(\pi_f)\otimes_{\QQ}{\CC}$, we write $\mu \sim \mu^{\prime}$ if there exists $\lambda \in E(\pi_f)^{\times}$ such that $\mu = \lambda \mu^{\prime}$. 
\end{definition}

\begin{lemma}
\label{Lemma: linear form and rational structure}
    Let ${v}_{D}$ be a non-zero vector in $\mathcal{D}(\pi_f, V(2))$ and $v_{K}$ be a non-zero vector in $\mathcal{K}(\pi_{f}, V(2))$. Let $\tilde{v}_{D}$ and $\tilde{v}_{K}$ be any lifting of $v_{D}$ and $v_{K}$ in $M_{B}(\pi_f, V)^{-}_{\RR}(-1)$ using the last arrow in the exact sequence (\ref{exact seq: Ext^1}). Then for any $E(\pi_{f})\otimes_{\QQ}\RR$-linear map $\psi: M_{B}(\pi_f, V(2))^{-}_{\RR}(-1) \rightarrow E(\pi_f) \otimes_{\QQ} \CC$ which is trivial on $F^{0}M_{dR}(\pi_f, V(2))_{\RR}$, we have 
    \begin{equation*}
        \mathcal{K}(\pi_{f}, V(2)) = \frac{\psi(\tilde{v}_{K})}{\psi(\tilde{v}_{D})} \mathcal{D}(\pi_f, V(2)). 
    \end{equation*}
\end{lemma}

\begin{proof}
    It follows from the exact sequence (\ref{exact seq: Ext^1}).
\end{proof}

Our goal is to compute $\psi(\tilde{v}_{K})$ and $\psi(\tilde{v}_{D})$ seperately for a well-chosen linear functional $\psi$. One natural way to choose $\psi$ is to use Poincar\'{e} duality. Hence, let us first recall properties of 
Poincar\'{e} duality for Picard modular surfaces. By previous computations, we assume that 
\begin{equation}
\label{eq: Cond of coefficients}
    \left \{
    \begin{aligned}
        & 0 \le -r \le a \\
        & 0 \le -s \le b \\
        & a > 0 \quad \text{and} \quad b > 0 \\
        &  r \neq 0 \quad \text{or} \quad s \neq 0
    \end{aligned}
    \right. 
\end{equation}
Recall for a representation $V = V^{a, b}\{r, s\}$, its contragredient representation is $D = V^{*} = V^{b, a}\{-a - b - r, -a - b -r \}$. In other words, we have a perfect pairing \footnote{Both $V$ and $D$ are in $\mathrm{Rep}_{\QQ}(G)$. }
\begin{equation*}
    V \otimes D \rightarrow \QQ(0). 
\end{equation*}
The pairing induces a $G(\AAA_{f})$-equivariant pairing 
\begin{equation*}
    \langle \cdot, \cdot \rangle_{B} \colon \mathrm{H}^{2}_{B, !}(S, V(2)) \otimes \mathrm{H}^{2}_{B, !}(S, D) \rightarrow \mathrm{H}^{4}_{B, !}(S, \QQ(2)) \rightarrow \QQ(0),
\end{equation*}
where the last map is defined by the trace. The pairing becomes perfect after restriction to the vectors which are invariant by a compact open subgroup of $G(\AAA_f)$.
\begin{fact}
    Here, $\QQ(0)$ is given by an action of $G(\AAA_f)$ by $|\mu|^{-2}$.
\end{fact}
\begin{proof}
    We have 
    \begin{equation}
    \label{eq: isom duality of Betti interior cohomology}
        \mathrm{H}^{4}_{B, !}(S, \QQ(2)) \cong  \mathrm{H}^{4}_{B, c}(S, \QQ(2)) \cong  \mathrm{H}^{0}_{B}(S, \QQ(-2)) \cong \bigoplus (\chi |\mu|^{-2}),
    \end{equation}
    where the sum is over finite Hecke characters $\chi \colon \QQ^{\times} \backslash \AAA_{f, \QQ}^{\times} \rightarrow \CC^{\times}$. The first isomorphism in (\ref{eq: isom duality of Betti interior cohomology}) comes from Corollary \ref{Corollary: Localization long exact seq} (1) and the fact $\dim \partial S = 0$; the second isomorphism in (\ref{eq: isom duality of Betti interior cohomology}) holds by Poincar\'{e} duality, and the last isomorphism in (\ref{eq: isom duality of Betti interior cohomology}) can be deduced from \cite[Theorem 5.17]{Milne17}. The trace map is a projection onto the factor with $\chi$ trivial. 
\end{proof}
\begin{remark}
    This fact is similar to \cite[p295]{Taylor93} in the Siegel 3-folds case.
\end{remark}
Hence, this induces a $G(\AAA_f)$-equivariant morphism of Hodge structures \footnote{It is a $\QQ$-morphism.}
\begin{equation}
\label{eq: M_{B} pairing}
    \langle \cdot, \cdot \rangle_{B} \colon M_{B}(\pi_f, V(2)) \otimes M_B(\tilde{\pi}_f|\mu|^{-2}, D) \rightarrow E(\pi_f)(0). 
\end{equation}

Recall by Proposition \ref{Prop: Hodge decomp for motives} that we have the Hodge decomposition
\begin{align*}
    & M_{B}(\tilde{\pi}_f|\mu|^{-2}, D)_{\CC} \\
    \cong & M_{B}^{a + b + r + 2, s} \oplus M_{B}^{a + r + 1, b + s + 1} \oplus M_{B}^{r, a + b + s + 2} \\
    \oplus & M_{B}^{s, a + b + r + 2} \oplus M_{B}^{b + s + 1, a + r + 1} \oplus M_{B}^{a + b + s + 2, r}. 
\end{align*}

\begin{lemma}
\label{lemma: abstrac paring with Omega}
    Let $\Omega \in M(\tilde{\pi}_f|\mu|^{-2}, D)_{\CC}^{+}$. Let 
    \[
        \begin{tikzcd}
            M_{B}(\pi_f, V(2))^{-}_{\RR}(-1) \ar[r, "{\langle\Omega, \cdot \rangle_{B}}"] & E(\pi_f) \otimes_{\QQ} \CC 
        \end{tikzcd}
    \]
    be the morphism defined by the composition of the inclusion $M_{B}(\pi_f, V(2))^{-}_{\RR}(-1) \rightarrow M_{B}(\pi_f, V(2))_{\CC}$ and the pairing with $\Omega$. Assume $\Omega$ belongs to $M_{B}^{a + r + 1, b + s + 1} \oplus M_{B}^{b + s + 1, a + r + 1}$. Then 
    \begin{equation}
    \label{eq:comparion_rat_struc}
        \mathcal{K}(\pi_{f}, V(2)) = \frac{\langle \Omega, \tilde{v}_{K} \rangle_{B}}{\langle \Omega, \tilde{v}_{D} \rangle_{B}} \mathcal{D}(\pi_f, V(2))
    \end{equation}
\end{lemma}

\begin{proof}
    This follows from a Hodge-type computation. 
\end{proof}

\begin{convention}
    From now on, we will use $\Sh_{G}$ (resp. $\Sh_{H}$) to denote $\Sh_{G, \CC}^{an}$ (resp., $\Sh_{H, \CC}^{an}$). 
\end{convention}

\begin{remark}
\label{remark: pairing}
    Let $\mathbf{1}$ be the multiplicative identities of $E(\pi_f) \otimes_{\QQ} \CC$. 
    \begin{enumerate}
        \item By definition, we have the following isomorphisms 
              \begin{align*}
                  & M_{B}(\pi_f, V(2))_{\CC} \cong ((\mathrm{H}^{2}_{B, !}(\Sh_{G}, V(2))_{\CC} \oplus  \mathrm{H}^{2}_{B, !}(\overline{\Sh_{G}}, V(2))_{\CC})[\pi_f]) \otimes_{\CC} (E(\pi_f) \otimes_{\QQ} \CC), \\
                  & M_B(\tilde{\pi}_f|\mu|^{-2}, D)_{\CC} \cong ((\mathrm{H}^{2}_{B, !}(\Sh_{G}, D)_{\CC} \oplus \mathrm{H}^{2}_{B, !}(\overline{\Sh_{G}}, D))_{\CC})[\tilde{\pi}_f|\mu|^{-2}]) \otimes_{\CC} (E(\pi_f) \otimes_{\QQ} \CC),
              \end{align*}
              where the notation $[\cdot]$ means the corresponding isotypic subspaces. 
        \item If we could construct a differential form 
        \begin{equation*}
            \omega_{\Psi} = \omega \otimes \Psi_f \in \mathrm{H}^{2}_{B, !}(\Sh_{G}, D)_{\CC}[\pi_f] = \mathrm{H}^{2}(\lieg_{\CC}, K_{G}; D_{\CC} \otimes_{\CC} \tilde{\pi}_{\infty})\otimes \tilde{\pi}_f|\mu|^{-2}
        \end{equation*}
        associated to a $\Psi = \Psi_{f} \otimes \Psi_{\infty}$
        that belongs to a cuspidal automorphic representation $\tilde{\pi} = \tilde{\pi}_{f}|\mu|^{-2} \otimes \tilde{\pi}_{\infty}$ of $G(\AAA)$, then it is natural to let $\Omega \in M(\tilde{\pi}_f|\mu|^{-2}, D)_{\CC}^{+}$ be 
        \begin{equation*}
            \frac{1}{2} (\omega_{\Psi} + \overline{\omega_{\Psi}}) \otimes \mathbf{1}, 
        \end{equation*}
        where $\overline{\omega_{\Psi}} \in \mathrm{H}^{2}_{B, !}(\overline{\Sh_{G}}, D)_{\CC}$ is the complex conjugate of $\omega_{\Psi}$. 
        \item Since the Eisenstein symbol is defined over $\QQ$ (see subsubsection \ref{SSS: Eisenstein symbol}),  
              $\tilde{v}_{K}$ can be represented by 
              \begin{equation*}
                (\tilde{v}_{K}^{1}, \tilde{v}_{K}^{1}) \otimes \mathbf{1}, 
              \end{equation*}
              where the first $\tilde{v}_{K}^{1}$ belongs to $\mathrm{H}^{2}_{B, !}(\Sh_{G}, V(2))_{\CC}[\pi_f]$ and the second $\tilde{v}_{K}^{1}$ belongs to 
              \begin{equation*}
                   \mathrm{H}^{2}_{B, !}(\overline{\Sh_{G}}, V(2))_{\CC}[\pi_f].
              \end{equation*}
        \item The Poincar\'{e} duality pairing 
            \begin{equation*}
                 \langle \cdot, \cdot \rangle_{B} \colon \mathrm{H}^{2}_{B, !}(S, V(2))_{\CC} \otimes \mathrm{H}^{2}_{B, !}(S, D)_{\CC} \rightarrow \CC(0),
            \end{equation*}
            can be decomposed into 
            \begin{equation*}
                 \langle \cdot, \cdot \rangle_{B} = \langle \cdot, \cdot \rangle_{B}^{\Sh_{G}} + \overline{\langle \cdot, \cdot \rangle_{B}^{\Sh_{G}}}, 
            \end{equation*}
            where $\langle \cdot, \cdot \rangle_{B}^{\Sh_{G}}$ is the Poincar\'{e} duality pairing on $\Sh_{G}$: 
            \begin{equation*}
                \langle \cdot, \cdot \rangle_{B}^{\Sh_{G}} \colon \mathrm{H}^{2}_{B, !}(\Sh_{G}, V(2))_{\CC} \otimes \mathrm{H}^{2}_{B, !}(\Sh_{G}, D)_{\CC} \rightarrow \CC(0), 
            \end{equation*}
            and its complex conjugate $\overline{\langle \cdot, \cdot \rangle_{B}^{\Sh_{G}}}$ is t Poincar\'{e} duality pairing on $\overline{\Sh_{G}}$: 
            \begin{equation*}
                \overline{\langle \cdot, \cdot \rangle_{B}^{\Sh_{G}}} \colon \mathrm{H}^{2}_{B, !}(\overline{\Sh_{G}}, V(2))_{\CC} \otimes \mathrm{H}^{2}_{B, !}(\overline{\Sh_{G}}, D)_{\CC} \rightarrow \CC(0). 
            \end{equation*}
        \item  We can see that 
                \begin{align*}
                    & \langle \Omega, \tilde{v}_{K} \rangle_{B} \\
                 =  & \langle \frac{1}{2} (\omega_{\Psi} + \overline{\omega_{\Psi}}), (\tilde{v}_{K}^{1}, \tilde{v}_{K}^{1}) \rangle_{B} \otimes \mathbf{1} \\
                 =  & \frac{1}{2} \langle \omega_{\Psi}, \tilde{v}_{K}^{1} \rangle_{B}^{\Sh_{G}} \otimes \mathbf{1} + \frac{1}{2} \overline{\langle \overline{\omega_{\Psi}}, \tilde{v}_{K}^{1} \rangle_{B}^{\Sh_{G}}}  \otimes \mathbf{1}\\
                 =  & \langle \omega_{\Psi}, \tilde{v}_{K}^{1} \rangle_{B}^{\Sh_{G}} \otimes \mathbf{1}. 
                \end{align*}
                Therefore, in order to compute the pairing $\langle \Omega, \tilde{v}_{K} \rangle_{B}$, it suffices to construct a differential form $\omega_{\Psi}$ and compute $\langle \omega_{\Psi}, \tilde{v}_{K}^{1} \rangle_{B}^{\Sh_{G}}$. 
    \end{enumerate}
\end{remark}

\begin{definition}
    If $\mu$ and $\mu^{\prime}$ are two elements of $E(\pi_f)\otimes_{\QQ}{\CC}$ (resp., $\CC$), we denote $\mu \sim \mu^{\prime}$ if there exists $\lambda \in (E(\pi_f) \otimes_{\QQ} \overline{\QQ})^{\times}$ (resp., $\overline{\QQ}^{\times}$) such that $\mu = \lambda \mu^{\prime}$. 
\end{definition}

\begin{convention}
    In this paper, because we can only compute $\langle \omega_{\Psi}, \tilde{v}_{K}^{1} \rangle_{B}^{\Sh_{G}}$ up to $\overline{\QQ}^{\times}$, we can freely choose $\tilde{v}_{K}$ up to $\overline{\QQ}^{\times}$. In other words, we can use $\mathcal{B}_{n, \overline{\QQ}}$ as the domain of $\mathcal{E}is_{H}$. 
\end{convention}

\subsection{Differential forms}
\label{SS: the test vector}

The next step is to explain how to construct a differential form $\omega_{\Psi}$ on $S$. It is based on some elementary representation-theoretic considerations. 

\par By equation (\ref{eq: decomosition of Betti into (g,K)-cohomology}), Theorem \ref{Thm: multiplicity 1} and Proposition \ref{Prop: (g, K)-cohomology = Hom}, we have the decomposition 
\begin{equation*}
     \mathrm{H}^{2}_{B, !}(S, D)_{\CC} \cong \mathrm{H}^{2}_{dR, !}(S, D)_{\CC} = \bigoplus_{\tilde{\pi} = \tilde{\pi}_{\infty} \otimes \tilde{\pi}_f} \Hom_{K_G}(\wedge^{2} \liep_{G, \CC}, D_{\CC} \otimes_{\CC} \tilde{\pi}_{\infty}) \otimes \tilde{\pi}_f. 
\end{equation*}
Hence, in order to associate a differential form $\omega_{\Psi} = \omega \otimes \Psi_{f}$ to a cusp form $\Psi = \Psi_{f} \otimes \Psi_{\infty}$ in the cuspidal automorphic representation  $\tilde{\pi} = \tilde{\pi}_{f}|\mu|^{-2} \otimes \tilde{\pi}_{\infty}$, it suffices to associate some
\begin{equation*}
    \omega \in \Hom_{K_G}(\wedge^{2} \liep_{G, \CC}, D_{\CC} \otimes_{\CC} \pi_{\infty})
\end{equation*}
to $\Psi_{\infty} \in \tilde{\pi}_{\infty}$. Here $\Psi_{f}$ is any vector in $\tilde{\pi}_{f} |\mu|^{-2}$, and the choice of $\Psi_{\infty} \in \tilde{\pi}_{\infty}$ is explained below. 

\par Now we explain the way to contruct $\omega$. Recall that the complexified Lie algebra of $G$ can be decomposed as $\lieg_{\CC} = \liek_{G, \CC} \oplus \liep^{+}_{G, \CC} \oplus \liep^{-}_{G, \CC}$. 
Let
\begin{align*}
    & X_{(1, -1, 0)} = \begin{pmatrix} 
                            0 & 1 & 0 \\
                            0 & 0 & 0 \\
                            0 & 0 & 0 
                       \end{pmatrix}, \ X_{(-1, 1, 0)} = \begin{pmatrix} 
                                                            0 & 0 & 0 \\
                                                            1 & 0 & 0 \\
                                                            0 & 0 & 0 
                                                         \end{pmatrix} \\
    & X_{(1, 0, -1)} = \begin{pmatrix} 
                            0 & 0 & 1 \\
                            0 & 0 & 0 \\
                            0 & 0 & 0 
                       \end{pmatrix}, \   X_{(-1, 0, 1)} = \begin{pmatrix} 
                                                            0 & 0 & 0 \\
                                                            0 & 0 & 0 \\
                                                            1 & 0 & 0 
                                                         \end{pmatrix} \\
     & X_{(0, 1, -1)} = \begin{pmatrix} 
                            0 & 0 & 0 \\
                            0 & 0 & 1 \\
                            0 & 0 & 0 
                       \end{pmatrix}, \ X_{(0, -1, 1)} = \begin{pmatrix} 
                                                            0 & 0 & 0 \\
                                                            0 & 0 & 0 \\
                                                            0 & 1 & 0 
                                                         \end{pmatrix} 
\end{align*}
be some fixed root vectors in $\lieg_{\CC}$. 
\par Let $\tilde{\pi}_{\infty}$ be the representation of $G(\RR)^{+}$ in the discrete series $L$-packet $P(D_{\CC})$ with Blattner parameter $(1 + a + r -s, -1 -b + r - s, r - s)$ \footnote{This is the $\pi_2$ in the talble (\ref{talble: DS L-packt}) in Proposition \ref{Prop: DS L-packet}.}. 

\begin{proposition}
\label{Prop: explicit construnction of the differential form}
    Let $v \in D_{\CC}$ be a vector of weight $(-a + s - r, b + s - r, s - r; -b - 2s + r)$ that is the lowest weight vector in the $K_G$-type $\tau_{(b + s - r, -a + s - r, s - r)}$ of the algebraic representation $D_{\CC}$ and let $\Psi_{\infty} \in \tilde{\pi}_{\infty}$ be a vector of weight $(-1 - b + r - s, 1 + a + r - s, r - s; b + 2s - r)$ that is the lowest weight vector of the minimal $K_{G}$-type $\tau_{(1 + a + r -s, -1 - b + r  + s, r - s)}$ of $\tilde{\pi}_{\infty}$. Let 
    \begin{equation*}
        X_{(1, 0, -1)} \otimes X_{(0, -1, 1)} \in \liep^{+}_{G, \CC} \otimes \liep^{-}_{G, \CC} = \tau_{(1, -1, 0)} \oplus \tau_{(0, 0, 0)}
    \end{equation*}
    be a highest weight vector of the $K_{G}$-type $\tau_{(1, -1, 0)}$. Then there exists a unique non-zero map 
    \begin{equation*}
        \omega \in \Hom_{K_G}(\liep^{+}_{G, \CC} \otimes \liep^{-}_{G, \CC}, D_{\CC} \otimes \tilde{\pi}_{\infty}) 
    \end{equation*}
    such that 
    \begin{equation}
    \label{eqn:form_satisfied}
        \omega( X_{(1, 0, -1)} \otimes X_{(0, -1, 1)}) = \sum\limits_{i = 0}^{a + b} (-1)^{i} X_{(1, -1 ,0)}^{i} v \otimes X^{2 + a + b - i}_{(1, -1, 0)} \Psi_{\infty}. 
    \end{equation}
\end{proposition}

\begin{proof}
    First, the $K_{G}$-weight of $X_{(1, -1 ,0)}^{i} v \otimes X^{2 + a + b - i}_{(1, -1, 0)} \Psi_{\infty}$ is $(1, -1, 0)$, which is the weight of $X_{(1, 0, -1)} \otimes X_{(0, -1, 1)}$. Hence, the map $\omega$ preserves $K_{G}$-weights. 
    \par Second, we have 
       \begin{align*}
           & X_{(1, -1, 0)}(\sum\limits_{i = 0}^{a + b} (-1)^{i} X_{(1, -1 ,0)}^{i} v \otimes X^{2 + a + b - i}_{(1, -1, 0)} \Psi_{\infty} ) \\
           =  & \sum\limits_{i = 0}^{a + b}(-1)^{i} X_{(1, -1 ,0)}^{i + 1} v \otimes X^{2 + a + b - i}_{(1, -1, 0)} \Psi_{\infty}  
           + \sum\limits_{i = 0}^{a + b}(-1)^{i} X_{(1, -1 ,0)}^{i} v \otimes X^{3 + a + b - i}_{(1, -1, 0)} \Psi_{\infty}  \\
           = \,  & (-1)\sum\limits_{i = 1}^{a + b}(-1)^{i} X_{(1, -1 ,0)}^{i} v \otimes X^{3 + a + b - i}_{(1, -1, 0)} \Psi_{\infty}
             + \sum\limits_{i = 1}^{a + b}(-1)^{i} X_{(1, -1 ,0)}^{i} v \otimes X^{3 + a + b - i}_{(1, -1, 0)} \Psi_{\infty} \\
           = \,  & 0. 
       \end{align*}
    Therefore, the vector $\sum\limits_{i = 0}^{a + b} (-1)^{i} X_{(1, -1 ,0)}^{i} v \otimes X^{2 + a + b - i}_{(1, -1, 0)} \Psi_{\infty}$ is a highest weight vector of $K_G$-type $\tau(1, -1, 0)$. So the map $\omega$ is non-zero and by 
    \begin{equation*}
        \dim \Hom_{K_G}(\liep^{+}_{G, \CC} \otimes \liep^{-}_{G, \CC}, D_{\CC} \otimes \tilde{\pi}_{\infty}) = 1, 
    \end{equation*}
    $\omega$ is uniquely determined. 
\end{proof}

\begin{remark}
    \begin{itemize}
        \item  Since $\omega$ is uniquely determined by the equaiton (\ref{eqn:form_satisfied}), $\omega$ must satisfy
                \begin{equation*}
                    \omega(X_{(0, 0, 0)}) = 0,
                \end{equation*}
                for any $X_{(0, 0, 0)} \in \tau_{(0, 0, 0)}$. 
        \item Since $v \in D_{\CC}$ is a vector of weight $(-a + s - r, b + s - r, s - r; -b - 2s + r)$ and the highest weight of $D_{\CC}$ is $(b + s - r, s - r, -a + s - r, -a + s - r; -b -2s + r)$, we can see that $v$ is lowest weight vector in $D_{\CC}$ with respect to the positive root system 
        \begin{equation*}
            \{ (1, -1, 0), (0, -1, 1), (-1, 0, 1) \},
        \end{equation*}
        which guarantees the existence of $v$. 
        Hence, we have 
        \begin{equation}
        \label{eq: Xv = 0}
            X_{(1, 0, -1)}^{a + b + 1}v = 0. 
        \end{equation}
    \end{itemize}
\end{remark}

\subsection{Restriction of differential forms}
\label{SS: res of the diff form}

Recall that we have the following inclusion of algebraic groups 
\[
    \iota\colon H \hookrightarrow G,  \quad (\begin{pmatrix} 
                                            a & b \\
                                            c & d \\
                                         \end{pmatrix}, z) \mapsto (\begin{pmatrix}
                                                                        a & & b \\
                                                                        & z & \\
                                                                        c & & d 
                                                                     \end{pmatrix}, z\bar{z}), 
\]
which induces the closed immersion of Shimura varieties 
\[
    \iota \colon \Sh_{H} \hookrightarrow \Sh_{G}. 
\]
Now, we compute the pullback of the differential form 
\begin{equation*}
    \iota^{*} \omega_{\Psi} = \iota^{*} (\omega \otimes \Psi_f) = \iota^{*}(\omega) \otimes \Psi_{f}
\end{equation*}
along the map $\iota$. Hence, it suffices to compute $\iota^{*}(\omega)$. 
\par Recall the complexified Lie algebra of $H$ is $\lieh_{\CC} = \liek_{H, \CC} \oplus \liep_{H, \CC}^{+} \oplus \liep_{H, \CC}^{-}$. Let 
\begin{equation}
\label{eq: def of v^{+} and v^{-}}
    v^{+} = \begin{pmatrix}
                0 & 1 \\
                0 & 0 
            \end{pmatrix} \in \liep_{H, \CC}^{+},\ v^{-} = \begin{pmatrix}
                                                                    0 & 0 \\
                                                                    1 & 0
                                                            \end{pmatrix} \in \liep_{H, \CC}^{-} . 
\end{equation}
be two root vectors in $\liegl_{2, \CC}$. So under the tangent map $\iota_{*}$ of the map $\iota$ of Shimura varieties, we have 
\begin{equation*}
    \iota_{*}(v^{+}) = X_{(1, 0, -1)}, \  \iota_{*}(v^{-}) = X_{(-1, 0, 1)}, 
\end{equation*}
so
\begin{equation*}
    \iota_{*} (v^{+} \otimes v^{-}) = \iota_{*}v^{+} \otimes \iota_{*} v^{-} = X_{(1, 0, -1)} \otimes X_{(-1, 0, 1)}.
\end{equation*}

\begin{lemma}
\label{Lemma: pushforward of v}
    If we set
    \begin{equation*}
        Y_{(1, -1, 0)}:= \mathrm{ad}_{X_{(-1, 1, 0)}}(X_{(1, 0, -1)} \otimes X_{(0, -1, 1)}). 
    \end{equation*} and $Y_{(0, 0, 0)}$ is a vector in $\tau_{(0, 0, 0)}$, then we have 
    \begin{equation*}
        \iota_{*} (v^{+} \otimes v^{-}) = -\frac{1}{2} Y_{(1, -1, 0)} + \beta Y_{(0, 0, 0)}
    \end{equation*}
    for some $\beta \in \CC$.
\end{lemma}

\begin{proof}
    First, it follows from the fact that $X_{(1, 0, -1)} \otimes X_{(0, -1, 1)}$ is a highest weight vector of $\tau_{(1, -1, 0)}$ as a $K_G$-representation that 
    $Y_{(1, -1, 0)}$ is vector in $\tau_{(1, -1, 0)}$ with weight $(0, 0, 0)$. Hence, we could let 
    \begin{equation*}
        \iota_{*} (v^{+} \otimes v^{-}) = \alpha Y_{(1, -1, 0)} + \beta Y_{(0, 0, 0)}
    \end{equation*}
    for some $\alpha, \beta \in \CC$, from which we have 
    \begin{equation}
    \label{eqn:claim_proof_pullback_form}
        \mathrm{ad_{X_{(1, -1, 0)}}}( \iota_{*} (v^{+} \otimes v^{-})) = \alpha \mathrm{ad_{X_{(1, -1, 0)}}}(\mathrm{ad}_{X_{(-1, 1, 0)}}(X_{(1, 0, -1)} \otimes X_{(0, -1, 1)})). 
    \end{equation}
    \par Second, we can compute the left hand side of the formula (\ref{eqn:claim_proof_pullback_form}) aså 
    \begin{align*}
        \mathrm{ad_{X_{(1, -1, 0)}}}( \iota_{*} (v^{+} \otimes v^{-})) & = \mathrm{ad_{X_{(1, -1, 0)}}}(X_{(1, 0, -1)} \otimes X_{(-1, 0, 1)}) \\
                                                                   & = X_{(1, 0, -1)} \otimes \mathrm{ad_{X_{(1, -1, 0)}}}(X_{(-1, 0, 1)}) \\
                                                                   & = X_{(1, 0, -1)} \otimes [X_{(1, -1, 0)}, X_{(-1, 0, 1)}] \\
                                                                   & = - X_{(1, 0, -1)} \otimes X_{(0, -1, 1)}. 
    \end{align*}
    \par Third, if we let 
    \begin{equation*}
        H := [X_{(1, -1, 0)}, X_{(-1, 1, 0)}] = \left(\begin{matrix}
                                                    1 & 0 & 0 \\
                                                    0 & -1 & 0 \\
                                                    0 & 0 & 0 
                                                \end{matrix}\right), 
    \end{equation*}
    then we have 
    \begin{align*}
        \alpha \mathrm{ad_{X_{(1, -1, 0)}}}(\mathrm{ad}_{X_{(-1, 1, 0)}}(X_{(1, 0, -1)} \otimes X_{(0, -1, 1)})) &= \alpha \mathrm{ad_{X_{(-1, 1, 0)}}}(\mathrm{ad}_{X_{(1, -1, 0)}}(X_{(1, 0, -1)} \otimes X_{(0, -1, 1)})) \\
        & + \alpha \mathrm{ad_{[X_{(1, -1, 0)}, X_{(-1, 1, 0)}]}}(X_{(1, 0, -1)} \otimes X_{(0, -1, 1)}) \\
        & = 0 + \alpha \mathrm{ad}_{H}(X_{(1, 0, -1)} \otimes X_{(0, -1, 1)}) \\
        & = \alpha X_{(1, 0, -1)} \otimes X_{(0, -1, 1)} + \alpha X_{(1, 0, -1)} \otimes X_{(0, -1, 1)} \\
        & = 2\alpha X_{(1, 0, -1)} \otimes X_{(0, -1, 1)}. 
    \end{align*}
    Finally, by equation (\ref{eqn:claim_proof_pullback_form}), we have 
    \begin{equation*}
        - X_{(1, 0, -1)} \otimes X_{(0, -1, 1)} = 2\alpha X_{(1, 0, -1)} \otimes X_{(0, -1, 1)}.
    \end{equation*}
    Hence, we conclude that $\alpha = -\frac{1}{2}$. \\
\end{proof}

\begin{lemma}
\label{Lemma: lower and up vector}
For $0 \le i \le a + b$, we have the following identities: 
\begin{enumerate} 
        \item $X_{(-1, 1, 0)}X^{2 + a + b - i}_{(1, -1, 0)} \Psi_{\infty}  = (1 + i)(2 + a + b - i)X_{(1, -1, 0)}^{1 + a + b - i} \Psi_{\infty}$, 
        \item $X_{(-1, 1, 0)} X_{(1, -1 ,0)}^{i} v = \begin{cases}
                                                            i(a + b + 1 - i) X^{i - 1}_{(1, -1, 0)} v & i > 0 \\
                                                            0 & i = 0
                                                          \end{cases}. $
\end{enumerate}    
\end{lemma}

\begin{proof}
    Since $\Psi_{\infty}$ is the lowest weight vector with weight $(-1 - b + r - s, 1 + a + r - s, r - s; b + 2s -r)$ in the minimal $K_G$-type $\tau_{(1 + a + r -s, -1-b +r -s, r-s)}$ of $\tilde{\pi}_{\infty}$, which is the vector $\nu_{0}$ in the representation $\tau_{(1 + a + r -s, -1-b +r -s, r-s)} $ in our setup of representations of $K_{G}$ (see equation (\ref{eqn: action on K-type})), 
        we have 
        \begin{align*}
            X_{(-1, 1, 0)} X^{2 + a + b - i}_{(1, -1, 0)} \Psi_{\infty}  &= X_{(-1, 1, 0)} X^{2 + a + b - i}_{(1, -1, 0)} \nu_0 \\
            & = (2 + a + b - i)! X_{(-1, 1, 0)} \nu_{2 + a + b - i}. \\
            & = (1 + i)(2 + a + b - i)! \nu_{1 + a + b - i}. 
        \end{align*}
        We also have
        \begin{equation*}
            X^{1 + a + b - i}_{(1, -1, 0)} \Psi_{\infty}  = (1 + a + b - i)! \nu_{1 + a + b - i}. 
        \end{equation*}
        Hence, we get
        \begin{equation*}
            X_{(-1, 1, 0)} X^{2 + a + b - i}_{(1, -1, 0)} \Psi_{\infty}  = (1 + i)(2 + a + b - i) X^{1 + a + b - i}_{(1, -1, 0)} \Psi_{\infty}. 
        \end{equation*}
        The computation of $X_{(-1, 1, 0)} X_{(1, -1 ,0)}^{i} v$ is similar, which we finishes the proof of the lemma. \\
\end{proof}

\begin{proposition}
\label{Prop:pullback_form}
    If we pullback the $\omega$ defined in Proposition \ref{Prop: explicit construnction of the differential form} along the map $\iota$, then we have the following formula:
    \begin{equation*}
        \iota^{*}\omega(v_{+} \otimes v_{-}) = - \sum_{i = 0}^{a + b} (-1)^{i}(i + 1) X_{(1, -1 ,0)}^{i}v \otimes X^{1 + a + b - i}_{(1, -1 , 0)} \Psi_{\infty}. 
    \end{equation*}
\end{proposition}

\begin{proof}
    The pullback functor $\iota^{*}$ preserves the type of the differential form and the type of a differential form on a Shimura variety is determined by the archimedean part, $\iota^{*}\omega$ is determined by its value on $v^{+} \otimes v^{-}$. 
    It follows from Lemma \ref{Lemma: pushforward of v} that
    \begin{align*}
        \iota^{*}\omega(v^{+} \otimes v^{-}) &= \omega(\iota_{*}(v^{+} \otimes v^{-})) \\
                                             &= \omega(-\frac{1}{2}\mathrm{ad}_{X_{(-1, 1, 0)}}(X_{(1, 0, -1)} \otimes X_{(0, -1, 1)})) \\
                                             & = -\frac{1}{2} X_{(-1, 1, 0)} (\omega(X_{(1, 0, -1)} \otimes X_{(0, -1, 1)})) \\
                                             & = -\frac{1}{2} X_{(-1, 1, 0)}(\sum\limits_{i = 0}^{a + b} (-1)^{i} X_{(1, -1 ,0)}^{i} v \otimes X^{2 + a + b - i}_{(1, -1, 0)} \Psi_{\infty}) \\
                                             & = -\frac{1}{2} \sum\limits_{i = 0}^{a + b} (-1)^{i} X_{(-1, 1, 0)} X_{(1, -1 ,0)}^{i} v \otimes X^{2 + a + b - i}_{(1, -1, 0)} \Psi_{\infty} \\
                                             & -\frac{1}{2} \sum\limits_{i = 0}^{a + b} (-1)^{i} X_{(1, -1 ,0)}^{i} v \otimes X_{(-1, 1, 0)}X^{2 + a + b - i}_{(1, -1, 0)} \Psi_{\infty}. 
    \end{align*} 
    Plugging the formula in Lemma \ref{Lemma: lower and up vector} into the formula for $\iota^{*}\omega(v^{+} \otimes v^{-})$ above, we get 
    \begin{align*}
        \iota^{*}\omega(v^{+} \otimes v^{-}) & = -\frac{1}{2} \sum\limits_{i = 1}^{a + b} (-1)^{i}i(a + b + 1 - i) X_{(1, -1 ,0)}^{i - 1} v \otimes X^{2 + a + b - i}_{(1, -1, 0)} \Psi_{\infty} \\
                                             & -\frac{1}{2} \sum\limits_{i = 0}^{a + b} (-1)^{i} (i + 1)(2 + a + b - i) X_{(1, -1 ,0)}^{i} v \otimes X^{1 + a + b - i}_{(1, -1, 0)} \Psi_{\infty} \otimes \\
                                             & = \frac{1}{2} \sum\limits_{i = 0}^{a + b - 1} (-1)^{i}(i + 1)(a + b - i) X_{(1, -1 ,0)}^{i} v \otimes X^{1 + a + b - i}_{(1, -1, 0)} \Psi_{\infty} \\
                                             & -\frac{1}{2} \sum\limits_{i = 0}^{a + b} (-1)^{i} (i + 1)(2 + a + b - i) X_{(1, -1 ,0)}^{i} v \otimes X^{1 + a + b - i}_{(1, -1, 0)} \Psi_{\infty} \\
                                             & = - \sum\limits_{i = 0}^{a + b} (-1)^{i}(i + 1) X_{(1, -1 ,0)}^{i} v \otimes X^{1 + a + b - i}_{(1, -1, 0)} \Psi_{\infty}. 
    \end{align*}
\end{proof}

\subsection{Deligne{\textendash}Beilinson cohomology and tempered currents}
 \label{SS: DB cohomology}
 In this subsection, we collect some facts about Deligne{\textendash}Beilinson cohomology and tempered currents, mainly following \cite{BCLRJ24}. 

\subsubsection{Classical Definition of Deligne{\textendash}Beilinson Cohomology}
We first recall the classical definition of Deligne{\textendash}Beilinson Cohomology (DB-cohomology for short). 

\par Let $X$ be a smooth, quasi-projective complex analytic variety of pure dimension $d$. Let $\overline{X}$ be a smooth compactification of $X$ such that $D = \overline{X} - X$ is a simple normal crossings divisor. We let $j \colon X \rightarrow \overline{X}$ be the open immersion. We will assume that $X$ is defined as the analytification of the base change to $\CC$ of a smooth, quasi-projective $\RR$-scheme. For $p \in \ZZ$, let $\RR(p)$ denote the subgroup  $(2\pi i)^{p} \RR$ of $\CC$. We will also use the same notation to denote the constant sheaf with value $\RR(p)$ on $X$. 

\par Let $\Omega^{*}_{X}$ be the sheaf of holomorphic differential forms on $X$ and let $\Omega_{\overline{X}}^{*}(\mathrm{log}\, D)$ be the sheaf of differential forms on $X$ with logarithmic singularities along $D$.  It is defined to be the sheaf of sub-$\mathcal{O}_{\overline{X}}$-module of $j_{*} \Omega_{X}$ locally generated by sections $\xi_{i}, \, 1 \le i \le d$ (see \cite[(3.1.2)]{HodgeII}). Here, we let 
\begin{equation*}
    \xi_{i} = \begin{cases}
                \frac{dz{i}}{z_{i}} &  1 \le i \le m \\
                dz_{i} & m < i \le d. 
               \end{cases}
\end{equation*}
The Hodge filtration  on $\Omega_{\overline{X}}^{*}(\mathrm{log}\, D)$ is defined as 
\begin{equation*}
    F^{p}\Omega_{\overline{X}}^{*}(\mathrm{log}\, D) = \bigoplus_{p^{\prime} \ge p}\Omega_{\overline{X}}^{p^{\prime}}(\mathrm{log}\, D). 
\end{equation*}
\begin{fact}
\label{Fact: Betti to log_de Rham}
    There are natural quasi-isomorphisms: 
    \begin{enumerate}
        \item  $Rj_{*} \CC \rightarrow Rj_{*} \Omega_{X}^{*}$,
        \item  $\Omega^{*}_{\overline{X}}(\mathrm{log}\, D) \rightarrow Rj_{*}\Omega_{X}^{*}$ \cite[(3.1.8.2)]{HodgeII}. 
    \end{enumerate}
\end{fact}

\begin{definition}
   \begin{enumerate}
       \item For each $p \in \ZZ$, the \textit{DB-cohomology group} $\mathrm{H}_{D}^{n}(X, \RR(p))$ of $X$ with coefficients in $\RR(p)$ is defined as the $n$-th hypercohomology of the complex 
       \begin{equation}
           \RR(p)_{D} := \mathrm{Cone}(Rj_{*}\RR(p) \oplus F^{p}\Omega_{\overline{X}}^{*}(\mathrm{log}\, D) \rightarrow Rj_{*} \Omega_{X}^{*})[-1], 
       \end{equation}
       where the arrow is given by the difference of the natural maps. 
       \item Let $\overline{F^{*}_{\infty}} = F_{\infty}^{*} \otimes c$ be the de Rham involution given by the complex conjugation $F_{\infty}$ on $X$ and $c$ on the coefficients. The \textit{real DB-cohomology groups} are defined as 
       \begin{equation}
           \mathrm{H}_{D}^{n}(X/\RR, \RR(p)) := \mathrm{H}_{D}^{n}(X, \RR(p))^{\overline{F_{\infty}^{*}} = 1}. 
       \end{equation}
   \end{enumerate}
\end{definition}    

\subsubsection{Tempered Currents}
We recall the definition of tempered currents in \cite{BCLRJ24} here. 

Let $\Delta = \{ z \in \CC | |z| < \frac{1}{2} \}$ be the open disc of radius $\frac{1}{2}$ and $\Delta^{*} = \{z \in \CC | 0 < |z| < \frac{1}{2} \}$ be the open punctured disc of radius $r$. 

\par For each point $x \in \overline{X}$, there is an open neighborhood $U$ of $x$ that is isomorphic to $\Delta^{d}$ with coordinates $(z_1, z_2, \ldots, z_{d})$ for which $x = (0, 0, \ldots, 0)$, and such that there exist some integers $m, n$ with $0 \le n, m \le d$ and $m + n = d$ such that 
\begin{equation*}
    X \cap U = (\Delta^{*})^{m} \times (\Delta)^{n} = \{ (z_1, z_2, \cdots, z_{d}) \in \Delta^{d} | z_1 z_2 \cdots z_{m} \neq 0 \}.
\end{equation*}
We call such a subset $U$ an open coordinate neighborhood of $x$. 

\par We denote by $\mathcal{A}_{\overline{X}}^{0}$ the sheaf of smooth functions on $\overline{X}$ and let $\mathcal{A}^{*}_{\overline{X}}$ be the complex of sheaves of smooth differential forms. The complex $\mathcal{A}^{*}_{\overline{X}}(\mathrm{log}\, D)$ of smooth differential forms on $X$ with logarithmic growth at $D$ is defined to be the $\mathcal{A}_{\overline{X}}$-algebra subsheaf of $j_{*} \mathcal{A}_{X}$ locally generated by sections $\log|z_{i}|\, (1 \le i \le m)$ and $\xi_{i}, \bar{\xi_{i}} \, (1 \le i \le d)$. \cite[\S 2]{Burgos94} 

\par We first give the definition of sheaves of differential forms with growth conditions. 
\begin{definition}
    We denote by $\mathcal{A}_{si}^{0}$ (resp., $\mathcal{A}_{rd}^{0}$) the sheaf on $\overline{X}$ whose sections on an open $U \subset \overline{X}$ is given by complex-valued functions on $U \cap X$, which are locally slowly increasing (resp. rapidly decreasing) \cite[Definition 2.3]{BCLRJ24} at each point of $U$. The graded sheaf $\mathcal{A}_{si}^{*}$ of slowly increasing differential forms is defined to be the $\mathcal{A}_{si}^{0}$-subalgebra of $j_{*} \mathcal{A}_{X}^{*}$ locally generated by $\xi_{i}, \bar{\xi_{i}}$ for $1 \le i \le d$.  Similarly, the graded sheaf $\mathcal{A}_{rd}^{*}$ of rapidly decreasing differential forms is defined to be the $\mathcal{A}_{rd}^{0}$-subalgebra of $j_{*}\mathcal{A}_{X}^{*}$ locally generated by $\xi_{i}, \bar{\xi_{i}}$ for $1 \le i \le d$. We denote by $\mathcal{A}_{si}(\overline{X})$ and $\mathcal{A}_{rd}(\overline{X})$ the corresponding complex of global sections. 
\end{definition}

\begin{remark}
    \begin{enumerate}
        \item There are natural inclusions 
        \begin{equation*}
            \mathcal{A}_{rd}^{*} \subseteq \mathcal{A}_{\overline{X}}^{*}(\log D) \subseteq \mathcal{A}_{si}^{*} \subseteq j_{*} \mathcal{A}^{*}_{X}. 
        \end{equation*}
        Moreover, all these sheaves are fine sheaves. 
        \item The complex structure on $\overline{X}$ induces compatible bigradings
        \begin{equation*}
            \mathcal{A}_{rd}^{n} = \bigoplus_{p + q = n} \mathcal{A}_{rd}^{p,q}, \,  \mathcal{A}_{si}^{n} = \bigoplus_{p + q = n} \mathcal{A}_{si}^{p,q}, \, \mathcal{A}_{\overline{X}}^{n}(\log \, D) = \bigoplus_{p + q = n} \mathcal{A}_{\overline{X}}^{p, q}(\log D), 
        \end{equation*}
        with correspoding Hodge filtrations 
        \begin{equation*}
            F^{p}\mathcal{A}_{rd}^{n} = \bigoplus_{p^{\prime} \ge p} \mathcal{A}_{rd}^{p^{\prime},q}, F^{p}\mathcal{A}_{si}^{n} = \bigoplus_{p^{\prime} \ge p} \mathcal{A}_{si}^{p^{\prime},q}, 
            F^{p}\mathcal{A}_{\overline{X}}^{n} (\log \, D) = \bigoplus_{p^{\prime} \ge p} \mathcal{A}_{\overline{X}}^{p^{\prime}, q} (\log \, D). 
        \end{equation*}
        \item We denote by 
        \begin{equation*}
            \mathcal{A}_{si, \RR}^{*} \subseteq \mathcal{A}_{si}^{*},  \mathcal{A}_{rd, \RR}^{*} \subseteq \mathcal{A}_{rd}^{*}, \mathcal{A}_{\overline{X}, \RR}^{*}(\log \, D) \subseteq \mathcal{A}_{\overline{X}}^{*}(\log \, D)
        \end{equation*}
        the subcomplexes of sheaves of $\RR$-valued differential forms. 
    \end{enumerate}
\end{remark}

    We now give the definition of sheaves of tempered currents. 

\begin{definition}
        \begin{enumerate}
            \item For any $0 \le p,q \le d$, the sheaf $\mathcal{D}_{si}^{p,q}$ of tempered currents is defined to be the sheaf on $\overline{X}$ assigning to any open coordinate neighborhood $U \subseteq \overline{X}$ the complex vector space $\mathcal{D}^{p,q}_{si}(U)$ of continuous linear forms on the compactly supported sections $\Gamma_{c}(U, \mathcal{A}_{rd}^{d - p, d - q})$. Similarly,  We denote by $\mathcal{D}_{si, \RR}^{p,q}$ the sheaf on $\overline{X}$ assigning to any open coordinate neighborhood $U \subseteq \overline{X}$ the real vector space $\mathcal{D}_{si, \RR}^{p,q}(U)$ of continuous linear forms on the compactly supported sections $\Gamma_{c}(U, \mathcal{A}_{rd, \RR}^{d - p, d -q})$. 
            \item For $n \in \ZZ_{\ge 0}$, let 
            \begin{equation*}
                \mathcal{D}_{si}^{n} = \bigoplus_{p + q = n} \mathcal{D}_{si}^{p, q},  \mathcal{D}_{si, \RR}^{n} = \bigoplus_{p + q = n} \mathcal{D}_{si, \RR}^{p, q}. 
            \end{equation*}
            The exterior differentials $\partial$ and $\overline{\partial}$ on $\mathcal{D}_{si}^{*}$ are defined as follows: for any open coordinate neighborhood $U \subseteq \overline{X}$, any $T \in \mathcal{D}^{p, q}_{si}(U)$ and any $\omega \in \Gamma_{c}(U, \mathcal{A}_{rd}^{d - p, d - q})$, we have 
            \begin{equation*}
                \partial T(w) := (-1)^{p + q + 1} T(\partial w),  \overline{\partial} T(w) := (-1)^{p + q + 1} T(\overline{\partial} w). 
            \end{equation*}
            The exterior differential $d \colon \mathcal{D}_{si}^{n} \rightarrow \mathcal{D}_{si}^{n + 1}$ is defined as $d = \partial + \overline{\partial}$. This defines complexes of sheaves of tempered currents $\mathcal{D}^{*}_{si}, \mathcal{D}_{si}^{*, *}$ (with differential $d, \partial, \overline{\partial}$). We will denote by $\mathcal{D}_{si}^{*, *}(\overline{X})$ the corresponding complex of global sections. 
            \item The complex $D^{*}_{si}$ is equipped with a Hodge filtration given by 
            \begin{equation*}
                F^{p}D^{*}_{si} = \bigoplus_{p^{\prime} \ge p} \mathcal{D}_{si}^{p^{\prime}, q}. 
            \end{equation*}
            \item For any open $U \subseteq \overline{X}$, there is a natural way to associate any form $\omega \in \mathcal{A}_{si}^{p, q}(U)$ a current $T_{\omega} \in \mathcal{D}_{si}^{p, q}(U)$ given by 
            \begin{equation*}
                T_{\omega}(\eta) = \frac{1}{(2\pi i)^{d}} \int_{U} \omega \wedge \eta , \, (\eta \in \Gamma_{c}(U, \mathcal{A}_{rd}^{d - p, d - q})). 
            \end{equation*}
        \end{enumerate}
\end{definition}

\begin{remark}
    Since currents are modules over $\mathcal{A}^{0}_{si}$, all these complexes are complexes of fine sheaves. 
\end{remark}

\begin{proposition} \textnormal{\cite[Theorem 1.1]{BCLRJ24}}
\label{Prop: BGCLRJ24 Theoreom 1.1}
    The natural inclusions 
    \begin{equation*}
        (\Omega^{*}_{\overline{X}}(\log\, D), F) \rightarrow (\mathcal{A}^{*}_{\overline{X}} (\log\, D), F) \rightarrow (\mathcal{A}_{si}^{*}, F) \rightarrow (\mathcal{D}^{*}_{si}, F)
    \end{equation*}
    are filtered quasi-isomorphisms. Futhermore, the last two quasi-isomorphisms are compatible for the corresponding real structures. 
\end{proposition}

\begin{remark}
\label{remark: Betti coh and tempered currents}
    By Proposition \ref{Prop: BGCLRJ24 Theoreom 1.1}, Fact \ref{Fact: Betti to log_de Rham} and, the fact that $\mathcal{D}_{si}^{*}$ is a fine sheaf, we can see that Betti cohomology classes of $X$ can be represented by closed tempered currents. 
\end{remark}


\subsubsection{Deligne{\textendash}Beilinson cohomology in terms of tempered currents}
In \cite{BCLRJ24}, the authors give a new definition of Deligne{\textendash}Beilinson cohomology in terms of smooth slowly increasing differential forms and tempered currents, which will be recalled here. This definition is convenient for computing Beilinson regulators. 

\begin{proposition}
\label{Prop: DB-coh_si forms}
    There is a quasi-isomorphism 
    \begin{equation*}
        \RR(p)_{D} \simeq \mathrm{Cone}(F^{p}\mathcal{A}^{*}_{si} \rightarrow \mathcal{A}_{si, \RR(p-1)}^{*})[-1],
    \end{equation*}
    where the arrow is induced by the projection $\pi_{p-1} \colon \CC \rightarrow \RR(p - 1)$ defined by $\pi_{p - 1}(z) = \frac{z + (-1)^{p-1} \overline{z}}{2}$. In particular, we have the canonical isomorphism 
    \begin{equation}
        \mathrm{H}_{D}^{n}(X, \RR(p)) \simeq \frac{\{ (\phi, \phi^{\prime}) \in  \mathcal{A}_{si, \RR(p-1)}^{n - 1}(\overline{X}) \oplus F^{p} \mathcal{A}_{si}^{n}(\overline{X}) | d \phi^{\prime} = 0, d \phi = \pi_{p - 1}(\phi^{\prime})  \}}{ \{ d(\tilde{\phi}, \tilde{\phi^{\prime}})\}}, 
    \end{equation}
    where $d(\tilde{\phi}, \tilde{\phi^{\prime}}) = (d \tilde{\phi} - \pi_{p -1}(\tilde{\phi^{\prime}}), d\tilde{\phi^{\prime}})$. 
\end{proposition}

\begin{proof}
    See \cite[Proposition 2.23]{BCLRJ24}. 
\end{proof}

\begin{proposition}
\label{Prop: DB-coh_temperd currents}
    There is a quasi-isomorphism 
    \begin{equation*}
        \RR(p)_{D} \simeq \mathrm{Cone}(F^{p}\mathcal{D}^{*}_{si} \rightarrow \mathcal{D}_{si, \RR(p-1)}^{*})[-1].
    \end{equation*}
    In particular, we have the canonical isomorphism 
    \begin{equation}
        \mathrm{H}_{D}^{n}(X, \RR(p)) \simeq \frac{\{ (S, T) \in  \mathcal{D}_{si, \RR(p-1)}^{n - 1}(\overline{X}) \oplus F^{p} \mathcal{D}_{si}^{n}(\overline{X}) | d T = 0, d S = \pi_{p - 1}(T)  \}}{ \{ d(\tilde{S}, \tilde{T})\}}, 
    \end{equation}
    where $d(\tilde{S}, \tilde{T}) = (d \tilde{S} - \pi_{p -1}(\tilde{T}), d\tilde{T})$. 
\end{proposition}

\begin{proof}
    See \cite[Theorem 2.25]{BCLRJ24}. 
\end{proof}

In what follows, we will use $[(S, T)] \in \mathrm{H}_{D}^{n}(X, \RR(p))$ to denote the cohomology class of the pair $(S, T)$. 


\begin{proposition}
\label{Prop: DB-cohomology-si forms to tempered currents}
    Let $x \in  \mathrm{H}_{D}^{n}(X, \RR(p))$ be a DB-cohomology class which is represented by a pair $(\phi, \phi^{\prime})$ of smooth slowly increasing differential forms via Proposition \ref{Prop: DB-coh_si forms}. Then via the isomorphism of Proposition \ref{Prop: DB-coh_temperd currents}, the class $x$ is represented by the pairs of tempered currents $(T_{\phi}, T_{\phi^{\prime}})$. 
\end{proposition} 

\begin{proof}
    See \cite[Proposition 2.26]{BCLRJ24}. 
\end{proof}

The reinterpretation of DB-cohomology classes in terms of tempered currents allows us to describe explicitly the Gysin morphism as follows. 
Let $\iota \colon X^{\prime} \hookrightarrow X$ be a closed immersion of pure codimension $c$. Let $\overline{X^{\prime}}$ be a smooth compactification of $X^{\prime}$ such that $D^{\prime} = \overline{X^{\prime}} - X^{\prime}$ is a simple normal crossings divisor. We further assume that $\iota$ extends to a morphism $\iota \colon \overline{X^{\prime}} \rightarrow \overline{X}$ such that $\iota^{-1}(D) = D^{\prime}$. Then we define the Gysin morphism 
\begin{equation}
\label{eq: Gysin for DB-cohomology}
    \iota_{*} \colon \mathrm{H}_{D}^{n}(X^{\prime}, \RR(p)) \rightarrow \mathrm{H}_{D}^{n + 2c}(X, \RR(p + c))
\end{equation}
by $\iota_{*}[(S, T)] = [(\iota_{*}S, \iota_{*}T)]$. 
Here for a tempered current $T$, we denote by $\iota_{*}T$ the tempered current defined as: for any smooth rapidly decreasing differential form $\omega$,
\begin{equation}
\label{eq: pushforward of tempered currents}
    \iota_{*}T(\omega) = T(\iota^{*} \omega), 
\end{equation}
which makes sense because $\iota^{*} \omega$ is also rapidly decreasing. 

\begin{proposition}
    Let $n \in \ZZ_{\ge 0}$, $p \in \ZZ$ and let $\omega \in \mathcal{A}_{rd}^{2d - n}(\overline{X})$ be a smooth closed rapidly decreasing  differential form with Hodge type components inside $\{(a, b): a, b > d - p\}$. Then the assignment $(S, T) \mapsto S(\omega)$ induces a linear map 
    \begin{equation*}
        \langle -, \omega \rangle \colon \mathrm{H}_{D}^{n}(X^{\prime}, \RR(p)) \rightarrow \CC. 
    \end{equation*}
\end{proposition}

\begin{proof}
    See \cite[Proposition 2.27]{BCLRJ24}
\end{proof}

\subsection{DB-cohomology with coefficients}
\label{SS: DB-cohomology with coeffients}
In this subsection, we give the definition of DB-cohomology with coefficients using the ``Liebermann’s trick'':

Let $p \colon A \rightarrow S$ be the universal abelian $3$-fold over $S$. 

\begin{proposition}
\label{Prop:motiv_coh_abs_relative}
    We have a inclusion. 
    \begin{equation}
        \mathrm{H}^{3}_{H}(S, V(2)) \subset \mathrm{H}_{H}^{a + 5b - 3(r+s) + 3}(A^{a + 5b - 3(r+s)}, \RR(a + 3b -r - s + 2)). 
    \end{equation}
\end{proposition}

\begin{proof}
    Since $V = V^{a, b}\{a, b \} = \lambda(a + r - s, r - s,  r - s - b; 2s - r + b)$ is in $\mathrm{Rep}(G)$, we have   
    \begin{equation*}
        V \subset (\mathrm{std})^{\otimes(a + 5b - 3(r+s))}(2s + 2r -2b),
    \end{equation*}
    where $\mathrm{std}$ is the standard representation of $G$. We then get the inclusion
    \begin{equation}
    \label{eq: inclusion of sheaf in DB with coefficent}
        V \subset \mathcal{H}^{a + 5b - 3(r + s)}p_{*}\RR(a + 3b -r - s)_{A^{a + 5b - 3(r + s)}}
    \end{equation}
    in $\mathrm{D}^{b}(\mathrm{MHM}_{\RR}(S/\RR))$ by applying the functor $\mu_{H}$ to the inclusion (\ref{eq: inclusion of sheaf in DB with coefficent}). 
    Recall that for the morphism $p \colon A^{a + 5b - 3(r+s)} \rightarrow S$, we have the following isomorphism in $\mathrm{D}^{b}(\mathrm{MHM}_{\RR}(S/\RR))$ \cite[(4.5.4)]{MHM90}: 
    \begin{equation*}
        p_{*}\RR(a + 3b -r - s) \cong \bigoplus\limits_{i} \mathcal{H}^{i}p_{*} \RR(a + 3b -r - s)_{A^{a + 5b - 3(r + s)}}[-i]. 
    \end{equation*}
    Thus, we can see that
    \begin{equation*}
        \mathcal{H}^{a + 5b - 3(r + s)} p_{*}\RR(a + 3b - r - s)_{A^{a + 5b - 3(r + s)}} \subset p_{*}\RR(a + 3b - r - s)[a + 5b - 3(r+s)]. 
    \end{equation*}
    Finally, if we let $\mathcal{A} = A^{a + 5b - 3(r+s)}$, then it can be seen from the definition of absolute Hodge cohomology that  
    \begin{align*}
        \mathrm{H}^{3}_{H}(S, V(2)) &= \Hom_{\mathrm{D}^{b}(\mathrm{MHM}_{\RR}(S/\RR))}(\RR(0)_{S}, V(2)[3]) \\
                                    & \subset \Hom_{\mathrm{MHM}_{\RR}(S/\RR))}(\RR(0)_{S}, p_{*}\RR(a + 3b -r - s + 2)[a + 5b - 3(r+s) + 3]) \\ 
                                    & = \Hom_{\mathrm{MHM}_{\RR}(\mathcal{A}/\RR))}(\RR(0)_{\mathcal{A}}, \RR(a + 3b -r - s + 2)[a + 5b - 3(r+s) + 3]) \\
                                    & = \mathrm{H}_{H}^{a + 5b - 3(r+s) + 3}(\mathcal{A}, \RR(a + 3b - r -s + 2)). 
    \end{align*}
\end{proof}

\begin{definition}
\label{Def: DB-coh with coefficients}
    Let $i = a + 5b - 3(r + s)$ and $j = a + 3b -r - s + 2$. 
    \begin{enumerate}
        \item We define $\mathrm{H}^{3}_{D}(S, V(2))$ as 
              \begin{equation*}
                    \mathrm{H}^{3}_{D}(S, V(2)) := r_{H \rightarrow D} (\mathrm{H}^{3}_{H}(S, V(2))), 
              \end{equation*}
               where $r_{H \rightarrow D}$ is the natural map 
               \begin{equation*}
                   r_{H \rightarrow D} \colon \mathrm{H}_{H}^{i + 3}(A^{i}, \RR(j)) \rightarrow \mathrm{H}_{D}^{i + 3}(A^{i}, \RR(j))
               \end{equation*}
               and $A$ is the universal abelian $3$-fold over $S$;
        \item We define $\mathrm{H}^{1}_{D}(M, \iota^{*}V(1))$ as 
              \begin{equation*}
                    \mathrm{H}^{1}_{D}(M, \iota^{*}V(1)) := r_{H \rightarrow D} (\mathrm{H}^{1}_{H}(M, \iota^{*}V(1))), 
              \end{equation*}
               where $r_{H \rightarrow D}$ is the natural map 
               \begin{equation*}
                   r_{H \rightarrow D} \colon \mathrm{H}_{H}^{i + 1}(A^{i}, \RR(j - 1)) \rightarrow \mathrm{H}_{D}^{i + 1}(A^{i}, \RR(j - 1))
               \end{equation*}
               and $A$ is the pullback of the universal abelian $3$-fold over $S$ to $M$ through $\iota$;
        \item We define $\mathrm{H}^{1}_{D}(M, W(1))$ as 
              \begin{equation*}
                    \mathrm{H}^{1}_{D}(M, W(1)) := r_{H \rightarrow D}(\mathrm{H}_{H}(M, W(1))), 
              \end{equation*}
              where $r_{H \rightarrow D}$ is the map 
              \begin{equation*}
                  r_{H \rightarrow D} \colon \mathrm{H}^{n + 1}_{H} (E^{n}, \RR(n + 1)) \rightarrow \mathrm{H}^{n + 1}_{D} (E^{n}, \RR(n + 1))
              \end{equation*}
              and $E$ is the universal elliptic curve over $M$; 
        \item We define $\mathrm{H}^{1}_{D}(\Sh_{\GL_2}, W(1))$ as
              \begin{equation*}
                    \mathrm{H}^{1}_{D}(\Sh_{\GL_2}, W(1)) := r_{H \rightarrow D}(\mathrm{H}_{H}(\Sh_{\GL_2}, W(1))), 
              \end{equation*}
              where $r_{H \rightarrow D}$ is the map 
              \begin{equation*}
                  r_{H \rightarrow D} \colon \mathrm{H}^{n + 1}_{H} (E^{n}, \RR(n + 1)) \rightarrow \mathrm{H}^{n + 1}_{D} (E^{n}, \RR(n + 1))
              \end{equation*}
              and $E$ is the universal elliptic curve over $\Sh_{\GL_2}$.
    \end{enumerate}
\end{definition}

\begin{remark}
    \begin{itemize}
        \item This definition is enough for our application to Beilinson's conjectures and can be easily generalized to all Shimura varieties of PEL type. However, we do not give a definition of ``DB-cohomology with coefficients'' for general coefficients over general analytic varieties. 
        \item It follows from \cite[Page 60]{Bei83} that the maps $r_{H \rightarrow D}$ in (3) and (4) are isomorphic, but the maps $r_{H \rightarrow D}$ in (1) and (2) may not be isomorphic. 
    \end{itemize}
\end{remark}

\begin{convention}
    Since we only work with real DB-cohomology in this paper, from now on, we will write $\mathrm{H}_{D}^{n}(X, \RR(p))$ for 
    $\mathrm{H}_{D}^{n}(X/\RR, \RR(p))$. 
\end{convention}

 \subsection{Deligne-Beilinson classes}
 \label{SS: Eis symbol and Hodge realization}
 In this subsection, we compute the realization of the motivic classes constructed in Subsection \ref{SS: motivic class} in Deligne{\textendash}Beilinson cohomology. 

\subsubsection{Eisenstein symbol in DB-cohomology}

We first compute the realization of Eisenstein symbols (see Subsubsection \ref{SSS: Eisenstein symbol}) in Deligne-Beilinson cohomology. We follow \cite[6.3]{Kings98} closely but we change the setting to use the Hermitian form $J_{2}^{\prime}$.  

\begin{convention}
    Recall that when we write $\GL_2(\RR)$, we always mean $\GU(J_{2}^{\prime})^{\prime}(\RR)$. 
\end{convention}

\begin{notation}
\label{Notation: character on GL2}
    \begin{itemize}
        \item We let $\lambda^{\prime}(n, c)$ be the character of the torus $T_{c} = Z_{2}(\RR)^{+}\U(1)(\RR)$ of $\GL_2(\RR)$
        \begin{equation*}
            \begin{pmatrix}
                t_1 & \\
                    & \bar{t}_{1} \\
            \end{pmatrix} \in  T_c   \longmapsto (x + iy)^{n}(x^2 + y^2)^{\frac{c - n}{2}}, 
        \end{equation*}
        where $Z_2$ is the center of $\GL_2$, $t_1 = x + iy$, $n$ is the weight and $c$ is the central character. 
        \item Let $T_{spl} = Z_{2}(\RR) \{ \begin{pmatrix}
                                                \cosh t & \sinh t \\
                                                \sinh t & \cosh t 
                                           \end{pmatrix} \}$,         where $t \in \RR$, $\cosh t = \frac{e^{t} + e^{-t}}{2}$ is the hyperbolic cosine function and $\sinh t = \frac{e^t - e^{-t}}{2}$ is the hyperbolic sine function. 
        It can viewed as a split torus of the group $\GL_2(\RR)$ for the following reason. 
        If we let $C = \begin{pmatrix}
                                                                                                                 i  & i \\
                                                                                                                 -1 & 1 \\
                                                                                                            \end{pmatrix}$, 
                  for $\begin{pmatrix}
                        \cosh t & \sinh t \\
                        \sinh t & \cosh t \\ 
                      \end{pmatrix} \in \GL_2(\RR)$, then we have 
                \begin{equation*}
                    C \begin{pmatrix}
                        \cosh t & \sinh t \\
                        \sinh t & \cosh t \\ 
                      \end{pmatrix} C^{-1} = \begin{pmatrix}
                                                e^{t} & \\ 
                                                      & e^{-t} \\
                                             \end{pmatrix}. 
                \end{equation*}
        \item We denote by $\lambda(n, c)$ the character of $T_{spl}$ 
                \begin{equation*}
                    \lambda(n, c) \colon z \begin{pmatrix}
                                        \cosh t & \sinh t \\
                                        \sinh t & \cosh t \\
                                     \end{pmatrix} \in T_{spl} \longmapsto (e^t)^{n} z^{c},
                \end{equation*}
            where $z \in Z_{2}(\RR)$, $n$ is the weight and $c$ is the central character. 
    \end{itemize}
    
\end{notation}

\begin{notation}
    \begin{itemize}
        \item Let $(X, Y)$ be a basis of the standard representation $V_{2}$ of $\GL_{2}$ such that each $\begin{pmatrix}
                                                                                                    a & b \\
                                                                                                    c & d \\ 
                                                                                                \end{pmatrix} \in \GL_2(\RR)$ acts by 
                    \begin{align*}
                        \begin{pmatrix}
                            a & b \\
                            c & d \\
                        \end{pmatrix} X = aX + bY, \\
                        \begin{pmatrix}
                            a & b \\
                            c & d \\
                        \end{pmatrix} Y = cX + dY. 
                    \end{align*}
        \item Let us view $\Sym^{n}V_{2, \CC}$ as the $n$-dimensional vector space of homogeneous polynomials of degree $n$ in the variables $X$ and $Y$ with coefficients in $\CC$. For any integer $0 \le j \le n$, let $b_{j}^{n}$ be the vector $b_{j}^{n} = (-1)^{j}X^{j}Y^{n-j}$, so it has weight $\lambda^{\prime}(2j - n, n)$. The family $(b_{j}^{n})_{0 \le j \le n}$ forms a basis of $\Sym^{n}V_{2, \CC}$. We denote by $(a_{j}^{n})_{0 \le j \le n}$ the dual basis of $(b_{j}^{n})_{0 \le j \le n}$. Hence, the vector $a_{j}^{n}$ has weight $\lambda^{\prime}(n - 2j, -n)$.
    \end{itemize}
\end{notation}

\begin{definition}
\label{def: function phi}
    \begin{enumerate}
        \item We define the function $z_{2}$, $z_{2}^{\prime}$, $w_{2}$ and $w_{2}^{\prime}$ on $\GL_{2}(\RR)$ as: 
            \begin{align*}
                & z_{2}(\begin{pmatrix}
                    a & b \\
                    c & d \\
                \end{pmatrix}) := \frac{b + d}{-b + d}i,  \\ & z_{2}^{\prime}(\begin{pmatrix}
                                                                            a & b \\
                                                                            c & d \\
                                                                        \end{pmatrix}) := \frac{a + c}{-a + c}i; \\
                & w_{2}(\begin{pmatrix}
                            a & b \\
                            c & d \\
                        \end{pmatrix}) :=-b + d, \\ &      w_{2}^{\prime}(\begin{pmatrix}
                                                                            a & b \\
                                                                            c & d \\
                                                                        \end{pmatrix}) := a - c. 
            \end{align*}
        \item For any integer $r$ such that $r \equiv n (\text{mod} \ 2)$, we define the function $\phi_{r}^{n}$ on $\GL_2(\RR)$ as 
                \begin{equation*}
                    \phi_{r}^{n} = (z_2 - z_2^{\prime}) w_{2}^{-\frac{n - r}{2}} {w_{2}^{\prime}}^{-\frac{n + r}{2}}. 
                \end{equation*}
    \end{enumerate}
\end{definition}

\begin{remark}
    Since we identify $\GL_2(\RR)$ with $\GU(J_2^{\prime})^{\prime}(\RR)$,                                                                                                       $\begin{pmatrix}
                                                             a & b \\
                                                             c & d 
                                                            \end{pmatrix} \in \GL_2(\RR)$ if and only if $a = \bar{d}$ and $c = \bar{b}$. Hence, it is impossible that $a = c$ or $b = d$, and the functions $z_2$ and $z_2^{\prime}$ are well-defined on $\GL_{2}(\RR)$. 
\end{remark}

\begin{notation}
    Let $B_2$ be the standard Borel of $\GL_2$. Since we identify $\GL_{2}(\RR)$ with $\GU(J_2^{\prime})^{\prime}(\RR)$, we have 
    $B_{2}(\RR) = T_{spl} N$ where $N = \left\{ \begin{pmatrix} 
                                             1 - \frac{1}{2i}u & \frac{1}{2i}u \\
                                             -\frac{1}{2i}u & 1 + \frac{1}{2i}u 
                                            \end{pmatrix} | u \in \RR \right \}$. 
\end{notation}

\begin{proposition}
\label{prop: phi equivariant}
    We have 
    \begin{equation*}
        \phi_{r}^{n} \in \Ind_{B_{2}(\RR)^{+}}^{\GL_2(\RR)^{+}} \lambda(n + 2, -n). 
    \end{equation*}
   Furthermore, the compact torus $T_c$ acts on $\phi_{r}^{n}$ by right translation and $\phi_{r}^{n}$ has weight $\lambda^{\prime}(-r,-n)$. 
\end{proposition}

\begin{proof}
    \begin{enumerate}
        \item We have the following identity:
            \begin{align*}
                &\begin{pmatrix}
                    \cosh t & \sinh t \\
                    \sinh t & \cosh t 
                \end{pmatrix} \begin{pmatrix}
                                a & b \\
                                c & d 
                             \end{pmatrix} = \begin{pmatrix}
                                                a \cosh t + c \sinh t & b \cosh t + d\sinh t \\
                                                a \sinh t + c \cosh t & b \sinh t + d \cosh t
                                            \end{pmatrix}, \\
                &\begin{pmatrix}
                     1 - \frac{1}{2i}u & \frac{1}{2i}u \\
                     -\frac{1}{2i}u & 1 + \frac{1}{2i}u 
                 \end{pmatrix} \begin{pmatrix}
                                    a & b \\
                                    c & d 
                               \end{pmatrix} = \begin{pmatrix}
                                                    a + (c - a) \frac{1}{2i}u &  b + (d - b) \frac{1}{2i}u\\
                                                    c + (c - a) \frac{1}{2i}u &  d + (d - b) \frac{1}{2i}u
                                                \end{pmatrix}. 
            \end{align*}
            Then by the identity 
            \begin{align*}
               \cosh t \pm \sinh t = e^{\pm t}
            \end{align*}
            and $\phi_{r}^{n} = (z_2 - z_2^{\prime}) w_{2}^{-\frac{n - r}{2}} {w_{2}^{\prime}}^{-\frac{n + r}{2}}$, we can see that 
            \begin{equation*}
                 \phi_{r}^{n} \in \Ind_{B_{2}(\RR)^{+}}^{\GL_2(\RR)^{+}} \lambda(n + 2, -n). 
            \end{equation*}
        \item The weight of $\phi_{r}^{n}$ under $T_c$ can be seen from the identity 
        \begin{equation*}
            \begin{pmatrix}
                a & b \\
                c & d \\
            \end{pmatrix} \begin{pmatrix}
                            (x + iy) & 0 \\
                            0 & (x - iy) \\
                          \end{pmatrix} = \begin{pmatrix}
                                            (x + iy)a & (x - iy) b \\
                                            (x + iy)c & (x - iy) d 
                                         \end{pmatrix}
        \end{equation*} and the definition of $\phi_{r}^{n}$. 
    \end{enumerate}
\end{proof}

Recall that we have the decomposition 
\begin{equation*}
    \liegl_{2, \CC} = \liek_{\GL_{2}, \CC} \oplus \liep^{+}_{\GL_{2}, \CC} \oplus \liep^{-}_{\GL_{2}, \CC},
\end{equation*} and we let $v^{+} = \begin{pmatrix}
                                                            0 & 1 \\
                                                            0 & 0 
                                                       \end{pmatrix} \in \liep^{+}_{\GL_2, \CC}$ and 
                                                                                             $v^{-} = \begin{pmatrix}
                                                                                                0 & 0 \\
                                                                                                1 & 0  
                                                                                              \end{pmatrix} \in \liep^{-}_{\GL_2, \CC}$. 

Let us define 
\begin{equation*}
    w_{n}^{+}, w_{n}^{-} \in \Hom_{K_{\GL_2}}(\liep^{+}_{\GL_{2}, \CC} \oplus \liep^{-}_{\GL_{2}, \CC}, \Sym^{n}V_{2, \CC} \otimes \Ind_{B_{2}(\RR)^{+}}^{\GL_2(\RR)^{+}} \lambda(n + 2, -n))
\end{equation*}
by 
\begin{align*}
    & \omega_{n}^{+}(v^{+}) = (2\pi i)^{n+1} b^{n}_{0} \otimes \phi^{n}_{-(n + 2)}, \\ 
    & \omega_{n}^{+}(v^{-}) = 0, \\
    & \omega_{n}^{-}(v^{+}) = 0, \\
    & \omega_{n}^{-}(v^{-}) = (2 \pi i)^{n+1}b^{n}_{n} \otimes \phi^{n}_{n + 2}. 
\end{align*}
\begin{remark}
\begin{itemize}
    \item  The weight of $b^{n}_{0}$ is $\lambda^{\prime}(-n, n)$, the weight of $\phi^{n}_{-(n + 2)}$ is $\lambda^{\prime}( n + 2,-n)$ and the weight $v^{+}$ is $\lambda^{\prime}(2, 0)$; the weight of $b^{n}_{n}$ is $\lambda^{\prime}(n, n)$, the weight of $\phi^{n}_{n + 2}$ is $\lambda^{\prime}(-(n + 2),-n)$ and the weight $v^{-}$ is $\lambda^{\prime}(-2, 0)$. So $\omega_{n}^{\pm}$ preserves $K_{\GL_2}$-weight, so it is well-defined. 
    \item The vector $\phi^{n}_{-(n + 2)}$ is a minimal $K_{\GL_2}$-type vector in the discrete series $D_{n}^{+}$. 
\end{itemize}
\end{remark}

Let us state some basic properties. 

\begin{lemma}
We have the following identities of the action of $v^{\pm}$:
\begin{enumerate}
    \item 
        \begin{align*}
            & v^{+}(w_2) = -w_2^{\prime}, \  v^{+}(w_2^{\prime}) = 0, \\
            & v^{+}(z_2) = \frac{2(ad - bc)}{(-b + d)^{2}}i, \ v^{+}(z_2^{\prime}) = 0,\\
            & v^{-}(w_2) = 0, \ v^{-}(w_2^{\prime}) = -w_2, \\
            & v^{-}(z_2) = 0, \ v^{-}(z_2^{\prime}) = \frac{2(bc-ad)}{(-a+c)^{2}}i; 
        \end{align*} 
        \item 
        \begin{align*}
            & v^{+}(\phi_r^n) = \frac{n - r + 2}{2}\phi^{n}_{r - 2}, \ v^{-}(\phi_r^n) = \frac{n + 2 + r}{2}\phi^{n}_{r + 2}, 
        \end{align*}
        \item 
        \begin{equation*}
            v^{+} b^{n}_{r} = -(n - r)b^n_{r + 1},\  v^{-} b^{n}_{r} = -rb^n_{r - 1}. 
        \end{equation*}
\end{enumerate}
    
\end{lemma}

\begin{proof}
    If $f$ is a linear function on $\GL_{2}(\RR)$, then
        \begin{align*}
            (v^{+}f)(\begin{pmatrix}
                        a & b \\
                        c & d
                     \end{pmatrix}) &= \lim_{t \rightarrow 0} \frac{f(\begin{pmatrix}
                                                                        a & b \\
                                                                        c & d
                                                                     \end{pmatrix} \begin{pmatrix}
                                                                                        1 & t \\
                                                                                        0 & 1
                                                                                    \end{pmatrix}) - f(\begin{pmatrix}
                                                                                                            a & b \\
                                                                                                            c & d
                                                                                                         \end{pmatrix}) }{t} \\
                                   &= \lim_{t \rightarrow 0} \frac{f(\begin{pmatrix}
                                                                        a & b \\
                                                                        c & d
                                                                     \end{pmatrix} \begin{pmatrix}
                                                                                        0 & t \\
                                                                                        0 & 0
                                                                                    \end{pmatrix})}{t} = f(\begin{pmatrix}
                                                                                                                 a & b \\
                                                                                                                 c & d
                                                                                                            \end{pmatrix} \begin{pmatrix}
                                                                                                                                0 & 1 \\
                                                                                                                                0 & 0
                                                                                                                            \end{pmatrix}). 
        \end{align*}
        Similarly, we have 
        \begin{equation*}
             (v^{+}f)(\begin{pmatrix}
                        a & b \\
                        c & d
                     \end{pmatrix}) = f(\begin{pmatrix}
                                            a & b \\
                                            c & d
                                        \end{pmatrix} \begin{pmatrix}
                                                            0 & 0 \\
                                                            1 & 0
                                                       \end{pmatrix}). 
        \end{equation*}
        \begin{enumerate}
        \item It follows from the fact that $w_2$ and $w_2^{\prime}$ are linear functions that 
              \begin{align*}
                  & (v^{+}w_2)(\begin{pmatrix}
                            a & b \\
                            c & d
                          \end{pmatrix}) = -a + c = w_2^{\prime}(\begin{pmatrix}
                                                                    a & b \\
                                                                    c & d
                                                                  \end{pmatrix}), \\
                  & (v^{+}w_2^{\prime})(\begin{pmatrix}
                            a & b \\
                            c & d
                          \end{pmatrix}) = 0.                                                  
              \end{align*}
              The proof for $v^{-}(w_2)$ and $v^{-}(w_2^{\prime})$ is similar. \\
              It can be seen from the quotient rule that
              \begin{align*}
                  & (v^{+}z_2) = \frac{(v^{+}(b + d))(-b + d) - (v^{+}(-b + d))(b + d)}{(-b + d)^{2}}i = \frac{2(ad - bc)}{(-b + d)^{2}}i, \\
                  & (v^{+}z_2^{\prime}) = \frac{(v^{+}(a + c))(-a + c) - (v^{+}(-a + c))(a + c)}{(-a + c)^{2}}i = 0. 
              \end{align*}
              The proof for $v^{-}(z_2)$ and $v^{-}(z_2^{\prime})$ is similar. 
        \item It follows from explicit computation that
              \begin{align*}
                  z_2 - z_2^{\prime} &= \frac{2(bc - ad)}{(-b + d)(-a + c)}i \\
                                     &= (v^{+}(z_2 - z_2^{\prime}))\frac{w_2}{w_2^{\prime}}. 
              \end{align*}
              From this, we get 
              \begin{equation*}
                  v^{+}(z_2 - z_2^{\prime}) = (z_2 - z_2^{\prime}) \frac{w_2^{\prime}}{w_2}. 
              \end{equation*}
              Hence, 
              \begin{align*}
                  v^{+}\phi^{n}_{r} & =  (z_2 - z_2^{\prime})w_{2}^{-(\frac{n - r}{2} + 1)}{w_{2}^{\prime}}^{-(\frac{n + r}{2} - 1)}\\   
                  &+ (\frac{n - r}{2}) (z_2 - z_2^{\prime})w_{2}^{-(\frac{n - r}{2} + 1)}{w_{2}^{\prime}}^{-(\frac{n + r}{2} - 1)}\\
                  & = \frac{n -r + 2}{2}(z_2 - z_2^{\prime})w_{2}^{-(\frac{n - (r-2)}{2})}{w_{2}^{\prime}}^{-(\frac{n + (r-2)}{2})} \\
                  & = \frac{n -r + 2}{2} \phi^{n}_{r - 2}. 
              \end{align*}
              The computation of $v^{-}\phi^{n}_{r}$ is similar. 
        \item Since $X$ and $Y$ are linear functions, we have $v^{+}(X) = 0$ and $v^{+}(Y) = X$. 
              Hence, by the product rule, 
              \begin{equation*}
                  v^{+}(b^{n}_{r}) = v^{+}((-1)^{r}X^{r}Y^{n-r}) = (-1)^{j}(n - r) X^{r + 1} Y^{n - r - 1} = -(n - r) b^{n}_{r + 1}. 
              \end{equation*}
              Similar computations hold for $v^{-}(b^{n}_{j})$. 
        \end{enumerate}
\end{proof}

\begin{lemma}
        We have the following identities with complex conjuagtion: 
        \begin{align*}
           &  \overline{\phi^{n}_{r}} =  (-1)\phi^{n}_{-r}, \ \overline{b^n_r} = b^b_{n - r}, \overline{v^{+}} = v^{-}, \\
           &  \overline{\omega^{\pm}} =  (-1)^{n}\omega_{n}^{\mp},  
        \end{align*}
        where $\overline{(\cdot)}$ denotes the complex conjugation. 
\end{lemma}

\begin{proof}
        Recall that the Shimura data of $\Sh_{\GL_2}$ is defined on $\RR$-points as 
            \begin{equation*}
             \begin{tikzcd}[row sep = 0]
                 h \colon \mathbb{S}(\RR) \ar[r] & \GL_2(\RR) \\
                    z = x + iy \ar[r, mapsto] & \begin{pmatrix}
                                                    z & 0 \\
                                                    0 & \bar{z}
                                                \end{pmatrix}
             \end{tikzcd},
            \end{equation*}
            and the reflex field of the Shimura variety $\Sh_{\GL_2}$ is $\QQ$ and the complex conjugation on it is given by conjugation by 
                           $N = \begin{pmatrix}
                                    0 & -1 \\
                                    -1 & 0 
                                \end{pmatrix}$. From this we can see that 
        \begin{equation*}
            \overline{v^{+}} = \mathrm{Ad}_{N}(v^{+}) = v^{-}. 
        \end{equation*}
        It follows from 
        \begin{equation*}
            \mathrm{Ad}_{N}( \begin{pmatrix}
                                a & b \\
                                c & d
                             \end{pmatrix}) = \begin{pmatrix}
                                                 d & c\\
                                                 b & a 
                                              \end{pmatrix},
        \end{equation*}
        that $\overline{a} = d$ and $\overline{b} = c$.  So we have 
        \begin{align*}
            & \overline{w_2} = -\overline{b} + \overline{d} = a - c = w_2^{\prime}, \\
            & \overline{z_2} = \frac{\overline{b} + \overline{d}}{-\overline{b} + \overline{d}}(-i) = \frac{a + c}{c - a}i = z_2^{\prime}. 
        \end{align*}
        We can see from this that 
        \begin{equation*}
            \overline{\phi^{n}_{r}} = (-1) (z_2 - z_2^{\prime}) w_{2}^{-(\frac{n - (-r)}{2})}{w_{2}^{\prime}}^{-(\frac{n + (-r)}{2})} = (-1) \phi^{n}_{-r}. 
        \end{equation*}
        The complex conjugate on $X, Y$ is given by $\overline{(X, Y)} = (X, Y)N = (-Y, -X)$. Hence, we have 
        \begin{equation*}
            \overline{b^{n}_{r}} = \overline{(-1)^{r}X^{r}Y^{n - r}} = (-1)^{n+r}Y^{r}X^{n - r} = b^{n}_{n - r}. 
        \end{equation*}
        Finally, we can deduce that 
        \begin{align*}
            \overline{\omega_{n}^{-}}(v^{-}) &= \overline{(2 \pi i)^{n + 1}} \overline{b^n_n} \otimes \overline{\phi^{n}_{n + 2}} \\
                                        & = (-1)^{n + 1} (2\pi{i})^{n + 1}  b^{n}_{0} \otimes (-1) \phi^{n}_{-(n + 2)} \\
                                        & = (-1)^{n}(2\pi{i})^{n + 1} b^{n}_{0} \otimes \phi^{n}_{-(n + 2)} \\
                                        & = (-1)^{n}\omega^{+}_{n}(v^{+}). 
        \end{align*}
        Therefore, $\overline{\omega^{\pm}_{n}} = (-1)^{n}\omega_n^{\mp}$. 
\end{proof}

Given $\phi_{f} \in \mathcal{B}_{n}$, we let 
\begin{equation*}
   Eis_{B}^{n}(\phi_f) = \sum_{\gamma \in B_{2}(\QQ) \backslash \GL_2(\QQ)} \gamma^{*}(\omega_n^{+} \otimes \phi_f),
\end{equation*}
which is absolutely convergent \cite[(6.3.1)]{Kings98} and defines an element of 
\begin{equation*}
    \Hom_{K_{\GL_2}}(\liep_{\GL_{2}, \CC}^{+}\oplus \liep_{\GL_{2}, \CC}^{-}, \Sym^{n}V_{2, \CC} \otimes C^{\infty}(\GL_2(\QQ) \backslash \GL_2(\AAA)). 
\end{equation*}
Consider
\begin{equation*}
    \theta_{n} = \frac{(2 \pi i)^{n + 1}}{2(n + 1)}\sum\limits_{j = 0}^{n}  b_{n - j}^{n} \otimes \phi_{n - 2j}^{n} \in \Sym^{n}V_{2, \CC} \otimes \Ind_{B_{2}(\RR)^{+}}^{\GL_2(\RR)^{+}} \lambda(n + 2, -n),
\end{equation*}
Similar to before, for $\phi_{f} \in \mathcal{B}_{n}$ we define 
\begin{equation}
\label{eqn:Eis_symbol_Hodge}
    Eis_{H}^{n}(\phi_f) = \sum_{\gamma \in B_{2}(\QQ) \backslash \GL_2(\QQ)} \gamma^{*}(\theta_{n} \otimes \phi_f).
\end{equation}
This infinite series is absolutely convergent for $n \ge 1$. In this paper, we only treat the case $n \ge 1$. 

\begin{lemma}\textnormal{\cite[(6.3.5)]{Kings98}}
\label{Lemma: Delgine representative of Eisenstein symbol}
    We have the following identity: 
    \begin{equation*}
        \mathrm{d}(Eis_{H}^{n}(\phi_f)) = \pi_{n} (Eis_{B}^{n}(\phi_f)). 
    \end{equation*}
    In other words, for any $n \ge 1$ and $\phi_{f} \in \mathcal{B}_{n}$, the class $Eis_{D}^{n}(\phi_f)$ is represented by 
    \begin{equation*}
        (Eis_{H}^{n}(\phi_f), Eis_{B}^{n}(\phi_f)) \in \mathrm{H}_{D}^{1}(\Sh_{\GL_2}, \Sym^{n} V_{2}(1)), 
    \end{equation*}
    which is a direct factor of $\mathrm{H}_{D}^{n + 1}(E^{n}, \RR(n + 1))$. Here $E \rightarrow \Sh_{\GL_2}$ is the universal elliptic curve over the Shimura variety $\Sh_{\GL_2}$.
\end{lemma}

\begin{proof}
    We have 
    \begin{align*}
        d\theta_n(v^{+}) = v^{+}\theta_n = \frac{(2\pi{i})^{n+1}}{2(n + 1)} ( & b^{n}_{n} \otimes \phi^{n}_{n - 2} \\
                                                                       & + 2 b^{n}_{n - 1} \otimes \phi^{n}_{n - 4} - b^{n}_{n} \otimes \phi^{n}_{n - 2} \\
                                                                       & + 3 b^{n}_{n - 2} \otimes \phi^{n}_{n - 6} - 2 b^{n}_{n-1} \otimes \phi^{n}_{n - 4} \\
                                                                       & + \cdots \\
                                                                       & + (n + 1) b^{n}_{0} \otimes \phi^{n}_{-(n + 2)} - n b^{n}_{1} \otimes \phi^{n}_{-n}) \\
                                         & = \frac{(2\pi{i})^{n+1}}{2 } b^{n}_{0} \otimes \phi^{n}_{-(n + 2)}. 
    \end{align*}
    By a similar computation, we have $d\theta_{n}(v^{-}) = \frac{(2\pi{i})^{n + 1}}{2} b^{n}_{n} \otimes \phi^{n}_{n + 2}$. 
    It follows from the definitions that 
    \begin{equation*}
        \pi_n(\omega^{+}_{n}) = \frac{\omega_{n}^{+} + (-1)^{n}\overline{\omega_{n}^{+}}}{2} = \frac{\omega_{n}^{+} + {\omega_{n}^{-}}}{2}. 
    \end{equation*}
    Hence, 
    \begin{align*}
        & \pi_n(\omega^{+}_{n})(v^{+}) = \frac{\omega_{n}^{+} + {\omega_{n}^{-}}}{2}(v^{+}) = \frac{(2\pi{i})^{n+1}}{2 } b^{n}_{0} \otimes \phi^{n}_{-(n + 2)} = d\theta_n(v^{+}),  \\
        & \pi_n(\omega^{+}_{n})(v^{-}) = \frac{\omega_{n}^{+} + {\omega_{n}^{-}}}{2}(v^{-}) = \frac{(2\pi{i})^{n+1}}{2 } b^{n}_{n} \otimes \phi^{n}_{n + 2} = d\theta_{n}(v^{-}). 
    \end{align*}
    In summary, we have shown $\pi_n(\omega_n^{+}) = d\theta_n$ so $\mathrm{d}(Eis_{H}^{n}(\phi_f)) = \pi_{n} (Eis_{B}^{n}(\phi_f))$. 
\end{proof}

\begin{remark}
    This proof is essentially the same as in \cite[\S 6.3]{Kings98}, but we chose to identify $\GL_{2}(\RR)$ with $\GU(J_2^{\prime})^{\prime}$, so we reproduce the proof here. 
\end{remark}

\subsubsection{Motivic classes in DB-cohomology}
Now we give an expression of the realization of the motivic classes constructed in Construction \ref{Construction: motivic classes} in Deligne{\textendash}Beilinson cohomology using functoriality of motivic cohomology, DB-cohomology and the Beilinson regulator. 

\begin{notation}
    We use the following notations: 
    \begin{itemize}
        \item $* := a + 5b - 3 (r + s) + 1$, $** :=  a + 5b - 3 (r + s) + 3$, $\Box := a + 5b - 3 (r + s)$, $\circ := a + 3b -r - s + 1$ and $\triangle := a + 3b -r - s + 2$; 
        \item for $\diamond \in \{M, H, D\}$, $i_{\diamond}$ stands for the natural inclusion induced by $W \hookrightarrow \iota^{*}V$.
    \end{itemize}
\end{notation}

\begin{proposition}
    We have the following commutative diagram: 
    \[
        \begin{tikzcd}[column sep = 0.8em]
            \mathcal{B}_n \ar[r,"Eis^n_M"] \ar[d, "r_{H}"] & \mathrm{H}^{1}_{M}(\Sh_{\GL_2}, \mathrm{Sym}^{n}V_2(1)) \ar[r, "p^{*}_{M}"] \ar[d, "r_{H}"] & \mathrm{H}^{1}_{M}(M, W(1)) \ar[r, hook, "i_{M}"] \ar[d, "r_{H}"] & \mathrm{H}^{1}_{M}(M, \iota^{*}V(1)) \ar[r, "\iota_{M, *}"] \ar[d, "r_{H}"] & \mathrm{H}^{3}_{M}(S,V(2)) \ar[d, "r_{H}"] \\
            \mathcal{B}_{n, \RR} \ar[r, "Eis^n_H"] \ar[d, "r_{H \rightarrow D}"] & \mathrm{H}^{1}_{H}(\Sh_{\GL_2}, \mathrm{Sym}^{n}V_2(1)) \ar[r,"p^{*}_{H}"] \ar[d, "r_{H \rightarrow D}"] & \mathrm{H}^{1}_{H}(M, W(1)) \ar[r, hook, "i_{H}"] \ar[d, "r_{H \rightarrow D}"] & \mathrm{H}^{1}_{H}(M, \iota^{*}V(1)) \ar[r, "\iota_{H, *}"]  \ar[d, "r_{H \rightarrow D}"] & \mathrm{H}^{3}_{H}(S,V(2)) \ar[d, "r_{H \rightarrow D}"] \\
            \mathcal{B}_{n, \RR} \ar[r, "Eis^n_D"] \ar[d, "\cong"] & \mathrm{H}^{1}_{D}(\Sh_{\GL_2}, \mathrm{Sym}^{n}V_2(1)) \ar[r, "p^{*}_{D}"] \ar[d, hook] & \mathrm{H}^{1}_{D}(M, W(1)) \ar[r, hook,"i_D"] \ar[d, hook] & \mathrm{H}^{1}_{D}(M, \iota^{*}V(1)) \ar[r, "\iota_{D, *}"] \ar[d, hook] & \mathrm{H}^{3}_{D}(S,V(2)) \ar[d, hook] \\
            \mathcal{B}_{n, \RR} \ar[r, "Eis^n_D"]  & \mathrm{H}^{n + 1}_{D}(E^{n}, \RR(n + 1)) \ar[r, "p^{*}_{D}"] & \mathrm{H}^{n + 1}_{D}(E^{n}, \RR(n + 1)) \ar[r, hook, "i_D"] & \mathrm{H}_{D}^{*}(A^{\Box}, \RR(\circ)) \ar[r, "\iota_{D, *}"] & \mathrm{H}_{D}^{**}(A^{\Box}, \RR(\triangle)) 
        \end{tikzcd}, 
    \]
    where in the last row, $E^{n}$ in the first $\mathrm{H}^{n + 1}_{D}(E^{n}, \RR(n + 1))$ is the $n$-th fiber product of the universal elliptic curve over $\Sh_{\GL_2}$, $E^{n}$ in the second $\mathrm{H}^{n + 1}_{D}(E^{n}, \RR(n + 1))$ is the $n$-th fiber product of the universal elliptic curve over $M$, $A^{\Box}$ in $\mathrm{H}_{D}^{*}(A^{\Box}, \RR(\circ))$ is the $\Box$-th fiber product of the pullback of the universal abelian $3$-fold from $S$ to $M$ through $\iota$, and $A^{\Box}$ in $\mathrm{H}_{D}^{**}(A^{\Box}, \RR(\triangle))$ is the $\Box$-th fiber product of the universal abelian $3$-fold over $S$. 
\end{proposition}

\begin{proof}
    \par First, the commutativity of the diagram comes from the functoriality of $r_{H}$ and $r_{H \rightarrow D}$. 
    \par Second, the decomposition of the Gysin morphism 
    \begin{equation*}
        \mathrm{H}^{1}_{\mathrm{\diamond}}(M, W(1)) \rightarrow \mathrm{H}^{3}_{\mathrm{\diamond}}(S, V(2))
    \end{equation*}
    into $\iota_{\diamond, *} \circ i_{\diamond}$ is natural from the construction of Gysin morphism (Proposition \ref{Prop: Gysin for motivic cohomology}). 
    \par Finally, the embebding of the third row into the fourth row is from ``Liebermann’s trick'' as explained in Proposition \ref{Prop:motiv_coh_abs_relative}. 
\end{proof}

\begin{remark}
\label{remark: Eis_D class}
    From Construction \ref{Construction: motivic classes}, the Eisenstein classes $c \in \mathrm{H}^{3}_{M}(S, V(2))$ are 
    \begin{equation*}
        c = \iota_{M, *} \circ i_{M} \circ p_{M}^{*} \circ Eis_{M}^{n} (\phi_f), 
    \end{equation*}
    for $\phi_{f} \in \mathcal{B}_{n}$. 
    By commutativity of the above diagram, their image under the Deligne regulator $r_{D}: = r_{H \rightarrow D} \circ r_{H}$ 
    is 
    \begin{align*}
        r_{D}(c) &= r_{D} (\iota_{M, *} \circ i_{M} \circ p_{M}^{*} \circ Eis_{M}^{n} (\phi_f)) \\
                 &= \iota_{D, *} \circ i_{D} \circ p_{D}^{*} \circ Eis_{D}^{n} (r_{D}(\phi_f)) \\
                &= \iota_{D, *} \circ i_{D} \circ p_{D}^{*}(Eis_{H}^{n}(\phi_f), Eis_{B}^{n}(\phi_f)).
    \end{align*}
    By ``Liebermann’s trick'', Proposition \ref{Prop: DB-coh_temperd currents}, Proposition \ref{Prop: DB-cohomology-si forms to tempered currents} and the fact that $Eis_{H}^{n}(\phi_f)$ and $Eis_{B}^{n}(\phi_f)$ are slowly increasing as explained in \cite[page 120]{Kings98}, we can view 
    \begin{equation*}
         i_{D} \circ p_{D}^{*} \circ (Eis_{H}^{n}(\phi_f), Eis_{B}^{n}(\phi_f)). 
    \end{equation*}    
    as a pair of tempered currents 
    \begin{equation*}
        (T_{p^{*}Eis_{H}^{n}(\phi_f)}, T_{p^{*}Eis_{B}^{n}(\phi_f)}) \in \mathrm{H}^{1}_{D} (M, \iota^{*}V(1)) \subset \mathrm{H}^{a + 5b - 3 (r + s) + 1}_{D} (A^{a + 5b - 3 (r + s)} /M, \RR(a + 3b - r - s + 1)), 
    \end{equation*}
    where $A^{a + 5b - 3 (r + s)} /M$ is the $a + 5b - 3(r + s)$-th fiber product of the pullback of the universal abelian $3$-fold from $S$ to $M$ through $\iota$. 
    Hence, by the explicit description of the Gysin morphism for DB-cohomology (cf. equation (\ref{eq: Gysin for DB-cohomology})), we have 
    \begin{align*}
        r_{D}(c) = (\iota_{*}T_{p^{*}Eis_{H}^{n}(\phi_f)}, \iota_{*} T_{p^{*}Eis_{B}^{n}(\phi_f)}) & \in \mathrm{H}^{3}_{D} (S, V(2)) \\ & \subset \mathrm{H}^{a + 5b - 3 (r + s) + 3}_{D} ( A^{a + 5b - 3 (r + s)} /S, \RR(a + 3b -r - s + 2)), 
    \end{align*}
     where $A^{a + 5b - 3 (r + s)} /S$ is the $a + 5b - 3(r + s)$-th fiber product of the universal abelian $3$-fold over $S$. 
\end{remark}



\subsection{The pairing}
\label{SS the pairing}
In this subsection, we give an explicit formula for the Poincar\'{e} duality pairing with the tool of tempered currents. 

\subsubsection{From Poincar\'{e} duality pairing to an integral of differential forms} 
We first express the Poincar\'{e} duality pairing in terms of an integration of differential forms on Shimura varieties. 

\begin{lemma}
\label{Lemma: tempered current associated to Eis classes}
    Let $\widetilde{A}^{a + 5b - 3(r + s)}$ be a smooth toroidal compactification of $A^{a + 5b - 3(r + s)}$ such that the complement
    $\widetilde{A}^{a + 5b - 3(r + s)} \backslash A^{a + 5b - 3(r + s)}$ is a simple normal crossings divisor. Then there exists a closed tempered current 
    \begin{equation*}
        \rho \in \mathcal{D}_{si, \RR(a + 3b - r - s + 1)}^{a + 5b - 3(r + s) + 2}(\widetilde{A}^{a + 5b - 3(r + s)})\
    \end{equation*}
    such that: 
    \begin{enumerate}
        \item the cohomology class $[\rho]$ of $\rho$ belongs to $\mathrm{H}^{2}_{B, !}(S, V(2))^{-}(-1)$; 
        \item the class $[\rho]$ maps to $\mathcal{E}is_{H}^{n}(\phi_f) = r_{H}(\mathcal{E}is_{M}^{n}(\phi_f))$ 
            through the third map in the exact sequence (\ref{exact seq: Ext^1}):
            \begin{equation*}
                \mathrm{H}^{2}_{B, !}(S, V(2))_{\RR}^{-}(-1) \rightarrow \Ext^{1}_{\mathrm{MHS}_{\RR}^{+}}(\RR(0), \mathrm{H}^{2}_{B, !}(S, V(2))_{\RR}); 
            \end{equation*}
        \item the pair of tempered currents $(\iota_{*}T_{p^{*}Eis_{H}^{n}(\phi_f)}, \iota_{*} T_{p^{*}Eis_{B}^{n}(\phi_f)})$ and $(\rho, 0)$ represent the same cohomology class in $\mathrm{H}^{3}_{D} (S, V(2))$. 
    \end{enumerate}
\end{lemma}

\begin{proof}
   This is a direct consequence of \cite[\S 5.7.]{Bei83}.
\end{proof}

\begin{corollary}
\label{corollary: Omega pairing in terms of omega paring with rho}
    We have the following identity: 
    \begin{equation*}
        \langle \Omega, \tilde{v}_{K} \rangle_{B} = \langle \omega_{\Psi}, [\rho] \rangle_{B} \otimes \mathbf{1}, 
    \end{equation*}
    where $\mathbf{1}$ be unit of $E(\pi_f) \otimes_{\QQ} \CC$. 
\end{corollary}

\begin{proof}
    It is direct from Lemma \ref{Lemma: tempered current associated to Eis classes} (2) and Remark \ref{remark: pairing} (5). 
\end{proof}

We first take the dual of the map of representations
\begin{equation*}
    \Sym^{n}V_{2} \hookrightarrow \iota^{*} V,
\end{equation*}
 and get the map $\iota^{*} D \rightarrow \Sym^{n}V_{2}^{\vee} $. 
Composing the above map with the natural pairing $\Sym^{n}V_2^{\vee} \otimes \Sym^{n} V_2 \rightarrow \QQ(0)$ defined by 
\begin{equation*}
    a^{n}_{j} \otimes b^{n}_{k} \mapsto \begin{cases}
                                            0 & \text{if} \ j \neq k, \\
                                            \binom{n}{j} & \text{if} \ j = k,   \\
                                        \end{cases}
\end{equation*}
we have a natural pairing 
\begin{equation*}
    \langle \cdot, \cdot \rangle \colon \iota^{*}D \otimes \Sym^{n} V_2 \rightarrow \QQ(0). 
\end{equation*} 

\begin{convention}
We normalize the embedding 
\begin{equation*}
\mathrm{H}^{2}_{B, !}(S, V(2)) \hookrightarrow \mathrm{H}^{a + 5b - 3(r + s) + 2}_{B, !}(A^{a + 5b - 3(r + s)}, \RR(a + 3b -r -s + 2))
\end{equation*}
and 
\begin{equation*}
\mathrm{H}^{2}_{B, !}(S, D) \hookrightarrow \mathrm{H}^{a + 5b - 3(r + s) + 2}_{B, !}(A^{a + b - 3(r + s)}, \RR(2b - 2r - 2s))
\end{equation*}
such that we have the following commutative diagram: 
\[
    \begin{tikzcd}
        \mathrm{H}^{2}_{B, !}(S, V(2)) \otimes \mathrm{H}^{2}_{B, !}(S, D)  \ar[d, hook] \ar[r,"{\langle \cdot , \cdot \rangle_B}"] & \QQ(0) \ar[d, "="] \\
        \mathrm{H}^{*}_{B, !}(A^{\Box}, \RR(a + 3b -r - s + 2)) \otimes \mathrm{H}^{*}_{B, !}(A^{\Box}, \RR(2b - 2r -2s )) \ar[r,"{\langle \cdot, \cdot \rangle_B}"] & \QQ(0)
    \end{tikzcd}, 
\]
where $* = a + 5b - 3(r + s) + 2$ and $\Box = a + 5b - 3(r + s)$. 
\end{convention}

Let $\Psi = \Psi_{f} \otimes \Psi_{\infty}$ be a factorizable cusp form in the irreducible cuspidal automorphic representation $\tilde{\pi} = \tilde{\pi}_f \otimes \tilde{\pi}_{\infty}$ of $G(\AAA)$ with central character\footnote{Here $\tilde{\pi}_{\infty}$ is a $(\lieg_{\CC}, K_G)$-module, but $A_{G} \subset K_{G}$, so the word ``central character of $\tilde{\pi}$ makes sense.}
\[
    \begin{tikzcd}[row sep = 0]
       \omega_{\pi} \colon Z_{G}(\QQ) \backslash Z_{G}(\AAA) \ar[r] & \CC^{\times} \\
                      z \ar[r, mapsto] & {|z|^{n} \nu(z)^{-1}}, 
    \end{tikzcd}
\]
where $\nu$ is a finite order Hecke character of sign $(-1)^{n}$. Let 
\begin{equation*}
    \omega_{\Psi} := \omega \otimes \Psi_{f}
\end{equation*}
be a differential form associated to $\Psi$, where $\omega$ is the Lie algebra cohomology class constructed in Proposition \ref{Prop: explicit construnction of the differential form}. 

Recall that we let $n = a + b + r + s$ and we use $\Sh_{G}(L)$ (resp., $\Sh_{H}(K)$) to denote $\Sh_{G}(L)_{\CC}^{an}$ (resp., $\Sh_{H}(K)_{\CC}^{an}$). 
\begin{proposition}
\label{Prop: pairing to integration of diff form}
 Let $\phi_f \in I_{n}(\nu) \subset \mathcal{B}_{n, \overline{\QQ}}$ defined in Definition \ref{Def: In(v)} and let $\rho$ be the closed tempered current associated to the class $\mathcal{E}is_{D}^{n}(\phi_f) = r_{D}(\mathcal{E}is_{M}^{n}(\phi_f))$ by Lemma \ref{Lemma: tempered current associated to Eis classes}. We have 
 \begin{equation*}
      \langle \omega_{\Psi}, [\rho] \rangle_{B} = \frac{1}{(2\pi i)^{n + 1}} \int_{\Sh_H(K)} p^{*} Eis_{H}(\phi_f) \iota^{*} \omega_{\Psi},
 \end{equation*}
 where $K$ is the level of $\phi_f$. 
    
\end{proposition}

\begin{proof}
    For the pair of tempered currents $(\iota_{*}T_{p^{*}Eis_{H}^{n}(\phi_f)}, \iota_{*} T_{p^{*}Eis_{B}^{n}(\phi_f)})$ and $(\rho, 0)$ that represent the same cohomology class in $\mathrm{H}^{3}_{D} (\Sh_{G}, V(2))$, there exists 
    \begin{equation*}
        (S^{\prime}, T^{\prime}) \in \mathcal{D}_{si, \RR(a + 3b -r - s + 1)}^{a + 5b - 3(r + s) + 1}(\widetilde{A}^{a + 5b - 3(r + s)}) \oplus F^{a + 3b -r - s + 2} \mathcal{D}_{si}^{a + 5b - 3(r + s) + 2}(\widetilde{A}^{a + 5b - 3(r + s)})
    \end{equation*}
    such that 
    \begin{equation*}
        \rho = \iota_{*}T_{p^{*}Eis_{H}^{n}(\phi_f)} + dS^{\prime} + \pi_{a + 3b -r -s + 1} (T^{\prime}). 
    \end{equation*}
    It follows from the fact that $\omega_{\Psi}$ is a diffferential form on $\Sh_{G}$ that 
    \begin{equation*}
        \langle \omega_{\Psi}, [\rho] \rangle_{B} = \langle \omega_{\Psi}, [\rho] \rangle_{B}^{\Sh_{G}}. 
    \end{equation*}
    Then we can see that 
    \begin{align*}
        \langle \omega_{\Psi}, [\rho] \rangle_{B}         &= \rho(\omega_{\Psi}) \\
                                                          &= \iota_{*}T_{p^{*}Eis_{H}^{n}(\phi_f)} (\omega_{\Psi}) + dS^{\prime} (\omega_{\Psi}) + \pi_{a + 2r -s + 1} (T^{\prime}) (\omega_{\Psi}) \\
                                                          &= \iota_{*}T_{p^{*}Eis_{H}^{n}(\phi_f)} (\omega_{\Psi}) + \pi_{a + 2r -s + 1} (T^{\prime}) (\omega_{\Psi}) \\
                                                          &= \iota_{*}T_{p^{*}Eis_{H}^{n}(\phi_f)} (\omega_{\Psi}) \\
                                                          &= T_{p^{*}Eis_{H}^{n}(\phi_f)} (\iota^{*}\omega_{\Psi}) \\
                                                          &= \frac{1}{(2\pi i)^{n + 1}} \int_{\Sh_H(K)} p^{*} Eis_{H}(\phi_f) \iota^{*} \omega_{\Psi}. 
    \end{align*}   
    The first equality holds by the compatibility between Poincar\'{e} duality pairings and integrations (see \cite[Prop.1.4.4.(c)]{Harris90}). 
    The third equality follows from the fact that $\omega_{\Psi}$ is closed. 
    The fourth equality follows from the fact that the Hodge type of $\omega_{\Psi}$ is $(a + r + 1, b + s + 1)$ and the fact that $T^{\prime} \in F^{a + 3b -r - s + 2} \mathcal{D}_{si}^{a + 5b - 3(r + s) + 2}(\widetilde{A}^{a + 5b - 3(r + s)})$. 
    The fifth  equality follows from the definition of the pushforward of tempered currents (see equation (\ref{eq: pushforward of tempered currents})). The last equality comes from the definition of tempered currents. This completes the proof of the proposition. 
\end{proof}

\subsubsection{From the integral of differential forms to an adelic integral}
We now translate the above integral into an adelic integral. In order to do so, we need to make the following choices of measures. 

\begin{convention}
\label{convention: measure}
    \begin{enumerate}
        \item We normalize the Haar measure for a reductive or unipotent group $G^{\prime}$ over $\Zp$ such that $\mathrm{Vol}(G^{\prime}(\Zp)) = 1$. 
        \item At the archimedean place, we use the Iwasawa decomposition to determine the measure. More precisely, we define the measure on the unipotent radical of the Borels\footnote{The algebraic group $G$ is quasi-split over $\QQ$.} of $G(\RR)$  and $H(\RR)$ so that the integral points have covolume $1$. We choose the Haar measure on $K_{G}$ (resp., $K_{H}$) such that the volume of $K_{G} / Z_{G}(\RR)$ (resp., $K_{H} / Z_{H}(\RR)$) is $1$. On the maximal split torus $\RR^{\times, n}$, we choose the multiplicative Haar measure $\frac{dx_1dx_{2} \cdots dx_{n}}{|x_1 x_2 \cdots x_{n}|}$. 
        \item The measures on $G(\AAA)$ and $H(\AAA)$ are the product of the above nonarchimedean measures and archimedean measures. 
        \item For quotients of groups, we use the unique measure compatible with these choices. 
        \item Once we fix $\mathbf{1} = v^{+} \otimes v^{-}$, we have fixed an equivalence of top diffferential forms on $\Sh_{H}(K)$ and invariant measures on $\Sh_{H}(K)$ (see \cite[Page 83]{Harris97}). 
    \end{enumerate}
\end{convention}

\begin{proposition}
\label{Prop: integration of diff form into adelic integral}
For $\phi_f$ and $\rho$ be as in Proposition \ref{Prop: pairing to integration of diff form}. We have 
\begin{equation*}
    \langle \omega_{\Psi}, [\rho] \rangle_{B} =  \frac{[H(\Zp): K]^{-1}}{(2\pi{i})^{n + 1}} \int_{H(\QQ) Z(\AAA) \backslash H(\AAA) } (p^{*} Eis_{H}(\phi_f) \iota^{*} \omega_{\Psi} (\mathbf{1}))(g) dg, 
\end{equation*}
where $\mathbf{1}$ is the vector $v^{+} \otimes v^{-}$ and the vectors $v^{+}$ and $v^{-}$ are defined in equation (\ref{eq: def of v^{+} and v^{-}}).
\end{proposition}

\begin{proof}
    By Proposition \ref{Prop: pairing to integration of diff form}, we have 
    \begin{equation*}
         \langle \omega_{\Psi}, [\rho] \rangle_{B} = \frac{1}{(2\pi i)^{n + 1}} \int_{\Sh_{H}(K)} p^{*} Eis_{H}(\phi_f) \iota^{*} \omega_{\Psi}. 
    \end{equation*}
    Thanks to the equivalence of top differential forms on $\Sh_{H}(K)$ and invariant measures on $\Sh_{H}(K)$ explained above, we can write the integral as an adelic integral. More precisely, we have 
    \begin{align*}
        \langle \omega_{\Psi}, [\rho] \rangle_{B} &= \frac{1}{(2\pi i)^{n + 1}} \int_{H(\QQ) \backslash H(\AAA) / K_{H} K}(p^{*} Eis_{H}(\phi_f) \iota^{*} \omega_{\Psi}(\mathbf{1}))(g) dg \\
                                                  &= \frac{[H(\Zp): K]^{-1}}{(2\pi i)^{n + 1}} \int_{H(\QQ) Z_{H}(\RR) \backslash H(\AAA) }(p^{*} Eis_{H}(\phi_f) \iota^{*} \omega_{\Psi}(\mathbf{1}))(g) dg \\    
                                                  &= \frac{[H(\Zp): K]^{-1}}{(2\pi i)^{n + 1}} \int_{H(\QQ) Z(\RR) \backslash H(\AAA) }(p^{*} Eis_{H}(\phi_f) \iota^{*} \omega_{\Psi}(\mathbf{1}))(g) dg \\
                                                   &= \frac{[H(\Zp): K]^{-1}}{(2\pi i)^{n + 1}} \int_{H(\QQ) Z(\AAA) \backslash H(\AAA) }(p^{*} Eis_{H}(\phi_f) \iota^{*} \omega_{\Psi}(\mathbf{1}))(g) dg. 
    \end{align*}
    The second equality follows from $\mathrm{Vol}(K) = [H(\Zp): K]^{-1}$ and $\mathrm{Vol}(K_H / Z_{H} (\RR)) = 1$. The third equality is from the fact that $\mathrm{Vol}(\mathbb{S}^{1}) = 1$ and the fourth comes from $\mathrm{Vol}(\Zp^{\times}) = 1$. 
\end{proof}

\subsubsection{An explicit form of the adelic integral}

\par Recall that we let $\Psi_{\infty} \in \tilde{\pi}_{\infty}$ be a vector of weight $(-1 - b + r - s, 1 + a + r - s, r - s; b + 2s -r)$ and $v \in D$ be a vector of weight $(-a + s - r, b + s - r, s - r; -b - 2s + r)$. We also have the differential form $\omega_{\Psi}$ associated to $\Psi$ by Proposition \ref{Prop: explicit construnction of the differential form}. The following proposition gives a precise formula for $\langle \omega_{\Psi}, [\rho] \rangle_{B}$. 


\begin{notation}
\label{notation: Xi}
    \begin{itemize}
        \item   We let 
                \begin{equation*}
                    C_{j} := [H(\Zp): K]^{-1} \langle X^{2j -b -r -s}_{(1, -1, 0)}v , b^{a + b + r + s}_{n - j} \rangle. 
                \end{equation*}
        \item  We denote by $\Xi_{m, t}(\phi_f)$ the functions on $H(\AAA)$ defined by 
                \begin{equation*}
                    \Xi_{m, t}(\phi_f) := X^{m}_{(1, -1, 0)} \Psi \sum\limits_{\gamma \in B_{2}(\QQ) \backslash \GU(J_{2}^{'})^{'}(\QQ)} \gamma^{*} (\phi_{t}^{a + b + r + s} \otimes \phi_{f}). 
                \end{equation*}
        \item We let 
             \begin{align*}
                & A = \max \{0, \frac{b + r + s}{2} \}, \\
                & B = \min\{ \frac{a + 2b + r + s}{2}, a + b + r + s\}. 
            \end{align*}
    \end{itemize}
\end{notation}

\begin{proposition}
\label{prop: explicit_pairing}
The pairing $\langle \omega_{\Psi},  [\rho] \rangle_{B} $ is equal to 
\begin{equation}
\label{eq:pairing}
    -\frac{1}{2(a + b + r + s + 1)} \sum\limits_{j = A}^{B} (-1)^{b + r + s} (2j - b -r - s + 1) C_j\int \Xi_{1 + a + 2b -2j + r + s, a + b + r + s - 2j}(\phi_f),
\end{equation}
where the integrals are over $H(\QQ) Z(\AAA) \backslash H(\AAA)$. 
\end{proposition}

\begin{proof}
    By Proposition \ref{Prop: integration of diff form into adelic integral}, we have 
    \begin{equation*}
        \langle \omega_{\Psi}, [\rho] \rangle_{B} =  \frac{[H(\Zp): K]^{-1}}{(2\pi{i})^{n + 1}} \int_{H(\QQ) Z(\AAA) \backslash H(\AAA) } (p^{*} Eis_{H}(\phi_f) \iota^{*} \omega_{\Psi} (\mathbf{1}))(g) dg,  
    \end{equation*}
    where $\mathbf{1}$ is the vector $v^{+} \otimes v^{-}$.  
    \par It follows from the formula for $Eis_{H}(\phi_f)$ in equation (\ref{eqn:Eis_symbol_Hodge}) and the formula for $\iota^{*} \omega_{\Psi}$ in Proposition \ref{Prop:pullback_form} that for $g \in H(\AAA)$, 
    \begin{align*}
        & (p^{*} Eis_{H}(\phi_f) \iota^{*} \omega_{\Psi})(\mathbf{1})(g) \\
        = & -\frac{(2\pi{i})^{n + 1}}{2(n + 1)} (\sum\limits_{i = 0}^{a + b} (-1)^{i} (i + 1) X^{1 + a + b - i}_{(1, -1, 0)} \Psi_{\infty} \otimes \Psi_f) (g)\\
         \times & (\sum\limits_{\gamma \in B_{2}(\QQ) \backslash \GL_2(\QQ)} \sum\limits_{j = 0}^{n} \gamma^{*} (\phi_{n - 2j}^{n} \otimes \phi_f))(g) \langle X^{i}_{(1, -1, 0)}v, b^{n}_{n-j} \rangle.
    \end{align*}
    The weight of $X^{i}_{(1, -1, 0)}v$ is $\lambda^{\prime}(i - a, -a -b -r -s)$ when restricting to $\GL_{2}$ since its weight  is $(i - a + s - r, -i + b + s - r, s - r; -b - 2s + r)$. It can be seen from the definition of the pairing and the fact that the weight of $b^{n}_{n - j}$ is $\lambda^{\prime}(a + b + r + s - 2j, a + b + r + s)$ 
    that if $a + b + r + s - 2j \neq -(i - a)$, then 
    \begin{equation*}
        \langle X^{i}_{(1, -1, 0)}v, b^{n}_{n-j} \rangle = 0. 
    \end{equation*}
    Thus only the term satisfying $i = 2j - b - r - s$ remains. 
    By the inequality $0 \le i \le a + b$, we have 
    \begin{equation*}
        \frac{b + r + s}{2} \le j \le \frac{a + 2b + r + s}{2}. 
    \end{equation*}
    By $0 \le j \le a + b + r + s$ and the definition of $A$ and $B$, we have $A \le j \le B$. Hence, we get 
    \begin{align*}
        & (p^{*} Eis_{H}(\phi_f) \iota^{*} \omega_{\Psi})(\mathbf{1})(g) \\ 
        =& -\frac{(2\pi{i})^{n + 1}}{2(n + 1)} \sum\limits_{j = A}^{B} (-1)^{b + r + s}(2j -b -r -s + 1) [H(\Zp): K]C_j \\
        \times & (X^{1 + a + 2b -2j + r + s}_{(1, -1, 0)} \Psi_{\infty} \otimes \Psi_f ) \times (\sum\limits_{\gamma \in B_{2}(\QQ) \backslash \GL_2(\QQ)} \gamma^{*} (\phi^{a + b + r + s}_{a + b + r + s -2j} \otimes \phi_f)). 
    \end{align*}
    Plugging this into the integral, we get
    \begin{equation*}
        \langle \omega_{\Psi}, [\rho] \rangle_{B} = -\frac{1}{2(n + 1)} \sum\limits_{j = A}^{B} (-1)^{b + r + s}(2j - b - r -s  + 1)C_j \int \Xi_{1 + a + 2b -2j + r + s, a + b + r + s - 2j}(\phi_f). 
    \end{equation*}
\end{proof}

\begin{remark}
    \begin{enumerate}
        \item  By the inequality $A \le j \le B$, we can see that $2j - b -r - s + 1 > 0$, which is important for later applications. 
        \item  The $K_{\GL_2}$-weight of $\Xi_{1 + a + 2b -2j + r + s, a + b + r + s - 2j}(\phi_f)$ is $0$, so the integral 
        \begin{equation*}
            \int \Xi_{1 + a + 2b -2j + r + s, a + b + r + s - 2j}(\phi_f)
        \end{equation*}
        could be non-zero. 
    \end{enumerate}
\end{remark}

\begin{lemma}
\label{Lemma:C_nonvanish}
    The constant $C_{b + s}$ is non-zero. 
\end{lemma}

\begin{proof}
    First, recall that
    \begin{equation*}
        C_{b + s} = [H(\Zp): K]^{-1} \langle X^{b + s - r}_{(1, -1, 0)}v, b^{a + b + r + s}_{a + r} \rangle. 
    \end{equation*}
    The weight of $X^{b + s - r}_{(1, -1, 0)}v$ is $(b - a + 2(s - r), 0, s - r; -b -2s + r)$, so the weight of it is $\lambda^{\prime}(b + s - (a + r), -(a + b + r + s))$ when restricting to $\GL_2$. By the fact that the weight of the representation $\Sym^{n}V_2^{\vee}$ of $\GL_2$ is multiplicity -free, we have 
    \begin{equation*}
        X^{b + s - r}_{(1, -1, 0)}v|_{\GL_2} = \lambda^{b+s}(v) a^{a + b + r + s}_{a + r},
    \end{equation*}
    where $a^{a + b + r + s}_{a + r}$ is $(b^{a + b + r +s}_{a + r})^{\vee}$ and $\lambda^{b+s}(v) \in \ZZ_{\ge 0}$.
    By the definition of the pairing $\langle \cdot, \cdot \rangle$, in order to show that $C_{b + s} \neq 0$, it suffices to show that 
    \begin{equation*}
         \rho(X^{b + s - r}_{(1, -1, 0)}v) = X^{b + s - r}_{(1, -1, 0)}v|_{\GL_2} \neq 0. 
    \end{equation*}
    Second, if we untwist by the character $\mu^{-b - 2s + r}$,  the weight of $X^{a + r}_{(1, 0, -1)}X^{b + s - r}_{(1, -1, 0)}v$ is $(b + 2s - r, 0, -a + s - 2r)$. After the untwist, the highest weight of 
    \begin{equation*}
        \Sym^{n}V_{2}^{\vee} \hookrightarrow \iota^{*}D
    \end{equation*}
    is also $(b + 2s - r, 0, -a + s - 2r)$, which is the same as the weight of $X^{a + r}_{(1, 0, -1)}X^{b + s - r}_{(1, -1, 0)}v$. \\
    Thirdly, we have the following two formulas: for a representation $V$ of the Lie algebra $\lieg_{\CC}$, for any vector $v \in V$, 
    we have
    \begin{align*}
        X_{(1, 0, -1)}X_{(1, -1, 0)}v &= [X_{(1, 0, -1)}, X_{(1, -1, 0)}]v + X_{(1, -1, 0)}X_{(1, 0, -1)}v \\
                                      &= X_{(1, -1, 0)}X_{(1, 0, -1)}v.
    \end{align*}
    Finally, by the above formula, we have 
    \begin{equation*}
        X_{(1, 0, -1)}(X^{a + r}_{(1, 0, -1)}X^{b + s - r}_{(1, -1, 0)}v) = X^{b + s - r}_{(1, -1, 0)}(X_{(1, 0, -1)}^{a + r + 1} v) = 0,
    \end{equation*}
    where the last equality follows from equation (\ref{eq: Xv = 0}).
    Hence, $X^{a + r}_{(1, 0, -1)}X^{b + s - r}_{(1, -1, 0)}v$ is a highest weight vector of 
    \begin{equation*}
        \Sym^{n}V_{2}^{\vee}  \hookrightarrow \iota^{*}D . 
    \end{equation*}
    By the branching law of $\GL_2 \subset \GL_3$, the highest weight vector is unique up to a non-zero constant, so $X^{a + r}_{(1, 0, -1)}X^{b + s - r}_{(1, -1, 0)}v \neq 0$. Thus
    \begin{equation*}
        \rho(X^{a + r}_{(1, 0, -1)}X^{b + s - r}_{(1, -1, 0)})v = X^{a + r}_{(1, 0, -1)} \rho(X^{b + s - r}_{(1, -1, 0)}v) \neq 0. 
    \end{equation*}
    Hence, we have 
    \begin{equation*}
        \rho(X^{b + s - r}_{(1, -1, 0)}v) \neq 0. 
    \end{equation*}
\end{proof}

\begin{remark}
    \begin{enumerate}
        \item From the computation of an archimedean zeta integral in the next section, we can see that the only non-zero summand in the summation of Lemma \ref{prop: explicit_pairing} is associated to $j = b + s$, so we only need to verify that $C_{b + s}$ is non-zero. 
        \item We can replace $v$ by $C_{b + s}^{-1}v$ so that we replace $\omega_{\Psi}$ by $C_{b + s}^{-1}\omega_{\Psi}$. This will not affect the result because our goal is compute the quotient 
        \begin{equation*}
            \frac{\langle \omega_{\Psi}, \tilde{\nu}_{K} \rangle_{B}}{\langle \omega_{\Psi},  \tilde{\nu}_{D} \rangle_{B}}. 
        \end{equation*}
    \end{enumerate}
\end{remark}
