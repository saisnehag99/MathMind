\section{Local actions on odd and covering Khovanov homology}
\label{sec:action_on_homology}

In this section, we describe actions on odd and covering Khovanov homology.
Following \cite{KR_PositiveHalfWitt_2016,QRS+_Symmetries$mathfrakgl_N$foams_2024a},
\cref{subsec:relative_homotopy_category} introduces the relative homotopy category, which gives the formal framework where the invariant is defined.
Our exposition is slightly different from \emph{op.\ cit.} (beyond using graded structures), as we avoid the use of triangulated categories.
We then define the marked tangle invariant in \cref{subsec:defn_tangle_invariant}.
Finally, \cref{subsec:topological_invariance} and \cref{subsec:marking_slide} show topological invariance and marking slide, respectively.

\subsection{The relative homotopy category}
\label{subsec:relative_homotopy_category}

\begin{convention}
  All chain complexes are bounded chain complexes.
\end{convention}

Let $A$ be a $\fg$-category.
A \emph{$\fg$-equivariant chain complex} is a chain complex in $A$ whose differential has $\fg$-equivariant components.
% In the sequel, we only consider $\fg$-equivariant chain complexes, referred to as ``(chain) complexes''.
A chain morphism is said to be \emph{$\fg$-equivariant} if each of its components is $\fg$-equivariant.
We denote $\Ch(A)$ the category of $\fg$-equivariant chain complexes and $\fg$-equivariant chain morphisms, and $\underline{\Ch}(A)$ the category of $\fg$-equivariant chain complexes and \emph{all} chain morphisms.
There is an embedding $\Ch(A)\hookrightarrow\underline{\Ch}(A)$.

Homotopies have the standard meaning; that is, homotopies for the pre-additive category underlying $A$. A \emph{$\fg$-equivariant homotopy equivalence} is a homotopy equivalence which is also $\fg$-equivariant as a chain morphism.
Note that if $f$ is a $\fg$-equivariant homotopy equivalence, its inverse needs not be $\fg$-equivariant.
A $\fg$-equivariant chain complex $C_\bullet$ is \emph{contractible} if it is contractible in the standard sense, that is, if the (necessarily $\fg$-equivariant) chain morphism $C_\bullet\to 0$ (or equivalently, $0\to C_\bullet$) is a homotopy equivalence.

\begin{definition}
  \label{defn:formal_relative_homotopy_category}
  Let $A$ be a $\fg$-category.
  The \emph{relative homotopy category} $\cK^\fg(A)$ is the localization of $\Ch(A)$ at $\fg$-equivariant homotopy equivalences.
  We denote $\simeq^\fg$ an isomorphism in $\cK^\fg(A)$.
\end{definition}

We unpack the definition; see \cite[section~III.2.2]{GM_MethodsHomologicalAlgebra_2003} for a more thorough review of localization.
Objects of $\cK^\fg(A)$ are the same objects as those in $\Ch(A)$; namely, $\fg$-equivariant chain complexes in $A$.
In this context, a \emph{path} is formal composition of arrows
\[u_0\overset{f_1}{\longrightarrow}u_1\overset{f_2}{\longrightarrow}\ldots\overset{f_n}{\longrightarrow}u_n,\]
where $f_i\colon u_{i-1}\to u_i$ is either a $\fg$-equivariant chain morphism or the inverse of a $\fg$-equivariant homotopy equivalence.
Two paths are \emph{equivalent} if they can be joined by a chain of the following elementary equivalences:
\begin{itemize}
  \item two consecutive arrows are replaced by their composition;
  \item the composition of a $\fg$-equivariant homotopy equivalence with its inverse is replaced by the identity.
\end{itemize}
Morphisms in $\cK^\fg(A)$ are equivalences classes of paths.

Denote $\cK(A)$ (resp.\ $\underline{\cK}(A)$) the homotopy category of $\Ch(A)$ (resp.\ $\underline{\Ch}(A)$), that is, the localization of $\Ch(A)$ (resp.\ $\underline{\Ch}(A)$) at $\fg$-equivariant homotopy equivalences with $\fg$-equivariant inverses (resp.\ at homotopy equivalences).
The relative homotopy category $\cK^\fg(A)$ can be understood as sitting in between $\cK(A)$ and $\underline{\cK}(A)$, inverting homotopy equivalence that are $\fg$-equivariant but may not have $\fg$-equivariant inverses.
Namely, there is a commutative diagram
\begin{center}
  \begin{tikzcd}
    \cK(A) \ar[dr]\ar[rr] && \underline{\cK}(A)
    \\
    & \cK^\fg(A)\ar[ur]
  \end{tikzcd}
\end{center}
with each arrow being the obvious quotient functor.
The category $\cK^\fg(A)$ satisfies the universal property that if $F\colon \Ch(A)\to T$ is a functor sending $\fg$-equivariant homotopy equivalences to isomorphisms, then the functor $F$ factors through the quotient $\Ch(A)\to \cK^\fg(A)$.

\begin{remark}
  As a triangulated category, the category $\cK^\fg(\ring)$ coincides with the relative homotopy category $\cC^\fg(\ring)$ as defined in \cite{QRS+_Symmetries$mathfrakgl_N$foams_2024a}.
\end{remark}

If $C_\bullet$ is a $\fg$-equivariant chain complex in $\gMod$, its homology $H_\bullet(C)$ is canonically endowed with a structure of $\fg$-module, preserving the homological grading.
If $f\colon C_\bullet\to D_\bullet$ is a $\fg$-equivariant chain morphism, it induces a linear map $[f]\colon H_\bullet(C)\to H_\bullet(D)$, preserving both the homological grading and the $\fg$-action.
Furthermore, if $f$ is a homotopy equivalence then $[f]$ an isomorphism.
Recall $\ugMod$ from \cref{ex:ugMod} and write $\cK^\fg(\ring)\coloneqq\cK^\fg(\ugMod)$.
It follows from the above discussion and the universal property of $\cK^\fg(\ring)$ that the homology functor $H_\bullet$ descends to a functor from $\cK^\fg(\ring)$ to $(\gMod)^\bZ$.
This justifies the definition of the relative homotopy category as the category capturing which $\fg$-equivariant complexes have the same homology, with the same induced $\fg$-action.

If $A$ is a $\fg$-category, any choice of object $w$ in $A$ defines a representable functor
\[\Hom_{A}(w,-)\colon A\to\ugMod.\]
It descends to the relative homotopy categories, leading to a homology functor:
\[\cK^\fg(A)\overset{\Hom_{A}(w,-)}{\longrightarrow} \cK^\fg(\ring)\overset{H_\bullet}{\longrightarrow}(\gMod)^\bZ.\]



\subsection{Definition of the marked tangle invariant}
\label{subsec:defn_tangle_invariant}

In this subsection, we adapt the construction of the tangle invariant in \cite{SV_OddKhovanovHomology_2023} to carry a $\fg$-action.
In this article, a tangle diagram always refers to a sliced tangle diagram.

\medbreak

In contrast with \cite{SV_OddKhovanovHomology_2023}, the construction in this article applies to ``marked tangles''.
A \emph{marked tangle diagram} is a tangle diagram with extra markings $\greenmarking[2]$, each labelled with a triple of scalars, as for webs.
For us, a \emph{marked tangle} is an equivalence class of marked tangle diagrams with respect to the standard relations on tangle diagrams, together with the relations (here $\omega=(\alpha,\beta_1,\beta_2)$ is a generic triple of scalars):
\begin{gather}
  \label{eq:marking_slide_through_cup_cap}
  \tikzpic{
    \draw[web1] (1,0) to[out=180,in=-90] (.5,.5) to[out=90,in=180] (1,1);
    \node[green_mark] (B) at (1-.35,1-.15) {};
    \node[above left={-5pt} of B] {\scriptsize $\omega$};
  }[xscale=.8,yscale=.6][(0,.5*.6)]
  \;\leftrightarrow\;
  \tikzpic{
    \draw[web1] (1,0) to[out=180,in=-90] (.5,.5) to[out=90,in=180] (1,1);
    \node[green_mark] (B) at (1-.35,.15) {};
    \node[below left={-5pt} of B] {\scriptsize $\omega$};
  }[xscale=.8,yscale=.6][(0,.5*.6)]
  \quad\an\quad
  \tikzpic{
    \draw[web1] (0,0) to[out=0,in=-90] (.5,.5) to[out=90,in=0] (0,1);
    \node[green_mark] (B) at (.35,1-.15) {};
    \node[above right={-5pt} of B] {\scriptsize $\omega$};
  }[xscale=.8,yscale=.6][(0,.5*.6)]
  \;\leftrightarrow\;
  \tikzpic{
    \draw[web1] (0,0) to[out=0,in=-90] (.5,.5) to[out=90,in=0] (0,1);
    \node[green_mark] (B) at (.35,.15) {};
    \node[below right={-5pt} of B] {\scriptsize $\omega$};
  }[xscale=.8,yscale=.6][(0,.5*.6)]
  \;.
\end{gather}
That is, markings can slide along strands, but (a priori) not over or under crossings.
One could give a topological description, with markings being points where the tangle is ``glued onto the plane'' or ``attached to the point at infinity'', depending on the topological model. We omit the details.

A \emph{(marked) tangled web} is a (marked) web where one may further use the following crossings:
\begin{gather*}
  \def\webscl{.6}
  \tikzpic{
    \webncr
  }[scale=\webscl][(0,.5*\webscl)]
  \qquad\an\qquad
  \tikzpic{
    \webpcr
  }[scale=\webscl][(0,.5*\webscl)]
  \;.
\end{gather*}
If $W$ is a marked tangled web, then $\undcomp(W)$ is a marked tangle diagram. As explained in \cite{SV_OddKhovanovHomology_2023} (and following \cite{LQR_KhovanovHomologySkew_2015}), we have that:

\begin{lemma}
  For any marked tangle diagram $T$, there exists a marked tangled web $W$ such that $\undcomp(W)=T$.
\end{lemma}

To realise the above lemma in practice, it is useful to introduce mixed crossings:
\begin{gather*}
  \tikzpic{\webcrDS}[xscale=.6,yscale=.4]\coloneqq
  \tikzpic{
    \webm[0][1]\draw[web1] (1,2) to (2,2);
    \draw[web1] (0,0) to (1,0);\websRoundMarkedT[1][0]
  }[xscale=.6,yscale=.4]
  \qquad
  %
  \tikzpic{\webcrSD}[xscale=.6,yscale=.4]\coloneqq
  \tikzpic{
    \draw[web1] (0,2) to (1,2);\websRoundMarkedT[1][1]
    \webm\draw[web1] (1,0) to (2,0);
  }[xscale=.6,yscale=.4]
  \qquad\an\qquad
  %
  \tikzpic{\webcrDD}[xscale=.6,yscale=.4]\coloneqq
  \tikzpic{
    \draw[web2] (0,1) to (1,1);
    \draw[web2] (0,0) to (1,0);
  }[xscale=.6,yscale=.4]
  \;.
\end{gather*}
Two different tangled webs $D_1$ and $D_2$ can have the same underlying tangle diagram $\undcomp(D_1) =\undcomp(D_2)$. Indeed, we may have that $D_1\in\fg\foam_{d_1}$ and $D_2\in\fg\foam_{d_2}$ for $d_1\neq d_2$, as we can always add a double line on the top or bottom of a web; and even if $d_1=d_2$, the webs $D_1$ and $D_2$ may not have the same input and output coordinates; as we can always compose a web horizontally with a mixed crossing.
Instead, we would want to think of tangled webs
\begin{itemize}
  \item up to adding double lines on the top or bottom of the web
  \item and up to adding mixed crossings on the right or the left of the web.
\end{itemize}
This is formalize as follows.
On the one hand, adding a double line to a web (resp.\ a 2-facet to a foam) on top (resp.\ on the back) defines a $\fg$-2-functor $\fg\foam_d^{\greenmarking}\to\fg\foam^{\greenmarking}_{d+2}$ (see also the front-back composition from \cref{rem:monoidal-2-categorical-structure}). In fact, it follows from the basis theorem shown in \cite{Schelstraete_RewritingModuloDiagrammatic_2025} that these $\fg$-2-functors are embeddings. We refer to this type of $\fg$-2-functors as ``adding double lines''.
On the other hand, pre- and post-composing with mixed crossings define various $\fg$-2-functors $\fg\foam_d^{\greenmarking}(\lambda,\mu)\to\fg\foam_d^{\greenmarking}(\lambda',\mu')$, where $\lambda$ (resp.\ $\mu$) has the same number of $1$'s as $\lambda'$ (resp.\ $\mu'$). In fact, these $\fg$-2-functors are isomorphisms. We refer to this type of $\fg$-2-functors as ``changing the endpoints''.

\begin{definition}
  \label{defn:colimit_of_foam}
  The $\fg$-2-category $\fg\foam^{\greenmarking}$ is the colimit over ``adding double lines'' and ``changing the endpoints'' $\fg$-2-functors.
  % \begin{center}
  %   \begin{tikzcd}
  %     & \fg\foam^{\greenmarking}_d \ar[dl,hook']\ar[dr,hook]
  %     \\
  %     \fg\foam^{\greenmarking}_{d+2} && \fg\foam^{\greenmarking}_{d+2}
  %   \end{tikzcd}
  % \end{center}
  % where the arrows are embeddings given by adding a 2-facet at the back and at the front, respectively.
\end{definition}

The tangle invariant in \cite{SV_OddKhovanovHomology_2023} is defined as a certain tensor product of chain complexes in $\fg\foam$.
In this context, one needs a new notion of tensor product of chain complexes to account for the graded interchange law, which we review now.

In practice, it means that there exists a graded analogue of the Koszul rule, suitably compatible with homotopy equivalence; this was shown in the second author's master thesis \cite{Schelstraete_SupercategorificationKhovanovlikeTangle_2020}.
Recall that the usual tensor product of two chain complexes looks like a grid, and the Koszul rule is a way to assign signs to edges, such that each square has an odd number of signs; this makes the induced differential squares to zero.
In fact, one does not need to follow the Koszul rule: any two ways of assigning signs to edges such that each square has an odd number of signs lead to isomorphic chain complexes.

In a similar way, if $A_\bullet$ and $B_\bullet$ are two chain complexes with homogeneous differentials, the graded Koszul rule defines $A_\bullet\otimes B_\bullet$ by assigning invertible scalars to edges; denote it $\epsilon$, and view it as 1-cochain on the oriented grid.
For each square
\begin{center}
  \begin{tikzcd}
    \bullet \ar[r,"\alpha\otimes\id"] \ar[d,"\id\otimes\beta"'] & \bullet
    \ar[d,"\id\otimes\beta"]\\
    \bullet \ar[r,"\alpha\otimes\id"] & \bullet
  \end{tikzcd}
\end{center}
in the grid, the assignment $\epsilon$ is such that $\partial\epsilon = \bil(\alpha,\beta)$.
In fact, one does not need to follow the graded Koszul rule: any two ways of assigning invertible scalars to edges such that each square has this property lead to isomorphic chain complexes.
One can proceed inductively and define a tensor product for ``sufficiently homogeneous complexes'', called ``homogeneous polycomplexes''.
We refer to \cite{SV_OddKhovanovHomology_2023} for the precise definition.
As for the classical tensor product, this graded tensor product is suitably compatible with homotopy equivalence.

It is not hard to extend the above construction to the equivariant setting.
We omit the details, and summarize the main points in the following proposition:

\begin{proposition}
  \label{prop:tensor_product_chain_complexes}
  Let $\cA$ be an additive $\fg$-2-category.
  For a certain family of complexes called \emph{$\fg$-equi\-va\-riant homogeneous polycomplexes}, there exists a procedure that, given two $\fg$-equi\-va\-riant homogeneous polycomplexes $A_\bullet$ and $B_\bullet$, defines a $\fg$-equi\-va\-riant homogeneous polycomplex $A_\bullet\otimes B_\bullet$.
  We call it the \emph{graded tensor product of chain complexes}.
  This procedure is such that:
  \begin{equation*}
    A_\bullet\simeq^\fg C_\bullet
    \quad\an\quad
    B_\bullet\simeq^\fg D_\bullet
    \qquad\Rightarrow\qquad
    A_\bullet\otimes B_\bullet\simeq^\fg C_\bullet\otimes D_\bullet,
  \end{equation*} 
  where $\simeq^\fg$ denotes isomorphism in the relative homotopy category $\cK^\fg(\cA)$.
\end{proposition}

We can now define the tangle invariant:

\begin{definition} 
\label{defn:tangle_invariant}
  \def\webscl{.6}
  Assume $2$ is invertible in $\ringfoam$.
  Let $D$ be a diagram of tangled webs with markings.
  The complex $\fg\glKom(D)\in\Ch(\fg\markedfoam[])$ is defined on elementary marked webs as follows (the homological degree zero is underlined):
  \begin{gather*}
    \tikzpic{
      \draw[web1] (-1,0) to (1,0);
      \node[green_mark] (B) at (0,0){};
      \node[below={-3pt} of B] {\scriptsize $(\alpha,\beta_1,\beta_2)$};
    }[scale=.7][(0,0)]
    \mspace{10mu}\mapsto\mspace{10mu}
    \tikzpic{
      \draw[web1] (-1,0) to (1,0);
      \node[green_mark] (B) at (0,0){};
      \node[below={-3pt} of B] {\scriptsize $(\alpha,\beta_1,\beta_2)$};
      \draw[decorate,decoration=snake] (-1,-1) to (1,-1);
    }[scale=.7][(0,0)]
    %
    \mspace{60mu}
    %
    \tikzpic{
      \webm
    }[scale=\webscl][(0,.5*\webscl)]
    \mspace{10mu}\mapsto\mspace{10mu}
    \tikzpic{
      \webm
      \draw[decorate,decoration=snake] (0,-.5) to (1,-.5);
    }[scale=\webscl][(0,.5*\webscl)]
    %
    \mspace{60mu}
    %
    \tikzpic{
      \webs
    }[scale=\webscl][(0,.5*\webscl)]
    \mspace{10mu}\mapsto\mspace{10mu}
    \tikzpic{
      \websRoundMarkedB[0][0][]
      \draw[decorate,decoration=snake] (0,-.5) to (1,-.5);
    }[scale=\webscl][(0,.5*\webscl)]
    \\[2ex]
    %%%%%%%%%%%%%%%%%%%%%%%%%%%%%%
    %%%%%%%%%%%%%%%%%%%%%%%%%%%%%%
    \tikzpic{
      \webncr
    }[scale=\webscl][(0,.5*\webscl)]
    %
    \quad\mapsto\quad
    %
    \tikzpic{
      \websRoundMarkedB[0][0][]
      \webm[1][0]
    }[scale=\webscl][(0,.5*\webscl)]
    \;\langle-\frac{1}{2},\frac{1}{2}\rangle
    %
    \quad
    \xrightarrow{\tikzpic{\funzip}[scale=.6]}
    \quad
    %
    \tikzpic{
      \webid
      \draw[decorate,decoration=snake] (0,-.5) to (2,-.5);
    }[scale=\webscl][(0,.5*\webscl)]
    \\[2ex]
    %%%%%%%%%%%%%%%%%%%%%%%%%%%%%%
    %%%%%%%%%%%%%%%%%%%%%%%%%%%%%%
    \tikzpic{
      \webpcr
    }[scale=\webscl][(0,.5*\webscl)]
    %
    \quad\mapsto\quad
    %
    \tikzpic{
      \webid
    }[scale=\webscl][(0,.5*\webscl)]
    \;\langle-\frac{1}{2},\frac{1}{2}\rangle
    %
    \quad
    \xrightarrow{\tikzpic{\fzip}[scale=.6]}
    \quad
    %
    \tikzpic{
      \websRoundMarkedB[0][0][]
      \webm[1][0]
      \draw[decorate,decoration=snake] (0,-.5) to (2,-.5);
    }[scale=\webscl][(0,.5*\webscl)]
  \end{gather*}
  and extending to $D$ by taking graded tensor product of chain complexes, as given in \cref{prop:tensor_product_chain_complexes}.
\end{definition}

\begin{remark}
  Contrary to \cite{QRS+_SymmetriesEquivariantKhovanovRozansky_2023}, twists do not only arise from crossings. This should be explained by the fact that while both theories are oriented (ie use webs), the setting of \cite{SV_OddKhovanovHomology_2023} uses the more restricted setting of directed webs.
  At the time of writing, there is no non-directed model for odd (or covering) Khovanov homology.
\end{remark}

The next two subsections explore its property, namely topological invariance, and how markings can slide through crossings.



\subsection{Topological invariance}
\label{subsec:topological_invariance}

In this subsection, we prove topological invariance:


\begin{theorem}
  \label{thm:topological_invariance}
  Assume $2$ is invertible in $\ringfoam$.
  Let $T$ be a marked tangle and $D$ a marked tangled web presenting $T$.
  Denote $N_+$ (resp.\ $N_-$) the number of positive (resp.\ negative) crossings in $D$, and $w\coloneqq N_+-N_-$ its writhe.
  In the relative homotopy category $\cK^\fg(\fg\markedfoam[])$, the object
  \begin{gather*}
    \fg\glKh(D)\coloneqq t^{N_+}\fg\glKom(D)\langle\frac{w+N_+}{2},-\frac{w+N_+}{2}\rangle
  \end{gather*}
  only depends on $T$, up to isomorphism.
  We write $\glCKh(D)\coloneqq \fg\glKh(D)$ when $\fg=\grgl_2^\geq$ and $\glOKh(D)\coloneqq \fg\glKh(D)$ when $\fg=\gloo$.
\end{theorem}

Here we remind the reader that as defined in the beginning of \cref{subsec:twist_foam}, every twist $(\alpha,\beta_1,\beta_2)$ carries a shift in the quantum grading by $q^{\beta_2-\beta_1}$.
Note that this theorem does not say anything on how markings can slide through crossings: this is discussed in the next subsection.

\begin{remark}
  \label{rem:local_invariant_2_invertible}
  If one restricts $\grgl^\geq_2$ to $\grsl^\geq_2$ as in \cref{ex:defn_covering_sl2} (resp.\ $\gloo$ to $\sloo$ as in \cref{ex:defn_sloo}), then one can do away with the condition that $2$ is invertible in the ground ring $\ringfoam$.
\end{remark}

The remainder of this subsection is devoted to the proof of \cref{thm:topological_invariance}.
The proof is adapted from the proof of invariance in \cite{SV_OddKhovanovHomology_2023}, incorporating $\fg$-equivariance as in the analogous proof in \cite{QRS+_SymmetriesEquivariantKhovanovRozansky_2023}.
Given our description of the relative homotopy category given in \cref{subsec:relative_homotopy_category}, finding an isomorphism in $\cK^\fg(\fg\foam^{\greenmarking})$ amounts to finding a zigzag of $\fg$-equivariant homotopy equivalences in $\underline{\Ch}(\fg\foam^{\greenmarking})$.

Thanks to the properties of the graded tensor product of complexes (\cref{prop:tensor_product_chain_complexes}), we can work locally.
We first show invariance under planar isotopy in \cref{lem:invariance_planar_isotopies}, where a \emph{planar isotopy} between two marked tangles is a planar isotopy between the underlying tangles, such that markings do not slide through crossings. (In other words, it consists of the usual planar isotopy relations, together with \eqref{eq:marking_slide_through_cup_cap}).
We then show invariance under Reidemeister I, Reidemeister II and Reidemeister III in \cref{lem:invariance_Reidemeister_I}, \cref{lem:invariance_Reidemeister_II} and \cref{lem:invariance_Reidemeister_III}, respectively.

\begin{lemma}
  \label{lem:invariance_planar_isotopies}
  Let $D_1$ and $D_2$ be two marked tangled webs. If $\undcomp(D_1)$ and $\undcomp(D_2)$ are planar isotopic, then there is an equivariant isomorphism between $\fg\glKom(D_1)$ and $\fg\glKom(D_2)$ in $\Ch(\fg\markedfoam[])$.
\end{lemma}

\begin{proof}
  We have already seen in \cref{lem:twist_dot_slide_cup_cap} that markings can slide through cups and caps.
  Hence, it suffices to prove invariance under elementary planar isotopies (see \cite[Figure~3.2]{SV_OddKhovanovHomology_2023}), following the proof of Lemma 3.8 in \cite{SV_OddKhovanovHomology_2023}.
  On the one hand,
  invariance under planar isotopies interchanging two elementary tangles is realised by foam crossings, which are always $\fg$-equivariant;
  on the other hand,
  invariance under zigzags isotopies and pitchfork isotopies essentially use the isomorphisms given in \cref{lem:web_defining_relations_equivariance}, which we showed to be $\fg$-equivariant.
  For instance, the isomorphism for one of the pitchfork isotopies is given as follows:
  \begin{center}
    \begin{tikzpicture}
      \def\vspc{2}\def\hspc{5}
      \def\webxscl{.4}\def\webyscl{.3}
      \node (A1) at (0,\vspc) {
        \tikzpic{
          \websRoundMarkedT[1][0]
          \webpcr[0][1]
          \draw[web1] (0,0) to (1,0);
          \draw[web1] (1,2) to (2,2);
        }[xscale=\webxscl,yscale=\webyscl]
      };
      \node (A2) at (\hspc,\vspc) {
        \tikzpic{
          \websRoundMarkedT[1][0]
          \draw[web1] (1,2) to (2,2);
        }[xscale=\webxscl,yscale=\webyscl]
        $\langle -\frac{1}{2},\frac{1}{2} \rangle$
      };
      \node (A3) at (2*\hspc,\vspc) {
        \tikzpic{
          \websRoundMarkedT[1][0]
          \websRoundMarkedT[-1][1]\webm[0][1]
          \draw[web1] (0,0) to (1,0);
          \draw[web1] (1,2) to (2,2);
          \draw[web1] (-1,0) to (0,0);
        }[xscale=\webxscl,yscale=\webyscl]
      };
      %
      \node (B1) at (0,0) {
        \tikzpic{
          \websRoundMarkedB[1][1]\webm[2][1]
          \webncr[0][0]\websRoundMarkedB[3][0]
          \draw[web1] (0,2) to (1,2);
          \draw[web1] (3,2) to (4,2);
          \draw[web1] (1,0) to (3,0);
        }[xscale=\webxscl,yscale=\webyscl]
      };
      \node (B2) at (\hspc,0) {
        \tikzpic{
          \websRoundMarkedB[1][1]\webm[2][1]
          \websRoundMarkedB[-1][0]\webm[0][0]\websRoundMarkedB[3][0]
          \draw[web1] (-1,2) to (1,2);
          \draw[web1] (3,2) to (4,2);
          \draw[web1] (1,0) to (3,0);
        }[xscale=\webxscl,yscale=\webyscl]$\langle -\frac{1}{2},\frac{1}{2} \rangle$
      };
      \node (B3) at (2*\hspc,0) {
         \tikzpic{
          \websRoundMarkedB[1][1]\webm[2][1]
          \websRoundMarkedB[3][0]
          \draw[web1] (3,2) to (4,2);
          \draw[web1] (1,0) to (3,0);
        }[xscale=\webxscl,yscale=\webyscl]
      };
      %
      \draw[->] (A2) to (A3);
      \node[above] at (1.5*\hspc,\vspc) {
        \tikzpic{\fzip[-.5][0][2]\frst[1][0]}[scale=.4]
      };
      \draw[->] (B2) to (B3);
      \node[above] at (1.5*\hspc,0) {
        \tikzpic{\funzip[-.5][0]\frst[1][0][2]\flst[2][0][2]\frst[2.5][0]}[scale=.4]
      };
      %
      \draw[->] (B2) to (A2);
      \node[left] at (\hspc,.5*\vspc) {$\lambda\;
        \tikzpic{
          \clip (-2,0) rectangle (2,2);
          \funzip[0][0][2]
          \begin{scope}[scale=2]
            \fcap[-.25][0]
          \end{scope}
          \draw[foamdraw1] (-1.5,0) to (-1.5,2);
        }[scale=.4]$
      };
      \draw[->] (B3) to (A3);
      \node[right] at (2*\hspc,.5*\vspc) {$\id$};
      %
      \node[right={5pt} of A1] {:};
      \node[right={5pt} of B1] {:};
    \end{tikzpicture}
  \end{center}
  where $\lambda$ is some scalar that we do not need to compute; here we use the squeezing relation. This concludes.
\end{proof}

Before proving invariance under Reidemeister moves, we recall the following homological fact.

\begin{lemma}
  \label{lem:abstract_homotopy_equivalence}
  Let $A$ be an additive category and let
  \[P_\bullet\overset{f}{\longrightarrow} C_\bullet\overset{g}{\longrightarrow} D_\bullet\overset{h}{\longrightarrow}Q_\bullet\]
  be a chain complex in $A$ which is split exact at $C_\bullet$ and $D_\bullet$. If $P_\bullet$ and $Q_\bullet$ are contractible, then $g$ is a homotopy equivalence with inverse given by the splitting.
\end{lemma}

\begin{proof}
  Let $\ov{f}$ and $\ov{g}$ the maps giving the splitting at $C_\bullet$, so that $f\circ\ov{f}+\ov{g}\circ g=\id_{C_\bullet}$.
  Let $h^P$ be the homotopy between $\id_{P_\bullet}$ and $0$, so that $h^P\circ d_P+d_P\circ h^P=\id_P$.
  The map $f\circ h^P\circ \ov{f}$ defines a homotopy between $\ov{g}\circ g$ and $\id_{C_\bullet}$, as one can check that:
  \begin{align*}
    (f\circ h^P\circ \ov{f})\circ d_{C_\bullet}+d_{C_\bullet}\circ (f\circ h^P\circ \ov{f})
    &= 
    f\circ h^P\circ d_{P_\bullet}\circ \ov{f}+f\circ d_{P_\bullet}\circ h^P\circ \ov{f}
    \\
    &=
    f\circ\ov{f}
    =
    \id_{C_\bullet}-\ov{g}\circ g.
  \end{align*}
  A similar argument gives a homotopy between $g\circ\ov{g}$ and $\id_{D_\bullet}$.
\end{proof}

\begin{lemma}
  \label{lem:invariance_Reidemeister_I}
  Let $D$ be a marked tangled web.
  In the relative homotopy category $\cK^\fg(\fg\foam^{\greenmarking})$, the object $\fg\glKh(D)$ is invariant under Reidemeister I moves, up to isomorphism.
\end{lemma}

\begin{proof}
  We can proceed locally.
  We must check that, in the relative homotopy category:
  \begin{gather*}
    \def\webxscl{.4}\def\webyscl{.3}
    \tikzpic{
      \draw[web1] (0,2) to (1,2);\webncr[1][1]\draw[web1] (2,2) to (3,2);
      \webm[0][0]\draw[web1] (1,0) to (2,0);\websRoundMarkedB[2][0]
    }[xscale=\webxscl,yscale=\webyscl][(0,1.25*\webyscl)]
    \;\simeq^\fg\;
    \tikzpic{
      \draw[web1] (0,2) to (3,2);
      \draw[web2] (0,.5) to (3,.5);
    }[xscale=\webxscl,yscale=\webyscl]
    \langle \frac{1}{2},-\frac{1}{2} \rangle
    \quad\an\quad
    \tikzpic{
      \draw[web1] (0,2) to (1,2);\webpcr[1][1]\draw[web1] (2,2) to (3,2);
      \webm[0][0]\draw[web1] (1,0) to (2,0);\websRoundMarkedB[2][0]
    }[xscale=\webxscl,yscale=\webyscl][(0,1.25*\webyscl)]
    \;\simeq^\fg\;
    t^{-1}
    \tikzpic{
      \draw[web1] (0,2) to (3,2);
      \draw[web2] (0,.5) to (3,.5);
    }[xscale=\webxscl,yscale=\webyscl]
    \langle -1,1 \rangle
    \;.
  \end{gather*}
  Using a split exact sequence in the spirit of \cref{lem:web_defining_relations_equivariance}, we can fit each left-hand side in a sequence which is split exact at the two middle chain complexes (we omit labelling the arrows in the second case):
  \begin{center}
    \begin{tikzpicture}
      \def\vspc{2}\def\hspc{4.5}
      \def\webxscl{.4}\def\webyscl{.3}
      \node (A2) at (\hspc,\vspc) {
        \tikzpic{
          \draw[web1] (0,1.5) to (3,1.5);
          \draw[web2] (0,0) to (3,0);
        }[xscale=\webxscl,yscale=\webyscl]
      };
      \node[left={-8pt} of A2] {\footnotesize $t^{-1}$};
      \node (A3) at (2*\hspc,\vspc) {
        \tikzpic{
          \draw[web1] (0,1.5) to (3,1.5);
          \draw[web2] (0,0) to (3,0);
        }[xscale=\webxscl,yscale=\webyscl]
      };
      \node[right={-8pt} of A3] {\footnotesize $\langle -\frac{1}{2},\frac{1}{2} \rangle$};
      %
      \node (B2) at (\hspc,0) {
        \tikzpic{
          \websRoundMarkedB[1][1]\webm[2][1]
          \webm[0][0]\websRoundMarkedB[3][0]
          \draw[web1] (0,2) to (1,2);
          \draw[web1] (3,2) to (4,2);
          \draw[web1] (1,0) to (3,0);
        }[xscale=\webxscl,yscale=\webyscl]
      };
      \node[left={-8pt} of B2] {\footnotesize $t^{-1}$};
      \node (B3) at (2*\hspc,0) {
        \tikzpic{
          \draw[web1] (0,2) to (1,2);\webid[1][1][1]\draw[web1] (2,2) to (3,2);
          \webm[0][0]\draw[web1] (1,0) to (2,0);\websRoundMarkedB[2][0]
        }[xscale=\webxscl,yscale=\webyscl]
      };
      %
      \node (C3) at (2*\hspc,-\vspc) {
        \tikzpic{
          \draw[web1] (0,1.5) to (3,1.5);
          \draw[web2] (0,0) to (3,0);
        }[xscale=\webxscl,yscale=\webyscl]
      };
      \node[right={-8pt} of C3] {\footnotesize $\langle \frac{1}{2},-\frac{1}{2} \rangle$};
      \node (D3) at (2*\hspc,-2*\vspc) {$0$};
      %
      % EDGES
      \draw[->] (A2) to (A3);
      \node[above] at (1.6*\hspc,\vspc) {\tiny $\id$};
      \draw[->] (B2) to (B3);
      \node[above] at (1.6*\hspc,0) {\tiny 
        \tikzpic{
          \flst[-.5][0]\funzip[0][0][2]\frst[1.5][0]
        }[scale=.3]
      };
      \draw[->] ($(B2.north)+(.1,0)$) to ($(A2.south)+(.1,0)$);
      \draw[<-,dashed] ($(B2.north)-(.1,0)$) to ($(A2.south)-(.1,0)$);
      \node[right] at (\hspc,.5*\vspc) {
        \tikzpic{
          \clip (-1,0) rectangle (2,2);
          \funzip[0][0][2]
          \begin{scope}[scale=2]
            \fcap[-.25][0]
          \end{scope}
        }[scale=.2]
      };
      \node[left] at (\hspc,.5*\vspc) {\tiny $Z^{-1}
        \tikzpic{
          \clip (-1,0) rectangle (2,2);
          \funzip[0][0][2]
          \begin{scope}[scale=2]
            \fcap[-.25][0]
          \end{scope}
        }[scale=.2,yscale=-1]\;$
      };
      \draw[->] ($(B3.north)+(.1,0)$) to ($(A3.south)+(.1,0)$);
      \draw[<-,dashed] ($(B3.north)-(.1,0)$) to ($(A3.south)-(.1,0)$);
      \node[right] at (2*\hspc,.5*\vspc) {
        \tikzpic{\fcap}[scale=.3]
      };
      \node[left] at (2*\hspc,.5*\vspc) {\tiny $
        \tikzpic{\fcup\fdot[.5][.8][2][1.5]}[scale=.3]
        +XY
        \tikzpic{\fcup\fdot[.5][.8][3][1.5]}[scale=.3]\;$
      };
      \draw[->] ($(C3.north)+(.1,0)$) to ($(B3.south)+(.1,0)$);
      \draw[<-,dashed] ($(C3.north)-(.1,0)$) to ($(B3.south)-(.1,0)$);
      \node[right] at (2*\hspc,-.5*\vspc) {
        \tikzpic{\fcup}[scale=.3]
      };
      \node[left] at (2*\hspc,-.5*\vspc) {\tiny $
        \tikzpic{\fcap\fdot[.5][.2][2][1.5]}[scale=.3]
        -
        \tikzpic{\fcap\fdot[.5][.2][3][1.5]}[scale=.3]\;$
      };
      \draw[->] (D3) to (C3);
      %
      \node[right={-8pt} of B2,fill=white,inner sep=1pt] {\footnotesize $\langle -\frac{1}{2},\frac{1}{2} \rangle$};
      \node[right={-8pt} of A2,fill=white,inner sep=1pt] {\footnotesize $\langle -\frac{1}{2},\frac{1}{2} \rangle$};
    \end{tikzpicture}
    %
    \hspace{1cm}
    %
    \begin{tikzpicture}
      \def\vspc{2}\def\hspc{3.5}
      \def\webxscl{.4}\def\webyscl{.3}
      \node (A2) at (\hspc,\vspc) {
        \tikzpic{
          \draw[web1] (0,1.5) to (3,1.5);
          \draw[web2] (0,0) to (3,0);
        }[xscale=\webxscl,yscale=\webyscl]
      };
      \node[left={-8pt} of A2] {\footnotesize $t^{-1}$};
      \node (A3) at (2*\hspc,\vspc) {
        \tikzpic{
          \draw[web1] (0,1.5) to (3,1.5);
          \draw[web2] (0,0) to (3,0);
        }[xscale=\webxscl,yscale=\webyscl]
      };
      %
      \node (B2) at (\hspc,0) {
        \tikzpic{
          \draw[web1] (0,2) to (1,2);\webid[1][1][1]\draw[web1] (2,2) to (3,2);
          \webm[0][0]\draw[web1] (1,0) to (2,0);\websRoundMarkedB[2][0]
        }[xscale=\webxscl,yscale=\webyscl]
      };
      \node[left={-8pt} of B2] {\footnotesize $t^{-1}$};
      \node (B3) at (2*\hspc,0) {
        \tikzpic{
          \websRoundMarkedB[1][1]\webm[2][1]
          \webm[0][0]\websRoundMarkedB[3][0]
          \draw[web1] (0,2) to (1,2);
          \draw[web1] (3,2) to (4,2);
          \draw[web1] (1,0) to (3,0);
        }[xscale=\webxscl,yscale=\webyscl]
      };
      %
      \node (C2) at (\hspc,-\vspc) {
        \tikzpic{
          \draw[web1] (0,1.5) to (3,1.5);
          \draw[web2] (0,0) to (3,0);
        }[xscale=\webxscl,yscale=\webyscl]
      };
      \node[left={-8pt} of C2] {\footnotesize $t^{-1}$};
      \node[right={-8pt} of C2] {\footnotesize $\langle -1,1\rangle$};
      \node (D2) at (\hspc,-2*\vspc) {$0$};
      %
      \draw[->] (A2) to (A3);
      \draw[->] (B2) to (B3);
      \draw[->] (A2) to (B2);
      \draw[->] (B2) to (C2);
      \draw[->] (C2) to (D2);
      \draw[->] (A3) to (B3);
      %
      \node[right={-8pt} of B2,fill=white,inner sep=1pt] {\footnotesize $\langle -\frac{1}{2},\frac{1}{2} \rangle$};
    \end{tikzpicture}
  \end{center}
  Colours $\textdot[1]$, $\textdot[2]$ and $\textdot[3]$ are labels $1$, $2$ and $3$, respectively.
  The top chain complex is the cone of an identity while the bottom chain complex is zero: we are in the situation of \cref{lem:abstract_homotopy_equivalence}.
  Finally, the middle chain morphism is $\fg$-equivariant, so that it defines a $\fg$-equivariant homotopy equivalence.
\end{proof}

\begin{lemma}
  \label{lem:invariance_Reidemeister_II}
  Let $D$ be a marked tangled web.
  In the relative homotopy category $\cK^\fg(\fg\foam^{\greenmarking})$, the object $\fg\glKh(D)$ is invariant under Reidemeister II moves, up to isomorphism.
\end{lemma}

\begin{proof}
  We can proceed locally.
  We must check that, in the relative homotopy category:
  \begin{gather*}
    \tikzpic{\webpcr[0][0]\webncr[1][0]}[scale=.4]
    \simeq^{\fg}
    t^{-1}
    \tikzpic{\webid}[scale=.4]
    \langle -\frac{1}{2},\frac{1}{2}\rangle
    \simeq^{\fg}
    \tikzpic{\webncr[0][0]\webpcr[1][0]}[scale=.4]
    \;.
  \end{gather*}
  We focus on the first isomorphism, the other one being the same up to reordering direct sums.
  Using a split exact sequence in the spirit of \cref{lem:web_defining_relations_equivariance}, we can fit the left-hand side in a sequence which is split exact at the two middle chain complexes:
  \begin{center}
    \begin{tikzpicture}
      \def\vspc{2.5}\def\hspc{6}
      \def\webxscl{.4}\def\webyscl{.3}
      %
      % VERTICES
      \node (Z2) at (\hspc,2*\vspc) {$0$};
      %
      \node (A1) at (0,\vspc) {$0$};
      \node (A2) at (\hspc,\vspc) {
        $\tikzpic{\webid}[xscale=\webxscl,yscale=\webyscl][(0,.5*\webyscl)]
        \;\langle -\frac{1}{2},\frac{1}{2} \rangle$
      };
      \node (A3) at (2*\hspc,\vspc) {$0$};
      %
      \node (B1) at (0,0) {
        $\tikzpic{\websRoundMarkedB\webm[1][0]}[xscale=\webxscl,yscale=\webyscl][(0,.5*\webyscl)]
        \;\langle -1,1 \rangle$
      };
      \node (B2) at (\hspc,0) {
        $\tikzpic{\webid}[xscale=\webxscl,yscale=\webyscl][(0,.5*\webyscl)]
        \;\langle -\frac{1}{2},\frac{1}{2} \rangle
        \;\oplus\;
        \tikzpic{
          \websRoundMarkedB\webm[1][0]\websRoundMarkedB[2][0]\webm[3][0]
        }[xscale=\webxscl,yscale=\webyscl][(0,.5*\webyscl)]
        \;\langle -\frac{1}{2},\frac{1}{2} \rangle$
      };
      \node (B3) at (2*\hspc,0) {
        $\tikzpic{\websRoundMarkedB\webm[1][0]}[xscale=\webxscl,yscale=\webyscl][(0,.5*\webyscl)]$
      };
      %
      \node (C1) at (0,-\vspc) {
        $\tikzpic{\websRoundMarkedB\webm[1][0]}[xscale=\webxscl,yscale=\webyscl][(0,.5*\webyscl)]
        \;\langle -1,1 \rangle$
      };
      \node (C2) at (\hspc,-\vspc) {
        $\tikzpic{
          \websRoundMarkedB\webm[1][0]
        }[xscale=\webxscl,yscale=\webyscl][(0,.5*\webyscl)]
        \;\langle -1,1 \rangle
        \;\oplus\;
        \tikzpic{
          \websRoundMarkedB\webm[1][0]
        }[xscale=\webxscl,yscale=\webyscl][(0,.5*\webyscl)]$
      };
      \node (C3) at (2*\hspc,-\vspc) {
        $\tikzpic{
          \websRoundMarkedB\webm[1][0]
        }[xscale=\webxscl,yscale=\webyscl][(0,.5*\webyscl)]$
      };
      %
      % EDGES
      % horizontal
      \draw[->] (A1) to (A2);
      \draw[->] (A2) to (A3);
      \draw[->] (B1) to (B2);
      \node[above] at (.35*\hspc,0) {\scriptsize
        $\left(\!\begin{array}{c}
          \tikzpic{\funzip}[scale=.2] \\
          \tikzpic{\frst[0][0]\flst[1][0]\fzip[2][0]}[scale=.2]
        \end{array}\!\right)$
      };
      \draw[->] (B2) to (B3);
      \node[above] at (1.7*\hspc,0) {\scriptsize
        $\left(\!\begin{array}{cc}
          \tikzpic{\fzip}[scale=.2] &
          -Z\;\tikzpic{\funzip[0][0]\frst[2][0]\flst[3][0]}[scale=.2] 
        \end{array}\!\right)$
      };
      \draw[->] (C1) to (C2);
      \node[above] at (.5*\hspc,-\vspc) {$
      \left(\!\begin{array}{c}
        1 \\ 0
      \end{array}\!\right)$};
      \draw[->] (C2) to (C3);
      \node[above] at (1.5*\hspc,-\vspc) {$
      \left(\!\begin{array}{cc}
        0 & 1
      \end{array}\!\right)$};
      %
      % vertical
      \draw[<-] (Z2) to (A2);
      %
      \draw[<-] ($(A2.south)+(.1,0)$) to ($(B2.north)+(.1,0)$);
      \draw[->,dashed] ($(A2.south)-(.1,0)$) to ($(B2.north)-(.1,0)$);
      \node[right] at (1*\hspc,.5*\vspc) {\scriptsize
        $\left(\begin{array}{cc}
          1 & -Z^{-3}\;\tikzpic{
          \fcap[0][0][1]
          \draw[foamdraw1] (-.5,0) to (-.5,1);
          \draw[foamdraw1] (1.5,0) to (1.5,1);
          \begin{scope}[scale=2]
            \funzip[-.25][.5]
          \end{scope}
          }[scale=.2]
        \end{array}\right)$
      };
      \node[left] at (1*\hspc,.5*\vspc) {\scriptsize
        $\left(\begin{array}{c}
          1\\
          \tikzpic{
          \fcap[0][0][1]
          \draw[foamdraw1] (-.5,0) to (-.5,1);
          \draw[foamdraw1] (1.5,0) to (1.5,1);
          \begin{scope}[scale=2]
            \funzip[-.25][.5]
          \end{scope}
          }[scale=.2,yscale=-1]
        \end{array}\right)$
      };
      %
      \draw[<-] ($(B1.south)+(.1,0)$) to ($(C1.north)+(.1,0)$);
      \draw[->,dashed] ($(B1.south)-(.1,0)$) to ($(C1.north)-(.1,0)$);
      \node[right] at (0,-.5*\vspc) {$\id$};
      \node[left] at (0,-.5*\vspc) {$\id$};
      %
      \draw[<-] ($(B2.south)+(.1,0)$) to ($(C2.north)+(.1,0)$);
      \draw[->,dashed] ($(B2.south)-(.1,0)$) to ($(C2.north)-(.1,0)$);
      \node[right] at (1*\hspc,-.5*\vspc) {\scriptsize
        $\left(\!\begin{array}{cc}
          \tikzpic{\funzip}[scale=.2] & 0 \\
          \tikzpic{\frst[0][0]\flst[1][0]\fzip[2][0]}[scale=.2] 
          & -Z^{-3}\;\tikzpic{
            \fcap[0][0][1]
            \draw[foamdraw1] (-.5,0) to (-.5,1);
            \draw[foamdraw1] (1.5,0) to (1.5,1);
          }[scale=.2,yscale=-1]
        \end{array}\!\right)$
      };
      \node[left] at (1*\hspc,-.5*\vspc) {\scriptsize
        $\left(\!\begin{array}{cc}
          0 & \tikzpic{
            \fcap[0][0][1]
            \draw[foamdraw1] (-.5,0) to (-.5,1);
            \draw[foamdraw1] (1.5,0) to (1.5,1);
          }[scale=.2]
          \\
          \tikzpic{\fzip}[scale=.2]
          &
          -Z\;\tikzpic{\funzip[0][0]\frst[2][0]\flst[3][0]}[scale=.2] 
        \end{array}\!\right)$
      };
      %
      \draw[<-] ($(B3.south)+(.1,0)$) to ($(C3.north)+(.1,0)$);
      \draw[->,dashed] ($(B3.south)-(.1,0)$) to ($(C3.north)-(.1,0)$);
      \node[right] at (2*\hspc,-.5*\vspc) {$\id$};
      \node[left] at (2*\hspc,-.5*\vspc) {$\id$};
    \end{tikzpicture}
  \end{center}
  % \ls{split thanks to super 4tu (see notes)}
  We omitted the homological degree: the middle column is in homological degree zero.
  The top chain complex is zero while the bottom chain complex is the cone of an identity: we are in the situation of \cref{lem:abstract_homotopy_equivalence}.
  Moreover, the middle chain morphism is $\fg$-equivariant thanks to \cref{lem:crossing_complex_equivariant}, so that it defines a $\fg$-equivariant homotopy equivalence.
\end{proof}

\begin{lemma}
  \label{lem:invariance_Reidemeister_III}
  Let $D$ be a marked tangled web.
  In the relative homotopy category $\cK^\fg(\fg\foam^{\greenmarking})$, the object $\fg\glKh(D)$ is invariant under Reidemeister III moves, up to isomorphism.
\end{lemma}

\begin{proof}
    The proof is an equivariant version of the proof in \cite{SV_OddKhovanovHomology_2023}, following the general strategy of Bar-Natan \cite{Bar-Natan_KhovanovsHomologyTangles_2005}.
\end{proof}




%%%%%%%%%%%%%%%%%%%%%%%%%%%%%%
%%%%%%%%%%%%%%%%%%%%%%%%%%%%%%
%%%%%%%%%%%%%%%%%%%%%%%%%%%%%%
\subsection{Marking slide}
\label{subsec:marking_slide}

In this subsection, we prove the following ``marking slide'' lemma:

\begin{lemma}[marking slide lemma]
  \label{lem:dot_slide_lemma}
  Let $\omega=(\alpha,\beta_1,\beta_2)$ be a generic local twist.
  The identity chain map induces an isomorphism in the relative homotopy category $\cK^{\gloo}(\sfoam^{\greenmarking})$:
  \begin{gather*}
    \tikzpic{
      \webncr
      \node[green_mark] (B) at (.3,.15) {};
      \node[below={-1pt} of B] {\scriptsize $\omega$};
    }[scale=.7][(0,.5*.7)]
    \simeq^{\gloo}
    \tikzpic{
      \webncr
      \node[green_mark] (B) at (1-.3,1-.15) {};
      \node[above={-1pt} of B] {\scriptsize $\omega$};
    }[scale=.7][(0,.5*.7)]
    \;.
  \end{gather*}
  If one considers $\sfoam'$ (see \cref{rem:variant_graded_gl2_foams}) instead, then the roles of the overcrossing and the undercrossing are swapped.
\end{lemma}




\begin{remark}
  \label{rem:marking_cannot_undercross}
  By considering the example 
  \begin{gather*}
  \tikzpic{
  \webpcr[0][0]\webncr[1][0]
  \webs[2][1]\webs[2][-1]
  \webm[-1][1]\webm[-1][-1]
  \draw[web1] (0,2) to (2,2);
  \draw[web1] (0,-1) to (2,-1);
  \node[green_mark] (B) at (1,-1) {};
  \node[above={-1pt} of B] {\scriptsize $-\omega$};
  \node[green_mark] (C) at (1,1) {};
  \node[above={-1pt} of C] {\scriptsize $\omega$};
  }[scale=.5]
  \;,
\end{gather*}
  one can check that indeed, the analogue statement for the other crossing does not hold:
  \begin{gather*}
    \tikzpic{
      \webpcr
      \node[green_mark] (B) at (.3,.15) {};
      \node[below={-1pt} of B] {\scriptsize $\omega$};
    }[scale=.7][(0,.5*.7)]
    \not\simeq^{\gloo}
    \tikzpic{
      \webpcr
      \node[green_mark] (B) at (1-.3,1-.15) {};
      \node[above={-1pt} of B] {\scriptsize $\omega$};
    }[scale=.7][(0,.5*.7)]
    \;,
  \end{gather*}
  and vice-versa if one considers $\sfoam'$. Indeed, if both dot slides did hold, then we would have 
  \begin{gather*}
  \tikzpic{
  %\webpcr[0][0]\webncr[1][0]
  \webs[2][1]\webs[2][-1]
  \webm[-1][1]\webm[-1][-1]
  \draw[web1] (0,2) to (2,2);
  \draw[web1] (0,-1) to (2,-1);
  \draw[web1] (0,1) to (2,1);
  \draw[web1] (0,0) to (2,0);
  }[scale=.5][(0,.5*.5)]
  \simeq^{\gloo}
  \tikzpic{
  \webs[2][1]\webs[2][-1]
  \webm[-1][1]\webm[-1][-1]
  \draw[web1] (0,2) to (2,2);
  \draw[web1] (0,-1) to (2,-1);
  \draw[web1] (0,1) to (2,1);
  \draw[web1] (0,0) to (2,0);
  \node[green_mark] (B) at (1,-1) {};
  \node[above={-1pt} of B] {\scriptsize $-2\omega$};
  \node[green_mark] (C) at (1,2) {};
  \node[above={-1pt} of C] {\scriptsize $2\omega$};
  }[scale=.5][(0,.5*.5)]
  \;.
\end{gather*}
\end{remark}

Before giving the proof, we discuss some consequences.

\begin{lemma}
  \label{lem:csq_dot_slide_lemma}
  Let $\omega=(\alpha,\beta_1,\beta_2)$ be a generic local twist.
  For each of the following cases, the identity chain map induces an isomorphism in the relative homotopy category $\cK^{\gloo}(\sfoam^{\greenmarking})$:
  \begin{gather*}
    \tikzpic{
      \webpcr
      \node[green_mark] (B) at (.3,.15) {};
      \node[below={-1pt} of B] {\scriptsize $\omega$};
    }[scale=.7][(0,.5*.7)]
    \simeq^{\gloo}
    \tikzpic{
      \webpcr
      \node[green_mark] (B) at (1-.3,1-.15) {};
      \node[above={-1pt} of B] {\scriptsize $-\omega$};
      \node[green_mark] (C) at (1-.3,.15) {};
      \node[below={-1pt} of C] {\scriptsize $2\omega$};
    }[scale=.7][(0,.5*.7)]
    %
    \;,\quad
    \tikzpic{
      \webncr
      \node[green_mark] (B) at (.3,1-.15) {};
      \node[above={-1pt} of B] {\scriptsize $\omega$};
    }[scale=.7][(0,.5*.7)]
    \simeq^{\gloo}
    \tikzpic{
      \webncr
      \node[green_mark] (B) at (1-.3,.15) {};
      \node[below={-1pt} of B] {\scriptsize $-\omega$};
      \node[green_mark] (C) at (1-.3,1-.15) {};
      \node[above={-1pt} of C] {\scriptsize $2\omega$};
    }[scale=.7][(0,.5*.7)]
    %
    \quad\an\quad
    \tikzpic{
      \webpcr
      \node[green_mark] (B) at (.3,1-.15) {};
      \node[above={-1pt} of B] {\scriptsize $\omega$};
    }[scale=.7][(0,.5*.7)]
    \simeq^{\gloo}
    \tikzpic{
      \webpcr
      \node[green_mark] (B) at (1-.3,.15) {};
      \node[below={-1pt} of B] {\scriptsize $\omega$};
    }[scale=.7][(0,.5*.7)]
    \;.
  \end{gather*}
  If one considers $\sfoam'$ instead, then the roles of the overcrossing and the undercrossing are swapped.
\end{lemma}

\begin{proof}
  The follows from \cref{lem:dot_slide_lemma} using invariance under planar isotopies \cref{lem:invariance_planar_isotopies} and the fact that in general, the identity chain map induces an isomorphism in $\cK^\gloo(\sfoam)$:
  \begin{gather*}
    \tikzpic{
      \webncr
      \node[green_mark] (B) at (.3,1-.15) {};
      \node[above={-1pt} of B] {\scriptsize $-\omega$};
      \node[green_mark] (B) at (.3,.15) {};
      \node[below={-1pt} of B] {\scriptsize $\omega$};
    }[scale=.7][(0,.5*.7)]
    \simeq^\fg
    \tikzpic{
      \webncr
      \node[green_mark] (B) at (1-.3,.15) {};
      \node[below={-1pt} of B] {\scriptsize $\omega$};
      \node[green_mark] (B) at (1-.3,1-.15) {};
      \node[above={-1pt} of B] {\scriptsize $-\omega$};
    }[scale=.7][(0,.5*.7)]
  \end{gather*}
  This concludes.
\end{proof}

The remainder of this subsection is devoted to the proof of \cref{lem:dot_slide_lemma}.
The proof is inspired by the proof of the analogue result in \cite[lemma~4.4]{QRS+_SymmetriesEquivariantKhovanovRozansky_2023}; see also \cite[Lemma~5.2]{Roz_$mathfraksl_2$ActionLink_2023}. Their proof originates from \cite{KR_PositiveHalfWitt_2016}.

We begin with an outline.
Write $D_\bullet$ for the complex associated to the crossing, and ${}_\omega D_\bullet$ (resp.\ $D_\bullet^\omega$) for the complex with the additional $\omega$-twist at the top left (resp.\ bottom right).
We aim to show that ${}_\omega D_\bullet\simeq^\fg D_\bullet^\omega$.
The main idea is to add a circle to the web, and move the $\omega$-twist to that circle. More formally, we define in \cref{lem:web_resolution_twist} a partial resolution of a generic $\omega$-twist via $\omega$-twisted circles.
Applying this to ${}_\omega D_\bullet$ gives another complex ${}_\omega C_\bullet$ together with a $\gloo$-equivariant homotopy equivalence ${}_\omega C_\bullet\to {}_\omega D_\bullet$; similarly, we find a $\gloo$-equivariant homotopy equivalence $C_\bullet^\omega\to D_\bullet^\omega$.
We are then able to give an $\gloo$-equivariant isomorphism between ${}_\omega C_\bullet$ and $C_\bullet^\omega$.
This leads to a zigzag of $\gloo$-equivariant homotopies between ${}_\omega D_\bullet$ and $D_\bullet^\omega$, and hence an isomorphism in the relative homotopy category.

\medbreak

We now give the details, beginning with some preliminary definitions.
For a web $W$, we shall write $\Phi(W)$ the web obtained by extending $W$ forward with a marked circle, as shown below.
\begin{gather*}
  \Phi(W)
  \coloneqq\quad
  \tikzpic{
    \node[anchor=mid,align=center] at (.4,1.7) {\footnotesize $t(W)$};
    \node[anchor=mid,align=center] at (1.5,1.7) {$W$};
    \node[anchor=mid,align=center] at (2.6,1.7) {\footnotesize $s(W)$};
    \draw[web1,dashed] (1,0) to (2,0);
    \draw[web1,dashed] (1,1) to (2,1);
    \webmRoundMarkedB
    \webs[2][0]
    %
    \draw[dotted] (1,-.3) to (1,2);
    \draw[dotted] (2,-.3) to (2,2);
  }
  \qquad\an\qquad
  \Phi_\omega(W)
  \coloneqq\quad
  \tikzpic{
    \node[anchor=mid,align=center] at (.4,1.7) {\footnotesize $t(W)$};
    \node[anchor=mid,align=center] at (1.5,1.7) {$W$};
    \node[anchor=mid,align=center] at (2.6,1.7) {\footnotesize $s(W)$};
    \draw[web1,dashed] (1,0) to (2,0);
    \draw[web1,dashed] (1,1) to (2,1);
    \webmRoundMarkedB
    \websMarkedB[2][0][$\omega$]
    %
    \draw[dotted] (1,-.3) to (1,2);
    \draw[dotted] (2,-.3) to (2,2);
  }\;.
\end{gather*}
More formally, we define a family of superfunctors $\Phi\colon\sfoam^{\greenmarking}(n,m)\to\sfoam^{\greenmarking}(n,m)$, defined on objects as
\[\Phi(W)\coloneqq(M^{\roundmarking}\mathrel{\square}\id_W)\mathrel{\square} (\id_{(1,1)}\otimes W)\otimes (S\mathrel{\square}\id_W),\]
where $M^{\roundmarking}$ is a merge web with an extra marking $\roundmarking[2]$ and $S$ is a split web, and $\square$ is the front-back composition from \cref{rem:monoidal-2-categorical-structure}.
We also define the variant $\Phi_\omega(W)$ where the circle carries an extra local marking $\omega$.



Consider an identity web $W\in\sfoam^{\greenmarking}$ with a distinguished strand $i$:
\begin{gather*}
  W = \tikzpic{
    \node at (1,.8){};
    \node at (1,-.8){};
    \node at (1,.5) {$\vdots$};
    \draw[web1] (0,0) to (2,0);
    \node at (1,-.3) {$\vdots$};
    \node[right] at (2,0) {\footnotesize $i$};
  }
\end{gather*}
Below we consider $\Phi(W)$ and $\Phi_\omega(W)$, using the color blue ($\textdot$) for the label of the added circle and the color red ($\textdot[2]$) for the label of the distinguished strand $i$ in $W$.

\begin{lemma}
  \label{lem:web_resolution_twist}
  Let $\omega=(\alpha,\beta_1,\beta_2)$ be a generic local  marking.
  The following is a sequence in $\sfoam^{\greenmarking}$, split exact at the two middle vertices, and with each forward (plain) arrow being $\gloo$-equivariant:
  \begin{center} 
  \begin{tikzpicture}
    \def\hsh{3}
    \def\wxscl{.5}
    \def\wyscl{.35}
    % \node (L) at (-\hsh+1*\wxscl,.5) {$\ldots$};
    % \node at (-\hsh+3*\wxscl,1.5*\wyscl) {\footnotesize $\langle -3,3\rangle$};
    % %
    % \begin{scope}[xscale=\wxscl,yscale=\wyscl]
    %   \node[shape=rectangle,anchor=center] at (1,3.7) {$\vdots$};
    %   \draw[web1] (0,2.5) to (2,2.5);
    %   \node[shape=rectangle,anchor=center] at (1,2) {$\vdots$};
    %   \webmRoundMarkedB
    %   \websMarkedB[1][0][$\omega$]
    %   \node at (3,1.5) {\footnotesize $\langle -2,2\rangle$};
    % \end{scope}
    %
    \begin{scope}[shift={(\hsh,0)},xscale=\wxscl,yscale=\wyscl]
      \node[shape=rectangle,anchor=center] at (1,3.7) {$\vdots$};
      \draw[web1] (0,2.5) to (2,2.5);
      \node[shape=rectangle,anchor=center] at (1,2) {$\vdots$};
      \webmRoundMarkedB
      \websMarkedB[1][0][$\omega$]
      \node at (3,1.5) {\footnotesize $\langle -\frac{3}{2},\frac{3}{2}\rangle$};
    \end{scope}
    %
    \begin{scope}[shift={(2*\hsh,0)},xscale=\wxscl,yscale=\wyscl]
      \node[shape=rectangle,anchor=center] at (1,3.7) {$\vdots$};
      \draw[web1] (0,2.5) to (2,2.5);
      \node[shape=rectangle,anchor=center] at (1,2) {$\vdots$};
      \webmRoundMarkedB
      \websMarkedB[1][0][$\omega$]
      \node at (3,1.5) {\footnotesize $\langle -\frac{1}{2},\frac{1}{2}\rangle$};
    \end{scope}
    %
    \begin{scope}[shift={(3*\hsh,0)},xscale=\wxscl,yscale=\wyscl]
      \node[shape=rectangle,anchor=center] at (1,3.7) {$\vdots$};
      \draw[web1] (0,2.5) to (2,2.5);
      \node[green_mark=2.5pt] (M) at (1,2.5) {};
      \node[below right={-1pt} of M] {\scriptsize $\omega$};
      \node[shape=rectangle,anchor=center] at (1,1.8) {$\vdots$};
      \draw[web2] (0,.5) to (2,.5);
    \end{scope}
    %
    \node (R) at (3.7*\hsh,.5) {$0$};
    % Arrow
    % \draw[->,thick] (-\hsh+2*\wxscl,1.5) to[out=45,in=135] (0,1.5);
    % \draw[->,thick] (2*\wxscl,1.5) to[out=45,in=135] (\hsh,1.5);
    \draw[->,thick] (\hsh+2*\wxscl,1.5) to[out=45,in=135] (2*\hsh,1.5);
    \draw[->,thick] (2*\hsh+2*\wxscl,1.5) to[out=45,in=135] (3*\hsh,1.5);
    \draw[->,thick] (3*\hsh+2*\wxscl+.2,.5) -- (R);
    %
    % \node[above] at (-.5*\hsh+\wxscl,2) {%
    %   $\tikzpic{
    %     \flst\frst[1][0]\fdot[.5][.5]
    %   }[scale=.5,xscale=.6]
    %   -
    %   \tikzpic{
    %     \flst\frst[1][0]\fdot[.5][.5][2]
    %   }[scale=.5,xscale=.6]
    %   $};
    % \node[above] at (.5*\hsh+\wxscl,2) {%
    %   $\tikzpic{
    %     \flst\frst[1][0]\fdot[.5][.5]
    %   }[scale=.5,xscale=.6]
    %   -
    %   \tikzpic{
    %     \flst\frst[1][0]\fdot[.5][.5][2]
    %   }[scale=.5,xscale=.6]
    %   $};
    \node[above] at (1.5*\hsh+\wxscl,2) {%
      $\tikzpic{
        \flst\frst[1][0]\fdot[.5][.5]
      }[scale=.5,xscale=.6]
      -
      \tikzpic{
        \flst\frst[1][0]\fdot[.5][.5][2]
      }[scale=.5,xscale=.6]
      $};
    \node[above] at (2.5*\hsh+\wxscl,2) {
      $\tikzpic{
        \fcap\fdot[.5][.3]
      }[scale=.5,xscale=.6]
      -
      \tikzpic{
        \fcap\fdot[.5][.3][2]
      }[scale=.5,xscale=.6]
      $};
    %
    % \draw[<-,thick,dashed] (-\hsh+2*\wxscl,-.5) to[out=-45,in=-135] (0,-.5);
    % \draw[<-,thick,dashed] (2*\wxscl,-.5) to[out=-45,in=-135] (\hsh,-.5);
    \draw[<-,thick,dashed] (\hsh+2*\wxscl,-.5) to[out=-45,in=-135] (2*\hsh,-.5);
    \draw[<-,thick,dashed] (2*\hsh+2*\wxscl,-.5) to[out=-45,in=-135] (3*\hsh,-.5);
    %
    % \node[below=15pt] at (-.5*\hsh+\wxscl,-.5) {%
    %   $\tikzpic{
    %     \fcap\fcup[0][1]
    %   }[scale=.3,xscale=1]
    %   $};
    % \node[below=15pt] at (.5*\hsh+\wxscl,-.5) {%
    %   $\tikzpic{
    %     \fcap\fcup[0][1]
    %   }[scale=.3,xscale=1]
    %   $};
    \node[below=15pt] at (1.5*\hsh+\wxscl,-.5) {%
      $\tikzpic{
        \fcap\fcup[0][1]
      }[scale=.3,xscale=1]
      $};
    \node[below=15pt] at (2.5*\hsh+\wxscl,-.5) {%
      $\tikzpic{
        \fcup
      }[scale=.3,xscale=1]
      $};
  \end{tikzpicture}
  \end{center}
\end{lemma}

\begin{proof}
  The fact that the sequence split is a direct computation.
  Equivariance with respect to $\lieh_1$ and $\lieh_2$ can be checked using \cref{lem:h-equivariance}.
  Equivariance with respect to $\liee$ and $\lief$ follows from
  \cref{lem:crossing_complex_equivariant} and respectively \cref{lem:e-equivariance} and \cref{lem:f-equivariance}.
\end{proof}

Using this partial resolution, we construct a partial resolution in $\underline{\Ch}(\sfoam^{\greenmarking})$ of ${}_\omega D_\bullet$, as pictured in \cref{fig:proof_dot_slide}.
The fact that the complexes are $\gloo$-equivariant was checked already in \cref{lem:web_resolution_twist}. Note that up to scalar, we have ${}_\omega C_\bullet\cong\Cone(F)$ for $F\colon\Phi_\omega(C_\bullet)\to\Phi_\omega(C_\bullet)$ a $\gloo$-equivariant chain map consisting of dots.
Note moreover that $P_\bullet=\Cone(\id_{\Phi_\omega(D_\bullet)}\langle-\frac{3}{2},\frac{3}{2}\rangle)$. In particular, the complex $P_\bullet$ is contractible.
We are in the situation of \cref{lem:abstract_homotopy_equivalence}, and conclude that the chain map ${}_\omega C_\bullet\to{}_\omega D_\bullet$ is a ($\gloo$-equivariant) homotopy equivalence.

Use the colour brown ($\textdot[3]$) for  the label of the backmost strand amongst to two strands involved in $D_\bullet$.
The very same argument applies to $D_\bullet^\omega$, only replacing red dots ($\textdot[{2}]$) with brown dots ($\textdot[{3}]$), and swapping the dots from left to right in the two vertical arrows in the middle of the diagram.
We get a ($\gloo$-equivariant) homotopy equivalence $C_\bullet^\omega\to D_\bullet^\omega$.

\begin{figure}[p]
  \begin{tikzpicture}
    \def\hsh{5}\def\vsh{5}\def\hoplussh{.5}\def\voplussh{1}
    \def\wxscl{.35}\def\wyscl{.35}
    %%%%%%%%%%%%%%%%%%%%%%%%%%%%%%
    % SOME DIAGRAMS
    \def\tempwebA{
      \webid[0][1.5][1]\websRoundMarkedB[1][1.5]\webm[2][1.5]\webid[3][1.5][1]
      \webmRoundMarkedB\webid[1][0][1]\webid[2][0][1]\websMarkedB[3][0][$\omega$]
    }
    \def\tempwebB{
      \webid[0][1.5][1]\webid[1][1.5]\webid[3][1.5][1]
      \webmRoundMarkedB\webid[1][0][1]\webid[2][0][1]\websMarkedB[3][0][$\omega$]
    }
    \newcommand{\tempfoamid}{
    \tikzpic{
        \flst\frst[1][0]\fdot[.5][.5]
      }[scale=.7,xscale=.5]
      -
      \tikzpic{
        \flst\frst[1][0]\fdot[.5][.5][2]
      }[scale=.7,xscale=.5]
    }
    \newcommand{\tempfoamdbid}{
      \tikzpic{
        \draw[fill,foamshade2] (1-.3,0) rectangle (1+.3,2);
        \draw[foamdraw2] (1-.3,0) to (1-.3,2);
        \draw[foamdraw2] (1+.3,0) to (1+.3,2);
        %
        \draw[fill,foamshade1] (-.5,0) rectangle (0,2);
        \draw[foamdraw1] (0,0) to (0,2);
        \draw[fill,foamshade1] (2.5,0) rectangle (2,2);
        \draw[foamdraw1] (2,0) to (2,2);
        %
        \fdot[.4][1][1]
      }[scale=.35]
      -
      \tikzpic{
        \draw[fill,foamshade2] (1-.3,0) rectangle (1+.3,2);
        \draw[foamdraw2] (1-.3,0) to (1-.3,2);
        \draw[foamdraw2] (1+.3,0) to (1+.3,2);
        %
        \draw[fill,foamshade1] (-.5,0) rectangle (0,2);
        \draw[foamdraw1] (0,0) to (0,2);
        \draw[fill,foamshade1] (2.5,0) rectangle (2,2);
        \draw[foamdraw1] (2,0) to (2,2);
        %
        \fdot[.4][1][2]
      }[scale=.35]
    }
    \newcommand{\tempfoamdbunzip}{
      \tikzpic{
        \funzip[.5][0][2]
        %
        \draw[fill,foamshade1] (-.5,0) rectangle (0,2);
        \draw[foamdraw1] (0,0) to (0,2);
        \draw[fill,foamshade1] (2.5,0) rectangle (2,2);
        \draw[foamdraw1] (2,0) to (2,2);
      }[scale=.35]
    }
    %%%%%%%%%%%%%%%%%%%%%%%%%%%%%%
    % VERTICES
    %
    % top layer
    \node[] (T0) at (\hsh-\hoplussh,\vsh) {$\tikzpic{
      \webid[0][1.5][1]\websRoundMarkedB[1][1.5]\webm[2][1.5]\webid[3][1.5][1]
      \node[green_mark] (M) at (.5,1.5) {};
      \node[below={-3pt} of M] {\scriptsize $\omega$};
      \draw[web2] (0,.5) to (4,.5);
    }[xscale=\wxscl,yscale=\wyscl]$};
    \node[] (R0) at (2*\hsh,\vsh) {$\tikzpic{
      \webid[0][1.5][4]
      \node[green_mark] (M) at (.5,1.5) {};
      \node[below={-3pt} of M] {\scriptsize $\omega$};
      \draw[web2] (0,.5) to (4,.5);
    }[xscale=\wxscl,yscale=\wyscl]$};
    %
    % middle layer
    \node[] (L1) at (0,0) {$\tikzpic{\tempwebA}[xscale=\wxscl,yscale=\wyscl]$};
    \node[] (T1) at (\hsh-\hoplussh,\voplussh) {$\tikzpic{\tempwebA}[xscale=\wxscl,yscale=\wyscl]$};
    \node[anchor=mid] at (\hsh,.1) {$\bigoplus$};
    \node[] (B1) at (\hsh+\hoplussh,-\voplussh) {$\tikzpic{\tempwebB}[xscale=\wxscl,yscale=\wyscl]$};
    \node[] (R1) at (2*\hsh,0) {$\tikzpic{\tempwebB}[xscale=\wxscl,yscale=\wyscl]$};
    %
    % bottom layer
    \node[] (L2) at (0,-\vsh) {$\tikzpic{\tempwebA}[xscale=\wxscl,yscale=\wyscl]$};
    \node[] (T2) at (\hsh-\hoplussh,\voplussh-\vsh) {$\tikzpic{\tempwebA}[xscale=\wxscl,yscale=\wyscl]$};
    \node[anchor=mid] at (\hsh,.1-\vsh) {$\bigoplus$};
    \node[] (B2) at (\hsh+\hoplussh,-\voplussh-\vsh) {$\tikzpic{\tempwebB}[xscale=\wxscl,yscale=\wyscl]$};
    \node[] (R2) at (2*\hsh,-\vsh) {$\tikzpic{\tempwebB}[xscale=\wxscl,yscale=\wyscl]$};
    %
    %%%%%%%%%%%%%%%%%%%%%%%%%%%%%%
    % EDGES
    % top layer
    \draw[->] (T0) to[out=0,in=180] (R0);
    \node[above] at (1.5*\hsh-.5*\hoplussh,\vsh) {\tikzpic{\funzip[0][0][2]}[scale=.7]};
    %
    % mid-top
    \draw[->] (T1) to (T0);
    \draw[->] (R1) to (R0);
    \node[left=15pt] at (\hsh,.6*\vsh) {$
      \tikzpic{
        \clip (-.5,0) rectangle (2.5,2);
        \begin{scope}[scale=2]
          \fcap
        \end{scope}
        \draw[fill,foamshade2] (1-.3,0) rectangle (1+.3,2);
        \draw[foamdraw2] (1-.3,0) to (1-.3,2);
        \draw[foamdraw2] (1+.3,0) to (1+.3,2);
        \fdot[.4][.5][1]
      }[xscale=1,scale=.4]
      -
      \tikzpic{
        \clip (-.5,0) rectangle (2.5,2);
        \begin{scope}[scale=2]
          \fcap
        \end{scope}
        \draw[fill,foamshade2] (1-.3,0) rectangle (1+.3,2);
        \draw[foamdraw2] (1-.3,0) to (1-.3,2);
        \draw[foamdraw2] (1+.3,0) to (1+.3,2);
        \fdot[.4][.5][2]
      }[xscale=1,scale=.4]
    $};
    \node[right] at (2*\hsh,.5*\vsh) {$
      \tikzpic{
        \fcap\fdot[.5][.3]
      }[scale=.7,xscale=.5]
      -
      \tikzpic{
        \fcap\fdot[.5][.3][2]
      }[scale=.7,xscale=.5]
    $};
    %
    % T2 to T1
    \draw[->] ($(T2.north)+(-.2,0)$) to ($(T1.south)+(-.2,0)$);
    %
    % middle layer
    \draw[->] (L1) to[out=20,in=180] (T1);
    \draw[->] (T1) to[out=0,in=160] (R1);
    \draw[->] (B1) to[out=0,in=-160] (R1);
    \draw[{preaction={draw,white,line width=5pt}},->] (L1) to[out=-20,in=180] (B1);
    %
    \node[fill=white,inner sep=1pt] at (.4*\hsh-.5*\hoplussh,.25*\vsh) {$
      \tempfoamdbid$};
    \node[fill=white,inner sep=1pt] at (.4*\hsh+.5*\hoplussh,-.2*\vsh) {$-\;
      \tempfoamdbunzip$};
    \node[fill=white,inner sep=1pt] at (1.5*\hsh-.5*\hoplussh,.25*\vsh) {$-\;
      \tempfoamdbunzip$};
    \node[fill=white,inner sep=1pt] at (1.6*\hsh+.5*\hoplussh,-.2*\vsh) {$\tempfoamid$};
    %
    % bot-mid
    \draw[->] (L2) to (L1);
    \draw[->] (R2) to (R1);
    \node[right] at (2*\hsh,-.5*\vsh) {$\tempfoamid$};
    \node[left,fill=white,inner sep=1pt] at (1*\hsh,-.45*\vsh) {$\tempfoamdbid$};
    \node[left] at (0,-.5*\vsh) {$\id$};
    %
    % bottom layer
    \draw[->] (L2) to[out=20,in=180] (T2);
    \draw[->,{preaction={draw,white,line width=5pt}}] (L2) to[out=-20,in=180] (B2);
    \draw[->] (T2) to[out=0,in=160] (R2);
    \draw[->] (B2) to[out=0,in=-160] (R2);
    %
    % SHIFTS
    \node[right={-3pt} of T0,fill=white,inner sep=1pt] {\scriptsize $\langle -\frac{1}{2},\frac{1}{2}\rangle$};
    %
    \node[right={-9pt} of L1,fill=white,inner sep=1pt] {\scriptsize $\langle -2,2\rangle$};
    \node[right={-12pt} of T1,fill=white,inner sep=1pt] {\scriptsize $\langle -1,1\rangle$};
    \node[right={-12pt} of B1,fill=white,inner sep=1pt] {\scriptsize $\langle -\frac{3}{2},\frac{3}{2}\rangle$};
    \node[right={-9pt} of R1,fill=white,inner sep=1pt] {\scriptsize $\langle -\frac{1}{2},\frac{1}{2}\rangle$};
    %
    \node[right={-12pt} of L2] {\scriptsize $\langle -2,2\rangle$};
    \node[right={-9pt} of T2,fill=white,inner sep=1pt] {\scriptsize $\langle -2,2\rangle$};
    \node[right={-12pt} of B2,fill=white,inner sep=1pt] {\scriptsize $\langle -\frac{3}{2},\frac{3}{2}\rangle$};
    \node[right={-9pt} of R2] {\scriptsize $\langle -\frac{3}{2},\frac{3}{2}\rangle$};
    %
    \node[fill=white,inner sep=1pt] at (.4*\hsh-.5*\hoplussh,.2*\vsh-\vsh) {$
      \id$};
    \node[fill=white,inner sep=1pt] at (.4*\hsh+.5*\hoplussh,-.25*\vsh-\vsh) {$
      -\;\tempfoamdbunzip$};
    \node[fill=white,inner sep=1pt] at (1.5*\hsh-.5*\hoplussh,.25*\vsh-\vsh) {$
      \tempfoamdbunzip$};
    \node[fill=white,inner sep=1pt] at (1.6*\hsh+.5*\hoplussh,-.2*\vsh-\vsh) {$\id$};
    %
    % B2 to B1
    \draw[{preaction={draw,white,line width=5pt}},->] ($(B2.north)+(.2,0)$) to ($(B1.south)+(.2,0)$);
    \node[right=-8pt] at (\hsh+\voplussh,-.4*\vsh-\voplussh) {$\id$};
    %
    % below-bot
    % \draw[->] ($(L2.south)-(0,.3*\vsh)$) to (L2);
    % \draw[dashed] ($(L2.south)-(0,.5*\vsh)$) to[] ($(L2.south)-(0,.3*\vsh)$);
    % \draw[->] ($(B2.south)+(.2,-.3*\vsh)$) to ($(B2.south)+(.2,0)$);
    % \draw[dashed] ($(B2.south)+(.2,-.5*\vsh)$) to[] ($(B2.south)+(.2,-.3*\vsh)$);
    % \draw[dashed] ($(T2.south)+(-.2,-.5*\vsh)$) to[] ($(T2.south)+(-.2,-.3*\vsh)$);
    % \draw[->] ($(R2.south)-(0,.3*\vsh)$) to (R2);
    % \draw[dashed] ($(R2.south)-(0,.5*\vsh)$) to[] ($(R2.south)-(0,.3*\vsh)$);
    %
    %%%%%%%%%%%%%%%%%%%%%%%%%%%%%%
    % RIGHT BAR
    \node (zero) at (2.6*\hsh,1.5*\vsh) {$0$};
    \node (C) at (2.6*\hsh,\vsh) {${}_\omega D_\bullet$};
    \node (P0) at (2.6*\hsh,0) {${}_\omega C_\bullet$};
    \node (P1) at (2.6*\hsh,-\vsh) {$P_\bullet$};
    \draw[->] (C) to (zero);
    \draw[->] (P0) to (C);
    \draw[->] (P1) to (P0);
    % \draw[->] ($(P1.south)-(0,.3*\vsh)$) to (P1);
    % \draw[dashed] ($(P1.south)-(0,.5*\vsh)$) to[] ($(P1.south)-(0,.3*\vsh)$);
    \draw[dotted] (R0) to (C);
    \draw[dotted] (R1) to (P0);
    \draw[dotted] ($(R2.east)+(.7,0)$) to (P1);
  \end{tikzpicture}
  \caption{Partial resolution of the marked crossing ${}_\omega D_\bullet$. Colour blue ($\textdot[1]$) corresponds to the label of the foremost strand on the circle and colour red ($\textdot[2]$) corresponds to the label of the foremost strand amongst to two strands involved in $D_\bullet$.}
  \label{fig:proof_dot_slide}
\end{figure}

Finally, we construct a $\gloo$-equivariant isomorphism ${}_\omega C_\bullet\to C_\bullet^\omega$, as follows:
\begin{equation}
  \label{eq:proof_dot_slide_isomorphism}
  \begin{tikzpicture}
    \def\hsh{5}\def\vsh{5}\def\hoplussh{.5}\def\voplussh{1}
    \def\wxscl{.35}\def\wyscl{.35}
    %%%%%%%%%%%%%%%%%%%%%%%%%%%%%%
    % SOME DIAGRAMS
    \def\tempwebA{
      \webid[0][1.5][1]\websRoundMarkedB[1][1.5]\webm[2][1.5]\webid[3][1.5][1]
      \webmRoundMarkedB\webid[1][0][1]\webid[2][0][1]\websMarkedB[3][0][$\omega$]
    }
    \def\tempwebB{
      \webid[0][1.5][1]\webid[1][1.5]\webid[3][1.5][1]
      \webmRoundMarkedB\webid[1][0][1]\webid[2][0][1]\websMarkedB[3][0][$\omega$]
    }
    \newcommand{\tempfoamid}{
    \tikzpic{
        \flst\frst[1][0]\fdot[.5][.5]
      }[scale=.7,xscale=.5]
      -
      \tikzpic{
        \flst\frst[1][0]\fdot[.5][.5][2]
      }[scale=.7,xscale=.5]
    }
    \newcommand{\tempfoamdbid}{
      \tikzpic{
        \draw[fill,foamshade2] (1-.3,0) rectangle (1+.3,2);
        \draw[foamdraw2] (1-.3,0) to (1-.3,2);
        \draw[foamdraw2] (1+.3,0) to (1+.3,2);
        %
        \draw[fill,foamshade1] (-.5,0) rectangle (0,2);
        \draw[foamdraw1] (0,0) to (0,2);
        \draw[fill,foamshade1] (2.5,0) rectangle (2,2);
        \draw[foamdraw1] (2,0) to (2,2);
        %
        \fdot[.4][1][1]
      }[scale=.35]
      -
      \tikzpic{
        \draw[fill,foamshade2] (1-.3,0) rectangle (1+.3,2);
        \draw[foamdraw2] (1-.3,0) to (1-.3,2);
        \draw[foamdraw2] (1+.3,0) to (1+.3,2);
        %
        \draw[fill,foamshade1] (-.5,0) rectangle (0,2);
        \draw[foamdraw1] (0,0) to (0,2);
        \draw[fill,foamshade1] (2.5,0) rectangle (2,2);
        \draw[foamdraw1] (2,0) to (2,2);
        %
        \fdot[.4][1][2]
      }[scale=.35]
    }
    \newcommand{\tempfoamdbunzip}{
      \tikzpic{
        \funzip[.5][0][2]
        %
        \draw[fill,foamshade1] (-.5,0) rectangle (0,2);
        \draw[foamdraw1] (0,0) to (0,2);
        \draw[fill,foamshade1] (2.5,0) rectangle (2,2);
        \draw[foamdraw1] (2,0) to (2,2);
      }[scale=.35]
    }
    %%%%%%%%%%%%%%%%%%%%%%%%%%%%%%
    % VERTICES
    %
    % top layer
    \node[] (L1) at (0,0) {$\tikzpic{\tempwebA}[xscale=\wxscl,yscale=\wyscl]$};
    \node[] (T1) at (\hsh-\hoplussh,\voplussh) {$\tikzpic{\tempwebA}[xscale=\wxscl,yscale=\wyscl]$};
    \node[anchor=mid] at (\hsh,.1) {$\bigoplus$};
    \node[] (B1) at (\hsh+\hoplussh,-\voplussh) {$\tikzpic{\tempwebB}[xscale=\wxscl,yscale=\wyscl]$};
    \node[] (R1) at (2*\hsh,0) {$\tikzpic{\tempwebB}[xscale=\wxscl,yscale=\wyscl]$};
    %
    % bottom layer
    \node[] (L2) at (0,-\vsh) {$\tikzpic{\tempwebA}[xscale=\wxscl,yscale=\wyscl]$};
    \node[] (T2) at (\hsh-\hoplussh,\voplussh-\vsh) {$\tikzpic{\tempwebA}[xscale=\wxscl,yscale=\wyscl]$};
    \node[anchor=mid] at (\hsh,.1-\vsh) {$\bigoplus$};
    \node[] (B2) at (\hsh+\hoplussh,-\voplussh-\vsh) {$\tikzpic{\tempwebB}[xscale=\wxscl,yscale=\wyscl]$};
    \node[] (R2) at (2*\hsh,-\vsh) {$\tikzpic{\tempwebB}[xscale=\wxscl,yscale=\wyscl]$};
    %
    %%%%%%%%%%%%%%%%%%%%%%%%%%%%%%
    % EDGES
    %
    % T2 to T1
    \draw[->] ($(T2.north)+(-.3,0)$) to ($(T1.south)+(-.3,0)$);
    \node[,fill=white,inner sep=1pt] at (.78*\hsh,-.4*\vsh) {$\id$};
    %
    % top layer
    \draw[->] (L1) to[out=20,in=180] (T1);
    \draw[->] (T1) to[out=0,in=160] (R1);
    \draw[->] (B1) to[out=0,in=-160] (R1);
    \draw[{preaction={draw,white,line width=5pt}},->] (L1) to[out=-20,in=180] (B1);
    %
    \node[right={-12pt} of L1] {\scriptsize $\langle -2,2\rangle$};
    \node[right={-12pt} of T1,fill=white,inner sep=1pt] {\scriptsize $\langle -1,1\rangle$};
    \node[right={-12pt} of B1,fill=white,inner sep=1pt] {\scriptsize $\langle -\frac{3}{2},\frac{3}{2}\rangle$};
    \node[right={-12pt} of R1,fill=white,inner sep=1pt] {\scriptsize $\langle -\frac{1}{2},\frac{1}{2}\rangle$};
    %
    \node[fill=white,inner sep=1pt] at (.4*\hsh-.5*\hoplussh,.25*\vsh) {$
      \tikzpic{
        \draw[fill,foamshade2] (1-.3,0) rectangle (1+.3,2);
        \draw[foamdraw2] (1-.3,0) to (1-.3,2);
        \draw[foamdraw2] (1+.3,0) to (1+.3,2);
        %
        \draw[fill,foamshade1] (-.5,0) rectangle (0,2);
        \draw[foamdraw1] (0,0) to (0,2);
        \draw[fill,foamshade1] (2.5,0) rectangle (2,2);
        \draw[foamdraw1] (2,0) to (2,2);
        %
        \fdot[1.6][1][1]
      }[scale=.35]
      -
      \tikzpic{
        \draw[fill,foamshade2] (1-.3,0) rectangle (1+.3,2);
        \draw[foamdraw2] (1-.3,0) to (1-.3,2);
        \draw[foamdraw2] (1+.3,0) to (1+.3,2);
        %
        \draw[fill,foamshade1] (-.5,0) rectangle (0,2);
        \draw[foamdraw1] (0,0) to (0,2);
        \draw[fill,foamshade1] (2.5,0) rectangle (2,2);
        \draw[foamdraw1] (2,0) to (2,2);
        %
        \fdot[1.6][1][3]
      }[scale=.35]$};
    \node[fill=white,inner sep=1pt] at (.4*\hsh+.5*\hoplussh,-.2*\vsh) {$-\;
      \tempfoamdbunzip$};
    \node[fill=white,inner sep=1pt] at (1.5*\hsh-.5*\hoplussh,.25*\vsh) {$-\;
      \tempfoamdbunzip$};
    \node[fill=white,inner sep=1pt] at (1.6*\hsh+.5*\hoplussh,-.2*\vsh) {$
      \tikzpic{
        \flst\frst[1][0]\fdot[.5][.5]
      }[scale=.7,xscale=.5]
      -
      \tikzpic{
        \flst\frst[1][0]\fdot[.5][.5][3]
      }[scale=.7,xscale=.5]
    $};
    %
    % bot-top
    \draw[->] (L2) to (L1);
    \draw[->] (R2) to (R1);
    \node[right] at (2*\hsh,-.5*\vsh) {$\id$};
    \node[left] at (0,-.5*\vsh) {$\id$};
    %
    % bottom layer
    \draw[->] (L2) to[out=20,in=180] (T2);
    \draw[->,{preaction={draw,white,line width=5pt}}] (L2) to[out=-20,in=180] (B2);
    \draw[->] (T2) to[out=0,in=160] (R2);
    \draw[->] (B2) to[out=0,in=-160] (R2);
    %
    \node[right={-12pt} of L2] {\scriptsize $\langle -2,2\rangle$};
    \node[right={-12pt} of T2,fill=white,inner sep=1pt] {\scriptsize $\langle -1,1\rangle$};
    \node[right={-12pt} of B2,fill=white,inner sep=1pt] {\scriptsize $\langle -\frac{3}{2},\frac{3}{2}\rangle$};
    \node[right={-12pt} of R2,fill=white,inner sep=1pt] {\scriptsize $\langle -\frac{1}{2},\frac{1}{2}\rangle$};
    %
    \node[fill=white,inner sep=1pt] at (.4*\hsh-.5*\hoplussh,.25*\vsh-\vsh) {$
      \tempfoamdbid$};
    \node[fill=white,inner sep=1pt] at (.4*\hsh+.5*\hoplussh,-.25*\vsh-\vsh) {$
      -\;\tempfoamdbunzip$};
    \node[fill=white,inner sep=1pt] at (1.5*\hsh-.5*\hoplussh,.25*\vsh-\vsh) {$
      -\;\tempfoamdbunzip$};
    \node[fill=white,inner sep=1pt] at (1.6*\hsh+.5*\hoplussh,-.2*\vsh-\vsh) {$\tempfoamid$};
    %
    % B2 to B1
    \draw[{preaction={draw,white,line width=5pt}},->] ($(B2.north)+(.3,0)$) to ($(B1.south)+(.3,0)$);
    \node[right=-8pt] at (\hsh+\voplussh,-.4*\vsh-\voplussh) {$\id$};
    %
    \draw ($(B2.north)+(.3,1)$) to[out=90,in=-90] ($(T1.south)+(-.3,-1.4)$);
    \node[fill=white,inner sep=1pt] at (\hsh,-.5*\vsh) {$\tikzpic{
        \fzip[.5][1][2]
        %
        \draw[fill,foamshade1] (-.5,0) rectangle (0,2);
        \draw[foamdraw1] (0,0) to (0,2);
        \draw[fill,foamshade1] (2.5,0) rectangle (2,2);
        \draw[foamdraw1] (2,0) to (2,2);
      }[scale=.3]$};
  \end{tikzpicture}
\end{equation}
Equivariance follows from \cref{lem:crossing_complex_equivariant}, and the fact that this is indeed an isomorphism of complexes follows from the following two computations (in the first case, we additionally use dot migration to change from $\textdot[2]$ to $\textdot[3]$):
\begin{IEEEeqnarray*}{rCll}
  \tikzpic{
    \draw[fill,foamshade2] (1-.3,0) rectangle (1+.3,2);
    \draw[foamdraw2] (1-.3,0) to (1-.3,2);
    \draw[foamdraw2] (1+.3,0) to (1+.3,2);
    %
    \draw[fill,foamshade1] (-.5,0) rectangle (0,2);
    \draw[foamdraw1] (0,0) to (0,2);
    \draw[fill,foamshade1] (2.5,0) rectangle (2,2);
    \draw[foamdraw1] (2,0) to (2,2);
    %
    \fdot[1.6][1][1]
  }[scale=.35]
  -
  \tikzpic{
    \draw[fill,foamshade2] (1-.3,0) rectangle (1+.3,2);
    \draw[foamdraw2] (1-.3,0) to (1-.3,2);
    \draw[foamdraw2] (1+.3,0) to (1+.3,2);
    %
    \draw[fill,foamshade1] (-.5,0) rectangle (0,2);
    \draw[foamdraw1] (0,0) to (0,2);
    \draw[fill,foamshade1] (2.5,0) rectangle (2,2);
    \draw[foamdraw1] (2,0) to (2,2);
    %
    \fdot[1.6][1][3]
  }[scale=.35]
  %
  &=&
  %
  \tikzpic{
    \draw[fill,foamshade2] (1-.3,0) rectangle (1+.3,2);
    \draw[foamdraw2] (1-.3,0) to (1-.3,2);
    \draw[foamdraw2] (1+.3,0) to (1+.3,2);
    %
    \draw[fill,foamshade1] (-.5,0) rectangle (0,2);
    \draw[foamdraw1] (0,0) to (0,2);
    \draw[fill,foamshade1] (2.5,0) rectangle (2,2);
    \draw[foamdraw1] (2,0) to (2,2);
    %
    \fdot[.4][1][1]
  }[scale=.35]
  -
  \tikzpic{
    \draw[fill,foamshade2] (1-.3,0) rectangle (1+.3,2);
    \draw[foamdraw2] (1-.3,0) to (1-.3,2);
    \draw[foamdraw2] (1+.3,0) to (1+.3,2);
    %
    \draw[fill,foamshade1] (-.5,0) rectangle (0,2);
    \draw[foamdraw1] (0,0) to (0,2);
    \draw[fill,foamshade1] (2.5,0) rectangle (2,2);
    \draw[foamdraw1] (2,0) to (2,2);
    %
    \fdot[.4][1][2]
  }[scale=.35]
  %
  \mspace{10mu}-\mspace{10mu}
  %
  \tikzpic{
    \funzip[.5][0][2]
    \fzip[.5][1][2]
    %
    \draw[fill,foamshade1] (-.5,0) rectangle (0,2);
    \draw[foamdraw1] (0,0) to (0,2);
    \draw[fill,foamshade1] (2.5,0) rectangle (2,2);
    \draw[foamdraw1] (2,0) to (2,2);
  }[scale=.35]
  %%%%%%%%%%%%%%%%%%%%
  &
  %%%%%%%%%%%%%%%%%%%%
  \quad
  \text{thanks to}
  \quad
  \tikzpic{
    \funzip[.5][0][2]
    \fzip[.5][1][2]
  }[scale=.35]
  =
  \tikzpic{
    \draw[fill,foamshade2] (0,0) rectangle (1,2);
    \draw[foamdraw2] (0,0) to (0,2);
    \draw[foamdraw2] (1,0) to (1,2);
    \fdot[1.5][1][2]
  }[scale=.35]
  \;-\;
  \tikzpic{
    \draw[fill,foamshade2] (0,0) rectangle (1,2);
    \draw[foamdraw2] (0,0) to (0,2);
    \draw[foamdraw2] (1,0) to (1,2);
    \fdot[-.5][1][2]
  }[scale=.35]
  %%%%%%%%%%%%%%%%%%%%
  %%%%%%%%%%%%%%%%%%%%
  %%%%%%%%%%%%%%%%%%%%
  \\
  \tikzpic{
    \flst\frst[1][0]\fdot[.5][.5]
  }[scale=.7,xscale=.5]
  -
  \tikzpic{
    \flst\frst[1][0]\fdot[.5][.5][2]
  }[scale=.7,xscale=.5]
  &=&
  \tikzpic{
    \flst\frst[1][0]\fdot[.5][.5]
  }[scale=.7,xscale=.5]
  -
  \tikzpic{
    \flst\frst[1][0]\fdot[.5][.5][3]
  }[scale=.7,xscale=.5]
  %
  \mspace{10mu}-\mspace{10mu}
  %
  \tikzpic{
    \funzip[.5][1][2]
    \fzip[.5][0][2]
    %
    \draw[fill,foamshade1] (-.5,0) rectangle (0,2);
    \draw[foamdraw1] (0,0) to (0,2);
    \draw[fill,foamshade1] (2.5,0) rectangle (2,2);
    \draw[foamdraw1] (2,0) to (2,2);
  }[scale=.35]
  %%%%%%%%%%%%%%%%%%%%
  &
  %%%%%%%%%%%%%%%%%%%%
  \quad
  \text{thanks to}
  \quad
  \tikzpic{
    \funzip[.5][2][2]
    \fzip[.5][1][2]
  }[scale=.35]
  =
  \tikzpic{
    \fdot[0][0][2]
  }[scale=.35]
  \;-\;
  \tikzpic{
    \fdot[0][0][3]
  }[scale=.35]
\end{IEEEeqnarray*}
This gives a zigzag of $\gloo$-equivariant homotopy equivalences between ${}_\omega D_\bullet$ and $D_\bullet^\omega$. One checks that their composition (or their inverse, using the splitting given in \cref{lem:web_resolution_twist}) is the identity.
The last statement in \cref{lem:dot_slide_lemma} is discussed in the following remark.\hfill\qed

\begin{remark}
  Let us try to prove the understrand variant of \cref{lem:dot_slide_lemma}, namely that:
  \[\tikzpic{
      \webncr
      \node[green_mark] (B) at (.3,1-.15) {};
      \node[above={-1pt} of B] {\scriptsize $\omega$};
    }[scale=.5][(0,.5*.5)]
  \;\simeq\;
  \tikzpic{
    \webncr
    \node[green_mark] (B) at (1-.3,.15) {};
    \node[below={-1pt} of B] {\scriptsize $\omega$};
  }[scale=.5][(0,.5*.5)]
  \;.\]
  The beginning of the proof would go through, and we could try to build an isomorphism as in \eqref{eq:proof_dot_slide_isomorphism}.
  The only difference would be that red dots ($\textdot[2]$) are swapped with brown dots ($\textdot[3]$).
  To get a chain map, we would need the following relations, with $\lambda$ some invertible scalar:
  \begin{IEEEeqnarray*}{rCl}
    \tikzpic{
      \draw[fill,foamshade2] (1-.3,0) rectangle (1+.3,2);
      \draw[foamdraw2] (1-.3,0) to (1-.3,2);
      \draw[foamdraw2] (1+.3,0) to (1+.3,2);
      %
      \draw[fill,foamshade1] (-.5,0) rectangle (0,2);
      \draw[foamdraw1] (0,0) to (0,2);
      \draw[fill,foamshade1] (2.5,0) rectangle (2,2);
      \draw[foamdraw1] (2,0) to (2,2);
      %
      \fdot[1.6][1][1]
    }[scale=.35]
    -
    \tikzpic{
      \draw[fill,foamshade2] (1-.3,0) rectangle (1+.3,2);
      \draw[foamdraw2] (1-.3,0) to (1-.3,2);
      \draw[foamdraw2] (1+.3,0) to (1+.3,2);
      %
      \draw[fill,foamshade1] (-.5,0) rectangle (0,2);
      \draw[foamdraw1] (0,0) to (0,2);
      \draw[fill,foamshade1] (2.5,0) rectangle (2,2);
      \draw[foamdraw1] (2,0) to (2,2);
      %
      \fdot[1.6][1][2]
    }[scale=.35]
    %
    &\overset{?}{=}&
    %
    \tikzpic{
      \draw[fill,foamshade2] (1-.3,0) rectangle (1+.3,2);
      \draw[foamdraw2] (1-.3,0) to (1-.3,2);
      \draw[foamdraw2] (1+.3,0) to (1+.3,2);
      %
      \draw[fill,foamshade1] (-.5,0) rectangle (0,2);
      \draw[foamdraw1] (0,0) to (0,2);
      \draw[fill,foamshade1] (2.5,0) rectangle (2,2);
      \draw[foamdraw1] (2,0) to (2,2);
      %
      \fdot[.4][1][1]
    }[scale=.35]
    -
    \tikzpic{
      \draw[fill,foamshade2] (1-.3,0) rectangle (1+.3,2);
      \draw[foamdraw2] (1-.3,0) to (1-.3,2);
      \draw[foamdraw2] (1+.3,0) to (1+.3,2);
      %
      \draw[fill,foamshade1] (-.5,0) rectangle (0,2);
      \draw[foamdraw1] (0,0) to (0,2);
      \draw[fill,foamshade1] (2.5,0) rectangle (2,2);
      \draw[foamdraw1] (2,0) to (2,2);
      %
      \fdot[.4][1][3]
    }[scale=.35]
    %
    \mspace{10mu}-\mspace{10mu}\lambda\;
    %
    \tikzpic{
      \funzip[.5][0][2]
      \fzip[.5][1][2]
      %
      \draw[fill,foamshade1] (-.5,0) rectangle (0,2);
      \draw[foamdraw1] (0,0) to (0,2);
      \draw[fill,foamshade1] (2.5,0) rectangle (2,2);
      \draw[foamdraw1] (2,0) to (2,2);
    }[scale=.35]
    %%%%%%%%%%%%%%%%%%%%
    %%%%%%%%%%%%%%%%%%%%
    %%%%%%%%%%%%%%%%%%%%
    \\
    \tikzpic{
      \flst\frst[1][0]\fdot[.5][.5]
    }[scale=.7,xscale=.5]
    -
    \tikzpic{
      \flst\frst[1][0]\fdot[.5][.5][3]
    }[scale=.7,xscale=.5]
    &\overset{?}{=}&
    \tikzpic{
      \flst\frst[1][0]\fdot[.5][.5]
    }[scale=.7,xscale=.5]
    -
    \tikzpic{
      \flst\frst[1][0]\fdot[.5][.5][2]
    }[scale=.7,xscale=.5]
    %
    \mspace{10mu}-\mspace{10mu}\lambda\;
    %
    \tikzpic{
      \funzip[.5][1][2]
      \fzip[.5][0][2]
      %
      \draw[fill,foamshade1] (-.5,0) rectangle (0,2);
      \draw[foamdraw1] (0,0) to (0,2);
      \draw[fill,foamshade1] (2.5,0) rectangle (2,2);
      \draw[foamdraw1] (2,0) to (2,2);
    }[scale=.35]
  \end{IEEEeqnarray*}
  The first identity imposes $\lambda=1$, while the second imposes $\lambda=-1$: we do not have a chain map.
  However, if we instead work with $\sfoam'$, then the bubble evaluation is replaced by the relation
  \begin{gather*}
    \tikzpic{
      \funzip[.5][2][2]
      \fzip[.5][1][2]
    }[scale=.35]
    =
    \;-\;
    \tikzpic{
      \fdot[0][0][2]
    }[scale=.35]
    \;+\;
    \tikzpic{
      \fdot[0][0][3]
    }[scale=.35]
    \;,
  \end{gather*}
  so that setting $\lambda=1$ works.
  In other words, if we work with $\sfoam$, \cref{lem:dot_slide_lemma} (overstrand) works but not its understrand variant; while if we work with $\sfoam'$, the understrand variant of \cref{lem:dot_slide_lemma} holds, but not \cref{lem:dot_slide_lemma} itself.
\end{remark}

