\section{Introduction}
\label{sec:introduction}

\subsection{Overview}

Quantum link homologies are homology theories for links in $S^3$ that arise as categorifications of polynomial link invariants associated with quantum groups.
Following Khovanov's construction of a categorification of the Jones polynomial \cite{Khovanov_CategorificationJonesPolynomial_2000}, several related homology theories have been developed.
These constructions are closely connected with higher representation theory and have led to many fruitful interactions between these fields.

\medbreak

Actions on these homologies, or on the categories underlying them, have been studied by various authors in different contexts and with different motivations.

Gorsky, Oblomkov, and Rasmussen \cite{GOR_StableKhovanovHomology_2013} conjectured that certain colored link homologies have graded dimensions given by the characters of representations of affine Lie algebras.
An $\slt$-action on triply-graded homology was constructed by Gorsky, Hogancamp and Mellit \cite{GHM_TautologicalClassesSymmetry_2024} used to show certain symmetries of triply-graded homology, giving a new proof of a conjecture of Dunfield, Gukov, and Rasmussen \cite{DGR_SuperpolynomialKnotHomologies_2006}; this action is further studied in \cite{CG_StructuresHOMFLYPTHomology_2024}.

Actions of Steenrod algebras have been constructed on even and odd Khovanov homology \cite{LS_SteenrodSquareKhovanov_2014,Schutz_TwoSecondSteenrod_2022}, induced from (or at least motivated by) the existence of (odd) Khovanov stable homotopy type \cite{LS_KhovanovStableHomotopy_2014,HKK_FieldTheoriesStable_2016,SSS_OddKhovanovHomotopy_2020}. See \cite{Rajapakse_SteenrodSquaresEven_2025} for recent developments.

For annular theories, Grigsby, Licata and Wehrli \cite{GLW_AnnularKhovanovHomology_2018} constructed an action the $\slt$ current algebra on annular Khovanov homology, while Grigsby and Wehrli constructed an action of $\gloo$ on odd annular Khovanov homology \cite{GW_Action$mathfrakgl1|1$Odd_2020}.

In \cite{KR_PositiveHalfWitt_2016}, Khovanov and Rozansky constructed an action of the positive half of the Witt algebra $\fW^+$ on triply-graded homology.
Inspired by this action, Qi, Robert, Wagner and Sussan \cite{QRS+_Symmetries$mathfrakgl_N$foams_2024a} constructed an action of $\fW^\infty_{-1}=\fW^+\cup\langle L_{-1}\rangle$ on equivariant $\fgl_N$-foams \cite{MSV_$mathfrakslN$linkHomology$N_2009,Khovanov_Sl3LinkHomology_2004,RW_ClosedFormulaEvaluation_2020}, where $L_{-1}$ is the degree $-2$ operator in the Witt algebra $\fW$; Guérin and Roz \cite{GR_ActionWittAlgebra_2025} later extended this action to equivariant Khovanov--Rozansky homology \cite{KR_MatrixFactorizationsLink_2008}, building on \cite{QRS+_SymmetriesEquivariantKhovanovRozansky_2023}.
Over a field of characteristic $p$, one can restrict to non-equivariant parameters, and the degree $2$ operator $L_1\in\fW^+$ recovers the $p$-DG structure used in
\cite{%
  % Khovanov_HopfologicalAlgebraCategorification_2016,
  QRS+_CategorificationColoredJones_2021,%
  QS_PdifferentialGradedLink_2022%
} to categorify the (resp.\ colored) Jones polynomial at root of unity.
On the other hand, the operator $L_{-1}$ recovers Wang's extension of Shumakovitch operation \cite{Wang_$mathfrakslN$LinkHomology_2024}.
These work have lead to certain topological applications and structural properties; see \cite{QRS+_SymmetriesEquivariantKhovanovRozansky_2023,QRS+_RemarksInfinitesimalSymmetries_2024,Roz_$mathfraksl_2$ActionLink_2023}.

In connection with some of the above work, Elias and Qi realised that various categories appearing in higher representation theory carried an $\slt$-action \cite{EQ_CategorifyingHeckeAlgebras_2020,EQ_Actions$mathfraksl_2$Algebras_2023}. In a related direction, Grlj and Lauda recently constructed an action of the positive Witt algebra on simply-laced categorified quantum groups \cite{GL_ActionWittAlgebra_2025}.
% \cite{BC_SteenrodStructuresCategorified_2018}

In this article, in analogy with this line of work, we describe a  $\gloo$-action on \emph{odd} Khovanov homology.

\medbreak

Odd Khovanov homology \cite{ORS_OddKhovanovHomology_2013} is a homological invariant of links. As (even) Khovanov homology, it categorifies the Jones polynomial. While the two theories are identical over $\bF_2$, they are distinct over $\bZ$, in the sense that one can find pair of knots distinguished by one theory but not the other \cite{Shumakovitch_PatternsOddKhovanov_2011}.
It was discovered in an attempt to lift to the integers the Ozsváth--Szabó spectral sequence from Khovanov homology to the Heegaard--Floer homology of the branched double cover \cite{OS_HeegaardFloerHomology_2005}.
While the existence of this spectral sequence remains conjectural, odd Khovanov homology is thought as more closely related to Heegaard--Floer theory than its even counterpart.
Various authors have explored odd Khovanov homology; see \cite{NP_OddKhovanovHomology_2020,Spyropoulos_HochschildCohomologyOdd_2025,Spyropoulos_JonesWenzlProjectorsOdd_2024,MW_FunctorialityOddGeneralized_2024} for recent structural results using this original construction.

Since its discovery, odd Khovanov homology has been expected to relate to various odd analogues in higher representation theory, and in particular to so-called ``supercategorification'' \cite{EL_OddCategorification$U_qmathfraksl_2$_2016}; see e.g.\ \cite{%
  EKL_OddNilHeckeAlgebra_2014,%
  BE_SuperKacMoody_2017,%
  % BE_MonoidalSupercategories_2017,
  BK_OddGrassmannianBimodules_2022,%
  % Ebert_NewPresentationOsp1|2polynomial_2022,
  % EQ_DifferentialGradedOdd_2016,
  EL_DGStructuresOdd_2020,%
  EL_OddCategorification$U_qmathfraksl_2$_2016,
  ELV_DerivedSuperequivalencesSpin_2022,
  ENW_RealSpringerFibers_2021,
  % KKO_SupercategorificationQuantumKacMoody_2013,
  % KKO_SupercategorificationQuantumKac_2014,
  LR_OddificationCohomologyType_2014,
  % NV_OddKhovanovsArc_2018,
}.
An explicit connection in that direction was given in \cite{SV_OddKhovanovHomology_2023}, where the second author and Vaz gave a foamy construction of odd Khovanov homology.
The main players are \emph{super $\glt$-foams}, gathering together as the super-2-category $\sfoam$; they lead to an invariant of tangles in the homotopy category of $\sfoam$.
A \emph{super-2-category} is a structure akin to a linear 2-category, but where 2-morphisms have parities and the interchange law only hold up to sign.
In the original construction of odd Khovanov homology, signs depend on whether a saddle is a split or a merge (a global data); in the foamy construction, parities only depend on wether a saddle is a zip or an unzip (a local data).
Despite these conceptual differences, the two constructions lead to the same invariant (when restricted to links).

Moreover, each construction comes in two flavours. The original construction is either ``type X'' or ``type Y'';
and the super-2-category $\sfoam$ admits a (essentially unique) variant, denoted $\sfoam'$ \cite{Schelstraete_RewritingModuloDiagrammatic_2025}.
Through the isomorphism between the two constructions, $\sfoam$ relates to type Y, while $\sfoam'$ relates to type X.
On the topological side, the existence of these variants comes from a sign choice ambiguity on so-called ``ladybug squares'', similar to the choice ambiguity appearing in Khovanov stable homotopy type \cite{LS_KhovanovStableHomotopy_2014}.
Despite this ambiguity, type X and type Y have been shown to be isomorphic \cite{Beier_IntegralLiftStarting_2012,Putyra_2categoryChronologicalCobordisms_2014}.

\medbreak

In work in progress, Migdail and Wehrli \cite{MW_ModuleStructureOdd_} (building on Migdail's PhD thesis \cite{Migdail_FunctorialityOddKhovanov_2025}) define an action of the first homology group of the branched double cover of the link, and study some of its topological consequences.
We learned about their work while preparing this manuscript; see \cref{rem:literature_comparison_Migdail_Wehrli} for details on how our work relates with theirs.

We now summarize our result:
\medbreak

\noindent \textsc{Extended abstract:}
\begin{enumerate}[(i)]
  \item There exists a $\gloo$-action on the super-2-category $\sfoam$ which gives rise to a $\gloo$-action on (the foamy construction of) odd Khovanov homology, well-defined for any tangle and choice of ``markings'' (see below).
  \item Markings behave differently in type X (i.e.\ $\sfoam'$) and type Y (i.e.\ $\sfoam$); while the type X and type Y odd Khovanov homologies are isomorphic, they are \emph{not} expected to be $\gloo$-equivariantly isomorphic.
  \item When restricting to links and comparing with the original construction of odd Khovanov homology, part of that action recovers Migdail and Wehrli's action of the first homology group of the branched double cover of the link.
  \item The pretzel link $P(n,n,-n)$ has torsion $\bZ/n\bZ$; this copy lies in the image of the $\gloo$-action. In particular, all torsions appear in odd Khovanov homology.
\end{enumerate}
Through (iii), item (ii) recovers a similar observation by Migdail and Wehrli in their work in progress \cite{MW_ModuleStructureOdd_}. Item (iv) in particular answers a question of Shumakovitch \cite{Shumakovitch_PatternsOddKhovanov_2011}, who showed that $P(n,n,-n)$ had torsion $\bZ/n\bZ$ for small $n$ and suggested this was a general pattern. Through (iii) again, this extends a remark of Migdail and Wehrli \cite{MW_ModuleStructureOdd_}, who showed that torsion in $P(3,3,-3)$ lies in the image of their action.

\newpage

\subsection{Results}

We now describe our results in more details.
Throughout we work over any ring in which $2$ is invertible; alternatively, one can ignore the condition that $2$ is invertible by restricting to $\sloo\subset\gloo$ (see \cref{rem:local_invariant_2_invertible}).
Recall that the super Lie algebra $\gloo$ has generators $\liee$, $\lief$, $\lieh_1$ and $\lieh_2$; see \cref{ex:defn_gloo}.

The $\gloo$-action depends on a choice of ``markings'' on the tangle.
Namely, a \emph{choice of markings} is a choice of diagram together with points on this diagram, each endowed with a triple of scalars $(\alpha,\beta_1,\beta_2)$ with $\alpha=\beta_1+\beta_2$. These scalars $\alpha$, $\beta_1$ and $\beta_2$ correspond to twists of the action of $\lief$, $\lieh_1$ and $\lieh_2$, respectively.

The super-2-category of super $\glt$-foams is reviewed in \cref{defn:foam_and_sfoam} and \cref{defn:review_foam}, which we write as $\sfoam$ (and $\sfoam'$ its variant) in this introduction for simplicity; see also \cref{defn:twisted_foams} and \cref{defn:colimit_of_foam} for the relevant versions with markings.

\begin{bigtheorem}[\cref{thm:topological_invariance} and \cref{lem:dot_slide_lemma}]
  \label{thm:main_local_action}
  There exists a $\gloo$-action on $\sfoam$, given on generators in \cref{tab:definition_action_foam_intro}, which extends to a $\gloo$-action on (the foamy construction of) Khovanov homology for any tangle and choice of markings. Moreover, the action is invariant under any move not involving the markings, under markings sliding along strands, and under markings sliding accross crossings as follows (here $\omega=(\alpha,\beta_1,\beta_2)$):
  \begin{gather*}
    \tikzpic{
      \webncr
      \node[green_mark] (B) at (.3,.15) {};
      \node[below={-1pt} of B] {\scriptsize $\omega$};
    }[scale=.7][(0,.5*.7)]
    \simeq^{\gloo}
    \tikzpic{
      \webncr
      \node[green_mark] (B) at (1-.3,1-.15) {};
      \node[above={-1pt} of B] {\scriptsize $\omega$};
    }[scale=.7][(0,.5*.7)]
    \quad\an\quad
    \tikzpic{
      \webpcr
      \node[green_mark] (B) at (.3,.15) {};
      \node[below={-1pt} of B] {\scriptsize $\omega$};
    }[scale=.7][(0,.5*.7)]
    \simeq^{\gloo}
    \tikzpic{
      \webpcr
      \node[green_mark] (B) at (1-.3,1-.15) {};
      \node[above={-1pt} of B] {\scriptsize $-\omega$};
      \node[green_mark] (C) at (1-.3,.15) {};
      \node[below={-1pt} of C] {\scriptsize $2\omega$};
    }[scale=.7][(0,.5*.7)]
    \;.
  \end{gather*}
  Here $\simeq^{\gloo}$ denotes isomorphism in the relative homotopy category (\cref{defn:formal_relative_homotopy_category}).
  Considering $\sfoam'$ instead exchanges the role of the overcross and the undercross in the above statement.
\end{bigtheorem}

\begin{table}
  \def\spc{2ex}
  \def\wdth{1.5cm}
  \centering
  \begin{tabular}{@{}l@{\hskip 5ex}*{4}{r@{\hskip 5ex}}r@{}}
    &
    $\vcenter{\hbox{\includegraphics[width=\wdth]{foam/dot.pdf}}}$
    &
    $\vcenter{\hbox{\includegraphics[width=\wdth]{foam/cup.pdf}}}$
    &
    $\vcenter{\hbox{\includegraphics[width=\wdth]{foam/zip.pdf}}}$
    &
    $\vcenter{\hbox{\includegraphics[width=\wdth]{foam/cap.pdf}}}$
    &
    $\vcenter{\hbox{\includegraphics[width=\wdth]{foam/unzip.pdf}}}$
    \\*[2ex]
    \midrule
    %%%%%%%%%%%%%%%%%%%%%%%%%%%%%%
    $\lief$
    &
    $0$
    &
    $\vcenter{\hbox{\includegraphics[width=\wdth]{foam/dotted_cup.pdf}}}$
    &
    $\vcenter{\hbox{\includegraphics[width=\wdth]{foam/dotted_zip.pdf}}}$
    &
    $-\vcenter{\hbox{\includegraphics[width=\wdth]{foam/dotted_cap.pdf}}}$
    &
    $\vcenter{\hbox{\includegraphics[width=\wdth]{foam/dotted_unzip.pdf}}}$
    \\*[\spc]
    %%%%%%%%%%%%%%%%%%%%%%%%%%%%%%
    $\liee$
    &
    $\vcenter{\hbox{\includegraphics[width=\wdth]{foam/facet.pdf}}}$
    &
    $0$
    &
    $0$
    &
    $0$
    &
    $0$
    \\*[\spc]
    %%%%%%%%%%%%%%%%%%%%%%%%%%%%%%
    $\lieh_1$
    &
    $-\vcenter{\hbox{\includegraphics[width=\wdth]{foam/dot.pdf}}}$
    &
    $\vcenter{\hbox{\includegraphics[width=\wdth]{foam/cup.pdf}}}$
    &
    $0$
    &
    $0$
    &
    $-\vcenter{\hbox{\includegraphics[width=\wdth]{foam/unzip.pdf}}}$
    \\*[\spc]
    %%%%%%%%%%%%%%%%%%%%%%%%%%%%%%
    $\lieh_2$
    &
    $\vcenter{\hbox{\includegraphics[width=\wdth]{foam/dot.pdf}}}$
    &
    $0$
    &
    $\vcenter{\hbox{\includegraphics[width=\wdth]{foam/zip.pdf}}}$
    &
    $-\vcenter{\hbox{\includegraphics[width=\wdth]{foam/cap.pdf}}}$
    &
    $0$
    \\*[\spc]
    % \cmidrule(r){2-6}
    %%%%%%%%%%%%%%%%%%%%%%%%%%%%%%
    % $\lieh\coloneqq\lieh_1+\lieh_2$
    % &
    % $0$
    % &
    % \includegraphics[width=\wdth]{foam/cup.pdf}
    % &
    % \includegraphics[width=\wdth]{foam/zip.pdf}
    % &
    % $-$\includegraphics[width=\wdth]{foam/cap.pdf}
    % &
    % $-$\includegraphics[width=\wdth]{foam/unzip.pdf}
  \end{tabular}
  \caption{Definition of the action of $\gloo$ by derivation on the generators of $\sfoam$.
  The source is given on the top row, and the target on the associated row; for instance, we have $\lief\cdot\vcenter{\hbox{\includegraphics[width=1cm]{foam/dot.pdf}}}=0$.}
  \label{tab:definition_action_foam_intro}
\end{table}

Note that while markings can ``freely'' overcross, the rule for undercrossing is more intricate; in fact, one can check that it cannot both freely overcross \emph{and} undercross (\cref{rem:marking_cannot_undercross}).
It follows that (see \cref{thm:main_global_action}):

\begin{corollary}
  The homology theories for marked tangles associated to $\sfoam$ and $\sfoam'$ are isomorphic, but in general 
   \emph{not} $\gloo$-equivariantly isomorphic.
\end{corollary}

As a natural odd analogue to the work of Elias and Qi \cite{EQ_Actions$mathfraksl_2$Algebras_2023}, and having in mind the work of Grlj and Lauda \cite{GL_ActionWittAlgebra_2025}, one wonders:

\begin{question}
  Are there other actions of super Lie algebras appearing in supercategorification, for instance on super Kac--Moody 2-categories \cite{BE_SuperKacMoody_2017}?
\end{question}

There is a unifying approach between even and odd Khovanov homology, replacing signs by scalars and super structures by graded structures.
In particular, there exists a \emph{graded-2-category $\gfoam$ of graded $\glt$-foams}, which specializes both to $\glt$-foams and super $\glt$-foams.
Working in this framework allows an explicit comparison of the two theories.

The action of $\liee$ does not work in the even setting, for a simple reason. For grading reason, it must act on dots as $\liee(\textdot)=\lambda\id$ for some scalar $\lambda$, but by the Leibniz rule, $\liee(\textdot^2)=\lambda\;\textdot+\lambda\;\textdot=2\lambda\;\textdot$, which contradicts $\textdot^2=0$ (at least if $2\lambda$ is invertible).
In the super context, the super Leibniz rule replaces ``$+$'' by ``$-$'', and hence there is no contradiction.
This is parallel to the fact that the $\slt$-action in (non-equivariant) $\fgl_p$-Khovanov--homology \cite{QRS+_SymmetriesEquivariantKhovanovRozansky_2023} is well-defined only over a field of characteristic $p$.

Nonetheless, one can define the action excluding $\liee$, and this can be unified at the level of graded $\glt$-foams.
For that purpose, we define $\grgl^\geq_2$ as a certain \emph{graded Lie algebra} (a structure that interpolates between Lie algebras and super Lie algebras; this is \emph{not} just a Lie algebra with a grading) interpolating between $\fgl_2$ and $\gloo$.
The homology interpolating even and odd Khovanov homology is known as \emph{covering} (or \emph{generalized}) \emph{Khovanov homology} \cite{Putyra_2categoryChronologicalCobordisms_2014}.

\begin{proposition}
  There exists a $\grgl^\geq_2$-action on $\gfoam$, which extends to a $\grgl^\geq_2$-action on covering $\glt$-Khovanov homology for any tangle and choice of markings.
  Moreover, the action is invariant under any move not involving the markings and under markings sliding along strands, away from crossings.
\end{proposition}

Note that in the graded case, markings do not seem to verify any particular crossing slide relation\footnote{Although see the relation in the proof of \cref{lem:csq_dot_slide_lemma}, which holds in general.}; the result that markings can slide over crossing is specific to the odd case.

\medbreak

Next, we compare with the original construction of odd Khovanov homology.
Our construction provides a certain ``$\gloo$-equivariant homotopy equivalence of complexes $\glOKh(T)$'' associated to a tangle with markings; to compare with the original construction, we need to apply a homology functor, given by the composition of the standard homology functor and a representable functor.
We denote $\slOKh^Y(L)$ the type Y (original construction of) odd Khovanov homology.

\begin{bigtheorem}[\cref{thm:equivalence_local_global}]
  \label{thm:main_global_action}
  Let $L$ be an oriented link and $D$ a diagram of $L$.
  There exists a $\gloo$-action on (the original construction of) odd Khovanov homology, for any oriented link and choice of markings.
  Moreover, there is a $\gloo$-equivariant isomorphism
  \begin{gather*}
    H_\bullet\Hom(\emptyset,\glOKh(D))\cong^{\gloo}\slOKh^Y(D).
  \end{gather*}
  Similarly, we have a $\gloo$-equivariant isomorphism considering $\sfoam'$ and type X instead.
\end{bigtheorem}

This allows us to relate with other constructions appearing in literature; see \cref{rem:literature_comparison_Shumakovitch} for comparison with Shumakovitch's operation $\nu$ \cite{Shumakovitch_TorsionKhovanovHomology_2014}, \cref{rem:literature_comparison_Migdail_Wehrli} for comparison with Manion's work \cite{Manion_SignAssignmentTotally_2014}\footnote{We thank Stephan Wehrli for pointing out that reference to us.} and Migdail and Wehrli's work in progress \cite{MW_ModuleStructureOdd_,Migdail_FunctorialityOddKhovanov_2025}, and \cref{rem:literature_comparison_Grigsby_Wehrli} for comparison with Grigsby and Wehrli's $\gloo$-action on odd annular Khovanov homology \cite{GW_Action$mathfrakgl1|1$Odd_2020}.

\medbreak

As noticed by Shumakovitch \cite{Shumakovitch_PatternsOddKhovanov_2011}, even and odd Khovanov homology typically have very different torsions.
As an example of that heuristics, Shumakovitch noticed that for certain pretzel links, reduced even and odd Khovanov homologies have the same torsion-free part, with only odd Khovanov homology having a non-trivial torsion part.
In particular, he computed that $P(n,n,-n)$ had $\bZ/n\bZ$ torsion in odd Khovanov homology for small $n\in\bN$, and asked whether this was a general pattern.

We verify this expectation, and relate it to our $\gloo$-action:

\begin{bigtheorem}
  \label{thm:main_pretzel}
  \def\webscl{.4}
  Let $n\in\bN$.
  The odd Khovanov homology of the pretzel link
  \begin{gather*}
    P(n,n,-n)\coloneqq
    \tikzpic{
      \webpcr\node at (2,.5) {$\ldots$};\webpcr[3][0]
      \webpcr[0][2]\node at (2,2.5) {$\ldots$};\webpcr[3][2]
      \webncr[0][4]\node at (2,4.5) {$\ldots$};\webncr[3][4]
      %
      \draw[web1] (0,0) to[out=180,in=180] (0,5);
      \draw[web1] (0,1) to[out=180,in=180] (0,2);
      \draw[web1] (0,3) to[out=180,in=180] (0,4);
      %
      \draw[web1] (4,0) to[out=0,in=0] (4,5);
      \draw[web1] (4,1) to[out=0,in=0] (4,2);
      \draw[web1] (4,3) to[out=0,in=0] (4,4);
      %
      \node[green_mark] at (0,5) {};
      \node[above left=-1pt] at (0,5) {\scriptsize $(1,\frac{1}{2},\frac{1}{2})$};
      \node[green_mark] at (4,5) {};
      \node[above right=-1pt] at (4,5) {\scriptsize $(-1,-\frac{1}{2},-\frac{1}{2})$};
      %
      \draw [decorate,decoration={brace,amplitude=5pt,mirror,raise=1ex}]
      (0,0) -- (4,0) node[midway,yshift=-1.5em]{$n$};
    }[scale=\webscl][(0,2*\webscl)]
  \end{gather*}
  has torsion $\bZ/n\bZ$. Moreover, this copy of $\bZ/n\bZ$ lies in the image of the action of $\lief\in\gloo$, for the given choice of markings (for both type X or type Y).
\end{bigtheorem}

In particular, the $\gloo$-action is non-trivial on $P(n,n,-n)$. As mentioned above, Migdail and Wehrli \cite{Migdail_FunctorialityOddKhovanov_2025} have shown an analoguous statement using their action (see \cref{rem:literature_comparison_Migdail_Wehrli}), for the pretzel knots $P(3,3,-3)$ and $P(3,4,-3)$.

It follows that:

\begin{corollary}
  All torsions appear in odd Khovanov homology.
\end{corollary}

To the authors' knowledge, this result has not appeared in the literature.
In contrast, and to the authors' knowledge again, it is not known whether all torsions appear in even Khovanov homology, in spite of active research on the question; see e.g.\ \cite{Shumakovitch_TorsionKhovanovHomology_2014,PS_TorsionKhovanovHomology_2014,MPS+_SearchTorsionKhovanov_2018,MS_ArbitrarilyLargeTorsion_2021}.

\begin{question}
  How much of the torsion in odd Khovanov homology can be explained by the $\gloo$-action?
\end{question}




\subsection{Organization}

\Cref{sec:action_on_foams} describes the action on $\sfoam$, \cref{sec:action_on_homology} describes the action on (the foamy definition of) odd Khovanov homology, \cref{sec:global_action_odd_khovanov} compares with the original construction when restricting to links, and \cref{sec:computation} does the torsion computation for pretzel links.


\subsection{Acknowledgments}

We thank Alexis Guérin, You Qi, Louis-Hadrien Robert, Felix Roz, Josh Sussan for discussing their previous work, and we thank Stephan Wehrli for discussing his work in progress with Jacob Migdail.
L.S.\ is supported by the Max Planck Institute for Mathematics (Bonn, Germany). M.E. is supported by the Simons Collaboration Award 994328 “New Structures in Low-Dimensional Topology.”
