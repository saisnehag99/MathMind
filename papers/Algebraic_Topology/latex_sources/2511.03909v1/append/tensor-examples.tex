\section{Examples of Tensor Operations}
\label{append:tensor-examples}

In this appendix, we provide a few examples to illustrate the tensor operations
defined in \ssref{tensorops}.
Consider the following~tensors:
\[
    T = \begin{bmatrix}
        1 & 2 \\
        3 & 4 \\
        5 & 6
    \end{bmatrix}, \quad
    U = \begin{bmatrix}
        1 \\
        2
    \end{bmatrix}, \quad
    V = \begin{bmatrix}
        5
    \end{bmatrix}
\]

% Cumulative sum
The $\cumsum$ operation computes the cumulative sum of each row of the tensor.
For the tensor $T$, we see
\[
    \cumsum(T) = \begin{bmatrix}
        1 & 1 + 2 \\
        3 & 3 + 4 \\
        5 & 5 + 6
    \end{bmatrix} = \begin{bmatrix}
        1 & 3 \\
        3 & 7 \\
        5 & 11
    \end{bmatrix}.
\]

% Reduced max
The $\rmax$ operation computes the maximum value with respect to a specified axis.
That is, for the tensor $T$ and axis $p = 1$, we have
\[
    \rmax(T, 1) = \begin{bmatrix}
        \max(1, 2) \\
        \max(3, 4) \\
        \max(5, 6)
    \end{bmatrix} = \begin{bmatrix}
        2 \\
        4 \\
        6
    \end{bmatrix}.
\]

% Advanced Indexing
Next, we consider the advanced indexing operation, using $U$ to index into $T$.
Note that~$T$ has shape $(3, 2)$ and $U$ has shape $(2, 1)$, so the resulting tensor
has shape $(2, 1, 2)$.
Then, for each $k \in [[2]]$ we have
\[
    T[U] = \begin{bmatrix}
            T[U[0, 0], k] \\
            T[U[0, 1], k]
        \end{bmatrix} =
        \begin{bmatrix}
            T[1, k] \\
            T[2, k]
        \end{bmatrix} =
        \begin{bmatrix}
            \begin{bmatrix}
                3 & 4
            \end{bmatrix} \\
            \begin{bmatrix}
                5 & 6
            \end{bmatrix}
        \end{bmatrix}.
\]

% Scatter Add
To demonstrate the $\scadd$ operation, we need to ensure dimensions align correctly, so consider the transpose $S = T^T$, which is
\[
    S = \begin{bmatrix}
        1 & 3 & 5 \\
        2 & 4 & 6
    \end{bmatrix},
\]
Next, we compute $\scadd(U, V)(S)$.
To do so, we compute the difference tensor~$D$, which is used to modify $S$ in place.
The tensor $D$ has the same shape as $S$, which is~$(2, 3)$.
Because~$V$ has shape $(1)$, we note that $k=0$ is the only possible value of
$k$ in the definition of $D$ given in \eqref{difftensor}.  As a result, we look
for indices $i,j$ such that $U[i,0]=j$.
Thus, we start with $D=\zeros((2,3))$ and compute the values of $D$ by adding
the following values:
\begin{enumerate}
    \item For the first row ($i = 0$), the index from $U$ is $U[0,0] = 1$,
        and the value from $V$ is~$5$.
        Thus, we add the value $5$ to the element at position~\mbox{$(i, U[i, k]) =
        (0,U[0,0])= (0, 1)$},
        so the first row of $D$ is $[0, 5, 0]$.
    \item For the second row ($i = 1$), the index from $U$ is $U[1, 0] = 2$,
        and the value from $V$ is~$5$.
        Thus, we add the value $5$ to the element at position~\mbox{$(i, U[i, k]) =
        (1,U[1,0]) = (1, 2)$},
        so the second row of $D$ is $[0, 0, 5]$.
\end{enumerate}
Thus, the difference tensor is
\[
    D = \begin{bmatrix}
        0 & 5 & 0 \\
        0 & 0 & 5
    \end{bmatrix},
\]
and we have
\[
    \scadd(U, V)(S) = S + D = \begin{bmatrix}
        1 + 0 & 3 + 5 & 5 + 0 \\
        2 + 0 & 4 + 0 & 6 + 5
    \end{bmatrix} = \begin{bmatrix}
        1 & 8 & 5 \\
        2 & 4 & 11
    \end{bmatrix}.
\]
Recall that the $\scadd$ operation modifies the input tensor in place,
so the new value of $S$ is
\[
    S = \begin{bmatrix}
        1 & 8 & 5 \\
        2 & 4 & 11
    \end{bmatrix}.
\]
