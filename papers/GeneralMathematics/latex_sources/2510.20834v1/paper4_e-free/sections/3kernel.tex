\newpage

%------------------------------------------------------------------
\section{Kernel estimate of the operator}\label{sec:kernel}
%------------------------------------------------------------------

We show that for $\lambda \ge 2$
\begin{equation}\label{eq:kernel-claim}
   \|K\|_{L^{2} \to L^{2}}
   \;\lesssim\;
   \lambda^{-9/2}\,D^{-3},
   \qquad D = \lambda^{1/12}.
\end{equation}
This bound in $\lambda$ is \emph{stronger}, while in $D$ it is \emph{weaker}, than the benchmark $\lambda^{-4} D^{-7}$; together with the other
blocks (see~\S\ref{sec:balance}) it suffices for negativity of the cumulative exponents.

%------------------------------------------------------------------
\subsection{Full kernel}\label{sec:kernel-1}
%------------------------------------------------------------------
For the decomposition \(F=\sum_{\Theta} F_{\Theta}\) into caps (see §\ref{subsec:caps}) define
\[
   K(t,x;\,s,y)
   \;=\;
   \int_{\Theta^{6}}
        e^{\,i\Phi(t,x;\xi)\,-\,i\Phi(s,y;\xi)}\;
        a(t,\xi)\,a(s,\xi)\;
   d\xi,
\]
where integration is over the Cartesian product \(\Theta^{6}\) of the active caps, 
\(d\xi=\prod_{m=1}^{6}d\xi_m\), and the phase and amplitude are
\[
   \Phi(t,x;\xi)
   \;=\;
   t\!\!\sum_{m\le3}\!|\xi_m|^{2}
   \;-\;
   t\!\!\sum_{m>3}\!|\xi_m|^{2}
   \;+\;
   x'\!\cdot\!\Bigl(\sum_{m\le3}\xi'_m-\sum_{m>3}\xi'_m\Bigr),
   \qquad x'=(x_2,x_3),
\]
\[
   a(t,\xi)=\omega(t)\,\vartheta(\xi),\qquad
   \partial_t^{\,k}\omega(t)=O\!\bigl(\lambda^{3k/2}\bigr)\quad(k\ge0).
\]
Here \(\xi'_m=(\xi_{m,2},\xi_{m,3})\) is the transverse projection; when needed we also use \(\xi'=(\xi_2,\xi_3)\).

\begin{remark}[Time window with plateau]\label{rem:time-plateau}
Take \(\omega(t)=\chi(\lambda^{3/2}t)\) with \(\chi\in C_0^\infty([-1/2,1/2])\) and \(\chi\equiv1\) on \([-1/4,1/4]\).
On the central plateau \(|t|\le \tfrac14\lambda^{-3/2}\) we have \(\partial_t\omega\equiv0\), while all derivatives
\(\partial_t^{\,k}\omega\) are supported in the two edge layers \(|t|\in[\tfrac14,\tfrac12]\lambda^{-3/2}\) of total measure
\(\lesssim\lambda^{-3/2}\). This allows time IBP in the analysis of \(TT^{\!*}\) without paying for
derivatives of \(\omega\) on the plateau; the contribution of the edges of the window is subcritical (see also App.~\ref{app:C4}).
\end{remark}

\begin{remark}[Why the integral over \(\Theta^6\)]
The kernel \(K\) arises after one Cauchy–Schwarz step in the trilinear insertion
of Bennett–Carbery–Tao: this produces a product of six terms \(F_{\Theta}\), so the phase
\(\Phi\) depends on six frequencies \((\xi_1,\dots,\xi_6)\).
\end{remark}

\begin{remark}[On $x$–dependence of the amplitude]
In the main text we use \(a(t,\xi)\), independent of \(x\). If needed one may insert
a window \(\chi(x/\lambda^{1/2})\) and take \(a(t,x,\xi)=\omega(t)\,\vartheta(\xi)\,\chi(x/\lambda^{1/2})\). Then applying
the operator \(L_{x'}\) to \(\chi\) yields an additional gain \(\lambda^{-1/2}\) on top of the basic \((\lambda\alpha)^{-1}\);
the final exponent in \(\lambda\) only improves (see App.~\ref{app:damp}).
\end{remark}

%------------------------------------------------------------------
\subsection{Frequency localization and transverse gradients}\label{subsec:gradients}
%------------------------------------------------------------------

\paragraph{Partition by the size of $\mu_6(\xi):=|\partial_t\Phi_6|$.}
Let $\Phi_6$ be the six–frequency phase of the kernel, and 
\[
\mu_6(\xi_1,\dots,\xi_6)
:=\Bigl|\sum_{m\le3}|\xi_m|^2-\sum_{m>3}|\xi_m|^2\Bigr|
\quad(\text{independent of }t).
\]
Split $\Theta^6=\mathcal B_{\ge}\,\dot\cup\,\mathcal B_{<}$, where
\[
\mathcal B_{\ge}:=\{\mu_6 \ge c\,\lambda^{1/2}\},\qquad
\mathcal B_{<}:=\{\mu_6 < c\,\lambda^{1/2}\}.
\]

\begin{lemma}[Quantitative lower bound in broad--3]\label{lem:Blt-quant}
Let $\Theta_1,\Theta_2,\Theta_3$ be $\alpha$–caps in the \emph{broad--3} regime, i.e.
$\inf_{\xi_m\in\Theta_m}\|n(\xi_1)\wedge n(\xi_2)\wedge n(\xi_3)\|\ge c_{\mathrm{tr}}\alpha^2$.
Then for any $(t,x)\in Q_\lambda$ and any $\xi\in\Theta_1\cup\Theta_2\cup\Theta_3$
\[
  \mu(\xi)=|\partial_t\Phi(t,x;\xi)| \ \gtrsim\ \lambda^{1/2}\alpha^2 .
\]
\end{lemma}

\noindent Note that the lower bound in Lemma \ref{lem:Blt-quant} is \emph{uniform in $(t,x)\in Q_\lambda$}:
the constants are controlled via the triple minor of normals (see \S\ref{app:A5}--\S\ref{app:A6}) and do not depend on the position in $Q_\lambda$.

\begin{proof}[Idea of the proof]
Pass to $\tau,\zeta,\eta$; compute $\partial_\tau\Phi=|\eta|^2$ and
$\partial_\eta\Phi=\zeta+2\tau\eta$; apply \ref{app:A5}–\ref{app:A6} to relate the triple minor of normals
to the nondegeneracy of phase gradients in $(\tau,\eta)$; see also §\ref{sec:broad3}.
\end{proof}


\begin{lemma}[R/N dichotomy on $B_{<}$]\label{lem:near-res-dichotomy}
Let $\xi\in\Theta^6$ be such that $\mu_6(\xi)=|\partial_t\Phi_6(\xi)|<c\,\lambda^{1/2}$
(i.e. $\xi\in\mathcal B_{<}$). Then the following alternative holds:
\begin{enumerate}\setlength\itemsep{2pt}
\item[\textup{(R)}] (\emph{robust}) there exists an $\alpha$–cap that contains $>c_\ast D$ centers of active caps 
(in the sense of §\,\ref{subsec:kakeya-density}); the Robust–Kakeya block is activated (§\,\ref{sec:kakeya}).
\item[\textup{(N)}] (\emph{narrow}) there exists an $O(\alpha)$–cluster containing at least $5$ out of the $6$ frequencies of the sextuple; then the narrow cascade applies (§\,\ref{sec:narrow}).
\end{enumerate}
\end{lemma}

\begin{proof}[Idea of the proof]
From $\mu_6< c\,\lambda^{1/2}$ and the sign splitting we get pairwise radial proximity
$\bigl||\xi_m|-|\xi_{\pi(m)}|\bigr|\lesssim \mu_6/\lambda$ for some permutation $\pi\in S_3$.
Assume the six directions split into three pairwise separated $O(\alpha)$–clusters.
Then, using the bilipschitzness of $\xi\mapsto n(\xi)$ (see §\,\ref{subsec:lip}) and the estimates for
$4\times4$ minors/Gram matrices (App.\,\ref{app:A5}–\ref{app:A6}), we obtain a contradiction:
the mixed minors become too small relative to the principal ones when $\mu_6$ is small.
Hence either there is high angular density (branch (R)),
or at least five directions lie in a single $O(\alpha)$–cluster (branch (N)).
\end{proof}

\noindent\emph{Detail for the previous proof (radial proximity).}
Since $|\xi_m|,|\xi_{\pi(m)}|\sim\lambda$, we have
\[
\bigl||\xi_m|^2-|\xi_{\pi(m)}|^2\bigr|
=\bigl||\xi_m|-|\xi_{\pi(m)}|\bigr|\cdot\bigl(|\xi_m|+|\xi_{\pi(m)}|\bigr)
\ \ge\ \tfrac{\lambda}{2}\,\bigl||\xi_m|-|\xi_{\pi(m)}|\bigr|.
\]
Splitting $\sum_{m\le3}|\xi_m|^2-\sum_{m>3}|\xi_m|^2$ by signs and pairing the summands yields a permutation $\pi\in S_3$ such that
\[
\bigl||\xi_m|-|\xi_{\pi(m)}|\bigr|\ \lesssim\ \mu_6/\lambda,\qquad m=1,2,3.
\]
This “almost radial” tying triggers the (R)/(N) fork when $\mu_6$ is small.

\medskip
Next we estimate the contribution $K_{\ge}$ (the integral over $\mathcal B_{\ge}$). The contribution of the almost-resonant basket $K_{<}$
is absorbed by the blocks of \S\ref{sec:kakeya} (Robust Kakeya) and \S\ref{sec:narrow} (narrow cascade), used mutually exclusively.
On the support of the integral we take
\begin{equation}
\label{eq:phi-t-derivative}
\boxed{\ |\partial_t \Phi| \ \gtrsim\ \lambda^{1/2}\ }\quad(\xi\in\mathcal B_{\ge}).
\end{equation}

\noindent\textbf{Convention (IBP $\Rightarrow$ only on $\mathcal B_{\ge}$).}
Throughout §\ref{sec:kernel} integrations by parts in $t$ and in $x'$ are performed \emph{on the basket $\mathcal B_{\ge}$}.
The basket $\mathcal B_{<}$ is entirely handled by the blocks of §\ref{sec:kakeya} and §\ref{sec:narrow}.

\paragraph{Transverse dichotomy on \texorpdfstring{$\mathcal{B}_{\ge}$}{B\_ge}.}
Let $u_m := \xi'_m / |\xi_m|$ be the directions, and let $\Phi_6$ be the six–frequency phase of the kernel. 
On the basket $\mathcal{B}_{\ge}$ the following alternative holds:

\smallskip
\emph{(T) Transversal case:} there exists $c_1 = c_1(c_0) > 0$ such that
\begin{equation}
\label{eq:grad-xp}
   |\nabla_{x'}\Phi_6|
   = \Bigl|\sum_{m\le 3} \xi'_m - \sum_{m>3} \xi'_m\Bigr|
   \ \ge\ c_1\,\lambda\,\alpha
   = c_0 \lambda^{1/3} D^{1/2} \,=\, c_0 \lambda^{3/8}.
\end{equation}
In this case in the kernel block (\S\ref{sec:kernel}) six integrations by parts in $x'$ are applicable 
(see~\S\ref{subsec:IBP}).

\emph{(P) Paired case:} there is a permutation $\pi \in S_3$ with
\[
   \angle(u_m,\,u_{\pi(m)})\ \le\ C\,\alpha,
   \qquad
   \bigl||\xi_m| - |\xi_{\pi(m)}|\bigr| \ \lesssim\  \mu_6/\lambda,
   \qquad m=1,2,3.
\]
Then each of the three pairs lies in an $O(\alpha)$–cluster; in total the six frequencies
belong to the union of three $O(\alpha)$–classes. In the sum over active caps this leads either
to high angular density in a single $\alpha$–cap (Robust–Kakeya block, §\,\ref{subsec:kakeya-robust}),
or to the \emph{narrow} situation (at least five out of six in one $O(\alpha)$–cluster, §\,\ref{sec:narrow}).
These regimes are used mutually exclusively with the kernel block (§\,\ref{sec:kernel}).

\medskip
\noindent\textbf{Remark (transverse gradients in $TT^*$).}
For the phase difference $\Psi(t,x;s,y;\xi)=\Phi(t,x;\xi)-\Phi(s,y;\xi)$
\[
\boxed{\ \nabla_y\Psi(t,x;s,y;\xi)\ =\ -\,\nabla_{x'}\Phi(s,y;\xi)\ },
\]
hence on $\mathcal B_{\ge}$ from \eqref{eq:grad-xp} we obtain
\begin{equation}\label{eq:grad-y-psi}
|\nabla_y\Psi|\ \gtrsim\ \lambda\alpha\qquad(\text{and similarly for }y'),
\end{equation}
which is precisely what is used for four transverse IBPs at the $TT^*$ stage (§\,\ref{subsec:measures}).


%------------------------------------------------------------------
\subsection{IBP operators}\label{subsec:IBP}
%------------------------------------------------------------------

\paragraph{In time.}
Introduce the “structural” operator
\[
   L_t^{\mathrm{std}}
   := \frac{1}{i\,\partial_t\Phi}\,\partial_t,
   \qquad
   \bigl(L_t^{\mathrm{std}}\bigr)^{\!*} e^{\,i\Phi}=e^{\,i\Phi}.
\]
On the basket $\mathcal B_{\ge}$ we have $|\partial_t\Phi|\gtrsim \lambda^{1/2}$ (see~\eqref{eq:phi-t-derivative}),
and $\partial_t^{\,k}\omega(t)=O(\lambda^{3k/2})$. Therefore \emph{one} step of IBP with $L_t^{\mathrm{std}}$
produces a factor $\lambda^{+1}$ inside the integral. To fix the “scaling part,” we work
with the \emph{normalized} operator
\[
   L_t\ :=\ \lambda^{-1/2}\,L_t^{\mathrm{std}},
\]
so that one step of $L_t$ gives \(\lambda^{+1/2}\), and six steps — \(\lambda^{+3}\). This is just bookkeeping:
the identity $\bigl(L_t^{\mathrm{std}}\bigr)^{\!*}e^{i\Phi}=e^{i\Phi}$ remains valid, while the compensating physical
Jacobian appears at the Schur step (see~\eqref{eq:scale-jac} and §\ref{subsec:measures}).

\paragraph{In the transverse coordinates \(x'=(x_2,x_3)\).}
Set
\[
   L_{x'}\ :=\ |\nabla_{x'}\Phi|^{-2}\,\nabla_{x'}\Phi\cdot i\nabla_{x'},
   \qquad
   L_{x'}^{\!*} e^{\,i\Phi}=e^{\,i\Phi}.
\]
In the broad regime from~\eqref{eq:grad-xp} we have
\(|\nabla_{x'}\Phi|\gtrsim \lambda\alpha = c_0\,\lambda^{1/3}D^{1/2}\),
hence \(\|L_{x'}\|\lesssim \lambda^{-1/3}D^{-1/2}\).
Six integrations by parts in \(x'\) give
\[
   \bigl\|L_{x'}^{\,6}\bigr\|\ \lesssim\ (\lambda^{-1/3}D^{-1/2})^{6}
   \ =\ \lambda^{-2}D^{-3}.
\]

\begin{remark}[On $x$–dependence]
In the main text the amplitude \(a(t,\xi)\) does not depend on \(x\), and IBP in \(x'\) is realized
at the stage of integration in \((t,x)\) (Schur/TT$^\ast$): derivatives are transferred to the second
copy of the kernel or to the physical window. If desired, one may explicitly introduce a smooth
window \(\chi(x/\lambda^{1/2})\) (localization on \(Q_\lambda\)); then the action of \(L_{x'}\) on \(\chi\)
produces an \emph{additional} factor \(\lambda^{-1/2}\) on top of the basic \((\lambda\alpha)^{-1}\),
which only improves the $\lambda$–exponent (see App.~\ref{app:damp}).
\end{remark}

\noindent\emph{Edge layers of the window.}
Split the integral in $t$ into the plateau $\{|\partial_t\omega|=0\}$ and two edges where 
$|\{t:\partial_t\omega\ne0\}|\lesssim \lambda^{-3/2}$ (see Remark~\ref{rem:time-plateau}). 
On the plateau perform 6 IBPs in $t$ as in §\ref{subsec:IBP}. 
On the edges we do not integrate by parts: we estimate the contribution directly, 
using only $|\partial_t^k\omega|\lesssim \lambda^{3k/2}$ and the small measure of the edges; 
after 6 IBPs in $x'$ it falls under the same scale $\lambda^{-2}D^{-3}$ as the plateau contribution. 
Thus the “raw” Schur remains within \eqref{eq:Schur-raw}.

%------------------------------------------------------------------
\subsection{Separation of measures: Schur + TT$^\ast$}\label{subsec:measures}
%------------------------------------------------------------------
After six integrations by parts in $t$ and six in $x'$ (see §\ref{subsec:IBP})
we obtain the modified kernel $K^\#$. Passing to dimensionless variables
\[
   t=\lambda^{-3/2}\tau,\qquad x=\lambda^{-1/2}\zeta
\]
gives
\begin{equation}\label{eq:scale-jac}
   dt\,dx \;=\; \lambda^{-3}\,d\tau\,d\zeta, \qquad |Q_\lambda|\simeq \lambda^{-3}.
\end{equation}
In the Schur step \emph{only} the physical Jacobian \eqref{eq:scale-jac}
appears (angular/radial volumes are already accounted for in the measure $d\xi$),
while the factors from 12 IBPs
(six in $t$ and six in $x'$) are already \emph{built into the amplitude} of the modified kernel $K^\#$.

Combining the growth from $6L_t$ (see \eqref{eq:phi-t-derivative}) and the gain from $6L_{x'}$
(see \eqref{eq:grad-xp}), we obtain the “raw” Schur bound:
\begin{equation}\label{eq:Schur-raw}
  \|K\|_{2\to2}\ \lesssim\
  \underbrace{\lambda^{+3}}_{\text{$6L_t$}}\cdot
  \underbrace{\lambda^{-2}\,D^{-3}}_{\text{$6L_{x'}$}}\cdot
  \underbrace{\lambda^{-3}}_{\text{phys.\ Jacobian}}
  \;=\; \lambda^{-2}\,D^{-3}.
\end{equation}

\medskip
\noindent\textit{Factorization.}
After six integrations in $t$ and six in $x'$ the kernel can be written as
\[
   K^\#(t,x;s,y) \;=\; B(t,x;s,y) \cdot K_{\mathrm{phase}}(t,x;s,y),
\]
with $\|B\|_{L^\infty} \lesssim \lambda^{-2} D^{-3}$.
To the kernel $K_{\mathrm{phase}}$ with unit amplitude we apply the $TT^\ast$ step,
giving $\|T_{\mathrm{phase}}\|_{2\to 2} \lesssim \lambda^{-5/2}$.
Therefore,
\[
   \|K\|_{2\to 2} \;\le\; \|B\|_{L^\infty} \cdot \|T_{\mathrm{phase}}\|_{2\to 2}
   \ \lesssim\ \lambda^{-9/2} D^{-3}.
\]
Thus the final bound \eqref{eq:K-final} is obtained as
the product of the amplitude gain and the $TT^\ast$ contribution for the phase part.


\begin{lemma}[TT$^\ast$ with a smooth amplitude]\label{lem:TTstar-ampl}
Let the kernel be $K^\#(t,x;s,y)=B(t,x;s,y)\,K_{\mathrm{phase}}(t,x;s,y)$, where
\[
\sup_{|\beta|\le 2}\|\partial_{(t,x;s,y)}^\beta B\|_{L^\infty}\ \lesssim\ \lambda^{-2}D^{-3}.
\]
Then the operator $T$ with kernel $K^\#$ satisfies
\[
\|T\|_{2\to2}\ \lesssim\ \lambda^{-2}D^{-3}\cdot \|T_{\mathrm{phase}}\|_{2\to2}.
\]
In particular, if under the conditions of §\ref{subsec:measures} one has 
$\|T_{\mathrm{phase}}\|_{2\to2}\lesssim \lambda^{-5/2}$, then
$\|T\|_{2\to2}\lesssim \lambda^{-9/2}D^{-3}$.
\end{lemma}

\begin{proof}
Consider $(T T^\ast)f$ and integrate in $(s,y),(s',y')$ with integrations by parts in $s,s'$ and in the transverse variables, 
as in §\ref{subsec:measures}. Derivatives may also land on $B$, but by assumption all 
$\partial^\beta B$ with $|\beta|\le2$ remain $\lesssim \lambda^{-2}D^{-3}$, so each term in the 
expansion of $T T^\ast$ is dominated by the same constant. Note that with the chosen amplitude $a(t,\xi)$ (no $x$/$y$ dependence) the factor $B(t,x;s,y)$ is independent of $y,y'$, and in $s,s'$ we perform only one IBP each; hence at most two time derivatives hit $B$, and the control $\sup_{|\beta|\le 2}\|\partial^\beta B\|_\infty\lesssim \lambda^{-2}D^{-3}$ suffices. Separately we apply integration by parts to the phase component, giving $\|T_{\mathrm{phase}}\|_{2\to2}\lesssim \lambda^{-5/2}$ (see \eqref{eq:TTstar}). 
A standard Cauchy–Schwarz argument on kernels yields the stated inequality.
\end{proof}

\paragraph{TT$^\ast$ bound after 12 IBP.}
By the convention of §\ref{subsec:gradients}, all IBPs are performed on the basket $\mathcal B_{\ge}$ (see \eqref{eq:phi-t-derivative}, \eqref{eq:grad-xp}).
\noindent
Note that $TT^\ast$ is applied to the \emph{modified} kernel $K^\#$,
obtained \emph{after} 12 integrations by parts (six in $t$ and six in $x'$) and after accounting for
the physical Jacobian \eqref{eq:scale-jac}. Thus the factor $\lambda^{-5/2}$ from the $TT^\ast$ step
complements the amplitude gain $\lambda^{-2}D^{-3}$ recorded in \eqref{eq:Schur-raw};
we do not multiply independent upper bounds for the same operator.
Let $T$ be the operator with kernel $K^\#$, obtained after six IBPs in $t$ 
(producing $\lambda^{+3}$, see~\eqref{eq:phi-t-derivative}) and six IBPs in $x'$ 
(producing $\lambda^{-2}D^{-3}$, see~\eqref{eq:grad-xp}), as well as accounting for the physical Jacobian 
$\lambda^{-3}$ from \eqref{eq:scale-jac}. These factors are part of the amplitude of $K^\#$.

We estimate $\|T\|_{2\to2}^2=\|TT^\ast\|_{2\to2}$. 
On the plateau in $s,s'$ (see Remark~\ref{rem:time-plateau}) one IBP in each of these variables 
gives $\lambda^{-2}$ at the level of the squared norm. 
Then, in two independent transverse directions in $y$ and two in $y'$, we perform 
four more IBPs, giving $(\lambda\alpha)^{-8}=\lambda^{-3}$ at the level of the square.
Thus,
\[
  \|T\|_{2\to2}^2 \ \lesssim\ 
  \underbrace{\lambda^{+3} \cdot \lambda^{-2} D^{-3} \cdot \lambda^{-3}}_{\text{12 IBP + Jacobian}}
  \cdot
  \underbrace{\lambda^{-2} \cdot \lambda^{-3}}_{\substack{\text{IBP in $s,s'$}\\\text{and in transverse}}}
  \cdot O(1).
\]

\noindent\textit{Vector fields in $TT^\ast$.}
Set
\[
L_s:=\frac{1}{i\,\partial_s\Psi}\,\partial_s,\quad 
L_{s'}:=\frac{1}{-i\,\partial_{s'}\Psi}\,\partial_{s'},\quad
L_y:=\frac{\nabla_y\Psi}{i\,|\nabla_y\Psi|^2}\cdot\nabla_y,\quad
L_{y'}:=\frac{-\nabla_{y'}\Psi}{i\,|\nabla_{y'}\Psi|^2}\cdot\nabla_{y'}.
\]

\noindent\emph{Relation of transverse gradients.}
Since $\Psi(t,x;s,y;\xi)=\Phi(t,x;\xi)-\Phi(s,y;\xi)$, we have
\[
\nabla_y\Psi(t,x;s,y;\xi)=-\nabla_{x'}\Phi(s,y;\xi),
\]
hence on the basket $\mathcal B_{\ge}$ from \eqref{eq:grad-xp} we obtain
\[
|\nabla_y\Psi|\ \gtrsim\ \lambda\alpha\qquad(\text{and similarly for }y').
\]

Then $L_s^\ast e^{i\Psi}=e^{i\Psi}$ and so on. One IBP in $s$ and in $s'$ gives $\lambda^{-2}$ 
at the level of the squared norm, and two independent IBPs in $y$ and in $y'$ give $(\lambda\alpha)^{-8}=\lambda^{-3}$ 
at the level of the square (see \eqref{eq:phi-t-derivative}–\eqref{eq:grad-xp}), hence \(\lambda^{-5}\) on the square 
and \(\lambda^{-5/2}\) after taking the square root; see \eqref{eq:TTstar}.

\begin{remark}[On the independence of time integrations]
In the Schur step we used $6$ integrations in $t$ in the kernel $K$ (variable $t$).
At the $TT^{*}$ stage we perform one IBP in each of $s$ and $s'$ for the 
modified kernel $K^{\#}$; these differentiations are independent of $t$ 
and act at a different level of the quadratic form.
Thus the “time gain” is not counted twice.
This is consistent with the extraction of the $\lambda^{-5/2}$ factor in the $TT^{*}$ step.
\end{remark}



After the $TT^\ast$ step we obtain
\begin{equation}\label{eq:TTstar}
  \|T\|_{2\to 2} \ \lesssim\ \lambda^{-5/2}.
\end{equation}

Combining \eqref{eq:Schur-raw} and \eqref{eq:TTstar}, we conclude
\begin{equation}\label{eq:K-final}
  \|K\|_{2\to2}\ \lesssim\
  \underbrace{\lambda^{-2}D^{-3}}_{\text{Schur after 12 IBP + physical Jacobian}}
  \cdot
  \underbrace{\lambda^{-5/2}}_{\text{$TT^\ast$ contribution}}
  \;=\; \lambda^{-9/2}D^{-3}.
\end{equation}

\begin{remark}[On the independence of transverse IBP]
The six IBPs in $x'$ for the kernel $K$ are performed before forming $K^\#$ and act on the variables $(t,x)$.
The four IBPs in $TT^\ast$ are in the transverse variables $y,y'$ of the other copy of the kernel. 
Therefore the transverse gain in $TT^\ast$ is not counted twice.
\end{remark}

\begin{remark}[Where exactly the factor $\lambda^{-5/2}$ comes from]\label{rem:l-52}
One IBP in each of $s,s'$ gives a factor $\lambda^{-2}$ \emph{at the level of the square} (after taking the square root — $\lambda^{-1}$),
and four transverse IBPs in total give $(\lambda\alpha)^{-8}=\lambda^{-3}$ \emph{at the level of the square}
(after the square root — $\lambda^{-3/2}$). Thus $\lambda^{-1}\cdot\lambda^{-3/2}=\lambda^{-5/2}$; see also
\eqref{eq:phi-t-derivative}–\eqref{eq:grad-xp}.
\end{remark}

\begin{remark}[Separation of regimes]\label{rem:regimes}
The almost-resonant basket $\mathcal B_{<}$ is handled by the blocks of §\ref{sec:kakeya} and §\ref{sec:narrow},
which are used \emph{mutually exclusively}; the estimate \eqref{eq:K-final} applies only on $\mathcal B_{\ge}$.
\end{remark}

%------------------------------------------------------------------
\subsection{Kernel: final bound}\label{subsec:kernel-final}
%------------------------------------------------------------------
\begin{proposition}\label{prop:kernel-final}
For $\lambda\ge 2$ one has
\[
   \|K\|_{L^{2}\to L^{2}} \;\lesssim\; \lambda^{-9/2}\,D^{-3},\qquad D=\lambda^{1/12}.
\]
\end{proposition}
