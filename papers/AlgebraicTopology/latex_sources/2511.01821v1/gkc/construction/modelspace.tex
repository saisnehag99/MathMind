\section{The global Kuranishi chart construction}\label{sec:prelim-aux}

\subsection{Base space}\label{subsec:base-space} Fix integers $n\geq 0$ and $d > 0$. We write
$$\cB = \cB(d) \sub \Mbar_0(\bP^d,d)$$ for the locus of (equivalence classes of) regular embedded stable holomorphic maps $\varphi \cl C\to \bP^d$ of genus $0$ and degree $d$. In particular, $\im(\varphi)$ is not contained in a complex hyperplane of $\bP^d$ and $\varphi$ admits no nontrivial automorphisms. Thus, $\cB$ is a smooth quasi-projective variety and served as the base space of previous global Kuranishi chart constructions, \cite{AMS21,HS22,AMS23}. Given nonnegative integers $n^\pm\ge 0$, we let $$\cB_{n^+,n^-} = \cB_{n^+,n^-}(d)$$ be the preimage of $\cB$ under the forgetful map $\Mbar_{0,n^+,n^-}(\bP^d,d)\to \Mbar_0(\bP^d,d)$, where the $(n^++n^-)$-many marked points are divided into positive and negative marked points. It is an easy verification that this is again a complex manifold.\par 
While the spaces $\cB$ and $\cB_{n}$ capture the domain breaking of stable maps, they do not capture the degenerations of buildings, where nodes are constrained by the asymptotic conditions. Thus, we construct a real-oriented blow-up of $\cB$ that will serve as the base space for the Kuranishi chart of Pardon buildings; see \textsection\ref{subsec:framings}. It agrees with the base space used in Hamiltonian Floer theory, \cite{BX22,Rez22,AB24}. In \textsection\ref{subsec:base-with-levels}, we perform a generalized corner blow-up to obtain the base space for the Kuranishi chart of leveled buildings.

\subsubsection{Without levels}\label{subsec:framings} We will perform real-oriented blow-ups on the space $\cB_{n^+,n^-}(d)$ following \cite{BX22}. These spaces will serve as the base spaces for the global charts of Pardon buildings in \ref{subsec:construction}. Assume we are given an integer $d_i^\pm\geq 1$ for each marked point $z_i^\pm$ so that
    $$d =  p\lbr{\s{i}{d^+_i}-\s{j}{d^-_j}} + n - 2$$
for some uniform integer $p \ge 1$ where $n$ is the number of marked points.
The following definition will allow us to modify $\cB_{n^+,n^-}(d)$ to obtain the right boundary stratification for curves in symplectizations. 

\begin{definition}\label{de:type-for-symplectisation} Given a stable map $\varphi \cl C\to \bP^d$ whose domain has a unique node $x$, we say that $x$ is of \emph{type $0$} if it is non-separating or if it separates $C$ into irreducible components $C_0$ and $C_1$ of degree $d_0$, respectively $d_1$ so that 
\begin{equation}\label{eq:degree-comparison} \mathrm{d}_x:=
(d_0-p\hspace{-3pt}\s{z^+_i \in C_0}{d_i^+}+p\hspace{-3pt}\s{z^-_j \in C_0}{d_j^-}  -\deg(\omega_{C}|_{C_0})) - (d_1-p\hspace{-3pt}\s{z^+_i \in C_1}{d_i^+}+p\hspace{-3pt}\s{z^-_j \in C_1}{d_j^-} -\deg(\omega_{C}|_{C_1})) = 0.
\end{equation}
We say $x$ is of \emph{type $1$} with order $|\mathrm{d}_x|$ otherwise.\end{definition}

The following does not depend on the way we determine the type of the node. Given $[\varphi,C,z_*] \in \cB_{n^+,n^-}(d)$ whose domain has two nodes $x_1,x_2$, we call $x_i$ of \emph{type $j$} if $[\varphi,C,z_*]$ is the limit of a sequence of maps with one-nodal domains so that the node converges to $x_i$ and each of them is of type $j$. Inductively, this yields a decomposition $N_C = N_C\inn \sqcup N_C^\bullet$ of the nodes of the domain of any $[\varphi,C,z_*]\in \cB$.

\begin{definition}\label{de:sft-base-space}
    The space $\cBR_{n^+,n^-}(d)$ consists of elements of $\cB_{n^+,n^-}(d)$ equipped with the following additional data
 \begin{itemize}[leftmargin=20pt]
	\item an asymptotic marker $b_{k}^\pm \in S_{z_k^\pm}C$ at each marked point $z_k^\pm$ determining a real line $\ell_k^\pm \sub T_{z_k^\pm}C$,
	\item an isomorphism $m_z \cl S_zC_v \to S_zC_{v'}$, or, equivalently, an element of $(T_zC_v\otimes T_zC_{v'}\sm \{0\})/S^1$, for each $z\in \cN_C^\bullet$ between the irreducible components $C_v$ and $C_{v'}$ of $C$.\footnote{To see this equivalence, note that $\ell = [\xi \otimes \xi']$ yields a map $m_\ell([\eta]) := [\lspan{\xi,\wt\eta}\xi']$ where $\lspan{\cdot}$ is any Hermitian product on $T_zC$, while conversely given $m$ we can set $\ell_m := [\wt\xi \otimes \wt m(\xi)]$ where $\xi \in S_zC$ is arbitrary and the tilde means that we take a lift in the respective tangent space.} 
\end{itemize}
\end{definition}

For the next assertion, we recall the construction of the real-oriented blow-up of a complex manifold $Z$ along an effective normal crossing divisor $D$ from \cite[\textsection8.2]{Sab13}. Suppose first $D$ is smooth and principal, and let $S_D \to Z$ be the $S^1$-bundle of the line bundle $L_D \to Z$ associated to $D$. Let $f \cl Z\to L_D$ be a holomorphic section. Then $f\inv(0) = D$, so the composition $\wt f$
$$Z\sm D\xra{f} L_D\sm 0 \to S_D$$
is well-defined. The \emph{real-oriented blow-up} $\text{Bl}_Z(D)$ of $Z$ along $D$ is the closure of the image of $\wt f$ and the blow-down map $\text{Bl}_Z(D) \to Z$ is simply given by the restriction of the projection $S_D\to Z$.
By \cite[Lemma~8.1]{Sab13}, the space $ \text{Bl}_Z(D)$ admits a unique smooth structure with 
\begin{equation}\label{eq:boundary-blow-up}
    \del\text{Bl}_Z(D) \cong S_D|_D.
\end{equation}

\par 
If $\{D_i\}_i$ is a normal crossing divisor with finitely many irreducible components, then there exists a real blow-up of $Z$ along $\{D_i\}_i$. By \cite[Lemma~8.2]{Sab13}, it is given by
\begin{equation}\label{eq:ncd-blowup}\text{Bl}_Z(\{D_i\}_i) = \text{Bl}_Z(D_1)\times_{Z}\dots \times_Z\text{Bl}_Z(D_k).\end{equation}
which we can take as the definition for the purposes of this paper.

\begin{remark}
    In our construction we also encounter the case of blowing up a divisor $D$, which has normal crossings self-intersections, i.e., any point of $D$ admits a \emph{holomorphic} chart $\phi\cl U \to Z$ where $U$ is an open neighborhood of $0$ in $\bC^n$ and $\phi\inv(D) = U \cap \{z_1\dots z_k = 0\}$. 
    We define the smooth structure on the real-oriented blow-up, $\text{Bl}_Z(D)$ via Equation \eqref{eq:ncd-blowup} in each such coordinate chart $U$. As the transition functions on the base are holomorphic, they lift to diffeomorphisms between the blow-ups.
\end{remark}

\begin{lemma}\label{lem:building-base-smooth} $\cBR_{n^+,n^-}(d)$ is a smooth $\PGL_{d+1}(\bC)$-manifold with corners so that the depth in the boundary of $(\varphi,C,z_*,b_*,m_*)$ is given by the number of elements in $\cN^\bullet_C$. Moreover, each corner stratum is invariant under the $\PGL_{d+1}(\bC)$-action. 
\end{lemma}

\begin{proof}
	We abbreviate $\cBR = \cBR_{n^+,n^-}(d)$ and $\cB = \cB_{n^+,n^-}(d)$. Recall that $\bL_i \to\cB$ is the complex line bundle with fiber given by $T_{z_i}^*C$ at $(\varphi,C,z_*)$. Let 
	$$\cB^{am}:= \bigoplus\limits_{i =1}^{k^+}S(\bL\dul_{i})\oplus  \bigoplus\limits_{j =1}^{k^-}S(\bL\dul_{j})$$
	be a direct sum of $U(1)$ bundles corresponding to the line bundles $\bL_*$ and let $\pi_{am} \cl \cB^{am}\to \cB$ be the forgetful map. 
    Then $\cB^{am}$ is clearly smooth (without corners) and $\pi_{am}$ is a submersion.\par Let $D_1,\dots,D_\ell$ be the divisors in $\cB$ given by curves with at least one node in $\cN^\bullet$. They form a normal crossing divisor, so we can define $\cB^{mi} := \text{Bl}_\cB(\{D_i\}_i)$ with blow-down map $\pi_{mi}\cl \cB^{mi}\to \cB$. We claim that 
    \begin{equation}
        \cBR = \cB^{am}\times_\cB\cB^{mi}
    \end{equation}
    and that the forgetful map $\cBR\to \cB$ is induced by the blow-down map $\cB^{mi}\to \cB$. To see this, we note that the asymptotic markers at the marked points correspond exactly to the additional data of elements of $S(\bL\dul_{j})$. Forgetting them yields the canonical map $\cBR\to \cB^{mi}$. Meanwhile, the normal bundle of $D_i$ has fiber $N_{\varphi} = \bL_{z^+}\dul\otimes \bL_{z^-}\dul$ by \cite[\textsection XI.3]{ACG11}. Thus an element of its sphere bundle is exactly a matching isomorphism, so the claim without the group action follows from \eqref{eq:boundary-blow-up}.
    Finally, by \cite[Theorem~5.1]{AK10}\footnote{To be precise, the statement is for projective blow-up while we work with the spherical one. However, as mentioned in the paragraph above Remark~1.1 op. cit., their results also hold for spherical blow-ups.}, the smooth $\PGL_{d+1}(\bC)$-action on $\cB$ lifts to a smooth action on $\cBR$. For the last assertion, it suffices to show that each stratum of codimension $1$ is invariant under the $\PGL$-action. Let $S$ be such a stratum, and note that $S$ is a connected component of the space $S_1(\cBR_d)$ of codimension-$1$ points in $\cBR_d$. Since $S_1(\cBR_d)$ is preserved by the action and $\PGL_{d+1}(\bC)$ is connected, the orbit of $S$ under this action is connected as well. Hence, it has to agree with $S$.
\end{proof}

Returning to the notation of \S\ref{subsec:buildings}, recall that $\scS$ is the category of decorated trees defined in Definition~\ref{de:decorated-tree}. Let $\scS\inn$ be the category obtained from $\scS$ by forgetting the data of the Reeb orbits associated to the edges and replacing the relative homology class $\beta_v \in H_2(Y,\{\gamma_{v,e}\}_{e\in E_v})$ by an integer $d_v \geq 1$, corresponding to the degree of the map $\wch\varphi_v \cl  C_v \to \bP^d$, where $\wch\varphi$ is the image of $\varphi$ under the blow-down map $\scBR\to \cB$. We equip $\scS\inn$ with the dimension function 
\begin{equation}\label{de:dimension-tree}
    \dim(T) = 2(d-3)+ 2(d+1)d+3(\# \Gamma^- + \# \Gamma^+)  -\# E^{\text{int}}(T)
\end{equation}

\vspace*{1.5pt}

\begin{lemma}\label{lem:stratification-base-space}
There exists a canonical stratification $P\cl \cBR_{n^+,n^-}(d) \to \scS\inn$ which assigns the tree type of the domain to an element of $\cBR_{n^+,n^+}(d)$. It is cell-like in the sense that $\cBR_{/T} := P(\scS\inn_{/T})$ is a smooth manifold with corners of dimension $\dim(T)$ with interior given by $P\inv(\{T\})$.
\end{lemma}

\begin{proof} Define the function $P^\bR$ by letting $T_{\wh\varphi} = P^\bR(\wh\varphi)$ be the underlying dual graph of the domain of $\pi(\wh\varphi)$, where we have collapsed all edges corresponding to nodes of type $0$ (and added the degrees of the associated vertices). Given an element in (the preimage of) a gluing chart (under $\pi$) near $\wh \varphi$, we obtain a unique morphism $T_{\wh\varphi}\to T_{\wh\phi}$ in $\scS\inn$. Since $\cB$ is unobstructed, the induced germ of a map $\cBR\to \scS\inn_{T_{\wh\varphi}/}$ near $\wh\varphi$ is a stratification. The last claim follows from the dimension formula for $\cB$ and the description of the corner strata of the real blow-up above. Observing that $\cBR\to \cB^{mi}$ is a torus bundle over $\cB^{mi}$, this completes the proof.\end{proof}


 


