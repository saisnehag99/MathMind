\section{Introduction} 

One of the best understood examples of triangulated categories is the derived category of representations $\mathsf{D}^b(kQ)$, for $Q$ a quiver  and $k$ a field. This is largely due to Happel's work \cite{Hap87,Hap88} and his use of methods coming from representation theory ---mainly \emph{Auslander-Reiten theory} \cite{AusReiSma97}--- to find quite explicit combinatorial descriptions of such categories. Encoded in the Auslander-Reiten quiver $\Gamma(\mathsf{D}^b(kQ))$ of the derived category there is a plethora of information. For instance, it reveals that derived equivalent quivers must have the same unoriented underlying graph. Moreover, many relevant functors can be read from this quiver ---reflection functors, Serre functors, suspension, etc.--- and as shown by Miyachi and Yekutieli \cite{MiyYek01}, in some cases one can even recover all symmetries of $\mathsf{D}^b(kQ)$ (namely the derived Picard group) from those of $\Gamma(\mathsf{D}^b(kQ))$. 

However, beyond the setting of representations over a field, little is known when considering more general coefficients, such as the integers or arbitrary commutative rings. Our approach to the subject is rather radical but well-founded: it has recently been observed that certain well-known symmetries of categories of representations are actually mere consequences of the stability ---in the sense of homotopy--- of the categories involved, and so they exist in a much broader generality, often for the corresponding representations in any stable homotopy theory. This paper seeks to contribute to the development of an \emph{abstract representation theory} of quivers in the sense of Groth and Stovicek \cite{GroSto16,GroSto16b,GroSto18,GroSto18b} ---see also our \cref{sec:abstract-rep}. In particular, we aim to introduce techniques from Auslander-Reiten theory into the study of categories of representations over abstract stable homotopy theories, ranging from derived categories of arbitrary rings, schemes or dg algebras to the homotopy category of spectra. 

There are several approximations to formalizing the abstract notion of a stable homotopy theory. The most classical one of a triangulated category fails badly for our purposes: representations ---i.e., functors--- in a triangulated category form no longer a triangulated category, and moreover, the lack of essential functoriality properties in such categories obstructs some of the most fundamental constructions in representation theory. We choose to work within the framework of \emph{stable $\infty$-categories} \cite{Lur17}, where a well-behaved notion of homotopy coherent representations is at our disposal. For a quiver $Q$ and a stable $\infty$-category $\mathcal{C}$, one can simply consider the $\infty$-category of functors $\mathcal{C}^Q$ which is again stable. When specialized to the derived $\infty$-category $\D{R}$ ---the $\infty$-categorical enhancement of the ordinary derived category of a ring $R$---, one precisely recovers derived representations: 
\begin{equation} \label{eq:derived-cat}
\D{R}^Q \simeq \D{RQ}
\end{equation}
We should mention that similar statements can be formulated using alternatives to higher category theory. For instance, the foundational work of Groth and Stovicek is written in the language of (stable) derivators. However, the use of certain gluing operations between categories of representations, which play a crucial role in this paper, naturally led us to work in the more flexible setting of $\infty$-categories. In a similar spirit, there are already the works \cite{DycJasWal19,DycJasWal21}.

We study abstract stable $\infty$-categories of representations via \emph{coherent diagrams} of the shape of the \emph{Auslander-Reiten quiver}.
More precisely, for the repetitive quiver $\ZQ$ (i.e., the component of the Auslander-Reiten quiver containing the indecomposable projectives), we show that there is a natural equivalence of $\infty$-categories
\begin{equation} \label{eq:intro-mesh}
\mathcal{C}^Q \xrightarrow{ \ \simeq \ } \mathcal{C}^{\ZQmesh}
\end{equation}
with a certain \emph{mesh $\infty$-category} of representations of $\ZQ$ (\Cref{theo:ZQ-mesh}). Along the way, we give a new construction of abstract reflection functors ---alternative to the one given in \cite{DycJasWal21}--- via an adequate gluing of a source (\Cref{theo:Qpv-mesh}) or a sink (\Cref{theo:Qmv-mesh}) to the stable $\infty$-category of representations. When iterated, these yield the equivalence \eqref{eq:intro-mesh}. Our result generalizes and refines one of the main results in \cite{GroSto16}, which only considers Dynkin quivers of type A.

This new perspective offers immediate advantages. While the quiver $Q$ itself may possess only trivial automorphisms, $\ZQ$ has a ton of non-trivial ones. This allows us to transfer these symmetries encoded in the Auslander-Reiten quiver to the $\infty$-category of representations over $Q$. 
We thus obtain abstract versions of key functors in representation theory: reflection functors, Auslander-Reiten translations, Serre functors, etc. ---indeed, we show that when specialized to $\mathcal{C} = \D{k}$ of a field $k$, one recovers the classical version of these. Moreover, we prove using the above equivalence \eqref{eq:intro-mesh} that the group of automorphisms of $\ZQ$ naturally acts on any stable $\infty$-category of representations $\mathcal{C}^Q$ (\Cref{theo:actionZQ}):
\begin{equation}
\mathrm{Aut}(\ZQ) \ \rotatebox[origin=c]{-90}{$\circlearrowright$} \ \mathcal{C}^Q.
\end{equation}
In practice, we work with a possibly bigger subquiver\footnote{If $Q$ is Dynkin, then $\Gamma_Q = \Gamma(\mathsf{D}^b(kQ)) \cong \mathbb{Z}Q$, otherwise, $\Gamma_Q \cong \mathbb{Z}\times\mathbb{Z}Q$.} $\Gamma_Q \subseteq \Gamma(\mathsf{D}^b(kQ))$ of the Auslander-Reiten quiver, which allows to read the suspension functor, and a group homomorphism 
\begin{equation} \label{eq:intro-action}
\mathrm{Aut}(\Gamma_Q) \xrightarrow{ \ \ } \mathrm{Aut}(\mathcal{C}^Q).    
\end{equation}

Perhaps the most notable application of the methods introduced is that they contribute to the computation of \emph{spectral Picard groups} of quivers. Given finite acyclic quivers $Q$ and $Q'$, it is already known when they have equivalent representations over any stable $\infty$-category. For instance, the following are equivalent (\Cref{theo:equiv-quivers}):
\begin{enumerate}[(a)]
    \item $\mathcal{C}^Q \simeq \mathcal{C}^{Q'}$ for every $\mathcal{C}$ stable $\infty$-category.
    \item $\mathsf{D}^b(kQ) \simeq \mathsf{D}^b(kQ')$ as triangulated categories over a field $k$.
    \item $\Sp^Q \simeq \Sp^{Q'}$, where $\Sp$ denotes the $\infty$-category of spectra.
    \item $Q$ and $Q'$ are related by a sequence of source or sink reflections.
    \item $Q$ and $Q'$ have isomorphic repetitive quivers $\mathbb{Z}Q \cong \mathbb{Z}Q'$.
\end{enumerate}
This follows already from Happel's work and the existence of abstract reflection functors (\Cref{coro:reflections} or \cite{DycJasWal21}). Therefore, the next natural step is to try to classify these equivalences, and this leads to the investigation of the corresponding groups of autoequivalences. There is one among them which is in some sense \emph{universal}: this is the spectral Picard group $\mathsf{Pic}_\Sp(Q)$, which encodes all autoequivalences of $\Sp^Q$. 
For instance, we show that every element of the spectral Picard group induces natural autoequivalences $\mathcal{C}^Q \simeq \mathcal{C}^Q$ for all $\mathcal{C}$ stable $\infty$-categories (\Cref{coro:sp-equiv-quivers}).

In this paper, we formulate Picard groups using a new definition of \emph{spectral path algebras} (\Cref{def:path-algebra}): for a commutative ring spectrum $R$, the $R$-algebra spectrum $RQ$ is such that there is a canonical equivalence $\Mod_R^Q \simeq \Mod_{RQ}$, analogous to \eqref{eq:derived-cat}. The Picard group $\mathsf{Pic}_R(Q)$ is that of invertible $RQ$-bimodules over $R$. We explain that autoequivalences induced from symmetries of $\Gamma_Q$ are represented by such bimodules, and so the action \eqref{eq:intro-action} induces a group homomorphism
\begin{equation} \label{eq:homo-pic}
\mathrm{Aut}(\Gamma_Q) \xrightarrow{ \ \ } \mathsf{Pic}_R(Q).
\end{equation}
When $R$ is at least connective, we show that \eqref{eq:homo-pic} is a split monomorphism, i.e. $\mathsf{Pic}_R(Q)$ contains $\mathrm{Aut}(\Gamma_Q)$ as a semidirect factor (\Cref{coro:split-mono}). Moreover, when $Q$ is a tree, we obtain a very conceptual explanation of the theory developed: in this case, \eqref{eq:homo-pic} gives $\mathrm{Aut}(\Gamma_Q) \cong \mathsf{Pic}_{\D{k}}(Q)$ for any field $k$ \cite{MiyYek01}, and it follows that the canonical homomorphism 
\begin{equation}
Hk \otimes -: \mathsf{Pic}_\Sp(Q) \xrightarrow{ \ \ } \mathsf{Pic}_{\D{k}}(Q)  
\end{equation}
is a split epimorphism. That is, every symmetry that we find in the representation theory of $Q$ over a field is a shadow of its counterpart in the abstract/spectral setting. In particular, this gives an important generalization of one of the main results in \cite{GroSto16} from Dynkin quivers of type A to all finite trees.

The content of the sections in this paper is as follows. In \cref{sec:preliminaries} we give the preliminaries on $\infty$-categories that we need to develop our constructions, with particular emphasis on stable $\infty$-categories. In \cref{sec:abstract-rep} we give context for the rest of the paper: we introduce the $\infty$-categories of representations we are interested in, their natural notion of equivalence and an overview of what is known about them. Since they will be used to build the main equivalence in the next section, we give in \cref{sec:reflections} a new construction of abstract BGP reflection functors by means of a gluing argument. In \cref{sec:equiv-repet} we explain homotopy coherent Auslander-Reiten diagrams in the derived category of a quiver over a field and give an abstract version of them over arbitrary stable $\infty$-categories, introducing the mesh $\infty$-category of $\ZQ$ and proving its equivalence \eqref{eq:intro-mesh} with representations of $Q$. In \cref{sec:actions} we use the previous result to produce autoequivalences of representations of $Q$ from automorphisms of the irregular Auslander-Reiten quiver, which assemble into the action group homomorphism \eqref{eq:intro-action}. Finally, in \cref{sec:picard} we mainly focus on the representations over an $\mathbb{E}_\infty$-ring $R$, showing that the equivalences produced land in the Picard group over $R$ and using the induced action to reveal new structural properties of these Picard groups.

\subsection*{Acknowledgments} I would like to thank my supervisors Manuel Saorín and Jan Stovicek for their guidance and constant help, and for offering me the incredible opportunity to work in this project.