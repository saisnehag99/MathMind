\subsection{The motivic periods}
\label{SS: motivic periods}
In this subsection, we compute the pairing $\langle \Omega, \tilde{v}_{D} \rangle_{B}$ defined in Lemma \ref{lemma: abstrac paring with Omega}. 

Recall that $V = V^{a, b}\{r, s\}$ and we have the following exact sequence: 
\begin{equation}
\label{eq: def of Ext1 in section 9}
    0 \rightarrow F^{0}M_{dR}(\pi_f, V(2))_{\RR} \rightarrow M_{B}(\pi_f, V(2))^{-}_{\RR}(-1) \rightarrow \Ext^{1}_{\mathrm{MHS_{\RR}^{+}}}(\RR(0), M_{B}(\pi_f, V(2))_{\RR}) \rightarrow 0.
\end{equation}
The Beilinson $E(\pi_f)$-structure of $\mathrm{det}_{E(\pi_f)\otimes_{\QQ}\RR}\Ext^{1}_{\mathrm{MHS_{\RR}^{+}}}(\RR(0), M_{B}(\pi_f, V(2))_{\RR})$ is defined as  
\begin{equation*}
    \mathcal{B}(\pi_f, V(2)) = \mathrm{det}_{E(\pi_f)} F^{0}M_{dR}(\pi_f, V(2))^{*} \otimes_{E(\pi_f)} \mathrm{det}_{E(\pi_f)} M_{B}(\pi_f, V(2))^{-}_{\RR}(-1). 
\end{equation*}
Let $\delta(\pi_f, V(2)) \in (E(\pi_f) \otimes_{\QQ} \CC)^{\times}$ be the determinant of the isomorphism 
\begin{equation*}
I_{\infty} \colon M_{B}(\pi_f, V(2))_{\CC} \rightarrow M_{dR}(\pi_f, V(2))_{\CC}
\end{equation*}
computed using the basis defined over $E(\pi_f)$ on both sides. The Deligne $E(\pi_f)$-structure of $\mathrm{det}_{E(\pi_f)\otimes_{\QQ}\RR}\Ext^{1}_{\mathrm{MHS_{\RR}^{+}}}(\RR(0), M_{B}(\pi_f, V(2))_{\RR})$ is 
\begin{equation*}
    \mathcal{D}(\pi_f, V(2)) := (2\pi i)^{3} \delta(\pi_f, V(2))^{-1} \mathcal{B}(\pi_f, V(2)). 
\end{equation*}
Let $v_D$ be a generator of $\mathcal{D}(\pi_f, V(2))$, $\tilde{v}_{D}$ be a lifting of  $v_{D}$ via the last arrow in the exact sequence (\ref{eq: def of Ext1 in section 9})
and $\Omega = \frac{1}{2} (\omega_{\Psi} + \overline{\omega_{\Psi}}) \cdot \mathbf{1}$, where $\omega_{\Psi}$ is constructed in Subsection \ref{SS: the test vector} and $\mathbf{1}$ the multiplicative identities of $E(\pi_f) \otimes_{\QQ} \CC$. 

\par It can be seen from Remark \ref{remark: parameter of repn transform} and Proposition \ref{Prop: Hodge decomp for motives} that we have the following Hodge decompositions: 
     \begin{align*}
        & M_{B}({\pi}_f, V(2))_{\CC} \\
        \cong & M_{B}^{-r, -2-a-b-s} \oplus M_{B}^{-1-a-r, -1-b-s} \oplus M_{B}^{-2-a-b-r, -s} \\
        \oplus & M_{B}^{-2-a-b-s, -r} \oplus M_{B}^{-1-b-s, -1-a-r} \oplus M_{B}^{-s, -2-a-b-r}, 
    \end{align*} and 
    \begin{align*}
        & M_{B}(\tilde{\pi}_f|\mu|^{-2}, D)_{\CC} \\
        \cong & M_{B}^{a + b + r + 2, s} \oplus M_{B}^{a + r + 1, b + s + 1} \oplus M_{B}^{r, a + b + s + 2} \\
        \oplus & M_{B}^{s, a + b + r + 2} \oplus M_{B}^{b + s + 1, a + r + 1} \oplus M_{B}^{a + b + s + 2, r}, 
    \end{align*}
    from which we can see that 
    $M_{B}(\pi_f, V(2))^{-}_{\RR}(-1)$ is $3$-dimensional.
    By the condition of the parameters stated in (\ref{eq: Cond of coefficients}), we get that $F^{0}M_{dR}(\pi_f, V(2))_{\RR}$ is $2$-dimensional. 

    \begin{notation}
        We use $M_{dR}^{s, t}$ to denote $I_{\infty}^{-1} (M_{B}^{s, t})$ for $s, t \in \ZZ$.  
    \end{notation}

    We now recall the definition of Deligne periods as in \cite[\S 1.7]{Deligne79}. 
    \begin{definition}[Deligne period]
    \label{def: Deligne period}
    Assume that $a + r \neq b + s$\footnote{This assumption implies that condition that the infinite Frobenius $F_{\infty}$ acts on the component with Hodge type $(n, n)$ for all $n \in \ZZ$ by a scalar $1$ or $-1$(see \cite[\S 1.7]{Deligne79}) is satisfied. } and let 
    \begin{align*}
           & F^{+}(\tilde{\pi}_{f}|\mu|^{-2}, D) \\
         = & F^{-}(\tilde{\pi}_{f}|\mu|^{-2}, D) \\
         = & M_{dR}^{a + b + r + 2, s} \oplus M_{dR}^{a + b + s + 2, r} \oplus M_{dR}^{A, B},
    \end{align*}
    where $A = \max\{a + r + 1, b + s + 1\}$, $B = \min\{a + r + 1, b + s + 1\}$. 
    It is shown in \cite[\S 1.7]{Deligne79} that the map 
    \begin{equation*}
        I_{\infty}^{\pm} \colon M_{B}(\tilde{\pi}_{f}|\mu|^{-2}, D)_{\CC}^{\pm} \longrightarrow M_{dR}(\tilde{\pi}_{f}|\mu|^{-2}, D)_{\CC} / F^{\pm}(\tilde{\pi}_{f}|\mu|^{-2}, D)
    \end{equation*}
    is an isomorphism.
        \par The Deligne period $c^{\pm}(\tilde{\pi}_{f}|\mu|^{-2}, D)$ is defined as $\det(I_{\infty}^{\pm}) \in (E(\pi_f) \otimes_{\QQ} \CC)^{\times}$. 
\end{definition}

\begin{remark}
    We can define the Deligne period $c^{\pm}(\pi_f, V(2))$ similarly. This is left to the reader.
\end{remark}

 We have the following commutative diagram: 
    \begin{equation}{\label{commutative diagram_pairing}}
        \begin{tikzcd}
            M_B(\pi_f, V(2))_{\CC} \otimes M_{B}(\tilde{\pi}_{f}|\mu|^{-2}, D)_{\CC}  \ar[r, "{\langle \cdot, \cdot \rangle_{B}}"] \ar[d, "I_{\infty}"] &  E(\pi_f) \otimes_{\QQ} {\CC} \ar[d, "{I_{\infty}}"]  \\
             M_{dR}(\pi_f, V(2))_{\CC} \otimes M_{dR}(\tilde{\pi}_{f}|\mu|^{-2}, D)_{\CC} \ar[r, "{\langle \cdot, \cdot \rangle_{dR}}"] & E(\pi_f) \otimes_{\QQ} \CC 
        \end{tikzcd}, 
    \end{equation}
where the vertical arrows are isomorphisms. 

\begin{construction}
\label{construnction: basis for periods}
\par    We contruct a ``nice" basis of the spaces in the commutative diagram (\ref{commutative diagram_pairing}) as follows. 
         \begin{itemize}
             \item Let $v_1, v_2, v_3$ be an $E(\pi_f)$-basis of $M_{B}(\pi_f, V(2))^{-}_{\RR}(-1)$ such that $v_1 \in M_{B}^{-r, -2-a-b-s}$ and $v_2 \in M_{B}^{-s, -2-a-b-r}$. Then $2\pi{i}v_1, 2\pi{i}v_2, 2\pi{i}v_3$ is an $E(\pi_f)$-basis of $M_B(\pi_f, V(2))^{+}$.
             \item Let $w_1, w_2, w_3$ be an $E(\pi_f)$-basis of $M_{B}(\tilde{\pi}_{f}|\mu|^{-2}, D)^{+}_{\CC}$ such that 
                    \begin{equation*}
                        \langle 2\pi{i}v_i, w_j \rangle_{B} = \delta_{ij},
                    \end{equation*}
                    for $i,j \in \{1, 2, 3\}$.
            \item Let $\theta_1, \theta_2$ be the $E(\pi_f)$-basis of  $F^{0}M_{dR}(\pi_f, V(2))$ such that the image of $\theta_{i}$ for $i = 1,2$ under the map
            \begin{equation*}
                 F^{0}M_{dR}(\pi_f, V(2))_{\RR} \rightarrow M_{B}(\pi_f, V(2))^{-}_{\RR}(-1) 
            \end{equation*}
            is $v_i$.
            \item Let $\theta_{1}^{\prime}, \theta_{2}^{\prime} \in M_{dR}(\tilde{\pi}_{f}|\mu|^{-2}, D)_{\CC}$ be the dual vectors of $\theta_1, \theta_2$, which means that 
            \begin{align*}
                & \langle \theta_1, \theta_1^{\prime} \rangle_{dR} = \langle \theta_2, \theta_2^{\prime} \rangle_{dR} = 1_{dR}, \\
                & \langle \theta_1, \theta_2^{\prime} \rangle_{dR} = \langle \theta_2, \theta_1^{\prime} \rangle_{dR} = 0. 
            \end{align*}
            It follows that $\theta_1^{\prime} \in M_{dR}^{r, a + b + s + 2}$ and $\theta_2^{\prime} \in M_{dR}^{s, a + b + r + 2}$. If we let $\Omega^{\prime} = I_{\infty}(\Omega)$, then it follows from the fact that 
            \begin{equation*}
            \Omega \in M_{B}^{a + r + 1, b + s + 1} \oplus M_{B}^{b + s + 1, a + r + 1}.
            \end{equation*}
            that 
            \begin{equation*}
                \Omega^{\prime} \in M_{dR}^{a + r + 1, b + s + 1} \oplus M_{dR}^{b + s + 1, a + r + 1}. 
            \end{equation*}
            Hence, $\Omega^{\prime}, \theta_1^{\prime}, \theta_{2}^{\prime}$ is a basis of $M_{dR}(\tilde{\pi}_{f}|\mu|^{-2}, D)_{\CC} / F^{\pm}(\tilde{\pi}_{f}|\mu|^{-2}, D)$. 
            \item It follows from $\theta_1 \in M_{dR}^{-r, -2-a-b-s}$, $\theta_2 \in M_{dR}^{-s, -2-a-b-r}$ and 
            \begin{equation*}
            \Omega^{\prime} \in M_{dR}^{a + r + 1, b + s + 1} \oplus M_{dR}^{b + s + 1, a + r + 1},
            \end{equation*}
            that 
            \begin{equation*}
            \langle \Omega^{\prime} , \theta_1 \rangle_{dR} =  \langle \Omega^{\prime} , \theta_2 \rangle_{dR} = 0. 
            \end{equation*}
    \end{itemize}
    
\end{construction}

\begin{lemma}
\label{lemma: inverse matric for periods}
    Using the basis $w_1, w_2, w_3$ of $M_{B}(\tilde{\pi}_{f}|\mu|^{-2}, D)^{+}_{\CC}$ and basis $\Omega^{\prime}, \theta_1^{\prime}, \theta_2^{\prime}$ of 
    \begin{equation*}
        M_{dR}(\tilde{\pi}_{f}|\mu|^{-2}, D)_{\CC} / F^{\pm}(\tilde{\pi}_{f}|\mu|^{-2}, D), 
    \end{equation*}
    the inverse matrix of 
    \begin{equation*}
         I_{\infty}^{+} \colon M_{B}(\tilde{\pi}_{f}|\mu|^{-2}, D)_{\CC}^{+} \longrightarrow M_{dR}(\tilde{\pi}_{f}|\mu|^{-2}, D)_{\CC} / F^{+}(\tilde{\pi}_{f}|\mu|^{-2}, D)
    \end{equation*}
    can be written as 
    \begin{equation*}
         c^{+}(\tilde{\pi}_{f}|\mu|^{-2}, D)^{-1}\begin{pmatrix}
                                                C & * & * \\
                                                B & * & * \\
                                                (2\pi{i})^{-2} & * & * 
         \end{pmatrix}. 
    \end{equation*}
\end{lemma}

\begin{proof}
    \par Using the basis $w_1, w_2, w_3$ and $\Omega^{\prime}, \theta_1^{\prime}, \theta_2^{\prime}$, if we write the map 
    \begin{equation*}
         I_{\infty}^{+} \colon M_{B}(\tilde{\pi}_{f}|\mu|^{-2}, D)_{\CC}^{+} \longrightarrow M_{dR}(\tilde{\pi}_{f}|\mu|^{-2}, D)_{\CC} / F^{+}(\tilde{\pi}_{f}|\mu|^{-2}, D)
    \end{equation*}
    explicitly as 
    \begin{equation*}
        I_{\infty}^{+} \colon (w_1, w_2, w_3) \mapsto (\Omega^{\prime}, \theta_1^{\prime}, \theta_2^{\prime}) \begin{pmatrix}
                                                                                                            \alpha_1 & \alpha_2 & \alpha_3 \\ 
                                                                                                            \beta_1 & \beta_2 & \beta_3 \\
                                                                                                            \gamma_1 & \gamma_2 & \gamma_3 \\
                                                                                                        \end{pmatrix}, 
    \end{equation*}
    then $(I_{\infty}^{+})^{-1}$ can be written as 
    \begin{equation*}
        (I_{\infty}^{+})^{-1} \colon (\Omega^{\prime}, \theta_1^{\prime}, \theta_2^{\prime}) \mapsto (w_1, w_2, w_3) (\det(I_{\infty}^{+}))^{-1}   \begin{pmatrix}
                                                                                                                    C & * & * \\
                                                                                                                    B & * & * \\
                                                                                                                    A & * & * 
                                                                                                                \end{pmatrix}, 
    \end{equation*}
    where $A = \det \begin{pmatrix}
                  \beta_1 & \beta_2 \\
                  \gamma_1 & \gamma_2 \\
               \end{pmatrix} = \beta_1\gamma_2 - \beta_2 \gamma_1$ and $\det(I_{\infty}^{+}) = c^{+}(\tilde{\pi}_{f}|\mu|^{-2}, D)$. 
    \par Now we compute $A$. We have  
    \begin{align*}
        & I_{\infty}^{+} (w_1) = \alpha_1 \Omega^{\prime} + \beta_1 \theta_1^{\prime} + \gamma_1 \theta_2^{\prime}, \\
        & I_{\infty}^{+} (w_2) = \alpha_2 \Omega^{\prime} + \beta_2 \theta_1^{\prime} + \gamma_2 \theta_2^{\prime}. 
    \end{align*}
    By the definition of $w_i$ and commutativity of the diagram (\ref{commutative diagram_pairing}), we have the following identities: 
    \begin{align*}
        & \beta_1 = \langle \theta_1, I_{\infty}^{+}(w_1) \rangle_{dR} = \langle v_1, w_1 \rangle_{B} = (2\pi{i})^{-1}, \\
        & \gamma_1 = \langle \theta_2, I_{\infty}^{+}(w_1) \rangle_{dR} = \langle v_2, w_1 \rangle_{B} = 0, \\
        & \beta_2 = \langle \theta_1, I_{\infty}^{+}(w_2) \rangle_{dR} = \langle v_1, w_2 \rangle_{B} = 0, \\
        & \gamma_2 = \langle \theta_2, I_{\infty}^{+}(w_2) \rangle_{dR} = \langle v_2, w_2 \rangle_{B} = (2\pi{i})^{-1}. 
    \end{align*}
    So we have $\begin{pmatrix}
                    \beta_1 & \beta_2 \\
                    \gamma_1 & \gamma_2 
                \end{pmatrix} = \begin{pmatrix}
                                    (2\pi{i})^{-1} & 0 \\
                                    0 &  (2\pi{i})^{-1}
                                 \end{pmatrix}$ and $A = (2\pi{i})^{-2}$. 
\end{proof}

\begin{notation}
    Recall that we had introduced the following notation: 
    if $\mu$ and $\mu^{\prime}$ are two elements of $E(\pi_f)\otimes_{\QQ}{\CC}$, we write $\mu \sim \mu^{\prime}$ if there exists $\lambda \in E(\pi_f)^{\times}$ such that $\mu = \lambda \mu^{\prime}$. 
\end{notation}

\begin{proposition}
\label{prop: periods}
Assuming that $a + r \neq b + s$, we have the following equality:
\begin{equation*}
    \langle \Omega, \tilde{v}_D \rangle_{B} \sim  c^{-}(\pi_f, V(2))^{-1}. 
\end{equation*}
    
\end{proposition}

\begin{proof}
    \par First, it can be seen from the construction that $v_3$ is a generator of the Beilinson $E(\pi_f)$-structure $\mathcal{B}(\pi_f, V(2))$. Therefore, we have 
    \begin{equation*}
        \langle \Omega, \tilde{v}_D \rangle_{B} \sim (2\pi i)^{3} \delta(\pi_f, V(2))^{-1} \langle \Omega, v_3 \rangle_{B}. 
    \end{equation*}
    \par Second, we compute $\langle \Omega, v_3 \rangle_{B}$. It can seen from Lemma \ref{lemma: inverse matric for periods} that 
    \begin{align*}
        \langle \Omega, v_3 \rangle_{B} &= c^{+}(\tilde{\pi}_{f}|\mu|^{-2}, D)^{-1}\langle Cw_1 + Bw_2 + (2\pi i)^{-2}w_3, v_3 \rangle_{B} \\
                                    &= c^{+}(\tilde{\pi}_{f}|\mu|^{-2}, D)^{-1} (2\pi i)^{-2} \langle w_3, v_3 \rangle_{B} \\
                                    &= c^{+}(\tilde{\pi}_{f}|\mu|^{-2}, D)^{-1} (2\pi{i})^{-3}. 
    \end{align*}
    \par Finally, we can see that 
    \begin{align*}
         \langle \Omega, \tilde{v}_D \rangle_{B} & \sim (2\pi i)^{3} \delta(\pi_f, V(2))^{-1} \langle \Omega, v_3 \rangle_{B} \\
                                           & = c^{+}(\tilde{\pi}_{f}|\mu|^{-2}, D)^{-1} \delta(\pi_f, V(2))^{-1}. 
    \end{align*}
    It follows from \cite[(5.1.7)]{Deligne79} that 
    \begin{equation*}
        c^{+}(\tilde{\pi}_{f}|\mu|^{-2}, D)^{-1} \delta(\pi_f, V(2))^{-1} \sim c^{-}(\pi_f, V(2))^{-1}. 
    \end{equation*}
    So we conclude that 
    \begin{equation*}
        \langle \Omega, \tilde{v}_d \rangle_{B} \sim c^{-}(\pi_f, V(2))^{-1}. 
    \end{equation*}
\end{proof}

\begin{remark}
    The condition $a + r \neq b + s$ guarantees that the weight of $M_B(\pi_f, V(2))$ is $\le -3$. 
\end{remark}

\subsection{The Whittaker period}
\label{SS: Whittaker period}
In this subsection, we define a special automorphic period called the Whittaker period.  

\par For any cuspidal automorphic representation $\pi = \pi_f \otimes \pi_{\infty}$ of $G(\AAA)$ such that $\pi_{\infty}$ is a discrete series representation of $G(\RR)^{+}$, $\pi_{f}$ is defined over a number field $E(\pi_f) \subseteq \overline{\QQ}$ by Theorem \ref{Thm: rational field}. Hence, for the vector space $V_{\pi_f}$ under $\pi_f$, there is a model $V_{\overline{\QQ}}$ of $V_{\pi_f}$ such that 
\begin{equation*}
    V_{\pi_f} = V_{\overline{\QQ}} \otimes_{\overline{\QQ}} \CC.
\end{equation*}

Therefore, by Proposition \ref{prop: rational whittaker}, for the Whittaker functional $\Lambda$ associated to $\pi_f$ by Definition \ref{def: Whittaker function}, there exists a $\Lambda_f \colon V_{\pi_f} \rightarrow \CC$ sending $V_{\overline{\QQ}}$ to $\overline{\QQ}$, which is unique up to multiplication by an element of $\overline{\QQ}^{\times}$. 

\begin{definition}[Whittaker period]
\label{def: Whittaker period}
    By uniqueness of Whittaker models, there exists a $c \in \CC^{\times}$ such that 
    \begin{equation*}
        \Lambda = c \cdot \Lambda_f. 
    \end{equation*}
    The \textit{Whittaker period} $W(\pi) \in \CC^{\times} / \QQ^{\times}$ is defined to be the class of $c$ in $\CC^{\times} / \QQ^{\times}$. 
\end{definition}