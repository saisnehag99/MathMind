\ifdefined\isMainDocument
\else
    \documentclass[10pt,oneside,reqno]{amsart}
    \usepackage{xr}
    \externaldocument{../main} % Main の .aux を参照
    \usepackage{latexsym}
\usepackage{amscd}
\usepackage{amsmath}
\usepackage{amssymb}
\usepackage[mathscr]{euscript}
\usepackage{graphicx}
\usepackage{geometry}
\usepackage{mathtools}
\usepackage[all]{xy}
\usepackage[dvipsnames]{xcolor}
\usepackage[colorlinks,final,hyperindex]{hyperref}
\usepackage{tikz}
\usepackage{tikz-cd}
\usetikzlibrary{decorations.pathmorphing,decorations.markings,arrows,calc,shapes.geometric,arrows.meta,positioning}
\usepackage{enumerate}
\usepackage{multirow}

\usepackage[bb=dsserif]{mathalpha}
\usepackage{bm}

\newcommand{\QQ}{\ensuremath{\mathbb{Q}}}
\newcommand{\DD}{\ensuremath{\mathbb{D}}}
\newcommand{\CC}{\ensuremath{\mathbb{C}}}
\newcommand{\RR}{\ensuremath{\mathbb{R}}}
\newcommand{\ZZ}{\ensuremath{\mathbb{Z}}}
\newcommand{\NN}{\ensuremath{\mathbb{N}}}

\newcommand{\g}{\ensuremath{\mathfrak{g}}}
\renewcommand{\SS}{\ensuremath{\mathbb{S}}}

\newcommand{\id}{\ensuremath{\mathrm{id}}}

\makeatletter

\newcommand{\bbone}{\mathbb{1}}

\newcommand{\calA}{\mathcal{A}}
\newcommand{\calB}{\mathcal{B}}
\newcommand{\calC}{\mathcal{C}}
\newcommand{\calD}{\mathcal{D}}
\newcommand{\calE}{\mathcal{E}}
\newcommand{\calF}{\mathcal{F}}
\newcommand{\calG}{\mathcal{G}}
\newcommand{\calH}{\mathcal{H}}
\newcommand{\calI}{\mathcal{I}}
\newcommand{\calJ}{\mathcal{J}}
\newcommand{\calK}{\mathcal{K}}
\newcommand{\calL}{\mathcal{L}}
\newcommand{\calM}{\mathcal{M}}
\newcommand{\calN}{\mathcal{N}}
\newcommand{\calO}{\mathcal{O}}
\newcommand{\calP}{\mathcal{P}}
\newcommand{\calQ}{\mathcal{Q}}
\newcommand{\calR}{\mathcal{R}}
\newcommand{\calS}{\mathcal{S}}
\newcommand{\calT}{\mathcal{T}}
\newcommand{\calU}{\mathcal{U}}
\newcommand{\calV}{\mathcal{V}}
\newcommand{\calW}{\mathcal{W}}
\newcommand{\calX}{\mathcal{X}}
\newcommand{\calY}{\mathcal{Y}}
\newcommand{\calZ}{\mathcal{Z}}
\newcommand{\kk}{\Bbbk}

\newcommand{\scrY}{\mathscr{Y}}
\newcommand{\scrV}{\mathscr{V}}
\newcommand{\scrP}{\mathscr{P}}

\newcommand{\Rep}{\mathop{\mathrm{Rep}}\nolimits}
\newcommand{\alg}{\mathop{\mathrm{alg}}\nolimits}
\newcommand{\Span}{\mathop{\mathrm{Span}}\nolimits}
\newcommand{\proj}{\mathop{\mathrm{proj}}\nolimits}
\newcommand{\Hom}{\mathop{\mathrm{Hom}}\nolimits}
\newcommand{\PHom}{\mathop{\mathrm{PHom}}\nolimits}
\newcommand{\Tw}{\mathop{\mathrm{Tw}}\nolimits}
\newcommand{\Dim}{\mathop{\mathrm{Dim}}\nolimits}
\newcommand{\Imm}{\mathop{\mathrm{Im}}\nolimits}
\newcommand{\Aus}{\mathop{\mathrm{Aus}}\nolimits}
\newcommand{\Ann}{\mathop{\mathrm{Ann}}\nolimits}
\newcommand{\APC}{\mathop{\mathrm{APC}}\nolimits}
\newcommand{\Barr}{\mathop{\mathrm{Bar}}\nolimits}
\newcommand{\Cone}{\mathop{\mathrm{Cone}}\nolimits}
\newcommand{\modu}{\mathop{\mathrm{mod}}\nolimits}
\newcommand{\dgMod}{\mathop{\mathrm{dgMod}}\nolimits}
\newcommand{\soc}{\mathop{\mathrm{soc}}\nolimits}
\newcommand{\dg}{\mathop{\mathrm{dg}}\nolimits}
\newcommand{\red}{\mathop{\mathrm{red}}\nolimits}
\newcommand{\Map}{\mathop{\mathrm{Map}}\nolimits}
\newcommand{\inj}{\mathop{\mathrm{inj}}\nolimits}
\newcommand{\op}{\mathop{\mathrm{op}}\nolimits}
\newcommand{\Ker}{\mathop{\mathrm{Ker}}\nolimits}
\newcommand{\Tor}{\mathop{\mathrm{Tor}}\nolimits}
\newcommand{\Tot}{\mathop{\mathrm{Tot}}\nolimits}
\newcommand{\Ext}{\mathop{\mathrm{Ext}}\nolimits}
\newcommand{\sg}{\mathop{\mathrm{sg}}\nolimits}
\newcommand{\Sl}{\mathop{\mathfrak{sl}}\nolimits}
\newcommand{\nc}{\mathop{\mathrm{nc}}\nolimits}
\newcommand{\Tr}{\mathop{\mathrm{Tr}}\nolimits}
\newcommand{\Irr}{\mathop{\mathrm{Irr}}\nolimits}
\newcommand{\HC}{\mathop{\mathrm{HC}}\nolimits}
\newcommand{\HH}{\mathop{\mathrm{HH}}\nolimits}
\newcommand{\THH}{\mathop{\mathrm{TH}}\nolimits}
\newcommand{\Perf}{\mathop{\mathrm{Perf}}\nolimits}
\newcommand{\del}{\partial}
\newcommand{\ev}{\mathop{\mathrm{ev}}\nolimits}
\newcommand{\co}{\mathop{\mathrm{co}}\nolimits}
\newcommand{\pt}{\mathop{\mathrm{pt}}\nolimits}
\newcommand{\wt}{\mathop{\mathrm{wt}}\nolimits}
\newcommand{\pCY}{\mathop{\mathrm{pCY}}\nolimits}
\newcommand{\gr}{\mathop{\mathrm{gr}}\nolimits}
\newcommand{\coker}{\mathop{\mathrm{coker}}\nolimits}


\newcommand{\Top}{\mathop{\mathrm{Top}}\nolimits}
\newcommand{\Set}{\mathop{\mathrm{Set}}\nolimits}
\newcommand{\Vect}{\mathop{\mathrm{Vect}}\nolimits}
\newcommand{\Mod}{\mathop{\mathrm{Mod}}\nolimits}
\newcommand{\Ob}{\mathop{\mathrm{Ob}}\nolimits}
\newcommand{\Ind}{\mathop{\mathrm{Ind}}\nolimits}
\newcommand{\Sym}{\mathop{\mathrm{Sym}}\nolimits}
\newcommand{\Alt}{\mathop{\mathrm{Alt}}\nolimits}
\newcommand{\End}{\mathop{\mathrm{End}}\nolimits}
\newcommand{\Ch}{\mathop{\mathrm{Ch}}\nolimits}


\newcommand{\Ass}{\mathop{\mathrm{Ass}}\nolimits}
\newcommand{\Com}{\mathop{\mathrm{Com}}\nolimits}
\newcommand{\Lie}{\mathop{\mathrm{Lie}}\nolimits}
\newcommand{\colim}{\mathop{\mathrm{colim}}\nolimits}

\newcommand{\Op}{\mathop{\mathsf{Operads}}\nolimits}
\newcommand{\Diop}{\mathop{\mathsf{Dioperads}}\nolimits}
\newcommand{\PDiop}{\mathop{\mathsf{PDioperads}}\nolimits}
\newcommand{\Coop}{\mathop{\mathsf{Cooperads}}\nolimits}
\newcommand{\Codiop}{\mathop{\mathsf{Codioperads}}\nolimits}
\newcommand{\PCodiop}{\mathop{\mathsf{PCodioperads}}\nolimits}

\newcommand{\antish}{\ensuremath{\mbox{!`}}}


\tikzset{->-/.style={decoration={
			markings,
			mark=at position #1 with {\arrow{>}}},postaction={decorate}}}
\tikzset{-w-/.style={decoration={
			markings,
			mark=at position #1 with {\arrow{Stealth[fill=white,scale=1.4]}}},postaction={decorate}}}
\tikzset{->-/.default=0.65}
\tikzset{-w-/.default=0.65}
\tikzstyle{bullet}=[circle,fill=black,inner sep=0.5mm]
\tikzstyle{circ}=[circle,draw=black,fill=white,inner sep=0.5mm]
\tikzstyle{vertex}=[circle,draw=black,thick,inner sep=0.5mm]
\tikzstyle{dot}=[draw,circle,fill=black,minimum size=0.5mm,inner sep = 0mm, outer sep = 0mm]
\tikzset{darrow/.style={double distance = 4pt,>={Implies},->},
	darrowthin/.style={double equal sign distance,>={Implies},->},
	tarrow/.style={-,preaction={draw,darrow}},
	qarrow/.style={preaction={draw,darrow,shorten >=0pt},shorten >=1pt,-,double,double
		distance=0.2pt}}
\newcommand\tikcirc[1][2.5]{\tikz[baseline=-#1]{\draw[thick](0,0)circle[radius=#1mm];}}
\newcommand{\tikzfig}[1]{\begin{tikzpicture}[auto,baseline={([yshift=-.5ex]current bounding box.center)}]#1\end{tikzpicture}}
\usetikzlibrary{decorations.pathreplacing}

\usepackage{amsthm}
\usepackage{thmtools}
\usepackage[noabbrev,capitalize]{cleveref}
\newtheorem{theorem}{Theorem}
\newtheorem{lemma}[theorem]{Lemma}
\newtheorem{proposition}[theorem]{Proposition}
\newtheorem{corollary}[theorem]{Corollary}
\newtheorem{conjecture}[theorem]{Conjecture}
\newtheorem{Claim}[theorem]{Claim}

\newtheorem*{theorem*}{Theorem}

\theoremstyle{definition}
\newtheorem{definition}{Definition}
\newtheorem{example}{Example}
\newtheorem{cexample}{Counterexample}
\newtheorem{exercise}{Exercise}
\newtheorem{question}{Question}
\newtheorem{remark}{Remark}

\addtolength{\textheight}{1.75cm}
\addtolength{\voffset}{-0.5cm}
\addtolength{\footskip}{+0.6cm}
\geometry{margin=1.2in}


\usepackage[backend=biber, maxbibnames=99, bibencoding=utf8, doi=false, isbn=false, url=false, style=alphabetic]{biblatex}
\addbibresource{refs.bib}

    \begin{document}
\fi

\section{Concluding remarks}% Section名

%\setcounter{section}{6}

\nsubsection{Questions related to $\AdDMR_0$}

In this subsection, we discuss remaining questions concerning $\AdDMR_0$.
Our first question asks whether $\addmr\cap \Fad$ is contained in $V_{\strprty}$.

\begin{qu}
The affine scheme $\addmr\cap \Fad$ is a closed affine subscheme of $V_{\strprty}$. In other words, $\Psi^{11} + \Psi^{10} + \Psi^{01} = 0$ follows from the defining relation of $\addmr$ for every $\Psi \in \addmr\cap \Fad$.
\end{qu}

Indeed, imposing $V_{\strprty}$ does not change the dimensions:

\begin{center}
\begin{tabular}{c|ccccccccccccc}
$k$ & $0$ & $1$ & $2$ & $3$ & $4$ & $5$ & $6$ & $7$ & $8$ & $9$ & $10$ & $11$ & $\cdots$ \\
\hline
$\dim \addmr^{(k)} \cap \Fad$
& $0$ & $0$ & $0$ & $0$ & $1$ & $0$ & $1$ & $0$ & $1$ & $1$ & $1$ & $1$ & $\cdots$ \\
$\dim \addmr^{(k)} \cap \Fad \cap V_{\strprty}$
& $0$ & $0$ & $0$ & $0$ & $1$ & $0$ & $1$ & $0$ & $1$ & $1$ & $1$ & $1$ & $\cdots$ \\
\end{tabular}
\end{center}


Lastly, let us consider the exponential map of $\TM_1$.
Let $\tau : \TM_1 \rightarrow \Aut(\ide{h}^{\vee}) $ be a collection $(\tau^R : \TM_1(R) \rightarrow \Aut_{R\text{-\textbf{alg}}}(R\cotimes\ide{h}^{\vee}))_{R \in \Qalg}$.
One notes that a differential map $d\tau : \ide{tm}_1 \rightarrow \ide{der}(\ide{h}^{\vee}) $ of $\tau^{\Q}$ induces a natural transformation $\tm_{1,\ide{a}} \rightarrow \ide{der}(\ide{h}^{\vee})$. We denote this natural transformation by $d\tau_{\ide{a}}$.
Since $d\tau_{\ide{a}}(R)$ is a pronilpotent derivation map on $R\cotimes \ide{h}^{\vee}$, then, there exist naturally isomorphisms of affine schemes  
\[
\exp^{\circledast_1} : \tm_{1,\ide{a}} \rightarrow \TM_1 \quad \text{and} \quad \exp : \ide{der}(\ide{h}^{\vee}) \rightarrow \Aut(\ide{h}^{\vee})
\]
that make the following diagram commute:
\[
\xymatrix{
\TM_1 \ar[r]^-{\tau} & \Aut(\ide{h}^{\vee}) \\
\tm_{1,\ide{a}} \ar[u]^{\exp^{\circledast_1}}\ar[r]_-{d\tau_{\ide{a}}} &\ide{der}(\ide{h}^{\vee})_{\ide{a}} \ar[u]_{\exp}
}
\]
That is, for every \(\Q\)-algebra \(R\) and any \(\psi \in \tm_{1,\ide{a}}(R)\), the following equality holds:
\[
\kappa_{\exp^{\circledast_1,R}(\psi)} = \tau^R \circ \exp^{\circledast_1,R}(\psi) = \exp^R \circ d\tau_{\ide{a}}^R(\psi) = \exp^R(d_{\psi})
\]
Substituting $x_1$ into the above, we obtain the explicit formula for the exponential map:
\[
\exp^{\circledast_1,R}(\psi) = \exp^R(d_{\psi})(x_1).
\]


We then expect the following to hold:

\begin{qu}\label{conj:AdDMR-exp}
Is the following exponential map
\[
\exp^{\circledast_1} : \addmr_{\ide{a}} \times_{\tm_{1,\ide{a}}} \ide{F}_{2,\ide{a}}^{\ad(x_1)} \rightarrow \AdDMR_0\times_{\TM_1} \FAd
\]
isomorphic? 
\end{qu}
One may note that the tangent space of $\AdDMR_0 \times_{\TM_1} \FAd$ at $x_1$ is $\addmr \cap \Fad$ and the following $(\addmr \cap \Fad)_{\ide{a}} \cong \addmr_{\ide{a}} \times_{\tm_{1,\ide{a}}} \ide{F}_{2,\ide{a}}^{\ad(x_1)}$ holds.


\nsubsection{Toward formal Kaneko-Zagier conjecture}

The formal Kaneko-Zagier conjecture, stated by Kaneko and Zagier \cite{Kan1} and recently rearranged by Bachmann and Risan \cite{Risan}, is one of the lifts of the Kaneko-Zagier conjecture.
To begin with, we recall a study on the FMZVs and SMZVs.
\begin{df}
\begin{enumerate}
\item For $(k_1,\ldots,k_r)$, we define finite multiple zeta values as elements of $\mathscr{A}$ (see Section \ref{section:Introduction}) by
\[
\zeta_{\A}(k_1,\ldots,k_r) := \left( \sum_{0<m_1<\cdots < m_r<p} \frac{1}{m_1^{k_1}\cdots m_r^{k_r}} \right)_p \in \A.
\]
We set $\zeta_{\A}(\emptyset) = 1$.
\item For $(k_1,\ldots,k_r)$, we define SMZVs as elements of $\mathcal{Z}/\zeta(2)\mathcal{Z}$ by
\[
\zeta_{\S}(k_1,\ldots,k_r) := \zeta_{\mathrm{Ad}}(k_1,\ldots,k_r;0).
\]
We set $\zeta_{\S}(\emptyset) = 1$.
\end{enumerate}
\end{df}
Let $\mathcal{Z}_{\A}$ (respectively, $\mathcal{Z}_{\S}$) be the $\Q$-linear space generated by finite multiple zeta values (respectively, SMZVs).
By Yasuda's theorem (\cite[Theorem 6.1]{Yasuda}), it follows that $\mathcal{Z}_{\S} = \mathcal{Z}/ \zeta(2)\mathcal{Z}$.
Put $\bullet \in \{\A,\S\}$. We define two $\Q$-linear maps 
\begin{align*}
Z_{\bullet}^{\shuffle} : &\Q + x_1 \ide{h} \rightarrow \mathcal{Z}_{\bullet} ; x_1x_0^{k_1-1}\cdots x_1x_0^{k_r-1} \mapsto \zeta_{\bullet}(k_1,\ldots,k_r)\\
Z_{\bullet}^{*} : & \Q\langle Y \rangle \rightarrow \mathcal{Z}_{\bullet} ; y_{k_1}\cdots y_{k_r} \mapsto \zeta_{\bullet}(k_1,\ldots,k_r)
\end{align*}
for $k_1,\ldots,k_r \in \mathbb{Z}_{>0}$.

Both analogs of multiple zeta values satisfy the following $\Q$-linear relations

\begin{thm}\label{thm:ds-AS}
\begin{enumerate}
\item Let $x_1u$, $x_1v \in  x_1\ide{h}$. Then, it follows that
\[
Z_{\bullet}^{\shuffle}(x_1u \shuffle x_1v) = (-1)^{\wt u + 1} Z_{\bullet}^{\shuffle}(x_1( \overset{\leftarrow}{v} x_1u )).
\]
\item Let $u$, $v \in \Q\langle Y \rangle$. Then, it follows that
\[
Z_{\bullet}^* (u * v) = Z_{\bullet}^*(u) Z_{\bullet}^*(v).
\]
\end{enumerate}
\end{thm}
From the point of view of Theorem \ref{thm:ds-AS}, the conjecture, stated by Kaneko and Zagier\cite{Kan1}, can be seen as a lift of the Kaneko-Zagier conjecture.

\begin{df}\label{df:ffmzs}
We define a formal finite multiple zeta space $\mathcal{Z}_{\A}^f$ as the $\Q$-algebra generated by formal symbols $\zeta_{\A}^f (u)$ ($u\in X^*$) satisifying the following:
\begin{enumerate}
\item\label{df:ffmzs-1} $\zeta_{\A}^f(\emptyset) = 1$,
\item\label{df:ffmzs-2} $\zeta_{\A}^f(x_0) = \zeta_{\A}^f(x_1) = 0$,
\item\label{df:ffmzs-3} $\zeta_{\A}(x_1 u \shuffle x_1 v) =  (-1)^{\wt u + 1} \zeta_{\A}^f(x_1( \overset{\leftarrow}{v} x_1u ))$ for  $x_1u$, $x_1v \in  x_1\ide{h}$, and
\item\label{df:ffmzs-4} $\zeta_{\A}^f (\mathbf{p}(u * v)) = \zeta_{\A}^f(\mathbf{p}(u)) \zeta_{\A}^f(\mathbf{p}(v))$ for $u$, $v \in \Q\langle Y \rangle$.
\end{enumerate}
\end{df}
Then by Definition \ref{df:ffmzs}. (\ref{df:ffmzs-1}), and (\ref{df:ffmzs-4}), $\mathcal{Z}_{\A}^f$ forms an unital, commutative, and associative $\Q$-algebra.

\begin{conj}[{\cite{Kan1}}]\label{conj:formal-KZ}
We denote $\zeta_{\A}^f (x_1x_0^{k_1-1}\cdots x_1x_0^{k_r-1})$ by $\zeta_{\A}^f (k_1,\ldots,k_r)$.
Then,  the following map is the isomorphism as $\Q$-algebra:
\[
\mathcal{Z}_{\A}^f \rightarrow \mathcal{Z}^f/\zeta^f(2)\mathcal{Z}^f ; \zeta_{\A}^f(k_1,\ldots,k_r) \mapsto \zeta_{\S}^f(k_1,\ldots,k_r).
\]
Here, $\zeta_{\S}^f(k_1,\ldots,k_r) : = \sum_{j = 0}^r (-1)^{k_{j+1}+ \cdots +k_r} \zeta^f(k_1,\ldots,k_j)\zeta^f(k_r,\ldots,k_{j+1})$.
\end{conj}



We define the affine scheme
\[
  \DMR_0^{\A}:=\Hom_{\Qalg}(\mathcal{Z}_{\A}^f,-):\Qalg\rightarrow\Set ,
\]
the functor represented by the $\Q$-algebra $\mathcal{Z}_{\A}^f$.
Our target is the conjectural identification
\[
 \DMR_0^{\A} \;\cong\; \AdDMR_0 \times_{\TM_1} \FAd .
\]
This identification is closely connected with Rosen's lifting conjecture \cite[Conjecture A]{Rosen-asymptotic}.
Namely, it asks whether every adjoint multiple zeta value in the regularized range $l>0$ can be expressed as a $\Q$-linear combination of those with $l = 0$ (i.e. SMZVs).
Indeed, AdMZVs admit iterated integral expressions \cite{Hirose}. When an integral diverges, its value is defined by a canonical regularization. For $l=0$ (SMZVs) no regularization is needed, while for $l>0$ regularization is required. The lifting problem can therefore be restated as follows: are AdMZVs with $l>0$ expressible as $\Q$-linear combinations of those with $l=0$?
One may note that Yasuda proved that SMZVs (the case $l=0$) span $\mathcal{Z}$ \cite{Yasuda}.
Thus, AdMZVs with $l>0$ can be written as $\Q$-linear combinations of SMZVs by Yasuda's theorem. However, Rosen's lifting requires an explanation within the adjoint/iterated integral framework, and this remains an open question.




\ifdefined\isMainDocument
\else
    %\bibliographystyle{amsplain}
    %\bibliography{reference-adjoint}
    \end{document}
\fi