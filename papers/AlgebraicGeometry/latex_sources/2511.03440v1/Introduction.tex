A central result in convex programming is the algorithmic equivalence of optimization and separation,
emerging from a body of work around the ellipsoid method~\cite{YudinNemirovski1976, Khachiyan:LP, Schrijver:Ellipsoid, Karp_1980, padberg:inria-00076483, GrötschelLovaszSchrijver:ellipsoid}.
% 
However, this equivalence crucially requires \emph{effective solution bounds}, even when we allow approximations:
% 
if optimal solutions exist, then (approximately) optimal solutions must exist such that the logarithm of their norm is polynomially bounded in the input size.
% 
The ellipsoid method requires such solution bounds when determining the radius of its initial ellipsoid and the running time depends polynomially on the logarithm of this radius.\footnote{This radius is sometimes called ``big $R$''. 
  The ellipsoid method has another parameter, often referred to as ``little~$r$'', relating to the existence of a small ball inside the feasible region.
  However, the need for ``little $r$'' can usually be avoided by accepting approximate solutions that may violate optimality and the feasibility constraints by a tiny amount.
}
Effective solution bounds are necessary in order to be able to output (approximately) optimal solutions in (worst-case) polynomial time.
Without such bounds, the corresponding decision problem  is typically not known to be in \textbf{NP}, as its natural witnesses may have size superpolynomial in the input size.

Linear programs are prominent examples of optimization problems that enjoy effective solution bounds.
%
Their optimal solutions arise as solutions to systems of (consistent and independent) linear equations~\cite{Dantzig1963}.
%
By Cramer's rule, such solutions have polynomial bit length~\cite{Edmonds1967}, and thus the logarithm of their norms is polynomially bounded.
%
This effective solution bound plays a crucial role for the seminal result that linear programming has polynomial time algorithms \cite{Khachiyan:LP}.
%
For quadratic objective functions, optimal solutions continue to have a linear characterization (because the gradient of a quadratic is an affine linear map).
Correspondingly, convex quadratic programs, where one minimizes a convex polynomial of degree two over a polyhedron, have effective solution bounds and polynomial-time algorithms like linear programs~\cite{Kozlov1980}.
%
On the other hand, for other natural generalizations of linear programs, like semidefinite programs, the situation is drastically different and no effective solution bound is possible:
%
There are feasible semidefinite programs such that all feasible and even approximately feasible solutions have huge coefficients and the logarithm of their norms is exponential in the input size.\footnote{
  These examples are commonly attributed to Khachiyan;
  see~\cite{Ramana1997, ODonnell2017, Pataki2024}.
}
%
Therefore, the ellipsoid method does not yield a polynomial-time algorithm for semidefinite programming (even approximately), even though the corresponding separation problem has polynomial-time algorithms.
%
The lack of effective solution bounds is also the reason why semidefinite programming is not known to be in~\textbf{NP}~\cite{Porkolab1997, Ramana1997}.

The unconstrained minimization of convex polynomials of degree four and higher is a fundamental optimization problem whose optimal solutions have no linear characterization
(because critical points are defined by equations of degree three and higher).
At the same time, previous ``bad'' examples (like for semidefinite programs) do not apply.
In this paper, we develop new techniques to show effective solution bounds for this problem.
These techniques extend to \emph{convex polynomial programming},
where one minimizes a convex polynomial (of arbitrary degree) over a polyhedron. 
Our solution bounds, combined with the ellipsoid method,
yield the first efficient algorithm for solving convex polynomial programs at degree four and above (to an arbitrary degree of accuracy).
Previously, no efficient algorithm for computing the approximate minimum of a convex polynomial of degree four was known (even without constraints).

\subsection{Minimizers of (convex) polynomials} 

If $p$ is a polynomial of degree $2$, its critical points are solutions to the \emph{linear} system $\nabla p (x) = 0$. 
This permits us to control the norm of its minimizers, similarly to the case of linear programs. For polynomials of larger degree, this approach does not work. 
In fact, already for degree $4$, \emph{nonconvex}~polynomials may not admit an effectively bounded minimizer at all. 
Consider, e.g., 
\begin{equation} \label{EQ:exmp_doublyexp}
    p(\x) \coloneqq (x_{1}^2 - x_{2})^2 +  \ldots + (x_{n-1}^2 - x_n)^2 + (x_n - 2)^2.
\end{equation}
The minimum of $p$ over $\R^n$ is $0$, and it is attained if and only if $x_n = 2$, and $x_{i-1} = x_{i}^2$ for all $1 \leq i < n$. Thus, the unique minimizer $\x^*$ of $p$ satisfies $x^*_1 = 2^{2^{n-1}}$, meaning its norm is \emph{doubly}-exponential in~$n$.

The upshot is that, in order to establish effective solution bounds for \emph{convex} polynomials, it is imperative to exploit their convexity. 
On the other hand, it is \textbf{NP}-hard to decide whether a given polynomial is convex~\cite{Ahmadi2011}. This means we cannot rely on an algorithmic characterization of convexity for polynomials.  
Furthermore, convex polynomials do not (obviously) satisfy the type of quantitative properties typically used in the analysis of optimization algorithms. For example, even strictly convex polynomials are not \emph{strongly} convex in general (consider ${f(x) = x^4}$).

\paragraph{The Hessian determinant.}
Instead, we will work with a purely \emph{qualitative} characterization: $f$~is convex if and only if its matrix $\nabla^2 f(x)$ of second derivatives is positive semidefinite for all $x \in \R^n$. 
To translate from this qualitative description to the quantitative bounds we are after, the key is to consider the \emph{Hessian determinant} $h_f(x) \coloneqq \det 
\left(\nabla^2 f(x)\right)$ of $f$, which describes its (local) curvature.
Note that, by convexity, we have $h_f(x) \geq 0$ for all $x \in \R^n$. The Hessian determinant captures the local behavior of $f$ around a point $a \in \R^n$ via the following dichotomy:
\begin{itemize}[noitemsep]
    \item If $h_f(a) > 0$, then $f$ is \emph{locally $\mu$-strongly convex} around $a$, meaning there are $\mu, r > 0$ such that
    \[f(x) \geq f(a) + \langle \nabla f(a), x-a \rangle + \mu \cdot \|x-a\|^2 \quad \forall x \in B(a, r).
    \]
    \item If $h_f(a) = 0$, then $f$ has a \emph{local direction of linearity} at $a$, meaning there is a $v \in \R^n$ such that
    \[
        \frac{\partial^2}{\partial t^2} f(a + tv) = 0. 
    \]
\end{itemize}
\paragraph{Local to global.}
The core of our analysis is to show that both sides of the dichotomy above exhibit \emph{local-to-global} behavior.
First, we show that if the Hessian determinant is positive at even a single point $a \in \R^n$, this immediately implies a \emph{global} $\mu$-strongly convex \emph{lower bound} on~$f$. Moreover, the parameter $\mu$ can be controlled in terms of the binary encoding length of $f$ independently of $a$.
Second, we show that if~$f$ has a \emph{local} direction of linearity \emph{everywhere}, then it has a \emph{global} direction of linearity. That is, if $h_f(x) = 0$ for all $x \in \R^n$, there is $v \in \R^n$ (independent of $x$) such that
\[
    \frac{\partial^2}{\partial t^2} f(x + tv) = 0 \quad (\forall x \in \R^n),
\]
which implies that $f$ is linear along $v$.
Projecting onto the complement of this direction allows us to reduce to a problem in fewer variables. This fact has an interesting connection to work of O. Hesse himself~\cite{OHesse}, who (mistakenly) claimed it holds for general (nonconvex) polynomials.
Together, these observations will allow us to write any convex polynomial as the sum of a linear function and a polynomial (of fewer variables) that admits a strongly convex lower bound~(\Cref{THM:quantstructure}). 

\subsection{Main contributions}
As our main contribution, we show a singly-exponential upper bound on the norm of a global minimizer of a convex polynomial $f$ on a convex polyhedron $P = \{Ax \leq b\}$.
To state our precise results, we must first specify our input model. We write $\bc(q)$ for the bit length of $q \in \Q$. The bit length of a vector or matrix is the sum of the bit lengths of its entries.
We write $\enc{P} \coloneqq \bc{(A)} + \bc{(b)}$ for the (total) encoding length of $P$.
We encode a polynomial $f(x) = \sum_{\alpha} f_\alpha x^\alpha$ using a \emph{binary} representation for its (nonzero) coefficients $f_\alpha \in \Q$, and a \emph{unary} representation for the multi-indices~${\alpha \in \N^n}$. Thus, its encoding length is $\enc{f} = \Theta(n + \mathrm{deg}(f) + \sum_{\alpha: f_\alpha \neq 0} \bc(f_\alpha))$.\footnote{This encoding is more efficient than the ``naive'' encoding of $f$ as a (dense) vector of coefficients in the monomial basis (whose size scales exponentially in $\deg(f)$). Our main results \Cref{THM:mainpolyhedron} and \Cref{COR:mainpolyhedron} are novel even in this dense model; in fact already for \emph{fixed} degree $\deg(f) = 4$.}

\begin{restatable}{theorem}{THMmainpolyhedron}
    \label{THM:mainpolyhedron}
    Let $f \in \Q[x]$ be a convex polynomial.
    Let $P$ be a (nonempty) convex polyhedron.
    Then, either $f$ is unbounded from below on $P$, or it attains its minimum on $P$ at a point $x^*$ of norm 
    \[
    \log \|x^*\| \leq \poly(\enc{f}, \,\enc{P}).
    \]
\end{restatable} \noindent
As a consequence, we can minimize $f$ over $P$ in polynomial time using the ellipsoid method.
\begin{restatable}{corollary}{CORmainpolyhedron}
    \label{COR:mainpolyhedron}
    Let $f \in \Q[x]$ be a convex polynomial. Let $P$ be a (nonempty) convex polyhedron. Let~${\epsilon >0}$. We can decide in time $\poly(\enc{f},\, \enc{P},\, \log(1/\epsilon))$ whether $f$ is unbounded from below on $P$, and, if not, output a point~$\tilde x \in P$ with $f(\tilde x) \leq \min_{x \in P} f(x) + \epsilon$.
\end{restatable}
The main technical tool in our proof of~\Cref{THM:mainpolyhedron} is a structure theorem for convex polynomials, which we believe to be of independent interest. It shows that any convex polynomial can be written as the sum of a linear function and a polynomial (of fewer variables) that has a $\mu$-strongly convex \emph{lower bound}, where $\mu$ is at least inversely exponential in $\enc{f}$. 

\begin{restatable}{theorem}{quantstructurethm}
\label{THM:quantstructure}
Let $f \in \Q[x]$ be a convex polynomial. Then, there is a subspace $\mU \subseteq \R^n$, and a rational vector $w \in \mU^\perp$ (possibly $w = 0$), such that, writing $x_\mU$ for the projection of $x$ onto $\mU$,
\[
    f(x) = f(x_\mU) - \langle w, x \rangle,
\]
where $f(x_\mU)$ has a $\mu$-strongly convex lower bound on $\mU$. That is, there is a matrix $U \in \Q^{k \times n}$ whose rows are orthogonal and span $\mU$, and a $k$-variate, $\mu$-strongly convex, quadratic polynomial $q \in \Q[y_1, \ldots, y_k]$ such that $f(x_\mU) \geq q(Ux)$ for all $x \in \R^n$.
Moreover, the matrix $U$ and the vector $w$ can be computed in polynomial time in~$\enc{f}$, the lower bound~$q$ satisfies~$\enc{q} \leq \poly(\enc{f})$, and $\mu \geq 2^{-\poly(\enc{f})}$.
\end{restatable} \noindent
\Cref{THM:quantstructure} allows us to give a quantitative description of the minimizers of $f$ on $P$. For example, in the unconstrained case (where $P = \R^n)$, we have $f_{\min} = -\infty$ if and only if $w \neq 0$. If $w = 0$, the $\mu$-strongly convex lower bound on $f$ straightforwardly shows that its global minimizer is attained at a point of norm at most exponential in $\poly(\enc{f})$. For general $P$, the situation is more complicated. There, we show that $f$ is unbounded from below if and only if a certain linear system of inequalities (involving $U$ and $w$) has a solution. If not, Farkas' lemma gives us a witness for this fact, which we use to show that any minimizer of $f$ on $P$ has small norm when projected onto the subspace~${\mU \oplus \spann{w}}$.
Finally, by a lifting argument, we show that there exists a minimizer of small norm in the full space.

\subsection{Complexity landscape of polynomial programming}
A \emph{polynomial program} asks to minimize a rational polynomial $f$ over a convex polyhedron $P$ described by rational linear inequalities $Ax \leq b$; that is, to output a binary representation of
\begin{equation} \tag{POP} \label{EQ:PolynomialProgram}
    x^* = \mathrm{argmin}_{x \in \R^n} \left\{ f(x): x \in P \right\}.
\end{equation}
In this section, we discuss the complexity of this problem (in the Turing model) in the convex and general setting, and for different values of $d = \deg(f)$. Our discussion is summarized in~\Cref{TABLE:complexity}. For ease of exposition, we assume that $x^*$ is well-defined, i.e., that the minimum above is attained and unique.
We are interested in algorithms whose runtime is bounded polynomially in $n$ and the number of bits required to specify $f$ and $P$. 
Consider the associated \emph{decision problem}:
\begin{equation} \label{EQ:decision}\tag{D}
    \text{Given } (f, P), \text{ output } 
    \begin{cases}
        \textsc{yes} \quad & \text{ if } \quad \exists x \in P : f(x) \leq 0, \\
        \textsc{no} & \text{ if } \quad \forall x \in P : f(x) > 0.
    \end{cases}
\end{equation}
\paragraph{Compact witnesses.} A \emph{witness} for~\eqref{EQ:decision} is a point $w \in P$ satisfying $f(w) \leq 0$. Clearly, the answer to~\eqref{EQ:decision} is \textsc{yes} if and only if there exists a witness. Furthermore, if there exists a witness $w \in \Q^n$ of polynomial bit length (in the size of the input), then this \textsc{yes}-answer can be verified efficiently by evaluating $f(w)$ and the inequalities $Aw \leq b$. We call such a witness \emph{compact}.
Thus, the existence of a compact witness for every \textsc{yes}-instance of~\eqref{EQ:decision} implies that the problem lies in \textbf{NP}. We have already seen that this is the case for linear programs ($d=1$) and for quadratic programs~($d=2$)~\cite{Vavasis1990}.
In both cases, the ``canonical'' witness $w = x^*$ happens to be compact.\footnote{For linear programs, $x^*$ is a vertex of $P$, and thus the solution to a subsystem of $Ax = b$ \cite{Dantzig1963, Edmonds1967}. 
For unconstrained quadratic programs ($P = \R^n$), $x^*$ is a critical point of $f$, and thus a solution to the linear system $\nabla f(x) = 0$.
In the general setting, it turns out $x^*$ is still the solution to a linear system, involving KKT(-like) conditions~\cite{Kozlov1980, Vavasis1990}.} 
As a direct result, linear programming and \emph{convex} quadratic programming lie in \textbf{P}, as $x^*$ can be identified efficiently by the ellipsoid method~\cite{Khachiyan:LP, Kozlov1980}. On the other hand, (nonconvex) quadratic programming is~\textbf{NP}-complete (while $x^*$ is compact, it might be hard to find~\cite{Sahni1974}).

For $d \geq 4$, the minimizer $x^*$ is typically not compact. In fact, there might not exist any compact witness for~\eqref{EQ:decision}. For example, the global minimum of $f(x) = (x^2-2)^2$ over $P = \R$ is $0$, but the only witnesses of this fact are $\pm \sqrt{2} \not \in \Q$. 
Even if we assume $f$ is convex, there are examples of degree~$4$ where $x^* \not \in \Q^n$, and examples of degree $6$ where no rational witnesses exist  at all (see~\Cref{APP:Complexity}). 

\paragraph{Effective solution bounds.}
If the canonical witness $x^*$ is not rational (compact), this means that the polynomial program~\eqref{EQ:PolynomialProgram} \emph{cannot} be solved (efficiently), as it is not possibly to output a binary representation of~$x^*$ (in polynomial time). 
This appears a somewhat artificial limitation. 
Instead, one could ask to output an \emph{approximate} minimizer $\tilde x \in P$, satisfying $f(\tilde x) \leq f(x^*) + \epsilon$, where $\epsilon > 0 $ is part of the input. This corresponds to the following \emph{promise problem}:
\begin{equation} \label{EQ:promise}\tag{P}
    \text{Given } (f, P, \epsilon), \text{ output } 
    \begin{cases}
        \textsc{yes} \quad & \text{ if } \quad \exists x \in P : f(x) \leq 0, \\
        \textsc{no} & \text{ if } \quad \forall x \in P : f(x) > \epsilon.
    \end{cases}
\end{equation}
A witness for~\eqref{EQ:promise} is a point $\tilde w \in P$ with $f(\tilde w) \leq \epsilon$. Existence of such a $\tilde w$ is equivalent to \textsc{yes} being a valid answer to~\eqref{EQ:promise}. As before, the existence of a compact witness for each \textsc{yes}-instance implies that~\eqref{EQ:promise} is in \textbf{NP}.
The upshot is that any witness $w$ for~\eqref{EQ:decision} whose \emph{norm} is exponentially bounded in (a polynomial of) the size of the input can be rounded to a \emph{compact} witness $\tilde w$ for~\eqref{EQ:promise}. 
Thus, to show that \eqref{EQ:promise} is in \textbf{NP}, it suffices to show that $x^*$ has exponentially bounded norm. 

Notably, polynomial programs generally do not admit an exponentially bounded solution, already when $d=4$. Indeed, the quartic polynomial defined in~\eqref{EQ:exmp_doublyexp} has a unique global minimizer of norm $2^{2^{n-1}}$ on $P = \R^n$. Our main contribution (\Cref{THM:mainpolyhedron}) is to show that \emph{convex} polynomials (of any degree) \emph{do} admit an exponentially bounded minimizer. Contrary to the degree-$2$ case, the existence of compact (approximate) witnesses for $d\geq 4$ thus depends on convexity. Using the ellipsoid method, this shows that (approximate) convex polynomial programming is in \textbf{P}~(\Cref{COR:mainpolyhedron}). 

\begin{table}[ht]
\centering
\begin{tabular}{lll|ll}
\toprule
\makebox[3cm][l]{\textbf{objective}} & \makebox[3cm][l]{\textbf{compact wit.}~\eqref{EQ:decision}} & \makebox[3cm][l]{\textbf{compact wit.}~\eqref{EQ:promise}} & \makebox[2.5cm][l]{\textbf{complexity}~\eqref{EQ:decision}} & \textbf{complexity}~\eqref{EQ:promise}\\
\midrule
linear& yes$^{a}$~{\scriptsize\cite{Dantzig1963,Edmonds1967}}& --- & \textbf{P} {\scriptsize\cite{Khachiyan:LP}} & --- \\
convex quadratic & yes$^{a}$ {\scriptsize\cite{Kozlov1980}}& --- & \textbf{P}~{\scriptsize\cite{Kozlov1980}} & --- \\
quadratic & yes$^{a}$ {\scriptsize\cite{Vavasis1990}}& --- & --- & \textbf{NP}-hard {\scriptsize\cite{Sahni1974}} \\
convex quartic & unknown$^{b}$ & yes~ {\scriptsize (Thm. \ref{THM:mainpolyhedron})} & unknown & \textbf{P}~{\scriptsize (Cor. \ref{COR:mainpolyhedron})}  \\
 quartic & --- & \makebox[0.65cm][l]{no}~{\scriptsize (Eq. \eqref{EQ:exmp_doublyexp})} & --- & \textbf{NP}-hard$^{c}$ \\
convex $d \geq 6$ & no~{\scriptsize (App. \ref{APP:Complexity})} & yes~ {\scriptsize (Thm. \ref{THM:mainpolyhedron})} & unknown & \textbf{P}~{\scriptsize (Cor. \ref{COR:mainpolyhedron})} \\
\bottomrule
\end{tabular}
\caption{Complexity and availability of compact witnesses for problems~\eqref{EQ:decision} and~\eqref{EQ:promise}, corresponding to exact and approximate polynomial programming, respectively. Omitted entries (``---'') follow directly from adjacent ones (e.g., a compact witness for~\eqref{EQ:decision} is also a compact witness for~\eqref{EQ:promise}).
{ \footnotesize
\\ $^a$the canonical witness $w = x^*$ is compact. 
\\ $^b$the canonical witness $w = x^*$ is not compact in general, but a compact witness might exist. See~\Cref{APP:Complexity}.
\\ $^c$the problem is \textbf{NP}-hard even in the unconstrained setting ($P = \R^n$)~\cite{Laurent2008}.
}}
\label{TABLE:complexity}
\end{table}

\paragraph{Local polynomial optimization.}
Even \emph{local} optimization tasks involving polynomials are typically hard:
Determining whether a polynomial of degree $4$ has a local minimum or a critical point is hard; it is even hard to decide if a given (critical) point \emph{is} a local minimum~\cite{Ahmadi2022}. Determining an (approximate) local minimizer of a quadratic on a polytope is also hard~\cite{Ahmadi2022a}.

\paragraph{Real models of computation.}
In real models of computation (e.g., in the \textit{Blum-Schub-Smale} model~\cite{BSS1989} or in the \emph{real RAM} model), the decision problem~\eqref{EQ:decision} is always in \textbf{NP} (as evaluating $f(x)$ and $Ax \leq b$ at any point $x \in \R^n$ is easy there). However, in these models, it is not even known whether \emph{linear} programs can be solved in polynomial time~\cite{Smale91}
(as the runtime of an algorithm is no longer allowed to depend on the size of the entries of $A$ and $b$). 

\subsection{Related work}
\label{SEC:relatedwork}
\paragraph{Higher-order Newton methods.}
The global minimization of convex polynomials comes up naturally in Newton's method for optimization. This method attempts to find the minimizer of a smooth function $g : \R^n \to \R$ by successively approximating it by its second-order Taylor expansion around an iterate $x^k \in \R^n$, setting $x^{k+1}$ to be a global minimizer of this approximation. 
If the Hessian matrix $\nabla^2 g (x^k) \succeq 0$ is positive semidefinite at $x^k$, then the Taylor expansion around $x^k$ is a convex quadratic, and its critical points are global minimizers.
Newton's method converges quadratically within a small basin around local minimizers where the Hessian of $g$ is positive definite and locally Lipschitz.
It has been extensively studied whether faster convergence can be achieved by considering Taylor expansions of~$g$ of higher order, see e.g.,~\cite{Nesterov2019,Ahmadi2024}. A difficulty is that the degree-$d$ expansion of $g$ around~$x^k$ is typically nonconvex when $d > 2$ (even if $\nabla^2 g(x^k) \succeq 0$), and it is not clear how to compute its global minimizer (if it even exists). 
To avoid this problem, the authors of~\cite{Nesterov2019} and \cite{Ahmadi2024} add a regularizing term that ensures convexity. 
Still, efficient optimization of the resulting convex polynomial then relies on additional properties inherited from the specifics of the regularization scheme: in~\cite{Nesterov2019}, it satisfies a \emph{relative smoothness condition}; in~\cite{Ahmadi2024} it is \emph{sum-of-squares-convex} (see below). 
This raises the question whether such additional properties are necessary.\footnote{In fact, this question is raised already by the author in~\cite{Nesterov2019}, who states that they were unable to locate any method in the literature aimed at solving the general problem of minimizing a convex, multivariate polynomial.} 
Our work shows that convexity alone is enough.


\paragraph{Complexity of SDP.}
Semidefinite programming (SDP) is a natural extension of linear programming with important applications in (combinatorial) optimization. 
This includes the Goemans-Williamson approximation for max-cut~\cite{GW1995}, and an efficient algorithm to compute the independence number of a perfect graph~\cite{Schrijver:Ellipsoid}.
Under additional assumptions, SDPs can be solved in polynomial time using the ellipsoid method~\cite{Schrijver:Ellipsoid} or interior-point methods~\cite{Klerk2016}.
However, the computational complexity of general SDP is not known. The reason is that SDPs do not admit effective solution bounds~\cite{Porkolab1997, Pataki2024}; consider the  following example due to Khachiyan:
\begin{equation} \label{EQ:expSDP}
   \exists x \in \R^n :
\begin{bmatrix}
x_{1} & x_{2} \\
x_{2} & 1
\end{bmatrix} 
\succeq 0, \, \ldots \, , 
\begin{bmatrix}
x_{n-1} & x_{n} \\
x_{n} & 1
\end{bmatrix} \succeq 0,
\,
x_n \geq 2?
\end{equation}   
Any solution to this problem must satisfy~$x_1 \geq 2^{2^{n-1}}$, meaning it has doubly-exponential norm. 
The best available results are that, in the Turing model, SDP is either in $\mathbf{NP} \cap \mathbf{coNP}$ or not in $\mathbf{NP} \cup \mathbf{coNP}$~\cite{Ramana1997};  in the real (Blum-Shub-Smale) model, it is in $\mathbf{\textbf{NP}} \cap \mathbf{coNP}$~\cite{Ramana1997}.

\paragraph{Effective solution bounds for sum-of-squares relaxations.} 
There is a well-developed literature on convex relaxations that produce tractable bounds on the (global) minimum $p_{\min}$ of a polynomial~$p$. 
These are based on the observation that 
$
    p_{\min} = \max_{\lambda \in \R} \{ \lambda : p(x) - \lambda \geq 0 \, \forall x \in \R^n\},
$
which transforms a (nonconvex) polynomial optimization problem into a (convex) problem over the cone $\mathcal{P}$ of nonnegative polynomials. Testing membership in $\mathcal{P}$ is \textbf{NP}-hard, but one may obtain a tractable \emph{lower bound} on $p_{\min}$ by optimizing over a subcone of $\mathcal{P}$ that admits an efficient separation oracle. 
The most famous example is the cone $\Sigma \subseteq \mathcal{P}$ of \emph{sums of squares (sos)}, i.e., polynomials of the form $\sigma(x) =  s_1(x)^2 + \ldots + s_\ell(x)^2$, $s_i \in \R[x]$. It is possible to optimize over $\Sigma$ via a semidefinite program. 
This approach can be generalized to constrained problems, leading to a sequence of lower bounds called the \emph{sum-of-squares hierachy}~\cite{Parrilo2000, Lasserre2001}. Algorithms based on the sos hierarchy have had a significant impact in theoretical computer science~\cite{BarakSteurer2014} . 
However, its precise computational complexity is not known: Pathological examples similar to~\eqref{EQ:expSDP} can be realized as the sos relaxation of a (constrained) polynomial optimization problem~\cite{ODonnell2017}. 
Recent work shows that polynomial-time computability of the sos hierarchy can be guaranteed under additional assumptions~\cite{Raghavendra2017,Gribling2023, Bortolotti2025, Bortolotti2025a}. Essentially, these works show that certain special classes of semidefinite programs admit effectively bounded solutions.

\paragraph{Sum-of-squares-convexity.}
The interplay between convex polynomials and sums of squares has also been studied.
A polynomial $p \in \R[x]$ is called \emph{sum-of-squares-convex} if the $2n$-variate polynomial $\langle y, \nabla^2 p(x) \cdot y \rangle$
is a sum of squares of polynomials in $\R[x ,y]$. This implies that~${\nabla^2 p(x) \succeq 0}$ for all $x \in \R^n$. 
Thus, sos-convex polynomials are convex in the regular sense, and their convexity has an algebraic proof. (But, not all convex polynomials are sos-convex~\cite{Ahmadi2013}.) 
Contrary to regular convexity (which is \textbf{NP}-hard to detect~\cite{Ahmadi2011}), sos-convexity of a polynomial can be determined by testing feasibility of a semidefinite program~\cite{Ahmadi2024}. 
Moreover, the (global) minimum of an sos-convex polynomial can be determined by solving an SDP. Indeed, the first level of the sum-of-squares hierarchy is exact for sos-convex polynomials (even under sos-convex constraints)~\cite{LasserreConvex}. In light of the discussion on sum-of-squares relaxations above, we remark that (to the best of the authors' knowledge) it has never been examined whether these semidefinite programs can be solved in polynomial time.

