\newpage

%==================================================================
%  Appendix D.  X–dependent amplitudes
%==================================================================

\section{Proof of sign–independence for \texorpdfstring{$x$}{x}–dependent amplitude}
\label{app:damp}

In the main text (see \S\ref{sec:kernel}) the kernel amplitude
was chosen in the form
\[
    a(t,\underline{\xi})=\omega(t)\,\vartheta(\underline{\xi}),
\]
that is, without explicit dependence on~\(x\).
In practice one often requires an \emph{\(x\)–dependent}
amplitude
\[
    a(t,x,\underline{\xi})=\omega(t)\,\vartheta(\underline{\xi})
    \,\chi\!\bigl(x/\lambda^{1/2}\bigr),
\]
where the smooth function \(\chi\) satisfies
\(\|\partial^{\beta}_{x}\chi\|\_{L^{\infty}}\le C\_\beta\)
for all multiindices~\(\beta\).
This appendix shows that such a modification
\emph{strengthens} the negative exponent in~\(\lambda\)
and does not spoil the global balance.

%------------------------------------------------------------------
\subsection{Integration by parts in~\(x'\)}
\label{app:damp-IBP}

\begin{lemma}\label{lem:D1}
Let $L_{x'}$ be the integration–by–parts operator
defined in §\ref{subsec:IBP} (see also \eqref{eq:grad-xp}). 
Then for any function \(\chi=\chi(x/\lambda^{1/2})\) in the definition
of the amplitude one has
\[
    \|L\_{x'}(\chi)\|\_{L^{\infty}}
    \;\;\lesssim\;\;
    \lambda^{-5/6} D^{-1/2}.
\]
\end{lemma}

\begin{proof}
From \eqref{eq:grad-xp} we have
\(
   |\nabla_{x'}\Phi|\gtrsim\lambda\alpha
   =c_{0}\lambda^{1/3}D^{1/2}.
\)
The operator \(L\_{x'}\) itself contributes the factor
\(\lambda^{-1/3}D^{-1/2}\).
Applying \(\partial\_{x'}\chi(x/\lambda^{1/2})\)
produces an additional factor \(\lambda^{-1/2}\).
Combining both gains yields the claimed bound
\(\lambda^{-5/6}D^{-1/2}\).
\end{proof}

\begin{remark}
Without the \(x\)–dependent \(\chi\) the operator \(L\_{x'}\)
gives only \(\lambda^{-1/3}D^{-1/2}\).
Therefore the new amplitude makes the final
coefficient \emph{even more negative} in~\(\lambda\),
which only improves the balance (the “Kernel’’ line in
Table~\ref{tab:global-balance}).
\end{remark}

%------------------------------------------------------------------
\subsection{Conclusion for the full kernel estimate}

None of the stages of the kernel estimate in \S\ref{sec:kernel}
is worsened by replacing the amplitude with
\(\omega(t)\vartheta(\xi)\chi(x/\lambda^{1/2})\);
the formula
\[
    \|K\|\_{L^{2}\to L^{2}}\lesssim\lambda^{-9/2}D^{-3}
\]
remains valid, and under exact counting the exponent
in~\(\lambda\) decreases by another~\(\tfrac12\).

\bigskip
\noindent
\textbf{Conclusion.}\;
The assumption of no \(x\)–dependence in the amplitude
is not essential: adding any
\(\chi(x/\lambda^{1/2})\in C^{\infty}\)
does not worsen, and slightly \emph{improves}, the overall balance of exponents.
All results of the main sections and
Theorem~\ref{thm:main} remain fully valid.
