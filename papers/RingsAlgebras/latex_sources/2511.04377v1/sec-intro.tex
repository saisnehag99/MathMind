\section{Introduction}\label{sec:introin}
Let $\mathbf{C}$ be the category of groups or algebras.
An element $\omega$ from a finitely generated free group or free polynomial ring will be referred to as a \emph{word}. 
For $A\in \mathbf{C}$, and a word $\omega$ on $n$ generators, $\omega$ induces a set-theoretic map
\begin{align}\label{eq:1}
    \widetilde{\omega}:A^n\longrightarrow A,
\end{align}
by means of evaluation. The image will be denoted by $\omega(A)$, by abuse of notation. In the case of groups, these are known as \emph{word maps} and in the case of algebras they are known as \emph{polynomial maps}; however, we will stick with the term word map throughout the article.
After the settlement of Ore's conjecture in \cite{LOST10}, there has been a growing interest in the direction of word maps in groups. 
The study of the polynomial maps was revamped after a somewhat positive solution in \cite{KaMaRo12} to a very old conjecture by L'vov and Kaplansky in the case of a quadratically closed field $F$ and $\M_2(F)$, the full matrix ring of $2\times 2$ matrices with entries from $F$. 
These results incorporate results from the theory of algebraic groups, representation theory of groups, \emph{etc}, to handle the problems in group theory, whereas results from algebraic geometry, arithmetic properties of polynomial equations in associative algebras \emph{etc} are being used to obtain results in the case of algebras.
An interesting related problem in the theory of word maps (or polynomial maps) is a \emph{Waring-type} problem which asks given a word (or polynomial) $\omega$ and a group (or an algebra) $A$ what is the minimal $m$ such that $\langle \omega (A)\rangle=\omega(A)^m$, where $\langle\omega(A)\rangle$ denotes the subgroup (or subalgebra) of $A$, generated by $\omega(A)$; here in case of algebra, $\omega(A)^m$ denotes the set consisting of sums of $m$ elements from $\omega(A)$. 
There are several results in different directions and of different kinds.
We mention a few of them; in a series of three papers \cite{LiebeckShalev01}, \cite{Shalev09} and \cite{LST11} it was proved that for any finite nonabelian simple group $G$ of sufficiently high order and a word $\omega$, one has $\omega(G)^2=G$. 
After \cite{KaMaRo12} appeared there has been results on the $3\times 3$ full matrix ring over algebraically closed fields (see \cite{KaMaRo16}), on upper-triangular matrices in \cite{GargateMello22}, \cite{PanjaPrasad23}, for the matrix ring over reals or quaternions in \cite{PaSaSi24Surj} and on more general algebras in \cite{Matej20}, \cite{MatejPeter23}. 
There are several generalizations of these concepts as well, for example, word maps with constants (see \cite{GoKuPlo18}) or polynomial maps with constants (see \cite{PaSaSi23,PanjaSainiSingh2025O2}).
Given that the literature in this area has grown so much, it is not possible to include all of them, hence we suggest the readers to the articles \cite{Shalev13} for the group theoretic aspect and \cite{KaMaRoYa20} for the ring theoretic aspects.
 
In this article, we introduce a notion of Fatou and Julia sets for $r$-tuples of word maps on groups and for $r$-tuples of polynomial maps on finite-dimensional complex algebras (see \Cref{sec:prelim}).
The subsequent sections analyze two main examples: in \Cref{sec:ring-poly} we describe the Fatou set of $(p,\M_n(\C))$, where $p$ is a polynomial map, and in \Cref{sec:group-power} we determine the Fatou set for the power map $(x^M,\GL_n(\C))$.