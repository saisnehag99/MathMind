\section{Gluing Results}\label{sec:gluing}

This section develops the tools required for showing that the thickening $\cT$ is a topological manifold over the base-space $\cB^{\bR}$ where the fibers carry a smooth structure. Our presentation of the gluing results is similar to \cite[Appendix A]{AMS23} and \cite[\textsection 5]{Par19}.

Let $(\wh Y,J)$ be the symplectization of a contact manifold $(Y,\lambda)$ equipped with an $\bR$-invariant almost complex structure $J$, which satisfies $J(\partial_s) = R$ where $\partial_s$ is the infinitesimal vector field generated by the translation $\bR$ action and $R$ is the Reeb vector field. Let $\Mbar \to \cB^{\bR}$ be a smooth map and define $\cC$ via the pull-back square 
$$
\begin{tikzcd}
\Cbar \arrow{r} \arrow{d} & \overline{\cC_{\cB}} \arrow{d} \\
\Mbar \arrow{r}{}& \cB^{\bR}.
\end{tikzcd}
 $$

\noindent Recall that $\cB^{\bR}$ is the base space, which was obtained from a real-oriented blow-up of certain divisors of $\Mbar^*(\bP^{N}).$ In particular, each fiber in the universal bundle consists of the data of
\begin{itemize}
    \item a curve $C$
    \item matching conditions at certain nodes of $C$
\end{itemize}
Let $\Cbar^0$ denote the complement of the nodal points in $\Cbar$ and $\mathfrak{Y}_{\Cbar}:= \Omega^{0,1}_{\Cbar/\Mbar} \otimes_{\bC} T\wh Y$
be the bundle over $\Cbar^0 \times \wh Y$ which consists of $T \wh Y$-valued anti-holomorphic forms on the vertical bundle of $\Cbar \to \Mbar.$  Let $W$ be a real vector space with a linear map $$ A  \cl W \to \Gamma^\infty_c (\mathfrak{Y}_{\Cbar} ).$$

Given a decorated corolla $T$, we define the regular locus of moduli of buildings $\Mbar_T^{\,\normalfont\text{reg}}(\wh Y)$ to be 

\begin{align}\label{gluing-square} 
    \Mbar_T^{\,\normalfont\text{reg}}(\wh{Y}) = 
	 \left\{
		\begin{array}{l|l}
			\nu \in \Mbar & u \ \textnormal{is smooth and is of type }   T \\ 
			u \cl \Cbar|_\nu \rightarrow{} \wh Y & \delbar_J u + A(w)(\cdot ,u(\cdot)) = 0 \\
			w \in W & u \ \textnormal{is regular} \\
		\end{array} 
		\right\}.
\end{align}
We denote the fiber over a point $\nu \in \Mbar$ by $\Mbar^{\,\normalfont\text{reg}}_T(\wh{Y})|_\nu$. The regularity assumption in \eqref{gluing-square} implies that each fiber $\Mbar^{\,\normalfont\text{reg}}_T(\wh{Y})|_\nu$ carries a unique smooth structure. However, this smooth structure does not necessarily extend to a smooth structure on all of $\Mbar^{\reg}_T(\wh Y)$ due to the resolution of nodes. The next gluing statement shows a somewhat weaker assertion. It is the main result of this section.

\begin{theorem}\label{thm:gluing-main}
    For any $(\nu, u, w) \in  \Mbar_T^{\,\normalfont\text{reg}}(\wh{Y})$, there exists a neighborhood $N$ of $(\nu, u, w)$ in the fiber $\Mbar^{\,\normalfont\text{reg}}_T(\wh{Y})|_\nu$ and a neighborhood $B$ of $\nu$ in $\Mbar$ admitting an embedding $g\cl B \times N \to \Mbar^{\,\normalfont\text{reg}}_T(\wh{Y})$ that fits in the commuting square 
   $$ \begin{tikzcd}
			{B \times N} & {\Mbar^{\,\normalfont\text{reg}}_T(\wh{Y})} \\
			B & { \Mbar}
			\arrow[ from=1-2, to=2-2]
			\arrow["g",hook, from=1-1, to=1-2]
			\arrow["i"', hook, from=2-1, to=2-2]
			\arrow["\proj"', from=1-1, to=2-1]
    \end{tikzcd}$$
    such that the restriction of $g$ to each fiber of the trivial fibration $B\times N \to B$ is smooth.
\end{theorem}

We first prove a special case of Theorem \ref{thm:gluing-main}, assuming that $\Mbar \to \cB^{\bR}$ is an open embedding. Then, we use an exponentiation-along-pullback-of-the-tangent-bundle trick to extend it to any map $\Mbar \to \cB^{\bR}$.

\begin{proposition}\label{prop:c1loc}
    For any two gluing maps $g_1,g_2$ as constructed in Theorem \ref{thm:gluing-main} the restrictions to the fibers over $B_1 \cap B_2$ 
    \begin{equation}
        g_{21,b} \coloneqq g_2\inv \circ g_1|_{b} \cl (\{b\} \times N_1 )\cap g_1\inv(g_2(B_2 \times N_2)) \to \{b\} \times N_2 
    \end{equation}
     vary continuously in the $C^1_{loc}-$topology.
\end{proposition}

\begin{corollary}\label{cor:tangent-bundle}
    There exists a well-defined \emph{vertical tangent bundle}
\begin{equation}\label{eq:vert-tangent-bundle}T^v\Mbar_T^{\reg}(\wh Y) \to \Mbar_T^{\reg}(\wh Y)\end{equation}
with fibers given by
\begin{equation}\label{eq:vert-tangent-bundle-fiber}
T^v_{(b,u)}\Mbar_T^{\reg}(\wh Y) = \ker(D_\varphi+ P).\end{equation}
\end{corollary}

\subsection{Setup for Gluing} 
We will present the gluing treatment as done in \cite[\S 5]{Par19} for the sake of completeness. The main gluing theorem as stated in op. cit. is a local homeomorphism result, but actually the gluing map is smooth for a fixed gluing parameter. In our discussion we will point out how to extract that result from \cite[\S 5]{Par19} and use it to prove Theorem \ref{thm:gluing-main}.
To be consistent with notations in \cite{Par19} we recall some relevant concepts here. 

\subsubsection{Preliminaries on \textsection 2.2 and 2.5 of \cite{Par19}}
In \textsection\ref{subsec:buildings}, we defined the categories $\scS$ stratifying the moduli spaces $\Mbar^J_P(\Gamma^+,\Gamma^-;\beta)$ of Pardon buildings. The space of gluing parameters of decorated tree $T$ as in Definition~\ref{de:decorated-tree} is defined to be 
$$G_{T/} \coloneqq (0,\infty]^{E^{int}(T)}.$$
Given a gluing parameter $\mathbf{\ell}= \{\ell_e \}_e$, the tree type $T_\ell$ is obtained by contracting all the edges for which $\ell_e < \infty.$ In certain setups, it is conceptually beneficial to reparametrize $G_{T/} \cong [0,1)^{E^{int}(T)}$ by applying the function $x\to e^{-x}$ to each factor.

\begin{convention}
    In this section we will implicitly assume that the gluing parameter space $\bC^{N_0}$ actually denotes a small open neighborhood of $0$ in $\bC^{N_0}$.
\end{convention}

\subsubsection{Target Gluing}\label{ssc:targ_glu}

 Fix a point $(\nu,[u],w)\in \Mbar_T^{\reg}(\wh Y)$ in the fiber over $\nu \in \cBR$. For a gluing parameter $\ell \in G_{T_\nu/}$, we define the \textit{glued target} $\wh Y_\ell$ as follows. For each positive (resp. negative) end we truncate $[0,\infty) \times Y$ (resp. $(-\infty,0] \times Y$) to $[0,\ell_e]$ (resp. $[-\ell_e,0]$)and identify the truncated ends corresponding to the edge $e\in E^{int}(T)$ by translation by $\ell_e$. If $\ell_e = \infty$ for some edge $e$, we do nothing for that edge. The target constructed by these gluing operations is denoted as $\wh Y_\ell$. 

 We also select local sections $q_i'$ of the universal curve over $\cBR$ near $\nu$  for every curve corresponding to a vertex of $T_\nu$, such that $\pi_\bR (u(q_i'(\nu)) = 0.$  In our gluing construction, we will send these marked points to their corresponding $0-$levels in the glued target $\wh{Y}_{\textbf{g}}$. This choice of sections is intuitively a gauge-fixing for the $\bR$ action on the target. In particular, it allows us to pick a map $u\cl \cC^0_{\nu} \to \bR \times Y $ in the equivalence class of maps $[u]$.

\subsubsection{Gluing in the base}\label{ssc:base_gluing}

Before elaborating on the gluing chart in the base space, we show how we can reduce to the case of the real-oriented blow-up of the moduli space of stable curves.

\begin{lemma}\label{thm:stab_map_to_curv}
    For any $\varphi \in \cBR(d)$ there exist $d'=d(d+2)$ hyperplanes $D_i$ in $\bP^d$ such that $\wh\varphi$ intersects $D_i$ transversely away from the marked points and there is an open embedding 
    \begin{align*}
        U &\to \Mbar^\bR_{0,d'+ E^{\normalfont\text{ext}}} \\ 
        [\check{\phi},C,j,x,m]&\mapsto  [C,j,x\cup\check{\phi}\inv(D_1)\cup \dots\cup\check{\phi}\inv(D_{d'}),x]
    \end{align*}
    of a neighborhood of $\varphi_0$ into real-oriented blow-up $\Mbar^\bR_{0,d'+ E^{\normalfont\text{ext}}(T)}$ of $\Mbar_{0,d'+E^{\normalfont\text{ext}}(T)}$ at the boundary divisors.
\end{lemma}

\begin{proof}
    This is a generalization of \cite[Lemma~5.15(3)]{BX22}, which can be shown in the same way, using \cite[Proposition~6.5]{AMS21}.
\end{proof}


For the element $\nu = (\varphi_0,C_0,j_0, m)$ of $\cBR$, we will construct a neighborhood in $\cBR$ based on the asymptotics of the map $u$ . Recall that $(C_0,j_0)$ is a marked closed Riemann surface, $\varphi_0 \cl C_0 \hkra \bP^N$ is a holomorphic embedding into $\bP^N$ and $m$ is the additional decoration of asymptotic markers and matching conditions at certain nodes. Pick cylindrical charts of the form $[M,\infty) \times S^1$ or $(-\infty,M]\times S^1$ near each puncture of $C_0$ such that the map $u$ satisfies $$u(s,t) = (Ls,\gamma(t)) + O(e^{-ks}).$$
\noindent Note that such a choice of cylindrical end is equivalent to the choice of a tangent vector $v\in T_pC_v$ at the puncture $p$.

To get a gluing chart near $\nu \in \cBR$, we use the identification from Lemma~\ref{thm:stab_map_to_curv}. We will use the local model of the moduli space of genus $0$ curves with asymptotic markers and matching conditions as \cite[\S 2.6 - 2.7, 5.1.3]{Par19}. We denote the component of the curve $C_0$ corresponding to the vertex $v\in V(T)$ as $C_0^v$. Assume that $C_0^v$ has $N_v$ nodes. Recall that nodes do not carry matching conditions.
Pick a chart
$$\scJ_v :=  \bC^{2\#p_v + \# q_v  -N_v - 3} \to \Mbar_{0,q_{v,i}, {p_{v,i}}_{(2)} }^{\#\text{nodes}=\#N_v} $$ 
which sends $0$ to the curve $(C_0^v,j_0^v)$ and $z \in \scJ_v$ to $(C_0^v,j_z^v)$, where $j_z^v$ is a complex structure agreeing with $j_0^v$ near the special points. 
The subscript $(2)$ denotes that the points $p_{v,i}$ are doubly-marked, i.e., the tangent space $T_{p_{v,i}}C_0^v \sm \{0 \}$ carries a marking (equivalently, there is an associated complex isomorphism $\bC \to T_{p_{v,i}}C_0^v$).
Moreover, we can identify $\scJ_v$ with an open neighborhood of $0$ in the cokernel of the map  
\begin{align*}
&\set{ 
  X \in C^\infty(C^v_0, TC^v_0) \;\middle|\;
  \begin{array}{l}
    X \text{ is holomorphic near } p, q, \\
    X(p_{v,i}) = X(q_{v,i}) = 0, \\
    dX(p_i) = 0
  \end{array}}
\,\to\, C^\infty_c(C^v_0 \sm \cup \{ p_{v,i}\} \cup \{ q_{v,i}\},TC^v_0 \otimes \Omega^{0,1}_{C^v_0})\\
&\qquad\qquad \qquad \qquad \qquad \qquad \qquad\qquad\qquad \qquad \qquad \quad X\mapsto \scL_X{j_0}
\end{align*}


\noindent We can then obtain a local diffeomorphism $$ \modulo{\lbr{\prod_{ v\in V(T)} \scJ_v \times \bC^{\# N_v}}}{\sim} \; \; \to f\Mbar_{0,d'+ E^{\normalfont\text{ext}}(T)}|_{T_\nu}\to \cBR|_{T_\nu}.$$
where the quotient $\sim$ is taken over relations induced from the $\bR_>0$ action on the tensor at the doubly marked points. In particular, for every edge $e = (v,v')$, we quotient by the $\bR_{>0}$ action on $T_{p_v} C^v_0 \otimes  T_{p_{v'}} C^{v'}_0 $. Here, $\cBR|^{\text{nodes fixed}}_{T_\nu}$ is the locus of all framed curves of tree type $T_\nu$ with a fixed number of nodes.
Abusing notation, we denote the image of this diffeomorphism by $\cBR|^{\text{nodes fixed}}_{T_\nu}$.
We construct a local chart near $\cBR|_{T_\nu}$ of the form,
\begin{equation}\label{base-gluing-stratum}
    \text{Glue}_{\cBR}:\cBR|^{\text{nodes fixed}}_{T_\nu}\times G_{T_\nu / }\times \bC^{N^{\inn}_{C_0}} \to \cBR,
\end{equation}
where the map $\Glue_\cBR$ is defined by a slight variation of the usual gluing in moduli of genus $0$ curves.
In particular, over the nodes in $\cN_C^\bullet$ we restrict it to the angle $ \theta = (\theta_e)_e$ determined by the matching isomorphisms in $\nu$, while the gluing parameter $\ell = (\ell_e)_e \in G_{T_\nu/}$ determines the radial component of each coordinate $\textbf{y} = (y_e)_e \in \bC^{E^{int}(T_\nu)}.$
More precisely, suppose $(C_0,j_0,m) \in \cBR$ is a curve with matching conditions $m$, which is over the fiber $(C_0,j_0) \in \cB$ under the blow-down map $\cBR \to \cB$.
Then, 
\begin{equation}
    \text{Glue}_{\cBR} ((\varphi,C,j_0,m) ,\ell,\textbf{z})  = (\varphi_{\textbf{g}},C_{\normalfont\textbf{g}},j_{\normalfont\textbf{g}},m_{\normalfont\textbf{g}}),
\end{equation} 
\noindent where $\textbf{g} = ((y_e)_e,\textbf{z}), \; y_e = (\ell_e e^{i\theta_e})$ and $(C_{\normalfont\textbf{g}},j_{\normalfont\textbf{g}})$ is obtained by the usual truncation-followed-by-gluing construction. The decoration $m_{\normalfont\textbf{g}}$ for the glued curve $(C_{\normalfont\textbf{g}},j_{\normalfont\textbf{g}})$ is induced from $(C_0,j_0,m)$ by keeping the asymptotic markers fixed and remembering the matching conditions at the edges $e$ for which $\ell_e \neq 0.$

\begin{notation*}
    From now on, we will write $\textbf{g} = (\ell,z) \in G_{T_\nu / }\times \bC^{N^{\inn}_{C_0}}$ to denote the `total' gluing parameter, which accounts for gluing along both Reeb punctures and nodes. We also abbreviate $\text{Glue}_{\cBR} (\varphi,C,j,m) , \textbf{g})$ as simply $\nu_{\textbf{g}}=(\varphi_{\textbf{g}},C_{\textbf{g}}).$
\end{notation*}

We can rewrite the gluing map in \eqref{base-gluing} in a conciser form by replacing $\cBR|_{T_\nu} \times \bC^{N^o_{C_0}}$ with the locus of $\cBR$ corresponding to the tree type $T_\nu$,  $\cBR|_{T_\nu}$ .
\begin{equation}\label{base-gluing}
    \text{Glue}_{\cBR}:\cBR|_{T_\nu}\times G_{T_\nu / } \to \cBR,
\end{equation}


\subsubsection{Linearization with respect to varying domain complex structures}
Recall that the linearization of the section $\delbar (\cdot ) + A(\cdot ) (\cdot , \cdot ) $ at the point $(\nu,u,w) \in \Mbar$ is given as 

\begin{equation}
    D^v_0 \cl \wkpd(C_\nu, u^* T\wh Y) \oplus W \to W^{k-1,p,\delta } (C_\nu, u^* T\wh Y\otimes \Omega^{0,1}TC_\nu),
\end{equation}
where the superscript $v$ denotes that this is the `vertical' part of the total linearization with respect to the fibration $\Mbar^{reg}_T(\wh Y)\to \cBR.$
If $(\nu,u,w)$ is in $\Mbar^{reg}_T(\wh Y)$, then the vertical linearization $D^v_0$ is surjective. We also have the linearization $D_0$
\begin{equation*}
     D_0 \cl \wkpd(C_\nu, u^* T\wh Y)\oplus W \oplus \cJ_{T_v}  \to W^{k-1,p,\delta } (C_\nu, u^* T\wh Y\otimes \Omega^{0,1}TC_\nu)  
\end{equation*}
given by
\begin{equation*}
     D_0(\eta,w,\xi) = D_0^v(\eta,w) + J \circ du \circ\xi. 
\end{equation*} 
Due to the regularity assumption, we have the local diffeomorphisms
\begin{align}\label{eq:kernel_at_0_identif}
    \ker D^v_0 \xrightarrow{\sim} \Mbar^{reg}_T(\wh Y)_{\nu}  \\    \ker D_0 \xrightarrow{\sim} \Mbar^{reg}_T(\wh Y)_{T_\nu}, 
\end{align}
near the origin, where $ \Mbar^{reg}_T(\wh Y)_{\nu} $ is the restriction to fiber over $\nu \in \cBR$ and $\Mbar^{reg}_T(\wh Y)_{T_\nu}$ is the restriction to strata $\cBR_{T_\nu}$corresponding to the tree type $T_\nu$.

\subsubsection{Pre-Glued maps}  
For a gluing parameter, $\textbf{g} \in G_{T_\nu/} \times \bC^{N^{\inn}_{C_0}}$. Recall that $\textbf{g}$ defines a positive real parameter $ \ell_e$ for each internal edge $e$ and that there is a Reeb orbit $\widetilde\gamma_e$ for each edge $e$ such that $u_0$ is asymptotic to the trivial cylinder $(Ls,\widetilde \gamma_e(t))$ where $L$ is the $\lambda$-action of $\widetilde \gamma_e.$ We define the \textit{flattened} building $u_0|_{\normalfont\textbf{g}}$ as follows: for every internal edge $e,$ in the chosen cylindrical coordinates $(s,t)$ near the positive puncture corresponding to the edge $e$, we define

$$u_0|_{\normalfont\textbf{g}} (s,t) := \begin{cases}
   \begin{array}{cc}
       u_0(s,t)\quad& s< \frac {1}{6} \ell_e\\
        \exp_{(Ls,\widetilde\gamma_{e}(t))} [ \chi(s-\frac{1}{6}\ell_e). \exp_{(Ls,\widetilde\gamma_{e}(t))}^{-1} u_0(s,t)] \quad  &\frac{1}{6} \ell_e \leq s \leq \frac{1}{6} S + 1 \\
        (Ls,\widetilde\gamma_{e}(t))\quad   \frac{1}{6} &\ell_e + 1<s.
        
   \end{array}    
\end{cases}$$
Near the positive end of a nodal point $n$ of $(C_0,u_0)$, we define the flattening as 


$$u_0|_{\normalfont\textbf{g}} (s,t) := \begin{cases}
   \begin{array}{cc}
       u_0(s,t)\quad& s< \frac {1}{6} S_{\mathsf{n}}\\
        \exp_{u_0(\mathsf{n})} [ \chi(s-\frac{1}{6}S). \exp_{u_0(\mathsf{n})}^{-1} u_0(s,t)] \quad&  \frac{1}{6} S_{\mathsf{n}} \leq s \leq \frac{1}{6} S_{\mathsf{n}} + 1 \\
        u_0(\mathsf{n})  \quad& \frac{1}{6} S_{\mathsf{n}} + 1<s.
        
   \end{array}

    
\end{cases}$$
where $S_{\mathsf{n}} = -\log |z_{\mathsf{n}}|$ for the gluing parameter $z_{\mathsf{n}}$ corresponding to the node $\mathsf{n}$. 
We define the flattening analogously in the negative ends of the punctures and nodes. Here the exponential maps are taken with respect to a fixed $\bR-$invariant metric on $\wh{Y}$.

\begin{definition}
    
We define the \textit{pre-glued} map $u_{\textbf{g}} \cl C_{\textbf{g}} \to \widehat Y_{\textbf{g}}$ to be the natural descent of the flattened map $u_0|_{\normalfont\textbf{g}}.$  
\end{definition}



\subsection{Gluing Estimates} 

We will now cover the required Fredholm setup and compare the linearization of the usual `Floer-function' on the pre-glued map with the linearization before gluing. We will also prove a `kernel-gluing' which is an isomorphism between kernels of linearization before and after pre-gluing. In this section our analysis slightly deviates from Pardon's setup since we consider the gluing map without variation of the domain curve.\\

\noindent\textbf{Fredholm Setup}
We recall the relevant Sobolev spaces required for gluing. We start off with selecting metrics and connections.

\begin{convention}

We fix an $\bR-$invariant metric on the target $\wh{Y}$ and  a $J-$linear connection
on $\wh{Y}$ that is induced from the pullback of the natural map  $\wh{Y} \to \{ 0  \} \times Y$.  On the domain $C$, we fix a metric that is equal to $ds^2 + dt^2$ in the cylindrical coordinates near each puncture $p_e$. We also equip the tangent bundle $TC$ with a connection for which $\partial_s$ is parallel in the cylindrical coordinates near each puncture.
\par
Different choices of metrics or connections are commensurable, so these choices do not affect the topology.
\end{convention}

\subsubsection{Weighted Sobolev norms.}

Recall that the weighted Sobolev space $W^{k,p,\delta} (E)$ of a bundle $E \to C_\g$ consists of those sections that decay at the rate $O(e^{\delta s})$ near each cylindrical end of $C_\g$. In particular, the weighted Sobolev norm $\| \cdot \|_{k,p,\delta}$ has the usual $W^{k,p}$ norm contribution away from the ends and near each cylindrical end of $C_\g$ the contribution is $$\int (\lvert \xi \rvert^p + \lvert \nabla \xi \rvert^p+ \cdots + \lvert \nabla^{k} \xi \rvert^p)e^{p\delta s} ds\; dt$$    
for a section $\xi$ supported in the cylindrical end. The norm is finally constructed by the usual bump function trick.



\subsubsection{Floer Function}
Given a point $(\nu, u_0, w_0) \in \Mbar_T^{\,\normalfont\text{reg}}(\wh{Y}),$ and a gluing parameter, $\textbf{g},$ we know that the pre-glued map, $u_{\normalfont\textbf{g}}$, does not satisfy the equation $\delbar u_{\normalfont\textbf{g}} + A(w_0) ( \cdot, u_{\normalfont\textbf{g}}(\cdot)) = 0.$ But the `defect' of being a true solution can be explicitly measured by the following function in a neighborhood of $(u_{\normalfont\textbf{g}}, w_{\normalfont\textbf{g}})$ in $\Maps(C_{\normalfont\textbf{g}},\wh{Y}).$


\begin{equation}
    \scF_{\textbf{g}}\cl \wkpd(C_{\normalfont\textbf{g}},u_{\normalfont\textbf{g}}^* T\wh Y_{\normalfont\textbf{g}}) \oplus \scJ_{T_v} \oplus W \to W^{k-1,p,\delta}(C_{\normalfont\textbf{g}},u_{\normalfont\textbf{g}}^* T\wh Y_{\normalfont\textbf{g}} \otimes \Omega^{0,1}TC_{\normalfont\textbf{g}}) \oplus \bR^{V(T_{\normalfont\textbf{g}})}
\end{equation}
$$\scF_{\textbf{g}}(\xi,y,w) :=  (\text{P{T}}_{\exp_{u_{\normalfont\textbf{g}}} \xi \to u_{\normalfont\textbf{g}}}\otimes I_{y})\left(d(\exp_{u_{\normalfont\textbf{g}}}\xi) + A(w+w_0)(\cdot, \exp_{u_{\normalfont\textbf{g}}}\xi(\cdot)) \right)^{0,1}_{j_{y}}\oplus \bigoplus_{v\in V(T_{\normalfont\textbf{g}}) } \pi_\bR (\exp_{u_{\normalfont\textbf{g}}} \xi)(q'(v)).  $$
In the above equation, the exponential maps are taken with respect to an $\bR-$equivariant metric on $\wh Y$, $I_y$ is the composition of the natural maps 
$$\Omega^{0,1}_{C_{\normalfont\textbf{g}},j_y} \to \Omega^{0,1}_{C_{\normalfont\textbf{g}}} \otimes_\bR \bC \to \Omega^{0,1}_{C_{\normalfont\textbf{g}},j_0},$$
and the parallel transport $\text{PT}$ is taken with respect to a fixed $\wh{J} _0$ linear connection which is $\bR$ equivariant.
We recall some estimates about the Floer function $\scF_{\normalfont\textbf{g}}$ from \cite[\S 5]{Par19}.

\begin{lemma}[{\cite[Lemma~5.5]{Par19}}]
    We have $\|\scF_{\normalfont\textbf{g}}(0)\|_{k-1,2,\delta} \to 0$ as $\normalfont\textbf{g} \to 0$ for all $k\geq 1.$
\end{lemma}

\begin{lemma}[{\cite[Proposition~5.6]{Par19}}]
    For $\left\|\zeta\right\|_{k,2,\delta},\left\|\eta\right\|_{k,2,\delta}\leq c_{k,\delta}'$, we have
\begin{equation}\label{quad}
\left\|\scF_{\normalfont\textbf{g}}'(0,\eta)-\scF_{\normalfont\textbf{g}}'(\zeta,\eta)\right\|_{k-1,2,\delta}\leq c_{k,\delta}\left\|\zeta\right\|_{k,2,\delta}\left\|\eta\right\|_{k,2,\delta}
\end{equation}
for constants $0 < c_{k,\delta}' , c_{k,\delta}<\infty$ which are bounded uniformly in $\normalfont\textbf{g}$ near $0$, for all $k\geq 4$.
\end{lemma}

We denote the linearization of $\scF_{\normalfont\textbf{g}}$ at $0$ by $D_{\normalfont\textbf{g}}$. Recall that, by assumption, the restriction of $D_0$ to $\wkpd(u_0^* T\wh Y) \oplus W$ is a surjective Fredholm operator. Thus, there is a right inverse $Q_0$. We now state the main kernel gluing and existence of right inverses.

\begin{proposition}
    There is a right inverse $Q_{\normalfont\textbf{g}}$ of $D_{\normalfont\textbf{g}}$ whose norm is bounded uniformly in $\textbf{g}$ and, for sufficiently small $\normalfont\textbf{g}$, an isomorphism of vector spaces 
    $$\Glue_{\ker} \cl \ker D_0 \to \ker D_{\normalfont\textbf{g}}.$$
\end{proposition}

\begin{proof}
    We omit the proof and refer the reader to \cite[\textsection 5.2.8]{Par19} for the construction of the right inverse $Q_{\normalfont\textbf{g}}$. The kernel gluing isomorphism is constructed in Equation (5.40) in op. cit.
\end{proof}

By Equation \eqref{quad}, the derivative $\scF_{\normalfont\textbf{g}}'(v,\cdot)$ is surjective for $\|v \|_{k,p,\delta} < c_{k,\delta} $. Thus $\scF_{\normalfont\textbf{g}}^{-1}(0)$ is a $C^{k-2}-$manifold, which is transverse to $\im \;Q_{\normalfont\textbf{g}}$.

\begin{proposition}\label{prop:proj_g}
    The projection map $\proj  :\scF_{\normalfont\textbf{g}}^{-1}(0) \to \ker D_{\normalfont\textbf{g}}$ along $\im \; Q_{\normalfont\textbf{g}}$ is a local diffeomorphism whose image contains $0 \in \ker D_{\normalfont\textbf{g}}.$
\end{proposition}

\begin{proof}
    This is shown in \cite[\S 5.3.1]{Par19}.
\end{proof}


\subsection{Gluing map}
We can now define a gluing map $\Glue(\textbf{g}, \cdot )\cl $The main gluing theorem in \cite{Par19} proves the following result about local homeomorphism.

\begin{proposition}\label{fixed-g-gluing} 
    Fix a point $\nu$ in the base-space $\cBR.$
    For a given $\normalfont\textbf{g}\in G_{T_\nu/} \times \bC^{N^{\inn}_{C_0}}$, the restriction of the gluing map 
    $$\Glue(\normalfont{\textbf{g}}, \;\underline{\;\;\;}\;) \cl \Mbar_{T}^{\,\normalfont\text{reg}}(\wh{Y})|_{T_\nu} \to \Mbar_{T}^{\,\normalfont\text{reg}}(\wh{Y})|_{T_{\Glue_{\cBR}(\nu,\textbf{g})}} $$ given by the equation 
    \begin{equation}\label{eq:gluing_map}
    \Glue(\textbf{g}, \;\underline{\;\;\;}\;) =  \exp_{u_{\normalfont\textbf{g}}} \circ \proj_{\ker D_{\textbf{g}}}^{-1}\circ \Glue_{\ker}\circ \proj_{\ker D_0}
\end{equation}
    is a local diffeomorphism.
\end{proposition}

\begin{proof}
    Fix a point $(u_0,w_0) \in  \Mbar_{T}^{\,\normalfont\text{reg}}(\wh{Y})|_\nu$. For a fixed $\textbf{g}$, the gluing map is the composition of local diffeomorphisms and thus itself is a local diffeomorphism. The map $\proj_{\ker D_0} \cl \Mbar_{T}^{\,\normalfont\text{reg}}(\wh{Y})|_\nu \to \ker D_0 $ is the inverse of the local diffeomorphism $\ker D_0 \to \Mbar_{T}^{\,\normalfont\text{reg}}(\wh{Y})|_\nu$ defined on a small neighborhood of the origin. The middle map $\Glue_{\ker}$ is the kernel gluing map $$\Glue_{\ker} \cl \ker D_0 \to \ker D_{\textbf{g}},$$ and $\proj_{\ker D_{\textbf{g}}}$ is the restriction of the projection map in Proposition \ref{prop:proj_g}. This finishes the proof.
\end{proof}

\begin{proof}[Proof of Theorem \ref{thm:gluing-main}]
We will only prove the case of $\Mbar \to \cB^{\bR}$ being an open embedding since the general result then follows from a similar argument as \cite[Corollary A.2]{AMS23}. Due to the strong regularity of $(\nu,u,w)$, there exists a neighborhood $B_{T_\nu}$ of $\nu$ in the submanifold $ \cBR|_{T_\nu} $ of domains of tree type $T_\nu$ and a neighborhood $N$ of $(\nu,u,w)$ in the fiber $\Mbar^{reg}_T(\wh Y)|_{(\nu,u,w)}$ such that there is an embedding $B_{T_\nu} \times N \to \Mbar^{reg}_T(\wh Y)|_{T_\nu}$. Fix a chart in a neighborhood of $B_{T_\nu}$ given by the map 
$$ \text{Glue}_{\cBR}:B|_{T_\nu}\times G_{T_\nu / } \to B \underset{open}{\subset} \cBR,$$
as described in \eqref{base-gluing}. Under the identification of a neighborhood of $(\nu,u,w)$ in the fiber $\Mbar^{reg}_T(\wh Y)|_{T_\nu}$  with $B|_{T_\nu} \times N$, we have that the gluing map, $\Glue$ can be written as 

$$\Glue \cl B|_{T_\nu} \times N \times G_{T_\nu/}  \to \Mbar^{reg}_T(\wh Y). $$

Now the result is a direct consequence of Proposition~\ref{fixed-g-gluing}.
\end{proof}

\begin{proof}[Proof of Proposition \ref{prop:c1loc}]
We rephrase the proposition using the language developed in this appendix. Let $(\nu_i,u_i,w_i) \in \Mbar^{reg}_T(\hat{Y}), \; i\in \{ 1,2\}$  be a pair of  points and let $\Glue^i$ be gluing maps constructed as above. We also assume that these gluing neighborhoods intersect non-trivially. In particular, let $N_i \sub \Mbar^{reg}_T(\hat{Y})|_{\nu_i}$ be neighborhoods of $(\nu_i,u_i,w_i)$ in the fiber and suppose $B_i$ are neighborhoods of $\nu_i$ in the $\cBR$ such that $B_1 \cap B_2 \neq \emptyset$ and $\Glue^1(B_1\cap B_2,  N_1) \cap \Glue^2(B_1\cap B_2,  N_2) \neq \emptyset.$ We can further assume that $N_i's$ are open neighborhoods of $0$ in the kernel of the respective linearization, $\ker D_0^i$. Thus, we can identify $N_1$ and $N_2$ with open neighborhoods of the origin in finite-dimensional Euclidean space. After potentially replacing $N_1$ with a subset, we have a map  $$G:=(\Glue^2)\inv \circ \Glue^1 \cl ( B_1 \cap B_2)\times N_1 \to ( B_1 \cap B_2)\times N_2 .$$  Now it is enough to check that the map $G(g,\;\underline{\;\;} \;) \cl N_1 \to N_2$ depends continuously on $g$ in the $C^1_{loc}$ topology. By viewing the map $G$ using the definition of gluing map as defined in Equation~\eqref{eq:gluing_map}, we see that it is enough to check continuity of derivative of 
$$ \Glue_{\ker D^2_0}\inv\circ\proj_{\ker D^2_g} \inv\circ\exp_{u_2,\textbf{g}}\inv \circ\exp_{u_1,\textbf{g}} \circ \proj_{\ker D^1_g}\inv \circ \Glue_{\ker D^1_0}\cl N_1 \to N_2$$ 
where $D^i_\textbf{g}$ is defined similarly to $D_\textbf{g}$ above. Now the result follows from the fact that the derivative of $\exp_{u_2,\textbf{g}}\inv \circ \exp_{u_1,\textbf{g}} $ depends continuously on $\textbf{g}$ and the construction of the right inverse $Q_\textbf{g}$ depends continuously on $\textbf{g}$, see \cite[\S 5.2.7-5.2.8]{Par19}.
\end{proof}




