\section{Actions on super and graded \texorpdfstring{$\glt$}{gl2}-foams}
\label{sec:action_on_foams}




\subsection{Graded structures}

% \begin{verbatim}
%   NOTATIONS
%   - generic ring: \ring
%   - generic group: \grp
%   - elements of \grp: g, h
%   - generic bilinear map: \bil
%   - scalars: \lambda, (\mu)

%   - generic category: A
%   - objects in category: u, v, w
%   - morphisms in category: \alpha, \beta, \gamma
  
%   - generic 2-category: \cA
%   - objects in 2-categories: i, j, k
%   - 1-morphisms in 2-categories: u, v, w
%   - 2-morphisms in 2-categories: \alpha, \beta, \gamma

%   - vertical composition: \circ
%   - horizontal composition: \otimes
% \end{verbatim}

% \newpage

In this subsection, we describe the super, and more generally graded, analogue of various structures familiar in the commutative setting.
After defining graded associative algebras and graded Lie algebras, we review graded-2-categories from \cite{SV_OddKhovanovHomology_2023}, and define a $\fg$-2-category as a graded-2-category endowed with an action of a graded Lie algebra $\fg$; this specializes to the notion of $\slt$-categories from \cite{EQ_Actions$mathfraksl_2$Algebras_2023}.
Finally, we describe the graded analogue of twists \cite{KR_PositiveHalfWitt_2016,QRS+_Symmetries$mathfrakgl_N$foams_2024a}.

We fix throughout a commutative ring $\ring$, an abelian group $\grp$ and a pairing $\bil\colon\grp\times\grp\to\ring^\times$, that is, a bilinear map.
We further assume that $\bil$ is \emph{symmetric}, in the sense that
\[\bil(g,h)\bil(h,g)=1\quad\forall g,h\in\grp.\]
We write $\deg(v)$ the degree of an element $v$ in a $G$-graded object, although we often abuse notation and write $\mu(\deg v,\deg w)$ simply as $\mu(v,w)$.
Given two $G$-graded $\ring$\nbd-modules $M$ and $N$, we write $\Hom(M,N)$ the $\ring$\nbd-module of degree-preserving $\ring$-linear maps between $M$ and $N$, and $\uHom(M,N)$ the $\grp$-graded $\ring$-module of all $\ring$-linear maps, not necessarily degree-preserving.
We write $\End(M)\coloneqq\Hom(M,M)$ and $\uEnd(M)\coloneqq\uHom(M,M)$.

We denote $\Mod_{\grp,\bil}$ the closed symmetric monoidal category of $\grp$-graded $\ring$-modules and deg\-ree-preserving linear maps.
Its monoidal structure is the usual one on $G$-graded $\ring$-modules; note that it does not depend on $\bil$, and we write $\Mod_{\grp}=\Mod_{\grp,\bil}$ when considered only as a monoidal category.
The symmetric structure is given by $(x,y)\mapsto\bil(x,y)(y,x)$ and the inner Hom is given by $\uHom$.
% \[\underline{\Hom}_{\Mod_{\grp,\bil}}(M,N)=\uHom(M,N).\]

We denote $\uMod_{\grp,\bil}$ the symmetric monoidal category whose objects are $\grp$-graded $\ring$-modules (the same as $\Mod_{\grp,\bil}$) and with $\uHom(M,N)$ as $G$-graded homspace between $M$ and $N$.
In other words, the category $\uMod_{\grp,\bil}$ is the $\Mod_{\grp,\bil}$-enriched category determined by the closed monoidal structure of $\Mod_{\grp,\bil}$; as an $\Mod_{\grp,\bil}$-enriched category, its underlying category is $\Mod_{\grp,\bil}$ (see e.g.\ \cite[section~3.4]{Riehl_CategoricalHomotopyTheory_2014}).

We sometimes simplify notation and write $\Mod=\Mod_{\grp,\bil}$ and $\uMod=\uMod_{\grp,\bil}$.

\subsubsection{Graded associative algebras}
\label{subsubsec:graded_associative_algebras}

\begin{definition}
  A \emph{$\grp$-graded (associative) algebra} is a unital associative algebra object in the monoidal category $\Mod_{\grp}$.
\end{definition}

That is, a $\grp$-graded algebra is a unital and associative algebra $(A,\cdot_A,1_A)$, such that $A$ is $\grp$-graded as a $\ring$-module, the multiplication is degree-preserving and the unit has trivial degree.
Similarly, a \emph{morphism of $\grp$-graded algebras} is a morphism of unital associative algebra objects in the monoidal category $\Mod_{\grp}$; that is, a degree-preserving linear map preserving the unit and the product.

\medbreak

Let $M$ be a $\grp$-graded $\ring$-module. The algebra $\uEnd(M)$ of linear maps on $M$ has a canonical structure of $\grp$-graded algebra. If $A$ is a $\grp$-graded algebra, an \emph{action of $A$ on $M$} is morphism of $\grp$-graded algebra $A\to\uEnd(M)$. We say that $M$ is an \emph{$A$-module}, and a \emph{morphism of $A$-modules} is a degree-preserving linear map intertwining the actions.

\begin{definition}
  \label{defn:graded-commutative}
  A \emph{$(\grp,\bil)$-graded commutative algebra} is a commutative unital associative algebra object in the symmetric monoidal category $\Mod_{\grp,\bil}$.
\end{definition}

That is, a $(\grp,\bil)$-graded commutative algebra is a $\grp$-graded algebra where for every homogeneous $x$ and $y$, we have $xy=\bil(x,y)yx$.
Note that a $\grp$-graded algebra is always an algebra, while a $(\grp,\bil)$-graded commutative algebra needs not be a commutative algebra.




\subsubsection{Graded Lie algebras}
\label{subsubsec:graded_lie_algebras}

\begin{definition}
  A \emph{$(\grp,\bil)$-graded Lie algebra} is a Lie algebra object in the symmetric monoidal category $\Mod_{\grp,\bil}$.
\end{definition}

That is, a $(\grp,\bil)$-graded Lie algebra is a $\grp$-graded $\ring$-module $\fg$ equipped with a degree-preserving map $[-,-]\colon\fg\otimes\fg\to\fg$ such that
\begin{gather*}
  [x,y]+\bil(x,y)[y,x]=0\\
  % \bil(y,x)\bil(z,x)\bil(z,y)[x,[y,z]]+\bil(z,y)[y,[z,x]]+\bil(y,x)[z,[x,y]]=0\\
  % \Leftrightarrow
  [x,[y,z]]+\bil(x,y+z)[y,[z,x]]+\bil(x+y,z)[z,[x,y]]=0
\end{gather*}
Similarly, a \emph{morphism of $\grp$-graded Lie algebras} is a morphism of Lie algebra objects in the monoidal category $\Mod_{\grp}$; that is, a degree-preserving linear map preserving the bracket.

\medbreak

Let $A$ be a $\grp$-graded algebra.
We endow $A$ with the structure of a $(\grp,\bil)$-graded Lie algebra, stating that:
\begin{gather*}
  [f,g]\coloneqq f\circ g -\bil(f,g)\;g\circ f.
\end{gather*}
This applies in particular if $A=\uEnd(M)$ for some $\grp$-graded $\ring$-module $M$.
If $\fg$ is a $(\grp,\bil)$-graded Lie algebra, an \emph{action of $\fg$ on $M$} is a morphism of $(\grp,\bil)$-graded Lie algebras $\fg\to\uEnd(M)$. We say that $M$ is a \emph{$\fg$-module}, and a \emph{morphism of $\fg$-modules} is a degree-preserving linear map intertwining the $\fg$-action.
Given two $\fg$-modules $M$ and $N$, we write $\Hom^\fg(M,N)$ the $\ring$-module of morphisms of $\fg$-modules.
Abusing notation, we denote $\uHom(M,N)$ the $G$-graded $\ring$-module of all linear maps, now endowed with the following $\fg$-action:
\begin{equation}
  \label{eq:inner_hom_g_categories}
  g\cdot \alpha \coloneqq\tau_g^M\circ\alpha - \bil(g,\alpha)\;\alpha\circ\tau_g^N,
\end{equation}
for $g\in\fg$ and $\alpha\in\uHom(M,N)$, and where $\tau_g^M$ (resp.\ $\tau_g^N$) denotes the action of $g$ on $M$ (resp.\ $N$).




\begin{example}
  \label{ex:Lie_algebra_as_graded_algebra}
  If $(\grp,\bil)$ is trivial, a $(\grp,\bil)$-graded Lie algebra is a Lie algebra over $\ring$. If only $\bil$ is trivial, a $(\grp,\bil)$-graded Lie algebra is a Lie algebra over $\ring$ equipped with a $\grp$-grading.
\end{example}

\begin{example}[super Lie algebra]
  \label{ex:super_Lie_algebra_as_graded_algebra}
  If $\grp=\bZ/2\bZ=\{\ov{0},\ov{1}\}$ and $\bil(n,m) = (-1)^{nm}$, a $(\grp,\bil)$-graded Lie algebra is a super Lie algebra over $\ring$.
  In this setting, we often write $\abs{v}\coloneqq\deg v$.
  Explicitly, a \emph{Lie superalgebra} is a super vector space $\fg$ endowed with a bilinear degree-preserving map $[-,-]\colon\fg\otimes \fg\to\fg$, satisfying the following axioms:
  \begin{IEEEeqnarray*}{Cl}
    [v,w] = -(-1)^{\abs{v}\abs{w}}[w,v]&\text{graded symmetry}\\{}
    [u,[v,w]] + (-1)^{\abs{u}(\abs{v}+\abs{w})}
    [v,[w,u]] + (-1)^{\abs{w}(\abs{u}+\abs{v})}[w,[u,v]] 
    = 0
    \qquad&\text{graded Jacobi identity}
  \end{IEEEeqnarray*}
\end{example}

\begin{example}[$\gloo$]
  \label{ex:defn_gloo}
  The Lie superalgebra $\gloo$ is presented by generators $\{\lieh_1,\lieh_2,\liee,\lief\}$, where $\abs{\lieh_1}=\abs{\lieh_2}=\ov{0}$ and $\abs{\liee}=\abs{\lief}=\ov{1}$, and relations
  \begin{align*}
    [\liee,\lief] &= \lieh_1+\lieh_2 & [\liee,\liee]&=[\lief,\lief]=[\lieh_i,\lieh_j]=0\\
    [\lieh_1,\liee] &=\liee & [\lieh_1,\lief] &=-\lief\\
    [\lieh_2,\liee] &=-\liee & [\lieh_2,\lief] &=\lief
  \end{align*}
\end{example}

\begin{example}[$\sloo$]
  \label{ex:defn_sloo}
  Setting $\lieh\coloneqq\lieh_1+\lieh_2$ defined the Lie super algebra $\sloo$ as a sub-algebra $\sloo\subset\gloo$.
  In other words, the Lie superalgebra $\sloo$ presented by generators $\{\lieh,\liee,\lief\}$, where $\abs{\lieh}=\ov{0}$ and $\abs{\liee}=\abs{\lief}=\ov{1}$, and relations
  \begin{align*}
    [\liee,\lief] &= \lieh & [\liee,\liee]&=[\lief,\lief]=[\lieh,\lieh]=0\\
    [\lieh,\liee] &=0 & [\lieh,\lief] &=0
  \end{align*}
\end{example}

Anticipating, we give some specific data for $\ring$, $\grp$ and $\bil$ which will be used in the definition of graded $\glt$-foams, and define certain ``covering'' Lie algebras.

\begin{definition}
  \label{defn:graded_structure_foam}
  Let $\ringfoam$ be a commutative ring together with three invertible elements $X$, $Y$ and $Z\in{\ringfoam}^\times$ such that $X^2=Y^2=1$.
  Given this data, let $\bilfoam$ be the following bilinear form for the abelian group $G\coloneq \bZ^2$:
  \begin{align*}
    \bilfoam\colon \bZ^2\times\bZ^2 &\to {\ringfoam}^\times,\\
    ((a,b),(c,d)) &\mapsto X^{ac}Y^{bd} Z^{ad-bc}.
  \end{align*}
  We say ``restrict to the even case'' to mean choosing $X=Y=Z=1$, and ``restrict to the odd, or super, case'' to mean choosing $X=Z=1$ and $Y=-1$.
\end{definition}

\begin{example}[$\grgl_2$]
  \label{ex:defn_covering_gl2}
  Let $\ringfoam$ and $\bilfoam$ as in \cref{defn:graded_structure_foam}.
  Let $\grgl_2$, called \emph{covering $\fgl_2$}, be the $(\bZ^2,\bilfoam)$-graded Lie algebra defined as follows.
  As a $\ringfoam$-module, $\grgl_2$ is generated by the following homogeneous vectors:
  \begin{gather*}
    \deg(\lief)=(1,1),\;
    \deg(\liee)=(-1,-1),\;
    \deg(\lieh_1)=(0,0)
    \an\deg(\lieh_2)=(0,0).
  \end{gather*}
  The structure of graded Lie algebra is then given as follows:
  \begin{align*}
    [\liee,\lief] &= \lieh_1+XY\lieh_2 & [\liee,\liee]&=[\lief,\lief]=[\lieh_i,\lieh_j]=0\\
    [\lieh_1,\liee] &=\liee & [\lieh_1,\lief] &=-\lief\\
    [\lieh_2,\liee] &=-\liee & [\lieh_2,\lief] &=\lief. 
  \end{align*}
  % Leibniz: (automatically of if two of f,g,h are equal)
  % [e,[h_1,h_2]]: e-e=0
  % [f,[h_1,h_2]]: f-f=0
  % [e,[f,h_1]]: [e,f]+\bil(e,f)[f,e]=0
  We further denote $\grgl_2^-\coloneqq\langle\lief\rangle$ and $\grgl_2^{\leq}\coloneqq\langle\lief,\lieh_1,\lieh_2\rangle$, and $\grgl_2^+\coloneqq\langle\liee\rangle$ and $\grgl_2^{\geq}\coloneqq\langle\liee,\lieh_1,\lieh_2\rangle$.
  Restricting to even and odd, we have $\grgl_2^{\leq}\vert_{X=Y=Z=1}=\fgl_2^{\leq}$ and $\grgl_2^{\leq}\vert_{X=Z=1,Y=-1}=\gloo$, respectively.
\end{example}

\begin{example}[$\grsl_2$]
  \label{ex:defn_covering_sl2}
  Following \cref{ex:defn_covering_gl2}, set $\lieh\coloneqq\lieh_1-XY\lieh_2$. The $(\bZ^2,\bilfoam)$-graded Lie algebra $\grsl_2\subset\grgl_2$, called \emph{covering $\fsl_2$}, is defined as generated by $\lief$, $\liee$ and $\lieh$.
  In other words, it has the following defining relations:
  \begin{align*}
    [\liee,\lief] &= \lieh & [\liee,\liee]&=[\lief,\lief]=[\lieh,\lieh]=0\\
    [\lieh,\liee] &=(1+XY)\liee & [\lieh,\lief] &=-(1+XY)\lief
  \end{align*}
  Evaluating to even recovers $\slt\subset\glt$, while evaluating to odd recovers $\mathfrak{sl}_{1|1}\subset \mathfrak{gl}_{1|1}$.
  Note that when working over a field of characteristic two, $\fsl_2=\fsl_{1|1}$.
  %
  Similarly to \cref{ex:defn_covering_gl2}, one can define $\grsl_2^-$, $\grsl_2^{\leq 0}$, $\grsl_2^+$ and $\grsl_2^{\geq 0}$.
\end{example}




\subsubsection{$\fg$-categories}
\label{subsubsec:g-categories}

We denote $\gMod$ the closed symmetric monoidal category of $\fg$-mo\-dules and morphisms of $\fg$\nbd-mo\-dules.
Its closed symmetric monoidal structure coincides with the closed symmetric monoidal structure of $\Mod_{\grp,\bil}$ via the forgetful functor; to complete the definition of the structure, it suffices to define the relevant $\fg$-actions.
For the monoidal structure, the $\fg$-action on the monoidal unit $\ring$ is trivial, and the $\fg$-action on the tensor product $M\otimes N$ is defined as
\[
g\cdot (m\otimes n)\coloneqq (g\cdot m)\otimes n + \bil(g,m) \,m\,(g\otimes n).
\]
One could view this symmetric monoidal structure as coming from some graded Hopf structure on the enveloping algebra of $\fg$; we omit this point of view.
The inner Hom is $\uHom$ with the structure of $\fg$-module given in \eqref{eq:inner_hom_g_categories}.

% Recall that given a monoidal category $\cV$, there is a notion of $\cV$-enriched category (see e.g.\ \cite[Chapter~6]{Borceux_HandbookCategoricalAlgebra_1994a}).

\begin{definition}
  A \emph{$\fg$-category} (resp.\ a \emph{$\fg$-functor}) is a $(\gMod)$-enriched category (resp.\ a $(\gMod)$-enriched functor).
\end{definition}

Note that this definition does not depend on the symmetric structure on $\gMod$.

We unpack the definition.
Given the forgetful functor $\gMod\to\Mod_{\grp}$, a $\fg$-category $A$ is in particular a $\grp$-graded category.
In addition, the $\fg$-category $A$ carries a family of linear maps
\begin{equation}
  \label{eq:g-cat-family-of-actions}
\fg\to \uEnd(\Hom_A(u,v))
\end{equation}
for each pair of objects $(u,v)$, that satisfies the \emph{$(\grp,\bil)$-graded Leibniz rule}:
\begin{equation}
  \label{eq:graded_Leibniz_rule}
  g\cdot (\alpha\circ\beta) = (g\cdot \alpha)\circ \beta + \bil(g,\alpha)\, f\circ(g\cdot \beta),
\end{equation}
where $\alpha$ and $\beta$ are suitably composable morphisms of $A$.
Whenever a $\grp$-graded category $A$ is equipped with a family of $\fg$-module morphisms as in \eqref{eq:g-cat-family-of-actions} satisfying the graded Leibniz rule \eqref{eq:graded_Leibniz_rule}, we say that \emph{$\fg$ acts by derivation on $A$}.

\begin{lemma}
  A $\fg$-category is the same as $\grp$-graded category equipped with an action of $\fg$ by derivation.\hfill\qed
\end{lemma}

\begin{remark}
  If $w$ is an object of $A$, it follows from the graded Leibniz rule that $g\cdot \id_w = g\cdot(\id_w\circ\id_w) = g\cdot \id_w + g\cdot \id_w$, so that $g\cdot \id_w=0$.
\end{remark}

\begin{example}
  \label{ex:ugMod}
  Let $\ugMod$ be the symmetric monoidal category whose objects are $\fg$-modules and with $\uHom(M,N)$ as the $\fg$-module homspace between $M$ and $N$.
  By definition, the category $\ugMod$ is a $\fg$-category.
  In fact, it is the $(\gMod)$-enriched category determined by the closed monoidal structure on $\gMod$, whose underlying category (as a $(\gMod)$-enriched category) is $\gMod$.
\end{example}

\begin{definition}
  \label{defn:g_equivariant}
  Let $A$ be a $\fg$-category. A morphism $\alpha$ is said to be \emph{$\fg$-equivariant} if $g\cdot\alpha=0$ for all $g\in\fg$.
\end{definition}

If $A=\ugMod$, then a morphism $\alpha$ is $\fg$-equivariant in the sense of \cref{defn:g_equivariant} if and only if it is $\fg$-equivariant in the usual sense, that is, if $\alpha$ intertwines the $\fg$-action on its source and target.

% Given a $\fg$-category $A$, one can consider the contravariant Yoneda embedding:
% \begin{align}
%   \label{eq:yoneda_embedding}
%   Y^{\mathrm{op}}\colon A^{\mathrm{op}}&\to \Fun(A,\gMod),\\
%   w&\mapsto\Hom_A(-,w),
% \end{align}



\subsubsection{$\fg$-2-categories}
\label{subsubsec:g-2-categories}

Recall that if $\cV$ is a symmetric monoidal category, then the category $\cV\md\cC at$ of $\cV$-enriched categories is itself symmetric monoidal, and one can enriched over $\cV\md\cC at$. A \emph{$\cV$-enriched 2-category} is a $(\cV\md\cC at)$-enriched category.

\begin{definition}[{\cite[Remark~2.7]{SV_OddKhovanovHomology_2023}}]
  A $(\grp,\bil)$-graded-2-category is a $(\Mod_{\grp,\bil})$-enriched 2-cate\-gory.
\end{definition}

Unpacking the definition, a $(\grp,\bil)$-graded-2-category is akin to a $\grp$-graded $\ring$-linear strict 2-category, except that the interchange law is replaced by the \emph{graded interchange law}:
\begin{equation*}
  \tikzpic{
    \draw (0,2) node[above] {\scriptsize $v'$} to
      node[fill=black,circle,inner sep=2pt,pos=.3] {}
      node[right,pos=.3] {\scriptsize $\beta$}
        (0,0) node[below] {\scriptsize $u'$};
    \draw (1,2) node[above] {\scriptsize $v$} to
      node[fill=black,circle,inner sep=2pt,pos=.7] {}
      node[right,pos=.7] {\scriptsize $\alpha$}
        (1,0) node[below] {\scriptsize $u$};
  }[scale=0.8]
  \;=\;\bil(\deg\alpha,\deg\beta)
  \tikzpic{
    \draw (0,2) node[above] {\scriptsize $v'$} to
      node[fill=black,circle,inner sep=2pt,pos=.7] {}
      node[right,pos=.7] {\scriptsize $\beta$}
        (0,0) node[below] {\scriptsize $u'$};
    \draw (1,2) node[above] {\scriptsize $v$} to
      node[fill=black,circle,inner sep=2pt,pos=.3] {}
      node[right,pos=.3] {\scriptsize $\alpha$}
      (1,0) node[below] {\scriptsize $u$};
  }[scale=0.8]
\end{equation*}

\begin{definition}
  A \emph{$\fg$-2-category} (resp.\ $\fg$-2-functor) is a $(\gMod)$-enriched 2-category (resp.\ $(\gMod)$-enriched 2-functor).
  A \emph{$\fg$-monoidal category} is a one-object $\fg$-2-category.
\end{definition}

We unpack the definition.
A $\fg$-2-category is in particular a $(\grp,\bil)$-graded-2-category, denoting its horizontal (resp.\ vertical) composition by $\otimes$ (resp.\ $\circ$). In addition, for each pair of objects $(x,y)$ the hom-category $\Hom(x,y)$ is a $\fg$-category.
Furthermore, the action of $\fg$ satisfy the $(\grp,\bil)$-graded Leibniz rule with respect to the horizontal composition;
equivalently, the action commutes with horizontal whiskering:
\begin{equation}
  \label{eq:axiomC_gcategories}
  \fg\cdot(\id_u\otimes\alpha\otimes\id_v) = \id_u\otimes(\fg\cdot\alpha)\otimes\id_v,
\end{equation}
where $u,v$ are 1-morphisms and $\alpha$ is a 2-morphism, suitably composable.

A $(\grp,\bil)$-graded-2-category $\cA$ equipped with a family of $\fg$-module morphisms
\[\fg\to\uEnd(\Hom_\cA(u,v))\]
indexed by pair of 1-morphisms $(u,v)$ with the same source and target, such that the action of $\fg$ defines an action by derivation on each Hom-category $\Hom_\cA(i,j)$ for pair of objects $(i,j)$, and furthermore verifies axiom \eqref{eq:axiomC_gcategories}, we say that \emph{$\fg$ acts by derivation on $\cA$}.

\begin{lemma}
  \label{lem:equivalent_defn_g2cat}
  A $\fg$-2-category is the same as a $(\grp,\bil)$-graded-2-category equipped an action of $\fg$ by derivation.\hfill\qed
\end{lemma}




\begin{example}
  \label{ex:sl2category_as_gcategory}
  Following up on \cref{ex:Lie_algebra_as_graded_algebra}, if $\ring=\bZ$, if $(\grp,\bil)=(\bZ,1)$ and if $\fg=\fsl_2$ equipped with the $\bZ$-grading $\abs{\lief}=2$, $\abs{\liee}=-2$ and $\abs{\lieh}=0$, then a $\fg$-monoidal category is an $\slt$-category in the sense of \cite{EQ_Actions$mathfraksl_2$Algebras_2023}.
\end{example}

\begin{example}
  \label{ex:dg_as_gcategory}
  Let $(\grp,\bil)=(\bZ,1)$ and $\ring$ a ring of characteristic $p$.
  If $\fg=\ring\partial$ is the one-dimensional abelian $(\grp,\bil)$-graded Lie algebra concentrated in degree $\abs{\partial}=2$, a $\fg$-monoidal category is a graded monoidal category equipped with an action by derivation $\partial$ of degree $2$.
  If this action is $p$-nilpotent, then this category is a $p$-DG-category in the sense of hopfological algebra \cite{Khovanov_HopfologicalAlgebraCategorification_2016,KQ_ApproachCategorificationSmall_2015,Qi_HopfologicalAlgebra_2014}.
\end{example}

\begin{example}
  \label{ex:super_dg_as_gcategory}
  Let $(\grp,\bil)$ as in \cref{ex:super_Lie_algebra_as_graded_algebra}.
  If $\fg=\ring\partial$ is the one-dimensional abelian super Lie algebra concentrated in degree $\abs{\partial}=\ov{1}$, then a $\fg$-2-category is a dg-2-super\-ca\-te\-gory in the sense of \cite{EL_DGStructuresOdd_2020}.
\end{example}

\begin{example}
  \label{ex:graded_commutative_dg_algebra}
  A $\fg$-2-category with one object and one morphism is a $(\grp,\bil)$-graded-com\-mu\-ta\-tive algebra equipped with an action of $\fg$ by derivation.
  In the setting of , then $(\grp,\bil)$-graded-commutativity recovers graded-commutativity in the usual sense, and if further the action of $\partial$ is nilpotent, we recover the notion of a graded-com\-mu\-ta\-tive DG-algebra.
\end{example}

\begin{remark}
  \label{rem:derivation_uniquely_defn_on_generators_of_A}
  If $\cA$ is a $(\grp,\bil)$-graded-2-category defined by generators and relations, an action by derivation is soly determined by the action on the generators. Conversely, to define an action by derivation, it suffices to define it on the generators and verify that it preserves the defining relations.
  The graded interchange law needs not be verified: it follows from the graded Leibniz rule that any action by derivation preserves the graded interchange law.
\end{remark}

\begin{remark}
  \label{rem:action_uniquely_defn_by_generators_of_g}
  If $\cA$ is a $(\grp,\bil)$-graded-2-category and $\fg$ is a $(\grp,\bil)$-graded Lie algebra defined by generators and relations, an action of $\fg$ on $\cA$ by derivation is solely determined by the action of the generators of $\fg$.
  Conversely, to define an action $\fg$ on $\cA$ by derivation, it suffices to define it on the generators of $\fg$, and verify that it satisfies the defining relations of $\fg$.
  % Indeed, the bracket of two elements that verify the graded Leibniz rule also verifies the graded Leibniz rule, the bracket of two elements that commute with the horizontal whiskering also commute with the horizontal whiskering.
  % \begin{align*}
  %   [f,g](xy)
  %   =&(f\circ g -\bil(f,g)g\circ f)(xy)
  %   =fg(xy)-\bil(f,g)gf(xy)
  %   \\
  %   =&fg(x)y+\bil(f,g(x))g(x)f(y)+\bil(g,x)f(x)g(y)+\bil(f,x)\bil(g,x)xfg(y)
  %   \\
  %   &{}-\bil(f,g)\big[gf(x)y+\bil(g,f(x))f(x)g(y)+\bil(f,x)g(x)f(y)+\bil(g,x)\bil(f,x)xgf(y)\big]
  %   \\
  %   =&[f,g](xy)+\bil(f+g,x)x[f,g](y)
  % \end{align*}
\end{remark}




\subsubsection{Twisting $\fg$-2-categories}
\label{subsubsec:twisting_g2categories}

Let $\cA$ be a $\fg$-2-category.
Consider a family of degree-preserving linear maps
\[\tau = (\tau_w\colon\fg\to\End_\cA(w))_w,\]
indexed by 1-morphisms $w$ of $\cA$.
We say that $\tau$ is \emph{flat} if for each $w$, we have
\begin{gather*}
  \tau_w([g,h])=g\cdot\tau_w(h)-\bil(g,h)\, h\cdot\tau_w(g),
\end{gather*}

\begin{definition}
  \label{defn:family_of_twists}
  Let $\cA$ be a $\fg$-2-category.
  A family $\tau$ as above is said to be a \emph{a family of twists} if it is flat, satisfies the Leibniz rule and has a graded-com\-mu\-ta\-tive image.
\end{definition}

Here ``satisfies the Leibniz rule'' means that $\tau_{u\otimes v}(g) = \tau_{u}(g)\otimes v + u\otimes\tau_v(g)$ and ``has a graded-com\-mu\-ta\-tive image'' means that the image of each $\tau_w$ is $(\grp,\bil)$-graded-com\-mu\-ta\-tive (see \cref{defn:graded-commutative}).

\begin{remark}
  \label{rem:twist_only_check_on_generators}
  A family of twists is determined by its value on generators of 1-morphisms. Moreover, flatness and graded-com\-mu\-ta\-tive image need only be checked on the generators.
  % \begin{align*}
  %   \tau_{u\otimes v}([g,h]) 
  %   &= \tau_{u}([g,h])\otimes v + u\otimes\tau_v([g,h])\\
  %   &=\big(g\cdot\tau_u(h)-\bil(g,h)\, h\cdot\tau_u(g)\big)\otimes v 
  %   + u\otimes\big(g\cdot\tau_v(h)-\bil(g,h)\, h\cdot\tau_v(g)\big)\\
  %   &= g\cdot\tau_{u\otimes v}(h)-\bil(g,h)\, h\cdot\tau_{u\otimes v}(g)
  % \end{align*}
\end{remark}

\begin{proposition}
  \label{prop:family_of_twists_gives_g2cat}
  Let $\cA$ be a $\fg$-2-category and $\tau$ a family as above.
  For each pair of 1-morphisms $(u,v)$ with the same source and target, define a degree-preserving linear map
  \[\fg\to\uEnd(\Hom_\cA(u,v)),\quad g\mapsto g\cdot_\tau (-)\]
  where for $\alpha\colon u\to v$  a 2-morphism in $\cA$:
  \begin{gather*}
    g\cdot_\tau \alpha \coloneqq \tau_v(g)\circ\alpha + g\cdot \alpha - \bil(g,\alpha)\,\alpha\circ\tau_u(g).
  \end{gather*}
  Let $\cA^\tau$ be the underlying $(\grp,\bil)$-graded-2-category of $\cA$ equipped with this family of maps. If $\tau$ is a family of twists, then $\cA^\tau$ is a $\fg$-2-category.
\end{proposition}

\begin{proof}
  We first check that the action is well-defined; that is, each map $\fg\to\uEnd_\ring(\Hom_\cA(u,v))$ is a $\fg$-morphism.
  For a 2-morphism $\alpha\colon u\to v$, we compute:
  \begin{IEEEeqnarray*}{rCl}
    g\cdot_\tau(h\cdot_\tau \alpha)
    &=&
    g\cdot_\tau(\tau_v(h)\circ \alpha + h\cdot \alpha - \bil(h,\alpha)\;\alpha\circ\tau_u(h))
    \\[1ex]
    &=&
    \tau_v(g)\;(\tau_v(h)\circ\alpha + h\cdot \alpha - \bil(h,\alpha)\;\alpha\circ\tau_u(h))
    \\
    &&{}+
    g\cdot(\tau_v(h)\circ\alpha + h\cdot \alpha - \bil(h,\alpha)\;\alpha\circ\tau_u(h))
    \\
    &&{}-\bil(g,h+\alpha)\;
    (\tau_v(h)\circ\alpha + h\cdot \alpha - \bil(h,\alpha)\;\alpha\circ\tau_u(h))\;
    \tau_u(g)
    \\[1ex]
    &=&
    \tau_v(g)\;(\underline{\tau_v(h)\circ\alpha}_1 + \underline{h\cdot \alpha}_2 - \underline{\bil(h,\alpha)\;\alpha\circ\tau_u(h)}_3)
    \\
    &&{}+
    \underline{(g\cdot\tau_v(h))\circ\alpha}_4 + \underline{\bil(g,h)\;\tau_v(h)\circ(g\cdot \alpha)}_2
    + \underline{g\cdot (h\cdot \alpha)}_5
    \\
    &&{}- \bil(h,\alpha) \big[\underline{(g\cdot \alpha)\circ\tau_u(h)}_6 + \underline{\bil(g,\alpha)\;\alpha\;g\cdot \tau_u(h)}_7\big]
    \\
    &&{}-\bil(g,h+\alpha)\;
    (\underline{\tau_v(h)\;\alpha}_3 + \underline{h\cdot \alpha}_6 - \underline{\bil(h,\alpha)\;\alpha\;\tau_u(h)}_8)\;
    \tau_u(g)
  \end{IEEEeqnarray*}
  Here we labelled each term with a number according to how they simplify in the computation below:
  \begin{IEEEeqnarray*}{rCl}
    g\cdot_\tau(h\cdot_\tau \alpha)-\bil(g,h)h\cdot_\tau(g\cdot_\tau \alpha)
    &=&
    \underline{\tau_v([g,h])\;\alpha}_4 + \underline{[g,h]\cdot \alpha}_5 - \underline{\bil(h+g,\alpha)\;\alpha\;\tau_u([g,h])}_7
    \\
    &=& [g,h]\cdot_\tau \alpha.
  \end{IEEEeqnarray*}
  Terms 4 and 7 simplify thanks to flatness, term 5 simplify as $\cdot$ is an action of $\fg$, and the remaining terms cancel, with terms 1 and 8 cancelling thanks to graded commutativity.


  Following \cref{lem:equivalent_defn_g2cat},
  it remains to check that the $\fg$-action verifies the Leibniz rule and commutes with horizontal whiskering.
  The former follows from graded Leibniz rule for $\cdot$, and the latter follows from the fact that $\tau$ .
\end{proof}

% \begin{remark}
%   It follows directly from the graded Leibniz rule that:
%   \begin{align*}
%     \partial^2(xy)&= \partial\big[\partial(x)y+\bil(\partial,x)x\partial(y)\big]\\
%     &=\partial^2(x)y+\big[\bil(\partial,\partial(x))+\bil(\partial,x)\big]\partial(x)\partial(y)+\bil(\partial,x)^2x\partial^2(y)\\
%     &=\partial^2(x)y+\big[\bil(\partial,\partial)+1\big]\bil(\partial,x)\partial(x)\partial(y)+\bil(\partial,x)^2x\partial^2(y)
%   \end{align*}
%   Assume $\partial^2(x)=\partial^2(y)=0$.
%   On the one hand, if $\bil(\partial,\partial)=-1$ (super case), then $\partial^2(xy)=0$.
%   On the other hand, if $\bil(\partial,\partial)=1$ and if the characteristic of $\ring$ is two ($2$-DG case), then $\partial^2(xy)=0$.
%   In particular, if we are in one of the above cases and $\partial^2$ is zero on a set of generators of $A$, then $(A,\partial)$ is a $(\grp,\bil)$-graded dg-algebra.
% \end{remark}





\subsection{Review of graded \texorpdfstring{$\glt$}{gl2}-foams}
\label{subsec:review_graded_foam}

In this subsection, we review the graded-2-category of $\glt$-foams $\pregfoam_d$ as introduced in \cite{SV_OddKhovanovHomology_2023}, and refer to \textit{op.\ cit.} for further details.

Fix a positive integer $d\in\bN$.
The objects of $\pregfoam_d$ are
\begin{gather*}
  \ob(\pregfoam_d)\coloneqq
  \bigsqcup_{k\in\bN}\{\lambda\in\{1,2\}^k\mid \lambda_1+\ldots+\lambda_k=d\}.
\end{gather*}
For each $\lambda\in\ob(\pregfoam_d)$ with $k$ coordinates, we label its coordinates with
\[l_\lambda\colon\{1,\ldots,k\}\to\{1,\ldots,d\},\]
setting $l_\lambda(i)=\sum_{j<i}\lambda_j+1.$
For instance, $l_{(1,1,2,1)}=(1,2,3,5)$.
In other words, the label $l_\lambda(i)$ is a sort of ``weighted coordinate'', where coordinate with value $2$ counts double.
Foreseeing the diagrammatics, we call this label the \emph{colour} of the coordinate.

The 1-morphisms of $\pregfoam_d$ are \emph{directed $\glt$-webs} (or simply \emph{webs}), such as:
\begin{center}
  \tikzpic{
    \webid[0][1.5][1]\webs[1][1.5]\webm[2][1.5]\webid[3][1.5][1]
    \webm\webid[1][0][1]\webid[2][0][1]\webs[3][0]
  }[xscale=.5,yscale=.4]
\end{center}
In general, a web is obtained from \emph{merge webs} ($M\coloneqq\tikzpic{\webm}[xscale=.5,yscale=.4]$) and \emph{split webs} ($S\coloneqq\tikzpic{\webs}[xscale=.5,yscale=.4]$), by adding single lines ($\tikzpic{\draw[web1] (0,0) to (1,0);}[xscale=.5,yscale=.4]$) and double lines ($\tikzpic{\draw[web2] (0,0) to (1,0);}[xscale=.5,yscale=.4]$) on top and on the bottom and then composing horizontally.
Note that we read webs from right to left.
Our webs are \emph{directed}, in the sense that when reading from right to left, the vertical cross section always has the same width (counting double the double lines); that is, the integer $d$ is fixed.
We sometimes emphasize that point by orienting our webs from right to left.
A web $W$ has an underlying unoriented flat tangle diagram, denoted $\undcomp(W)$, given by forgetting the double lines and the orientation.


We now turn to the 2-morphisms of $\pregfoam_d$.
For convenience, and in contrast to the introduction, we shall use the shading diagrammatics \cite{Schelstraete_OddKhovanovHomology_2024,Schelstraete_RewritingModuloDiagrammatic_2025} throughout the rest of the paper.
It is given by projecting $\glt$-foams onto the plane along the front-to-back direction, and recording only the seams and 2-facets:
\begin{IEEEeqnarray*}{CcC}
  \vcenter{\hbox{\includegraphics[width=2.5cm]{foam/local_model_foams_mini.png}}}
  &
  \mspace{30mu}
  \leftrightarrow
  \mspace{30mu}
  &
  \tikzpic{\frst\node[left] at (0,1) {\footnotesize $i$};}[scale=1.5]
  \\*
  \text{\small $\glt$-foams}
  &&
  \text{\small shading diagrammatics}
\end{IEEEeqnarray*}
Recall the data $\ringfoam$, $\grpfoam$ and $\bilfoam$ from \cref{defn:graded_structure_foam}.

% \ringfoam for ring in foams
% \grpfoam for group in foams
% \bilfoam for bilinear map in foams


\begin{definition}
  \label{defn:review_foam}
  The $(\bZ^2,\bilfoam)$-graded-2-category $\pregfoam_d$ has its $(\bZ^2,\bilfoam)$-graded structure given as in \cref{defn:graded_structure_foam} and is presented with generators given in \cref{fig:string_generators_graded_foams} and relations given in \cref{fig:rel_diagfoam_graded}. 
\end{definition}

The \emph{quantum grading} is defined as $\qdeg(a,b)=a+b$ where $(a,b)$ is the $\grpfoam$-grading.
Although the quantum grading is defined from the $\grpfoam$-grading, we view it as a distinct grading. We denote $\gfoam_d$ the additive $q$-shifted closure of $\pregfoam$; that means we allow formal direct sums and shifts in the quantum grading on objects, and restrict to foams with quantum degree zero (see \cite[subsection~2.1]{SV_OddKhovanovHomology_2023} for details).
Compare to \cite{SV_OddKhovanovHomology_2023}, our notation is such that $\pregfoam_d=\gfoam_d^{\text{\cite{SV_OddKhovanovHomology_2023}}}$ 
and $\gfoam_d=((\underline{\gfoam_d})^\oplus_q)^{\text{\cite{SV_OddKhovanovHomology_2023}}}$.

Recall from \cref{defn:graded_structure_foam} what we mean by ``restrict to odd'' and ``restrict to even''.

\begin{definition}
  \label{defn:foam_and_sfoam}
  We denote $\prefoam_d=\pregfoam_d\vert_{X=Y=Z=1}$ the restriction of $\pregfoam_d$ to even and $\presfoam_d=\pregfoam_d\vert_{X=Z=1,Y=-1}$ the restriction of $\pregfoam_d$ to odd.
  We similarly define $\foam_d$ and $\sfoam_d$.
\end{definition}

This article mainly deals with three graded Lie algebras: the Lie algebra $\fg=\fgl_2^\leq$, the super Lie algebra $\fg=\gloo$, and the graded Lie algebra $\fg=\grgl_2^\leq$.
For each of these cases, we write $\fg\foam_d$ for $\foam_d$, for $\sfoam_d$ and for $\gfoam_d$, respectively.
We shall use similar notations throughout, depending on the choice of $\fg$.

\begin{remark}[monoidal 2-categorical structure]
  \label{rem:monoidal-2-categorical-structure}
  One could gather the graded-2-categories $\pregfoam_d$ together as a certain ``monoidal graded-2-category'', leveraging the canonical graded-2-functors
  \[
    \pregfoam_{d_1}\times\pregfoam_{d_2}\to\pregfoam_{d_1+d_2}
  \]
  given on the pair $(F_1,F_2)$ by putting $F_1$ in front of $F_2$; in shading diagrammatics, it amounts to shifting the labels of $F_2$ and superposing the diagrams. 
  While we avoid making this precise here, certain parts of our discussion implicitly use this extra monoidal structure. We refer to it as the \emph{front-back composition}, and denote it $\square$.
\end{remark}

\begin{remark}
  \label{rem:variant_graded_gl2_foams}
  There exists a variant of $\pregfoam_d$, denoted $\widetilde{\gfoam'_d}$ and with the same generators and relations, except for the following two relations:
  \begin{gather*}
    \tikzpic{
      \funzip[0][1]\fzip\node at (.8+.5,-.3+1) {\scriptsize $i$};
    }[scale=.7]
    =
    XYZ
    \mspace{10mu}
    \tikzpic{
      \fdot\node at (.2,-.2) {\scriptsize $i$};
    }[][(0,0)]
    + Z
    \mspace{10mu}
    \tikzpic{
      \fdot[0][0][2]\node at (.5,-.2) {\scriptsize $i+1$};
    }[][(0,0)]
    \quad\an\quad
    \tikzpic{
      \flst[0][3]\flst[0][2]\flst[0][1]\flst[0][0]
      \frst[.5][3][2]\frst[.5][2][2]\frst[.5][1][2]\frst[.5][0][2]
      \flst[1.5][3][2]\flst[1.5][2][2]\flst[1.5][1][2]\flst[1.5][0][2]
      \frst[2][3]\frst[2][2]\frst[2][1]\frst[2][0]
      \node[below=-2pt] at (0,0) {\scriptsize $i$};
      \node[below=-2pt] at (1.5,0) {\scriptsize $i+1$};
    }[scale=.35][(0,2*.35)]
    %
    \;=\;XYZ^{-1}\;
    %
    \tikzpic{
      \fzip[0][3][2]
      \funzip[0][0][2]
      \begin{scope}[scale=2]
        \fcup[-.25][1]
        \fcap[-.25][0]
      \end{scope}
      \node[below=-2pt] at (-.5,0) {\scriptsize $i$};
      \node[below=-2pt] at (1,0) {\scriptsize $i+1$};
    }[scale=.35][(0,2*.35)]
    \;.
  \end{gather*}
  It was shown in \cite{Schelstraete_RewritingModuloDiagrammatic_2025} that $\pregfoam_d$ and $\widetilde{\gfoam'_d}$ are to only two deformations of $\foam_d$, in a suitable sense.
  When comparing with the classical definition of odd Khovanov homology \cite{ORS_OddKhovanovHomology_2013}, working with $\pregfoam_d$ gives type Y odd Khovanov homology, while working with $\widetilde{\gfoam'_d}$ gives type X odd Khovanov homology.
  See also \cref{subsec:comparison_local_global}.
\end{remark}

\begin{figure}[p]
  \centering

  % SHADING DIAGRAMMATICS GENERATORS
  \begin{equation*}
    \allowdisplaybreaks
    \begin{array}{cc}
      \begin{tabular}{*{5}{c@{\hskip 7ex}}c}
        $\tikzpic{\fdot\node at (.2,-.2) {\scriptsize $i$};}$
        &
        $\tikzpic{\fcup\node at (1,.4) {\scriptsize $i$};}$
        &
        $\tikzpic{\fcap\node at (1,.4) {\scriptsize $i$};}$
        &
        $\tikzpic{\fzip\node at (1,.6) {\scriptsize $i$};}$
        &
        $\tikzpic{\funzip\node at (1,.6) {\scriptsize $i$};}$
        \\*[1ex]
        %
        \small dot &\small cup & \small cap &  \small zip & \small unzip
        \\*[1ex]
        $(1,1)$ & $(0,-1)$ & $(-1,0)$ & $(1,0)$ & $(0,1)$
      \end{tabular}
      &
      %
      \\[7ex]
      %
      \begin{tabular}{*{4}{c@{\hskip 3ex}}c}
        %%%% downward crossing %%%%
        $\tikzpic{
          \fbcro
          \node[left=-3pt] at (0,0) {\scriptsize $i$};
          \node[right=-3pt] at (1,0){\scriptsize $j$};
          % \node at (1.4,0) {\scriptsize $\lambda$};
        }$
        &
        %%%% leftward crossing %%%%
        $\tikzpic{
          \flcro
          \node[left=-3pt] at (0,0) {\scriptsize $i$};
          \node[right=-3pt] at (1,0){\scriptsize $j$};
          % \node at (1.4,0) {\scriptsize $\lambda$};
        }$
        &
        %%%% upward crossing %%%%
        $\tikzpic{
          \ftcro
          \node[left=-3pt] at (0,0) {\scriptsize $i$};
          \node[right=-3pt] at (1,0){\scriptsize $j$};
          % \node at (1.4,0) {\scriptsize $\lambda$};
        }$
        &
        %%%% rightward crossing %%%%
        $\tikzpic{
          \frcro
          \node[left=-3pt] at (0,0) {\scriptsize $i$};
          \node[right=-3pt] at (1,0){\scriptsize $j$};
          % \node at (1.4,0) {\scriptsize $\lambda$};
        }$
        &
        if $\abs{i-j}>1$
        \\*[1ex]
        % \small \stackunder{\text{downward}}{\text{crossing}} & \small \stackunder{\text{rightward}}{\text{crossing}} & \small \stackunder{\text{upward}}{\text{crossing}} & \small \stackunder{\text{leftward}}{\text{crossing}}
        \small \text{downward crossing} & \small \text{leftward crossing} & \small \text{upward crossing} & \small \text{rightward crossing}
        \\*[1ex]
        $(0,0)$ & $(0,0)$ & $(0,0)$ & $(0,0)$
      \end{tabular}
    \end{array}
  \end{equation*}
  \vspace*{-.5cm}

  \caption{Generators in $\pregfoam_d$. Each generator has a grading in $\bZ\times\bZ$.}
  \label{fig:string_generators_graded_foams}

  % SHADING DIAGRAMMATICS RELATIONS
  \def\scl{.7}
  \newcommand{\ins}{0pt}% space between diagram and text
  \newcommand{\outs}{7pt}% space between two (set of) relation(s)
  \begin{gather*}
    %%%%% BRAID-LIKE MOVES %%%%%
    \tikzpic{
      \draw[foamdraw1]  +(0,-.75) node[below] {\textcolor{black}{\scriptsize $i$}}
        .. controls (0,-.375) and (1,-.375) .. (1,0)
        .. controls (1,.375) and (0, .375) .. (0,.75);
      \draw[foamdraw2]  +(1,-.75) node[below] {\textcolor{black}{\scriptsize $j$}}
        .. controls (1,-.375) and (0,-.375) .. (0,0)
        .. controls (0,.375) and (1, .375) .. (1,.75);
    }[scale=.8*\scl][(0,0)]
    \;=\;
    \tikzpic{
      \draw[foamdraw1] (0,-.75) node[below] {\textcolor{black}{\scriptsize $i$}} to (0,.75);
      \draw[foamdraw2] (1,-.75) node[below] {\textcolor{black}{\scriptsize $j$}} to (1,.75);
      % \node at (1.3,0) {\scriptsize $\lambda$};
    }[scale=.8*\scl][(0,0)]
    %
    \mspace{80mu}
    %
    \tikzpic{
      \draw[foamdraw1]  +(0,0)node[below] {\textcolor{black}{\scriptsize $i$}}
        .. controls (0,0.5) and (2, 1) ..  +(2,2);
      \draw[foamdraw3]  +(2,0)node[below] {\textcolor{black}{\scriptsize $k$}}
        .. controls (2,1) and (0, 1.5) ..  +(0,2);
      \draw[foamdraw2]  (1,0)node[below] {\textcolor{black}{\scriptsize $j$}}
        .. controls (1,0.5) and (0, 0.5) ..  (0,1)
        .. controls (0,1.5) and (1, 1.5) ..  (1,2);
      % \node at (2.1,1) {\scriptsize $\lambda$};
    }[scale=.7*\scl][(0,1*.7*\scl)]
    \;=\;
    \tikzpic{
      \draw[foamdraw1]  +(0,0)node[below] {\textcolor{black}{\scriptsize $i$}}
        .. controls (0,1) and (2, 1.5) ..  +(2,2);
      \draw[foamdraw3]  +(2,0)node[below] {\textcolor{black}{\scriptsize $k$}}
        .. controls (2,.5) and (0, 1) ..  +(0,2);
      \draw[foamdraw2]  (1,0)node[below]{\textcolor{black} {\scriptsize $j$}}
        .. controls (1,0.5) and (2, 0.5) ..  (2,1)
        .. controls (2,1.5) and (1, 1.5) ..  (1,2);
      % \node at (2.3,1) {\scriptsize $\lambda$};
    }[scale=.7*\scl][(0,1*.7*\scl)]
    \\[\ins]
    \text{\footnotesize braid-like relations}
    \\[\outs]
    %%%%% PITCHFORKS %%%%%
    \tikzpic{
      \draw[foamdraw1](-.5,.4) to (0,-.3);
      \draw[foamdraw2] (0.3,-0.3)
        to[out=90, in=0] (0,0.2)
        to[out = -180, in = 40] (-0.5,-0.3);
      \node at (-0.5,-.5) {\scriptsize $j$};
      \node at (0,-.5) {\scriptsize $i$};
      % \node at (0.4,0) {\scriptsize $\lambda$};
    }[scale=1.2*\scl][(0,0)]
    \;=\;
    \tikzpic{
      \draw[foamdraw1](.6,.4) to (.1,-.3);
      \draw[foamdraw2] (0.6,-0.3)
        to[out=140, in=0] (0.1,0.2)
        to[out = -180, in = 90] (-0.2,-0.3);
      \node at (-0.2,-.5) {\scriptsize $j$};
      \node at (0.1,-.5) {\scriptsize $i$};
      % \node at (0.7,0) {\scriptsize $\lambda$};
    }[scale=1.2*\scl][(0,0)]
    %
    \mspace{80mu}
    %
    \tikzpic{
      \draw[foamdraw1](-.5,-.3) to (0,.4);
      \draw[foamdraw2] (0.3,0.4)
        to[out=-90, in=0] (0,-0.1)
        to[out = 180, in = -40] (-0.5,0.4);
      \node at (-0.5,.6) {\scriptsize $j$};
      \node at (-0.05,.6) {\scriptsize $i$};
      % \node at (0.5,0.1) {\scriptsize $\lambda$};
    }[scale=1.2*\scl][(0,0)]
    \;=\;
    \tikzpic{
      \draw[foamdraw1](.6,-.3) to (.1,.4);
      \draw[foamdraw2] (0.6,0.4)
        to[out=-140, in=0] (0.1,-0.1)
        to[out = 180, in = -90] (-0.2,0.4);
      \node at (-0.25,.6)  {\scriptsize $j$};
      \node at (0.15,.6) {\scriptsize $i$};
      % \node at (0.7,0.1) {\scriptsize $\lambda$};
    }[scale=1.2*\scl][(0,0)]
    \\[\ins]
    \text{\footnotesize pitchfork relations}
    \\[\outs]
    %%%%% ZIGZAGS %%%%%
    \begingroup
      \tikzpic{
        \fcap[0][1]\flst[2][1]\flst\fzip[1][0]
        \node[left=-3pt] at (2,2) {\scriptsize $i$};
      }[scale=.5][(0,1*.5)]
      =
      \tikzpic{
        \flst[0][1]\flst\node[left=-3pt] at (0,2) {\scriptsize $i$};
      }[scale=.5][(0,1*.5)]
      %
      \mspace{30mu}
      %
      \tikzpic{
        \frst[0][1]\fcap[1][1]\fzip[0][0]\frst[2][0]
        \node[left=-3pt] at (0,2) {\scriptsize $i$};
      }[scale=.5][(0,1*.5)]
      = X
      \tikzpic{
        \frst[0][1]\frst\node[left=-3pt] at (0,2) {\scriptsize $i$};
      }[scale=.5][(0,1*.5)]
      %
      \mspace{30mu}
      %
      \tikzpic{
        \funzip[0][1]\frst[2][1]\frst\fcup[1][0]
        \node[left=-3pt] at (2,2) {\scriptsize $i$};
      }[scale=.5][(0,1*.5)]
      = Z^2
      \tikzpic{
        \frst[0][1]\frst\node[left=-3pt] at (0,2) {\scriptsize $i$};
      }[scale=.5][(0,1*.5)]
      %
      \mspace{30mu}
      %
      \tikzpic{
        \flst[0][1]\funzip[1][1]\fcup[0][0]\flst[2][0]
        \node[left=-3pt] at (0,2) {\scriptsize $i$};
      }[scale=.5][(0,1*.5)]
      =
      YZ^2\;
      \tikzpic{
        \flst[0][1]\flst\node[left=-3pt] at (0,2) {\scriptsize $i$};
      }[scale=.5][(0,1*.5)]
    \endgroup
    \\[\ins]
    \text{\footnotesize zigzag relations (or adjunction relations)}
    \\[\outs]
    %%%%% DOT-TYPE RELATIONS %%%%%
    \begin{IEEEeqnarraybox}{cCcCc}
      %%%%% DOT ANNIHILATION %%%%%
      \left(\tikzpic{
        \fdot\node at (.2,-.2) {\scriptsize $i$};
      }[][(0,0)]\right)^2
      =
      0
      &\mspace{50mu}&
      %%%%% DOT SWAP %%%%%
      \tikzpic{
        \fdot[-1][1]\node at (-.7,.8) {\scriptsize $i$};
        \frst[0][1]\frst\node[left=-3pt] at (0,2) {\scriptsize $i$};
      }[scale=.5][(0,1*.5)]
      =\;
      \tikzpic{
        \fdot[-1][1][2]\node at (-.7,.5) {\scriptsize $i+1$};
        \frst[0][1]\frst\node[left=-3pt] at (0,2) {\scriptsize $i$};
      }[scale=.5][(0,1*.5)]
      &\mspace{50mu}&
      %%%%% DOT SLIDE %%%%%
      \tikzpic{
        \fdot[-1][1][2]\node at (-.7,.8) {\scriptsize $j$};
        \frst[0][1]\frst\node[left=-3pt] at (0,2) {\scriptsize $i$};
      }[scale=.5][(0,1*.5)]
      \;=\;
      \tikzpic{
        \fdot[1][1][2]\node at (-.7+2,.8) {\scriptsize $j$};
        \frst[0][1]\frst\node[left=-3pt] at (0,2) {\scriptsize $i$};
      }[scale=.5][(0,1*.5)]
      %
      \mspace{20mu}
      %
      \text{if }j\neq i,i+1
      \\[\ins]
      \text{\footnotesize dot annihilation}
      &&
      \text{\footnotesize dot migration}
      &&
      \text{\footnotesize dot slide}
    \end{IEEEeqnarraybox}
    \\[\outs]
    %%%%% BUBBLES %%%%%
    \begin{IEEEeqnarraybox}{cCc}
      %%%%% COUNTER-CLOCKWISE BUBBLES %%%%%
      \tikzpic{
        \fdot[.5][1]\node at (.2+.5,-.2+1) {\scriptsize $i$};
        \fcap[0][1]\fcup\node at (.8+.5,-.3+1) {\scriptsize $i$};
      }[scale=.7]
      =
      1
      \mspace{40mu}
      \tikzpic{
        \fcap[0][1]\fcup\node at (.8+.5,-.3+1) {\scriptsize $i$};
      }[scale=.7]
      =
      0
      & \mspace{70mu} &
      %%%%% CLOCKWISE BUBBLES %%%%%
      \begingroup
        \tikzpic{
          \funzip[0][1]\fzip\node at (.8+.5,-.3+1) {\scriptsize $i$};
        }[scale=.7]
        =
        {Z}
        \mspace{10mu}
        \tikzpic{
          \fdot\node at (.2,-.2) {\scriptsize $i$};
        }[][(0,0)]
        + XYZ
        \mspace{10mu}
        \tikzpic{
          \fdot[0][0][2]\node at (.5,-.2) {\scriptsize $i+1$};
        }[][(0,0)]
      \endgroup
      \\[\ins]
      \text{\footnotesize evaluation of bubbles}
      &&
      \text{\footnotesize evaluation of shaded disks}
    \end{IEEEeqnarraybox}
    \\[\outs]
    %%%%% SHADING RELATED RELATION %%%%%
    \begin{IEEEeqnarraybox}{cCc}
      %%%%% NECK-CUTTING RELATION %%%%%
      \begingroup
        \tikzpic{
          \flst[0][1]\flst\node[left=-3pt] at (0,2) {\scriptsize $i$};
          \frst[1][1]\frst[1][0];
        }[scale=.7][(0,1*.7)]
        %
        \;=\;
        %
        \tikzpic{
          \fdot[.5][1.8]\fcup[0][1]\fcap
          \node[left=-3pt] at (0,2) {\scriptsize $i$};
        }[scale=.7][(0,1*.7)]
        \;+\;
        \tikzpic{
          \fdot[.5][.2]\fcup[0][1]\fcap
          \node[left=-3pt] at (0,2) {\scriptsize $i$};
        }[scale=.7][(0,1*.7)]
      \endgroup
      & \mspace{120mu} &
      %%%%% ISOTOPY RELATION %%%%%
      \begingroup
      \tikzpic{
          \flst[0][3]\flst[0][2]\flst[0][1]\flst[0][0]
          \frst[.5][3][2]\frst[.5][2][2]\frst[.5][1][2]\frst[.5][0][2]
          \flst[1.5][3][2]\flst[1.5][2][2]\flst[1.5][1][2]\flst[1.5][0][2]
          \frst[2][3]\frst[2][2]\frst[2][1]\frst[2][0]
          \node[below=-2pt] at (0,0) {\scriptsize $i$};
          \node[below=-2pt] at (1.5,0) {\scriptsize $i+1$};
        }[scale=.35][(0,2*.35)]
        %
        \;=\;Z^{-1}\;
        %
        \tikzpic{
          \fzip[0][3][2]
          \funzip[0][0][2]
          \begin{scope}[scale=2]
            \fcup[-.25][1]
            \fcap[-.25][0]
          \end{scope}
          \node[below=-2pt] at (-.5,0) {\scriptsize $i$};
          \node[below=-2pt] at (1,0) {\scriptsize $i+1$};
        }[scale=.35][(0,2*.35)]
      \endgroup
      \\[\ins]
      \text{\footnotesize neck-cutting relation}
      &&
      \text{\footnotesize squeezing relation}
    \end{IEEEeqnarraybox}
  \end{gather*}
  \vspace*{-.5cm}

  \caption{Relations in $\pregfoam_d$. We omit the objects labelling the regions of each diagram: this avoids clutter and emphasizes that relations are independent of the ambient object. If no shading is given, the relation holds for all shadings. In the case of the braid-like and pitchfork relations, colours should be so that the crossings exist.}
  \label{fig:rel_diagfoam_graded}
\end{figure}

\subsection{Generic derivations and actions}
\label{subsec:generic_derivation_action}

In this subsection, we define derivations on the graded-2-category $\pregfoam_d$ of graded $\glt$-foams generically, depending on a family of parameters. We then give minimal conditions so that these derivations gather into an action of $\grsl_{2}^{\leq}$ by derivation on $\pregfoam_d$.
We do the same analysis when restricting to the odd case $\presfoam_d$, extending to an action of $\gloo$.


\subsubsection{Graded case}
\label{subsubsec:generic_derivation_action_graded}

\begin{lemma}
  \label{lem:generic_derivations_on_graded_foam}
  Let $\varf\coloneqq\{\varf^i\}_{1\leq i\leq d-1}$, $\delh$ and $\varh\coloneqq\{\varh^i\}_{1\leq i\leq d-1}$ be scalars in $\ringfoam$.
  The graded-2-category $\pregfoam_d$ admits the following graded derivations $\sff_{\varf}$ and $\sfh_{\delh,\varh}$, of degree $(1,1)$ and $(0,0)$ respectively, and defined on the generators (\cref{fig:string_generators_graded_foams}) as zero on crossings and as:
  \begin{center}
    \def\spc{2ex}
    \begin{tabular}{@{}l@{\hskip 5ex}*{4}{r@{\hskip 4ex}}r@{}}
      &
      $\tikzpic{\fdot\node at (.2,-.2) {\scriptsize $i$};}[scale=.5]$
      &
      $\tikzpic{\fcup\node at (1,.4) {\scriptsize $i$};}[scale=.5]$
      &
      $\tikzpic{\fzip\node at (1,.4) {\scriptsize $i$};}[scale=.5]$
      &
      $\tikzpic{\fcap\node at (1,.6) {\scriptsize $i$};}[scale=.5]$
      &
      $\tikzpic{\funzip\node at (1,.6) {\scriptsize $i$};}[scale=.5]$
      \\*[2ex]
      \midrule
      %%%%%%%%%%%%%%%%%%%%%%%%%%%%%%
      $\sff_{\varf}$
      &
      $0$
      &
      $\varf^i\;\tikzpic{\fcup\fdot[.5][.8]\node at (1,.4) {\scriptsize $i$};}[scale=.5]$
      &
      $\varf^i\;\tikzpic{\fzip\fdot[-.3][.7]\node at (1,.4) {\scriptsize $i$};}[scale=.5]$
      &
      $-\varf^iXZ\;\tikzpic{\fcap\fdot[.5][.2]\node at (1,.6) {\scriptsize $i$};}[scale=.5]$
      &
      $-\varf^iYZ\;\tikzpic{\funzip\fdot[-.3][.3]\node at (1,.6) {\scriptsize $i$};}[scale=.5]$
      \\*[\spc]
      %%%%%%%%%%%%%%%%%%%%%%%%%%%%%%
      % $\lieh_1$
      % &
      % $-\;\tikzpic{\fdot\node at (.2,-.2) {\scriptsize $i$};}[scale=.5]$
      % &
      % $\tikzpic{\fcup\node at (1,.4) {\scriptsize $i$};}[scale=.5]$
      % &
      % $0$
      % &
      % $0$
      % &
      % $-\;\tikzpic{\funzip\node at (1,.6) {\scriptsize $i$};}[scale=.5]$
      % \\*[\spc]
      % %%%%%%%%%%%%%%%%%%%%%%%%%%%%%%
      % $\lieh_2$
      % &
      % $\tikzpic{\fdot\node at (.2,-.2) {\scriptsize $i$};}[scale=.5]$
      % &
      % $0$
      % &
      % $\tikzpic{\fzip\node at (1,.4) {\scriptsize $i$};}[scale=.5]$
      % &
      % $-\;\tikzpic{\fcap\node at (1,.6) {\scriptsize $i$};}[scale=.5]$
      % &
      % $0$
      % \\*[\spc]
      % \cmidrule(r){2-6}
      % %%%%%%%%%%%%%%%%%%%%%%%%%%%%%%
      $\sfh_{\delh,\varh}$
      &
      $-\delh\;\tikzpic{\fdot\node at (.2,-.2) {\scriptsize $i$};}[scale=.5]$
      &
      $(\delh-\varh^i)\;\tikzpic{\fcup\node at (1,.4) {\scriptsize $i$};}[scale=.5]$
      &
      $-\varh^i\;\tikzpic{\fzip\node at (1,.4) {\scriptsize $i$};}[scale=.5]$
      &
      $\varh^i\;\tikzpic{\fcap\node at (1,.6) {\scriptsize $i$};}[scale=.5]$
      &
      $-(\delh-\varh^i)\;\tikzpic{\funzip\node at (1,.6) {\scriptsize $i$};}[scale=.5]$
    \end{tabular}
  \end{center}
  % \begin{gather*}
  %   \sff\left(\xy(0,0)*{\begin{tikzpicture}
  %     \node[fdot1] at (0,0){};
  %   \end{tikzpicture}}\endxy\right)
  %   =0
  %   \qquad
  %   \sff\left(\xy(0,0)*{\begin{tikzpicture}[scale=.7]
  %     \pic[transform shape] at (0,0){rcup=foamdraw1};
  %   \end{tikzpicture}}\endxy\right)
  %   =
  %   \varf^i\;
  %   \xy(0,0)*{\begin{tikzpicture}[scale=.7]
  %     \pic[transform shape] at (0,0){rcup=foamdraw1};
  %     \node[fdot1] at (.5,.7){};
  %   \end{tikzpicture}}\endxy
  %   \qquad
  %   \sff\left(\xy(0,0)*{\begin{tikzpicture}[scale=.7]
  %     \pic[transform shape] at (0,0){lcup=foamdraw1};
  %   \end{tikzpicture}}\endxy\right)
  %   =
  %   \varf^i\;
  %   \xy(0,0)*{\begin{tikzpicture}[scale=.7]
  %     \pic[transform shape] at (0,0){lcup=foamdraw1};
  %     \node[fdot1] at (-.5,.7){};
  %   \end{tikzpicture}}\endxy
  %   \\[1ex]
  %   \sff\left(\xy(0,0)*{\begin{tikzpicture}[scale=.7]
  %     \pic[transform shape] at (0,0){lcap=foamdraw1};
  %   \end{tikzpicture}}\endxy\right)
  %   =
  %   -\varf^iXZ\;
  %   \xy(0,0)*{\begin{tikzpicture}[scale=.7]
  %     \pic[transform shape] at (0,0){lcap=foamdraw1};
  %     \node[fdot1] at (.5,.3){};
  %   \end{tikzpicture}}\endxy
  %   \qquad
  %   \sff\left(\xy(0,0)*{\begin{tikzpicture}[scale=.7]
  %     \pic[transform shape] at (0,0){rcap=foamdraw1};
  %   \end{tikzpicture}}\endxy\right)
  %   =
  %   -\varf^iYZ\;
  %   \xy(0,0)*{\begin{tikzpicture}[scale=.7]
  %     \pic[transform shape] at (0,0){rcap=foamdraw1};
  %     \node[fdot1] at (-.5,.3){};
  %   \end{tikzpicture}}\endxy
  %   %%%%%%%%%%%%%%%%%%%%%%%%%%%%%%
  %   \\[4ex]
  %   %%%%%%%%%%%%%%%%%%%%%%%%%%%%%%
  %   \sfh\left(\xy(0,0)*{\begin{tikzpicture}
  %     \node[fdot1] at (0,0){};
  %   \end{tikzpicture}}\endxy\right)
  %   =
  %   -\delh\;
  %   \xy(0,0)*{\begin{tikzpicture}
  %     \node[fdot1] at (0,0){};
  %   \end{tikzpicture}}\endxy
  %   \qquad
  %   \sfh\left(\xy(0,0)*{\begin{tikzpicture}[scale=.7]
  %     \pic[transform shape] at (0,0){rcup=foamdraw1};
  %   \end{tikzpicture}}\endxy\right)
  %   =
  %   (\delh-\varh^i)\;
  %   \xy(0,0)*{\begin{tikzpicture}[scale=.7]
  %     \pic[transform shape] at (0,0){rcup=foamdraw1};
  %   \end{tikzpicture}}\endxy
  %   \qquad
  %   \sfh\left(\xy(0,0)*{\begin{tikzpicture}[scale=.7]
  %     \pic[transform shape] at (0,0){lcup=foamdraw1};
  %   \end{tikzpicture}}\endxy\right)
  %   =
  %   -\varh^i\;
  %   \xy(0,0)*{\begin{tikzpicture}[scale=.7]
  %     \pic[transform shape] at (0,0){lcup=foamdraw1};
  %   \end{tikzpicture}}\endxy
  %   \\[1ex]
  %   \sfh
  %   \left(\xy(0,0)*{\begin{tikzpicture}[scale=.7]
  %     \pic[transform shape] at (0,0){lcap=foamdraw1};
  %   \end{tikzpicture}}\endxy\right)
  %   =
  %   \varh^i\;
  %   \xy(0,0)*{\begin{tikzpicture}[scale=.7]
  %     \pic[transform shape] at (0,0){lcap=foamdraw1};
  %   \end{tikzpicture}}\endxy
  %   \qquad
  %   \sfh
  %   \left(\xy(0,0)*{\begin{tikzpicture}[scale=.7]
  %     \pic[transform shape] at (0,0){rcap=foamdraw1};
  %   \end{tikzpicture}}\endxy\right)
  %   =
  %   -(\delh-\varh^i)\;
  %   \xy(0,0)*{\begin{tikzpicture}[scale=.7]
  %     \pic[transform shape] at (0,0){rcap=foamdraw1};
  %   \end{tikzpicture}}\endxy
  % \end{gather*}
\end{lemma}

\begin{proof}
  We show that $\sff_{\varf}$ is well-defined.
  Thanks to \cref{rem:derivation_uniquely_defn_on_generators_of_A}, it suffices to check it locally on the defining relations (\cref{fig:rel_diagfoam_graded}).
  It is straightforward for braid-like relations, pitchfork relations, dot annihilation, dot migration, dot slide, and evaluation of dotted bubbles.
  For the other evaluations, we have:
  \def\scl{.5}
  \begin{IEEEeqnarray*}{rCl}
    %%%%% COUNTER-CLOCKWISE BUBBLES %%%%%
    \sff_{\varf}\left(
      \tikzpic{
        \fcap[0][1]\fcup\node at (.8+.5,-.3+1) {\scriptsize $i$};
      }[scale=\scl]
    \right)
    &=&
    -\varf^iXZ \;
    \tikzpic{
        \fdot[.5][1]\node at (.2+.5,-.2+1) {\scriptsize $i$};
        \fcap[0][1]\fcup\node at (.8+.5,-.3+1) {\scriptsize $i$};
      }[scale=\scl]
    +\bilfoam\big((1,1),(-1,0)\big)
    \varf^i\;
    \tikzpic{
        \fdot[.5][1]\node at (.2+.5,-.2+1) {\scriptsize $i$};
        \fcap[0][1]\fcup\node at (.8+.5,-.3+1) {\scriptsize $i$};
      }[scale=\scl]
    =
    0
    \\[2ex]
    %%%%% CLOCKWISE BUBBLES %%%%%
    \sff_{\varf}\left(
      \tikzpic{
        \funzip[0][1]\fzip\node at (.8+.5,-.3+1) {\scriptsize $i$};
      }[scale=\scl]
    \right)
    &=&
    -\varf^iYZ
    \mspace{10mu}
    \tikzpic{
        \fdot[-.5][1]\node at (.2-.5,-.2+1) {\scriptsize $i$};
        \funzip[0][1]\fzip\node at (.8+.5,-.3+1) {\scriptsize $i$};
      }[scale=\scl]
    +\bilfoam\big((1,1),(0,1)\big)
    \varf^i
    \mspace{10mu}
    \tikzpic{
        \fdot[-.5][1]\node at (.2-.5,-.2+1) {\scriptsize $i$};
        \funzip[0][1]\fzip\node at (.8+.5,-.3+1) {\scriptsize $i$};
      }[scale=\scl]
    =0
  \end{IEEEeqnarray*}
  The neck-cutting gives:
  \begin{gather*}
    \sff_{\varf}\left(
      \tikzpic{
        \fdot[.5][1.8]\fcup[0][1]\fcap
        \node[left=-3pt] at (0,2) {\scriptsize $i$};
      }[scale=\scl][(0,1*\scl)]
      \;+\;
      \tikzpic{
        \fdot[.5][.2]\fcup[0][1]\fcap
        \node[left=-3pt] at (0,2) {\scriptsize $i$};
      }[scale=\scl][(0,1*\scl)]
    \right)
    =
    \Big[-\varf^iXZ \bilfoam\big((1,1),(1,0)\big)+\varf^i\Big]\;
    \tikzpic{
        \fdot[.5][.2]
        \fdot[.5][1.8]\fcup[0][1]\fcap
        \node[left=-3pt] at (0,2) {\scriptsize $i$};
      }[scale=\scl][(0,1*\scl)]
    =0
  \end{gather*}
  Finally, the squeezing relation gives:
  \begin{IEEEeqnarray*}{rCl}
    \sff_{\varf}\left(
      \tikzpic{
        \fzip[0][3][2]
        \funzip[0][0][2]
        \begin{scope}[scale=2]
          \fcup[-.25][1]
          \fcap[-.25][0]
        \end{scope}
        \node[below=-2pt] at (-.5,0) {\scriptsize $i$};
        \node[below=-2pt] at (1,0) {\scriptsize $i+1$};
      }[scale=.35][(0,2*.35)]
    \right)
    &=&
    \varf^{i+1}
    \tikzpic{
      \fzip[0][3][2]
      \funzip[0][0][2]
      \begin{scope}[scale=2]
        \fcup[-.25][1]
        \fcap[-.25][0]
      \end{scope}
      \node[below=-2pt] at (-.5,0) {\scriptsize $i$};
      \node[below=-2pt] at (1,0) {\scriptsize $i+1$};
      %
      \fdot[-.2][3.7][2][1.5]
    }[scale=.35][(0,2*.35)]
    +
    \varf^i
    \bilfoam\big((1,1),(1,0)\big)
    \tikzpic{
      \fzip[0][3][2]
      \funzip[0][0][2]
      \begin{scope}[scale=2]
        \fcup[-.25][1]
        \fcap[-.25][0]
      \end{scope}
      \node[below=-2pt] at (-.5,0) {\scriptsize $i$};
      \node[below=-2pt] at (1,0) {\scriptsize $i+1$};
      %
      \fdot[1-.5][3][1][1.5]
    }[scale=.35][(0,2*.35)]
    \\
    &&{}-\varf^iXZ
    \bilfoam\big((1,1),(1,-1)\big)
    \tikzpic{
      \fzip[0][3][2]
      \funzip[0][0][2]
      \begin{scope}[scale=2]
        \fcup[-.25][1]
        \fcap[-.25][0]
      \end{scope}
      \node[below=-2pt] at (-.5,0) {\scriptsize $i$};
      \node[below=-2pt] at (1,0) {\scriptsize $i+1$};
      %
      \fdot[1-.5][1][1][1.5]
    }[scale=.35][(0,2*.35)]
    -\varf^{i+1}YZ
    \bilfoam\big((1,1),(0,-1)\big)
    \tikzpic{
      \fzip[0][3][2]
      \funzip[0][0][2]
      \begin{scope}[scale=2]
        \fcup[-.25][1]
        \fcap[-.25][0]
      \end{scope}
      \node[below=-2pt] at (-.5,0) {\scriptsize $i$};
      \node[below=-2pt] at (1,0) {\scriptsize $i+1$};
      %
      \fdot[-.2][.3][2][1.5]
    }[scale=.35][(0,2*.35)]
    \\[1ex]
    &=&
    \Big[\varf^{i+1}\normafoam{+}{-}\varf^i\normafoam{-}{+}\varf^i-\varf^{i+1}\Big]
    \tikzpic{
      \fzip[0][3][2]
      \funzip[0][0][2]
      \begin{scope}[scale=2]
        \fcup[-.25][1]
        \fcap[-.25][0]
      \end{scope}
      \node[below=-2pt] at (-.5,0) {\scriptsize $i$};
      \node[below=-2pt] at (1,0) {\scriptsize $i+1$};
      %
      \fdot[-.2][3.7][2][1.5]
    }[scale=.35][(0,2*.35)]
    =0
  \end{IEEEeqnarray*}

  We show that $\sfh_{\delh,\varh}$ is well-defined.
  Given that $\sfh_{\delh,\varh}$ has trivial grading and it acts on each generator by multiplication with a certain scalar, its action on a generic diagram amounts to multiplying this diagram with the sum of the scalars associated to each of its generators.
  With this remark, braid-like relations, pitchfork relations, dot annihilation, dot slide and evaluation of undotted bubbles are straightforward, and do not depend on the choice of scalars.
  %
  Zigzag relations force the scalars associated to the cup and unzip (resp.\ the cap and zip) to be opposite of one another.
  %
  Dot migration forces the scalar associated to the dot to be independent of $i$.
  %
  Neck-cutting imposes a linear relation between the scalars associated to the dot, the cup and the cap.
  %
  All the conditions above lead to the choice of scalars given in the lemma.
  %
  One check compatibility with the remaining relations similarly (squeezing, evaluation of dotted bubbles and evaluation of shaded disks).

  This concludes.
\end{proof}

\begin{remark}[unicity of $\sff$ and $\sfh$]
  \label{rem:unicity_generic_graded}
  Recall the front-back composition from \cref{rem:monoidal-2-categorical-structure}.
  It is natural to ask for derivations to satisfy a Leibniz rule with respect to this composition as well. If so, then each derivation of degree $(1,1)$ is of the form $\sff_{\varf}$, where moreover all variables $\varf^i$ are equal.
  % in principle, the action of f could have "far-away" dots
  Similarly, in this case each derivation of degree $(0,0)$ is of the form $\sfh_{\delh,\varh}$, where moreover all variables $\varh^i$ are equal.
\end{remark}

\begin{lemma}
  \label{lem:commutator_graded_case}
  The commutators of the derivations $\sff_{\varf}$ and $\sfh_{\delh,\varh}$ defined in \cref{lem:generic_derivations_on_graded_foam} are
  \[[\sfh_{\delh,\varh},\sff_{\varf}]=-\delh\sff_{\varf}
  \quad\an\quad
  [\sff_{\varf},\sff_{\varf}]=[\sfh_{\delh,\varh},\sfh_{\delh',\varh'}]=0\]
  for any choice of (family of) parameters $\varf$, $(\delh,\varh)$ and $(\delh',\varh')$.
\end{lemma}

\begin{proof}
  Thanks to \cref{rem:derivation_uniquely_defn_on_generators_of_A}, it suffices to check the equalities on generators.
  Checking the claimed equalities amounts to straightforward computation.
  We give another argument for the relation $[\sfh_{\delh,\varh},\sff_{\varf}]=-\delh\sff_{\varf}$.
  Recall from the previous proof that $\sfh_{\delh,\varh}$ acts by multiplying a diagram by the sum of scalars associated to its generators; in particular, for any generator $D$, $\sfh_{\delh,\varh}$ acts by a certain scalar $\lambda_D$.
  On the other hand, the action of $\sff_{\varf}$ on the generator $D$ ``adds a dot'', up to scalar. It follows that 
  \[[\sfh_{\delh,\varh},\sff_{\varf}](D)=\sfh_{\delh,\varh}\sff_{\varf}(D)-\sff_{\varf}\sfh_{\delh,\varh}(D)
  =(\lambda_D-\delh)\sff(D)-\lambda_D\sff(D)=-\delh\sff(D).\]
  This concludes.
\end{proof}

With the help of \cref{rem:action_uniquely_defn_by_generators_of_g}, it follows that:

\begin{corollary}
  \label{cor:generic_action_on_graded_foam}
  For any choice of parameters $\varf$, $\varh$ and $\varh'$ as in \cref{lem:generic_derivations_on_graded_foam},
  The application
  \[\{\lief\mapsto \sff_{\varf},\lieh_1\mapsto\sfh_{1,\varh},\lieh_2\mapsto\sfh_{-1,\varh'}\}\]
  % (resp.\ the application $\{\liee\mapsto \sff_{\varf},\lieh_1\mapsto\sfh_{-1,\varh},\lieh_2\mapsto\sfh_{1,\varh'}\}$)
  defines an action of $\grgl_{2}^{\leq}$
  % (resp.\ $\grgl_{2}^{\geq}$)
  by derivation on the graded-2-category $\pregfoam_d$.
  \hfill\qed
\end{corollary}

We pick a standard choice of action:

\begin{definition}
  \label{defn:action_on_foam_graded_case}
  We view $\pregfoam_d$ as a $\grgl_{2}^{\leq}$-2-category with the action of $\grgl_{2}^{\leq}$ by derivation given in \cref{defn:action_on_foam_super_case} (ignoring the action of $\liee$).
\end{definition}


\subsubsection{Super case}
\label{subsubsec:generic_derivation_action_super}

\begin{lemma}
  \label{lem:generic_derivation_on_super_foam}
  Let $\vare\in\ringfoam$ be a choice of parameter.
  The super-2-category $\presfoam_d$ admits the derivation $\sfe_{\vare}$, defined on the generators (\cref{fig:string_generators_graded_foams}) as zero on crossings and as:
  \begin{center}
    \def\spc{2ex}
    \begin{tabular}{@{}l@{\hskip 5ex}*{4}{r@{\hskip 4ex}}r@{}}
      &
      $\tikzpic{\fdot\node at (.2,-.2) {\scriptsize $i$};}[scale=.5]$
      &
      $\tikzpic{\fcup\node at (1,.4) {\scriptsize $i$};}[scale=.5]$
      &
      $\tikzpic{\fzip\node at (1,.4) {\scriptsize $i$};}[scale=.5]$
      &
      $\tikzpic{\fcap\node at (1,.6) {\scriptsize $i$};}[scale=.5]$
      &
      $\tikzpic{\funzip\node at (1,.6) {\scriptsize $i$};}[scale=.5]$
      \\*[2ex]
      \midrule
      %%%%%%%%%%%%%%%%%%%%%%%%%%%%%%
      $\sfe_{\vare}$
      &
      $\normafoam{\vare}{(-1)^i\vare}\id_\emptyset$
      &
      $0$
      &
      $0$
      &
      $0$
      &
      $0$
    \end{tabular}
  \end{center}
\end{lemma}

\begin{proof}
  \def\scl{.5}
  It suffices to check that $\sfe_{\vare}$ is compatible with the defining relations (\cref{fig:rel_diagfoam_graded}).
  Relations that do not involve dots are straightforward.
  Compatibility with dot annihilation and neck-cutting relation essentially follows from the fact that $\sfe_{\vare}$ is a super derivation:
  \begin{gather*}
    \sfe_{\vare}\left(\;
    \tikzpic{
      \fdot[0][1]\fdot\node at (.2,-.2) {\scriptsize $i$};
    }[scale=\scl]
    \right)
    =
    \vare[1-1]\;\tikzpic{\fdot}
    =0
    \mspace{10mu}\an\mspace{10mu}
    \sfe_{\vare}\left(
      \tikzpic{
        \fdot[.5][1.8]\fcup[0][1]\fcap
        \node[left=-3pt] at (0,2) {\scriptsize $i$};
      }[scale=\scl][(.5,1*\scl)]
      \;+\;
      \tikzpic{
        \fdot[.5][.2]\fcup[0][1]\fcap
        \node[left=-3pt] at (0,2) {\scriptsize $i$};
      }[scale=\scl][(.5,1*\scl)]
    \right)
    =
    \vare[1-1]\;
    \tikzpic{
        \fcup[0][1]\fcap
        \node[left=-3pt] at (0,2) {\scriptsize $i$};
      }[scale=\scl][(.5,1*\scl)]
    =0.
  \end{gather*}
  This explains why $\sfe_{\vare}$ can only be defined in the super case.
  Compatibility with dot migration and evaluation of shaded disks follows from the fact that $\vare$ does not depend on $i$.
  % \begin{gather*}
  %   \sfe\left(\;
  %   \tikzpic{
  %     \fdot\node at (.2,-.2) {\scriptsize $i$};
  %   }
  %   % \mspace{10mu}
  %   -
  %   \mspace{10mu}
  %   \tikzpic{
  %     \fdot[0][0][2]\node at (.5,-.3) {\scriptsize $i+1$};
  %   }
  %   \right)
  %   =0
  % \end{gather*}
  Compatibility with dot slide and evaluation of bubbles is straightforward. This concludes.
\end{proof}

\begin{lemma}
  \label{lem:commutator_super_case}
  The (super) commutators of the super derivation $\sfe_{\vare}$ (\cref{lem:generic_derivation_on_super_foam}) with the (super) derivations $\sff_{\varf}$ and $\sfh_{\delh,\varh}$ (\cref{lem:generic_derivations_on_graded_foam}; restricted to the super case) are
  \begin{gather*}
    [\sfe_{\vare},\sff_{\varf}] = \sfh_{0,(-\normafoam{}{(-1)^i}\vare\varf^i)_i},
    \qquad
    [\sfh_{\delh,\varh},\sfe_{\vare}]=\delh\sfe_{\vare}
    \qquad\an\qquad
    [\sfe_{\vare},\sfe_{\vare}]=0,
  \end{gather*}
  for any choice of (family of) parameters $\varf$, $(\delh,\varh)$ and $\vare$.
\end{lemma}

\begin{proof}
  Thanks to \cref{rem:derivation_uniquely_defn_on_generators_of_A}, it suffices to check the equalities on generators.
  Checking $[\sfe_{\vare},\sfe_{\vare}]=0$ is straightforward, and the case of the commutator $[\sfh_{\delh,\varh},\sfe_{\vare}]$ follows from the equality
  \[[\sfh_{\delh,\varh},\sfe_{\vare}]=-\sfe_{\vare}\sfh_{\delh,\varh}.\]
  %
  The equality $[\sfe_{\vare},\sff_{\varf}](\tikzpic{\fdot})=0$ is straightforward. For $D$ one of the remaining generators, we have $[\sfe_{\vare},\sff_{\varf}](D) = \sfe_{\vare}\sff_{\varf}(D)$, leading to the remaining equality.
\end{proof}

\begin{corollary}
  \label{cor:generic_action_on_super_foam}
  Let $\varf$, $\varh$, $\varh'$ and $\vare$ be choice of (family of) scalars in $\ringfoam$ as in \cref{lem:generic_derivations_on_graded_foam,lem:generic_derivation_on_super_foam}.
  If
  \[\varh^i+\varh^{'i} = \normafoam{-\vare\varf^i}{-(-1)^i\vare\varf^i}\quad\text{ for all }1\leq i\leq d-1,\]
  then the application
  \[\{\lief\mapsto \sff_{\varf},\lieh_1\mapsto\sfh_{1,\varh},\lieh_2\mapsto\sfh_{-1,\varh'},\liee\mapsto \sfe_{\vare}\}\]
  defines an action of $\gloo$ by derivation on $\presfoam_d$.
\end{corollary}

\begin{proof}
  This follows from \cref{cor:generic_action_on_graded_foam} and \cref{lem:commutator_super_case} (with the help of \cref{rem:action_uniquely_defn_by_generators_of_g}), using that
  $\sfh_{1,\varh}+\sfh_{-1,\varh'}=\sfh_{0,\varh+\varh'}$.
\end{proof}

We pick a standard choice of action, corresponding to the choice $\varf^i=\vare=1$, $\varh^i=0$ and $\varh^{'i}=-1$:

\begin{definition}
  \label{defn:action_on_foam_super_case}
  We view $\presfoam_d$ as a $\gloo$-2-category with the action of $\gloo$ by derivation given by:
  \begin{gather}
    \label{eq:simplest_derivation}
    \def\spc{2ex}
    \def\scl{.5}
    \begin{tabular}{@{}l@{\hskip 4ex}r*{4}{@{\hskip 3ex}r}@{}}
      &
      $\tikzpic{\fdot\node at (.2,-.2) {\scriptsize $i$};}[scale=\scl]$
      &
      $\tikzpic{\fcup\node at (1,.4) {\scriptsize $i$};}[scale=\scl]$
      &
      $\tikzpic{\fzip\node at (1,.4) {\scriptsize $i$};}[scale=\scl]$
      &
      $\tikzpic{\fcap\node at (1,.6) {\scriptsize $i$};}[scale=\scl]$
      &
      $\tikzpic{\funzip\node at (1,.6) {\scriptsize $i$};}[scale=\scl]$
      \\*[2ex]
      \midrule
      %%%%%%%%%%%%%%%%%%%%%%%%%%%%%%
      $\lief$
      &
      $0$
      &
      $\normafoam{}{}\tikzpic{\fcup\fdot[.5][.8]\node at (1,.4) {\scriptsize $i$};}[scale=\scl]$
      &
      $\normafoam{}{}\tikzpic{\fzip\fdot[-.3][.7]\node at (1,.4) {\scriptsize $i$};}[scale=\scl]$
      &
      $\normafoam{-\;}{?}\tikzpic{\fcap\fdot[.5][.2]\node at (1,.6) {\scriptsize $i$};}[scale=\scl]$
      &
      $\normafoam{}{}\tikzpic{\funzip\fdot[-.3][.3]\node at (1,.6) {\scriptsize $i$};}[scale=\scl]$
      \\*[\spc]
      %%%%%%%%%%%%%%%%%%%%%%%%%%%%%%
      $\liee$
      &
      $\normafoam{}{(-1)^i}\id_\emptyset$
      &
      $0$
      &
      $0$
      &
      $0$
      &
      $0$
      \\*[\spc]
      %%%%%%%%%%%%%%%%%%%%%%%%%%%%%%
      $\lieh_1$
      &
      $-\;\tikzpic{\fdot\node at (.2,-.2) {\scriptsize $i$};}[scale=\scl]$
      &
      $\;\tikzpic{\fcup\node at (1,.4) {\scriptsize $i$};}[scale=\scl]$
      &
      $0$
      &
      $0$
      &
      $-\;\tikzpic{\funzip\node at (1,.6) {\scriptsize $i$};}[scale=\scl]$
      \\*[\spc]
      %%%%%%%%%%%%%%%%%%%%%%%%%%%%%%
      $\lieh_2$
      &
      $\tikzpic{\fdot\node at (.2,-.2) {\scriptsize $i$};}[scale=\scl]$
      &
      $0$
      &
      $\;\tikzpic{\fzip\node at (1,.4) {\scriptsize $i$};}[scale=\scl]$
      &
      $-\;\tikzpic{\fcap\node at (1,.6) {\scriptsize $i$};}[scale=\scl]$
      &
      $0$
      % \\*[\spc]
      % \cmidrule(r){2-6}
      %%%%%%%%%%%%%%%%%%%%%%%%%%%%%%
      % $\lieh\coloneqq\lieh_1+\lieh_2$
      % $\lieh$
      % &
      % $0$
      % &
      % $-\;\tikzpic{\fcup\node at (1,.4) {\scriptsize $i$};}[scale=\scl]$
      % &
      % $-\;\tikzpic{\fzip\node at (1,.4) {\scriptsize $i$};}[scale=\scl]$
      % &
      % $\;\tikzpic{\fcap\node at (1,.6) {\scriptsize $i$};}[scale=\scl]$
      % &
      % $\;\tikzpic{\funzip\node at (1,.6) {\scriptsize $i$};}[scale=\scl]$
    \end{tabular}
  \end{gather}
  Note that in term of the $\bZ^2$-grading $(a,b)$, we have $h_1(D)=-b D$, $h_2(D)=a D$ and $h(D)=(a-b) D$.
\end{definition}

\begin{remark}
  \label{rem:gloo_is_almost_unique}
  Under certain reasonable assumptions, the $\gloo$-action is almost unique.
  Arguing as in \cref{rem:unicity_generic_graded}, it is reasonable to assume that each family of scalars is independent of $i$. We then view $\varf$, $\varh$ and $\varh'$ as three scalars.
  Any graded derivation on $\pregfoam_d$ of degree $(-1,-1)$ is of the form $\sfe$.
  following \cref{rem:unicity_generic_graded} and \cref{lem:commutator_super_case}, under this assumption any $\gloo$-action by derivation arises as in \cref{cor:generic_action_on_super_foam}.
  Assuming further that $\varf$ and $\vare$ are invertible, one can renormalize the action of $\lief$ and $\liee$, leaving only one parameter $\varh$, having necessarily $\varh'=-1-\varh$.
\end{remark}

\begin{example}
  Another choice compatible with the assumptions of \cref{rem:gloo_is_almost_unique} is $\varf^i=-1$, $\vare=1$, $\varh^i=\varh^{'i}=\frac{1}{2}$, assuming that $2$ is invertible in the ground ring. This gives:
  \begin{center}
    \def\spc{2ex}
    \def\scl{.5}
    \begin{tabular}{@{}l@{\hskip 4ex}r*{4}{@{\hskip 3ex}r}@{}}
      &
      $\tikzpic{\fdot\node at (.2,-.2) {\scriptsize $i$};}[scale=\scl]$
      &
      $\tikzpic{\fcup\node at (1,.4) {\scriptsize $i$};}[scale=\scl]$
      &
      $\tikzpic{\fzip\node at (1,.4) {\scriptsize $i$};}[scale=\scl]$
      &
      $\tikzpic{\fcap\node at (1,.6) {\scriptsize $i$};}[scale=\scl]$
      &
      $\tikzpic{\funzip\node at (1,.6) {\scriptsize $i$};}[scale=\scl]$
      \\*[2ex]
      \midrule
      %%%%%%%%%%%%%%%%%%%%%%%%%%%%%%
      $\lief$
      &
      $0$
      &
      $\normafoam{-\;}{}\tikzpic{\fcup\fdot[.5][.8]\node at (1,.4) {\scriptsize $i$};}[scale=\scl]$
      &
      $\normafoam{-\;}{}\tikzpic{\fzip\fdot[-.3][.7]\node at (1,.4) {\scriptsize $i$};}[scale=\scl]$
      &
      $\normafoam{}{?}\tikzpic{\fcap\fdot[.5][.2]\node at (1,.6) {\scriptsize $i$};}[scale=\scl]$
      &
      $\normafoam{-\;}{}\tikzpic{\funzip\fdot[-.3][.3]\node at (1,.6) {\scriptsize $i$};}[scale=\scl]$
      \\*[\spc]
      %%%%%%%%%%%%%%%%%%%%%%%%%%%%%%
      $\liee$
      &
      $\normafoam{}{(-1)^i}\id_\emptyset$
      &
      $0$
      &
      $0$
      &
      $0$
      &
      $0$
      \\*[\spc]
      %%%%%%%%%%%%%%%%%%%%%%%%%%%%%%
      $\lieh_1$
      &
      $-\;\tikzpic{\fdot\node at (.2,-.2) {\scriptsize $i$};}[scale=\scl]$
      &
      $\frac{1}{2}\;\tikzpic{\fcup\node at (1,.4) {\scriptsize $i$};}[scale=\scl]$
      &
      $-\frac{1}{2}\;\tikzpic{\fzip\node at (1,.4) {\scriptsize $i$};}[scale=\scl]$
      &
      $\frac{1}{2}\;\tikzpic{\fcap\node at (1,.6) {\scriptsize $i$};}[scale=\scl]$
      &
      $-\frac{1}{2}\;\tikzpic{\funzip\node at (1,.6) {\scriptsize $i$};}[scale=\scl]$
      \\*[\spc]
      %%%%%%%%%%%%%%%%%%%%%%%%%%%%%%
      $\lieh_2$
      &
      $\tikzpic{\fdot\node at (.2,-.2) {\scriptsize $i$};}[scale=\scl]$
      &
      $(-1-\frac{1}{2})\;\tikzpic{\fcup\node at (1,.4) {\scriptsize $i$};}[scale=\scl]$
      &
      $-\frac{1}{2}\;\tikzpic{\fzip\node at (1,.4) {\scriptsize $i$};}[scale=\scl]$
      &
      $\frac{1}{2}\;\tikzpic{\fcap\node at (1,.6) {\scriptsize $i$};}[scale=\scl]$
      &
      $-(-1-\frac{1}{2})\;\tikzpic{\funzip\node at (1,.6) {\scriptsize $i$};}[scale=\scl]$
      % \\*[\spc]
      % \cmidrule(r){2-6}
      %%%%%%%%%%%%%%%%%%%%%%%%%%%%%%
      % $\lieh\coloneqq\lieh_1+\lieh_2$
      % $\lieh$
      % &
      % $0$
      % &
      % $-\;\tikzpic{\fcup\node at (1,.4) {\scriptsize $i$};}[scale=\scl]$
      % &
      % $-\;\tikzpic{\fzip\node at (1,.4) {\scriptsize $i$};}[scale=\scl]$
      % &
      % $\;\tikzpic{\fcap\node at (1,.6) {\scriptsize $i$};}[scale=\scl]$
      % &
      % $\;\tikzpic{\funzip\node at (1,.6) {\scriptsize $i$};}[scale=\scl]$
    \end{tabular}
  \end{center}
  % Note that in term of the $\bZ^2$-grading $(a,b)$, we have $h_1(D)=-\frac{1}{2}(a+b) D$, $h_2(D)=\frac{1}{2}(3b-a) D$ and $h(D)=(b-a) D$.
\end{example}

\begin{example}
  A choice which is not compatible with the assumptions of \cref{rem:gloo_is_almost_unique} is $\vare=1$, $\varf^i=0$, $\varh^i=\frac{1}{2}$ and $\varh^{'i}=-\frac{1}{2}$, assuming again that $2$ is invertible in the ground ring. This gives:
  \begin{center}
    \def\spc{2ex}
    \def\scl{.5}
    \begin{tabular}{@{}l@{\hskip 4ex}r*{4}{@{\hskip 3ex}r}@{}}
      &
      $\tikzpic{\fdot\node at (.2,-.2) {\scriptsize $i$};}[scale=\scl]$
      &
      $\tikzpic{\fcup\node at (1,.4) {\scriptsize $i$};}[scale=\scl]$
      &
      $\tikzpic{\fzip\node at (1,.4) {\scriptsize $i$};}[scale=\scl]$
      &
      $\tikzpic{\fcap\node at (1,.6) {\scriptsize $i$};}[scale=\scl]$
      &
      $\tikzpic{\funzip\node at (1,.6) {\scriptsize $i$};}[scale=\scl]$
      \\*[2ex]
      \midrule
      %%%%%%%%%%%%%%%%%%%%%%%%%%%%%%
      $\lief$
      &
      $0$
      &
      $0$
      &
      $0$
      &
      $0$
      &
      $0$
      \\*[\spc]
      %%%%%%%%%%%%%%%%%%%%%%%%%%%%%%
      $\liee$
      &
      $\normafoam{}{(-1)^i}\id_\emptyset$
      &
      $0$
      &
      $0$
      &
      $0$
      &
      $0$
      \\*[\spc]
      %%%%%%%%%%%%%%%%%%%%%%%%%%%%%%
      $\lieh_1$
      &
      $-\;\tikzpic{\fdot\node at (.2,-.2) {\scriptsize $i$};}[scale=\scl]$
      &
      $\frac{1}{2}\;\tikzpic{\fcup\node at (1,.4) {\scriptsize $i$};}[scale=\scl]$
      &
      $-\frac{1}{2}\;\tikzpic{\fzip\node at (1,.4) {\scriptsize $i$};}[scale=\scl]$
      &
      $\frac{1}{2}\;\tikzpic{\fcap\node at (1,.6) {\scriptsize $i$};}[scale=\scl]$
      &
      $-\frac{1}{2}\;\tikzpic{\funzip\node at (1,.6) {\scriptsize $i$};}[scale=\scl]$
      \\*[\spc]
      %%%%%%%%%%%%%%%%%%%%%%%%%%%%%%
      $\lieh_2$
      &
      $\tikzpic{\fdot\node at (.2,-.2) {\scriptsize $i$};}[scale=\scl]$
      &
      $-\frac{1}{2}\;\tikzpic{\fcup\node at (1,.4) {\scriptsize $i$};}[scale=\scl]$
      &
      $\frac{1}{2}\;\tikzpic{\fzip\node at (1,.4) {\scriptsize $i$};}[scale=\scl]$
      &
      $-\frac{1}{2}\;\tikzpic{\fcap\node at (1,.6) {\scriptsize $i$};}[scale=\scl]$
      &
      $\frac{1}{2}\;\tikzpic{\funzip\node at (1,.6) {\scriptsize $i$};}[scale=\scl]$
      % \\*[\spc]
      % \cmidrule(r){2-6}
      %%%%%%%%%%%%%%%%%%%%%%%%%%%%%%
      % $\lieh\coloneqq\lieh_1+\lieh_2$
      % $\lieh$
      % &
      % $0$
      % &
      % $-\;\tikzpic{\fcup\node at (1,.4) {\scriptsize $i$};}[scale=\scl]$
      % &
      % $-\;\tikzpic{\fzip\node at (1,.4) {\scriptsize $i$};}[scale=\scl]$
      % &
      % $\;\tikzpic{\fcap\node at (1,.6) {\scriptsize $i$};}[scale=\scl]$
      % &
      % $\;\tikzpic{\funzip\node at (1,.6) {\scriptsize $i$};}[scale=\scl]$
    \end{tabular}
  \end{center}
  % Note that in term of the $\bZ^2$-grading $(a,b)$, we have $h_1(D)=-\frac{1}{2}(a+b) D$, $h_2(D)=\frac{1}{2}(a+b) D$ and $h(D)=0$.
\end{example}


\subsection{Twist on graded \texorpdfstring{$\glt$}{gl2}-foams}
\label{subsec:twist_foam}

In this subsection, we define webs with green markings and twists on graded $\glt$-foams, following the general framework of \cref{subsubsec:twisting_g2categories} and in analogy with \cite[section~5.1]{QRS+_SymmetriesEquivariantKhovanovRozansky_2023}.
We fix $\fg$ to be either $\fg=\grgl_2^\leq$ or $\fg=\gloo$.
Recall the notation $\fg\foam_d$ after \cref{defn:foam_and_sfoam}, denoting either $\gfoam_d$ or $\sfoam_d$.
Recall that we fixed a structure of $\fg$-2-category on $\fg\prefoam_d$ in \cref{defn:action_on_foam_graded_case} and in \cref{defn:action_on_foam_super_case}, which extends to $\fg\foam_d$.

Below is an example of a web with markings:
\begin{equation*}
  \tikzpic{
    \webid[0][1.5][1]\webs[1][1.5]\webmMarkedT[2][1.5][$(0,3,-3)$]\webid[3][1.5][1]
    \webm\webid[1][0][1]\webid[2][0][1]\websMarkedB[3][0][$(2,-1,-1)$]
  }[xscale=.5,yscale=.4][(0,1.25*.4)]
  =
  \tikzpic{
    \webid[0][1.5][1]\webs[1][1.5]\webm[2][1.5]\webid[3][1.5][1]
    \webm\webid[1][0][1]\webid[2][0][1]\websMarkedB[3][0][$2$]
  }[xscale=.5,yscale=.4][(0,1.25*.4)]
  \;\langle 2,-4\rangle
  \qquad
  \epsilon_{W^{\greenmarking}}(h_1)=2,\quad\epsilon_{W^{\greenmarking}}(h_2)=-4.
\end{equation*}
More formally, a \emph{web with markings $W^{\greenmarking}$} (or \emph{marked web}) is the data of a web $W$ together with markings 
$\greenmarking[2]$ on its edges of width one, each equipped with a triple of scalars in $\ringfoam$, generically denoted $(\alpha,\beta_1,\beta_2)$.
% When $\fg=\gloo$, we further assume that
% \[\alpha=\beta_1+\beta_2.\]
% (This assumption is not necessary for the following few paragraphs, but as we shall see \cref{lem:family_of_twists_foam}, it is necessary for the rest of the paper.)
For a marked web $W^{\greenmarking}$ and $i\in \{1,2\}$, we set $\epsilon_{W^{\greenmarking}}(h_i)$ to be the sum of the $i+1$st entries of all the markings on $W$.
See the example above; the notation of the second web is explained in \cref{rem:notation_bracket_twist}.
If $W_s^{\greenmarking}$ and $W_t^{\greenmarking}$ are two marked webs with $W_s$ and $W_t$ as underlying webs respectively, then any foam $F\colon W_s\to W_t$ defines a foam $F^{{\greenmarking}}\colon W_s^{\greenmarking}\to W_t^{\greenmarking}$.
If $F$ has quantum grading $\qdeg F$, then
\[\qdeg F^{\greenmarking}\coloneqq \qdeg F - (\epsilon_{W_s^{\greenmarking}}(h_2)-\epsilon_{W_s^{\greenmarking}}(h_1)) + (\epsilon_{W_t^{\greenmarking}}(h_2)-\epsilon_{W_t^{\greenmarking}}(h_1)).\]
In other words, adding a twist $\greenmarking$ to a web $W$ shifts it by $q^{\epsilon_{W^{\greenmarking}}(h_2)-\epsilon_{W^{\greenmarking}}(h_1)}$.
Denote $\fg\foam_d^{\text{pre-}\greenmarking}$ the $\fg$-2-category consisting of marked webs and the same Hom-categories as $\fg\foam_d$, restricting to foams preserving the quantum grading. 

We now define a family of twists for the $\fg$-2-category $\gfoam_d^{\text{pre-}\greenmarking}$.

\begin{definition}
  \label{defn:twist_foam}
  Let $\alpha,\beta_1,\beta_2\in\ringfoam$ be three parameters.
  Defines:
  \begin{IEEEeqnarray*}{rClcrClcrCl}
    \lief\left(\;\tikzpic{
      \draw[web1] (-1,0) to (1,0);
      \node[green_mark] at (0,0){};
      \node[below] at (0,0){\scriptsize $(\alpha,\beta_1,\beta_2)$};
    }[scale=.7][(0,0)]\;\right)
    &=&
    \alpha\;
    \tikzpic{
      \draw (-1,0) rectangle (1,1);
      \fdot[0][.5]
    }[scale=.7]
    &\quad&
    \lieh_{i}\left(\;\tikzpic{
      \draw[web1] (-1,0) to (1,0);
      \node[green_mark] at (0,0){};
      \node[below] at (0,0){\scriptsize $(\alpha,\beta_1,\beta_2)$};
    }[scale=.7][(0,0)]\;\right)
    &=&
    \beta_{i}\;
    \tikzpic{
      \draw (-1,0) rectangle (1,1);
    }[scale=.7]
    &\quad&
    \liee\left(\;\tikzpic{
      \draw[web1] (-1,0) to (1,0);
      \node[green_mark] at (0,0){};
      \node[below] at (0,0){\scriptsize $(\alpha,\beta_1,\beta_2)$};
    }[scale=.7][(0,0)]\;\right)
    &=&
    0
    \\
    \tau_{\alpha,\beta_1,\beta_2}(\lief)
    &=&
    \alpha\;
    \tikzpic{
      \draw (-1,0) rectangle (1,1);
      \fdot[0][.5]
    }[scale=.7]
    &&
    \tau_{\alpha,\beta_1,\beta_2}(\lieh)
    &=&
    \beta_{i}\;
    \tikzpic{
      \draw (-1,0) rectangle (1,1);
    }[scale=.7]
    &&
    \tau_{\alpha,\beta_1,\beta_2}(\liee)
    &=&
    0
  \end{IEEEeqnarray*}
  Extending this definition by the Leibniz rule defines for each marked web $W^{\greenmarking}$ a degree-preserving linear map
  \[\tau_{W^{\greenmarking}}\colon\fg\to\End_{\fg\foam_d^{\text{pre-}\greenmarking}}(W^{\greenmarking}).\]
\end{definition}

Note that $\tau_{W^{\greenmarking}}(h_i)=\epsilon_{W^{\greenmarking}}(h_i)\id_W$.

\begin{remark}
  \label{rem:twist-foam-front-back-composition}
  In the definition above, ``extending by the Leibniz rule'' should be understood both with respect to the horizontal composition and with respect to the front-back composition (see \cref{rem:monoidal-2-categorical-structure}).
  Below we sometimes use 2-categorical statements, although we should really be using monoidal 2-categorical statements, and take the front-back composition into account.
  % We shall refer to this remark when encountering such caveat.
\end{remark}


\begin{remark}
  \label{rem:notation_bracket_twist}
  Only the action of $\lief$ depends on the position of the dot. For that reason, we shall use the notation
  \begin{gather*}
    \tikzpic{
      \draw[web1] (-1,0) to (1,0);
      \node[green_mark] (B) at (0,0){};
      \node[below={-3pt} of B] {\scriptsize $(\alpha,\beta_1,\beta_2)$};
    }[scale=.7][(0,0)]
    =
    \tikzpic{
      \draw[web1] (-1,0) to (1,0);
      \node[green_mark] (B) at (0,0){};
      \node[below={-3pt} of B] {\scriptsize $\alpha$};
    }[scale=.7][(0,0)]
    \langle \beta_1,\beta_2 \rangle.
  \end{gather*}
  In particular, marking a green dot with a single scalar $\alpha$ is a notation for marking it with the triple $(\alpha,0,0)$, and the notation $W\langle \beta_1,\beta_2 \rangle$ means ``the web $W$ with an additional marking $(0,\beta_1,\beta_2)$ anywhere''. See the example above.
\end{remark}

Note that $\tau_{W^{\greenmarking}}(f)$ is a sum over the identity foam with a single dot.
We write $\epsilon_{W^{\greenmarking}}(f)$ the sum over all the scalars in front of these dotted identities.

\begin{lemma}
  \label{lem:family_of_twists_foam}
  The family $\tau$ given in \cref{defn:twist_foam} is a family of twists in the sense of \cref{defn:family_of_twists}:
  \begin{enumerate}[(i)]
    \item in the graded case, for any $\grgl_2^{\leq}$-action defined in \cref{cor:generic_action_on_graded_foam};
    \item in the super case, for any $\gloo$-action defined in \cref{cor:generic_action_on_super_foam}, provided that $\epsilon_{W^{\greenmarking}}(f)=\epsilon_{W^{\greenmarking}}(h_1)+\epsilon_{W^{\greenmarking}}(h_2)$.
  \end{enumerate}
\end{lemma}

\begin{proof}
  It is clear that $\tau$ verifies the Leibniz rule and has a graded-commutative image.
  Thanks to \cref{rem:notation_bracket_twist}, we can redistribute twists with respect to $h_1$ and $h_2$, so that if the condition is verified, we may assume it is verified at the level of each twist.
  Following \cref{rem:twist_only_check_on_generators} (bearing \cref{rem:twist-foam-front-back-composition} in mind), it suffices to check flatness locally, that is, a single green marking $\omega=(\alpha,\beta_1,\beta_2)$.
  In the graded case, flatness holds in fact for any $\sff_{\varf}$ and $\sfh_{\delh,\varh}$ defined in \cref{lem:generic_derivations_on_graded_foam}, using \cref{lem:commutator_graded_case} (here we write $\epsilon(\sfh_{\delh,\varh})=\beta$ and $\epsilon(\sfh_{\delh,\varh}')=\beta'$):
  \begin{IEEEeqnarray*}{rCl}
    \IEEEeqnarraymulticol{3}{l}{
      \sfh_{\delh,\varh}\cdot \tau(\sff_{\varf})-\bilfoam(\sfh_{\delh,\varh},\sff_{\varf})\sff_{\varf}\cdot\tau(\sfh_{\delh,\varh})
    }
    \\
    \mspace{80mu}
    &=&
    \sfh_{\delh,\varh}(\alpha\;\tikzpic{\fdot}) - \sff_{\varf}(\beta\id)
    = -\delh\alpha\;\tikzpic{\fdot}
    = \tau(-\delh\sff_{\varf})
    = \tau([\sfh_{\delh,\varh},\sff_{\varf}])
    \\[1ex]
    %%%%%%%%%%%%%%%%%%%%%%%%%%%%%%
    \IEEEeqnarraymulticol{3}{l}{
      \sfh_{\delh,\varh}\cdot \tau(\sfh_{\delh',\varh'})-\bilfoam(\sfh_{\delh,\varh},\sfh_{\delh',\varh'})\sfh_{\delh',\varh'}\cdot\tau(\sfh_{\delh,\varh})
    }
    \\
    &=& \sfh_{\delh,\varh}(\beta')-\sfh_{\delh',\varh'}(\beta\id)=0
    = \tau([\sfh_{\delh,\varh},\sfh_{\delh',\varh'}])
    \\[1ex]
    %%%%%%%%%%%%%%%%%%%%%%%%%%%%%%
    \IEEEeqnarraymulticol{3}{l}{
      \sff_{\varf}\cdot \tau(\sff_{\varf'})-\bilfoam(\sff_{\varf},\sff_{\varf'})\sff_{\varf'}\cdot\tau(\sff_{\varf})
    }
    \\
    &=&
    \sff_{\varf}(\alpha'\;\tikzpic{\fdot})-XY\sff_{\varf'}(\alpha\;\tikzpic{\fdot})=0
    = \tau([\sff_{\varf},\sff_{\varf'}])
  \end{IEEEeqnarray*}
  We do a similar computation in the super case, using \cref{lem:commutator_super_case}:
  \begin{IEEEeqnarray*}{rCl}
    \IEEEeqnarraymulticol{3}{l}{
      \sfh_{\delh,\varh}\cdot \tau(\sfe_{\vare})-\bilfoam(\sfh_{\delh,\varh},\sfe_{\vare})\sfe_{\vare}\cdot\tau(\sfh_{\delh,\varh})
    }
    \\
    \mspace{80mu}
    &=&
    - \sfe_{\vare}(\beta\id) = 0 = \tau(\delh\sfe_{\vare})
    = \tau([\sfh_{\delh,\varh},\sfe_{\vare}])
    \\[1ex]
    %%%%%%%%%%%%%%%%%%%%%%%%%%%%%%
    \IEEEeqnarraymulticol{3}{l}{
      \sfe_{\vare}\cdot \tau(\sff_{\varf})-\bilfoam(\sfe_{\vare},\sff_{\varf})\sff_{\varf}\cdot\tau(\sfe_{\vare})
    }
    \\
    &=&
    \alpha\normafoam{\vare}{(-1)^i\vare}
    \overset{?}{=}
    \beta\id
    = \tau(\sfh_{0,(-\normafoam{}{(-1)^i}\vare\varf^i)_i})
    =
    \tau([\sfe_{\vare},\sff_{\varf}])
    \\[1ex]
    %%%%%%%%%%%%%%%%%%%%%%%%%%%%%%
    \IEEEeqnarraymulticol{3}{l}{
      \sfe_{\vare}\cdot \tau(\sfe_{\vare'})-\bilfoam(\sfe_{\vare},\sfe_{\vare'})\sfe_{\vare'}\cdot\tau(\sfe_{\vare})
    }
    \\
    &=& 0
    = \tau([\sfe_{\vare},\sfe_{\vare'}])
  \end{IEEEeqnarray*}
  The only equality that does not hold formally is the condition coming from the commutation between $\sff$ and $\sfe$, as it requires
  $\alpha
  =
  \beta$, where here $\beta=\beta_1+\beta_2$.
\end{proof}

\begin{definition}
  \label{defn:twisted_foams}
  We denote $\fg\markedfoam[d]\coloneqq\left(\fg\foam_d^{\text{pre-}\greenmarking}\right)^\tau$, where $\tau$ is the family of twists defined in \cref{defn:twist_foam}.
\end{definition}


We conclude this subsection by gathering some properties of twists.

\begin{lemma}[$\lieh$-equivariance]
  \label{lem:h-equivariance}
  Suppose that $G\in \Hom_{\markedgfoam[d]} (W_s^{\greenmarking}, W_t^{\greenmarking}) $ is homogeneous of degree $\deg(G)=(a,b)$. Then the following statements are true:
  \begin{enumerate}[(i)]
    \item $G$ is $h_1$-equivariant if and only if $\tau_{W_t^{\greenmarking}}(h_2)-b-\tau_{W_s^{\greenmarking}}(h_2)=0$;
    \item $G$ is $h_2$-equivariant if and only if $\tau_{W_t^{\greenmarking}}(h_2)+a-\tau_{W_s^{\greenmarking}}(h_2)=0$.
  \end{enumerate}
  In particular, if $G\colon W_s^{\greenmarking}\to W_t^{\greenmarking}$ is $\fg$-equivariant, then $G\colon W_s^{\greenmarking}\langle a,b\rangle\to W_t^{\greenmarking}\langle a,b\rangle$ is $\fg$-equivariant.
\end{lemma}

\begin{lemma}[$\lief$-equivariance]
  \label{lem:f-equivariance}
  Let $W_s^{\greenmarking}$ and $W_t^{\greenmarking}$ be two marked webs with the same underlying web $W$.
  Let $F\colon W_s^{\greenmarking}\to W_t^{\greenmarking}$ be a linear combination where each term is $\id_W$ decorated with a single dot.
  Let $G$ and $H$ be linear combinations of $\lief$-equivariant foams, suitably composable with $F$.
  If
  \[G\circ\tau_{W_s^{\greenmarking}}(f)\circ H=G\circ\tau_{W_t^{\greenmarking}}(f)\circ H,\]
  then $G\circ F\circ H$ is $f$-equivariant.
  In particular, if $\tau_{W_s^{\greenmarking}}(f)=\tau_{W_t^{\greenmarking}}(f)$, then $F$ is $\lief$-equivariant.
\end{lemma}

In some sense, the condition states that ``globally'', i.e.\ when dots are allowed to move in $G\circ F\circ H$, the marked webs $W_s^{\greenmarking}$ and $W_t^{\greenmarking}$ have identical $f$-markings. In practice, one can move $f$-markings along connected components of the underlying unmarked web $W$, and across components if they happen to be connected in $G\circ F\circ H$.

\begin{lemma}[$\liee$-equivariance]
  \label{lem:e-equivariance}
  Let $W_s^{\greenmarking}$ and $W_t^{\greenmarking}$ be two marked webs with the same underlying web $W$.
  Let $F\colon W_s^{\greenmarking}\to W_t^{\greenmarking}$ be a linear combination where each term is $\id_W$ decorated with a single dot.
  Write $\epsilon(F)$ for the sum of coefficients in $F$.
  If
  \[\epsilon(F)=0,\]
  then $F$ is $\liee$-equivariant.
\end{lemma}

Below we sometimes implicitly assume that $2$ is invertible in the ground ring.

\begin{notation}
We use the following notation:
\begin{equation*}
  \tikzpic{\draw (0,0) to (1,0);\node[round_mark] (A) at (.5,0) {};}
  \quad\coloneqq\quad
  \tikzpic{
    \coordinate (base) at (.5,-.5ex);
    \draw (0,0) to (1,0);\node[green_mark] (A) at (.5,0) {};
    \node[below={-3pt} of A] {\scriptsize $(-1,-\frac{1}{2},-\frac{1}{2})$};
  }[baseline=(base)]
\end{equation*}
\end{notation}

\begin{lemma}
  \label{lem:twist_dot_slide_cup_cap}
  \def\webscl{.6}
  The following are $\fg$-equivariant:
  \begin{gather*}
    \tikzpic{\websMarkedT[0][0][$\alpha$]}%
      [scale=\webscl][(.5,.5*\webscl)]
    =
    \tikzpic{\websMarkedB[0][0][$\alpha$]}%
      [scale=\webscl][(.5,.5*\webscl)]
    \qquad\an\qquad
    \tikzpic{\webmMarkedT[0][0][$\alpha$]}%
      [scale=\webscl][(.5,.5*\webscl)]
    =
    \tikzpic{\webmMarkedB[0][0][$\alpha$]}%
      [scale=\webscl][(.5,.5*\webscl)]
  \end{gather*}
\end{lemma}

\begin{lemma}
\label{lem:crossing_complex_equivariant}
  \def\webscl{.6}
  The following are $\fg$-equivariant:
  \begin{IEEEeqnarray*}{CcC}
    \tikzpic{\webid}[scale=\webscl][(.5,.5*\webscl)]
    %
    \quad
    \xrightarrow{\tikzpic{\fzip}[scale=.6]}
    \quad
    %
    \tikzpic{\websRoundMarkedB[0][0][]\webm[1][0]}%
      [scale=\webscl][(.5,.5*\webscl)]
    \;\langle \frac{1}{2},-\frac{1}{2}\rangle
    %%%%%%%%%%%%%%%%%%%%
    &\mspace{80mu}&
    %%%%%%%%%%%%%%%%%%%%
    \tikzpic{\websRoundMarkedB[0][0][]\webm[1][0]}%
      [scale=\webscl][(.5,.5*\webscl)]
    \;\langle -\frac{1}{2},\frac{1}{2}\rangle
    %
    \quad
    \xrightarrow{\tikzpic{\funzip}[scale=.6]}
    \quad
    %
    \tikzpic{\webid}[scale=\webscl][(.5,.5*\webscl)]
    %%%%%%%%%%%%%%%%%%%%
    \\[2ex]
    %%%%%%%%%%%%%%%%%%%%
    \tikzpic{\draw[web2] (-1,0) to (1,0);}%
      [scale=\webscl][(.5,0*\webscl)]
    %
    \quad
    \xrightarrow{\tikzpic{\fcup}[scale=.6]}
    \quad
    %
    \tikzpic{\webmRoundMarkedB[0][0][]\webs[1][0]}%
      [scale=\webscl][(.5,.5*\webscl)]
    \;\langle -\frac{1}{2},\frac{1}{2}\rangle
    %%%%%%%%%%%%%%%%%%%%
    &\mspace{80mu}&
    %%%%%%%%%%%%%%%%%%%%
    \tikzpic{\webmRoundMarkedB[0][0][]\webs[1][0]}%
      [scale=\webscl][(.5,.5*\webscl)]
    \;\langle \frac{1}{2},-\frac{1}{2}\rangle
    %
    \quad
    \xrightarrow{\tikzpic{\fcap}[scale=.6]}
    \quad
    %
    \tikzpic{\draw[web2] (-1,0) to (1,0);}%
      [scale=\webscl][(.5,0*\webscl)]
  \end{IEEEeqnarray*}
\end{lemma}
 
\begin{lemma}
  \label{lem:web_defining_relations_equivariance}
  \def\webscl{.6}
  The following isomorphisms are $\fg$-equivariant:
  \begin{gather*}
    \stackanchor{W_{i,s_1}W_{j,s_2}\cong^\fg W_{j,s_2}W_{i,s_1}}{(\text{for all }s_1,s_2\in\{-,+\}\an \abs{i-j}>1)}
    \\[2ex]
    \tikzpic{
      \websRoundMarkedB[1][1]\webm[2][1]
      \webmRoundMarkedB\webs[3][0]
      \draw[web1] (0,2) to (1,2);\draw[web1] (3,2) to (4,2);
      \draw[web1] (1,0) to (3,0);
    }[scale=.4][(2,1.25*.4)]
    \cong^\fg
    \tikzpic{
      \coordinate (C) at (2,1);
      \draw[web1] (0,2) to (4,2);
      \draw[web2] (0,.5) to (4,.5);
    }[scale=.4][(2,1.25*.4)]
    \quad\an\quad
    \tikzpic{
      \webmRoundMarkedB[0][1]\webs[3][1]
      \websRoundMarkedB[1][0]\webm[2][0]
      \draw[web1] (0,0) to (1,0);\draw[web1] (3,0) to (4,0);
      \draw[web1] (1,2) to (3,2);
    }[scale=.4][(2,.75*.4)]
    \cong^\fg
    \tikzpic{
      \draw[web1] (0,0) to (4,0);
      \draw[web2] (0,1.5) to (4,1.5);
    }[scale=.4][(2,.75*.4)]
  \end{gather*}
  Furthermore, the following is a $\fg$-equivariant split short exact sequence:
  \begin{gather*}
    \tikzpic{
      \draw[web2] (0,0) to (2,0);
    }[scale=\webscl]
    \;\langle \frac{1}{2},-\frac{1}{2}\rangle
    %
    \quad
    \xrightarrow{\tikzpic{\fcup}[scale=.6]}
    \quad
    %
    \tikzpic{\webmRoundMarkedB[0][0][]\webs[1][0]}%
      [scale=\webscl][(0,.5*\webscl)]
    %
    \quad
    \xrightarrow{\tikzpic{\fcap}[scale=.6]}
    \quad
    %
    \tikzpic{
      \draw[web2] (0,0) to (2,0);
    }[scale=\webscl]
    \;\langle -\frac{1}{2},\frac{1}{2}\rangle
  \end{gather*}
\end{lemma}
