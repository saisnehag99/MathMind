Next, we want to examine the morphism of schemes $\pi_f \colon \Spec(S_f) \to \Spec(S_{(f)})$, which is induced by the ring homomorphism $S_{(f)} \to S_f$. Most importantly, we show that $\Proj^D(S)$ coincides with the geometric quotient obtained from gluing the affine geometric quotients \(\Spec(S_f) \sslash  G\cong\Spec(S_{(f)})\).


\begin{definition}\label{def:veronese_ring}
Let $H \le D$ be a subgroup. The \emph{Veronese subring} of $S$ with respect to $H$ is defined as
\begin{align*}
    S_H \deq \bigoplus_{d \in H} S_d.
\end{align*}
$S_H$ is obviously graded by $H$ and $f \in S_H$ is $D$-homogeneous if and only if it is $H$-homogeneous.
\end{definition}



The following fact from \cite{BS} (page 6) states that periodic rings have \emph{enough} units to form a Laurent polynomial algebra with $\rank(D)$ variables. 

\begin{lemma}\label{lem:laurent_pol_algebra}
Let $S$ be periodic. Then there is a free abelian subgroup $F \le D$ of finite index such that the Veronese subring $S_F = \bigoplus_{d \in F} S_d$ is a \emph{Laurent polynomial algebra} $S_0[T_1^{\pm 1}, \ldots, T_r^{\pm 1}]$. 
\end{lemma}



\begin{proposition}\label{prop:index_integral}
Let $H \le D$ be of finite index. Then $S|S_H$ is integral.
\end{proposition}

\begin{proof}
Since integrality is closed under addition, homogeneous elements $f \in S$ are integral over $S_H$. So consider $f \in S_d$. The number of cosets $i d H$ for $i \in \BN$, has to be finite. Therefore, there is an element $N \in \BN$ such that $N d \in H$. But 
\begin{equation*}
    \deg_D(f^N) \equ \deg_D(f) \cdot \ldots \cdot \deg_D(f) \equ d^N
\end{equation*}
that is, $f^N \in S_H$. Thus $f$ is integral over $S_H$.
\end{proof}



Now we can examine $\pi_f$. Note that the corresponding statement in \cite{BS} (Lemma 2.1) is stated for arbitrary rings $S$, but the proof uses GIT for $S$  being a finitely generated algebra over a field. We will work with the following more general notion (cf.\   \cite[\href{https://stacks.math.columbia.edu/tag/04AE}{Definition 04AE}]{stacks-project}). In particular, our statement is new and strictly generalizes \cite{BS}, Lemma 2.1.

\begin{definition}\label{def:geom_quot_gen}
    Let $T$ be a scheme and $B$ an algebraic space over $T$. Let $j \colon R \to U \times_B U$ be a pre-relation. A morphism $\Phi\colon U \to X$ of algebraic spaces over $B$ is called a \emph{geometric} quotient, if
    \begin{enumerate}[label=(\arabic*)]
        \item $\Phi$ is an orbit space.
        \item condition (1) holds universally, i.e. $\Phi$ is universally submersive, and
        \item the functions on $X$ are the $R$-invariant functions on $U$.
    \end{enumerate}
\end{definition}




The concept of a geometric quotient is closely related to that of a pseudo $G$-torsor, but the latter is a strictly stronger notion (cf.\  \cite[\href{https://stacks.math.columbia.edu/tag/0498}{Definition 0498}]{stacks-project}). 

\begin{definition}\label{def:pseudo_G_torsor}
    Let $D$ be a finitely generated abelian group and $S$ a $D$-graded ring. Let $G = \Spec(S_0[D])$ and denote the group action on $U_f = \Spec(S_f)$ by $\alpha_f \colon G \times_{\Spec(S_0)} U_f \to U_f$.
    \begin{enumerate}[label=(\arabic*)]
        \item  We say that $U_f$ is a \emph{pseudo $G$-torsor} under $G$, if the induced morphism of schemes 
        \begin{align*}
            \Phi_f \colon G \times_{\Spec(S_0)} U_f \to U_f \times_{\Spec(S_0)} U_f,\ (g, x) \mapsto (\alpha_f(g, x), x)
        \end{align*}
   is an isomorphism of $S_0$-schemes. The map $\Phi_f$ is often called \emph{shear map} in the literature.
   \item $U_f$ is called a \emph{trivial pseudo $G$-torsor}, if $U_f$ is a pseudo $G$-torsor such that there exists an $G$-equivariant isomorphism $G \to U_f$ over $\Spec(S_0)$, where $G$ acts by left multiplication.        
    \end{enumerate}
\end{definition}


Every pseudo $G$-torsor is also a geometric quotient: As the map $\Phi_f$ is an isomorphism if and only if
\begin{align*}
    \varphi_f \colon S_f \otimes_{S_0} S_f \to S_f \otimes_{S_0} S_0[D],\ g \otimes h \mapsto gh \otimes \chi^{\deg(g)}
\end{align*}
is an isomorphism, we can see that we need to require some conditions on $S_f$. Taking $1 \otimes \chi^d \in S_f \otimes_{S_0} S_f$, it is immediate that the surjectivity of $\phi_f$ is equivalent to the existence of units in $(S_f)_d$ for every degree $d \in D$. This is apparently a strictly stronger condition than relevance, where units only exist in every degree $d \in D^f$.
We will also see that the pre-relation $j$ from Definition~\ref{def:geom_quot_gen} is exactly the scheme morphism $\Phi_f$.


\begin{theorem} \label{lem:periodic_quotient} 
Let $F$ be the free abelian part of the grading group $D$ and assume that the $D$-graded ring $S$ is a finitely generated $S_F$-module.
Then, for periodic $S$, the map $\pi \colon \Spec(S) \to \Spec(S_0)$ induced by $S_0 \to S$ is a geometric quotient in the category of schemes.

In particular, for all relevant $f \in S$ (where $S$ is not necessarily periodic), the corresponding statement holds for the maps $\pi_f \colon \Spec(S_f) \to \Spec(S_{(f)})$.
\end{theorem}




\begin{proof}
We start with identifying the roles in the above Definition.
Let $T = B = \Spec(S_0)$ and $\Phi = \pi_f$ (hence $U = U_f = \Spec(S_f)$ and $X = \Spec(S_{(f)})$). Then for $R := G \times_{\Spec(S_0)} U_f$ we see that $\Phi_f \colon G \times_{\Spec(S_0)}U_f \to U_f \times_{\Spec(S_0)} U_f$, given by $(g,x) \mapsto (gx, x)$ is a morphism  of schemes (induced action on $\Spec(S_f)$), and hence $j = \Phi_f$ is a pre-relation. As $j$ is exactly the morphism that corresponds to the induced action on $\Spec(S_f)$, that is, $\Phi_f$, we can apply Proposition~\ref{prop:S_f^G=S_(f)} directly to deduce (3). \\ 




Let $D' \le D$ denote the finite index subgroup of $D$ that is given by the degrees of homogeneous units, and denote the corresponding free group from Lemma~\ref{lem:laurent_pol_algebra} by $F$. 
We have a short exact sequence
 \begin{align*}
        0 \to F \to D \to D/F \to 0 ,
\end{align*}
giving rise to a short exact sequence of diagonalizable group schemes \begin{align*}
        0 \to G_{D/F} \to G_D \to G_F\to 0
\end{align*}
 by Proposition~\ref{prop:G_functor_represent}, where we denote $G_M = \Spec(S_0[M])$ for an abelian group $M$. It is also clear that we have ring maps $S_0 \to S_F \to S$, giving rise to a factorization of $\pi$ into $\Spec(S) \to \Spec(S_F) \to \Spec(S_0)$. 
We aim to show that:
    \begin{itemize}
        \item $\pi_1 \colon \Spec(S) \to \Spec(S_F)$ is a geometric quotient w.r.t $G_{D/F}$.
        \item $\pi_2 \colon \Spec(S_F) \to \Spec(S_0)$ is a geometric quotient w.r.t. $G_F$.
    \end{itemize}
    So let's start with $\pi_1$. By Proposition~\ref{prop:S_f^G=S_(f)} we know that $S^{G_{D/F}}$ coincides with the degree zero part of $S$ viewed as $D/F$-graded ring, which is exactly $S_F$, showing (3). Since  $S|S_F$ is integral by Proposition~\ref{prop:index_integral} and finite by assumption, we have a finite surjective morphism $\pi_1 \colon \Spec(S) \to \Spec(S_F)$ with a $G_{D/F}$--invariant action. By \cite[\href{https://stacks.math.columbia.edu/tag/01WM}{Lemma 01WM}]{stacks-project}, $\pi_1$ is also universally closed. Then by  \cite[\href{https://stacks.math.columbia.edu/tag/0AAU}{Lemma 0AAU}]{stacks-project} $\pi_1$ is submersive. As $\pi_1$ remains surjective and closed after base change, it also remains submersive after base change and (2) holds true. Finally, (1) follows from \cite[\href{https://stacks.math.columbia.edu/tag/049Z}{Lemma 049Z}]{stacks-project}, since $\pi_1$ is surjective. \\

    

Regarding $\pi_2$, it holds that $S_F \cong S_0[F]$, as $S_F$ is a $F$-graded Laurent polynomial ring and is of finite type over $S_0$ (since $\rank(D)$ is finite). Furthermore, the action of $G_F$ on $\Spec(S_F)$ is free and transitive, i.e.\ $\Spec(S_F) = G_F$ (and the corresponding shear map from Definition~\ref{def:pseudo_G_torsor} is an isomorphism). Thus $\pi_2$ is a Zariski locally trivial (pseudo) $G_F$-torsor and hence the projection $\Spec(S_F) \to \Spec(S_0)$ is a geometric quotient. 
As the composition of geometric quotients is a geometric quotient, $\pi\colon \Spec(S) \to \Spec(S') \to \Spec(S_0)$ is a geometric quotient by the action of $G_{D/F} \times_{G_F} G_F \cong G_D$. Since both $\pi_1$ and $\pi_2$ are affine, the geometric quotient actually is a scheme by construction (cf.\   \cite[\href{https://stacks.math.columbia.edu/tag/03JH}{Lemma 03JH}]{stacks-project}).
\end{proof}


For the pseudo $G$-torsor structure, we need stronger assumptions on $S_f$. We want to emphasize that the following theorem is a completely new result.


\begin{theorem}\label{thm:quotient_pi_f_general}
    Let $S$ be a $D$-graded ring, $f \in \Rel^D(S)$ and $\pi_f\colon \Spec(S_f) \to \Spec(S_{(f)})$ be the map induced by $S_{(f)} \to S_f$. Then $\Spec(S_f)$ is a pseudo $G$-torsor if and only if $\deg( (S_f)^\times) = D$ (or equivalently $D^f = D$) and $S$ is integral.
    In particular, if $\deg( (S_f)^\times) = D$ holds and $S$ is integral, then $\pi_f$ is Zariski locally trivial for the base change $S_0 \to S_{(f)}$.
\end{theorem}

\begin{proof}
First note that the new condition $\deg( (S_f)^\times) = D$ ensures that we have a unit $u_d \in (S_f)^\times$ of degree $d$ for every $d \in D$. Hence $\supp( (S_f)^\times) =D$ and $1 \in S_d S_{-d}$ for all $d \in D$, i.e. $S_f$ is \emph{stronlgy graded} (cf.\  \cite{Dade80}, Proposition 1.6)).
Let $\deg( (S_f)^\times) = D$ hold, where $S$ is integral.
Regarding the injectivity, we see that an element $g \otimes h$ lies in the kernel of $\varphi_f$ if and only if $gh = 0$. As we assume $S$ to be integral, we can deduce that $g=0$ or $h=0$. Now as $0 \otimes 0 = a \otimes 0 = 0 \otimes b$, we see that $g \otimes h \in \ker(\varphi_f)$ if and only if $g \otimes h$ is the $0$-tensor. Hence $\varphi_f$ is injective. 
Conversely, the injectivity of $\varphi_f$ forces $S$ to be an integral domain.   


For surjectivity, we have to show that an arbitrary element $g \otimes \chi^d \in S_f \otimes_{S_0} S_0[D]$ lies in the image of $\varphi_f$. By assumption, there exists a unit $u_d \in (S_f)_d$, hence for $g = u$ and $h = u^{-1} g$ we see that $\varphi_f(gh) = g \otimes \chi^d$. 
If $\varphi_f$ is surjective, this forces units in $S_f$ to exist in every degree $d\in D$.


For the local triviality, we have to show that there is an isomorphism $\varphi\colon S_f \to S_{(f)}[D]$. First, by assumption, we can choose homogeneous units $u_d \in (S_f)_d$ for every degree $d \in D$. We define $\varphi$ by $q \mapsto \frac{q}{u_{\deg(q)}} \chi^{\deg(q)}$. For $q \chi^d \in S_{(f)}[D]$, we see that $\varphi(u_d q) = q \chi^d$ and $\varphi$ is surjective.
Regarding injectivity, let $q = \sum_{d\in D} q_d$ such that $\varphi(q) = 0$, i.e.\ $\sum_{d\in D} \frac{q_d}{u_d} \chi^d=0$. 
As $\chi^d \neq 0$ and the various $\chi^d$ are linearly independent by Proposition~\ref{prop:group_like} (since $S$ is integral), we deduce that $\frac{q_d}{u_d} = 0$ for all $d \in D$ and hence $q_d = 0$ for all $d$ and $\varphi$ is injective. Finally, for $g \in G$ and $d := \deg(q)$ it holds that
    \begin{align*}
        \varphi(g \bullet q) \equ  \varphi(\chi^d(g) q) \equ
        \chi^{d}(g) \varphi(q) \equ
        \chi^{d}(g) \frac{q}{u_{d}} \chi^{d} \equ g \odot (\frac{q}{u_{d}} \chi^{d}) \equ g \odot \varphi(q),
    \end{align*}
    where $\bullet$ denotes the induced $G$-action on $S_f$ and $\odot$ denotes the induced $G$-action on $S_{(f)}[D]$ (that is induced via the base change $S_0 \to S_{(f)}$). Hence $\varphi$ is also $G$-equivariant.
\end{proof}




The above discussion gives rise to the following definition.

\begin{definition}
    Let $D$ be a finitely generated abelian group and $S$ a $D$-graded ring. We call a relevant element $f \in \Rel^D(S)$ \emph{strongly relevant}, if $S_f$ contains a homogeneous unit in every degree $d \in D$, i.e., when $D^f = D$.
\end{definition}

We can interpret the shear map in terms of properties of the action of $G$ on $U_f$.

\begin{lemma}
$\Phi_f$ is injective if and only if the action of $G$ on $\Spec(S_f)$ is free.
Moreover, $\Phi_f$ is surjective if and only if the action of $G$ on $\Spec(S_f)$ is transitive.
\end{lemma}

\begin{proof}
    If the action is free and there exist $x, y \in U_f$ and $g, h \in G$ such that $(gx,x) = (hy,y)$, clearly $x= y$ and hence $gx = hx$. Therefore, $g = h$ by freeness. Conversely, let $\Phi_f$ be injective and let $g \in G, x \in U_f$ such that $gx = x$, i.e.\ $(gx, x) = (1_G x, x)$. By injectivity, we deduce that $g = e$ and thus, the action is free.

    For the second statement, it holds that $(x, y) \in \im(\Phi_f)$ if and only if there exists a $g \in G$ such that $gy = x$, i.e.\ such that $(x, y) = (gy, y)$.
\end{proof}

For relevant $f \in S$, we define $G^f := \Spec(S_0[D^f])$.

\begin{corollary}\label{cor:G^f-torsor}
    Let $D$ be a finitely generated abelian group, $S$ a $D$-graded integral domain 
    and $f \in S$ relevant. Then $\pi_f \colon \Spec(S_f) \to \Spec(S_{(f)})$ is a $G^f$-torsor.
    In particular, if $\supp(S_f) = D^f$, then $\pi_f$ is Zariski locally trivial for the base change $S_0 \to S_{(f)}$.

Hence, for an integral $D$-graded domain $S$ and relevant $f \in S$, $\pi_f$ is a pseudo $G$-torsor if and only if $\supp((S_f)) = D^f = D$.



\end{corollary}

\begin{proof}
    We have to show that the map
    \begin{align*}
    \varphi_f\colon    S_f \otimes_{S_0} S_f \to S_f \otimes_{S_0} S_0[D^f],\ g \otimes h \mapsto gh \otimes \chi^{\deg(g)}
    \end{align*}
    is an isomorphism. However, as we assume the grading to be effective, we know that $\varphi_f$ is an isomorphism if and only if $\deg((S_f)^\times) = D^f$ by Theorem~\ref{thm:quotient_pi_f_general}. But this holds by definition of $D^f$, as every divisor of some power of $f$ corresponds to a unit in $S_f$. Alternatively, this statement is equivalent to the second part of the proof of Theorem~\ref{lem:periodic_quotient} and hence true.


    Now consider the base change $S_0 \to S_{(f)}$. Then $\pi_f$ is a trivial $G^f$-torsor if and only if $S_f \otimes_{S_{(f)}} S_{(f)} \cong S_{(f)} \otimes_{S_0} S_0[D^f] = S_{(f)}[D^f]$ as $S_{(f)}$-algebras. For all $d \in D^f$, we choose a unit $u_d \in ((S_f)_d)^\times$ and consider the following morphism of $S_{(f)}$-algebras
    \begin{align*}
        \varphi\colon S_f \to S_{(f)}[D^f],\ q \chi^d \mapsto \frac{q}{u_{\deg(q)}} \chi^d,
    \end{align*}
    which is well-defined by assumption (i.e.\ $\supp(S_f) = D^f$).
    Let $q \cdot \chi^d \in S_{(f)}[D^f]$. Then
    \begin{align*}
        \varphi(u_d q) \equ \frac{q u_d}{u_d} \chi^d \equ q \chi^d,
    \end{align*}
    i.e.\ it is surjective. Regarding injectivity, let $q = \sum_{d\in D^f}$ such that $\varphi(q) = 0$. Then $\sum_{d\in D^f} \frac{q_d}{u_d} \chi^d=0$. As $\chi^d \neq 0$ and the various $\chi^d$ are linearly independent (by Proposition~\ref{prop:group_like}), we deduce that $\frac{q_d}{u_d} = 0$ for all $d \in D^f$ and hence $q_d = 0$ for all $d$ and $\varphi$ is injective. Finally, $\varphi$ is $G^f$-equivariant by the same reasoning as in Theorem~\ref{thm:quotient_pi_f_general}.
\end{proof}





\begin{example}\label{ex:double_origin_3rd}
    The above conditions seem to be satisfied in many cases. For example, consider again the ring $S = \BC[x, y, z]$ graded by $\BZ^2$, where $x \mapsto (1, 0)$, $y \mapsto (0, 1)$ and $z \mapsto (1, 1)$. It holds $S_+ = (xy, xz, yz)$, and we have already shown that every generator is strongly relevant in Example~\ref{ex:double_origin_P^1_2nd}, so that $\pi_f$ is locally trivial in the Zariski topology for $f = xy, xz, yz$.
    In particular, as $D = \BZ^2$ has no torsion, we conclude that $\Proj^D(S)$ is a pseudo $G$-torsor. 
    Also note that Example~\ref{ex:standard_ex} (1) satisfies those conditions, whereas (2) does not.
\end{example}




The following definition is due to \cite{BS}.

\begin{definition}\label{def:irrelevant_ideal}
We call $V(S_+) \subseteq \Spec(S)$ the \emph{irrelevant subscheme} and $D(S_+) \subseteq \Spec(S)$ the \emph{relevant locus}. We denote the affine projection $\Spec(S) \setminus V(S_+) \to \Proj^D(S)$ by $\pi_+$.
\end{definition}


    Note that $\pi_+$ coincides with the quotient map $p$ (see Proposition~\ref{prop:Liu_quotient}) on $D(f)$ for $f \in S$ relevant, as $(S_f)^G  = S_{(f)}$.






\begin{proposition}\label{prop:irrel_quotient}
Let $S$ be a multigraded ring, graded by $D$ with free abelian part $F$.
\begin{enumerate}[label=(\alph*)]
    \item Let $S$ be a finitely generated $S_F$-module.
    Then the affine projection morphism
\begin{align*}
    \pi_+ \colon \Spec(S) \setminus V(S_+) \, \to\, \Proj^D(S)
\end{align*}
is a geometric quotient for the induced action. 

    \item Let $S$ be an integral domain such that $\Spec(S_0)$ is connected and $\deg((S_f)^\times) = D$ holds for all relevant $f \in S$. Then the affine projection morphism
\begin{align*}
    \pi_+ \colon \Spec(S) \setminus V(S_+) \, \to\, \Proj^D(S)
\end{align*}
is a pseudo $G$-torsor that is Zariski locally trivial over the base $\Spec(S_{(f)})$.
\end{enumerate}
\end{proposition}

\begin{proof}
    The statements follow from Theorem~\ref{lem:periodic_quotient} and Theorem~\ref{thm:quotient_pi_f_general} and gluing, as the conditions are local in either case.
\end{proof}



%%%%%%%%%%%%%%%
The affine projection comes with the expected properties. The following statement is the generalization of a standard result from affine GIT (cf.\  \cite{Hos}, Corollary 4.31).

\begin{proposition}\label{prop:orbit_closed}
Consider $\pi_+\colon D(S_+) \to \Proj^D(S)$. Then
    \begin{align*}
        \pi_+(x) = \pi_+(y) \ \iff \ \overline{G \cdot x} \cap \overline{G \cdot y} \neq \emptyset 
    \end{align*}
\end{proposition}

\begin{proof}
\Ra As $\pi_+$ is a geometric quotient by Theorem~\ref{lem:periodic_quotient}, fibers of $\pi_+$ are exactly orbits of $G$, so $\pi_+(x) = \pi_+(y)$ yields that $x$ and $y$ lie in the same orbit. Hence $\overline{G \cdot x} = \overline{G \cdot y}$ and thus their intersection cannot be empty (since $x, y$ are contained). 

\La Let $z \in \overline{G \cdot x} \cap \overline{G \cdot y}$. As the domain of $\pi_+$ is $D(S_+)$, there must exist some relevant $f \in S$ such that $z \in \Spec(S_f)$. As $\Spec(S_f)$ is open, $\Spec(S_f) \cap (G \cdot x) \neq \emptyset$, so that we may choose $x' \in G\cdot x \cap \Spec(S_f)$ and $y' \in G \cdot y \cap \Spec(S_f)$. We assume that $\pi_f(x') \neq \pi_f(y')$, so that there exists a $G$-invariant regular function $h \in S_{(f)}$ such that $h(x') \neq h(y')$. Note that $h(x')$ is defined to be the image of $h$ in the residue field $\kappa(x')= \Quot(S_f / \Fp_{x'})$.
But as $h$ is continuous and constant on orbits and $z \in \overline{G\cdot x'} \cap \overline{G \cdot y'}$, it must hold that $h(x') = h(z)$ and $h(y') = h(z)$. As this is a contradiction, the claim follows.
\end{proof}


%%%%%%%%%%%%%%%%%%%%%%%%%%%%%%%%%%%%%%%%%%%%%%%%%%%%%%%%%%%%%%%%%%%%%%%%%%




\begin{example}[Projective Space]\label{exa:special_proj}
Let $S = \BC[x_0, \ldots, x_n]$ be a $\BZ$-graded polynomial ring such that $\deg_D(x_i) =1$. Hence, $S_+ = (x_0, \ldots, x_n)$, therefore $\Proj^D(S)$ is the classical Proj, which we denote by $\Proj^\BN(S)$. The group action by $G = \Spec(S_0[D])= \BG_m$ on $\Spec(S) = \BA^{n+1}$ is given by 
\begin{align*}
    (\lambda, (a_0, \ldots, a_n)) \mapsto (\lambda a_0, \ldots, \lambda a_n).
\end{align*}
In particular, this coincides with the classical Proj construction. Also note that each weighted projective space is naturally given by a $D$-graded Proj. If $X = \BP(q_0, \ldots, q_n)$ is a weighted projective space with weights $q_i$, then $S = \BC[x_0, \ldots, x_n]$ with $\deg(x_i) = q_i$ (cf.\  \cite{Cox}, page 3). In particular $X = \Proj^\BN(S)$.
\end{example}






There are surprisingly many gradings yielding a trivial $D$-graded Projectivization (i.e.\ $|\Proj^D(S)| = 1$).



\begin{proposition}\label{prop:Proj_trivial_r=n}       
Let $S = \BC[x_1, \ldots, x_n]$ be $D$-graded such that the degrees of the $x_j$ are pairwise linearly independent and $r:=\rank(D) = n$. Then $\Proj^D(S) = \{(0)\}$.
\end{proposition}

\begin{proof}
Since all the $\deg(x_j)$ are pairwise linearly independent and $\rank(D) = n$, we conclude that relevant elements must have exactly $n$ linearly independent divisors. Hence $S_+ = (x_1 \cdot \ldots \cdot x_n)$, i.e.\ there is only one affine open giving Proj. It holds $S_{(x_1 \cdot \ldots \cdot x_n)} = \BC$ and hence $\Proj^D(S) = \Spec(\BC) = \{(0)\}$. 
\end{proof}

The following Examples give some well-known spaces.

\begin{example}\label{ex:P1_first}
    \begin{enumerate}[label=(\arabic*)]
    \item For $S = \BC[x_0, \ldots, x_{n+1}]$ and $D = \BZ^2$ such that $\deg(x_0) = e_2$ and $\deg(x_i) = e_1$ for $i=1, \ldots, n+1$, we deduce that $\Proj^D(S) = \BP^n$. Again, all relevant elements are strongly relevant.
    

    \item Products of projective spaces can be easily seen as $D$-graded Proj schemes. If $X = \BP^n\times\BP^m$, then for $S = \BC[x_0, \ldots, x_n; y_0, \ldots, y_n]$ and $\deg(x_i) = e_1$ and $\deg(y_j) = e_2$ we see that $\Proj^{\BN\times\BN}(S) = \BP^n\times\BP^m$ (cf.\  \cite{Cox}, page 3).
    \end{enumerate}
\end{example}


We can also show $\Proj^D(S) = (0)$ in Proposition~\ref{prop:Proj_trivial_r=n} by computing $\dim(\Proj^D(S))$. Note that this fact was already stated in \cite{KU}, Lemma 3.6 (2), but the proof given there is not sufficient.


\begin{lemma}\label{lem:dim_proj}
Let $S$ be a noetherian effectively $D$-graded ring (cf.\ Definition~\ref{def:grading_effective}), such that for all relevant $f \in S$, we have $(\bigcup_{m\in \BN}\Ann_S(f^m))\cap S_F = \{0\}$. Then it holds
\begin{align*}
\dim(\Proj^D(S)) \equ \dim(\Spec(S)) - \rank(D)
\end{align*}
In particular, this statement holds if $S$ is integral, or when $\Ann(S_f) = \{0\}$.
\end{lemma}

\begin{proof}
Let $D =F \oplus T$ be the decomposition into free and torsion part of $D$, where $F \cong \BZ^{\rank(D)}$. 
Applying Lemma~\ref{lem:laurent_pol_algebra} to the periodic ring $S_f$ for relevant $f \in S$ immediately gives that $(S_f)_F\ \cong S_{(f)}[F]$. As $S$ is noetherian, $\dim(S_f)$ and $\dim(S_{(f)})$ are finite and we can deduce that $\dim((S_f)_F) = \dim(S_{(f)}) + \rank(D)$. 
By Proposition~\ref{prop:index_integral}, $S_f$ is integral over $(S_f)_F$ and hence $\dim(S_f) = \dim( (S_f)_F)) = \dim(S_{(f)}) + \rank(D)$ by \cite{Bosch}, Proposition 3.3/6, as the inclusion $(S_f)_F \to S_f$ is injective (by the annihilator condition). 
Since $\Proj^D(S)$ is given the the union of all $\Spec(S_{(f)}) = \Spec(S_f) \sslash G$, the claim follows.
\end{proof}









%%%%%%%%%%%%%%%%%%%%%%%%%%%%%%%%%%%%%%%%%%%%%%%%%%%%%%%%%%%%%%%%%%%%%%%%%%%%%%%%%%%%%%%%%%







