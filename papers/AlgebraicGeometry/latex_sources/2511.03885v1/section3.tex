\section{Initial ideals}\label{sec3}
Let $S$ be a fixed iterated sequence for $\Gr\left(3,n\right)$, and assume without losing generality that it is iterated from a PBW sequence. By \cite[Theorem 1]{Bossinger:2021} and the Fundamental Theorem of Tropical Geometry, it is known that the initial ideal $\In_{M_S}\left(\I_{3,n}\right)$ is monomial-free. Since the set of Plücker relations is a (subset of a) Gröbner basis of $\I_{3,n}$ with respect to the weigthing matrix $M_S$, a first step in proving that the initial ideal is not only monomial-free but also binomial, is proving that the initial form of any Plücker relation is binomial.

Let $\V_S$ be the lowest term valuation, $d = 3(n-3)$ and $N=\binom{n}{3}$. Let $M_S\in M_{d\times N}\left(\Z\right)$ be the weigthing matrix $M_S := M_{\V_S}$ given by
\begin{equation}\label{weight-matrix}
    M_S := \begin{pmatrix}
        \V_S(\overline{p_K})^t
    \end{pmatrix}_{K\in I_{3,n}}.
\end{equation}

Let $I\in I_{2,6}, J\in I_{4,6}$ and $R_{I,J}\neq 0$ a Plücker relation. The number of terms of $R_{I,J}$ is
\begin{itemize}
    \item 3, if $I$ and $J$ have a common index;
    \item 4, if $I$ and $J$ have no common index.
\end{itemize}

For all $j\in J$ such that $p_{I\cup j}p_{J\setminus j}\neq 0$, let $\mathbf{m}_j\in \Z^{d}$ be the vector that satisfies $\mathbf{p}^{\mathbf{m}_j}=p_{I\cup j}p_{J\setminus j}$. From the definition of $M_S$, it holds
\begin{equation*}
    M_S\mathbf{m}_j^t = \V_S\left(\overline{p_{I\cup j}}\right) + \V_S\left(\overline{p_{J\setminus j}}\right) = \V_S\left(\overline{p_{I\cup j}p_{J\setminus j}}\right). 
\end{equation*}

The following remark is a consequence of Lemma \ref{homogeneidad}.
\begin{rem}\label{min-max}
    Let $S$ be an iterated sequence $\Gr\left(3,n\right)$. Let $I=(i_1,i_2)\in I_{2,n}, J=(j_1,j_2,j_3,j_4)\in I_{4,n}$ and $R_{I,J}\neq 0$. For all $\mathbf{m}\in \Z^{N}$ such that $\mathbf{p}^{\mathbf{m}}\in \supp(R_{I,J})$, the following equality holds
    \begin{equation*}
        \Psi_S\left(\mathbf{m}\right) = i_1+i_2+j_1+j_2+j_3+j_4 - 12.
    \end{equation*} Therefore
    \begin{equation*}
        \min_{\prec_{\Psi_S}}\left\lbrace M_S\mathbf{m}^t:\p^{\mathbf{m}}\in \supp(R_{I,J})\right\rbrace = \max_{<_{lex}}\left\lbrace  M_S\mathbf{m}^t:\p^{\mathbf{m}}\in \supp(R_{I,J})\right\rbrace.
    \end{equation*}
\end{rem}

In turn, this remark has the following consequence.

\begin{prop}\label{binomial}
    Let $\left(\beta_1,\beta_2,\beta_3\right)$ be a PBW sequence for $\Gr\left(3,4\right)$ and $S$ an iterated sequence for $\Gr\left(3,n\right)$. Let $I\in I_{2,n}, J\in I_{4,n}$ and $R_{I,J}\neq 0$. Then $\In_{M_S}\left(R_{I,J}\right)$ is binomial.
\end{prop}

By the Remark \ref{min-max}, finding the minima with respect to $\prec_{\Psi_S}$ is equivalent to finding the maxima with respect to $<_{lex}$. This fact will be used throughout the proof of Proposition \ref{binomial}, as well as the Algorithm \ref{algorithm} to compute the valuations. The notation used for the Algorithm \ref{algorithm} will be used here as well: for $u\in [3]$,  $f_u\in \Z^3$ is the unitary vector with $1$ in the $u$-th entry.

The proof is now divided in the two cases, corresponding to the number of terms of a Plücker relation.

\begin{proof}[Proof in the case of 3 terms]
    The proof is done by induction, starting with a Plücker relation for $\Gr\left(3,5\right)$. Consider the following notation for the birational sequences:
    \begin{align*}
        \Gr(3,4): \qquad & S_B  = \left(\beta_1,\beta_2,\beta_3\right),\\
        \Gr(3,5): \qquad & S''  = \left(\epsilon_{l_1}-\epsilon_5,\epsilon_{l_2}-\epsilon_5,\epsilon_{l_3}-\epsilon_5,S_B\right), \\
        \Gr(3,n-1): \qquad &S'  = \left(\epsilon_{j_1}-\epsilon_{n-1},\epsilon_{j_2}-\epsilon_{n-1},\epsilon_{j_3}-\epsilon_{n-1}\dots,S''\right),\\
        \Gr(3,n): \qquad  &S  = \left(\epsilon_{h_1}-\epsilon_n,\epsilon_{h_2}-\epsilon_n,\epsilon_{h_3}-\epsilon_n,S'\right).
    \end{align*} 

    \begin{proof}[Base case]
    There are two different cases
        \begin{itemize}
            \item Let $I=(r,5),J=(s_1,s_2,s_3,r)$, with $r,s_1,s_2,s_3\in [4] $ pairwise different.

            Let $t=\min\lbrace t'\in [3]:l_{t'}\neq r\rbrace$. Since $\lbrace r,s_1,s_2,s_3\rbrace=[4]$, then $l_t\in (s_1,s_2,s_3)$. Write $(s_1',s_2')=(s_1,s_2,s_3)\setminus l_t$ and compute the valuations to get
            \begin{align*}
                \V_S\left(\overline{p_{I\cup s_1'}p_{J\setminus s_1'}}\right) &= \left(f_t,\V_{S_B}(\overline{p_{I\cup s_1'\setminus 5\cup l_t}p_{J\setminus s_1'}})\right),\\
                \V_S\left(\overline{p_{I\cup s_2'}p_{J\setminus s_2'}}\right) &= \left(f_t,\V_{S_B}(\overline{p_{I\cup s_2'\setminus 5\cup l_t}p_{J\setminus s_2'}})\right),\\
                \V_S\left(\overline{p_{I\cup l_{t}}p_{J\setminus l_t}}\right) &= \left(f_u,\V_{S_B}(\overline{p_{I\cup l_t\setminus 5\cup l_{u}}p_{J\setminus l_t}})\right),
            \end{align*} with $u\in [3]$ satisfying $u>t$. Furthermore, the following equation holds
            \begin{equation*}
                \overline{p_{(r,5)\cup s_1' \setminus 5 \cup l_t}p_{(s_1,s_2,s_3,r)\setminus s_1'}} = \overline{p_{(r,5)\cup s_2' \setminus 5 \cup l_t}p_{(s_1,s_2,s_3,r)\setminus s_2'}}.
            \end{equation*} Therefore
            \begin{equation*}
                \V_S\left(\overline{p_{I\cup s_1'}p_{J\setminus s_1'}}\right) = \V_S\left(\overline{p_{I\cup s_2'}p_{J\setminus s_2'}}\right)>_{lex}\V_S\left(\overline{p_{I\cup l_{t}}p_{J\setminus l_t}}\right).
            \end{equation*}

            \item Let $I=(4,5)$ and $J=(1,2,3,5)$. Since $l_1,l_2,l_3\in [4]$, there is a term in $R_{I,J}$ of the form $p_{(l_1,l_2,5)}p_{K}$, with $K=(4,5)\cup s$ o $K=(1,2,3,5)\setminus s$ for some $s\in \lbrace 1,2,3\rbrace$. Let $K_1,K_2,L_1,L_2$ be such that
            \begin{equation*}
                p_{(l_1,l_2,5)}p_K,p_{L_1}p_{K_1}, p_{L_2}p_{K_2}\in \supp\left(R_{I,J}\right),
            \end{equation*} and the valuations are 
            \begin{align*}
                \V_S\left(\overline{p_{(l_1,l_2,5)}p_K}\right) &= \left(f_1+f_3,\mathbf{n}'\right)\\
                \V_S\left(\overline{p_{L_1}p_{K_1}}\right) &= \left(f_1+f_2,\mathbf{n}_1\right)\\
                \V_S\left(\overline{p_{L_2}p_{K_2}}\right) &= \left(f_1+f_2,\mathbf{n}_2\right),
            \end{align*} with $\mathbf{n}',\mathbf{n}_1,\mathbf{n}_2\in \Z^{3(n-1-3)}$. Therefore
            \begin{equation*}
                \V_S\left(\overline{p_{L_1}p_{K_1}}\right) = \V_S\left(\overline{p_{L_2}p_{K_2}}\right) >_{lex} \V_S\left(\overline{p_{(l_1,l_2,5)}p_{K_1}}\right).
            \end{equation*}
        \end{itemize}
        This proves the base case.
    \end{proof}

    Assume that for all Plücker relation $R_{I,J}$ for $\Gr\left(k,n-1\right)$, the initial form $\In_{M_S}\left(R_{I,J}\right)$ is binomial, this is, there are $s_1,s_2,s_3\in J$ such that
    \begin{equation*}
        \V_{S'}\left(\overline{p_{I\cup s_1}p_{J\setminus s_1}}\right) = \V_{S'}\left(\overline{p_{I\cup s_2}p_{J\setminus s_2}}\right) >_{lex} \V_{S'}\left(\overline{p_{I\cup s_3}p_{J\setminus s_3}}\right).
    \end{equation*}

    Let $I\in I_{2,n},J\in I_{4,n}$, and $R_{I,J}\neq 0$ a Plücker relation such that $n\in I\setminus J$ or $n\in I\cap J$. For the inductive step, notice that for $n-1\geq 5$, the equality $\lbrace s_1,s_2,s_3,r\rbrace=[n-1]$ does not hold. Consider the two previous cases.
    \begin{itemize}
        \item Let $I=(r,n)$ and $J=(s_1,s_2,s_3,r)$, with $s_1,s_2,s_3,r\neq n$ pairwise different.
        
        Let $t=\min \lbrace t'\in [3]:h_{t'}\neq r\rbrace$. If $h_t\in (s_1,s_2,s_3)$, the same argument used in the base case applies by changing $5$ for $n$. If this is not the case, for $s\in (s_1,s_2,s_3)$, the valuations are
        \begin{align*}
            \V_S\left(\overline{p_{I\cup s}p_{J\setminus s}}\right) & = \left(f_t,\V_{S'}(\overline{p_{I\cup s\setminus n\cup h_t}p_{J\setminus s}})\right).
        \end{align*} The following equality holds
        \begin{equation*}
            \left\lbrace p_{I\cup s\setminus n\cup h_t}p_{J\setminus s} : s=s_1,s_2,s_3\right\rbrace = \supp\left(R_{I\setminus n\cup h_t,J}\right).
        \end{equation*} By induction, $\In_{M_S}\left(R_{I\setminus n\cup h_t,J}\right)$ is binomial. Therefore, $\In_{M_S}\left(R_{I,J}\right)$ is binomial.

        \item Let $I=(r,n)$ and $J=(s_1,s_2,s_3,n)$, with $s_1,s_2,s_3,r\neq n$ pairwise different.

        If $\lbrace h_1,h_2\rbrace\subset \lbrace s_1,s_2,s_3,r \rbrace$, the same argument used in the base case applies by changing $5$ for $n$. Otherwise, use the inductive hypothesis to ``pass to a lower dimension". This argument is exemplified in the case $h_1\not \in \lbrace s_1,s_2,s_3,r\rbrace$: for $s=s_1,s_2,s_3$, the valuations are
        \begin{equation*}
            \V_S\left(2f_1,\V_{S'}(\overline{p_{I\cup s\setminus n\cup h_1}p_{J\setminus s\setminus n\cup h_1}})\right),
        \end{equation*} and the following equality holds
        \begin{equation*}
            \left\lbrace p_{I\cup s\setminus n\cup h_1}p_{J\setminus s\setminus n\cup h_1}:s=s_1,s_2,s_3\right\rbrace = \supp\left(R_{I\setminus n\cup h_1,J\setminus n\cup h_1}\right).
        \end{equation*} By induction, $\In_{M_S}\left(R_{I\setminus n\cup h_1,J\setminus n\cup h_1}\right)$ is binomial. Therefore, $\In_{M_S}\left(R_{I,J}\right)$ is binomial.
    \end{itemize}
\end{proof}

\begin{proof}[Proof in the case of 4 terms]
    The proof is done by induction, starting with a Plücker relation for $\Gr\left(3,6\right)$. Consider the notation $S_B, S'',S',S$ of the previous proof, and add
    \begin{equation*}
        S''' = \left(\epsilon_{b_1}-\epsilon_6,\epsilon_{b_2}-\epsilon_6,\epsilon_{b_3}-\epsilon_6,S''\right),
    \end{equation*} for the iterated sequence for $\Gr\left(3,6\right)$.

    \begin{proof}[Base case]
        Let $I\in I_{2,6}$, $J\in I_{4,6}$, and $R_{I,J}\neq 0$ a Plücker relation with four terms for $\Gr\left(3,6\right)$. Consider the following two cases
        \begin{itemize}
            \item Let $I=(r,6)$ and $J=(s_1,s_2,s_3,s_4)$ with $\lbrace r,s_1,s_2,s_3,s_4\rbrace=[5]$.

            Let $t=\min\lbrace t'\in [3]:b_{t'}\neq r\rbrace$. Then $b_t\in \lbrace s_1,s_2,s_3,s_4\rbrace$. For all  $s'\in \left(s_1,s_2,s_3,s_4\right)\setminus b_t$ the following inequality holds
            \begin{equation*}
                \V_{S'''}\left(\overline{p_{I\cup s'}p_{J\setminus s'}}\right) >_{lex} \V_{S'''}\left(\overline{p_{I\cup b_t}p_{J\setminus b_t}}\right)
            \end{equation*} and
            \begin{equation*}
                \V_{S'''}\left(\overline{p_{I\cup s'}p_{J\setminus s'}}\right) = \left(f_t,\V_{S''}(\overline{p_{I\cup s'\setminus 6\cup b_t}p_{J\setminus s'}})\right).
            \end{equation*} The following equality holds 
            \begin{equation*}
                \left\lbrace p_{I\cup s'\setminus 6\cup b_t}p_{J\setminus s'} : s'\in (s_1,s_2,s_3,s_4)\setminus b_t \right\rbrace = \supp\left(R_{I\setminus 6\cup b_t,J}\right),
            \end{equation*} and by the previous proof, $\In_{M_S}\left(R_{I\setminus 6\cup b_t,J}\right)$ is binomial. Therefore, $\In_{M_S}\left(R_{I,J}\right)$ is binomial.

            \item Let $I=(r_1,r_2)$ and $J=(s_1,s_2,s_3,6)$, with $\lbrace r_1,r_2,s_1,s_2,s_3\rbrace=[5]$.

            Since $b_1\in [5]$, there are two cases: $b_1\in (r_1,r_2)$ or $b_1\in (s_1,s_2,s_3)$. 

            In the first case, for all $s\in (s_1,s_2,s_3)$, it holds
            \begin{equation*}
                \V_{S'''}\left(\overline{p_{I\cup s}p_{J\setminus s}}\right)>_{lex} \V_{S'''}\left(\overline{p_{I\cup 6}p_{J\setminus 6}}\right)
            \end{equation*} and the following equality holds
            \begin{equation*}
                \left\lbrace p_{I\cup s}p_{J\setminus s\setminus 6\cup b_1}:s\in (s_1,s_2,s_3) \right\rbrace = \supp\left(R_{I,J\setminus 6\cup b_1}\right).
            \end{equation*} By the previous proof, $\In_{M_S}\left(R_{I,J\setminus 6\cup b_1}\right)$ is binomial. Therefore, $\In_{M_S}\left(R_{I,J}\right)$ is binomial.

            For the second case, for $s =  b_1,6$, $r\in (s_1,s_2,s_3)\setminus b_1$, it holds
            \begin{equation*}
                \V_{S'''}\left(\overline{p_{I\cup s}p_{J\setminus s}}\right) >_{lex} \V_{S'''}\left(\overline{p_{I\cup r}p_{J\setminus r}}\right).
            \end{equation*} Writing explicitly the left hand side for $s=6,b_1$, the valuations are
            \begin{align*}
                \V_{S'''}\left(\overline{p_{I\cup s}p_{J\setminus s}}\right) &= \V_{S'''}\left(\overline{p_{(r_1,r_2)\cup 6}p_{(s_1,s_2,s_3,6)\setminus 6}}\right) = \left(f_1,\V_{S''}(\overline{p_{(r_1,r_2)\cup b_1}p_{(s_1,s_2,s_3)}})\right)\\
                 \V_{S'''}\left(\overline{p_{I\cup s}p_{J\setminus s}}\right) &= \V_{S'''}\left(\overline{p_{(r_1,r_2)\cup b_1}p_{(s_1,s_2,s_3,6)\setminus b_1}}\right) = \left(f_1,\V_{S''}(\overline{p_{(r_1,r_2)\cup b_1}p_{(s_1,s_2,s_3,6)\setminus b_1\setminus 6\cup b_1}})\right),
            \end{align*} and thus, $\In_{M_S}\left(R_{I,J}\right)$ is binomial.
        \end{itemize}
    \end{proof}

    Assume that for all Plücker relations $R_{I,J}\neq 0$ with four terms for $\Gr\left(3,n-1\right)$, its initial form $\In_{M_S}\left(R_{I,J}\right)$ is binomial, and consider the two previous cases.

    \begin{itemize}
        \item Let $I = \left(r,n\right)$ and $J = \left(s_1,s_2,s_3,s_4\right)$ with $r,s_1,s_2,s_3,s_4\in [n-1]$ pairwise different.

        Let $t=\min \lbrace t':h_{t'}\neq r \rbrace$. If $h_t\in (s_1,s_2,s_3,s_4)$, then the same argument used in the base case applies by changing 5 for $n$. Otherwise, if  $h_t\not\in (s_1,s_2,s_3,s_4)$, then the valuations are 
        \begin{equation*}
            \V_S\left(\overline{p_{I\cup s}p_{J\setminus s}}\right) = \left(f_t,\V_{S'}(\overline{p_{I\cup s\setminus n\cup h_t}p_{J\setminus s}})\right),
        \end{equation*} and the following equality holds
        \begin{equation*}
            \left\lbrace p_{I\cup s \setminus n\cup h_t}p_{J\setminus s}:s\in (s_1,s_2,s_3,s_4) \right\rbrace = \supp\left(R_{I\setminus n\cup h_t,J}\right),
        \end{equation*} with $R_{I\setminus n\cup h_t,J}$ a Plücker relation with 4 terms for $\Gr\left(k,n-1\right)$. By induction, $\In_{M_S}\left(R_{I\setminus n\cup h_t,J}\right)$ is binomial. Therefore, $\In_{M_S}\left(R_{I,J}\right)$ is binomial.

        \item Let $I=(r_1,r_2)$ y $J=(s_1,s_2,s_3,n)$, with $r,s_1,s_2,s_3,s_4\in [n-1]$ pairwise different. 

        If $h_1\in (r_1,r_2,s_1,s_2,s_3)$, the same argument used in the base case applies. If this is not the case, this is, if $h_1\not\in (r_1,r_2,s_1,s_2,s_3)$, then for all $s\in (s_1,s_2,s_3)$, the valuations are 
        \begin{align*}
            \V_S\left(\overline{p_{I\cup n}p_{J\setminus n}}\right) &= \left(f_1,\V_{S'}(\overline{p_{I\cup h_1}p_{J\setminus n}})\right),\\
            \V_S\left(\overline{p_{I\cup s}p_{J\setminus s}}\right) &=\left(f_1,\V_{S'}(\overline{p_{I\cup s}p_{J\setminus s\setminus n\cup h_1}})\right).
        \end{align*}  The following equality holds
        \begin{equation*}
            \left\lbrace p_{I\cup h_1}p_{J\setminus n},p_{I\cup s}p_{J\setminus s\cup n\cup h_1}:s=s_1,s_2,s_3\right\rbrace = \supp\left(R_{I,J\setminus n\cup h_1}\right).
        \end{equation*} By induction, $\In_{M_S}\left(R_{I,J\setminus n\cup h_1}\right)$ is binomial. Therefore, $\In_{M_S}\left(R_{I,J}\right)$ is binomial.
    \end{itemize}
\end{proof}

\subsection*{Initial ideals}
Denote $\mathcal{S}_{3,n}:=\left\lbrace \textrm{iterated sequences for }\Gr(3,6)\right\rbrace$. Let $S\in \mathcal{S}_{3,n}$. By \cite[Theorem 1]{Bossinger:2021}, there exists $C_S\subset \trop\left(\Gr(3,n)\right)$ such that 
\begin{equation}\label{igualdad-de-iniciales}
    \In_{M_S}\left(\I_{3,n}\right) = \In_{C_S}\left(\I_{3,n}\right).
\end{equation}

In order to find $C_S$, the notion of \emph{order preserving projection} is needed: given the weighting matrix $M_S:=\left(\V_S(\overline{p_K})^t\right)_{K\in I_{3,n}}$, a projection $e_S:\Z^{3(n-3)}\rightarrow \Z$ is \emph{order preserving} if
\begin{equation*}
    \In_{M_S}\left(\I_{3,n}\right) = \In_{e_S M_S}\left(\I_{3,n}\right),
\end{equation*} with $e_S M_S := \left(e_S(\V_S(\overline{p_K}))\right)_{K\in I_{3,n}}$. If this is the case, then $w_S := e_SM_S\in \trop\left(\Gr(3,n)\right)$, which in turn defines $C_S$.

The inequalities from the computation of initial forms of Plücker relations will be used to compute these projections. Let $R_{I,J}\neq 0$ ($I\in I_{2,n},J\in I_{4,n}$) be a Plücker relation. A projection $e_S:\Z^{3(n-3)}\rightarrow \Z$ \emph{preserves inequalities} with respect to $S$ if the following condition is satisfied: if $s_1,s_2\in J$ satisfy 
\begin{equation*}
    \V_S\left(\overline{p_{I\cup s_1}p_{J\setminus s_1}}\right) \prec_{\Psi_S} \V_S\left(\overline{p_{I\cup s_2}p_{J\setminus s_2}}\right),
\end{equation*} then
\begin{equation*}
    e_S\left(\V_S\overline{p_{I\cup s_1}p_{J\setminus s_1}})\right) <e_S\left(\V_S(\overline{p_{I\cup s_2}p_{J\setminus s_2}})\right).
\end{equation*}

By the proof of \cite[Lemma 3]{Bossinger_2020}, before concluding that Equation \eqref{igualdad-de-iniciales} holds for the vector obtained only from the inequalities of the initial forms of Plücker relations, it is necessary to verify that the set $\left\lbrace R_{I,J}\neq 0:I\in I_{2,n},J\in I_{4,n}\right\rbrace$ is a reduced Gröbner basis for $\I_{3,6}$ (with respect to $M_S$). If this condition holds, the following two Propositions show that the total number of weight vectors $w_S\in \trop\left(\Gr(3,n)\right)$ reduces significantly, thus reducing the total number of degenerations to study.

\begin{prop}\label{perm-equality}
    Let $(\beta_1,\beta_2,\beta_3)$ be a PBW sequence for $\Gr\left(3,4\right)$ and $\sigma\in S_3$. Consider the following iterated sequences for $\Gr\left(3,n\right)$
    \begin{align*}
        S & := \left(\epsilon_{i_1}-\epsilon_{n}, \epsilon_{i_2}-\epsilon_n,\epsilon_{i_3}-\epsilon_n,\dots,\beta_1,\beta_2,\beta_3\right)\\
        S_{\sigma} & := \left(\epsilon_{i_1}-\epsilon_n,\epsilon_{i_2}-\epsilon_n,\epsilon_{i_3}-\epsilon_n,\dots,\beta_{\sigma(1)},\beta_{\sigma(2)},\beta_{\sigma(3)}\right).
    \end{align*}
    
    Given  $e_S$ a projection that preserves inequalities with respect to $S$, there exists a projection $e_{S_{\sigma}}$ that preserves inequalities with respect to $S_{\sigma}$ and satisfies
    \begin{equation*}
        e_{S_{\sigma}}M_{S_{\sigma}} = e_{S}M_S.
    \end{equation*}
\end{prop}

The following remark from the proof of Proposition \ref{binomial} will be needed.

\begin{rem}\label{obs1}
    Let $R_{I,J}\neq 0$ be a Plücker relation and $s_1,s_2\in J$ be such that
    \begin{equation*}
        \V_S\left(\overline{p_{I\cup s_1}p_{J\setminus s_1}}\right)\prec_{\Psi_S}\V_S\left(\overline{p_{I\cup s_2}p_{J\setminus s_2}}\right).
    \end{equation*} By the proof of Proposition \ref{binomial}, the comparison of these two elements is firstly done in the first three entries; if these entries coincide, pass to the next triad; inductively, repeat until reaching the second-to-last triad, where the tie must break. 
\end{rem}

\begin{proof}[Proof of Proposition \ref{perm-equality}]
    Let $S$ and $S_{\sigma}$ be the two iterated sequences and $\sigma\in S_3$. Write the projection $e_S:\Z^{3(n-3)}\rightarrow \Z$ as follows
    \begin{equation*}
        e_S = \left(e_1, \dots, e_{3(n-1-3)},e_{3(n-1-3) + 1},e_{3(n-1-3) + 2},e_{3(n-1-3) + 3}\right).
    \end{equation*} Let $I\in I_{3,n}$ be a multiindex, $\overline{p_I}$ a Plücker coordinate, and $\mathbf{m}\in \Z^{3(n-1-3)}, \mathbf{n}\in \Z^3$ such that 
    \begin{equation*}
        \V_S\left(\overline{p_I}\right) = \left(\mathbf{m},\mathbf{n}\right).
    \end{equation*} Then
    \begin{equation*}
        \V_{S_{\sigma}}\left(\overline{p_I}\right) = \left(\mathbf{m},\sigma\mathbf{n}\right),
    \end{equation*} with $\sigma\mathbf{n}$ is obtained by permuting the entries of $\mathbf{n}$ using $\sigma$.

    Define $e_{S_{\sigma}}:\Z^{3(n-3)}\rightarrow \Z$ as
    \begin{equation*}
        e_{S_{\sigma}} := \left(e_1,\dots,e_{3(n-1-3)}, e_{3(n-1-3)+\sigma(1)},e_{3(n-1-3)+\sigma(2)},e_{3(n-1-3)+\sigma(3)}\right).
    \end{equation*} Notice that for all $I\in I_{3,n}$, the following equality holds:
    \begin{equation*}
        e_S\left(\V_S(\overline{p_I})\right) = e_{S_{\sigma}}\left(\V_{S_{\sigma}}(\overline{p_I})\right).
    \end{equation*} Now, for a Plücker relation $R_{I,J}\neq 0$ and $s_1,s_2\in J$ satisfying
    \begin{equation*}
        \V_S\left(\overline{p_{I\cup s_1}p_{J\setminus s_1}}\right) \prec_{\Psi_S} \V_S\left(\overline{p_{I\cup s_2}p_{J\setminus s_2}}\right),
    \end{equation*} by the Remark \ref{obs1} and the computation of valuations $\V_{S_{\sigma}}$, the following inequality holds
    \begin{equation*}
        \V_{S_{\sigma}}\left(\overline{p_{I\cup s_1}p_{J\setminus s_1}}\right) \prec_{\Psi_{S_{\sigma}}} \V_{S_{\sigma}}\left(\overline{p_{I\cup s_2}p_{J\setminus s_2}}\right).
    \end{equation*} Since $e_S$ and $e_{S_{\sigma}}$ are $\Z$-linear, the following inequality is true
    \begin{equation*}
        e_{S_{\sigma}} \left(\V_{S_{\sigma}}(\overline{p_{I\cup s_1}p_{J\setminus s_1}})\right) = e_S\left(\V_S(\overline{p_{I\cup s_1}p_{J\setminus s_1}})\right) < e_S\left(\V_S(\overline{p_{I\cup s_2}p_{J\setminus s_2}})\right) = e_{S_{\sigma}}\left(\V_{S_{\sigma}}(\overline{p_{I\cup s_2}p_{J\setminus s_2}})\right),
    \end{equation*} which proves that $e_{S_{\sigma}}$ preserves inequalities with respect to $S_{\sigma}$.
\end{proof}

At this point, recall \cite[Proposition 1]{Bossinger:2021}: given an iterated sequence $S$ for $\Gr\left(2,n\right)$, there exists a trivalent tree $T_S$ that satisfies $\In_{M_S}\left(\I_{2,n}\right) = \In_{T_S}\left(\I_{2,n}\right)$. This tree is the output of \cite[Algorithm 1]{Bossinger:2021}, which only depends on the first index of each iteration. Thus, it is natural to wonder if this the case for an iterated sequence for $\Gr\left(3,n\right)$. The answer is positive, and is the content of the next Proposition.

\begin{prop}\label{change-equality}
    Let $S'$ be an iterated sequence for $\Gr\left(3,n-1\right)$. Let $i_1,i_2\in [n-1]$ and $i,j\in [n-1]\setminus \lbrace i_1,i_2\rbrace$. Consider the following iterated sequences for $\Gr\left(3,n\right)$
    \begin{align*}
        S_1 & = \left(\epsilon_{i_1}-\epsilon_n,\epsilon_{i_2}-\epsilon_n,\epsilon_{i}-\epsilon_n,S'\right)\\
        S_2 & = \left(\epsilon_{i_1}-\epsilon_n,\epsilon_{i_2}-\epsilon_n,\epsilon_j-\epsilon_n,S'\right).
    \end{align*} 
    
    Given a projection $e_{S_1}$ that preserves inequalities with respect to $S_1$, there exists a projection $e_{S_2}$ that preserves inequalities with respect to $S_{2}$ and satisfies
    \begin{equation*}
        e_{S_1}M_{S_1} = e_{S_2}M_{S_2}.
    \end{equation*}
\end{prop}

There is a remark analogous to Remark \ref{obs1}.

\begin{rem}\label{obs2}
    Let $S=\left(\epsilon_{i_1}-\epsilon_n,\epsilon_{i_2}-\epsilon_n,\epsilon_{i_3}-\epsilon_n,\dots\right)$ be an iterated sequence for $\Gr\left(3,n\right)$, $R_{I,J}\neq 0$ a Plücker relation with $n\in I$ or $n\in J$, and $s_1,s_2\in J$ such that
    \begin{equation*}
        \V_S\left(\overline{p_{I\cup s_1}p_{J\setminus s_1}}\right) \prec_{\Psi_S} \V_S\left(\overline{p_{I\cup s_2}p_{J\setminus s_2}}\right).
    \end{equation*}
    By the proof of Proposition \ref{binomial}, the comparison of  these two elements is first done in the first two entries of the first triad; if they coincide, pass to the first two elements of the second triad; inductively, repeat until reaching the second-to-last triad where the tie must break.
\end{rem}

\begin{proof}[Proof of Proposition \ref{change-equality}]
    Let $S_1$ and $S_2$ be two iterated sequences for $\Gr\left(3,n\right)$. Write the projection $e_{S_1}:\Z^{3(n-3)}\rightarrow \Z$ as follows
    \begin{equation*}
        e_{S_1} = \left(e_1,e_2,e_3,\dots,e_{3(n-3)}\right).
    \end{equation*}
    Let $I=(i_1,i_2,n)\in I_{3,n}$ and $\overline{p_I}$ the Plücker coordinate. The valuations of $\overline{p_I}$ are
    \begin{align*}
        \V_{S_1}\left(\overline{p_I}\right) & = \left(f_3,\V_{S'}(\overline{p_{I\setminus n\cup i}})\right),\\
        \V_{S_2}\left(\overline{p_I}\right) & = \left(f_3,\V_{S'}(\overline{p_{I\setminus n\cup j}})\right).
    \end{align*} Notice that this is the only Plücker coordinate that satisfies
    \begin{equation*}
        \V_{S_1}\left(\overline{p_I}\right) \neq \V_{S_2}\left(\overline{p_I}\right).
    \end{equation*} Define $e_{S_2}:\Z^{3(n-3)}\rightarrow \Z$ as
    \begin{equation*}
        e_{S_2} :=\left(e_1,e_2,e_{S_1}(\V_{S_1}(\overline{p_I})-\V_{S_2}(\overline{p_{I\setminus n\cup j}})),\dots,e_{3(n-3)}\right).
    \end{equation*} Notice that for $I\in I_{3,n}$ with $I\neq (i_1,i_2,n)$, the equality holds
    \begin{equation*}
        e_{S_2}\left(\V_{S_2}(\overline{p_I})\right) = e_{S_1}\left(\V_{S_1}(\overline{p_I})\right).
    \end{equation*} For $I=(i_1,i_2,n)$, the equality holds as well:
    \begin{equation*}
        e_{S_2}\left(\V_{S_2}(\overline{p_I})\right) = e_{S_1}\left(\V_{S_1}(\overline{p_I})\right) - e_{S_2}\left(\V_{S_2}(\overline{p_{I\setminus n\cup j}})\right) + e_{S_2}\left(\V_{S_2}(\overline{p_{I\setminus n\cup j}})\right) = e_{S_1}\left(\V_{S_1}(\overline{p_I})\right).
    \end{equation*}
    Now, given a Plücker relation $R_{I,J}\neq 0$ with $s_1,s_2\in J$ such that
    \begin{equation*}
        \V_{S_1}\left(\overline{p_{I\cup s_1}p_{J\setminus s_1}}\right) \prec_{\Psi_{S_1}} \V_{S_1}\left(\overline{p_{I\cup s_2}p_{J\setminus s_2}}\right),
    \end{equation*} by Remark \ref{obs2} and the computation of the valuations, the following inequality holds
    \begin{equation*}
        \V_{S_2}\left(\overline{p_{I\cup s_1}p_{J\setminus s_1}}\right) \prec_{\Psi_{S_2}} \V_{S_2}\left(\overline{p_{I\cup s_2}p_{J\setminus s_2}}\right).
    \end{equation*} Since $e_{S_1}$ and $e_{S_2}$ are $\Z$-linear, the following inequality is satisfied
    \begin{equation*}
        e_{S_1}\left(\V_{S_1}(\overline{p_{I\cup s_1}p_{J\setminus s_1}})\right)  = e_{S_2}\left(\V_{S_2}(\overline{p_{I\cup s_1}p_{J\setminus s_1}})\right) < e_{S_1}\left(\V_{S_1}(\overline{p_{I\cup s_2}p_{J\setminus s_2}})\right) = e_{S_2}\left(\V_{S_2}(\overline{p_{I\cup s_2}p_{J\setminus s_2}})\right),
    \end{equation*} which proves that $e_{S_2}$ preserves inequatilies with respect to $S_2$.
\end{proof}

Consider from now on that every iterated sequence for $\Gr\left(3,n\right)$ is iterated from a PBW sequence for $\Gr\left(3,4\right)$. A direct computation shows that
\begin{equation*}
    \# \mathcal{S}_{3,n} = \prod_{l=0}^{n-4} \left(n-l-1\right)\left(n-l-2\right)\left(n-l-3\right).
\end{equation*}

Let $S_1 ,S_2\in \mathcal{S}_{3,n}$ and assume that the first two indices of every iteration coincide. Given a projection $e_{S_1}$ that preserves inequalities with respect to $S_1$, by Proposition \ref{change-equality} there exists a projection $e_{S_2}$ that preserves inequalities with respect to $S_2$ and satisfies
\begin{equation*}
    e_{S_1}M_{S_1} = e_{S_2}M_{S_2}.
\end{equation*} Furthermore, if $S_1$ is iterated from the PBW sequence $(\beta_1,\beta_2,\beta_3)$ and $S_2$ is iterated from a PBW sequence $(\beta_{\sigma(1)},\beta_{\sigma(2)},\beta_{\sigma(3)})$ with $\sigma\in S_3$, then by Proposition \ref{perm-equality}, the same equality holds. Thus, and by the proof of \cite[Lemma 3]{Bossinger_2020}, the following proposition is proved.

\begin{prop}\label{number_of_vectors}
    If $\left\lbrace R_{I,J}\neq 0:I\in I_{2,n},J\in I_{4,n} \right\rbrace$ is a reduced Gröbner basis of $\I_{3,n}$ with respect to $M_{S}$ for all $S\in \mathcal{S}_{3,n}$, then
    \begin{equation*}
        \#\left\lbrace \In_{M_S}(\I_{3,n}):S\in \mathcal{S}_{3,n}\right\rbrace \leq \prod_{l=0}^{n-5}\left(n-l-1\right)\left(n-l-2\right).
    \end{equation*}
\end{prop}

\subsection*{The Grassmannian \texorpdfstring{$\Gr(3,6)$}{Gr(3,6)}}
Let $S \in \mathcal{S}_{3,6}$. Let $R_{I,J}\neq 0$ be any Plücker relation and $e_S:\Z^{9}\rightarrow \Z$ a projection that preserves inequalities with respect to $S$. This means that for $s_1,s_2\in J$ it satisfies
\begin{equation*}
    \V_S\left(\overline{p_{I\cup s_1}p_{J\setminus s_1}}\right) \prec_{\Psi_S} \V_S\left(\overline{p_{I\cup s_2}p_{J\setminus s_2}}\right),
\end{equation*} then 
\begin{equation}\label{inequalities}
    e_S\left(\V_S(\overline{p_{I\cup s_1}p_{J\setminus s_1}})\right) < e_S\left(\V_S(\overline{p_{I\cup s_2}p_{J\setminus s_2}})\right).
\end{equation} Regarding the projection $e_S:\Z^9\rightarrow \Z$ as an element $e_S\in \Z^9$, the inequalities of the form \eqref{inequalities} define a polyhedral cone in $\Z^9$ (cf. \cite[Chapter 2]{sturmfels1996grobner}). By implementing Algorithm \ref{algorithm}, all the different inequalities obtained from the computation of initial forms of Plücker relations are calculated. In turn, by using \verb|polymake| (cf. \cite{polymake:2000}), this cone is computed and a vector in its relative interior is a projection $e_S$ that preserves inequalities with respect to $S$. For all projections $e_S$, define as before $w_S:=e_S M_S$.

Using the computations from the previous section, the total number of iterated sequences for $\Gr\left(3,6\right)$ is $\# \mathcal{S}_{3,6}=8640$. Repeating the process above for all different iterated sequences for $\Gr\left(3,6\right)$ yields the total number of weight vectors
\begin{equation}
    \#\left\lbrace w_S\in \R^{20} :S\in \mathcal{S}_{3,6} \right\rbrace = 240,
\end{equation} which was expected by the Proposition \ref{number_of_vectors}.

 The initial ideals $\In_{w_S}\left(\I_{3,6}\right)$ are computed using \verb|macaulay2| (cf. \cite{M2}), and the following facts are obtained
\begin{itemize}
    \item For all $S\in \mathcal{S}_{3,6}$, the minimum number of generators of $\In_{w_S}\left(\I_{3,6}\right)$ coincides with the minimum number of generators of $\I_{3,6}$. Thus, the set of Plücker relations is a reduced Gröbner basis for $\I_{3,6}$ with respect to $M_S$ and the next equality holds
    \begin{equation*}
        \In_{w_S}\left(\I_{3,6}\right) = \In_{M_S}\left(\I_{3,6}\right).
    \end{equation*}
    \item For all $S\in \mathcal{S}_{3,6}$, the ideal $\In_{w_S}\left(\I_{3,6}\right)$ is binomial and prime. By \cite[Lemma 1.1]{sturmfels1996equationsdefiningtoricvarieties}, $\In_{w_S}\left(\I_{3,6}\right)$ is toric, and thus defines a toric variety.
    \item If $S_1$ and $S_2$ are two iterated sequences that do not satisfy the hypotheses of Proposition \ref{change-equality} (this is, the first or second or both indices of an iteration differ), then 
    \begin{equation*}
        \In_{M_{S_2}}\left(\I_{3,6}\right) \neq \In_{M_{S_2}}\left(\I_{3,6}\right).
    \end{equation*} 
\end{itemize}

This proves the following Theorem.

\begin{thm}[First Classification]\label{prim-class}
    The assignment $\mathcal{S}_{3,6}\rightarrow \left\lbrace \textrm{initial ideals of }\I_{3,6} \right\rbrace$ given by
    \begin{equation*}
        S\mapsto \In_{M_S}\left(\I_{3,6}\right)
    \end{equation*} is an assignment
    \begin{equation*}
        \mathcal{S}_{3,6} \rightarrow \left\lbrace \textrm{toric initial ideals of }\I_{3,6} \right\rbrace
    \end{equation*} and the image has cardinality $\#\left\lbrace \In_{M_S}(\I_{3,6}):S\in \mathcal{S}_{3,6}\right\rbrace = 240$.
\end{thm}

A direct consequence of this Theorem is related to the \emph{value semi-group} $S\left(A_{3,6},\V_S\right)$ (cf. e.g. \cite[$\S 2.1$]{Bossinger:2021}).

\begin{cor}
    For all $S\in \mathcal{S}_{3,6}$, the set $\left\lbrace \overline{p_I}:I\in I_{3,6}\right\rbrace$ forms a Khovanskii basis of $\left(A_{3,6},\V_S\right)$.
\end{cor}
\begin{proof}
    Let $S\in \mathcal{S}_{3,6}$. By the proof of \cite[Theorem 1]{Bossinger:2021}, the valuation $\V_S:A_{3,6}\setminus \lbrace 0\rbrace \rightarrow \left(\Z^{9},\preceq_{\Psi_S}\right)$ is a full-rank valuation. By Theorem \ref{prim-class}, the initial ideal $\In_{M_S}\left(\I_{3,6}\right)$ is prime. By \cite[Theorem 1]{Bossinger_2020}, the value semi-group $S\left(A_{3,6},\V_S\right)$ is generated by $\left\lbrace\V_S( \overline{p_I}):I\in I_{3,6}\right\rbrace$.
\end{proof}

By Theorem \ref{prim-class}, the notation can be simplified: consider an iterated sequence 
\begin{equation*}
    S = \left(\epsilon_{i_1}-\epsilon_6,\epsilon_{i_2}-\epsilon_6,\epsilon_{i_3}-\epsilon_6,\epsilon_{j_1}-\epsilon_5,\epsilon_{j_2}-\epsilon_5,\epsilon_{j_3}-\epsilon_5,\beta_1,\beta_2,\beta_3\right).
\end{equation*} Since the initial ideal $\In_{M_S}\left(\I_{3,6}\right)$ only depends on the first two indices of each iteration, the initial ideal will be written 
\begin{equation}
    I_{(i_1,i_2;j_1,j_2)} := \In_{M_S}\left(\I_{3,6}\right).
\end{equation}

The next step in the classification of the initial ideals considers the action of a subgroup of $\operatorname{Aut}\left(\I_{3,6}\right)$. Consider simple transpositions $s_i=\begin{pmatrix} i & i+1\end{pmatrix}\in S_6$. The action of $s_i$ on $[6]$ can be extended to $I_{3,6}$ as $s_i.(i_1,i_2,i_3)= (s_i(i_1), s_i(i_2), s_i(i_3))$. They can further be extended to Plücker variables as
\begin{equation*}
    s_i.p_{(i_1,i_2,i_3)} = \left\lbrace \begin{matrix} p_{s_i.(i_1,i_2,i_3)} & \textrm{if } (i_1,i_2) \neq (i,i+1) \textrm{ and } (i_2,i_3) \neq (i,i+1)\\
    -p_{(i_1,i_2,i_3)} &  \textrm{else}
    \end{matrix}\right. ,
\end{equation*} and finally extended as an isomorphism of $\I_{3,6}$. Consider the following subgroup of $\operatorname{Aut}\left(\I_{3,6}\right)$
\begin{equation*}
    G:=\left\langle s_i:i\in [5] \right\rangle_{\operatorname{Aut}\left(\I_{3,6}\right)}.
\end{equation*} Under the action of this group, two ideals $\In_{M_{S_1}}\left(\I_{3,6}\right)$ and $\In_{M_{S_2}}\left(\I_{3,6}\right)$ are equivalent if there exists $g\in G$ such that $\In_{M_{S_1}}\left(\I_{3,6}\right) = g\left(\In_{M_{S_1}}(\I_{3,6})\right)$; an equivalence class is called a $G$-orbit. Notice that this means that two ideals in the same $G$-orbit are isomorphic, and thus, the toric varieties they define are isomorphic. These isomorphisms are written in \verb|macaulay2| to compute the $G$-orbits and they yield the following classification up to the $G$-action.

\begin{thm}[Partial Classification]\label{second_class}
    Consider $\mathbb{I}_{3,6} := \lbrace \In_{M_S}(\I_{3,6}):S\in \mathcal{S}_{3,6}\rbrace$. The $G$-orbits of $\mathbb{I}_{3,6}$ intersected with $\mathbb{I}_{3,6}$ are
    \begin{align*}
    O_{1}\cap \mathbb{I}_{3,6} &:= \left\lbrace I_{(s_1,s_2;r,k)} : k\in [4]  \wedge (\lbrace s_1,s_2\rbrace = \lbrace r,5 \rbrace \textrm{ or } \lbrace s_1,s_2\rbrace = [5]\setminus \lbrace r,k,M_k  \rbrace ) \right\rbrace,\\
    O_{2} \cap \mathbb{I}_{3,6}&:= \left\lbrace I_{(k,s_1;s_2,k)} :k\in [4], s_1 \in [5]\setminus \lbrace k\rbrace \, \wedge\, s_2 \in [4]\setminus \lbrace k\rbrace  \right\rbrace,\\
    O_{3} \cap \mathbb{I}_{3,6}&:= \left\lbrace I_{(s_1,k;s_2,k)} : s_1 \in [5]\setminus \lbrace k\rbrace \, \wedge\, s_2 \in [4]\setminus \lbrace k\rbrace  \right\rbrace,\\
    O_4 \cap \mathbb{I}_{3,6} &:= \mathbb{I}_{3,6} \setminus \left(\bigcup_{i=1,2,3} O_i\right).
\end{align*} These intersections have the following cardinalities and are listed with its corresponding isomorphism class as described in \cite[$\S5$]{speyer2003tropicalgrassmannian}
\begin{table}[H]
    \centering
    \begin{tabular}{|c|c|c|}
        \hline
        Orbit & $\#$ & Isomorphism class \\
        \hline
        $O_1\cap\mathbb{I}_{3,6}$ & $48$ & $ EEFF1 $\\
        \hline
        $O_2\cap\mathbb{I}_{3,6}$ & $48$ & $EFFG$ \\
        \hline 
        $O_3\cap\mathbb{I}_{3,6}$ & $48$ & $EEFF2$\\
        \hline
        $O_4\cap\mathbb{I}_{3,6}$ & $96$ & $EEFG$\\
        \hline
    \end{tabular}
    \caption{Classification of the $G$-orbits in $\mathbb{I}_{3,6}$}
    \label{isom-class}
\end{table}
\end{thm}

\begin{rem}
    There exist initial ideals $\In_{M_S}\left(\I_{3,6}\right)$ whose image under one of the simple transpositions is not contained in $\mathbb{I}_{3,6}$. Therefore, the intersections $O\cap \mathbb{I}_{3,6}$ in the previous theorem are necessary. 
\end{rem}

\subsection*{Toric degenerations}
The main results of the previous sections can be briefly restated as follows: for every iterated sequence $S\in \mathcal{S}_{3,6}$, the ideal $\In_{M_S}\left(\I_{3,6}\right)$ is toric and up to strict equality, there are 240 different toric initial ideals labeled by the first two indices of each iteration (Theorem \ref{prim-class}); up to the action of $S_6\leq \operatorname{Aut}\left(\I_{3,6}\right)$, there are four different orbits (Theorem \ref{second_class}); equivalently, the set 
\begin{equation*}
    \mathbb{V}_{3,6} := \left\lbrace \operatorname{Spec}(\C[p_K:K\in I_{3,6}]/\In_{M_S}(\I_{3,6})):S\in \mathcal{S}_{3,6}\right\rbrace
\end{equation*}    
contains, up to isomorphism induced by $S_6$, four different toric varieties, classified according to Theorem \ref{second_class}. As was mentioned in the Introduction, one of the main reasons to study birational sequences and, in particular, iterated sequences, is to construct toric degenerations arising as Gröbner degenerations. The description of this construction can be found, for example, in \cite{Bossinger:2021,bossinger2023surveytoricdegenerationsprojective}. It will be briefly sketched here in the case of the Grassmannian $\Gr\left(3,6\right)$.

Let $w\in \trop\left(\Gr(3,6)\right)$ be an arbitrary weight vector and $\I_{3,6}$ the Plücker ideal.  For $t\in \C$, consider the following family of ideals
\begin{equation*}
    \Tilde{I}_t := \left\langle   t^{-\min_{u}\lbrace u\cdot w \rbrace}f\left(t^{w_{I_1}}p_{I_1},\dots,t^{w_{I_N}}p_{I_N}\right): f=\sum a_u p^u\in \I_{3,6}\right\rangle \subset \C\left[t,p_K^{\pm 1}:K\in I_{3,6}\right].
\end{equation*}  This describes a flat family over $\C$ (cf. \cite[Section 15.8]{Eisenbud:CommAlg}). Consider the following three varieties
\begin{equation*}
    \operatorname{Spec}\left(A_{3,6}\right), \quad \operatorname{Spec}\left(\C\left[t,p_K:K\in I_{3,6}\right]/\Tilde{I}_t\right),\quad \operatorname{Spec}\left(\C[p_K:K\in I_{3,6}]/\In_w(\I_{3,6})\right).
\end{equation*} Write $I_s := \left.\Tilde{I}_t\right|_{t=s}$. If $s\neq 0$, there is an automorphism of $\C\left[p_K:K\in I_{3,6}\right]$ sending $I_s$ to $\I_{3,6}$. Then $\operatorname{Spec}\left(\C\left[t,p_K:K\in I_{3,6}\right]/\Tilde{I}_t\right)$ defines a degeneration of $\operatorname{Spec}(A_{3,6})$, called the \emph{Gröbner degeneration}. If the ideal $\In_w\left(\I_{3,6}\right)$ is toric, then the degeneration is called \emph{toric}.

The consequence of Theorems \ref{prim-class} and \ref{second_class} is the following corollary.

\begin{cor}\label{class_toric}
    Every iterated sequence $S\in \mathcal{S}_{3,6}$ induces a toric degeneration of $\Gr\left(3,6\right)$. These degenerations are, up to the action of $S_6\leq \operatorname{Aut}\left(\I_{3,6}\right)$, classified according to the classification in Theorem \ref{second_class}.
\end{cor}

This paper concludes by noticing that the equivalences presented in \cite[Remark 1]{Bossinger:2021} for $\Gr\left(2,n\right)$ do not generalize when considering the Grassmannian $\Gr\left(3,6\right)$. By \cite[$\S 5$]{speyer2003tropicalgrassmannian}, there are 7 isomorphism classes of maximal cones in $\trop\left(\Gr(3,6)\right)$. By Theorem \ref{second_class}, only four of these classes ($EEFF1,EEFF2,EFFG,EEFG$) correspond to initial ideals induced by iterated sequences, so the function $\In_{M_S}\left(\I_{3,6}\right)\mapsto \In_{w_S}\left(\I_{3,6}\right)$ is injective but cannot be surjective. This is, there is no ``natural" arrow pointing left in the next diagram
\begin{equation*}
    \left\lbrace \begin{matrix}\textrm{toric degenerations of }\Gr(3,6)\\\textrm{induced by iterated sequences}\end{matrix}\right\rbrace \rightarrow \left\lbrace \begin{matrix}\textrm{toric degenerations of }\Gr(3,6)\\\textrm{induced by }\trop(\Gr(3,6))\end{matrix}\right\rbrace.
\end{equation*}   Furthermore, the classes $EEFF1,EEFF2,EFFG,EEFG,EEEG$, together with a class corresponding to an edge of the form $GG$, correspond to degenerations induced by plabic graphs (cf. \cite[Table 1]{bossinger2016toricdegenerationsgr2ngr36}), so there is an injective function 
\begin{equation*}
    \left\lbrace \begin{matrix}\textrm{toric degenerations of }\Gr(3,6)\\\textrm{induced by iterated sequences}\end{matrix}\right\rbrace \rightarrow \left\lbrace \begin{matrix}\textrm{toric degenerations of }\Gr(3,6)\\\textrm{induced by plabic graphs}\end{matrix}\right\rbrace.
\end{equation*} which cannot be surjective since there is no initial ideal $\In_{M_S}\left(\I_{3,n}\right)$ corresponding to the class $EEEG$ or to the class corresponding to an edge of the form $GG$.

