\newpage

%------------------------------------------------------------------
\section*{Introduction}\label{sec:intro}
\addcontentsline{toc}{section}{Introduction}
%------------------------------------------------------------------

\paragraph{Goal.}
We study the inequality
\[
   \bigl\|F\bigr\|_{L^{6}(Q_{\lambda})}
   \;\lesssim\;
   \lambda^{\Sigma_{\lambda}}\,
   D^{\Sigma_{D}}\,
   \Bigl(
        \sum_{\Theta}\|F_{\Theta}\|_{L^{6}}^{2}
   \Bigr)^{1/2},
   \qquad \lambda\to\infty ,
\]
for functions with the spectral decomposition
\(F=\sum_{\Theta}F_{\Theta}\)
over caps of angular radius \(r=\lambda^{-2/3}\).

The resulting exponents are
\(\Sigma_\lambda=-\frac{2557}{576}\approx -4.44<0\), \(\Sigma_D=-3<0\),
so that the traditional factors \(\lambda^\varepsilon\) and \(D^\varepsilon\) are absent in the final estimate.

The working region is the cylinder \(Q_{\lambda}\subset\mathbb{R}_{t}\times\mathbb{R}^{3}_{x}\)
from~\eqref{eq:def-cylinder}, with additional parameters
\(D=\lambda^{1/12}\) and
\(\alpha=c_{0}\,rD^{1/2}=c_{0}\lambda^{-5/8}\)
fixed in~\S\ref{sec:notation}.

\paragraph{Structure of the proof.}
\begin{enumerate}\setlength{\itemsep}{4pt}
\item
\emph{Preliminaries}
(\S\ref{sec:notation}):
introduction of the scales \(r,\rho,\alpha\) and conventions on estimates.
\item
\emph{Broad rank-3 geometry}
(\S\ref{sec:broad3}):
bilipschitz behavior of the map \(\xi\mapsto n(\xi)\)
guarantees the existence of a triple with
\(
   \|n_{i}\wedge n_{j}\wedge n_{k}\|
      \gtrsim\lambda^{-5/4}
\);
insertion into the trilinear Bennett–Carbery–Tao inequality
yields a contribution \(+\tfrac{5}{36}\) in \(\lambda\).
\item
\emph{Kernel estimate}
(\S\ref{sec:kernel}):
twelve integrations (6 in \(t\), 6 in \(x'\)) plus Schur/\(TT^{*}\)
   give \(\|K\|_{L^2\to L^2}\lesssim \lambda^{-9/2}D^{-3}\).
\item
\emph{Remaining blocks.}
  \begin{itemize}\setlength{\itemsep}{3pt}
  \item
  \emph{Robust Kakeya} (\S\ref{sec:kakeya}):  
  if the density is \(>c_{*}D\) we gain a factor \(D\),
  contributing \(+\tfrac{1}{12}\) in \(\lambda\) and \(+1\) in \(D\).
  \item
  \emph{Tube packing} (\S\ref{sec:tube-pack}):  
  presented for context; its contribution is \emph{not} used together with Robust Kakeya
  (the regimes are mutually exclusive; see \S\ref{sec:balance}).
  \item
  \emph{Algebraic “shell”} (\S\ref{sec:alg-skin}):  
  cutting off the layer \(N_{\beta}(P)\) contributes
  \(-\tfrac{1}{12}\) in \(\lambda\) and \(-1\) in \(D\).
  \item
  \emph{Narrow cascade} (\S\ref{sec:narrow}):  
  a double \(7/8\) rescaling removes the flow from the narrow regime
  and contributes a negative term \(-\tfrac{5}{64}\) in \(\lambda\) (zero in \(D\)).
  \end{itemize}
\item
\emph{Summary of exponents.}
\[
\Sigma_\lambda=\frac{5}{36}-\frac{9}{2}+\frac{1}{12}-\frac{1}{12}-\frac{5}{64}
=-\frac{2557}{576}\approx-4.44,\qquad
\Sigma_D=(-3)+1-1=-3.
\]
The negativity of both sums removes the traditional \(\lambda^\varepsilon\)- and \(D^\varepsilon\)-losses.
\end{enumerate}

\paragraph{Navigation.}
\begin{itemize}\setlength{\itemsep}{2pt}
\item
\S\ref{sec:notation} — parameters and notation;
\item
\S\ref{sec:broad3}--\ref{sec:narrow} — thematic blocks;
\item
\S\ref{sec:balance} — summary table and final estimate;
\item
Appendices A–D — technical lemmas, checkpoints,
and clarifications (Appendix D shows robustness with respect to \(x\)-dependence of the amplitude).
\end{itemize}

\paragraph{Notation.}
We write \(A\lesssim B\) for \(A\le C\,B\) with an absolute constant~\(C\);
notation \(A\lesssim_{\delta}B\) allows dependence \(C=C(\delta)\).
Fix a single \(\varepsilon\in(0,10^{-2}]\) and use the standard log-budget:
for any fixed \(k\) and all sufficiently large \(\lambda\)
\[
\log^{k}\!\lambda \ \le\ C_{\varepsilon}\,\lambda^{\varepsilon},\qquad
\log^{k}\!D \ \le\ C_{\varepsilon}\,D^{\varepsilon}\quad(D=\lambda^{1/12}).
\]
In all counts of \(\sigma_\lambda,\sigma_D\) we verify that the sums of exponents remain negative,
so no extra \(\lambda^{\varepsilon}\) or \(D^{\varepsilon}\) factors appear in the final formula,
and possible logarithms are absorbed into the constant \(C_{\varepsilon}\).


% ===== FAQ: short guide to delicate points (to be placed at the end of the Introduction) =====
\paragraph*{Short guide to delicate points (FAQ).}
\begin{enumerate}
\item \textbf{Statement of the result.}
Within the present framework (assumptions and notation of \S\ref{sec:notation})
we obtain
\[
  \|F\|_{L^{6}(Q_{\lambda})}
  \;\lesssim\;
  \lambda^{\Sigma_{\lambda}} D^{\Sigma_{D}}
  \Bigl(\sum_{\Theta}\|F_{\Theta}\|_{L^{6}}^{2}\Bigr)^{1/2},
\]
with negative cumulative exponents: $\Sigma_{\lambda}<0$, $\Sigma_{D}<0$.
Optimality of the exponents and constants is not discussed.

\item \textbf{The exponent table is a roadmap, not a proof.}
The actual inequalities and constants live in the sections and appendices; the table only summarizes their contribution.

\item \textbf{IBP and \(TT^{\ast}\) are not “double-counted.”}
Six integrations in \(t\) and six in the transverse variables \(x'\) belong to the phase kernel; measure separation is handled separately via \(TT^{\ast}\). These steps are independent.

\item \textbf{Time localization and log-freedom.}
The estimates are carried out on critical windows \(|I|\sim \lambda^{-2}\); gluing across windows/dyads does not create logs—the cumulative exponents are already negative.

\item \textbf{Robust–Kakeya and “tube packing” are mutually exclusive regimes.}
The main line uses Robust–Kakeya; the tube-packing section is provided for context and is not used simultaneously.

\item \textbf{Narrow zone and the passage \(L^{2}\to \dot H^{-1}\).}
The passage uses the zeroth-order operator \( |\nabla|^{-1}\nabla\!\cdot \) (after the Leray projector) together with the null-form factor; this yields a “clean” power in \(\lambda\) without \(\lambda^{\delta}\)-patches.

\item \textbf{“Few caps” scenario.}
When the number of active caps is small, we apply local \(\ell^{2}\!\to\!L^{6}\) orthogonality on \(Q_{\lambda}\); the strength of the final theorem is determined by the “broad” branch (broad geometry \(+\) kernel \(+\) Robust–Kakeya).

\item \textbf{Notation.}
We use \(A\lesssim B\) and \(A\gtrsim B\) for one-sided bounds, and \(A\asymp B\) for two-sided comparability. The symbol \(\simeq\) is interpreted as two-sided comparability \emph{only} where explicitly stated.
\end{enumerate}
% ===== end of FAQ =====
