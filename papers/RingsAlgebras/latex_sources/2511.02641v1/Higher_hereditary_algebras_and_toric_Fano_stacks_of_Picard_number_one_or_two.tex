\documentclass[a4paper,reqno]{amsart}

\setcounter{tocdepth}{1}

\oddsidemargin 0mm
\evensidemargin 0mm
\topmargin 0mm
\textwidth 160mm
\textheight 230mm

\usepackage{amsmath}
\usepackage{amssymb}
\usepackage{amscd}
\usepackage{amsthm}
\usepackage{type1cm}
\usepackage{tcolorbox}
\usepackage{mathrsfs}

\usepackage[colorlinks,citecolor=darkgreen,linkcolor=red]{hyperref}%arXiv用
\definecolor{darkgreen}{rgb}{0.0, 0.6, 0.0}
%\usepackage[bookmarks=true,draft=false,breaklinks,colorlinks,citecolor=darkgreen,linkcolor=red,dvipdfmx]{hyperref}%arXivのときはコメントアウト

\renewcommand\baselinestretch{1.1}

\usepackage[all]{xy}
\input xy
\xyoption{all}
\tcbuselibrary{breakable, skins, theorems}

\def\A{\mathcal{A}}
\def\C{\mathcal{C}}
\def\D{\mathcal{D}}
\def\E{\mathcal{E}}
\def\H{\mathcal{H}}
\def\I{\mathcal{I}}
\def\J{\mathcal{J}}
\def\L{\mathcal{L}}
\def\M{\mathcal{M}}
\def\O{\mathcal{O}}
\def\P{\mathcal{P}}
\def\S{\mathcal{S}}
\def\T{\mathcal{T}}
\def\U{\mathcal{U}}
\def\V{\mathcal{V}}
\def\X{\mathcal{X}}
\def\Y{\mathcal{Y}}
\def\Z{\mathcal{Z}}

\DeclareMathOperator{\Mod}{\mathsf{Mod}}
\DeclareMathOperator{\md}{\mathsf{mod}}
\renewcommand{\mod}{\md}
\DeclareMathOperator{\proj}{\mathsf{proj}}
\DeclareMathOperator{\fl}{\mathsf{fl}}
\DeclareMathOperator{\CM}{\mathsf{CM}}
\DeclareMathOperator{\add}{\mathsf{add}}
\DeclareMathOperator{\Add}{\mathsf{Add}}
\DeclareMathOperator{\Coh}{\mathsf{Coh}}
\DeclareMathOperator{\per}{\mathsf{per}}
\DeclareMathOperator{\thick}{\mathsf{thick}}
\DeclareMathOperator{\pvd}{\mathsf{pvd}}
\DeclareMathOperator{\Loc}{\mathsf{Loc}}
\DeclareMathOperator{\ind}{ind}
\DeclareMathOperator{\simp}{sim}
\DeclareMathOperator{\AR}{AR}
\DeclareMathOperator{\Hom}{Hom}
\DeclareMathOperator{\End}{End}
\DeclareMathOperator{\Ext}{Ext}
\DeclareMathOperator{\op}{op}
\DeclareMathOperator{\Frac}{Frac}
\DeclareMathOperator{\Ker}{Ker}
\DeclareMathOperator{\im}{Im}
\renewcommand{\Im}{\im}
\DeclareMathOperator{\Div}{Div}
\DeclareMathOperator{\Spec}{Spec}
\DeclareMathOperator{\Sing}{Sing}
\DeclareMathOperator{\Pic}{Pic}
\DeclareMathOperator{\rad}{rad}
\DeclareMathOperator{\HH}{HH}
\DeclareMathOperator{\RHom}{\mathbb{R}Hom}
\DeclareMathOperator{\REnd}{\mathbb{R}End}

\DeclareMathOperator{\tr}{tr}
\DeclareMathOperator{\ch}{char}
\DeclareMathOperator{\Conv}{Conv}

\def\gl{\mathop{\rm gl.dim}\nolimits}
\def\lgl{\mathop{\rm l.gl.dim}\nolimits}
\def\rgl{\mathop{\rm r.gl.dim}\nolimits}
\def\wgl{\mathop{\rm w/gl.dim}\nolimits}
\def\pd{\mathop{\rm proj.dim}\nolimits}
\def\id{\mathop{\rm inj.dim}\nolimits}

\theoremstyle{definition}
\newtheorem{Thm}{Theorem}[section]
\newtheorem{Lem}[Thm]{Lemma}
\newtheorem{Prop}[Thm]{Proposition}
\newtheorem{Cor}[Thm]{Corollary}
\newtheorem{Def}[Thm]{Definition}
\newtheorem{Ex}[Thm]{Example}
\newtheorem{Rem}[Thm]{Remark}
\newtheorem{Ques}[Thm]{Question}


\newcommand{\FRAC}[2]{\leavevmode\kern.1em\raise.5ex\hbox{\the\scriptfont0 #1}\kern-.1em/\kern-.15em\lower.25ex\hbox{\the\scriptfont0 #2}}

\title{Higher hereditary algebras and toric Fano stacks of Picard number one or two}
\author{Ryu Tomonaga}
\address{Graduate School of Mathematical Sciences, The University of Tokyo, 3-8-1 Komaba, Meguro-ku, Tokyo, 153-8914, Japan}
\email{ryu-tomonaga@g.ecc.u-tokyo.ac.jp}

\begin{document}
\begin{abstract}
We prove the existence and give a classification of all $d$-tilting bundles (and thus geometric Helices) consisting of line bundles on $d$-dimensional smooth toric Fano DM stacks of Picard number one or two. Here, a $d$-tilting bundle is a tilting bundle whose endomorphism algebra has global dimension $d$ or less.

In the case of Picard number one, tilting bundles consisting of line bundles correspond bijectively to non-trivial upper sets in its Picard group equipped with a certain partial order. Moreover, all of them are $d$-tilting bundles and their endomorphisms algebras become $d$-representation infinite algebras of type $\tilde{A}$. Conversely, all such algebras arise in this way. In this sense, we can think of smooth toric Fano DM stacks with Picard number one as geometric models of higher representation infinite algebras of type $\tilde{A}$. Using this geometric model, we give a new combinatorial description to $d$-APR tilting modules of them.

In the case of Picard number two, $d$-tilting bundles consisting of line bundles correspond bijectively to pairs $(I,I')$, where $I$ and $I'$ are non-trivial upper sets in certain partially ordered sets. Here, $I$ corresponds to a non-commutative crepant resolution (NCCR) of a certain Gorenstein toric singularity with divisor class group of rank one and $I'$ corresponds to a cut of the quiver of this NCCR. Moreover, the endomorphism algebras of these $d$-tilting bundles also become $d$-representation infinite algebras.
\end{abstract}

\maketitle
\tableofcontents

%%%%%%%%%%%%%%%%%%%%%%%%%%%%%%%%%%%%%%%%%%%%%%%%%%%%%%%%%%%%%
\section*{Introduction}
%%%%%%%%%%%%%%%%%%%%%%%%%%%%%%%%%%%%%%%%%%%%%%%%%%%%%%%%%%%%%

%%%%%%%%%%%%%%%%%%%%%%%%%%%%%%%%%%%%%%%%%%%%%%%%%%%%%%%%%%%%%
\subsection{Back grounds from tilting theory for toric stacks and higher Auslander-Reiten theory}
%%%%%%%%%%%%%%%%%%%%%%%%%%%%%%%%%%%%%%%%%%%%%%%%%%%%%%%%%%%%%

Tilting theory is an indispensable tool to establish derived equivalences and gives a bridge among many areas of mathematics including representation theory, algebraic geometry and mathematical physics. As for projective varieties, Beilinson first constructed tilting bundles for projective spaces $\mathbb{P}^d$ \cite{Bei}. After that, many tilting bundles are constructed \cite{BH,HIMO,HP14,Kap,Tom25a} including stacky varieties.

Now we focus on tilting theory for smooth toric stacks. The following question has attracted many people.

\begin{Ques}\label{probtoric}
Let $\X$ be a smooth toric stack. Does $\X$ have a tilting bundle consisting of line bundles?
\end{Ques}

It is conjectured in \cite{Kin97} that Question \ref{probtoric} is true for every smooth toric variety, which proves to be false in \cite{HP06}. After that, in \cite{BH}, it is conjectured that Question \ref{probtoric} is true for every smooth toric (weak) Fano DM stack. They proved that this is true for smooth toric Fano DM stack such that Picard number is at most two or Picard number is any in dimension two. Moreover, Question \ref{probtoric} is proven to be true for every smooth toric weak Fano DM stack in dimension two by using Dimer models \cite{IU09}. However, an infinite list of counterexample, which are smooth toric Fano varieties with Picard number three, to Question \ref{probtoric} is constructed \cite{Efi}. We remark here that Kawamata proved that arbitrary smooth toric DM stacks have full exceptional collections \cite{Kaw06}.

On the other hand, in representation theory of algebras, higher Auslander Reiten theory, first developed by Iyama \cite{Iya07a,Iya07b}, is fundamental to study higher structure of the module categories and the derived categories of algebras \cite{HIO,Iya11,IO11} and has deep connections with non-commutative crepant resolutions \cite{VdB04a}, Calabi-Yau dg algebras \cite{Gin06,KVdB} and additive categorification of cluster algebras \cite{BMRRT}. In \cite{HIO}, for $d\geq1$, the class of $d$-representation infinite algebras is introduced as a generalization of non-Dynkin path algebras to the case of global dimension $d$ in the viewpoint of higher Auslander-Reiten theory.

Examples of higher representation infinite algebras arise from projective geometry naturally in the following way: if a $d$-dimensional smooth proper (stacky) variety has a $d$-tilting bundle (that is, a tilting bundle whose endomorphism algebra has global dimension $d$ or less), then its endomorphism algebra becomes $d$-representation infinite. For example, Beilinson's tilting bundle $\bigoplus_{i=0}^d\O_{\mathbb{P}^d}(i)\in\Coh\mathbb{P}^d$ is a $d$-tilting bundle. For other examples of $d$-tilting bundles and their systematic treatment, see \cite{BHI,HI,HIMO,Tom25c,Tom25a}. Moreover, there are geometric interpretations of $d$-tilting bundles \cite{BuH,Tom25a}. For these reasons, it is natural to ask the following question.

\begin{Ques}\label{probtoricdtilt}
Let $\X$ be a $d$-dimensional smooth toric stack. Does $\X$ have a $d$-tilting bundle consisting of line bundles?
\end{Ques}

In this paper, we give an affirmative answer to Question \ref{probtoricdtilt} for smooth toric Fano DM stacks with Picard number at most two. Moreover, we classify all $d$-tilting bundles consisting of line bundles.

%%%%%%%%%%%%%%%%%%%%%%%%%%%%%%%%%%%%%%%%%%%%%%%%%%%%%%%%%%%%%
\subsection{The case of Picard number one}
%%%%%%%%%%%%%%%%%%%%%%%%%%%%%%%%%%%%%%%%%%%%%%%%%%%%%%%%%%%%%

Let $\X$ be a $d$-dimensional smooth toric Fano DM stack with Picard number one. If we put $G:=\Pic\X$, then by Gale duality, we have $d+1$ elements $\vec{x_0},\cdots,\vec{x_d}\in G$. These elements define a partial order on $G$ as follows:
\[\vec{g_1}\geq\vec{g_2}\Leftrightarrow\vec{g_1}-\vec{g_2}\in\sum_{i=0}^d\mathbb{Z}_{\geq0}\vec{x_i}\subseteq G.\]
Using these notations, we can give a classification of tilting bundles on $\X$ consisting of line bundles.

\begin{Thm}(Theorem \ref{classfitiltrk1})
Let $\X$ be a $d$-dimensional smooth toric Fano DM stack with Picard number one. In the above notations, we have a bijection between the following two sets.
\begin{enumerate}
\item The set of non-trivial upper sets in $G$.
\item $\{J\subseteq G\mid\bigoplus_{\vec{g}\in J}\O_\X(\vec{g})\in\Coh\X\text{ is a tilting bundle}\}$
\end{enumerate}
A bijection from (1) to (2) is given by $I\mapsto I\cap(I^c+\vec{p})$ where $\vec{p}=\sum_{i=0}^d\vec{x_i}$.
\end{Thm}

Moreover, these tilting bundles are $d$-tilting bundles whose endomorphism algebras become $d$-representation infinite algebras of type $\tilde{A}$.

\begin{Thm}(Theorem \ref{classfitiltrk1})
Let $\X$ be a $d$-dimensional smooth toric Fano DM stack with Picard number one.
\begin{enumerate}
\item For each non-trivial upper set $I\subseteq G$, $\End_\X(\bigoplus_{\vec{g}\in I\cap(I^c+\vec{p})}\O_\X(\vec{g}))$ is a $d$-representation infinite algebra of type $\widetilde{A}$.
\item Conversely, every $d$-representation infinite algebra of type $\widetilde{A}$ can be realized in this way.
\end{enumerate}
\end{Thm}

Therefore we can think of smooth toric Fano DM stacks with Picard number one as geometric models of higher representation infinite algebras of type $\tilde{A}$. By using this geometric models, as an application, we can prove the following folklore regarding to $d$-APR tilting modules of $d$-representation infinite algebras of type $\tilde{A}$. Here, recall that a $d$-representation infinite algebra $A$ of type $\tilde{A}$ is defined by a pair $(B,C)$ where $B$ is a cofinite subgroup of a fixed $d$-dimensional lattice and $C$ is a cut of a certain quiver $Q$ defined by $B$. In this sense, we write $A=A(B,C)$.

\begin{Thm}(Theorem \ref{dAPRAtilde}, \ref{derequiAtilde})
Let $A=A(B,C)$ be a $d$-representation infinite algebra of type $\tilde{A}$.
\begin{enumerate}
\item The endomorphism algebra of a $d$-APR tilting module of $A$ becomes a $d$-representation infinite algebra of type $\tilde{A}$ of the form $A(B,C')$ where $C'$ has the same type as $C$.
\item Let $C'$ be a cut of $Q$ with the same type as $C$. Then $A(B,C)$ and $A(B,C')$ can be connected by a finite sequence of $d$-APR tilting modules. In particular, $A(B,C)$ and $A(B,C')$ are derived equivalent.
\end{enumerate}
\end{Thm}

As for this theorem, see also \cite{DG}.

%%%%%%%%%%%%%%%%%%%%%%%%%%%%%%%%%%%%%%%%%%%%%%%%%%%%%%%%%%%%%
\subsection{The case of Picard number two}
%%%%%%%%%%%%%%%%%%%%%%%%%%%%%%%%%%%%%%%%%%%%%%%%%%%%%%%%%%%%%

Let $\X$ be a $d$-dimensional smooth toric Fano DM stack with Picard number two. We give a classification of $d$-tilting bundles on $\X$ consisting of line bundles. Remark that in \cite{BH}, they construct a tilting bundle consisting of line bundles on $\X$. However, we do not know whether their tilting bundle is $d$-tilting or not and their construction is far from giving a classification.

If we put $G:=\Pic\X$, then by Gale duality, we have $d+2$ elements $\vec{x_1},\cdots,\vec{x_{d+2}}\in G$. Put $\vec{p}:=\sum_{i=1}^{d+2}\vec{x_i}\in G$. These elements define a partial order on $G$ as follows:
\[\vec{g_1}\geq\vec{g_2}\Leftrightarrow\vec{g_1}-\vec{g_2}\in\sum_{i=1}^{d+2}\mathbb{Z}_{\geq0}\vec{x_i}\subseteq G.\]
If we put $\pi\colon G\to H:=G/\mathbb{Z}\vec{p}\to H/H_{\rm tors}\cong\mathbb{Z}$, then we may assume that
\[\pi(\vec{x_i})\left\{
\begin{array}{ll}
>0 & (1\leq i\leq l)\\
<0 & (l+1\leq i\leq l+l'=d+2)
\end{array}
\right.\]
holds. Here, we can prove $l,l'\geq2$. We let $q\colon G\to H$ be the natural surjection. Then we can define a partial order on $H$ as
\[h_1\geq h_2:\Leftrightarrow h_1-h_2\in\sum_{i=1}^l\mathbb{Z}_{\geq0}q(\vec{x_i})+\sum_{j=1}^{l'}\mathbb{Z}_{\geq0}q(-\vec{x_{l+j}})\text{ for }h_1,h_2\in H.\]
Put $s:=\sum_{i=1}^lq(\vec{x_i})=\sum_{j=1}^{l'}q(-\vec{x_{l+j}})\in H$. Using these notations, we can give a classification of $d$-tilting bundles on $\X$ consisting of line bundles.

\begin{Thm}(Theorem \ref{classfitiltrk2})
Let $\X$ be a $d$-dimensional smooth toric Fano stack with Picard number two. In the above notations, we have a bijection between the following two sets.
\begin{enumerate}
\item $\{(I,I')\mid I\subseteq H, I'\subseteq q^{-1}(I\cap(I^c+s))\text{ are non-trivial upper sets}\}$
\item $\{J\subseteq G\mid\bigoplus_{\vec{g}\in J}\O_\X(\vec{g})\in\Coh\X\text{ is a $d$-tilting bundle}\}$
\end{enumerate}
Here, $q^{-1}(I\cap(I^c+s))\subseteq G$ has a structure of partially ordered set inherited from $G$. A bijection from (1) to (2) is given by $(I,I')\mapsto I'\cap(I'^c+\vec{p})$.
\end{Thm}

We mention that their endomorphism algebras give rich examples of $d$-representation infinite algebras.

%%%%%%%%%%%%%%%%%%%%%%%%%%%%%%%%%%%%%%%%%%%%%%%%%%%%%%%%%%%%%
\section*{Conventions}
%%%%%%%%%%%%%%%%%%%%%%%%%%%%%%%%%%%%%%%%%%%%%%%%%%%%%%%%%%%%%
Throughout this paper, $k$ denotes an arbitrary field. All algebras and categories are defined over $k$. For an abelian group $G$ and a $G$-graded ring $A$, let $\mod^GA$ and $\proj^GA$ denote the categories of finitely generated $G$-graded right $A$-modules and finitely generated $G$-graded projective right $A$-modules respectively. For a $G$-graded dg ring $\Gamma$, we write $\D^G(\Gamma)$ and $\per^G\Gamma$ for the unbounded derived category of $G$-graded right dg $\Gamma$ modules and the perfect derived category of $G$-graded right dg $\Gamma$ modules respectively.

%%%%%%%%%%%%%%%%%%%%%%%%%%%%%%%%%%%%%%%%%%%%%%%%%%%%%%%%%%%%%
\section*{Acknowledgements}
%%%%%%%%%%%%%%%%%%%%%%%%%%%%%%%%%%%%%%%%%%%%%%%%%%%%%%%%%%%%%
The author is grateful to Osamu Iyama for fruitful discussions. This work was supported by the WINGS-FMSP program at the Graduate School of Mathematical Sciences, the University of Tokyo, and JSPS KAKENHI Grant Number JP25KJ0818.

%%%%%%%%%%%%%%%%%%%%%%%%%%%%%%%%%%%%%%%%%%%%%%%%%%%%%%%%%%%%%
\section{Preliminaries}
%%%%%%%%%%%%%%%%%%%%%%%%%%%%%%%%%%%%%%%%%%%%%%%%%%%%%%%%%%%%%

%%%%%%%%%%%%%%%%%%%%%%%%%%%%%%%%%%%%%%%%%%%%%%%%%%%%%%%%%%%%%
\subsection{Higher representation infinite algebras}
%%%%%%%%%%%%%%%%%%%%%%%%%%%%%%%%%%%%%%%%%%%%%%%%%%%%%%%%%%%%%

First, we recall the definition of higher representation infinite algebras introduced by \cite{HIO}.

\begin{Def}\cite[2.7]{HIO}
Let $A$ be a finite dimensional algebra. For $d\geq1$, $A$ is called {\it $d$-representation infinite} if $\gl A\leq d$ and 
\[\nu_d^{-n}A\in\mod A\subseteq\per A\]
holds for all $n\geq0$.
\end{Def}

This is a generalization of non-Dynkin path algebras to higher global dimensional case in the view point of higher Auslander-Reiten theory. As in the case of non-Dynkin path algebras, we have a $d$-preprojective components $\P:=\add\{\nu_d^{-n}A\mid n\geq0\}\subseteq\mod A$ and a $d$-preinjective components $\I:=\add\{\nu_d^n(DA)\mid n\geq0\}\subseteq\mod A$. For other beautiful properties of higher representation infinite algebras, see \cite{HIO}. In order to show a systematic way to give examples of higher representation infinite algebras, we introduce the following terminologies.

\begin{Def}
Let $\T$ be a triangulated category. Take an object $X\in\T$.
\begin{enumerate}
\item $X$ is called {\it pretilting} if $\T(X,X[\neq0])=0$ holds.
\item $X$ is called {\it tilting} if it is pretilting and $\thick X=\T$ holds.
\item For $d\geq1$, $X$ is called {\it d-tilting} if it is tilting and $\gl\End_{\T}(X)\leq d$ holds.
\end{enumerate}
\end{Def}

The following proposition says that we can get examples of higher representation infinite algebras through investigating tilting objects for certain abelian categories.

\begin{Prop}\cite{BuH,Tom25a}\label{SerreRI}
Let $\A$ be a Hom-finite abelian category and $T\in\A$ a $d$-tilting object of $\D^b(\A)$. If $\A$ has an auto-equivalence $F\curvearrowright\A$ such that $F[d]\curvearrowright\D^b(\A)$ gives a Serre functor, then $\End_\A(T)$ becomes $d$-representation infinite.
\end{Prop}

For further connections between higher representation infinite algebras and projective geometry, see \cite{BuH,Tom25a}.

Next, we give a family of higher representation infinite algebras, which are called of type $\tilde{A}$, introduced by \cite{HIO}. Let $e_i\in\mathbb{Z}^{d+1}$ be the $i$-th unit vector for $0\leq i\leq d$. Put $\alpha_i:=e_i-e_{i-1}$ for $1\leq i\leq d$ and $\alpha_0:=e_0-e_d$. Let $L:=\{v=(v_i)_{i=0}^d\in\mathbb{Z}^{d+1}\mid\sum_{i=0}^dv_i=0\}=\sum_{i=0}^d\mathbb{Z}\alpha_i\subseteq\mathbb{Z}^{d+1}$ be a $d$-dimensional lattice and $B\subseteq L$ a cofinite subgroup. Put $m:=\sharp(L/B)$. As in \cite{DG}, let $\hat{Q}:=(L,\{x\to x+\alpha_i\mid x\in L, 0\leq i\leq d\})$ be an infinite quiver. We say that an arrow $x\to x+\alpha_i$ in $\hat{Q}$ has {\it tyoe} $i$. A cycle of length $d+1$ in $\hat{Q}$ consisting of arrows of $d+1$ distinct types is called an {\it elementary cycle}. A subset $\hat{C}\subseteq\hat{Q}_1$ is called a {\it cut} if every elementary cycle has exactly one arrow in $\hat{C}$. A cut $\hat{C}\subseteq\hat{Q}_1$ is said to be $B$-{\it periodic} if $\hat{C}$ is invariant under $B$-translation.% For a cut $\hat{C}\subseteq\hat{Q}_1$, we define a quiver $\hat{Q}_{\hat{C}}:=(\hat{Q}_0,\hat{Q}_1\setminus\hat{C})$.

Similarly, let $Q:=(L/B,\bigsqcup_{i=0}^d\{x+B\to x+\alpha_i+B\mid x\in L\})$ be a finite quiver which may have multiple arrows. We define a cut of $Q$ similarly. For a cut $C\subseteq Q_1$, we call $\gamma(C):=(\sharp\{a\in C\mid\text{The type of $a$ is }i.\})_{i=0}^d\in\mathbb{Z}^{d+1}_{\geq0}$ the {\it type} of $C$. For a cut $C\subseteq Q_1$ of type $\gamma=(\gamma_i)_{i=0}^d$, we have $\sum_{i=0}^d\gamma_i=m$. Observe that cuts of $Q$ correspond bijectively to $B$-periodic cuts of $\hat{Q}$. In what follows, we identify $B$-periodic cuts of $\hat{Q}$ with cuts of $Q$ freely. For a cut $C\subseteq Q_1$, we define a quiver $Q_C:=(Q_0,Q_1\setminus C)$. A cut $C\subseteq Q_1$ is called {\it bounding} if the quiver $Q_C$ is acyclic.

\begin{Def}\cite[5.6(2)]{HIO}
Let $C\subseteq Q_1$ be a bounding cut. Consider the relation $I_C$ in path algebra $kQ_C$ which is generated by 
\[(x+B\to x+\alpha_i\to x+\alpha_i+\alpha_j+B)=(x+B\to x+\alpha_j\to x+\alpha_i+\alpha_j+B)\]
for $x\in L, 0\leq i,j\leq d$ such that the four arrows exist in $Q_C$. We call $A(B,C):=kQ_C/I_C$ a $d$-{\it representation infinite algebra of type} $\tilde{A}$.
\end{Def}

In \cite{HIO}, it is proved that this $A$ is $d$-representation infinite when $k$ is algebraically closed field of characteristic zero. Later, we give another proof of this fact which is valid for arbitrary field $k$.

%%%%%%%%%%%%%%%%%%%%%%%%%%%%%%%%%%%%%%%%%%%%%%%%%%%%%%%%%%%%%
\subsection{Smooth toric Fano stacks}
%%%%%%%%%%%%%%%%%%%%%%%%%%%%%%%%%%%%%%%%%%%%%%%%%%%%%%%%%%%%%

In this subsection, we recall the definition and basic properties of smooth toric DM Fano stacks from \cite{BCS}. Let $N\cong\mathbb{Z}^d$ be a free abelian group of rank $d$ and $P$ a simplicial convex lattice polytope in $N_\mathbb{R}:=N\otimes_\mathbb{Z}\mathbb{R}$ containing the origin as an interior point. Remark that we do not allow $N$ to have torsions. Let $\{v_i\}_{i=1}^n$ denote the set of the vertices of $P$. This $\{v_i\}_{i=1}^n$ defines a group homomorphism $\phi\colon\mathbb{Z}^n\to N$ with finite cokernel. Define an abelian group $G$ by the following exact sequence.
\[0\to N^*\xrightarrow{\phi^*}(\mathbb{Z}^n)^*\to G\to0\]
For $1\leq i\leq n$, we write $\vec{x_i}\in G$ for the image of the $i$-th unit vector of $(\mathbb{Z}^n)^*$. Then the polynomial ring $S:=k[x_1,\cdots, x_n]$ can be viewed as a $G$-graded $k$-algebra by $\deg x_i:=\vec{x_i}$. This grading induces an action of a group scheme $\Spec k[G]$ on $\mathbb{A}_k^n$. We define a Stanley-Reisner locus $SR(P)\subseteq\mathbb{A}_k^n$ as a closed subscheme defined by the reduced monomial ideal $(\prod_{v_i\notin Q}x_i\mid Q\subsetneq P\text{ is a proper face})\subseteq S$. Now we can associate to the polytope $P$ a smooth toric DM Fano stack $\X(P)$ as the quotient stack
\[\X(P):=[(\mathbb{A}_k^n\backslash SR(P))/\Spec k[G]].\]
Remark that $\X(P)$ becomes a Deligne-Mumford stack \cite[3.2]{BCS}. For Fano-ness, see \cite[3.11,3.12]{BH}.

\begin{Rem}
For a proper face $Q\subsetneq P$, we define a cone $\sigma_Q:=\sum_{v_i\in Q}\mathbb{R}_{\geq0}v_i\subseteq N_\mathbb{R}$. Then one gets a complete fan $\Sigma:=\{\sigma_Q\mid Q\subsetneq P\text{ is a proper face}\}$ in $N_\mathbb{R}$ and a data of a complete stacky fan $\boldsymbol{\Sigma}=(\Sigma, \{v_i\}_{i=1}^n)$. In this notation, our $\X(P)$ coincides with $\X(\boldsymbol{\Sigma})$ in \cite{BCS}.
\end{Rem}

Put $\X:=\X(P)$. We have a categorical equivalence
\[\Coh\X\simeq\Coh^{\Spec k[G]}\mathbb{A}_k^n\backslash SR(P)\simeq\mod^GS/\mod^G_{SR(P)}S,\]
where $\mod^G_{SR(P)}S\subseteq\mod^GS$ is a full subcategory consisting of modules supported by $SR(P)$. We put $\widetilde{(-)}:=(\mod^GS\to\mod^GS/\mod^G_{SR(P)}S\xrightarrow[\simeq]{}\Coh\X)$. For $\vec{g}\in G$, the auto-equivalence $(\vec{g})\colon\mod^GS\to\mod^GS$ induces an auto equivalence $(\vec{g})\colon\Coh\X\to\Coh\X$. If we put $\vec{p}:=\vec{x_1}+\cdots\vec{x_n}\in G$, then since $\omega_\X:=\O_\X(-\vec{p})$ is the canonical bundle, $(-\vec{p})[d]\colon\D^b(\Coh\X)\to\D^b(\Coh\X)$ gives a Serre functor. In addition, the group homomorphism
\[G\to\Pic\X;\vec{g}\mapsto\O_\X(\vec{g})\]
is an isomorphism.

For $a=(a_i)_{i=1}^n\in\mathbb{Z}^n$, put
\[\Delta_a:=\{I\subsetneq\{1,\cdots,n\}\mid\Conv\{v_i\}_{i\in I}\text{ is a face of }P\text{ and }a_i\geq0\text{ holds for all }i\in I\}\]
and define a subspace $X_a\subseteq P$ as
\[X_a:=\bigcup_{I\in\Delta_a}\Conv\{v_i\}_{i\in I}.\]
Observe that our definition of $\Delta_a$ (and so $X_a$) differ from those of \cite[2.6]{Tom25d} when $a\in\mathbb{Z}_{\geq0}^n$. In this terminology, the cohomology of the line bundles can be computed in the following way.

\begin{Prop}\label{calcoh}\cite[4.1]{BH}
Assume $\X$ is Fano. For $\vec{g}\in G$, we have
\[H^r(\X,\O_\X(\vec{g}))\cong\bigoplus_{\substack{a\in\mathbb{Z}^n \\ \sum_ia_i\vec{x_i}=\vec{g}}}\tilde{H}_{d-r-1}(X_a;k),\]
where $\tilde{H}_{d-r-1}(X_a;k)$ denotes the $(d-r-1)$-th reduced singular homology of $X_a$ with coefficients in $k$. Remark that we think $\tilde{H}_{-1}(X;k)\left\{
\begin{array}{ll}
=0 & (X\neq\emptyset)\\
\cong k & (X=\emptyset)
\end{array}
\right.$.
\end{Prop}
\begin{proof}
This can be proved in the same way as \cite[4.1]{BH}.
\end{proof}

Finally, we see some properties of our group $G$. By construction, and since $P$ contains the origin as an interior point, $G$ and $\vec{x_i}\in G$ satisfy the following conditions.
\begin{enumerate}
\item[(G1)] $\vec{x_i}\neq0$ for all $0\leq i\leq d$.
\item[(G2)] $G=\sum_{i=0}^d\mathbb{Z}\vec{x_i}$
\item[(G3)] If we put $G_{\geq0}:=\sum_{i=0}^d\mathbb{Z}_{\geq0}\vec{x_i}\subseteq G$, then we have $G_{\geq0}\cap(-G_{\geq0})=0$.
\end{enumerate}
Conversely, if a finitely generated abelian group $G$ of rank $n-d$ and $\vec{x_1},\cdots,\vec{x_n}\in G$ satisfying (G1), (G2) and (G3) are given, then we obtain $n$ lattice points $v_1,\cdots, v_n\in N$ whose convex hull $\Conv\{v_i\}_{i=1}^n$ contains the origin as an interior point. However, $v_i$ is not necessarily a vertex of $\Conv\{v_i\}_{i=1}^n$.

We remark here that if a finitely generated abelian group $G$ of rank and $\vec{x_1},\cdots,\vec{x_n}\in G$ satisfy (G1), (G2) and (G3), then we can define a partial order on $G$ as
\[\vec{g}\geq\vec{h}:\Leftrightarrow\vec{g}-\vec{h}\in G_{\geq0}.\]
If we view $S:=k[x_1,\cdots,x_n]$ as a $G$-graded $k$-algebra by $\deg x_i=\vec{x_i}$, then for $\vec{g}\in G$, $S_{\vec{g}}\neq0$ holds if and only if $\vec{g}\geq0$ holds.

%%%%%%%%%%%%%%%%%%%%%%%%%%%%%%%%%%%%%%%%%%%%%%%%%%%%%%%%%%%%%
\section{Combinatorics}
%%%%%%%%%%%%%%%%%%%%%%%%%%%%%%%%%%%%%%%%%%%%%%%%%%%%%%%%%%%%%

%%%%%%%%%%%%%%%%%%%%%%%%%%%%%%%%%%%%%%%%%%%%%%%%%%%%%%%%%%%%%
\subsection{Combinatorics of upper sets}
%%%%%%%%%%%%%%%%%%%%%%%%%%%%%%%%%%%%%%%%%%%%%%%%%%%%%%%%%%%%%

In this subsection, let $X$ be a partially ordered set.

\begin{Def}
We call a subset $I\subseteq X$ an {\it upper set} if for all $x\in I$ and $y\in X$ with $x\leq y$, $y\in I$ holds. An upper set $I\subseteq X$ is called {\it non-trivial} if $I\neq0,X$. We put $\I_X:=\{I\subseteq X\colon\text{non-trivial upper set}\}$.
\end{Def}

Assume $\mathbb{Z}$ acts on the set $X$ satisfying the following conditions. Here, we write $x+np:=n\cdot x$ for $n\in\mathbb{Z}$.
\begin{enumerate}
\item[(A1)] $x<x+p$ holds for all $x\in X$.
\item[(A2)] $x\leq y$ implies $x+np\leq y+np$ for all $x,y\in X$ and $n\in\mathbb{Z}$.
\item[(A3)] For any $x,y\in X$, there exists $n\in\mathbb{Z}$ such that $x+np\geq y$ holds.
\end{enumerate}

As \cite{Tom25d}, we put
\[\widetilde{\J}_X:=\{J\subseteq X\mid\text{For any }x,y\in J,\text{ we have }x\ngeq y+p.\}\text{ and}\]
\[\J_X:=\{J\in\widetilde{\J}_X\colon\text{maximal with respect to inclusion}\}\subseteq\widetilde{\J}_X.\]
Then we have the following bijection between $\I_X$ and $\J_X$.

\begin{Thm}\cite[1.3]{Tom25d}\label{upJX}
Consider the following sets.
\[J(-)\colon\I_X\rightleftarrows\J_X :I(-)\]
Then $J(I):=I\cap(I^c+p)$ and $I(J):=\{x\in X\mid\text{There exists }y\in J\text{ with }x\geq y.\}$ give inverse maps to each other.
\end{Thm}

Next, we introduce the notion of mutation for upper sets.

\begin{Def}\cite[1.6]{Tom25d}
Let $I\in\I_X$ and take a minimal element $m\in I$. Then we define the {\it mutation} $\mu_{m}(I)$ of $I$ at $m$ as
\[\mu_m^-(I):=I\setminus\{m\}.\]
\end{Def}

Finally, we focus on the following explicit setting. Let $G$ be a finitely generated abelian group of rank one. Assume we are given elements $\vec{x_0}\cdots,\vec{x_d}\in G$ satisfying (G1), (G2) and (G3). Put $\vec{p}:=\sum_{i=0}^d\vec{x_i}\in G$. Then $\mathbb{Z}$ acts on $G$ by $n\cdot\vec{g}:=\vec{g}+n\vec{p}$. This action satisfies the conditions (A1),(A2) and (A3). In this setting, we can describe $\J_G$ in the following way.

\begin{Prop}\label{GJX}
For a subset $J\subseteq G$, the following conditions are equivalent.
\begin{enumerate}
\item $J\in\J_G$
\item $J\subseteq G$ is a complete representative of $G/\mathbb{Z}\vec{p}$ and for every $\vec{g}\in J$ and $0\leq i\leq d$, we have $\vec{g}+\vec{x_i}\in J\sqcup(J+\vec{p})$.
\end{enumerate}
\end{Prop}
\begin{proof}
(1)$\Rightarrow$(2) $J\subseteq G$ is a complete representative by \cite[1.5]{Tom25d}. Take $\vec{g}\in J$ and $0\leq i\leq d$. Then there exists unique $n\in\mathbb{Z}$ with $\vec{g}+\vec{x_i}\in J+n\vec{p}$. If $n>1$, then $\vec{g}\geq\vec{g}-(\vec{p}-\vec{x_i})\geq(\vec{g}+\vec{x_i}-n\vec{p})+\vec{p}$ holds, but this contradicts to $\vec{g},\vec{g}+\vec{x_i}-n\vec{p}\in J$. If $n<0$, then $\vec{g}+\vec{x_i}-n\vec{p}\geq\vec{g}+\vec{p}$ holds, but this is a contradiction for the same reason. Thus we obtain $n=0$ or $n=1$.

(2)$\Rightarrow$(1) Assume there exists $\vec{g},\vec{h}\in J$ with $\vec{h}\geq\vec{g}+\vec{p}$. Then by the definition of the partial order on $G$, there exists $a_0,\cdots, a_d\in\mathbb{Z}_{\geq0}$ such that $\vec{h}=\vec{g}+\vec{p}+\sum_{i=0}^da_i\vec{x_i}$. By our assumption, there exists $m\geq0$ such that $\vec{g}+\sum_{i=0}^da_i\vec{x_i}\in J+m\vec{p}$ holds. This means $\vec{g}+\vec{p}+\sum_{i=0}^da_i\vec{x_i}\in J+(m+1)\vec{p}$ holds, but this contradicts to $\vec{h}\in J$. Thus $J\in\widetilde{\J}_G$ holds. By \cite[1.5]{Tom25d}, we obtain $J\in\J_G$.
\end{proof}

%%%%%%%%%%%%%%%%%%%%%%%%%%%%%%%%%%%%%%%%%%%%%%%%%%%%%%%%%%%%%
\subsection{Combinatorics of cuts}
%%%%%%%%%%%%%%%%%%%%%%%%%%%%%%%%%%%%%%%%%%%%%%%%%%%%%%%%%%%%%

Let $L:=\{v=(v_i)_{i=0}^d\in\mathbb{Z}^{d+1}\mid\sum_{i=0}^dv_i=0\}=\sum_{i=0}^d\mathbb{Z}\alpha_i\subseteq\mathbb{Z}^{d+1}$ be a $d$-dimensional lattice and $B\subseteq L$ a cofinite subgroup. Put $m:=\sharp(L/B)$.

First, we introduce a new object which we call cut detectors. This is an analogue of height functions.

\begin{Def}
A map $f\colon L/B\to\mathbb{Z}$ is called a {\it cut detector} of type $\gamma\in\mathbb{Z}^{d+1}_{\geq0}$ if satisfies the following conditions.
\begin{enumerate}
\item $f(0)=0$
\item For every $x\in L$, we have $f(x+\alpha_i+B)\in\{f(x+B)+\gamma_i,f(x+B)+\gamma_i-m\}$
\end{enumerate}
\end{Def}

Then we can prove that cut detectors correspond bijectively to cuts of $Q$.

\begin{Thm}\label{cutbij}
For $\gamma\in\mathbb{Z}^{d+1}_{\geq0}$, we have a bijection between the following sets.
\begin{enumerate}
\item The set of cut detectors $f\colon L/B\to\mathbb{Z}$ of type $\gamma$.
\item The set of cuts of $Q$ of type $\gamma$.
\end{enumerate}
\end{Thm}

First, we see that a cut $C$ of $Q$ induces cut detectors of the same type.

\begin{Def}
Let $\gamma$ be the type of $C$. For $a\in\hat{Q}_1$ of type $i$, we define
\[f_C(a):=
  \begin{cases}
    \gamma_i & a\notin C,\\
    \gamma_i-m & a\in C.
  \end{cases}
\]
For a path $p=a_n\cdots a_1$ in $\hat{Q}$, we define
\[f_C(p):=\sum_{i=1}^nf_C(a_i).\]
\end{Def}

\begin{Rem}
For a path $p$ in $\hat{Q}$ of length $0$, we think $f_C(p)=0$.
\end{Rem}

The following can be shown in the same way as \cite[2.5]{DG}.

\begin{Lem}
For paths $p,q$ in $\hat{Q}$ with same sources and targets, we have $f_C(p)=f_C(q)$.
\end{Lem}

Thanks to this lemma, for $x\in L$, we can define
\[f_C(x):=f_C(p_x),\]
where $p_x$ is any path from $0$ to $x$.

\begin{Prop}
Our $f_C\colon L\to\mathbb{Z}$ induces a cut detector $f_C\colon L/B\to\mathbb{Z}$ of type $\gamma$.
\end{Prop}
\begin{proof}
It is enough to show that $f_C\colon L\to\mathbb{Z}$ is invariant under the action of $B$ on $L$. Take $x\in L$ and $y\in B$. Let $p_x$ be a path in $\hat{Q}$ from $0$ to $x$. Since $C$ is $B$-periodic, for the path $p_x+y$ from $y$ to $x+y$, we have $f_C(p_x)=f_C(p_x+y)$. Thus we obtain
\[f_C(x+y)=f_C(y)+f_C(p_x+y)=f_C(y)+f_C(x).\]
Therefore it is enough to show $f_C(y)=0$.

In what follows, we mimic the proof of \cite[2.9]{DG}. Let $o_i$ be the order of $\alpha_i+B\in L/B$. First, we show $f_C(o_i\alpha_i)=0$. Consider the path $0\to\alpha_i\to\cdots\to o_i\alpha_i$, where each arrow is of type $i$ and put $\theta'_i:=\sharp\{1\leq j\leq o_i\mid((j-1)\alpha_i\to j\alpha_i)\in C\}$. Then we have $f_C(o_i\alpha_i)=o_i\gamma_i-\theta'_im$. Here, for any $x\in L$, we have
\[f_C(o_i\alpha_i)=f_C(x+o_i\alpha_i)-f_C(x)=f_C(x\to x+\alpha_i\to\cdots\to x+o_i\alpha_i).\]
This implies $\theta'_i=\sharp\{1\leq j\leq o_i\mid(x+(j-1)\alpha_i\to x+j\alpha_i)\in C\}$ holds. Take $x_1,\cdots,x_{\frac{m}{o_i}}\in L$ so that $\{x_l+B\}_l\subseteq L/B$ gives a complete representative of $(L/B)/\mathbb{Z}(\alpha_i+B)$. Then each arrow of type $i$ in $Q$ appears exactly once in cycles
\[x_l\to x_l+\alpha_i\to\cdots\to x_l+o_i\alpha_i\ (1\leq l\leq\frac{m}{o_i}).\]
This means $\frac{m}{o_i}\theta'_i=\gamma_i$. Therefore we have
\[f_C(o_i\alpha_i)=o_i\gamma_i-\theta'_im=0.\]

Finally, consider arbitrary $y\in B$. Since $f_C(my)=mf_C(y)$, it is enough to show $f_C(my)=0$. If we write $y=\sum_{i=0}^dy_i\alpha_i$, then we have
\[f_C(my)=\sum_{i=0}^dy_i\frac{m}{o_i}f_C(o_i\alpha_i)=0.\qedhere\]
\end{proof}

Using this, we can recover \cite[2.13,2.14]{DG} easily.

\begin{Cor}
Let $C$ be a cut of $Q$ and $\gamma$ its type.
\begin{enumerate}
\item\cite[2.13]{DG} Take $(m_i)_{i=0}^d\in\mathbb{Z}^{d+1}$. If $\sum_{i=0}^dm_i\alpha_i\in B$ holds, then we have $\sum_{i=0}^dm_i\gamma_i\in m\mathbb{Z}$.
\item\cite[2.14]{DG} The cut $C$ is bounding if and only if $\gamma\in\mathbb{Z}^{d+1}_{>0}$ holds.
\end{enumerate}
\end{Cor}
\begin{proof}
(1) By the definition of $f_C$, we have $f_C(\sum_{i=0}^dm_i\alpha_i)-\sum_{i=0}^dm_i\gamma_i\in m\mathbb{Z}$. Since $f_C(\sum_{i=0}^dm_i\alpha_i)=0$, we get the conclusion.

(2) The necessity is obvious. We prove the sufficiency. Take $x,y\in L/B$. Observe that if there exists a path in $Q_C$ from $x$ to $y$, then we have $f_C(x)<f_C(y)$ by the definition of $f_C$. This proves the conclusion.
\end{proof}

Now we prove Theorem \ref{cutbij}.
\begin{proof}[Proof of Theorem \ref{cutbij}]
Let $f\colon L/B\to\mathbb{Z}$ be a cut detector of type $\gamma$. We define a subset $C_f\subseteq Q_1$: for an arrow $a\colon x+B\to x+\alpha_i+B$ in $Q$,
\[a\in C_f\Leftrightarrow f(x+\alpha_i+B)=f(x+B)+\gamma_i-m.\]
Then this $C_f$ is a cut of $Q$. We show that the type of $C_f$ is $\gamma$. Let $o_i$ be the order of $\alpha_i+B\in L/B$. Then we have
\[0=f(o_i\alpha_i+B)=\sum_{j=1}^{o_i}(f(jo_i\alpha_i+B)-f((j-1)o_i\alpha_i+B))=o_i\gamma_i-\theta'_im,\]
where $\theta'_i=\sharp\{1\leq j\leq o_i\mid((j-1)\alpha_i\to j\alpha_i)\in C_f\}$. Here, for any $x\in L$, we have
\[0=f(x+o_i\gamma_i+B)-f(x+B)=\sum_{j=1}^{o_i}(f(x+jo_i\alpha_i+B)-f(x+(j-1)o_i\alpha_i+B)).\]
This implies $\theta'_i=\sharp\{1\leq j\leq o_i\mid(x+(j-1)\alpha_i\to x+j\alpha_i)\in C_f\}$. Thus by taking a complete representative of $(L/B)/\mathbb{Z}(\alpha_i+B)$, we can calculate that the number of the arrows in $Q_{C_f}$ of type $i$ is
\[\frac{m}{o_i}\theta'_i=\gamma_i.\]

By constructions, it is easy to check that $f_{C_f}=f$ and $C_{f_C}=C$ hold. This completes the proof.
\end{proof}

In \cite{DG}, the following theorem is proved by constructing an explicit cut which is periodic with respect to another cofinite subgroup of $L$.

\begin{Thm}\label{chartype}\cite[3.5]{DG}
For $\gamma=(\gamma_i)_{i=0}^d\in\mathbb{Z}_{\geq0}^{d+1}$, $\gamma$ is a type of a $B$-periodic cut if and only if both of the following conditions are satisfied.
\begin{enumerate}
\item $\sum_{i=0}^d\gamma_i=m$
\item For any $(m_i)_{i=0}^d\in\mathbb{Z}^{d+1}$ with $\sum_{i=0}^dm_i\alpha_i\in B$, we have $\sum_{i=0}^dm_i\gamma_i\in m\mathbb{Z}$.
\end{enumerate}
\end{Thm}

The necessity of these conditions are already proved. In the next subsection, we give a new proof of the sufficiency by introducing cut-upper set correspondence.

%%%%%%%%%%%%%%%%%%%%%%%%%%%%%%%%%%%%%%%%%%%%%%%%%%%%%%%%%%%%%
\subsection{Cut-upper set correspondence}
%%%%%%%%%%%%%%%%%%%%%%%%%%%%%%%%%%%%%%%%%%%%%%%%%%%%%%%%%%%%%

Let $\gamma=(\gamma_i)_{i=0}^d\in\mathbb{Z}_{\geq0}^{d+1}$ be an integer vector satisfying the conditions (1) and (2) in Theorem \ref{chartype}. We define a group homomorphism $\Phi\colon\mathbb{Z}^{d+1}\to\mathbb{Z}\oplus L/B$ by
\[\Phi(e_i):=(\gamma_i,\alpha_i+B)\]
and put $G=G(B,\gamma):=\Im\Phi$ and $\vec{x_i}:=\Phi(e_i)\in G$. Observe that $\vec{p}:=\sum_{i=0}^d\vec{x_i}=(m,0)$ holds. Thus the composition $G\hookrightarrow\mathbb{Z}\oplus L/B\twoheadrightarrow L/B$ induces a group homomorphism $\phi\colon G/\mathbb{Z}\vec{p}\to L/B$.

\begin{Lem}
The group homomorphism $\phi\colon G/\mathbb{Z}\vec{p}\to L/B$ is an isomorphism.
\end{Lem}
\begin{proof}
The surjectivity follows from $\phi(\vec{x_i}+\mathbb{Z}\vec{p})=\alpha_i+B$. Take $\vec{g}=\Phi(v)\in G$ with $\phi(\vec{g}+\mathbb{Z}\vec{p})=0$. If we put $v=(m_i)_{i=0}^d$, then we have $\sum_{i=0}^dm_i\alpha_i\in B$. Thus by our assumption, there exists $n\in\mathbb{Z}$ with $\sum_{i=0}^dm_i\gamma_i=mn$. This implies $\vec{g}=n\vec{p}$.
\end{proof}

We define
\[\J:=\{J\subseteq G\colon\text{a complete representative of }G/\mathbb{Z}\vec{p}\mid \vec{g}+\vec{x_i}\in J\sqcup(J+p)\text{ for all }\vec{g}\in J\text{ and }0\leq i\leq d\}.\]
Let $\pi:=(G\hookrightarrow\mathbb{Z}\oplus L/B\twoheadrightarrow\mathbb{Z})$ denotes the composition of natural group homomorphisms. The following proposition is key to prove Theorem \ref{chartype}.

\begin{Prop}\label{cutJcorr}
We have a surjective map
\[C(-)\colon\J\to\{\text{Cuts of $Q$ of type }\gamma\}.\]
For $J,J'\in\J$, $C(J)=C(J')$ holds if and only if $J=J'+n\vec{p}$ holds for some $n\in\mathbb{Z}$.
\end{Prop}
\begin{proof}
For $J\in\J$, let $C(J)\subseteq Q_1$ be a subset consisting of arrows which do not appear in the Cayley quiver of $J$. More precisely, we can describe $C(J)$ in terms of cut detectors as follows (see Theorem \ref{cutbij}). There exists a unique $n\in\mathbb{Z}$ with $n\vec{p}\in J$. Define a map $f_J\colon L/B\to\mathbb{Z}$ in the following way. For $x\in L$, take $\vec{g}\in J$ with $\phi(\vec{g}+\mathbb{Z}\vec{p})=x+B$. Then put $f_J(x+B):=\pi(\vec{g}-n\vec{p})=\pi(\vec{g})-nm$. Then we can check that $f_J$ is a cut detector of type $\gamma$ and put $C(J):=C_{f_J}$. By our definition, for $J,J'\in\J$, $f_J=f_{J'}$ holds if and only if $J=J'+n\vec{p}$ holds for some $n\in\mathbb{Z}$.

We prove the surjectivity of $C(-)$. We use Theorem \ref{cutbij}. Take a cut detector $f\colon L/B\to\mathbb{Z}$. Put $J:=\{x\in G\mid\pi(x)=f(\phi(x+B))\}\subseteq G$. Then we have $J\in\J$ and $f_J=f$.
\end{proof}

Now we can prove Theorem \ref{chartype}.

\begin{proof}[Proof of Theorem \ref{chartype}]
By Proposition \ref{cutJcorr}, it is enough to show $\J\neq\emptyset$. For example, if we put $J:=\{x\in G\mid0\leq\pi(x)<m\}\subseteq G$, then  we have $J\in\J$.
\end{proof}

Finally, to state cut-upper set correspondence, we focus on the case of $\gamma\in\mathbb{Z}^{d+1}_{>0}$. In this case, our $G$ and $\vec{x_i}\in G$ satisfy the conditions (G1), (G2) and (G3).

\begin{Thm}\label{cutupcorr}(Cut-upper set correspondence)
Assume $\gamma\in\mathbb{Z}^{d+1}_{>0}$. Then we have a surjective map
\[C(-)\colon\I_G\to\{\text{Cuts of $Q$ of type }\gamma\}.\]
For $I,I'\in\I_G$, $C(I)=C(I')$ holds if and only if $I=I'+n\vec{p}$ holds for some $n\in\mathbb{Z}$.
\end{Thm}
\begin{proof}
By Proposition \ref{GJX}, we have $\J=\J_G$. Thus the assertion follows from Theorem \ref{upJX} and Proposition \ref{cutJcorr}.
\end{proof}

%%%%%%%%%%%%%%%%%%%%%%%%%%%%%%%%%%%%%%%%%%%%%%%%%%%%%%%%%%%%%
\subsection{Starting from $G$}\label{startG}
%%%%%%%%%%%%%%%%%%%%%%%%%%%%%%%%%%%%%%%%%%%%%%%%%%%%%%%%%%%%%

Let $G$ be a finitely generated abelian group of rank one. Assume we are given elements $\vec{x_0},\cdots,\vec{x_d}\in G$ satisfying (G1), (G2), and (G3). In this subsection, we construct a cofinite subgroup $B\subseteq L$ and a type of a cut from $G$.

Put $\vec{p}:=\sum_{i=0}^d\vec{x_i}\in G$. Then we can define a surjective group homomorphism $L\to G/\mathbb{Z}\vec{p}$ sending $\alpha_i$ to $\vec{x_i}+\mathbb{Z}\vec{p}$. Let $B\subseteq L$ be the kernel of this homomorphism. Put $m:=\#(G/\mathbb{Z}\vec{p})=\#(L/B)$. Let $\pi':G\to G/G_{\rm tors}\cong\mathbb{Z}$ and put $m':=\pi'(\vec{p})$. Then since a surjective group homomorphism $G/\mathbb{Z}\vec{p}\to\mathbb{Z}/m'\mathbb{Z}$ is induced, we have $m\in m'\mathbb{Z}$. Define $\pi:=\frac{m}{m'}\pi'\colon G\to\mathbb{Z}$. If we put $\gamma_i:=\pi(\vec{x_i})$, then we can check that our $\gamma=(\gamma_i)_{i=0}^d\in\mathbb{Z}^d_{>0}$ satisfies the conditions in Theorem \ref{chartype}.

Define a group homomorphism $\Phi\colon\mathbb{Z}^{d+1}\to\mathbb{Z}\oplus L/B$ and $\Phi'\colon\mathbb{Z}^{d+1}\to G$ as
\[\Phi(e_i):=(\gamma_i,\alpha_i+B),\Phi'(e_i)=\vec{x_i}.\]
For $v=(m_i)_{i=0}^d\in\mathbb{Z}^{d+1}$, $\Phi(v)=0$ if and only if $\sum_{i=0}^dm_i\gamma_i=0$ and $\sum_{i=0}^dm_i\alpha_i\in B$ holds. $\sum_{i=0}^dm_i\gamma_i=0$ is equivalent to $\Phi'(v)\in G_{\rm tors}$. $\sum_{i=0}^dm_i\alpha_i\in B$ is equivalent to $\Phi'(v)\in\mathbb{Z}\vec{p}$ holds. Since $G_{\rm tors}\cap\mathbb{Z}\vec{p}=0$, we obtain $\Ker\Phi=\Ker\Phi'$. Thus we have an isomorphism $G\cong\Im\Phi=G(B,\gamma)$.

%%%%%%%%%%%%%%%%%%%%%%%%%%%%%%%%%%%%%%%%%%%%%%%%%%%%%%%%%%%%%
\section{Beilinson-type theorem for $G$-graded dg rings}
%%%%%%%%%%%%%%%%%%%%%%%%%%%%%%%%%%%%%%%%%%%%%%%%%%%%%%%%%%%%%

Let $G$ be a finitely generated abelian group whose rank is one. We assume that $G$ admits a partial order $\leq$ satysfying $G=\mathbb{Z}G_{\geq0}$ and $\vec{x}\leq\vec{y}\Rightarrow\vec{x}+\vec{z}\leq\vec{y}+z$ for any $\vec{x},\vec{y},\vec{z}\in G$. Let $\Gamma$ be a $G_{\geq0}$-graded dg ring. We introduce some notations for brevity.

\begin{Def}\label{subGder}
Let $I\subseteq G$ be a subset.
\begin{enumerate}
\item For $X\in\per^G\Gamma$, define $\thick^IX:=\thick\{X(-\vec{g})\mid\vec{g}\in I\}\subseteq\per^G\Gamma$.
\item $\per^I\Gamma:=\thick^I\Gamma\subseteq\per^G\Gamma$
\item $\D^G(\Gamma)_I:=\{X\in\D^G(\Gamma)\mid H^nX_{\vec{g}}=0\text{ for all }n\in\mathbb{Z}\text{ and }\vec{g}\in I^c\}$
\item $(\per^G\Gamma)_I:=\per^G\Gamma\cap\D^G(\Gamma)_I$
\end{enumerate}
\end{Def}

As in \cite[A.1]{Han24c}, we say $\Gamma$ has {\it Gorenstein parameter} $\vec{p}\in G$ if $\RHom_\Gamma(-,\Gamma)$ takes $\D^G(\Gamma)_0$ to $\D^G(\Gamma^{\op})_{-\vec{p}}$. Then we have an analogous statement to \cite[A.4]{Han24c}.

\begin{Lem}
Let $\Gamma$ be a $G_{\geq0}$-graded dg ring.
\begin{enumerate}
\item We have $\D^G(\Gamma)_0=\Loc\Gamma_0$.
\item $\Gamma$ has Gorenstein parameter $\vec{p}$ if and only if $\RHom_\Gamma(\Gamma_0,\Gamma)\in\D^G(\Gamma^{\op})_{-\vec{p}}$.
\end{enumerate}
\end{Lem}

Here, for any upper set $I\subseteq G$, we have a stable $t$-structure $\per^G\Gamma=\per^{I^c}\Gamma\perp\per^I\Gamma$. We can prove a similar result to \cite[A.8(1)]{Han24c}.

\begin{Lem}\label{perupper}
For any upper set $I\subseteq G$, we have $\per^I\Gamma=(\per^G\Gamma)_I$.
\end{Lem}

Remark that for any upper set $I\subseteq G$, we have a stable $t$-structure $\D^G(\Gamma)=\D^G(\Gamma)_I\perp\D^G(\Gamma)_{I^c}$.

\begin{Prop}\label{tstres}
For a $G_{\geq0}$-graded dg ring $\Gamma$, the following conditions are equivalent.
\begin{enumerate}
\item $\Gamma_0\in\per^G\Gamma$
\item $\Gamma_{\vec{g}}\in\per^G\Gamma$ for all $g\in G_{\geq0}$.
\item For any upper set $I\subseteq G$, the stable $t$-structure $\D^G(\Gamma)=\D^G(\Gamma)_I\perp\D^G(\Gamma)_{I^c}$ restricts to $\per^G\Gamma$.
\end{enumerate}
\end{Prop}
\begin{proof}
(3)$\Rightarrow$(1) Consider the triangle $\Gamma_{>0}\to\Gamma\to\Gamma_0\dashrightarrow$. Applying (3) to $I=G_{>0}$, we obtain (1). 

(1)$\Rightarrow$(2) Remark that for any $\vec{g}\in G_{\geq0}$, the number of elements $\vec{h}\in G_{\geq0}$ with $\vec{g}\nleq\vec{h}$ is finite. Thus as an induction hypothesis, we may assume that $\Gamma_{\vec{h}}\in\per^G\Gamma$ holds for all $h\in G_{\geq0}$ with $\vec{g}\nleq\vec{h}$. By the triangle $\Gamma_{\geq\vec{g}}\to\Gamma\to\Gamma_{\ngeq\vec{g}}\dashrightarrow$, we have $\Gamma_{\geq\vec{g}}\in\per^G\Gamma$. Then by Lemma \ref{perupper}, we have $\Gamma_{\geq \vec{g}}\in\per^{\geq\vec{g}}\Gamma$. By applying $(-)_{\vec{g}}\colon\D^G(\Gamma)\to\D(\Gamma_0)$, we have $\Gamma_{\vec{g}}\in\per\Gamma_0$. Thus we obtain $\Gamma_{\vec{g}}\in\per^G\Gamma$.

(2)$\Rightarrow$(3) It is enough to show $\Gamma_{I^c}\in\per^G\Gamma$. Since $G_{\geq0}\cap I^c$ is a finite set, this follows from (2).
\end{proof}

For a $G_{\geq0}$-graded dg ring $\Gamma$, we consider the following conditions.

\begin{enumerate}
\item[(D1)] $\Gamma_0\in\per^G\Gamma$ and $\Gamma_0^{\op}\in\per^G\Gamma^{\op}$.
\item[(D2)]  $\Gamma$ and $\Gamma^{\op}$ have Gorenstein parameter $\vec{p}\in G_{>0}$.
\end{enumerate}

\begin{Def}
In the setting (D1) and (D2), we define the {\it graded cluster category} as the Verdier quotient
\[\C^G(\Gamma):=\per^G\Gamma/\thick^G\Gamma_0.\]
\end{Def}

Remark that $\mathbb{Z}$ acts on $G$ by $n\cdot\vec{g}:=\vec{g}+n\vec{p}$ and this action satisfies the conditions (A1),(A2) and (A3). Thus for $I\in\I_G$, we write $J(I)=I\cap(I^c+\vec{p})\in\J_G$ as in Theorem \ref{upJX}.

We state one of the main theorems in this section.

\begin{Thm}\label{GBei}(Beilinson-type theorem for $G$-graded dg rings)
Assume the setting (D1) and (D2). Take $I\in\I_G$.
\begin{enumerate}
\item We have a weak semi-orthogonal decomposition
\[\per^G\Gamma=\thick^I\Gamma_0\perp\per^{J(I)}\Gamma\perp\thick^{I^c}\Gamma_0.\]
\item The composition
\[\per^{J(I)}\Gamma\hookrightarrow\per^G\Gamma\to\C^G(\Gamma)\]
is a triangle equivalence. Thus if we put $A:=[\Gamma_{\vec{g}-\vec{h}}]_{\vec{g},\vec{h}\in J(I)}$, then we have a triangle equivalence
\[\C^G(\Gamma)\simeq\per A.\]
\end{enumerate}
\end{Thm}

As in the case of (non-commutative) projective geometry, we write $\O_\Gamma\in\C^G(\Gamma)$ for the image of $\Gamma\in\D^G(\Gamma)$ in $\C^G(\Gamma)$. Remark that Theorem \ref{GBei}(2) says that $\bigoplus_{\vec{g}\in J(I)}\O_\Gamma(-\vec{g})$ is a thick generator of $\C^G(\Gamma)$.

Towards this theorem, we make some preparations.

\begin{Lem}\label{Gordual}
Under the setting (D1) and (D2),
\begin{enumerate}
\item $(\per^G\Gamma)_{\vec{g}}=\thick\Gamma_0(-\vec{g})$
%\item $\Gamma^{\op}$ also has Gorenstein parameter $\vec{p}$.
\item The duality $\RHom_\Gamma(-,\Gamma)\colon\per^G\Gamma\xrightarrow[\simeq]{}\per^G\Gamma^{\op}$ restricts to a duality $(\per^G\Gamma)_{\vec{g}}\xrightarrow[\simeq]{}(\per^G\Gamma^{\op})_{-\vec{p}-\vec{g}}$.
\end{enumerate}
\end{Lem}
\begin{proof}
These assertions can be shown in the same way as \cite[A.9]{Han24c}.
\end{proof}

We recall powerful results from \cite{IYang20}.

\begin{Prop}\cite[1.1]{IYang20}\label{IYang}
Let $\T$ be a triangulated category and $\S\subseteq\T$ a thick subcategory. Assume we have a $t$-structure $\S=\X\perp\Y$. Consider the Verdier quotient $\pi\colon\T\to\T/\S$.
\begin{enumerate}
\item For $M\in{}^\perp\Y[1]\subseteq\T$ and $N\in\X^\perp\subseteq\T$, the map
\[\pi\colon\T(M,N)\to(\T/\S)(\pi(M),\pi(N))\]
is bijective.
\item Assume furthermore that we have $t$-structures $\T=\X\perp\X^\perp={}^\perp\Y\perp\Y$. Put $\Z:=\X^\perp\cap{}^\perp\Y[1]$. Then we have $\T=\X\perp\Z\perp\Y[1]$. In particular, the composition
\[\Z\hookrightarrow\T\xrightarrow{\pi}\T/\S\]
is a triangle equivalence.
\end{enumerate}
\end{Prop}

Now we can prove Theorem \ref{GBei}.

\begin{proof}[Proof of Theorem \ref{GBei}]
We have stable $t$-structures
\[\thick^G\Gamma_0=\thick^I\Gamma_0\perp\thick^{I^c}\Gamma_0\text{ and}\]
\[\per^G\Gamma=(\per^G\Gamma)_I\perp(\per^G\Gamma)_{I^c}=\per^I\Gamma\perp\thick^{I^c}\Gamma_0.\]
By applying $\RHom_\Gamma(-,\Gamma)$ to $\per^G\Gamma^{\op}=\per^{-I^c-p}\Gamma^{\op}\perp\thick^{-I-p}\Gamma_0^{\op}$, we obtain a stable $t$-structure
\[\per^G\Gamma=\thick^I\Gamma_0\perp\per^{I^c+p}\Gamma.\]
Thus by Proposition \ref{IYang}, we obtain the desired result.
\end{proof}

Remark that the essential surjectivity of the functor $\per^{I\cap(I^c+p)}\Gamma\to\C^G(\Gamma)$ can be proven in the same way as \cite[A.10]{HI}.

As an application, we see that a result in \cite[4.12]{MM}, so called Minamoto-Mori correspondence, can be generalized.

\begin{Thm}\label{MMcorr}(Minamoto-Mori correspondence for $G$-graded dg algebras)
Let $\Gamma$ be a $G_{\geq0}$-graded dg $k$-algebra satisfying (D1) and (D2). We furthermore assume the following condition.
\begin{enumerate}
\item[(D3)] For $X\in\per^G\Gamma$ and $Y\in\thick^G\Gamma_0$, we have an isomorphism
\[\D^G(\Gamma)(X,Y)\cong D\D^G(\Gamma)(Y,X(-\vec{p})[d+1])\]
of $k$-linear space which is functorial in $X$ and $Y$ where $D=\Hom_k(-,k), d\geq1$ and $\vec{p}\in G_{>0}$.
\end{enumerate}
\begin{enumerate}
\item The functor $(-\vec{p})[d]\curvearrowright\C^G(\Gamma)$ is a Serre functor.
\end{enumerate}
In what follows, we assume $H^{>0}\Gamma=0$. Take a non-trivial upper set $I\subseteq G$ and put $\E:=\bigoplus_{\vec{g}\in J(I)}\O_\Gamma(-\vec{g})\in\C^G(\Gamma)$.
\begin{enumerate}
\setcounter{enumi}{1}
\item $\E\in\C^G(\Gamma)$ is a $d$-silting object.
\end{enumerate}
Moreover, we assume $H^{\neq0}\Gamma=0$.
\begin{enumerate}
\setcounter{enumi}{2}
\item $\E\in\C^G(\Gamma)$ is a $d$-tilting object and $\End_{\C^G(\Gamma)}(\E)\cong[H^0\Gamma_{\vec{g}-\vec{h}}]_{\vec{g},\vec{h}\in J(I)}$ is a $d$-representation infinite algebra.
\end{enumerate}
\end{Thm}
\begin{proof}
To begin with, we show that $\per^G\Gamma$ is Hom-finite. Then by Theorem \ref{GBei}(2), $\C^G(\Gamma)$ is also Hom-finite. Since $\per^G\Gamma=\thick^G\Gamma$, it is enough to show that $\D^G(\Gamma)(\Gamma,\Gamma(\vec{g})[n])\cong H^n\Gamma_{\vec{g}}$ is finite dimensional for every $\vec{g}\in G$ and $n\in\mathbb{Z}$. For $X,Y\in\thick^G\Gamma_0$, we have
\[\D^G(\Gamma)(X,Y)\cong D\D^G(\Gamma)(Y,X(-\vec{p})[d+1])\cong DD\D^G(\Gamma)(X,Y).\]
Thus $\D^G(\Gamma)(X,Y)$ must be finite dimensional. This implies $\per\Gamma_0$ is Hom-finite. Observe that by Proposition \ref{tstres} and Lemma \ref{Gordual}, we have $\Gamma_{\vec{g}}\in\per\Gamma_0$. Thus we obtain $\dim_kH^n\Gamma_{\vec{g}}<\infty$ for arbitrary $\vec{g}\in G$ and $n\in\mathbb{Z}$.

(1) Put $\S:=\thick^G\Gamma_0\subseteq\per^G\Gamma$. By \cite[1.3,1.4]{Ami09}, it is enough to show that for every $X,Y\in\per^G\Gamma$, there exists a local $\S$-envelope of $Y$ relative to $X$. Observe that there exists a finite subset $J\subseteq G$ such that $X\in\thick^J\Gamma$ holds. We can take $\vec{g}\in G$ so that for any $\vec{h}\in J$, we have $\vec{g}\nleq\vec{h}$. Consider the exact triangle $Y_{\geq\vec{g}}\to Y\to Y_{\ngeq\vec{g}}\dashrightarrow$. By Proposition \ref{tstres}, we have $Y_{\ngeq\vec{g}}\in\S$. Since $\D^G(\Gamma)(X,Y_{\geq\vec{g}})=0$ holds, the morphism $Y\to Y_{\ngeq\vec{g}}$ is a local $\S$-envelope relative to $X$.

(2) $\E\in\C^G(\Gamma)$ is silting by Theorem \ref{GBei}(2). By (1), we show $\C^G(\Gamma)(\E,\E(\vec{p})[>0])=0$. Put $\S^{\leq1}:=\{S\in\S\mid H^{\geq2}S=0\}$ and $\S^{\geq2}:=\{S\in\S\mid H^{\leq1}S=0\}$. Then we have a $t$-structure $\S=\S^{\leq1}\perp\S^{\geq2}$. By Proposition \ref{IYang}(1), for $M\in\thick^G\{\Gamma[\geq0]\}$ and $N\in\thick^G\{\Gamma[<d]\}$, the natural map $\D^G(\Gamma)(M,N)\to\C^G(\Gamma)(\pi(M),\pi(N))$ is bijective where $\pi\colon\per^G\Gamma\to\C^G(\Gamma)$ is the natural functor. Thus for any $\vec{g}\in G$ and $i<d$, we have $\C^G(\Gamma)(\O_\Gamma,\O_\Gamma(\vec{g})[i])\cong H^i\Gamma_{\vec{g}}$. Therefore for $0<i<d$, we obtain $\C^G(\Gamma)(\O_\Gamma,\O_\Gamma(\vec{g})[i])=0$. Take $\vec{g},\vec{h}\in J(I)$ and $i\geq0$. Then by (1), we have $\C^G(\Gamma)(\Gamma(-\vec{g}),\Gamma(-\vec{h}+\vec{p})[d+i])\cong D\C^G(\Gamma)(\Gamma(-\vec{h}+\vec{p})[d+i],\Gamma(-\vec{g}-\vec{p})[d])\cong H^{-i}\Gamma_{-\vec{g}+\vec{h}-2\vec{p}}$. If $-\vec{g}+\vec{h}-2\vec{p}\geq0$ holds, then we have $\vec{h}-\vec{p}\geq\vec{g}+\vec{p}$, but this contradicts to $\vec{h}-\vec{p}\notin I$ and $\vec{g}+\vec{p}\in I$. Thus we obtain $-\vec{g}+\vec{h}-2\vec{p}\ngeq0$ and $H^{-i}\Gamma_{-\vec{g}+\vec{h}-2\vec{p}}=0$.

(3) By (2), it is enough to show that $\C^G(\Gamma)(\E,\E(n\vec{p})[<0])=0$ holds for every $n\geq0$. As in the proof of (2), for any $\vec{g}\in G$, we have $\C^G(\Gamma)(\O_\Gamma,\O_\Gamma(\vec{g})[<0])\cong H^{<0}\Gamma_{\vec{g}}=0$.
\end{proof}

\begin{Ex}
Assume a $G_{\geq0}$-graded dg $k$-algebra $\Gamma$ satisfies the following conditions.
\begin{enumerate}
\item $\Gamma_0$ is proper as a dg $k$-algebra, that is, we have $\sum_{n\in\mathbb{Z}}\dim_kH^n\Gamma_0<\infty$.
\item $\Gamma$ is homologically smooth, that is, $\Gamma\in\per^G\Gamma^e$ holds where $\Gamma^e:=\Gamma^{\op}\otimes_k\Gamma$.
\item $\Gamma$ is $(d+1)$-Calabi-Yau of Gorenstein parameter $\vec{p}$, that is, $\RHom_{\Gamma^e}(\Gamma,\Gamma^e)\cong\Gamma(\vec{p})[-d-1]$ holds in $\D^G(\Gamma^e)$.
\end{enumerate}
Then $\Gamma$ satisfies (D1), (D2) and (D3).
\end{Ex}

In Theorem \ref{MMcorr}, we give a $d$-silting object of $\C^G(\Gamma)$. A partial converse holds in the following sense.

\begin{Prop}\label{classfipsilt}
Let $\Gamma$ be a $G$-graded dg $k$-algebra satisfying the conditions (D1), (D2) and (D3) with $H^{>0}\Gamma=0$. Assume $H^0\Gamma_{\vec{g}}\neq0$ holds for every $\vec{g}\geq0$. Then for a subset $J\subseteq G$, the following conditions are equivalent.
\begin{enumerate}
\item $\bigoplus_{-\vec{g}\in J}\O_\Gamma(\vec{g})\in\C^G(\Gamma)$ is presilting.
\item $J\in\widetilde{\J}_G$
\end{enumerate}
\end{Prop}
\begin{proof}
(2)$\Rightarrow$(1) follows from Theorem \ref{GBei} and \cite[1.4]{Tom25d}. We prove (1)$\Rightarrow$(2). Take elements $\vec{g},\vec{h}\in J$. Since $\bigoplus_{\vec{g}\in J}\O_\Gamma(-\vec{g})\in\C^G(\Gamma)$ is presilting, by Theorem \ref{MMcorr}(1), we have
\[H^0\Gamma_{\vec{g}-\vec{h}-\vec{p}}\cong\C^G(\Gamma)(\O_\Gamma(-\vec{g}+\vec{p}),\O_\Gamma(-\vec{h}))\cong D\C^G(\Gamma)(\O_\Gamma(-\vec{h}),\O_\Gamma(-\vec{g})[d])=0.\]
By our assumption, this forces $\vec{g}\ngeq\vec{h}+\vec{p}$.
\end{proof}

%%%%%%%%%%%%%%%%%%%%%%%%%%%%%%%%%%%%%%%%%%%%%%%%%%%%%%%%%%%%%
\section{Tilting theory for smooth toric Fano stacks of Picard number one}
%%%%%%%%%%%%%%%%%%%%%%%%%%%%%%%%%%%%%%%%%%%%%%%%%%%%%%%%%%%%%

In this section, we give a classification of tilting bundles consisting of line bundles on smooth toric Fano DM stacks of Picard rank one. Moreover, we prove that they are $d$-tilting and their endomorphism algebras are $d$-representation infinite algebras of type $\tilde{A}$. Furthermore, we can prove that all $d$-representation infinite algebras of type $\tilde{A}$ arise in this way. By using this derived equivalence, we give a new combinatorial description of $d$-APR tilting mutations, $d$-preprojective components and $d$-preinjective components of $d$-representation infinite algebras of type $\tilde{A}$.

Let $N$ be a free abelian group of rank $d$ and $P$ a lattice $d$-simplex in $N_\mathbb{R}$ containing the origin as an interior point with vertices $\{v_i\}_{i=0}^d$. Then resulting abelian group $G$ has rank one and $\vec{x_0}\cdots,\vec{x_d}\in G$ satisfy (G1), (G2) and (G3). Conversely, let $G$ be a finitely generated abelian group of rank one. Assume we are given elements $\vec{x_0}\cdots,\vec{x_d}\in G$ satisfying (G1), (G2) and (G3). Then the resulting lattice points $v_0,\cdots, v_d\in N$ become the vertex set of their convex hull. In summary, giving a lattice $d$-simplex in $N_\mathbb{R}$ containing the origin as an interior point is equivalent to giving a finitely generated abelian group $G$ of rank one and elements $\vec{x_0}\cdots,\vec{x_d}\in G$ satisfying (G1), (G2) and (G3).

Combining with the arguments in Subsection \ref{startG}, we obtain the following bijections.

\begin{Thm}\label{3corr}
We have bijections between the following three sets.
\begin{enumerate}
\item $\{(B,\gamma)\mid B\subseteq L\colon\text{cofinite subgroup},\gamma\in\mathbb{Z}_{>0}^{d+1}\text{ satisfies the conditions in Theorem \ref{chartype}})\}$
\item $\{(G,(\vec{x_i})_{i=0}^d)\mid G\colon\text{finitely generated abelian group of rank one}, \\
\vec{x_0},\cdots\vec{x_d}\in G\text{ satisfy (G1), (G2), and (G3)}\}/\cong$

Here, we write $(G,(\vec{x_i})_i)\cong(G',(\vec{x_i}')_i)$ if there exists a group isomorphism $G\cong G'$ sending each $\vec{x_i}$ to $\vec{x_i}'$.
\item $\{P\subseteq N_\mathbb{R}\colon\text{lattice $d$-simplex containing the origin as an interior point}\}/GL(N)$
\end{enumerate}
\end{Thm}

First, we give a classification of tilting bundles consisting of line bundles on smooth toric Fano stacks of Picard rank one.

\begin{Thm}\label{classfitiltrk1}
Let $P\subseteq N_\mathbb{R}$ be a lattice $d$-simplex containing the origin as an interior point and $\X:=\X(P)$. Then for a subset $J\subseteq G\cong\Pic\X$, the following conditions are equivalent.
\begin{enumerate}
\item $\E(J):=\bigoplus_{\vec{g}\in J}\O_\X(\vec{g})\in\D^b(\Coh\X)$ is a tilting bundle.
\item $J\in\J_G$
\end{enumerate}
Moreover, construct a cofinite subgroup $B\subseteq L$ and $\gamma\in\mathbb{Z}^{d+1}_{>0}$ satisfying the conditions in Theorem \ref{chartype} as Subsection \ref{startG}. Then for $J\in\J_G$, we have $\End_\X(\E(J))\cong A(B,C(J))$ where $C(J)$ is a cut (see Proposition \ref{cutJcorr}). Furthermore, every $d$-representation infinite algebra of type $\tilde{A}$ can be obtained in this way.
\end{Thm}
Thus by Theorem \ref{upJX}, tilting bundles on $\X$ consisting of line bundles correspond bijectively to non-trivial upper sets in $G$. In addition, from this theorem, we can say that the smooth toric Fano DM stacks of Picard rank one give geometric models of the higher representation infinite algebras of type $\tilde{A}$.
\begin{proof}[Proof of Theorem \ref{classfitiltrk1}]
Since $P$ is a $d$-simplex, we have $SR(P)=V(x_0,\cdots,x_d)$. Thus the equivalence (1)$\Leftrightarrow$(2) follows from Theorem \ref{GBei} and Proposition \ref{classfipsilt}. By Theorem \ref{MMcorr}(3), our $A:=\End_\X(\E(J))$ is a $d$-representation infinite algebra. Remark that this can also be deduced from Proposition \ref{SerreRI} and Theorem \ref{MMcorr}(2). Since $A\cong\End_S^G(\bigoplus_{\vec{g}\in J}S(\vec{g}))$, the quiver of $A$ has $J$ as a vertex set and $\bigsqcup_{i=0}^d\{\vec{g}\to\vec{g}+\vec{x_i}\mid\vec{g},\vec{g}+\vec{x_i}\in J\}$ as an arrow set. The relation is generated by the commutative relations. Thus we obtain $A\cong A(B,C)$. The last statement follows from Theorem \ref{3corr}.
\end{proof}

Remark that this theorem proves that our $A(B,C)$ defined by quiver with relation is certainly $d$-representation infinite.

Using the derived equivalence $\D^b(\Coh\X)\simeq\per A$ obtained by Theorem \ref{classfitiltrk1}, we can give a description of the $d$-preprojective component and the $d$-preinjective component $\P,\I\subseteq\mod A$. Remark that we have the following commutative diagram obtained by the uniqueness of the Serre functor.
\[\xymatrix{
\D^b(\Coh\X) \ar[r]_\simeq \ar[d]_{(\vec{p})} & \per A \ar[d]^{\nu_d^{-1}}\\
\D^b(\Coh\X) \ar[r]_\simeq & \per A
}\]

\begin{Prop}
Take $I\in\I_G$. Put $A:=\End_\X(\E(J(I)))$ in the notation of Theorem \ref{classfitiltrk1}. Then the derived equivalence $\D^b(\Coh\X)\simeq\per A$ restricts to equivalences
\[\add\{\O_\X(\vec{g})\mid\vec{g}\in I\}\simeq\P\text{ and }\add\{\O_\X(\vec{g})\mid\vec{g}\in I^c\}\simeq\I[-d].\]
In particular, we obtain a equivalence
\[\add\{\O_\X(\vec{g})\mid\vec{g}\in G\}\simeq\I[-d]\vee\P.\]
\end{Prop}
\begin{proof}
The assertion follows from the above commutative diagram.
\end{proof}

Next, we investigate the $d$-APR tilting module \cite{IO11} of $d$-representation infinite algebras of type $\tilde{A}$ through their geometric models. First, we give a proof to the following folklore: the endomorphism algebra of $d$-APR tilting module of a $d$-representation infinite algebra of type $\tilde{A}$ is again a $d$-representation infinite algebra of type $\tilde{A}$ having same $B$ and $\gamma$.

\begin{Thm}\label{dAPRAtilde}
Let $B\subseteq L$ be a cofinite subgroup and $\gamma\in\mathbb{Z}^{d+1}_{>0}$ a vector satisfying the conditions in Theorem \ref{chartype}. Put $G:=G(B,\gamma)$ and take $I\in\I_G$. Consider the cut $C(I)\subseteq Q_1$ in the notation of Theorem \ref{cutupcorr} and put $A:=A(B,C(I))$. Take a minimal element $\vec{m}\in I$ and let $T:=\nu_d^{-1}(e_{\vec{m}}A)\oplus\bigoplus_{\vec{g}\in J(I)\setminus\{\vec{m}\}}e_{\vec{g}}A\in\mod A$ be the $d$-APR tilting module with respect to $e_{\vec{m}}A$. Then we have
\[\End_A(T)\cong A(B,C(\mu^-_{\vec{m}}(I))).\]
\end{Thm}
\begin{proof}
If we consider the smooth toric Fano stack $\X$ constructed from $G$, we have
\[\End_A(T)\cong\End_\X(\E(J\sqcup\{\vec{m}+\vec{p})\}\setminus\{\vec{m}\}))\]
by the above commutative diagram. Thus the assertion follows from Theorem \ref{classfitiltrk1}.
\end{proof}

We emphasize that Theorem \ref{dAPRAtilde} is difficult to prove without using their geometric models. Thanks to Theorem \ref{dAPRAtilde}, we can prove that all $d$-representation infinite algebras having same $B$ and $\gamma$ are derived equivalent to each other. This is implicitly shown in \cite[5.2, 5.12]{DG}, but we could not find a proof for \cite[5.2]{DG}. Our method is different from \cite{DG} in that we use the combinatorics of upper sets.

\begin{Thm}\label{derequiAtilde}
Let $B\subseteq L$ be a cofinite subgroup and $\gamma\in\mathbb{Z}^{d+1}_{>0}$ a vector satisfying the conditions in Theorem \ref{chartype}. Take two cuts $C,C'\subseteq Q_1$. Then the two algebras $A(B,C)$ and $A(B,C')$ are derived equivalent.
\end{Thm}
\begin{proof}
By Theorem \ref{cutupcorr}, we can take $I,I'\in\I_{G(B,\gamma)}$ such that $C=C(I)$ and $C'=C(I')$ hold. Then by \cite[1.9]{Tom25d}, $I$ and $I'$ can be connected by a finite sequence of mutations by considering $I\cap I'$. Thus the assertion follows from Theorem \ref{dAPRAtilde}.
\end{proof}

We see several examples. First, as the simplest case, we see that we can obtain a classification of tilting bundles consisting of line bundles on the projective space $\mathbb{P}^d$. 

\begin{Ex}
Put $G:=\mathbb{Z}$ and $\vec{x_0}=\cdots=\vec{x_d}=1\in G$. Then the resulting toric stack $\X$ is isomorphic to the projective space $\mathbb{P}^d$. If we equip $G$ with our partial order, then the quiver of $G$ becomes the following.
\[\xymatrix{
\cdots \ar@/^1pc/[r]^{x_0}_{\scalebox{0.7}{\vdots}}\ar@/_1pc/[r]_{x_d} & \circ \ar@/^1pc/[r]^{x_0}_{\scalebox{0.7}{\vdots}}\ar@/_1pc/[r]_{x_d} & \circ \ar@/^1pc/[r]^{x_0}_{\scalebox{0.7}{\vdots}}\ar@/_1pc/[r]_{x_d} & \circ \ar@/^1pc/[r]^{x_0}_{\scalebox{0.7}{\vdots}}\ar@/_1pc/[r]_{x_d} & \circ \ar@/^1pc/[r]^{x_0}_{\scalebox{0.7}{\vdots}}\ar@/_1pc/[r]_{x_d} & \circ \ar@/^1pc/[r]^{x_0}_{\scalebox{0.7}{\vdots}}\ar@/_1pc/[r]_{x_d} & \cdots
}\]
Then there is the following just one kind of non-trivial upper sets in $G$ up to translations.
\[\xymatrix{
\circ \ar@/^1pc/[r]^{x_0}_{\scalebox{0.7}{\vdots}}\ar@/_1pc/[r]_{x_d} & \circ \ar@/^1pc/[r]^{x_0}_{\scalebox{0.7}{\vdots}}\ar@/_1pc/[r]_{x_d} & \circ \ar@/^1pc/[r]^{x_0}_{\scalebox{0.7}{\vdots}}\ar@/_1pc/[r]_{x_d} & \circ \ar@/^1pc/[r]^{x_0}_{\scalebox{0.7}{\vdots}}\ar@/_1pc/[r]_{x_d} & \cdots
}\]
Remark that we have $\vec{p}=d+1$. Therefore there is the following just one kinds of tilting bundles up to translations where there are $d+1$ vertices.
\[\xymatrix{
\circ \ar@/^1pc/[r]^{x_0}_{\scalebox{0.7}{\vdots}}\ar@/_1pc/[r]_{x_d} & \circ \ar@/^1pc/[r]^{x_0}_{\scalebox{0.7}{\vdots}}\ar@/_1pc/[r]_{x_d} & \circ \ar@/^1pc/[r]^{x_0}_{\scalebox{0.7}{\vdots}}\ar@/_1pc/[r]_{x_d} & \circ \ar@/^1pc/[r]^{x_0}_{\scalebox{0.7}{\vdots}}\ar@/_1pc/[r]_{x_d} & \cdots \ar@/^1pc/[r]^{x_0}_{\scalebox{0.7}{\vdots}}\ar@/_1pc/[r]_{x_d} & \circ
}\]
\end{Ex}

Next, we see that even when $d=1$, Theorem \ref{classfitiltrk1} gives us a new description of APR tilting mutations.
\begin{Ex}($d=1$)
(1) Put $G:=\mathbb{Z}$ and $\vec{x}=2, \vec{y}=3\in G$. Then the resulting toric stack $\X$ is isomorphic to the weighted projective stack $\mathbb{P}(2,3)$. If we equip $G$ with our partial order, then the quiver of $G$ becomes the following.
\[\xymatrix{
\cdots \ar@/^18pt/[rr]^x \ar@/^-18pt/[rrr]_y & \circ \ar@/^18pt/[rr]^x \ar@/^-18pt/[rrr]_y & \circ \ar@/^18pt/[rr]^x \ar@/^-18pt/[rrr]_y & \circ \ar@/^18pt/[rr]^x \ar@/^-18pt/[rrr]_y & \circ \ar@/^18pt/[rr]^x \ar@/^-18pt/[rrr]_y & \circ \ar@/^18pt/[rr]^x \ar@/^-18pt/[rrr]_y & \circ \ar@/^18pt/[rr]^x & \circ & \cdots
}\]
Then there is the following two kinds of non-trivial upper sets in $G$ up to translations.
\[\xymatrix{
\circ \ar@/^18pt/[rr]^x \ar@/^-18pt/[rrr]_y & \circ \ar@/^18pt/[rr]^x \ar@/^-18pt/[rrr]_y & \circ \ar@/^18pt/[rr]^x \ar@/^-18pt/[rrr]_y & \circ \ar@/^18pt/[rr]^x \ar@/^-18pt/[rrr]_y & \circ \ar@/^18pt/[rr]^x \ar@/^-18pt/[rrr]_y & \circ \ar@/^18pt/[rr]^x & \circ & \cdots
}\]
\[\xymatrix{
\circ \ar@/^18pt/[rr]^x \ar@/^-18pt/[rrr]_y & & \circ \ar@/^18pt/[rr]^x \ar@/^-18pt/[rrr]_y & \circ \ar@/^18pt/[rr]^x \ar@/^-18pt/[rrr]_y & \circ \ar@/^18pt/[rr]^x \ar@/^-18pt/[rrr]_y & \circ \ar@/^18pt/[rr]^x & \circ & \cdots
}\]
Remark that we have $\vec{p}=5$. Therefore there is the following two kinds of tilting bundles up to translations. Observe that by mutations of non-trivial upper sets in $G$, they are mutated to each other, which correspond to APR tilting mutations.
\[\xymatrix{
\circ \ar@/^18pt/[rr]^x \ar@/^-18pt/[rrr]_y & \circ \ar@/^18pt/[rr]^x \ar@/^-18pt/[rrr]_y & \circ \ar@/^18pt/[rr]^x & \circ & \circ
}\]
\[\xymatrix{
\circ \ar@/^18pt/[rr]^x \ar@/^-18pt/[rrr]_y & & \circ \ar@/^18pt/[rr]^x & \circ \ar@/^-18pt/[rrr]_y & \circ \ar@/^18pt/[rr]^x & & \circ
}\]

(2) Put $G:=\mathbb{Z}\oplus(\mathbb{Z}/2\mathbb{Z})$ and $\vec{x}=(1,0), \vec{y}=(1,1)\in G$. If we equip $G$ with our partial order, then the quiver of $G$ becomes the following.
\[\xymatrix{
\cdots \ar[r]^x \ar[dr]^y & \circ \ar[r]^x \ar[dr]^y & \circ \ar[r]^x \ar[dr]^y & \circ \ar[r]^x \ar[dr]^y & \circ \ar[r]^x \ar[dr]^y & \cdots \\
\cdots \ar[r]_x \ar[ur]^y & \circ \ar[r]_x \ar[ur]^y & \circ \ar[r]_x \ar[ur]^y & \circ \ar[r]_x \ar[ur]^y & \circ \ar[r]_x \ar[ur]^y & \cdots
}\]
Then there is the following two kinds of non-trivial upper sets in $G$ up to translations.
\[\begin{array}{c c}
\xymatrix{
\circ \ar[r]^x \ar[dr]^y & \circ \ar[r]^x \ar[dr]^y & \circ \ar[r]^x \ar[dr]^y & \cdots \\
\circ \ar[r]_x \ar[ur]^y & \circ \ar[r]_x \ar[ur]^y & \circ \ar[r]_x \ar[ur]^y & \cdots
}&\xymatrix{
 & \circ \ar[r]^x \ar[dr]^y & \circ \ar[r]^x \ar[dr]^y & \cdots\\
\circ \ar[r]_x \ar[ur]^y & \circ \ar[r]_x \ar[ur]^y & \circ \ar[r]_x \ar[ur]^y & \cdots
}\end{array}\]
Remark that we have $\vec{p}=(2,1)$. Therefore there is the following two kinds of tilting bundles up to translations. Observe that by mutations of non-trivial upper sets in $G$, they are mutated to each other, which correspond to APR tilting mutations.
\[\begin{array}{c c}
\xymatrix{
\circ \ar[r]^x \ar[dr]^y & \circ \\
\circ \ar[r]_x \ar[ur]^y & \circ
}&\xymatrix{
 & \circ \ar[dr]^y \\
\circ \ar[r]_x \ar[ur]^y & \circ \ar[r]_x & \circ
}\end{array}\]
\end{Ex}

Finally, we see a $2$-dimensional example which cannot be obtained as a weighted projective space in the sense of \cite{HIMO}.

\begin{Ex}($d=2$)
Put $G:=\mathbb{Z}\oplus(\mathbb{Z}/2\mathbb{Z}), \vec{x}=\vec{y}=(1,0), \vec{z}=(1,1)\in G$. If we equip $G$ with our partial order, then the quiver of $G$ becomes the following.
\[\xymatrix{
\cdots \ar@2[r]^x_y \ar[dr]^z & \circ \ar@2[r]^x_y \ar[dr]^z & \circ \ar@2[r]^x_y \ar[dr]^z & \circ \ar@2[r]^x_y \ar[dr]^z & \circ \ar@2[r]^x_y \ar[dr]^z & \cdots \\
\cdots \ar@2[r]^x_y \ar[ur]^z & \circ \ar@2[r]^x_y \ar[ur]^z & \circ \ar@2[r]^x_y \ar[ur]^z & \circ \ar@2[r]^x_y \ar[ur]^z & \circ \ar@2[r]^x_y \ar[ur]^z & \cdots
}\]
Then there is the following two kinds of non-trivial upper sets in $G$ up to translations.
\[\begin{array}{c c}
\xymatrix{
\circ \ar@2[r]^x_y \ar[dr]^z & \circ \ar@2[r]^x_y \ar[dr]^z & \circ \ar@2[r]^x_y \ar[dr]^z & \circ \ar@2[r]^x_y \ar[dr]^z & \cdots \\
\circ \ar@2[r]^x_y \ar[ur]^z & \circ \ar@2[r]^x_y \ar[ur]^z & \circ \ar@2[r]^x_y \ar[ur]^z & \circ \ar@2[r]^x_y \ar[ur]^z & \cdots
}&\xymatrix{
 & \circ \ar@2[r]^x_y \ar[dr]^z & \circ \ar@2[r]^x_y \ar[dr]^z & \circ \ar@2[r]^x_y \ar[dr]^z & \cdots \\
\circ \ar@2[r]^x_y \ar[ur]^z & \circ \ar@2[r]^x_y \ar[ur]^z & \circ \ar@2[r]^x_y \ar[ur]^z & \circ \ar@2[r]^x_y \ar[ur]^z & \cdots
}\end{array}\]
Remark that we have $\vec{p}=(3,1)$. Therefore there is the following two kinds of tilting bundles up to translations. Observe that by mutations of non-trivial upper sets in $G$, they are mutated to each other, which correspond to $2$-APR tilting mutations.
\[\begin{array}{c c}
\xymatrix{
\circ \ar@2[r]^x_y \ar[dr]^z & \circ \ar@2[r]^x_y \ar[dr]^z & \circ \\
\circ \ar@2[r]^x_y \ar[ur]^z & \circ \ar@2[r]^x_y \ar[ur]^z & \circ
}&\xymatrix{
 & \circ \ar@2[r]^x_y \ar[dr]^z & \circ \ar[dr]^z \\
\circ \ar@2[r]^x_y \ar[ur]^z & \circ \ar@2[r]^x_y \ar[ur]^z & \circ \ar@2[r]^x_y & \circ
}\end{array}\]
\end{Ex}

%%%%%%%%%%%%%%%%%%%%%%%%%%%%%%%%%%%%%%%%%%%%%%%%%%%%%%%%%%%%%
\section{Tilting theory for smooth toric Fano stacks of Picard number two}
%%%%%%%%%%%%%%%%%%%%%%%%%%%%%%%%%%%%%%%%%%%%%%%%%%%%%%%%%%%%%

In this section, we prove the existence and give a classification of $d$-tilting bundles consisting of line bundles on smooth toric Fano stacks of Picard rank two.

Let $N$ be a free abelian group of rank $d$ and $P$ a simplicial lattice polytope in $N_\mathbb{R}$ containing the origin as an interior point with $d+2$ vertices $\{v_i\}_{i=1}^{d+2}$. Then resulting abelian group $G$ has rank two and $\vec{x_1}\cdots,\vec{x_{d+2}}\in G$ satisfy (G1), (G2) and (G3). We put $\vec{p}:=\sum_{i=1}^{d+2}\vec{x_i}\in G$ and $\pi\colon G\to H:=G/\mathbb{Z}\vec{p}\to H/H_{\rm{tors}}\cong\mathbb{Z}$. Since $P$ is simplicial, we may assume
\[\pi(\vec{x_i})\left\{
\begin{array}{ll}
>0 & (1\leq i\leq l)\\
<0 & (l+1\leq i\leq l+l'=d+2)
\end{array}
\right..\]
Here, since $P$ is convex, we can deduce $l\geq2$ and $l'\geq2$ by the same arguments as \cite{Tom25d}. Conversely, if $G$ is a finitely generated abelian group of rank two and $\vec{x_1}\cdots,\vec{x_{d+2}}\in G$ satisfy all the properties above, then the resulting lattice points $v_1,\cdots, v_{d+2}\in N$ become the vertex set of their convex hull which is a simplicial lattice polytope.

Let $q\colon G\to H$ be the natural surjection. Then $q(\vec{x_1}), \cdots, q(\vec{x_l}), q(-\vec{x_{l+1}}),\cdots, q(-\vec{x_{d+2}})\in H$ satisfy (G1), (G2) and (G3). Thus $H$ has the following partial order.
\[h_1\geq h_2\Leftrightarrow h_1-h_2\in\sum_{i=1}^l\mathbb{Z}_{\geq0}q(\vec{x_i})+\sum_{j=1}^{l'}\mathbb{Z}_{\geq0}q(-\vec{x_{l+j}})\]
Put $s:=\sum_{i=1}^lq(\vec{x_i})=\sum_{j=1}^{l'}q(-\vec{x_{l+j}})\in H$. Then $\mathbb{Z}$ acts on $G$ by $n\cdot h:=h+ns$. This action satisfies the conditions (A1),(A2) and (A3).

\begin{Lem}\label{vancoh}
For $\vec{g}\in G$, the following conditions are equivalent.
\begin{enumerate}
\item For any $0\leq r\leq d-2$, and $a\in\mathbb{Z}^n$ with $\sum_ia_i\vec{x_i}\in\vec{g}+\mathbb{Z}\vec{p}$, we have $\tilde{H}_r(X_a;k)=0$.
\item $q(\vec{g})\ngeq s$ and $q(\vec{g})\nleq -s$ hold.
\end{enumerate}
\end{Lem}
\begin{proof}
This can be proved in the same way as \cite[4.1]{Tom25d} by using \cite[4.2]{Tom25d}. One thing we have to remark is that if $a\in\mathbb{Z}_{\geq0}^n$, then $X_a$ is homeomorphic to the $(d-1)$-dimensional sphere $S^{d-1}$.
\end{proof}

\begin{Cor}\label{rigid}
For a subset $J'\subseteq G$, the following conditions are equivalent.
\begin{enumerate}
\item For any $\vec{g},\vec{h}\in J', n\in\mathbb{Z}$ and $0<r<d$, we have $\Ext_\X^r(\O_\X(\vec{g}),\O_\X(\vec{h}+n\vec{p}))=0$.
\item For any $\vec{g},\vec{h}\in J', 0\leq r\leq d-2$ and $a\in\mathbb{Z}^n$ with $\sum_ia_i\vec{x_i}\in\vec{h}-\vec{g}+\mathbb{Z}\vec{p}$, we have $\tilde{H}_r(X_a;k)=0$.
\item For any $\vec{g},\vec{h}\in J'$, we have $q(\vec{g})\ngeq q(\vec{h})+s$.
\item There exists $I\in\I_H$ such that $q(J')\subseteq J(I)$ holds.
\end{enumerate}
\end{Cor}
\begin{proof}
(1)$\Leftrightarrow$(2) follows from Proposition \ref{calcoh}. (2)$\Leftrightarrow$(3) follows from Lemma \ref{vancoh}. (3)$\Leftrightarrow$(4) is \cite[1.4]{Tom25d}.
\end{proof}

We show the following lemma which is an important step to prove that our tilting bundles certainly generate the derived category.

\begin{Lem}\label{CTthick}
For any $I\in\I_H$, we have
\[\thick\{\O_\X(\vec{g})\mid q(\vec{g})\in J(I)\}=\D^b(\Coh\X).\]
\end{Lem}
\begin{proof}
Put $\T:=\thick\{\O_\X(\vec{g})\mid q(\vec{g})\in J(I)\}\subseteq\D^b(\Coh\X)$. Since $\thick\{\O_\X(\vec{g})\mid\vec{g}\in G\}=\D^b(\Coh\X)$ holds, it is enough to show that $\O_\X(\vec{g})\in\T$ holds for any $\vec{g}\in G$. Consider the graded Koszul complex of a regular sequence $x_1,\cdots, x_l\in S$.
\[0\to S(-\vec{x_1}-\cdots-\vec{x_l})\to\cdots\to S\to S/(x_1,\cdots, x_l)\to0\]
This yields an exact sequence
\[0\to\O_\X(-\vec{x_1}-\cdots-\vec{x_l})\to\cdots\to\O_\X\to0\]
in $\Coh\X$. Let $m\in I$ be a minimal element and take $\vec{m}\in q^{-1}(m)$. We have a short exact sequence
\[0\to\O_\X(\vec{m})\to\cdots\to\O_\X(\vec{m}+\vec{x_1}+\cdots+\vec{x_l})\to0.\]
By the minimality of $m\in I$, for any proper subset $\Lambda\subsetneq\{1,\cdots,l\}$, we have $\vec{m}+\sum_{i\in\Lambda}\vec{x_i}\in q^{-1}(J(I))$. Thus $\O_\X(\vec{m}+\vec{x_1}+\cdots+\vec{x_l})\in\T$ holds. This means that for any $\vec{g}\in q^{-1}(J(\mu_{\vec{m}}^-(I))$, we have $\O_\X(\vec{g})\in\T$. Moreover, the converse holds: for $I'\in\I_\H$, if $\O_\X(\vec{g})\in\T$ holds for any $\vec{g}\in q^{-1}(J(\mu_{\vec{m}}^-(I')))$, then we have $\O_\X(\vec{g})\in\T$ for any $\vec{g}\in q^{-1}(J(I'))$. Thus by \cite[1.10]{Tom25d}, we obtain $\O_\X(\vec{g})\in\T$ for any $\vec{g}\in G$.
\end{proof}

Since $G$ has a partial order, for $J\in\J_H$, we can equip $q^{-1}(J)\subseteq G$ with a partial order. Then observe that for $\vec{g},\vec{h}\in q^{-1}(J)$, we have
\[\vec{g}\leq\vec{h}\Leftrightarrow S_{\vec{h}-\vec{g}}\neq0\Leftrightarrow\Hom_\X(\O_\X(\vec{g}),\O_\X(\vec{h}))\neq0.\]
Then $\mathbb{Z}$ acts on $q^{-1}(J)$ by $n\cdot\vec{g}:=\vec{g}+n\vec{p}$. This action satisfies the conditions (A1),(A2) and (A3).

\begin{Thm}\label{classfitiltrk2}
Let $P\subseteq N_\mathbb{R}$ be a lattice polytope with $d+2$ vertices containing the origin as an interior point and $\X:=\X(P)$. Then for a subset $J'\subseteq G\cong\Pic\X$, the following conditions are equivalent.
\begin{enumerate}
\item $\E(J'):=\bigoplus_{\vec{g}\in J'}\O_\X(\vec{g})\in\D^b(\Coh\X)$ is a $d$-tilting bundle.
\item There exists $J\in\J_H$ containing $q(J')$ such that $J'\in\J_{q^{-1}(J)}$ holds.
\end{enumerate}
\end{Thm}
Thus by Theorem \ref{upJX}, $d$-tilting bundles on $\X$ consisting of line bundles correspond bijectively to the pairs $(I,I')$ where $I\in\I_H$ and $I'\in\I_{q^{-1}(J(I))}$.
\begin{proof}[Proof of Theorem \ref{classfitiltrk2}]
(2)$\Rightarrow$(1) Take $J\in\J_H$ and $J'\in\J_{q^{-1}(J)}$. From Corollary \ref{rigid}, we have
\[\Ext_\X^r(\O_\X(\vec{g}),\O_\X(\vec{h}+n\vec{p}))=0\]
for any $\vec{g},\vec{h}\in J', n\in\mathbb{Z}$ and $0<r<d$. In addition, for $\vec{g},\vec{h}\in J'$ and $n\geq0$, we have
\[\Ext_\X^d(\O_\X(\vec{g}),\O_\X(\vec{h}+n\vec{p}))\cong D\Hom_\X(\O_\X(\vec{h}+(n+1)\vec{p}),\O_\X(\vec{g}))=0\]
since $\vec{g}\ngeq\vec{h}+(n+1)\vec{p}$ holds by $J'\in\J_{q^{-1}(J)}$. Thus it is enough to show that $\thick\E(J')=\D^b(\Coh\X)$ holds. By Lemma \ref{CTthick}, it is enough to show that $\O_\X(\vec{g})\in\thick\E(J')$ holds only for $\vec{g}\in q^{-1}(J)$.

Put $\T:=\thick\{S(\vec{g})\mid q(\vec{g})\in J\}\subseteq\per^GS$ and $Q:=\bigoplus_{\vec{g}\in J'}S(\vec{g})\in\proj^GS$. If we define a $\mathbb{Z}_{\geq0}$-graded algebra $\Gamma$ as
\[\Gamma:=\bigoplus_{n\in\mathbb{Z}}\Hom_S^G(Q,Q(n\vec{p}))=\bigoplus_{n\in\mathbb{Z}_{\geq0}}\Hom_S^G(Q,Q(n\vec{p})),\]
then we have a triangle equivalence $F:=\bigoplus_{n\in\mathbb{Z}}\RHom_S^G(Q,-(n\vec{p}))\colon\T\xrightarrow[\simeq]{}\per^\mathbb{Z}\Gamma$. Since $\Gamma\cong\End_S^{G/\mathbb{Z}\vec{p}}(Q)$ holds as an ungraded algebra, by \cite[4.3]{Tom25d}, our $\Gamma$ gives a non-commutative crepant resolution of $R:=S^{(\mathbb{Z}\vec{p})}=\bigoplus_{n\geq0}S_{n\vec{p}}$. It is easy to check that the $\mathbb{Z}_{\geq0}$-graded algebra $R$ has Gorenstein parameter $1$. Observe that $\Gamma\cong\bigoplus_{\vec{g},\vec{h}\in J'}S(\vec{h}-\vec{g})^{(\mathbb{Z}\vec{p})}$ holds. By \cite[2.2]{Tom25d}, we have $\Hom_R(S(\vec{h}-\vec{g})^{(\mathbb{Z}\vec{p})},R)\cong S(\vec{g}-\vec{h})^{(\mathbb{Z}\vec{p})}$ and this isomorphism preserves $\mathbb{Z}$-grading. Thus we have $\Hom_R(\Gamma,R)\cong\Gamma$ as a $\mathbb{Z}$-graded $R$-module and we can check that this is an isomorphism even as a $\Gamma$-bimodule. Therefore by \cite[3.15]{HI}, we can conclude that $\Gamma$ has Gorenstein parameter $1$. This means that if we write the graded minimal projective resolution of $\Gamma/\rad\Gamma$ as
\[0\to P_{d+1}\to\cdots\to P_0\to\Gamma/\rad\Gamma\to0,\]
then we have $P_0=\Gamma$ and $P_{d+1}=\Gamma(-1)$. We write this resolution as $P_\bullet$ and then $P_\bullet\cong\Gamma/\rad\Gamma$ holds in $\per^\mathbb{Z}\Gamma$. Here, by Lemma \ref{invflvanq}, the image of $F^{-1}(P_\bullet)\cong F^{-1}(\Gamma/\rad\Gamma)$ in $\D^b(\Coh\X)$ vanishes. Take a minimal element $\vec{m}\in J'$. Then $F^{-1}(P_\bullet(1))$ has a direct summand of the form
\[S(\vec{m})\to Q_d\to\cdots Q_1\to S(\vec{m}+\vec{p})\]
with $Q_i\in\add Q$. This means that there exists an exact sequence
\[0\to\O_\X(\vec{m})\to\E_d\to\cdots\to\E_1\to\O_\X(\vec{m}+\vec{p})\to0\]
in $\Coh\X$ where $\E_i\in\add\E(J')$. Thus we obtain $\O_\X(\vec{m}+\vec{p})\in\thick\E(J')$. By combining with the dual argument, we can conclude that $\O_\X(\vec{g})\in\thick\E(J')$ holds for any $\vec{g}\in q^{-1}(J)$ by \cite[1.10]{Tom25d}.

(1)$\Rightarrow$(2) Since $\Ext_\X^r(\O_\X(\vec{g}),\O_\X(\vec{h}+n\vec{p}))=0$ holds for any $\vec{g},\vec{h}\in J', n\in\mathbb{Z}$ and $0<r<d$, by Corollary \ref{rigid}, there exists $I\in\I_H$ such that $q(J')\subseteq J(I)$ holds. For $\vec{g},\vec{h}\in J'$, we have
\[0=D\Ext^d_\X(\O_\X(\vec{g}),\O_\X(\vec{h}))\cong\Hom_\X(\O_\X(\vec{h}+\vec{p}),\O_\X(\vec{g}))\cong S_{\vec{g}-\vec{h}-\vec{p}}.\]
This means $\vec{g}\ngeq\vec{h}+\vec{p}$. Thus $J'\in\widetilde{\J}_{q^{-1}(J(I))}$ holds and by \cite[1.4]{Tom25d}, there exists $J''\in\J_{q^{-1}(J(I))}$ such that we have $J'\subseteq J''$. By (2)$\Rightarrow$(1), $\E(J'')\in\D^b(\Coh\X)$ is a tilting object. This forces $J'=J''$.
\end{proof}

\begin{Lem}\label{invflvanq}
Take $J\in\J_H$. For $M\in\mod^GS$, if $\#\{\vec{g}\in q^{-1}(J)\mid M_{\vec{g}}\neq0\}<\infty$ holds, then we have $M\in\mod^G_{SR(P)}S$.
\end{Lem}
\begin{proof}
Observe that if we put $\mathfrak{a}:=(x_1,\cdots,x_l)(x_{l+1},\cdots,x_{d+2})$, then $SR(P)=V(\mathfrak{a})$ holds. Thus to get the assertion, it is enough to show that for any homogeneous element $m\in M$, there exists $n\geq0$ such that $\mathfrak{a}^nm=0$. Take $\vec{g}\in G$ with $m\in M_{\vec{g}}$. Remark that there exists $n\in\mathbb{Z}$ such that $q(\vec{g})+ns\in J$ holds. If $n\geq0$ (respectively, $n\leq0$), then $(x_1\cdots x_l)^nm\in M_{q(\vec{g})+ns}$ (respectively, $(x_{l+1}\cdots x_{d+2})^{-n}m\in M_{q(\vec{g})+ns}$) holds. Thus we may assume first that $q(\vec{g})\in J$ holds.

Take $1\leq i\leq l$ and $1\leq j\leq l'$. Then there exists $a_{ij},b_{ij}>0$ such that $a_{ij}q(\vec{x_i})+b_{ij}q(\vec{x_{l+j}})=0\in H$ holds. Take $c>0$ such that $a_{ij}\vec{x_i}+b_{ij}\vec{x_{l+j}}=c\vec{p}$ holds. Then for any $n\geq0$, we have $(x_i^{a_{ij}}x_{l+j}^{b_{ij}})^nm\in M_{\vec{g}+nc\vec{p}}$ and $\vec{g}+nc\vec{p}\in q^{-1}(J)$. By our assumption, there exists $n\geq0$ such that $(x_i^{a_{ij}}x_{l+j}^{b_{ij}})^nm=0$. This proves the assertion.
\end{proof}

By our proof, we can see that $J$ corresponds to an NCCR of $R$ and that $J'$ corresponds to a cut of the quiver of this NCCR.

We classify $2$-tilting bundles consisting of line bundles on some examples of toric stacky surfaces and determine their quivers by using Theorem \ref{classfitiltrk2}. We remark that all the endomorphism algebras of the obtained $2$-tilting bundles are $2$-representation infinite algebras by Proposition \ref{SerreRI}.

%\begin{comment}%%%%%%

\begin{Ex}($d=2$)
We see that Theorem \ref{classfitiltrk2} gives a classification of $2$-tilting bundles consisting of line bundles on the Hirzebruch surfaces $\mathbb{P}^1\times\mathbb{P}^1$ and $\Sigma_1$. This is also known as the classification of geometric helices.

(1) Put $G:=\mathbb{Z}^2$ and $\vec{x}=\vec{y}=(1,0),\vec{z}=\vec{w}=(0,1)\in G$. We view $S:=k[x,y,z,w]$ as a $G$-graded $k$-algebra. Then the resulting toric stack $\X$ is isomorphic to $\mathbb{P}^1\times\mathbb{P}^1$. We have $\vec{p}=(2,2)$ and $H=G/\mathbb{Z}\vec{p}\cong\mathbb{Z}\oplus(\mathbb{Z}/2\mathbb{Z}); (a,b)+\mathbb{Z}\vec{p}\mapsto(a-b,a+2\mathbb{Z})$. If we equip $H$ with our partial order, then the quiver of $H$ becomes the following.
\[\xymatrix{
\cdots \ar@<0.25ex>[r]^{-z} \ar@<-0.25ex>[r]_{-w} \ar@<0.25ex>[dr]^x \ar@<-0.25ex>[dr]_y & \circ \ar@<0.25ex>[r]^{-z} \ar@<-0.25ex>[r]_{-w} \ar@<0.25ex>[dr]^x \ar@<-0.25ex>[dr]_y & \circ \ar@<0.25ex>[r]^{-z} \ar@<-0.25ex>[r]_{-w} \ar@<0.25ex>[dr]^x \ar@<-0.25ex>[dr]_y & \circ \ar@<0.25ex>[r]^{-z} \ar@<-0.25ex>[r]_{-w} \ar@<0.25ex>[dr]^x \ar@<-0.25ex>[dr]_y & \circ \ar@<0.25ex>[r]^{-z} \ar@<-0.25ex>[r]_{-w} \ar@<0.25ex>[dr]^x \ar@<-0.25ex>[dr]_y & \cdots\\
\cdots \ar@<0.25ex>[r]^{-z} \ar@<-0.25ex>[r]_{-w} \ar@<0.25ex>[ur]^x \ar@<-0.25ex>[ur]_y & \circ \ar@<0.25ex>[r]^{-z} \ar@<-0.25ex>[r]_{-w} \ar@<0.25ex>[ur]^x \ar@<-0.25ex>[ur]_y & \circ \ar@<0.25ex>[r]^{-z} \ar@<-0.25ex>[r]_{-w} \ar@<0.25ex>[ur]^x \ar@<-0.25ex>[ur]_y & \circ \ar@<0.25ex>[r]^{-z} \ar@<-0.25ex>[r]_{-w} \ar@<0.25ex>[ur]^x \ar@<-0.25ex>[ur]_y & \circ \ar@<0.25ex>[r]^{-z} \ar@<-0.25ex>[r]_{-w} \ar@<0.25ex>[ur]^x \ar@<-0.25ex>[ur]_y & \cdots
}\]
Then there are the following two kinds of non-trivial upper sets in $H$ up to translations.
\[\begin{array}{c c}
\xymatrix{
\circ \ar@<0.25ex>[r]^{-z} \ar@<-0.25ex>[r]_{-w} \ar@<0.25ex>[dr]^x \ar@<-0.25ex>[dr]_y & \circ \ar@<0.25ex>[r]^{-z} \ar@<-0.25ex>[r]_{-w} \ar@<0.25ex>[dr]^x \ar@<-0.25ex>[dr]_y & \circ \ar@<0.25ex>[r]^{-z} \ar@<-0.25ex>[r]_{-w} \ar@<0.25ex>[dr]^x \ar@<-0.25ex>[dr]_y & \cdots\\
\circ \ar@<0.25ex>[r]^{-z} \ar@<-0.25ex>[r]_{-w} \ar@<0.25ex>[ur]^x \ar@<-0.25ex>[ur]_y & \circ \ar@<0.25ex>[r]^{-z} \ar@<-0.25ex>[r]_{-w} \ar@<0.25ex>[ur]^x \ar@<-0.25ex>[ur]_y & \circ \ar@<0.25ex>[r]^{-z} \ar@<-0.25ex>[r]_{-w} \ar@<0.25ex>[ur]^x \ar@<-0.25ex>[ur]_y & \cdots
}&\xymatrix{
 & \circ \ar@<0.25ex>[r]^{-z} \ar@<-0.25ex>[r]_{-w} \ar@<0.25ex>[dr]^x \ar@<-0.25ex>[dr]_y & \circ \ar@<0.25ex>[r]^{-z} \ar@<-0.25ex>[r]_{-w} \ar@<0.25ex>[dr]^x \ar@<-0.25ex>[dr]_y & \cdots\\
\circ \ar@<0.25ex>[r]^{-z} \ar@<-0.25ex>[r]_{-w} \ar@<0.25ex>[ur]^x \ar@<-0.25ex>[ur]_y & \circ \ar@<0.25ex>[r]^{-z} \ar@<-0.25ex>[r]_{-w} \ar@<0.25ex>[ur]^x \ar@<-0.25ex>[ur]_y & \circ \ar@<0.25ex>[r]^{-z} \ar@<-0.25ex>[r]_{-w} \ar@<0.25ex>[ur]^x \ar@<-0.25ex>[ur]_y & \cdots
}\end{array}\]
Remark that the isomorphism $H\cong\mathbb{Z}\oplus(\mathbb{Z}/2\mathbb{Z})$ sends $s$ to $(2,0)$. Thus there are the following two kinds of sets in $\J_H$ up to translations. Here, we draw the quiver of the algebra $\End_S^H(\bigoplus_{h\in J}S(h))$ for $J\in\J_H$.
\[\begin{array}{c c}
\xymatrix{
\circ \ar@<0.25ex>[ddrr]^x \ar@<-0.25ex>[ddrr]_y & & \circ \ar@<0.25ex>[ll]^z \ar@<-0.25ex>[ll]_w \\
\\
\circ \ar@<0.25ex>[uurr]^x \ar@<-0.25ex>[uurr]_y & & \circ \ar@<0.25ex>[ll]^z \ar@<-0.25ex>[ll]_w
}&\xymatrix{
 & & \circ \ar@<0.6ex>[dd]^{\rotatebox{270}{\scriptsize$xz,xw$}} \ar@<0.2ex>[dd] \ar@<-0.2ex>[dd] \ar@<-0.6ex>[dd]_{\rotatebox{270}{\scriptsize$yz,yw$}} & & \circ \ar@<0.25ex>[ll]^z \ar@<-0.25ex>[ll]_w \\
\\
\circ \ar@<0.25ex>[uurr]^x \ar@<-0.25ex>[uurr]_y & & \circ \ar@<0.25ex>[uurr]^x \ar@<-0.25ex>[uurr]_y \ar@<0.25ex>[ll]^z \ar@<-0.25ex>[ll]_w 
}\end{array}\]
The quivers of $q^{-1}(J)$ become the following.
\[\begin{array}{c c}
\xymatrix{
 & & & \circ \ar@<0.25ex>[r]^x \ar@<-0.25ex>[r]_y & \cdots \\
 & & \circ \ar@<0.25ex>[r]^x \ar@<-0.25ex>[r]_y & \circ \ar@<0.25ex>[u]^z \ar@<-0.25ex>[u]_w \\
 & \circ \ar@<0.25ex>[r]^x \ar@<-0.25ex>[r]_y & \circ \ar@<0.25ex>[u]^z \ar@<-0.25ex>[u]_w \\
\cdots \ar@<0.25ex>[r]^x \ar@<-0.25ex>[r]_y & \circ \ar@<0.25ex>[u]^z \ar@<-0.25ex>[u]_w
}&\xymatrix{
 & & & & & \rotatebox{90}{$\ddots$} \\
 & & & \circ \ar@<0.25ex>[r]^x \ar@<-0.25ex>[r]_y & \circ \ar@<0.6ex>[ur]^{xz,xw} \ar@<0.2ex>[ur] \ar@<-0.2ex>[ur] \ar@<-0.6ex>[ur]_{yz,yw} \\
 & & & \circ \ar@<0.25ex>[r]^x \ar@<-0.25ex>[r]_y \ar@<0.25ex>[u]^z \ar@<-0.25ex>[u]_w & \circ \ar@<0.25ex>[u]^z \ar@<-0.25ex>[u]_w \\
 & \circ \ar@<0.25ex>[r]^x \ar@<-0.25ex>[r]_y & \circ \ar@<0.6ex>[ur]^{xz,xw} \ar@<0.2ex>[ur] \ar@<-0.2ex>[ur] \ar@<-0.6ex>[ur]_{yz,yw} \\
 & \circ \ar@<0.25ex>[r]^x \ar@<-0.25ex>[r]_y \ar@<0.25ex>[u]^z \ar@<-0.25ex>[u]_w & \circ \ar@<0.25ex>[u]^z \ar@<-0.25ex>[u]_w \\
\rotatebox{90}{$\ddots$} \ar@<0.6ex>[ur]^{xz,xw} \ar@<0.2ex>[ur] \ar@<-0.2ex>[ur] \ar@<-0.6ex>[ur]_{yz,yw}
}\end{array}\]
In the first case, there are the following two kinds of non-trivial upper sets in $q^{-1}(J)$ up to translations by $\vec{p}$.
\[\begin{array}{c c}
\xymatrix{
 & & \circ \ar@<0.25ex>[r]^x \ar@<-0.25ex>[r]_y & \cdots \\
 & \circ \ar@<0.25ex>[r]^x \ar@<-0.25ex>[r]_y & \circ \ar@<0.25ex>[u]^z \ar@<-0.25ex>[u]_w \\
\circ \ar@<0.25ex>[r]^x \ar@<-0.25ex>[r]_y & \circ \ar@<0.25ex>[u]^z \ar@<-0.25ex>[u]_w
}&\xymatrix{
 & \circ \ar@<0.25ex>[r]^x \ar@<-0.25ex>[r]_y & \cdots \\
\circ \ar@<0.25ex>[r]^x \ar@<-0.25ex>[r]_y & \circ \ar@<0.25ex>[u]^z \ar@<-0.25ex>[u]_w \\
\circ \ar@<0.25ex>[u]^z \ar@<-0.25ex>[u]_w
}\end{array}\]
Therefore there are the following two kinds of $2$-tilting bundles up to translations. Observe that by mutations of non-trivial upper sets in $q^{-1}(J)$, they are mutated to each other, which correspond to $2$-APR tilting mutations.
\[\begin{array}{c c}
\xymatrix{
 & \circ \ar@<0.25ex>[r]^x \ar@<-0.25ex>[r]_y & \circ \\
\circ \ar@<0.25ex>[r]^x \ar@<-0.25ex>[r]_y & \circ \ar@<0.25ex>[u]^z \ar@<-0.25ex>[u]_w
}&\xymatrix{
 & \circ \\
\circ \ar@<0.25ex>[r]^x \ar@<-0.25ex>[r]_y & \circ \ar@<0.25ex>[u]^z \ar@<-0.25ex>[u]_w \\
\circ \ar@<0.25ex>[u]^z \ar@<-0.25ex>[u]_w
}\end{array}\]
In the second case, there are the following four kinds of non-trivial upper sets in $q^{-1}(J)$ up to translations by $\vec{p}$.
\[\begin{array}{c c}
\xymatrix{
 & & & & \rotatebox{90}{$\ddots$} \\
 & & \circ \ar@<0.25ex>[r]^x \ar@<-0.25ex>[r]_y & \circ \ar@<0.6ex>[ur]^{xz,xw} \ar@<0.2ex>[ur] \ar@<-0.2ex>[ur] \ar@<-0.6ex>[ur]_{yz,yw} \\
 & & \circ \ar@<0.25ex>[r]^x \ar@<-0.25ex>[r]_y \ar@<0.25ex>[u]^z \ar@<-0.25ex>[u]_w & \circ \ar@<0.25ex>[u]^z \ar@<-0.25ex>[u]_w \\
\circ \ar@<0.25ex>[r]^x \ar@<-0.25ex>[r]_y & \circ \ar@<0.6ex>[ur]^{xz,xw} \ar@<0.2ex>[ur] \ar@<-0.2ex>[ur] \ar@<-0.6ex>[ur]_{yz,yw} \\
\circ \ar@<0.25ex>[r]^x \ar@<-0.25ex>[r]_y \ar@<0.25ex>[u]^z \ar@<-0.25ex>[u]_w & \circ \ar@<0.25ex>[u]^z \ar@<-0.25ex>[u]_w
}&\xymatrix{
 & & & & \rotatebox{90}{$\ddots$} \\
 & & \circ \ar@<0.25ex>[r]^x \ar@<-0.25ex>[r]_y & \circ \ar@<0.6ex>[ur]^{xz,xw} \ar@<0.2ex>[ur] \ar@<-0.2ex>[ur] \ar@<-0.6ex>[ur]_{yz,yw} \\
 & & \circ \ar@<0.25ex>[r]^x \ar@<-0.25ex>[r]_y \ar@<0.25ex>[u]^z \ar@<-0.25ex>[u]_w & \circ \ar@<0.25ex>[u]^z \ar@<-0.25ex>[u]_w \\
\circ \ar@<0.25ex>[r]^x \ar@<-0.25ex>[r]_y & \circ \ar@<0.6ex>[ur]^{xz,xw} \ar@<0.2ex>[ur] \ar@<-0.2ex>[ur] \ar@<-0.6ex>[ur]_{yz,yw} \\
 & \circ \ar@<0.25ex>[u]^z \ar@<-0.25ex>[u]_w
}\end{array}\]
\[\begin{array}{c c}
\xymatrix{
 & & & & \rotatebox{90}{$\ddots$} \\
 & & \circ \ar@<0.25ex>[r]^x \ar@<-0.25ex>[r]_y & \circ \ar@<0.6ex>[ur]^{xz,xw} \ar@<0.2ex>[ur] \ar@<-0.2ex>[ur] \ar@<-0.6ex>[ur]_{yz,yw} \\
 & & \circ \ar@<0.25ex>[r]^x \ar@<-0.25ex>[r]_y \ar@<0.25ex>[u]^z \ar@<-0.25ex>[u]_w & \circ \ar@<0.25ex>[u]^z \ar@<-0.25ex>[u]_w \\
\circ \ar@<0.25ex>[r]^x \ar@<-0.25ex>[r]_y & \circ \ar@<0.6ex>[ur]^{xz,xw} \ar@<0.2ex>[ur] \ar@<-0.2ex>[ur] \ar@<-0.6ex>[ur]_{yz,yw}
}&\xymatrix{
 & & & \rotatebox{90}{$\ddots$} \\
 & \circ \ar@<0.25ex>[r]^x \ar@<-0.25ex>[r]_y & \circ \ar@<0.6ex>[ur]^{xz,xw} \ar@<0.2ex>[ur] \ar@<-0.2ex>[ur] \ar@<-0.6ex>[ur]_{yz,yw} \\
 & \circ \ar@<0.25ex>[r]^x \ar@<-0.25ex>[r]_y \ar@<0.25ex>[u]^z \ar@<-0.25ex>[u]_w & \circ \ar@<0.25ex>[u]^z \ar@<-0.25ex>[u]_w \\
\circ \ar@<0.6ex>[ur]^{xz,xw} \ar@<0.2ex>[ur] \ar@<-0.2ex>[ur] \ar@<-0.6ex>[ur]_{yz,yw} \\
\circ \ar@<0.25ex>[u]^z \ar@<-0.25ex>[u]_w
}\end{array}\]
Therefore there are the following four kinds of $2$-tilting bundles up to translations. Observe that by mutations of non-trivial upper sets in $q^{-1}(J)$, they are mutated to each other, which correspond to $2$-APR tilting mutations.
\[\begin{array}{c c c c}
\xymatrix{
\circ \ar@<0.25ex>[r]^x \ar@<-0.25ex>[r]_y & \circ \\
\circ \ar@<0.25ex>[r]^x \ar@<-0.25ex>[r]_y \ar@<0.25ex>[u]^z \ar@<-0.25ex>[u]_w & \circ \ar@<0.25ex>[u]^z \ar@<-0.25ex>[u]_w
}&\xymatrix{
 & & \circ \\
\circ \ar@<0.25ex>[r]^x \ar@<-0.25ex>[r]_y & \circ \ar@<0.6ex>[ur]^{xz,xw} \ar@<0.2ex>[ur] \ar@<-0.2ex>[ur] \ar@<-0.6ex>[ur]_{yz,yw} \\
 & \circ \ar@<0.25ex>[u]^z \ar@<-0.25ex>[u]_w
}&\xymatrix{
 & & \circ \ar@<0.25ex>[r]^x \ar@<-0.25ex>[r]_y & \circ \\
\circ \ar@<0.25ex>[r]^x \ar@<-0.25ex>[r]_y & \circ \ar@<0.6ex>[ur]^{xz,xw} \ar@<0.2ex>[ur] \ar@<-0.2ex>[ur] \ar@<-0.6ex>[ur]_{yz,yw}
}&\xymatrix{
 & \circ \\
 & \circ \ar@<0.25ex>[u]^z \ar@<-0.25ex>[u]_w \\
\circ \ar@<0.6ex>[ur]^{xz,xw} \ar@<0.2ex>[ur] \ar@<-0.2ex>[ur] \ar@<-0.6ex>[ur]_{yz,yw} \\
\circ \ar@<0.25ex>[u]^z \ar@<-0.25ex>[u]_w
}\end{array}\]
(2) Put $G:=\mathbb{Z}^2$ and $\vec{x}=\vec{y}=(1,0),\vec{z}=(1,1),\vec{w}=(0,1)\in G$. We view $S:=k[x,y,z,w]$ as a $G$-graded $k$-algebra. Then the resulting toric stack $\X$ is isomorphic to $\Sigma_1$. We have $\vec{p}=(3,2)$ and $H=G/\mathbb{Z}\vec{p}\cong\mathbb{Z}; (a,b)+\mathbb{Z}\vec{p}\mapsto 2a-3b$. If we equip $H$ with our partial order, then the quiver of $H$ becomes the following.
\[\xymatrix{
\cdots \ar[r]_{-z} \ar@2@/^18pt/[rr]^x_y \ar@/^-18pt/[rrr]_{-w} & \circ \ar[r]_{-z} \ar@2@/^18pt/[rr]^x_y \ar@/^-18pt/[rrr]_{-w} & \circ \ar[r]_{-z} \ar@2@/^18pt/[rr]^x_y \ar@/^-18pt/[rrr]_{-w} & \circ \ar[r]_{-z} \ar@2@/^18pt/[rr]^x_y \ar@/^-18pt/[rrr]_{-w} & \circ \ar[r]_{-z} \ar@2@/^18pt/[rr]^x_y \ar@/^-18pt/[rrr]_{-w} & \circ \ar[r]_{-z} \ar@2@/^18pt/[rr]^x_y & \circ \ar[r]_{-z} & \cdots
}\]
Then there is the following just one kind of non-trivial upper sets in $H$ up to translations.
\[\xymatrix{
\circ \ar[r]_{-z} \ar@2@/^18pt/[rr]^x_y \ar@/^-18pt/[rrr]_{-w} & \circ \ar[r]_{-z} \ar@2@/^18pt/[rr]^x_y \ar@/^-18pt/[rrr]_{-w} & \circ \ar[r]_{-z} \ar@2@/^18pt/[rr]^x_y \ar@/^-18pt/[rrr]_{-w} & \circ \ar[r]_{-z} \ar@2@/^18pt/[rr]^x_y & \circ \ar[r]_{-z} & \cdots
}\]
Remark that the isomorphism $H\cong\mathbb{Z}$ sends $s$ to $4$. Thus there are the following just one kind of sets in $\J_H$ up to translations. Here, we draw the quiver of the algebra $\End_S^H(\bigoplus_{h\in J}S(h))$ for $J\in\J_H$.
\[\xymatrix{
\circ \ar@2@/^18pt/[rr]^x_y & \circ \ar[l]_z \ar@2@/^18pt/[rr]^x_y & \circ \ar@3[l]^{xw,yw}_z & \circ \ar[l]_z \ar@/^18pt/[lll]_w
}\]
The quiver of $q^{-1}(J)$ becomes the following.
\[\xymatrix{
 & & & & & & & \rotatebox{90}{$\ddots$} \\
 & & & & & \circ \ar@2[r]^x_y & \circ \ar@3[ur]^z_{xw,yw} \\
 & & & & \circ \ar@2[r]^x_y \ar[ur]^z & \circ \ar[u]^w \ar[ur]_z \\
 & & \circ \ar@2[r]^x_y & \circ \ar@3[ur]^z_{xw,yw} \\
 & \circ \ar@2[r]^x_y \ar[ur]^z & \circ \ar[u]^w \ar[ur]_z \\
\rotatebox{90}{$\ddots$} \ar@3[ur]^z_{xw,yw}
}\]
There are the following four kinds of non-trivial upper sets in $q^{-1}(J)$ up to translations by $\vec{p}$.
\[\begin{array}{c c}
\xymatrix{
 & & & & & & \rotatebox{90}{$\ddots$} \\
 & & & & \circ \ar@2[r]^x_y & \circ \ar@3[ur]^z_{xw,yw} \\
 & & & \circ \ar@2[r]^x_y \ar[ur]^z & \circ \ar[u]^w \ar[ur]_z \\
 & \circ \ar@2[r]^x_y & \circ \ar@3[ur]^z_{xw,yw} \\
\circ \ar@2[r]^x_y \ar[ur]^z & \circ \ar[u]^w \ar[ur]_z
}&\xymatrix{
 & & & & & \rotatebox{90}{$\ddots$} \\
 & & & \circ \ar@2[r]^x_y & \circ \ar@3[ur]^z_{xw,yw} \\
 & & \circ \ar@2[r]^x_y \ar[ur]^z & \circ \ar[u]^w \ar[ur]_z \\
\circ \ar@2[r]^x_y & \circ \ar@3[ur]^z_{xw,yw} \\
\circ \ar[u]^w \ar[ur]_z
}\end{array}\]
\[\begin{array}{c c}
\xymatrix{
 & & & & & \rotatebox{90}{$\ddots$} \\
 & & & \circ \ar@2[r]^x_y & \circ \ar@3[ur]^z_{xw,yw} \\
 & & \circ \ar@2[r]^x_y \ar[ur]^z & \circ \ar[u]^w \ar[ur]_z \\
\circ \ar@2[r]^x_y & \circ \ar@3[ur]^z_{xw,yw}
}&\xymatrix{
 & & & & \rotatebox{90}{$\ddots$} \\
 & & \circ \ar@2[r]^x_y & \circ \ar@3[ur]^z_{xw,yw} \\
 & \circ \ar@2[r]^x_y \ar[ur]^z & \circ \ar[u]^w \ar[ur]_z \\
\circ \ar@3[ur]^z_{xw,yw}
}\end{array}\]
Therefore there are the following four kinds of $2$-tilting bundles up to translations. Observe that by mutations of non-trivial upper sets in $q^{-1}(J)$, they are mutated to each other, which correspond to $2$-APR tilting mutations.
\[\begin{array}{c c c c}
\xymatrix{
 & \circ \ar@2[r]^x_y & \circ \\
\circ \ar@2[r]^x_y \ar[ur]^z & \circ \ar[u]^w \ar[ur]_z
}&\xymatrix{
 & & \circ \\
\circ \ar@2[r]^x_y & \circ \ar@3[ur]^z_{xw,yw} \\
\circ \ar[u]^w \ar[ur]_z
}&\xymatrix{
 & & \circ \ar@2[r]^x_y & \circ \\
\circ \ar@2[r]^x_y & \circ \ar@3[ur]^z_{xw,yw}
}&\xymatrix{
 & & \circ \\
 & \circ \ar@2[r]^x_y \ar[ur]^z & \circ \ar[u]^w \\
\circ \ar@3[ur]^z_{xw,yw}
}\end{array}\]
\end{Ex}

Finally, we see a stacky example. 

\begin{Ex}($d=2$)
Put $G:=\mathbb{Z}^2$ and $\vec{x}=(1,-1),\vec{y}=(1,0),\vec{z}=(1,1),\vec{w}=(0,1)\in G$. We view $S:=k[x,y,z,w]$ as a $G$-graded $k$-algebra. We have $\vec{p}=(3,1)$ and $H=G/\mathbb{Z}\vec{p}\cong\mathbb{Z}; (a,b)+\mathbb{Z}\vec{p}\mapsto a-3b$. If we equip $H$ with our partial order, then the quiver of $H$ becomes the following.
\[\xymatrix{
\cdots \ar[r]_y \ar@/^-15pt/[rr]^{-z} \ar@/^-21pt/[rrr]_{-w} \ar@/^15pt/[rrrr]^x & \circ \ar[r]_y \ar@/^-15pt/[rr]^{-z} \ar@/^-21pt/[rrr]_{-w} \ar@/^15pt/[rrrr]^x & \circ \ar[r]_y \ar@/^-15pt/[rr]^{-z} \ar@/^-21pt/[rrr]_{-w}  \ar@/^15pt/[rrrr]^x & \circ \ar[r]_y \ar@/^-15pt/[rr]^{-z} \ar@/^-21pt/[rrr]_{-w} \ar@/^15pt/[rrrr]^x & \circ \ar[r]_y \ar@/^-15pt/[rr]^{-z} \ar@/^-21pt/[rrr]_{-w} \ar@/^15pt/[rrrr]^x & \circ \ar[r]_y \ar@/^-15pt/[rr]^{-z} \ar@/^-21pt/[rrr]_{-w} & \circ \ar[r]_y \ar@/^-15pt/[rr]^{-z} & \circ \ar[r]_y & \cdots
}\]
Then there is the following just one kind of non-trivial upper sets in $H$ up to translations.
\[\xymatrix{
\circ \ar[r]_y \ar@/^-15pt/[rr]^{-z} \ar@/^-21pt/[rrr]_{-w}  \ar@/^15pt/[rrrr]^x & \circ \ar[r]_y \ar@/^-15pt/[rr]^{-z} \ar@/^-21pt/[rrr]_{-w} \ar@/^15pt/[rrrr]^x & \circ \ar[r]_y \ar@/^-15pt/[rr]^{-z} \ar@/^-21pt/[rrr]_{-w} \ar@/^15pt/[rrrr]^x & \circ \ar[r]_y \ar@/^-15pt/[rr]^{-z} \ar@/^-21pt/[rrr]_{-w} & \circ \ar[r]_y \ar@/^-15pt/[rr]^{-z} & \circ \ar[r]_y & \cdots
}\]
Remark that the isomorphism $H\cong\mathbb{Z}$ sends $s$ to $5$. Thus there are the following just one kind of sets in $\J_H$ up to translations. Here, we draw the quiver of the algebra $\End_S^H(\bigoplus_{h\in J}S(h))$ for $J\in\J_H$.
\[\xymatrix{
\circ \ar[r]^y \ar@/^20pt/[rrrr]^x & \circ \ar@2[r]^y_{xw} \ar@/^12pt/[rr]^{xz} & \circ \ar@2[r]^y_{xw} \ar@/^15pt/[ll]_z & \circ \ar[r]^y \ar@/^15pt/[ll]_z \ar@/^21pt/[lll]^w & \circ \ar@/^15pt/[ll]_z \ar@/^21pt/[lll]^w
}\]
The quiver of $q^{-1}(J)$ becomes the following.
\[\xymatrix{
 & & & & & \circ \ar[r]^y \ar[dr]^x & \circ \ar@2[r]^y_{xw} \ar@/^18pt/[rr]^{xz} & \circ \ar@2[r]^y_{xw} & \cdots\\
 & & \circ \ar[r]^y \ar[dr]^x & \circ \ar@2[r]^y_{xw} \ar@/^18pt/[rr]^{xz} & \circ \ar@2[r]^y_{xw} \ar[ur]^z & \circ \ar[r]^y \ar[ur]^z \ar[u]^w & \circ \ar[ur]_z \ar[u]^w \\
\cdots \ar@2[r]^y_{xw} \ar@/^18pt/[rr]^{xz} & \circ \ar@2[r]^y_{xw} \ar[ur]^z & \circ \ar[r]^y \ar[ur]^z \ar[u]^w & \circ \ar[ur]_z \ar[u]^w
}\]
There are the following five kinds of non-trivial upper sets in $q^{-1}(J)$ up to translations by $\vec{p}$.
\[\xymatrix{
 & & & & \circ \ar[r]^y \ar[dr]^x & \circ \ar@2[r]^y_{xw} \ar@/^18pt/[rr]^{xz} & \circ \ar@2[r]^y_{xw} & \cdots\\
 & \circ \ar[r]^y \ar[dr]^x & \circ \ar@2[r]^y_{xw} \ar@/^18pt/[rr]^{xz} & \circ \ar@2[r]^y_{xw} \ar[ur]^z & \circ \ar[r]^y \ar[ur]^z \ar[u]^w & \circ \ar[ur]_z \ar[u]^w \\
\circ \ar@2[r]^y_{xw} \ar[ur]^z & \circ \ar[r]^y \ar[ur]^z \ar[u]^w & \circ \ar[ur]_z \ar[u]^w
}\]
\[\begin{array}{c c}
\xymatrix{
 & & & \circ \ar[r]^y \ar[dr]^x & \circ \ar@2[r]^y_{xw} \ar@/^18pt/[rr]^{xz} & \circ \ar@2[r]^y_{xw} & \cdots\\
\circ \ar[r]^y \ar[dr]^x & \circ \ar@2[r]^y_{xw} \ar@/^18pt/[rr]^{xz} & \circ \ar@2[r]^y_{xw} \ar[ur]^z & \circ \ar[r]^y \ar[ur]^z \ar[u]^w & \circ \ar[ur]_z \ar[u]^w \\
\circ \ar[r]^y \ar[ur]^z \ar[u]^w & \circ \ar[ur]_z \ar[u]^w
}&\xymatrix{
 & & & \circ \ar[r]^y \ar[dr]^x & \circ \ar@2[r]^y_{xw} \ar@/^18pt/[rr]^{xz} & \circ \ar@2[r]^y_{xw} & \cdots\\
\circ \ar[r]^y \ar[dr]^x & \circ \ar@2[r]^y_{xw} \ar@/^18pt/[rr]^{xz} & \circ \ar@2[r]^y_{xw} \ar[ur]^z & \circ \ar[r]^y \ar[ur]^z \ar[u]^w & \circ \ar[ur]_z \ar[u]^w \\
 & \circ \ar[ur]_z \ar[u]^w
}\end{array}\]
\[\begin{array}{c c}
\xymatrix{
 & & \circ \ar[r]^y \ar[dr]^x & \circ \ar@2[r]^y_{xw} \ar@/^18pt/[rr]^{xz} & \circ \ar@2[r]^y_{xw} & \cdots\\
\circ \ar@2[r]^y_{xw} \ar@/^18pt/[rr]^{xz} & \circ \ar@2[r]^y_{xw} \ar[ur]^z & \circ \ar[r]^y \ar[ur]^z \ar[u]^w & \circ \ar[ur]_z \ar[u]^w \\
\circ \ar[ur]_z \ar[u]^w
}&\xymatrix{
 & & \circ \ar[r]^y \ar[dr]^x & \circ \ar@2[r]^y_{xw} \ar@/^18pt/[rr]^{xz} & \circ \ar@2[r]^y_{xw} & \cdots\\
\circ \ar@2[r]^y_{xw} \ar@/^18pt/[rr]^{xz} & \circ \ar@2[r]^y_{xw} \ar[ur]^z & \circ \ar[r]^y \ar[ur]^z \ar[u]^w & \circ \ar[ur]_z \ar[u]^w
}\end{array}\]
Therefore there are the following five kinds of $2$-tilting bundles up to translations. Observe that by mutations of non-trivial upper sets in $q^{-1}(J)$, they are mutated to each other, which correspond to $2$-APR tilting mutations.
\[\begin{array}{c c c}
\xymatrix{
 & \circ \ar[r]^y \ar[dr]^x & \circ \\
\circ \ar@2[r]^y_{xw} \ar[ur]^z & \circ \ar[r]^y \ar[ur]^z \ar[u]^w & \circ \ar[u]^w
}&\xymatrix{
\circ \ar[r]^y \ar[dr]^x & \circ \ar@2[r]^y_{xw} & \circ \\
\circ \ar[r]^y \ar[ur]^z \ar[u]^w & \circ \ar[ur]_z \ar[u]^w
}&\xymatrix{
\circ \ar[r]^y \ar[dr]_x & \circ \ar@2[r]^y_{xw} \ar@/^18pt/[rr]^{xz} & \circ \ar@2[r]^y_{xw} & \circ \\
 & \circ \ar[ur]_z \ar[u]^w
}\end{array}\]
\[\begin{array}{c c}
\xymatrix{
 & & \circ \\
\circ \ar@2[r]^y_{xw} \ar@/^18pt/[rr]^{xz} & \circ \ar@2[r]^y_{xw} \ar[ur]^z & \circ \ar[u]^w \\
\circ \ar[ur]_z \ar[u]^w
}&\xymatrix{
 & & \circ \ar[dr]^x \\
\circ \ar@2[r]^y_{xw} \ar@/^-18pt/[rr]_{xz} & \circ \ar@2[r]^y_{xw} \ar[ur]^z & \circ \ar[r]^y \ar[u]^w & \circ
}\end{array}\]
\end{Ex}

%\end{comment}%%%%%%

\bibliographystyle{amsplain} 
\bibliography{reference}

\end{document}