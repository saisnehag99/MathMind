\documentclass[12pt, reqno]{amsart}

\usepackage{amsmath, amsthm, amscd, amsfonts, amssymb, graphicx, xcolor}

\textheight 22.5truecm \textwidth 14.5truecm
\setlength{\oddsidemargin}{0.35in}\setlength{\evensidemargin}{0.35in}

\setlength{\topmargin}{-.5cm}

\newtheorem{theorem}{\indent\sc Theorem}[section]
\newtheorem{lemma}[theorem]{\indent\sc Lemma}
\newtheorem{proposition}[theorem]{\indent\sc Proposition}
\newtheorem{corollary}[theorem]{\indent\sc Corollary}
\newtheorem{maintheorem}{\indent\sc Theorem}
\newtheorem{conjecture}[theorem]{\indent\sc Conjecture}
\newtheorem{objective}{\indent\sc Objective}
\theoremstyle{definition}
\newtheorem{definition}[theorem]{\indent\sc Definition}
\newtheorem{example}[theorem]{\indent\sc Example}
\newtheorem{exercise}[theorem]{\indent\sc Exercise}
\newtheorem{conclusion}[theorem]{\indent\sc Conclusion}
\newtheorem{criterion}[theorem]{\indent\sc Criterion}
\newtheorem{summary}[theorem]{\indent\sc Summary}
\newtheorem{axiom}[theorem]{\indent\sc Axiom}
\newtheorem{problem}[theorem]{\indent\sc Problem}
\theoremstyle{remark}
\newtheorem{remark}[theorem]{\indent\sc Remark}
\numberwithin{equation}{section}

\newcommand{\Broer}[1]{}

\usepackage{latexsym,amssymb}
\usepackage{graphicx}
\usepackage{tikz-cd}
\usepackage{amsfonts}
\usepackage[T1]{fontenc}
\usepackage[utf8]{inputenc}
\usepackage{amsmath}
\usepackage{IEEEtrantools}
\usepackage{amsthm}
\usepackage{euscript}
\usepackage{tikz}
\usepackage{bm}
\usepackage{mathtools}
\usepackage[citestyle=alphabetic,bibstyle=alphabetic]{biblatex}
\usepackage{indentfirst}
\usepackage[british]{babel}
\usepackage{csquotes}
\usepackage{dsfont}
\usepackage{enumitem}
\usepackage{soul}
\usepackage{setspace}
\usepackage{perpage}

\usetikzlibrary{arrows,chains,matrix,positioning,scopes,shapes}
\tikzset{join/.code=\tikzset{after node path={\ifx\tikzchainprevious\pgfutil@empty\else(\tikzchainprevious)edge[every join]#1(\tikzchaincurrent)\fi}}}
\tikzset{>=stealth',every on chain/.append style={join},
         every join/.style={->}}
\newcommand{\Thmstop}{\hglue-6pt.\kern6pt}
\graphicspath{ {./images/} }

\usepackage{caption}
\usepackage{subcaption}

\usepackage[bookmarksnumbered, colorlinks, plainpages=false]{hyperref}
\hypersetup{pdfborder=0 0 0}
\hypersetup{citecolor = blue}
\hypersetup{urlcolor = black}

\usepackage[toc]{appendix}

\usepackage{array,booktabs}
\usepackage{amsmath}
\usepackage{nccmath}
\setcounter{MaxMatrixCols}{20}
\usepackage{tabularray}
\UseTblrLibrary{amsmath}
\usepackage{nicematrix}

\usepackage{setspace}
\usepackage{comment}

\let\originalleft\left
\let\originalright\right
\renewcommand{\left}{\mathopen{}\mathclose\bgroup\originalleft}
\renewcommand{\right}{\aftergroup\egroup\originalright}

\usepackage[myheadings]{fullpage}

\newenvironment{nohyphens}{
  \par
  \hyphenpenalty=10000
  \exhyphenpenalty=10000
  \sloppy
}{\par}

\relpenalty=9999
\binoppenalty=9999

\flushbottom

\DeclareMathSymbol{\mhyphen}{\mathord}{AMSa}{"39}
\newcommand{\itbf}[1]{\textit{\textbf{#1}}}
\newcommand{\slant}[2]{{\raisebox{.08em}{$#1$}\big/\raisebox{-.08em}{$#2$}}}
\newcommand{\restr}{\mathord\downarrow}
\newcommand{\induce}{\mathord\uparrow}
\newcommand{\mapsfrom}{\mathrel{\reflectbox{\ensuremath{\mapsto}}}}

\makeatletter
\let\save@mathaccent\mathaccent
\newcommand*\if@single[3]{
  \setbox0\hbox{${\mathaccent"0362{#1}}^H$}
  \setbox2\hbox{${\mathaccent"0362{\kern0pt#1}}^H$}
  \ifdim\ht0=\ht2 #3\else #2\fi
  }
\newcommand*\rel@kern[1]{\kern#1\dimexpr\macc@kerna}
\newcommand*\widebar[1]{\@ifnextchar^{{\wide@bar{#1}{0}}}{\wide@bar{#1}{1}}}
\newcommand*\wide@bar[2]{\if@single{#1}{\wide@bar@{#1}{#2}{1}}{\wide@bar@{#1}{#2}{2}}}
\newcommand*\wide@bar@[3]{
  \begingroup
  \def\mathaccent##1##2{
    \let\mathaccent\save@mathaccent
    \if#32 \let\macc@nucleus\first@char \fi
    \setbox\z@\hbox{$\macc@style{\macc@nucleus}_{}$}
    \setbox\tw@\hbox{$\macc@style{\macc@nucleus}{}_{}$}
    \dimen@\wd\tw@
    \advance\dimen@-\wd\z@
    \divide\dimen@ 3
    \@tempdima\wd\tw@
    \advance\@tempdima-\scriptspace
    \divide\@tempdima 10
    \advance\dimen@-\@tempdima
    \ifdim\dimen@>\z@ \dimen@0pt\fi
    \rel@kern{0.6}\kern-\dimen@
    \if#31
      \overline{\rel@kern{-0.6}\kern\dimen@\macc@nucleus\rel@kern{0.4}\kern\dimen@}%
      \advance\dimen@0.4\dimexpr\macc@kerna
      \let\final@kern#2%
      \ifdim\dimen@<\z@ \let\final@kern1\fi
      \if\final@kern1 \kern-\dimen@\fi
    \else
      \overline{\rel@kern{-0.6}\kern\dimen@#1}
    \fi
  }
  \macc@depth\@ne
  \let\math@bgroup\@empty \let\math@egroup\macc@set@skewchar
  \mathsurround\z@ \frozen@everymath{\mathgroup\macc@group\relax}
  \macc@set@skewchar\relax
  \let\mathaccentV\macc@nested@a
  \if#31
    \macc@nested@a\relax111{#1}
  \else
    \def\gobble@till@marker##1\endmarker{}
    \futurelet\first@char\gobble@till@marker#1\endmarker
    \ifcat\noexpand\first@char A\else
      \def\first@char{}
    \fi
    \macc@nested@a\relax111{\first@char}
  \fi
  \endgroup
}
\makeatother

\newcommand{\s}[0]{_{\mathrm{s}}}
\newcommand{\n}[0]{_{\mathrm{n}}}
\newcommand{\g}[0]{\mathfrak{g}}
\newcommand{\fl}[0]{\mathfrak{l}}
\newcommand{\h}[0]{\mathfrak{h}}
\newcommand{\fu}[0]{\mathfrak{u}}
\newcommand{\p}[0]{\mathfrak{p}}
\newcommand{\Z}[0]{\mathcal{Z}}
\newcommand{\z}[0]{\mathfrak{z}}
\newcommand{\R}[0]{\mathrm{R}}
\newcommand{\Ru}[0]{\mathrm{R_u}}
\newcommand{\cN}[0]{\mathcal{N}}
\newcommand{\Ng}[0]{\mathcal{N}(\mathfrak{g})}
\newcommand{\N}[0]{\mathbb{N}}
\newcommand{\Np}[0]{\mathbb{N}^+}
\newcommand{\reg}[0]{\mathrm{reg}}
\newcommand{\Ad}[0]{\mathrm{Ad}}
\newcommand{\ad}[0]{\mathrm{ad}}
\newcommand{\md}[0]{\mathrm{d}}
\newcommand{\GL}[0]{\mathrm{GL}}
\newcommand{\gl}[0]{\mathfrak{gl}}
\newcommand{\End}[0]{\mathrm{End}}
\newcommand{\midd}[0]{~\middle|~}
\newcommand{\bbK}[0]{\mathbb{K}}
\newcommand{\greg}[0]{{\g\text{-}\reg}}
\newcommand{\U}[0]{\mathrm{U}}
\newcommand{\Gm}[0]{\mathbb{G}_{\mathrm{m}}}
\newcommand{\Ga}[0]{\mathbb{G}_{\mathrm{a}}}
\newcommand{\fD}[0]{\mathfrak{D}}
\newcommand{\bbKx}[0]{\mathbb{K}^{\times}}
\newcommand{\ft}[0]{\mathfrak{t}}
\newcommand{\Le}[0]{\left(L;e_0\right)}
\newcommand{\fJ}[0]{\mathfrak{J}}
\newcommand{\am}[0]{\langle m \rangle}
\newcommand{\ls}[0]{\g_{\am}}
\newcommand{\dvJ}[0]{\overline{\mathfrak{J}}}

\DeclareMathOperator{\fc}{\mathfrak{c}}
\DeclareMathOperator{\C}{\mathrm{C}}
\DeclareMathOperator{\Ind}{\mathrm{Ind}}
\newcommand{\Co}[1]{\operatorname{\mathrm{C}}^{\,\circ}_{#1}}
\DeclareMathOperator{\Lie}{\mathrm{Lie}}
\newcommand{\JE}[1]{\operatornamewithlimits{\sim}\limits^{#1}}
\newcommand{\J}[1]{\operatorname{\mathfrak{J}}_{#1}}
\DeclareMathOperator{\fd}{\mathfrak{d}}
\newcommand{\dreg}[1]{\operatorname{\mathfrak{d}}^{\mathrm{reg}}_{#1}}
\newcommand{\dv}[2]{\widebar{\J{#1} #2}}
\newcommand{\llangle}[0]{\left\langle}
\newcommand{\rrangle}[0]{\right\rangle}

\begingroup
\makeatletter
\@ifundefined{ver@biblatex.sty}
  {\@latex@error
     {Missing 'biblatex' package}
     {The bibliography requires the 'biblatex' package.}
      \aftergroup\endinput}
  {}
\endgroup

\addbibresource{PaperDecompositionClassesinArbitraryCharacteristic.bib}

\begin{document}

\setcounter{page}{1}

\centerline{}

\centerline{}

\title[Lie Algebra Decomposition Classes in Arbitrary Characteristic]{Lie Algebra Decomposition Classes for Reductive Algebraic Groups in Arbitrary Characteristic}

\author[Joel Summerfield]{Joel Summerfield}

\address{School of Mathematics, University of Birmingham, UK.}
\email{\textcolor[rgb]{0.00,0.00,0.84}{jns228@student.bham.ac.uk}}

\begin{abstract}
\begin{nohyphens}
Decomposition classes provide a way of partitioning the Lie algebras of an algebraic group into equivalence classes based on the Jordan decomposition.
In this paper, we investigate the decomposition classes of the Lie algebras of connected reductive algebraic groups, over algebraically closed fields of arbitrary characteristic.
We extend some results previously proved under restrictions on the characteristic, and introduce Levi-type decomposition classes to account for some of the difficulties encountered in bad characteristic.
We also establish properties of Lusztig-Spaltenstein induction of non-nilpotent orbits, extending the known results for nilpotent orbits.
\end{nohyphens}
\end{abstract} \maketitle

\section{Introduction}

In \cite[\S 5.2]{BK79}, Borho and Kraft introduced \emph{Zerlegungsklassen} (decomposition classes) as a tool for studying \emph{Schichten der Lie-Algebra} (sheets of a Lie algebra).
They considered a (connected) semisimple algebraic group $G$ of adjoint type, over an algebraically closed field of characteristic $0$, acting via the adjoint action on its Lie algebra $\g = \Lie G$.
For an arbitrary element $x \in \g$, with Jordan decomposition $x = x\s + x\n$, define $\C_Gx\s \coloneqq \left\{ g \in G \midd g \cdot x\s = x\s \right\}$.
Another element $y \in \g$ then has \emph{\"{a}hnliche Jordanzerlegung} (similar Jordan decomposition) if there exists $g \in G$ such that the Jordan decomposition of $g \cdot y = y'\s + y'\n$ satisfies $\C_G y'\s = \C_G x\s$ and $\left(\C_Gx\s\right) \cdot y'\n = \left(\C_Gx\s\right) \cdot x\n$.
This yields an equivalence relation on $\g$, whose corresponding equivalence classes are decomposition classes.
Useful properties of these decomposition classes were then established in later parts of \cite{BK79} and \cite{B81}.

In \cite[\S 1.2]{S82}, Spaltenstein extended the idea as follows.
They considered a connected reductive algebraic group $G$, over an algebraically closed field of arbitrary characteristic, again acting via the adjoint action on $\g = \Lie G$.
Elements $x = x\s + x\n$ and $y=y\s+\nobreak y\n \in \g$ were said to be equivalent in $\g$ if there exists $g \in G$ such that $\fc_{\g}\left(g \cdot x\s\right) = \fc_{\g}y\s \coloneqq \left\{ z \in \g \midd \left[y\s,z\right] = 0 \right\}$ and $g \cdot x\n = y\n$.
This again yields an equivalence relation on $\g$, whose corresponding equivalence classes were named \emph{packets} by Spaltenstein.

This definition generalises the concept of decomposition classes introduced in \cite{BK79}, and coincides with them in the characteristic $0$ case.
Spaltenstein then established properties of \emph{packets} in \cite{S82} and \cite{S84}.
As with Borho and Kraft, \emph{packets} were introduced by Spaltenstein to study the maximal irreducible subsets of $\g$ consisting of equal-dimension orbits, known as sheets.
Spaltenstein demonstrated in \cite{S82} that some of the properties from \cite{BK79} generalised immediately to good characteristic, using the fact that connected stabilisers of semisimple elements of $\g$ are Levi subgroups of $G$.
Moreover, certain properties related to nilpotent orbits were also shown in \cite{S82} to hold in the classical cases in bad characteristic (see \S\ref{Subsection Levi-Type Sheets} for details).

In \cite[\S 3]{B98}, Broer also considered decomposition classes, working under the assumption that $G$ is the adjoint group of a semisimple Lie algebra $\g$, over an algebraically closed field.
They primarily used the additional assumption that the characteristic is very good, and generalised further results from \cite{BK79} relating to the closures of decomposition classes, as well as establishing that decomposition classes are smooth.
Further results on decomposition classes can be found in \cite{B98dv} and \cite{TP05} (both assuming characteristic $0$), \cite{PS18} (assuming the Standard Hypotheses), and \cite{A25} (partly assuming good characteristic), amongst other places. \\

This paper first defines decomposition classes for an arbitrary algebraic group $G$, over an algebraically closed field $\bbK$ of arbitrary characteristic, acting on its Lie algebra $\g = \Lie G$ via the adjoint action.
For any $x \in \g$, we define its connected stabiliser $\Co{G}x \subseteq G$ to be the identity component of $\Co{G}x \coloneqq \left\{ g \in G \midd g \cdot x = x \right\}$, and let $x = x\s + x\n$ denote its Jordan decomposition.
\emph{Decomposition classes} are then the equivalence classes of $\g$ under the relation $x \sim y$, which holds if and only if there exists $g \in G$ such that $\Co{G}\left(g \cdot x\s\right) = \Co{G}y\s$ and $g \cdot x\n = y\n$.
The decomposition class containing $x \in \g$ is denoted $\J{G}x$, and we prove that each decomposition class has constant stabiliser dimension, and constant centraliser dimension.

From \S\ref{Subsection Connected Reductive Algebraic Groups} onwards, we assume that $G$ is a connected reductive algebraic group, and establish that our definition of decomposition classes coincides with the definition of packets used by Spaltenstein.
We prove some initial properties of decomposition classes in \mbox{\textsc{Theorem}~\ref{Theorem Initial Decomposition Class Properties}}, including that they are $G$-stable, $\bbKx$-stable, irreducible, and constructible sets which form a finite partition of $\g$.

Then we turn our attention to decomposition varieties, which are defined as the Zariski-closures of decomposition classes.
We equip the set of decomposition classes $\fD[G]$ with the closure order, where $\J{G}x \preceq \J{G}y$ if and only if $\J{G}x \subseteq \dv{G}{y}$.
In \S\ref{Section Preservation of Decomposition Classes}, we explore how the structure of decomposition classes are affected by central surjections, which leads us to the following preservation result.

\begin{maintheorem}\label{Theorem Decomposition Class Structure and Separable Central Surjections Main Result}
    Suppose $\varphi \colon G \rightarrow H$ is a separable central surjection of connected reductive algebraic groups.
    For any $x \in \g$, let $\check{x} \coloneqq \md\varphi(x) \in \h$.
    Then $\J{G}x \mapsto \J{H}\check{x}$ defines a bijection $\fD[G] \rightarrow \fD[H]$, with the following properties$:$
\begin{itemize}
    \item[$\mathrm{(i)}$] $\md\varphi \colon \g \rightarrow \h$ restricts to a surjection $\J{G}x \rightarrow \J{H}\check{x}$.
    \item[$\mathrm{(ii)}$] Preservation of closure$:$ $\md\varphi\left(\dv{G}{x}\right) = \dv{H}{\check{x}}$.
    \item[$\mathrm{(iii)}$] Preservation of the partial order$:$ $\J{G}x \preceq \J{G}y$ if and only if $\J{H}\check{x} \preceq \J{H}\check{y}$.
    \item[$\mathrm{(iv)}$] $\dim \J{G}x = \dim\ker \md\varphi + \dim \J{H}\check{x}$.
\end{itemize}
\end{maintheorem}

Note that (within the generality in which we are working) the stabiliser dimension $\dim \C_G x$ and the centraliser dimension $\dim \fc_{\g}x$ do not necessarily coincide for arbitrary $x \in \g$.
The fibres of the stabiliser dimension map $\dim \C_G \colon \g \rightarrow \N$ are referred to as \emph{stabiliser level sets}, and the irreducible components of non-empty stabiliser level sets are called \emph{stabiliser sheets}.
Analogously, the fibres of the centraliser dimension map $\dim \C_G \colon \g \rightarrow \N$ are referred to as \emph{centraliser level sets}, and the irreducible components of non-empty centraliser level sets are called \emph{centraliser sheets}.
We then use \emph{level set} to refer to a subset of $\g$ which is (at least one of) a stabiliser level set or a centraliser level set, and \emph{sheet} to refer to a subset of $\g$ which is (at least one of) a stabiliser sheet or a centraliser sheet.
This is an important departure from the literature (see \S\ref{Section Sheets} for more details) which allows us to uniformly prove results regarding both situations.

Since decomposition classes have constant stabiliser dimension, they are each contained in a unique stabiliser level set; analogously, each decomposition class is contained in a unique centraliser level set.
Given a level set $\ls$ of $\g$, we let $\fD_{\am}[G] \coloneqq \left\{ \fJ \in \fD[G] \midd \fJ \subseteq \ls \right\}$ denote the set of decomposition classes contained in $\ls$.

\begin{maintheorem}\label{Theorem Level Sheets and their Irreducible Components Main Result}
    Suppose $\ls$ is a level set of $\g$, and $\fJ \in \fD_{\am}[G]$.
\begin{itemize}
    \item[$\mathrm{(i)}$] $\fJ$ is a dense subset of an irreducible component of $\ls$ if and only if $\fJ$ is maximal in $\ls$ $($with respect to the closure order$)$.
    \item[$\mathrm{(ii)}$] The irreducible components of $\ls$ are in bijection with the decomposition classes which are maximal in $\ls$ $($with respect to the closure order$)$, via $\overline{\fJ} \cap \ls \mapsfrom \fJ$.
    \item[$\mathrm{(iii)}$] If $\fJ$ coincides with an irreducible component of $\ls$, then $\fJ$ is isolated in $\ls$ $($with respect to the closure order$)$.
\end{itemize}
\end{maintheorem}

We next look at generalising Lusztig-Spaltenstein induction to arbitrary orbits, building upon the work in \cite{S82}.
For any Levi subgroup $L \subseteq G$, we consider the set of $L$-orbits in $\fl$ under the adjoint action, denoted $\slant{\fl}{L}$.
We then use \cite[\S 2.2]{S82} to establish the existence of an induction map $\Ind_{\fl}^{\g} \colon \slant{\fl}{L} \rightarrow \slant{\g}{G}$, which generalises the Lusztig-Spaltenstein induction of nilpotent orbits.
After covering the properties of this induction established in \cite{S82}, we prove the following results, generalising the corresponding known results about the Lusztig-Spaltenstein induction of nilpotent orbits.

\begin{maintheorem}\label{Theorem LS Induction Main Result}
    Suppose $\mathcal{O} \in \slant{\fl}{L}$ is an arbitrary $L$-orbit.
\begin{itemize}
    \item[$\mathrm{(i)}$] The induced orbit $\Ind_{\fl}^{\g} \mathcal{O}$ is independent of the choice of parabolic used in its construction. 
    \item[$\mathrm{(ii)}$] Induction is transitive$:$ $\Ind_{\fl}^{\g} \mathcal{O} = \Ind_{\mathfrak{m}}^{\g} \Ind_{\fl}^{\mathfrak{m}} \mathcal{O}$, for nested Levi subgroups $L \subseteq\nobreak M \subseteq\nobreak G$.
    \item[$\mathrm{(iii)}$] $\dim \Ind_{\fl}^{\g} \mathcal{O} = \dim \mathcal{O} + (\dim G - \dim L)$.
    \item[$\mathrm{(iv)}$] $\left(\Ind_{\fl}^{\g} \mathcal{O}\right) \cap \left(\mathcal{O} + \fu_{\p}\right)$ is a single $P$-orbit, where $P \subseteq G$ is any parabolic subgroup for which $L$ is a Levi factor, and $\fu_{\p} = \Lie \left(\Ru(P)\right)$.
\end{itemize}
\end{maintheorem}

Having worked in full generality up to this point, we narrow our scope in \S\ref{Section Levi-Type Decomposition Classes} to \mbox{Levi-type} decomposition classes; these are defined to be the decomposition classes of elements $x \in \g$ such that the connected stabiliser of their semisimple part $\Co{G}x\s \subseteq G$ is a Levi subgroup.
These are introduced as a tool to avoid some of the complications that arise in bad characteristic, and will allow us to prove the main result of this paper, which extends prior results of \cite{B81}, \cite{B98}, and \cite{A25}.

\begin{maintheorem}\label{Theorem Levi-Type Decomposition Varieties Main Result}
    Suppose $\J{G}\Le$ is a Levi-type decomposition class.
    Let $P \subseteq G$ be a parabolic with Levi factor $L$, and unipotent radical $U_P = \mathrm{R_u}(P)$.
\begin{itemize}
    \item[$\mathrm{(i)}$] $\dv{G}{\Le} = G \cdot \left( \mathfrak{z}(\mathfrak{l}) + \overline{L \cdot e_0} + \mathfrak{u}_{\p}\right)$.
    \item[$\mathrm{(ii)}$] $\dv{G}{\Le}$ is a union of decomposition classes.
    \item[$\mathrm{(iii)}$] $\dv{G}{\Le} = \bigcup\limits_{z \in \mathfrak{z}(\mathfrak{l})} \overline{\mathrm{Ind}_{\mathfrak{l}}^{\g}L \cdot \left(z + e_0\right)}$.
    \item[$\mathrm{(iv)}$] $\dv{G}{\Le}^\reg = \bigcup\limits_{z \in \mathfrak{z}(\mathfrak{l})} \mathrm{Ind}_{\mathfrak{l}}^{\g}L \cdot \left(z + e_0\right)$.
\end{itemize}
\end{maintheorem}

This paper concludes by considering a conjecture of Spaltenstein (see \textsc{Conjecture}~\ref{Conjecture Spaltenstein Sheets Contain Unique Nilpotent}) regarding stabiliser sheets and nilpotent orbits.
In particular, we show that (regardless of characteristic) every Levi-type stabiliser sheet contains a unique nilpotent orbit.

\subsection*{Notation}

Let $\bbK$ be an algebraically closed field of characteristic $p \geq 0$, with non-zero elements $\bbK^{\times}$.
All varieties and vector spaces will be over $\bbK$, and all spaces are equipped with the Zariski topology.
All algebraic groups are assumed to be affine and linear, and the Lie algebras of algebraic groups will be denoted by the corresponding lowercase fraktur letter (for example, $\g = \Lie G$).
If $X \subseteq \g$, then $\widebar{X}$ will always denote the closure of $X$ in $\g$ (with respect to the Zariski topology).

For any homomorphism of algebraic groups $\varphi \colon G \rightarrow H$, we denote its differential by $\md \varphi \colon \g \rightarrow \h$.
Let $[-,-] \colon \g \times \g \rightarrow \g$ denote the Lie bracket on $\g$.

We denote the set of all non-negative integers by $\N$, and the set of strictly positive integers by $\N^+$.
For each $n \in \N^+$, we let $\GL_n$ denote the group of $n \times n$ invertible matrices, and $\gl_n$ the Lie algebra of all $n \times n$ matrices. \\

Suppose $H \subseteq G$ is a closed subgroup of an algebraic group, and $X \subseteq \g$ is an arbitrary subset.
The connected component of $H$ containing the identity element is denoted $H^{\circ}$, and referred to as its \emph{identity component}.
The centre of $H$ is denoted $\mathcal{Z}_H$, and its identity component is also denoted $\mathcal{Z}^{\circ}_H = \left(\mathcal{Z}_H\right)^{\circ}$.

The adjoint action of $H$ on $\g$ is denoted $h \cdot x \coloneqq \Ad(h)(x)$, for any $h \in H$ and $x \in \g$, and the corresponding $H$-\emph{orbit} is $H \cdot x \coloneqq \left\{ h \cdot x \midd h \in H \right\}$.
The set of all adjoint $H$-orbits in $\g$ is denoted $\slant{\g}{H}$.
More generally, $H \cdot X \coloneqq \bigcup_{x \in X} H \cdot x$ denotes the $H$-\emph{saturation} of $X$, and we say that $X$ is $H$-\emph{stable} if $H \cdot X \subseteq X$ (equivalently, $H \cdot X = X$).

We define the $H$-\emph{stabiliser} of $x \in \g$ as $\C_H x \coloneqq \left\{ g \in H \midd g \cdot x = x \right\} = H \cap \C_G x$, and its $\h$-\emph{centraliser} as $\fc_{\h} x \coloneqq \left\{ y \in \h \midd [x,y] = 0 \right\} = \h \cap \fc_{\g} x$.
More generally, $\C_H X \coloneqq \bigcap_{x \in X} \C_H x$ and $\fc_{\h} X \coloneqq \bigcap_{x \in X} \fc_{\h} x$.
We also let $\z(\h) \coloneqq \fc_{\h} \h = \left\{ x \in \h \midd [x,y] = 0, \text{for~all~} y \in \h \right\}$ denote the \emph{centre} of $\h$.
When there is no ambiguity, we refer to the $G$-stabiliser and $\g$-centraliser as simply the \emph{stabiliser} and \emph{centraliser}, respectively.
Let $\Co{G} x \coloneqq \left(\C_G x\right)^{\circ}$ denote the \emph{connected stabiliser} of $x$.

The \emph{double centraliser} of $x \in \g$ is defined to be $\fd_{\g} x \coloneqq \fc_{\g}\left(\fc_{\g} x\right)$, the centraliser of its centraliser.
It follows readily from the definition that $\fd_{\g} x = \left\{ y \in \g \midd \fc_{\g} x \subseteq \fc_{\g} y \right\} = \z\left(\fc_{\g} x\right)$; that is, the double centraliser coincides with the centre of the centraliser.
We observe that $g \cdot \fc_{\g} x = \fc_{\g}\left(g \cdot x\right)$, for any $g \in G$.

\subsection*{Acknowledgements}

The author thanks their PhD supervisor, Simon Goodwin, for their continued guidance and support, as well as Matthew Westaway for originally introducing them to this topic.
The author also thanks Alexander Fr\"{u}h and Lauren Keane for assisting in the translation of \cite{BK79} and \cite{B81}.
The author's research is supported by EPSRC grant EP/V520275/1.

\section{Decomposition Classes}

For any $x \in \g$, let $x = x\s +x\n$ be the (\emph{additive}) \emph{Jordan decomposition}, as explained in \cite[\S 4.4.19]{S98}.
Then $x \in \g$ is \emph{semisimple} if and only if $x = x\s$, and $y \in \g$ is \emph{nilpotent} if and only if $y = y\n$.
This version of the Jordan decomposition is constructed by considering $x$ as a locally finite linear endomorphism of the coordinate algebra $\bbK[G]$.
However, we shall now demonstrate an alternative (but equivalent) way of defining semisimple and nilpotent elements of $\g$.

Following \cite[Example 1.6(8)]{B91}, an \emph{immersive representation} of $G$ is any injective homomorphism of algebraic groups $\rho \colon G \rightarrow \GL_n$ (for some $n \in \N^+$) such that $\rho$ induces an isomorphism of algebraic groups $G \cong \rho(G)$.
Using \cite[Theorem 2.3.7(i)]{S98}, every algebraic group has at least one immersive representation, so fix such a $\rho \colon G \rightarrow \GL_n$.
We then say that $x \in \g$ is \emph{semisimple} if there exists a basis of $\bbK^n$ consisting of eigenvectors of $\md \rho(x) \in \gl_n$, and $y \in \g$ is \emph{nilpotent} if $\left(\md \rho(y)\right)^n = 0$ is the zero-matrix.
It follows from \cite[Theorem 4.4.20]{S98} that these definitions are independent of the chosen immersive representation $\rho \colon G \rightarrow \GL_n$, and coincide with the definitions used throughout \cite{S98}.

We will make use of these alternative definitions in the proofs of \textsc{Lemma}~\ref{Lemma Stabiliser/Centraliser of Jordan Decomposition}(iii) and \textsc{\mbox{Proposition}}~\ref{Proposition Bijective morphism of UP varieties}(ii), where we can use an immersive representation to assume (without loss of generality) that $G \subseteq \GL_n$ is a closed subgroup for some $n \in \N^+$. \\

The set of all nilpotent elements of $\g$ (the \emph{nilpotent cone}) is denoted $\Ng$, and is a closed subset of $\g$.
The set of nilpotent $G$-orbits is then denoted $\slant{\Ng}{G}$.

Two elements $x, y \in \g$ are \emph{Jordan equivalent}, written $x \sim y$, if there exists $g \in G$ such that $\Co{G}\left(g \cdot x\s\right) = \Co{G}y\s$ and $g \cdot x\n = y\n$.
Then $\sim$ is an equivalence relation on $\g$, and thus we may consider its equivalence classes.

\begin{definition}\label{Definition Decomposition Class}
    The $G$\itbf{-decomposition class} of $x \in \g$ is defined as its equivalence class with respect to $\sim$, and is denoted $\J{G} x = \left\{ y \in \g \midd x \sim y \right\}$.
\end{definition}

We let $\fD[G]$ denote the set of $G$-decomposition classes, and note that this definition is different from the definition of packets found in \cite[\S 1.2]{S82}.
We shall prove in \textsc{Corollary}~\ref{Corollary Equivalent Jordan Equivalence Definition} that (assuming $G$ is connected reductive) the two definitions coincide.

\subsection{Stabilisers and Centralisers}\label{Subsection Stabilisers and Centralisers}

Observe that, if $\rho \colon G \rightarrow \GL_n$ is an immersive representation with $H = \rho(G) \subseteq \GL_n$, then $\md\rho \colon \g \rightarrow \h$ preserves the Jordan decomposition and restricts to a bijection $\fc_{\g} y \rightarrow \fc_{\h} \md \rho(y)$, for any $y \in \g$.

\begin{lemma}\label{Lemma Stabiliser/Centraliser of Jordan Decomposition}
    Suppose $x \in \g$.
\begin{itemize}
    \item[$\mathrm{(i)}$] $\C_G x = \C_G x\s \cap \C_G x\n = \C_{\C_G x\s} x\n$.
    \item[$\mathrm{(ii)}$] $\Co{G} x = \left(\Co{G} x\s \cap \C_G x\n\right)^{\circ} = \Co{\Co{G} x\s} x\n$.
    \item[$\mathrm{(ii)}$] $\fc_{\g} x = \fc_{\g} x\s \cap \fc_{\g} x\n = \fc_{\fc_{\g} x\s} x\n$.
\end{itemize}
\begin{proof}
    Since the adjoint action preserves the Jordan decomposition, we have that $g \cdot x\s = x\s$ and $g \cdot x\n = x\n$ (for any $g \in \C_Gx$), from which (i) follows.
    
    For any closed subgroups $H,K \subseteq G$, we observe that $H^{\circ} \cap K$ is a finite index closed subgroup of $H \cap K$, and thus $(H \cap K)^{\circ} = \left(H^{\circ} \cap K\right)^{\circ}$.
    Applying this to $H = \C_Gx\s$ and $K = \C_Gx\n$ shows that the first equality in (ii) follows from the first equality in (i), whereas the second equality is just notation.

    For (iii), we can use a suitable immersive representation to assume (without loss of generality) that $G \subseteq \GL_n$ is a closed subgroup, and consequently regard $\g$ as a Lie subalgebra of $\End(V)$, where $V = \bbK^n$.
    Following \cite[Proposition 4.2(2)]{B91}, there exists a univariate polynomial $q(z) \in z\bbK[z]$ (with no constant term) such that $x\s = q\left(x\right)$.
    If $y \in \fc_{\g} x$, then $y \circ x = x \circ y$ as maps $V \rightarrow V$, from which it follows that $y \circ q(x) = q(x) \circ y$, hence $y \in \fc_{\g} x_s$.
    Then $\left[x\n,y\right] = [x,y] - \left[x\s,y\right]$ implies that $y \in \fc_{\g}x\n$.
    Therefore, $\fc_{\g} x \subseteq \fc_{\g} x\s \cap \fc_{\g} x\n$ which (since the converse is immediate) proves that $\fc_{\g} x = \fc_{\g} x\s \cap \fc_{\g} x\n$.
\end{proof}
\end{lemma}

We define the \emph{stabiliser dimension map} $\dim \C_G \colon \g \rightarrow \N$ via $x \mapsto \dim\left(\C_G x \right) \in \N$; likewise for the \emph{centraliser dimension map} $\dim \fc_{\g} \colon \g \rightarrow \N$.
We immediately observe that both of these maps are constant on each $G$-orbit.
Using the version of Chevalley's Semi-Continuity Theorem from \cite[Corollary AG10.3]{B91}, we can establish the following lemma, in which a map $f \colon X \rightarrow \N$ (from an arbitrary topological space $X$) is \emph{upper semi-continuous} if $\left\{ x \in X \midd f(x) \geq n \right\}$ is closed for all $n \in \N$.

\begin{lemma}\label{Lemma Stabiliser/Centraliser Dimension Maps are USC}
    Both the stabiliser and centraliser dimension maps $\dim \C_G \colon \g \rightarrow \N$ and $\dim \fc_{\g} \colon \g \rightarrow \N$ are upper semi-continuous.
\end{lemma}

We then define the \emph{stabiliser level sets} of $\g$ as the fibres of the stabiliser dimension map, and denote them $\g_{(m)} \coloneqq \left\{ x \in \g \midd \dim \C_G x = m \right\}$, for each $m \in \N$.
Analogously, we define the \emph{centraliser level sets} of $\g$ as the fibres of the centraliser dimension map, and denote them $\g_{[m]} \coloneqq \left\{ x \in \g \midd \dim \fc_{\g} x = m \right\}$, for each $m \in \N$.
This coincides with the notation introduced in \cite[Remark 2.1]{PS18}.

We shall use the term \emph{level set} of $\g$ to refer collectively to any subset of $\g$ which is (at least one of) a stabiliser level set or a centraliser level set, and denote a generic level set by $\ls$.
Since $\C_G \lambda x = \C_G x$ and $\fc_{\g} \lambda x = \fc_{\g} x$ (for all $x \in \g$ and $\lambda \in \bbKx$), level sets are $\bbKx$-stable.
It follows from \textsc{Lemma}~\ref{Lemma Stabiliser/Centraliser Dimension Maps are USC} that each level set is locally closed in $\g$.
More generally, for any subspace $V \subseteq \g$, we define $V_{(m)} \coloneqq V \cap \g_{(m)}$ and $V_{[m]} \coloneqq V \cap \g_{[m]}$, and observe that $V_{(m)}$ and $V_{[m]}$ are also locally closed in $\g$. \\

Let $X \subseteq \g$ be an arbitrary subset.
We define the set of $G$-\emph{regular elements} of $X$ to be $X^{G\mhyphen\reg} \coloneqq \left\{ x \in X \midd \dim \C_G x \leq \dim \C_G y, \text{for~all~} y \in X \right\}$, the set of elements of $X$ with minimal stabiliser dimension.
Whenever the underlying group is unambiguous, we shall denote this set $X^\reg$ instead.
Analogously, we define the set of $\g$-\emph{regular elements} of $X$ to be $X^\greg \coloneqq \left\{ x \in X \midd \dim \fc_{\g} x \leq \dim \fc_{\g} y, \text{for~all~} y \in X \right\}$, the set of elements of $X$ with minimal centraliser dimension.
Since $X^\reg = X \cap \g_{(m)}$, where $m \in \N$ is minimal such that this intersection is non-empty, it follows that $X^\reg$ is open in $X$; a similar argument holds for $X^\greg$.

If $V \subseteq \g$ is a subsapce, then it follows that both $V^\reg$ and $V^\greg$ are open dense irreducible subsets of $V$, and are thus both irreducible and locally closed in $\g$.
In particular, since $\fd_{\g} x \subseteq \g$ is a subspace (for any $x \in \g$), we know that $\left(\fd_{\g} x\right)^\greg = \left\{ y \in \g \midd \fc_{\g} y = \fc_{\g} x \right\}$ is irreducible and locally closed in $\g$.

\begin{lemma}\label{Lemma Closure inside a Level Set}
    Suppose $\ls$ is a level set and $Y \subseteq \ls$.
\begin{itemize}
    \item[$\mathrm{(a)}$] If $\ls = \g_{(m)}$ is a stabiliser level set, then $Y \subseteq \widebar{Y}^\reg = \widebar{Y} \cap \ls$.
    \item[$\mathrm{(b)}$] If $\ls = \g_{[m]}$ is a centraliser level set, then $Y \subseteq \widebar{Y}^\greg = \widebar{Y} \cap \ls$.
\end{itemize}
\begin{proof}
    We shall prove (a), observing that an almost-identical proof works for (b), so suppose that $\ls = \g_{(m)}$.
    Recall that $\widebar{Y}$ denotes the closure of $Y$ in $\g$.
    Using \textsc{Lemma}~\ref{Lemma Stabiliser/Centraliser Dimension Maps are USC}, we know that $\widebar{Y} \subseteq \overline{\ls\vphantom{Y}} \subseteq \g_{(\geq m)} \coloneqq \bigsqcup_{n \geq m} \g_{(n)}$.
    Therefore, the minimal $k \in \N$ such that $\widebar{Y} \cap \g_{(k)} \neq \emptyset$ must be $m$, and hence $\widebar{Y}^\reg = \widebar{Y} \cap \ls$.
    The inclusion $Y \subseteq \widebar{Y}^\reg$ is then immediate since $Y \subseteq \widebar{Y} \cap \ls$.
\end{proof}
\end{lemma}

For some fixed $x \in \g$, let $\sigma_x \colon G \rightarrow \g$ denote the orbit map $g \mapsto g \cdot x$.
By considering its differential, we have the inclusion $\Lie\left(\C_G x\right) \subseteq \fc_{\g} x$; however, we do not have equality in general.
Using \cite[\S 9.1]{B91}, $\Lie\left(\C_G x\right) = \fc_{\g} x$ if and only if $\sigma_x \colon G \rightarrow \g$ is a separable morphism of affine varieties, if and only if $\dim \C_G x = \dim \fc_{\g} x$.

There are many extra conditions we could impose on $x$ and $G$ to force equality here, such as the Standard Hypotheses (see \cite[\S 2.9]{J04} for an explanation of these).
Other suitable conditions can be found throughout the literature, including (for example) \cite[\S 2]{J04}, \cite[Lemma 2.6.2]{L05}, and \cite[Proposition 3.10]{T16}, the latter of which uses results from \cite{H10}.

However, the most important condition for us follows from \cite[Proposition 9.1(2)]{B91}: if $x \in \g$ is semisimple, then $\Lie\left(\C_G x\right) = \fc_{\g} x$.
Consequently, if $X \subseteq \g$ only consists of semisimple elements, then $X^\reg = X^\greg$.

\begin{proposition}\label{Proposition Constant Dimension}
    Suppose that $x,y \in \g$ satisfy $x \sim y$.
\begin{itemize}
    \item[$\mathrm{(a)}$] $\dim \C_G x = \dim\C_G y$.
    \item[$\mathrm{(b)}$] $\dim \fc_{\g}x = \dim\fc_{\g}y$.
\end{itemize}
\begin{proof}
    Let $g \in G$ be such that $\Co{G} \left(g \cdot x\s\right) = \Co{G}y\s$ and $g \cdot x\n = y\n$.
    It then follows from \textsc{Lemma}~\ref{Lemma Stabiliser/Centraliser of Jordan Decomposition}(ii) that $g \cdot \Co{G}x = \left( \Co{G}\left(g \cdot x\s\right) \cap \C_G \left(g \cdot x\n\right)\right)^\circ = \left( \Co{G}y\s \cap \C_G y\n\right)^\circ = \Co{G}y$.
    Therefore, $\dim \C_G x = \dim \Co{G}x = \dim \Co{G}y = \dim\C_G y$, which proves (a).

    Since $\fc_{\g} \left(g \cdot x\s\right) = \Lie \C_G\left(g \cdot x\s\right) = \Lie \Co{G}\left(g \cdot x\s\right) = \Lie \Co{G}y\s = \Lie \C_G y\s = \fc_{\g} y\s$, it follows from \textsc{Lemma}~\ref{Lemma Stabiliser/Centraliser of Jordan Decomposition}(iii) that $g \cdot \fc_{\g} x = \fc_{\g}\left(g \cdot x\s\right) \cap \fc_{\g}\left(g \cdot x\n\right) = \fc_{\g} y\s \cap \fc_{\g} y\n = \fc_{\g} y$.
    Therefore, $\dim \fc_{\g}x = \dim\fc_{\g}y$, which proves (b).
\end{proof}
\end{proposition}

It follows from \textsc{Proposition}~\ref{Proposition Constant Dimension}(a) that each decomposition class lies in a unique stabiliser level set, and therefore each stabiliser level set is the finite disjoint union of the decomposition classes it contains.
Similarly, using \textsc{Proposition}~\ref{Proposition Constant Dimension}(b), each decomposition class lies in a unique centraliser level set, and so each centraliser level set is the finite disjoint union of the decomposition classes it contains.

Given a level set $\ls$ of $\g$, let $\fD_{\am}[G] \coloneqq \left\{ \fJ \in \fD[G] \midd \fJ \subseteq \ls \right\}$ denote the set of decomposition classes contained in $\ls$.
It follows from \textsc{Proposition}~\ref{Proposition Constant Dimension} that $\ls = \bigsqcup_{\fJ \in \fD_{\am}[G]} \fJ$.
If $\ls = \g_{(m)}$ is a stabiliser level set, then we shall also use $\fD_{(m)}[G]$ to denote $\fD_{\am}[G]$.
Similarly, if $\ls = \g_{[m]}$ is a centraliser level set, then we shall also use $\fD_{[m]}[G]$ to denote $\fD_{\am}[G]$.

\subsection{Connected Reductive Algebraic Groups}\label{Subsection Connected Reductive Algebraic Groups}

For the remainder of the paper, we always assume that $G$ is a connected reductive algebraic group.
Following \cite[\S 2]{S75}, we say that a connected reductive subgroup $H \subseteq G$ is \emph{regular} if it contains a maximal torus of $G$.

By a \emph{Levi subgroup} we mean a Levi factor of a parabolic subgroup, and observe that all Levi subgroups are regular connected reductive subgroups.
If $L \subseteq G$ is a Levi subgroup, then we let $\mathfrak{P}(G,L)$ denote the (finite) set of all parabolic subgroups of $G$ for which $L$ is a Levi factor.
Given any parabolic subgroup $P \subseteq G$, we let $U_P \coloneqq \Ru(P)$ denote its unipotent radical, with corresponding Lie algebra $\fu_{\p} \subseteq \p$. \\

For the remainder of the section, fix a choice of maximal torus $T \subseteq G$, and let $\Phi =\nobreak \Phi(G,T)$ denote the corresponding \emph{root system}.
For each $\alpha \in \Phi$, we let $\g_{\alpha}$ and $\U_{\alpha}$ denote the corresponding \emph{root subspace} and \emph{root subgroup}, respectively.
For any subset of roots $\Psi \subseteq \Phi$, let $G(\Psi) \coloneqq \llangle \U_{\alpha} \midd \alpha \in \Psi \rrangle$ denote the subgroup of $G$ generated by the corresponding root subgroups, and let $G_T(\Psi) \coloneqq \llangle T, \U_{\alpha} \midd \alpha \in \Psi \rrangle = \llangle T, G(\Psi) \rrangle$ denote the subgroup additionally generated by $T$.
We note that, without assumptions on the subset $\Psi \subseteq \Phi$, it may well be the case that $\U_{\beta} \subseteq G(\Psi)$ for some $\beta \notin \Psi$.
Moreover, we let $\g_{\Psi}$ be shorthand for $\bigoplus_{\alpha \in \Psi} \g_{\alpha}$.

Observe that $\mathfrak{P}(G,T)$ is precisely the set of Borel subgroups of $G$ which contain $T$.
For any $B \in \mathfrak{P}(G,T)$, let $\Phi^+_B \subseteq \Phi$ denote the corresponding set of positive roots.
It follows that $U_B = G\left(\Phi^+_B\right)$ and $B = G_T\left(\Phi^+_B\right)$, with corresponding Lie algebras $\fu_{\mathfrak{b}} = \g_{\Phi^+_B}$ and $\mathfrak{b} = \ft \oplus \fu_{\mathfrak{b}}$.
As proved in \cite[\S 2.7]{J04}, we have $\Ng = G \cdot \fu_{\mathfrak{b}}$.
Therefore, for any system of positive roots $\Phi^+ \subseteq \Phi$, we have $\Ng = G \cdot \g_{\Phi^+}$.

\begin{lemma}\label{Lemma Nilpotents of Regular Connected Reductive Subgroup}
    Suppose $H \subseteq G$ is a regular connected reductive subgroup, with $T \subseteq H$.
\begin{itemize}
    \item[$\mathrm{(i)}$] $\Phi(H,T) = \left\{\alpha \in \Phi \midd \U_{\alpha} \subseteq H \right\} = \left\{\alpha \in \Phi \midd \g_{\alpha} \subseteq \h \right\}$.
    \item[$\mathrm{(ii)}$] If $\Phi^+$ is a system of positive roots in $\Phi$, then $\Phi^+ \cap \Phi(H,T)$ is a system of positive roots in $\Phi(H,T)$.
    \item[$\mathrm{(iii)}$] If $X \subseteq \g$ is such that $H \cdot X \subseteq \g_{\Phi^+}$, then $\cN(\h) + X \subseteq \Ng$.
    \item[$\mathrm{(iv)}$] If $P \subseteq G$ is a parabolic subgroup with Levi factor $L \subseteq P$ and unipotent radical $U = \Ru(P)$, then $\cN(\fl) + \fu \subseteq \Ng$.
\end{itemize}
\begin{proof}
    (i) is a consequence of the proof of \cite[Proposition 13.20]{B91}, and (ii) is evident from \cite[\S 7.4.5]{S98}.
    If $\Phi^+_H = \Phi^+ \cap \Phi(H,T)$, then $\cN(\h) = H \cdot \g_{\Phi^+_H}$.
    For any $y \in \cN(\h)$, there exists $h \in H$ such that $h \cdot y \in \g_{\Phi^+_H}$.
    Suppose that $X \subseteq \g$ is such that $H \cdot X \subseteq \g_{\Phi^+}$.
    Then, for any $x \in X$, we have that $h \cdot (y + x) \in \g_{\Phi^+} \subseteq \Ng$.
    Therefore, $y + x \in \Ng$, as required for (iii).

    Since all Levi subgroups of $G$ are regular connected reductive subgroups, and $\fu$ is $L$-stable, (iv) follows from (iii) and the fact that $\fu \subseteq \bigoplus_{\alpha \in \Phi^+} \g_{\alpha}$ for some suitable system of positive roots $\Phi^+ \subseteq \Phi$.
\end{proof}
\end{lemma}

For any $y \in \ft$, let $\Phi_y \coloneqq \left\{ \alpha \in \Phi \midd \md \alpha (y) = 0 \right\}$ denote the set of roots $\alpha \colon T \rightarrow \Gm$ whose differential $\md \alpha \colon \ft \rightarrow \bbK$ has kernel containing $y$.
Since each semisimple element of $\g$ lies in $G \cdot \ft$, and is contained in the Lie algebra of some maximal torus of $G$, the following lemma describes the connected stabiliser and centraliser of any semisimple element of $\g$.

\begin{lemma}\label{Lemma Semisimple Centraliser Description}
    Suppose $y \in \ft$.
\begin{itemize}
    \item[$\mathrm{(i)}$] $\Co{G} y = G_T\left(\Phi_y\right) = \llangle T, \U_{\alpha} \midd \alpha \in \Phi \colon \md \alpha(y) = 0 \rrangle$ is a regular connected reductive algebraic group with root system $\Phi\left(\Co{G} y,T\right) = \Phi_y$.
    \item[$\mathrm{(ii)}$] $\Lie \left(\Co{G} y\right) = \fc_{\g} y = \ft \oplus \g_{\Phi_y} = \ft \oplus \bigoplus_{\alpha \in \Phi_y} \g_{\alpha}$.
    \item[$\mathrm{(iii)}$] There are only finitely many nilpotent $\Co{G} y$-orbits in $\fc_{\g} y$.
    \item[$\mathrm{(iv)}$] If $\Phi^+ \subseteq \Phi$ is any system of positive roots, then $\cN\left(\fc_{\g}y\right) = \Co{G}y \cdot \g_{\left( \Phi^+ \cap \Phi_y\right)}$.
\end{itemize}
\begin{proof}
    See \cite[Lemma 3.7]{S75} for (i).
    Since $H = \Co{G} y \subseteq G$ is a closed subgroup containing $T$, (ii) follows from \cite[Proposition 13.20]{B91}.
    Then (iii) is a consequence of the fact that a connected reductive group has only finitely many nilpotent orbits in its Lie algebra (see \cite[\S 2.8, Theorem 1]{J04}, for example).
    Finally, (iv) follows from the fact that $\Phi^+ \cap \Phi_y$ is a system of positive roots in $\Phi_y$, as proved in \textsc{Lemma}~\ref{Lemma Nilpotents of Regular Connected Reductive Subgroup}(ii).
\end{proof}
\end{lemma}

For each $\alpha \in \Phi$, we consider the subgroup $G_{\alpha} \coloneqq G(\{ \alpha, -\alpha\}) = \llangle \U_{\alpha}, \U_{-\alpha} \rrangle$.
Using the results in \cite[\S 8.3]{MT11}, we can show that $G_{\alpha} \subseteq G$ is a semisimple subgroup of rank $1$, and thus \cite[Theorem 7.2.4]{S98} shows that it is isomorphic (as an algebraic group) to $\mathrm{SL}_2$ or $\mathrm{PGL}_2$.
Moreover, \cite[\S 8.3]{MT11} demonstrates that there exists an isomorphism of algebraic groups $\varphi \colon H \rightarrow G_{\alpha}$, where $H$ is either $\mathrm{SL}_2$ or $\mathrm{PGL}_2$, such that the image of the standard maximal torus coincides with $T \cap G_{\alpha}$, and the differential maps the two root spaces of $\h$ to $\g_{\alpha}$ and $\g_{-\alpha}$.

\vspace{2mm}

\begin{proposition}\label{Proposition Double Centraliser Properties} 
    Suppose $y \in \ft$, and $\alpha \in \Phi$.
\begin{itemize}
    \item[$\mathrm{(i)}$] There exists $x \in \g_{\alpha}$ and $x' \in \ft \oplus \g_{-\alpha}$ such that $\left[x,x'\right] \neq 0$.
    \item[$\mathrm{(ii)}$] $\fd_{\g} y \subseteq \ft$, and thus $\fd_{\g} y$ consists only of semisimple elements.
    \item[$\mathrm{(iii)}$] $\left(\fd_{\g} y\right)^\reg = \left(\fd_{\g} y\right)^\greg = \left\{ z \in \g \midd \fc_{\g} y = \fc_{\g} z \right\}$.
    \item[$\mathrm{(iv)}$] $\fd_{\g} y = \left\{ z \in \ft \midd \Phi_y \subseteq \Phi_z \right\}$ and $\left(\fd_{\g} y\right)^\reg = \left\{ z \in \ft \midd \Phi_y = \Phi_z \right\}$.
    \item[$\mathrm{(v)}$] If $z \in \left(\fd_{\g} y\right)^\reg$, then $\Co{G}y = \Co{G}z$.
\end{itemize}
\begin{proof}
    Let $\varphi \colon H \rightarrow G_{\alpha}$ be an isomorphism of algebraic groups as described above, where $H$ is either $\mathrm{SL}_2$ or $\mathrm{PGL}_2$.
    If $H = \mathrm{SL}_2$, take $x = \md \varphi \begin{psmallmatrix} 0 & 1 \\ 0 & 0 \end{psmallmatrix} \in \g_{\alpha}$ and $x' = \md \varphi \begin{psmallmatrix} 0 & 0 \\ 1 & 0 \end{psmallmatrix} \in \g_{-\alpha}$.
    Otherwise, for $H = \mathrm{PGL}_2$, let $\pi \colon \GL_2 \rightarrow \mathrm{PGL}_2$ denote the canonical quotient homomorphism, and take $x = \md \varphi \left( \md \pi \begin{psmallmatrix} 0 & 1 \\ 0 & 0 \end{psmallmatrix} \right) \in \g_{\alpha}$ and $x' = \md \varphi \left( \md \pi \begin{psmallmatrix} 1 & 0 \\ 0 & 0 \end{psmallmatrix} \right) \in \ft$.
    Simple calculations in either case then show that $\left[x,x'\right] \neq 0$, as required for (i).
    
    Using \textsc{Lemma}~\ref{Lemma Semisimple Centraliser Description}(ii), we have that $\fd_{\g} y \subseteq \fc_{\g} y = \ft \oplus \g_{\Phi_y}$.
    Since $\fc_{\g}y$ is $T$-stable, it follows that $\fd_{\g}y = \fc_{\g}\left(\fc_{\g}y\right)$ is also $T$-stable.
    Thus, in order to establish the first part of (ii), it suffices to prove that $\fd_{\g} y \cap \g_{\alpha} = 0$ (for each $\alpha \in \Phi_y$).
    Fix $\alpha \in \Phi_y$, and -- using (i) -- let $x \in \g_{\alpha} \subseteq \fc_{\g} y$ and $x' \in \ft \oplus \g_{-\alpha} \subseteq \fc_{\g} y$ be such that $\left[x,x'\right] \neq 0$.
    It follows that $x \notin \fc_{\g} x'$, and thus $x \notin \fc_{\g}\left(\fc_{\g} y\right) = \fd_{\g} y$; therefore, $\fd_{\g} y \cap \g_{\alpha}$ is a proper subspace of $\g_{\alpha}$.
    Since $\dim\g_{\alpha} = 1$, we have that $\fd_{\g} y \cap \g_{\alpha} = 0$, as required.
    
    The second part of (ii) is then immediate since $\ft$ consists only of semisimple elements, which also proves the first equality in (iii); the second equality in (iii) was observed in \S\ref{Subsection Stabilisers and Centralisers}.
    Then (iv) follows from (ii) and (iii), along with \textsc{Lemma}~\ref{Lemma Semisimple Centraliser Description}(ii).
    Finally, (v) follows from (iv) and \textsc{Lemma}~\ref{Lemma Semisimple Centraliser Description}(i).
\end{proof}
\end{proposition}

We note that, unless $p = 2$ and $H = \mathrm{PGL}_2$, the element $x'$ in \textsc{Proposition}~\ref{Proposition Double Centraliser Properties}(i) can be chosen to lie in $\g_{-\alpha}$.
Explicitly, if we let $x' = \md \varphi \left( \md \pi \begin{psmallmatrix} 0 & 0 \\ 1 & 0 \end{psmallmatrix} \right) \in \g_{-\alpha}$, then $\left[x,x'\right] = 0$ if and only if $p = 2$.

For any semisimple $y \in \g$, let $\dreg{\g} y \coloneqq \left(\fd_{\g} y\right)^\reg$, which is an open and dense subset of $\fd_{\g}y$.

\begin{corollary}\label{Corollary Equivalent Jordan Equivalence Definition}
    Suppose $x, y \in \g$.
    Then $x \sim y$ if and only if there exists $g \in G$ such that $\fc_{\g}\left( g \cdot x\s\right) = \fc_{\g}y\s$ and $g \cdot x\n = y\n$.
\begin{proof}
    It suffices to show that, for any $g \in G$, we have $\Co{G}\left( g \cdot x\s\right) = \Co{G}y\s$ if and only if $\fc_{\g}\left( g \cdot x\s\right) = \fc_{\g}y\s$.
    Since $\Lie \Co{G} z = \fc_{\g}z$ for any semisimple $z \in \g$, the forward direction is clear.
    The converse direction follows from \textsc{Proposition}~\ref{Proposition Double Centraliser Properties}(iii) and (v).
\end{proof}
\end{corollary}

Therefore, (for connected reductive algebraic groups) our definition of decomposition class given in \textsc{Definition}~\ref{Definition Decomposition Class} coincides with the definition of packet given in \cite[\S 1.2]{S82}.
For the remainder of the paper, we will use this equivalent definition of $x \sim y$, without reference to \textsc{Corollary}~\ref{Corollary Equivalent Jordan Equivalence Definition}.

Recall that a subset $Y$ (of a topological space $X$) is called \emph{constructible} if it is a finite union of locally closed subsets (of $X$).
We then have the following initial properties of decomposition classes, some of which are found in \cite[\S 3.3]{B98}, but not in the generality presented here.

\begin{theorem}\label{Theorem Initial Decomposition Class Properties}
    Suppose $x \in \g$.
\begin{itemize}
    \item[$\mathrm{(i)}$] $\J{G} x = G \cdot \left( \dreg{\g} x\s + x\n \right)$.
    \item[$\mathrm{(ii)}$] $\J{G} x$ is $G$-stable, and $\bbKx$-stable.
    \item[$\mathrm{(iii)}$] $\J{G} x$ is irreducible and constructible.
    \item[$\mathrm{(iv)}$] $\J{G} x + \z(\g) = \J{G} x$.
    \item[$\mathrm{(v)}$] $\J{G}x\n = \z(\g) + G \cdot x\n$.
    \item[$\mathrm{(vi)}$] There are only finitely many $G$-decomposition classes in $\g$.
\end{itemize}
\begin{proof}
    If $y \in \J{G} x$, then there exists $g \in G$ such that $\fc_{\g}\left( g \cdot x\s\right) = \fc_{\g}y\s$ and $g \cdot x\n = y\n$.
    Using \textsc{Proposition}~\ref{Proposition Double Centraliser Properties}(iii), $g^{-1} \cdot y\s \in \dreg{\g} x\s$, and thus $y = g \cdot \left( g^{-1} \cdot y\s + x\n\right) \in G \cdot \left( \dreg{\g} x\s + x\n \right)$.
    Conversely, if $y \in G \cdot \left( \dreg{\g} x\s + x\n \right)$, then there exists $h \in G$ and $z \in \dreg{\g} x\s$ such that $y = g \cdot \left(z + x\n\right)$.
    Since $y\s = g \cdot z$, it follows from \textsc{Proposition}~\ref{Proposition Double Centraliser Properties}(iii) that $\fc_{\g} \left( g^{-1} \cdot y\s\right) = \fc_{\g} x\s$.
    Therefore, $x \sim y$, and so $y \in \J{G} x$, as required for (i).

    The first part of (ii) is immediate from (i), so suppose $\lambda \in \mathbb{K}^{\times}$.
    Observe that $\fc_{\g} \left(\lambda x\s\right) =\nobreak \fc_{\g} x\s$, and $x\n \in \fc_{\g} x\s$.
    Since \textsc{Lemma}~\ref{Lemma Semisimple Centraliser Description}(iii) implies that there are only finitely many nilpotent $\Co{G}x\s$-orbits in $\fc_{\g} x\s$, \cite[Lemma 2.10]{J04} shows that there exists $g \in \Co{G}x\s$ such that $g \cdot x\n = \lambda x\n$.
    Then $\fc_{\g} \left( g \cdot x\s\right) = \fc_{\g} \left(\lambda x\s\right)$ and $g \cdot x\n = \lambda x\n$, hence $\lambda x \in \J{G}x$.

    As observed in \S\ref{Subsection Stabilisers and Centralisers}, $\dreg{\g} x\s$ is irreducible and locally closed, and therefore so is $\dreg{\g} x\s +\nobreak x\n$.
    Since $G$ is connected, and $\J{G}x$ is the image of $G \times \left( \dreg{\g} x\s + x\n \right)$ under the adjoint action, \cite[Lemma 1.2.3]{S98} and \cite[Corollary AG10.2]{B91} show that $\J{G}x$ is irreducible and constructible, respectively.

    Suppose $z \in \dreg{\g} x\s$ and $g \in G$.
    If $y \in \z(\g)$, then the Jordan decomposition of $g \cdot\nobreak \left( z +\nobreak x\n\right) +\nobreak y$ is $g \cdot \left(z + y\right) + g \cdot x\n$, and $\fc_{\g}\left(z+y\right) = \fc_{\g} z$.
    Therefore, (i) implies that $x \sim g \cdot \left( z + x\n\right) \sim g \cdot\nobreak \left(z +\nobreak x\n\right) +\nobreak y$, from which (iv) is immediate.
    Observing that $\dreg{\g} 0 = \z(\g)$ is $G$-stable shows that (v) follows by applying (i) to $x\n$.

    Recall that $T \subseteq G$ is a maximal torus, with root system $\Phi$.
    Using \textsc{Lemma}~\ref{Lemma Semisimple Centraliser Description}(ii), each $y \in \ft$ determines a subset of roots $\Phi_y \subseteq \Phi$ from which $\fc_{\g} y$ is determined.
    Since each semisimple element of $\g$ is $G$-conjugate to an element of $\ft$, there are (up to $G$-conjugacy) only finitely many centralisers of semisimple elements.
    For each semisimple element $z \in \g$, \textsc{Lemma}~\ref{Lemma Semisimple Centraliser Description}(iii) shows there are only finitely many nilpotent $\Co{G} z$-orbits in $\fc_{\g} z$, hence (vi) follows.
\end{proof}
\end{theorem}

We observe the following consequence of \textsc{Theorem}~\ref{Theorem Initial Decomposition Class Properties}(iv) and (v): if $x \in \g$ satisfies $x\s \in \z(\g)$, then $\J{G}x = \J{G}x\n = \z(\g) + G \cdot x$.
We also note that \textsc{Theorem}~\ref{Theorem Initial Decomposition Class Properties}(v) implies that $\J{G} 0 = \z(\g)$, and thus the centre of $\g$ is always a decomposition class; moreover, each nilpotent orbit is a decomposition class if and only if $\z(\g) = 0$.

A $G$\emph{-decomposition datum} corresponding to a decomposition class $\fJ \in \fD[G]$ is a pair $\left(\Co{G}x\s;x\n\right)$, for some $x \in \fJ$.
It is clear from the definition of decomposition classes that decomposition data are unique up to $G$-conjugacy, where $G$ acts simultaneously via the adjoint action on both arguments.
Suppose that $M \subseteq G$ is the connected stabiliser of a semisimple element of $\g$, and $e_0 \in \cN\left(\mathfrak{m}\right)$.
Then we let $\J{G}\left(M;e_0\right)$ denote the corresponding $G$-decomposition class; explicitly, if $y = y\s \in \g$ is such that $M = \Co{G}y$, then $\J{G}\left(M;e_0\right) \coloneqq\nobreak \J{G}\left(y + e_0\right)$.

\subsection{Decomposition Varieties}

The closure of a $G$-decomposition class is referred to as a $G$\emph{-decomposition variety}.
It follows from \textsc{Theorem}~\ref{Theorem Initial Decomposition Class Properties} that each decomposition variety is $G$-stable, irreducible, and $\bbKx$-stable; in fact, they are stable under arbitrary scalar multiplication: if $z \in \dvJ$, then $\bbKx z \subseteq \dvJ$, and thus $\bbK z = \overline{\bbKx z} \subseteq \dvJ$.

We note that (in general) decomposition varieties do not have constant stabiliser dimension, or constant centraliser dimension.
However, \textsc{Lemma}~\ref{Lemma Closure inside a Level Set}(a) and \textsc{Proposition}~\ref{Proposition Constant Dimension}(a) imply that $\fJ \subseteq \dvJ^\reg$; analogously, \textsc{Lemma}~\ref{Lemma Closure inside a Level Set}(b) and \textsc{Proposition}~\ref{Proposition Constant Dimension}(b) imply that $\fJ \subseteq \dvJ^\greg$.

Since $0 \in \dvJ$, we have that $\dvJ \cap \Ng \neq \emptyset$.
Moreover, it follows from $\J{G}0 = \z(\g)$ that $\fJ = \dvJ$ if and only if $\fJ = \z(\g)$, which was stated in \cite[\S 3.1]{A25} (for good characteristic).
We define a relation $\preceq$ on $\fD[G]$ by $\fJ \preceq \fJ'$ if and only if $\fJ \subseteq \overline{\fJ'}$.

\begin{proposition}\label{Proposition Closure Order for Decomposition Classes}
    Suppose $x,y \in \g$.
\begin{itemize}
    \item[$\mathrm{(i)}$] There exists a maximal subset $U_x \subseteq \J{G}x$ which is open and dense in $\dv{G}{x}$.
    \item[$\mathrm{(ii)}$] If $\dv{G}{x} = \dv{G}{y}$, then $\J{G}x = \J{G}y$.
    \item[$\mathrm{(iii)}$] $\preceq$ is a partial order on $\fD[G]$.
\end{itemize}
\begin{proof}
    Since $\J{G}x$ is constructible by \textsc{Theorem}~\ref{Theorem Initial Decomposition Class Properties}(iii), \cite[Lemma 2.1]{A12} implies that there exists a subset of $\J{G}x$ which is open and dense in $\dv{G}{x}$.
    Taking $U_x$ to be the union of all such subsets yields (i).

    Let $U_x \subseteq \J{G}x$ and $U_y \subseteq \J{G}y$ be the subsets described in (i).
    If $\dv{G}{x} = \dv{G}{y}$, then $U_x$ and $U_y$ are both open and dense in $\dv{G}{x}$.
    Therefore, $U_x \cap U_y \neq \emptyset$, and so $\J{G}x \cap \J{G}y \neq \emptyset$.
    Since distinct decomposition classes are disjoint, it follows that $\J{G}x = \J{G}y$.

    Reflexivity and transitivity of $\preceq$ are immediate from its definition, so it remains to prove antisymmetry.
    If $\J{G}x$ and $\J{G}y$ are such that $\J{G}x \preceq \J{G}y$ and $\J{G}y \preceq \J{G}x$, then $\dv{G}x = \dv{G}y$; therefore, (iii) follows from (ii).
\end{proof}
\end{proposition}

Therefore, a decomposition class is uniquely determined by its corresponding decomposition variety.
We note that the proof in \textsc{Proposition}~\ref{Proposition Closure Order for Decomposition Classes}(i) works for any constructible subset in any topological space.

We refer to $\preceq$ as the \emph{closure order} on $\fD[G]$, which is consequently a (finite) partially ordered set.
If we let $\prec$ denote the corresponding strict partial order, then \textsc{Proposition}~\ref{Proposition Closure Order for Decomposition Classes} implies that $\fJ \prec \fJ'$ if and only if $\dvJ \subsetneq \overline{\fJ'}$.

We say that $\J{G}y$ \emph{covers} $\J{G}x$ (or $\J{G}x$ is \emph{covered by} $\J{G}y$) if $\J{G}x \prec \J{G}y$, and there does not exist any $\J{G}z \in \fD[G]$ such that $\J{G}x \prec \J{G}z \prec \J{G}y$.
A set $\left\{ \J{G}x, \J{G}y \right\} \subseteq \fD[G]$ is called a \emph{covering pair} if $\J{G}y$ covers $\J{G}x$ or $\J{G}x$ covers $\J{G}y$.

\begin{corollary}\label{Corollary Dimension and the Closure Order}
    Suppose $x,y \in \g$ are such that $\J{G}x \prec \J{G}y$.
    Then $\dim \J{G}x < \dim \J{G}y$.
\begin{proof}
    As observed above, we have $\dv{G}{x} \subsetneq \dv{G}{y}$, and so $\dv{G}{x}$ is a proper closed subset of $\dv{G}{y}$.
    \textsc{Theorem}~\ref{Theorem Initial Decomposition Class Properties}(iii) implies that $\dv{G}{y}$ is irreducible, and thus the result follows from \cite[Proposition 1.22]{MT11}.
\end{proof}
\end{corollary}

A useful way to visually represent the poset structure on $\fD[G]$ is with a Hasse diagram (see \cite[p.\,279]{S12}, for example), which we shall now describe.
Let $\Gamma$ be the finite (undirected) graph with vertex set $\fD[G]$ and an edge between the elements of each covering pair.
It follows from \textsc{Corollary}~\ref{Corollary Dimension and the Closure Order} that it is possible to draw $\Gamma$ in the plane in a way that has the following properties:
\begin{itemize}
    \item Two decomposition classes lie on the same horizontal line if and only if they have the same dimension.
    \item A decomposition class is further in the upwards direction than another if and only if it has strictly greater dimension.
    \item If $\J{G}y$ covers $\J{G}x$, then the corresponding edge goes upwards from $\J{G}x$ to $\J{G}y$, and does not touch any vertices other than its end points.
\end{itemize}
Such a drawing, with the vertices and dimensions labelled, is referred to as the \emph{Hasse diagram} of $\fD[G]$.

\subsection{Further Closure Results}

In order to establish more properties about decomposition varieties, we need some general topological results about closures.
Suppose that $V = \bigoplus V_i$ is a vector space, decomposed as a direct sum of finitely many subspaces, and let $U_i \subseteq V_i$ be a collection of arbitrary non-empty subsets.
Each $V_i \subseteq V$ is a closed irreducible subset, and so $\widebar{U_i} \subseteq V_i$.
A simple induction argument then shows that $\widebar{\sum U_i} = \sum \widebar{U_i}$.

\begin{lemma}\label{Lemma Closure of Saturated Set}
    Suppose $\eta \colon V \rightarrow W$ is a linear map between vector spaces, and $X \subseteq V$ is such that $X = X + \ker \eta$.
    Then $\eta\left(\overline{X}\right) = \widebar{\eta(X)}$.
\begin{proof}
    First suppose that $W = \slant{V}{\ker\eta}$, and thus $\eta \colon V \rightarrow \slant{V}{\ker\eta}$ is a quotient of vector spaces.
    Since this is a continuous open surjection, it is a (topological) quotient map.
    Hence, for any $\eta$-saturated $X \subseteq V$ (which means that $X = X + \ker \eta$), we have that $\eta\left(\overline{X}\right) =\nobreak \eta(V) \cap\nobreak \widebar{\eta(X)} =\nobreak \widebar{\eta(X)}$, where the last equality holds by surjectivity.
    The general case then follows from properties of isomorphisms, and the fact that $\eta(V) \subseteq W$ is closed.
\end{proof}
\end{lemma}

The following lemma encapsulates a crucial property of parabolic subgroups, which we shall use in \textsc{Theorem}~\ref{Theorem Levi-type Decomposition Variety description}.

\begin{lemma}[{\cite[Proposition 0.15]{H95conj}}]\label{Lemma Parabolic-Stable Closed Subsets}
    Suppose $H$ is a connected algebraic group, and $K \subseteq H$ is a parabolic subgroup.
    Let $X$ be an $H$-variety, and suppose that $Y \subseteq X$ is a closed $K$-stable subset.
    Then $H \cdot Y \subseteq X$ is closed.
\end{lemma}

Since Borel subgroups are themselves parabolic subgroups, we can use this lemma to prove the following result about the closure of certain $G$-stable sums.

\begin{lemma}\label{Lemma Closure of Semisimple + Nilpotent}
    Suppose $X \subseteq \z(\g)$, and $Y \subseteq \Ng$ is a union of $($nilpotent$)$ $G$-orbits.
    Then $\widebar{X + Y} = \widebar{X} + \widebar{Y}$.
\begin{proof}
    Fix $B \in \mathfrak{P}(G,T)$, and let $U = U_B$.
    Since $\Ng = G \cdot \fu$ is closed, it follows that $\widebar{Y} = G \cdot Z$, where $Z = \widebar{Y} \cap \fu$.
    Then $X \subseteq \ft$ and $Z \subseteq \fu$ imply that $\widebar{X + Z} = \widebar{X} + \widebar{Z}$ (inside the direct sum $\mathfrak{b} = \ft \oplus \fu$).
    Since $\fu$ is $B$-stable, and $\widebar{X}$ and $\widebar{Y}$ are both $G$-stable, it follows that $\widebar{X}$ and $\widebar{Z} = Z$ are both $B$-stable.
    Therefore, $\widebar{X} + Z$ is $B$-stable and closed; hence, \textsc{Lemma}~\ref{Lemma Parabolic-Stable Closed Subsets} implies that $\widebar{X} + \widebar{Y} = G \cdot \left( \widebar{X} + Z \right)$ is also closed, which proves that $\widebar{X + Y} \subseteq \widebar{X} + \widebar{Y}$.
    Since $\widebar{X} + \widebar{Y} = G \cdot \left( \widebar{X + Z}\right) \subseteq \overline{G \cdot (X + Z)} = \widebar{X + Y\vphantom{G \cdot (X + Z)}}$, the result follows.
\end{proof}
\end{lemma}

We note that \textsc{Lemma}~\ref{Lemma Closure of Semisimple + Nilpotent} is much simpler to prove if $\g = \z(\g) \oplus \Lie (G,G)$, where $(G,G)$ denotes the derived subgroup of $G$; see \cite[Corollary 2.3.9]{L05} for a sufficient condition on $p \geq 0$ for this to hold.
The following consequence of \textsc{Lemma}~\ref{Lemma Closure of Semisimple + Nilpotent} provides an analogue of \textsc{Theorem}~\ref{Theorem Initial Decomposition Class Properties}(iv) and (v) for decomposition varieties.

\begin{proposition}\label{Proposition Decomposition Varieties Properties}
    Suppose $x \in \g$.
\begin{itemize}
    \item[$\mathrm{(i)}$] $\dv{G}{x} + \z(\g) = \dv{G}{x}$.
    \item[$\mathrm{(ii)}$] $\dv{G}{x\n} = \z(\g) + \overline{G \cdot x\n}$.
    \item[$\mathrm{(iii)}$] If $y \in \dv{G}{x\n}$, then $y\s \in \z(\g)$ and $y\n \in \overline{G \cdot x\n}$.
\end{itemize}
\begin{proof}
    Using \textsc{Theorem}~\ref{Theorem Initial Decomposition Class Properties}(iv), we know that $\J{G} x + \z(\g) = \J{G} x$.
    Fix $z \in \z(\g)$, and consider the isomorphism of vector spaces $\eta \colon \g \rightarrow \g$ defined by $y \mapsto y + z$, under which $\J{G}x$ is stable.
    Since it is a homeomorphism of topological spaces, $\eta \colon \g \rightarrow \g$ preserves closures, and thus $\eta\left(\dv{G}{x}\right) = \widebar{\eta\left(\J{G}x\right)} = \dv{G}{x}$.
    Therefore, $\dv{G}{x} + z = \dv{G}{x}$, from which (i) follows.

    For (ii), first use \textsc{Theorem}~\ref{Theorem Initial Decomposition Class Properties}(v) to get $\J{G}x\n = \z(\g) + G \cdot x\n$.
    If $T \subseteq G$ is any maximal torus, then \textsc{Proposition}~\ref{Proposition Double Centraliser Properties}(ii) implies that $\z(\g) = \dreg{\g}0 \subseteq \ft$.
    Therefore, \textsc{Lemma}~\ref{Lemma Closure of Semisimple + Nilpotent} implies that $\overline{\z(\g) + G \cdot x\n} = \overline{\z(\g)} + \overline{G \cdot x\n\vphantom{\z(\g)}}$.
    Hence, (ii) follows from the fact that $\z(\g) \subseteq \g$ is closed, and (iii) follows from (ii) and the uniqueness of the Jordan decomposition.
\end{proof}
\end{proposition}

\section{Preservation of Decomposition Classes}\label{Section Preservation of Decomposition Classes}

In this section we shall explore how decomposition classes interact with direct products, central surjections, and separable central surjections.
Suppose throughout that $H$ is also a connected reductive algebraic group, just as $G$ is.

The direct product $G \times H$ is also a connected reductive algebraic group, with Lie algebra $\g \oplus \h$, whose structure is easily determined by the structures of $G$ and $H$.
Suppose $x \in \g$ and $y \in \h$, and consider the decomposition class of $x+y \in \g \oplus \h$.
It follows readily from the definitions that $\J{G \times H}(x + y) = \J{G}x + \J{H}y$; consequently, $\fD[G \times H] = \fD[G] \times \fD[H]$ as sets, where the closure order on $\fD[G \times H]$ coincides with the product order induced from $\fD[G]$ and $\fD[H]$.
By induction, this extends to arbitrary finite direct products of connected reductive algebraic groups.

\subsection{Preservation by Central Surjections}\label{Subsection Preservation by Central Surjections}

A surjective homomorphism of algebraic groups $\varphi \colon G \rightarrow H$ is said to be \emph{central} if $\ker \varphi \subseteq \mathcal{Z}_G$ and $\ker \md\varphi \subseteq \z(\g)$.
Fix a central surjection $\varphi \colon G \rightarrow H$, and note that the differential $\md\varphi \colon \g \rightarrow \h$ is not necessarily surjective (see \S\ref{Subsection Preservation by Separable Central Surjections}).

Suppose $T \subseteq G$ is a maximal torus, and $B \in \mathfrak{P}(G,T)$.
Then it follows from \cite[\S 2.7]{J04} that $\check{T} \coloneqq \varphi (T) \subseteq H$ is a maximal torus, $\check{B} \coloneqq \varphi(B) \in \mathfrak{B}\left(H,\check{T}\right)$, and $U_{\check{B}} = \varphi\left(U_B\right)$.
The induced comorphism $\varphi^* \colon \bbK[H] \rightarrow \bbK[G]$ restricts to a homomorphism of character groups $\left(\varphi\restr_T\right)^* \colon \mathrm{X}\left(\check{T}\right) \rightarrow \mathrm{X}(T)$.
Then \cite[Proposition 22.4]{B91} implies that this further restricts to a bijection of root systems $\check{\Phi} =\Phi\left(H,\check{T}\right) \rightarrow \Phi = \Phi(G,T)$.
Given any $x \in \g$, let $\check{x} \coloneqq \md \varphi (x) \in \h$ denote its image under the differential $\md \varphi \colon \g \rightarrow \h$, and let $\check{\g} \coloneqq \md\varphi(\g) \subseteq \h$.

\begin{lemma}[{\cite[Proposition 2.7(a)]{J04}}]\label{Lemma Central Surjections and Nilpotent Structure}
    The restriction of $\md\varphi \colon \g \rightarrow \h$ to $\Ng$ has the following properties, for each $x \in {\Ng}{:}$
\begin{itemize}
    \item[$\mathrm{(i)}$] It is a bijection $\Ng \rightarrow \cN(\h)$.
    \item[$\mathrm{(ii)}$] It induces a bijection $\slant{\Ng}{G} \rightarrow \slant{\cN(\h)}{H}$.
    \item[$\mathrm{(iii)}$] It restricts to a bijection $G \cdot x \rightarrow H \cdot \check{x}$.
    \item[$\mathrm{(iv)}$] $\varphi\left(\C_G x\right) = \C_H \check{x}$.
\end{itemize}
\end{lemma}

Therefore, the structure of the nilpotent cone is completely preserved by central surjections.
We note that \textsc{Lemma}~\ref{Lemma Central Surjections and Nilpotent Structure}(iv) is not necessarily true (in general) for non-nilpotent elements.
Since the Jordan decomposition is preserved by differentials of algebraic group homomorphisms, we have that $\check{x}\s = \md\varphi\left(x\s\right)$ and $\check{x}\n = \md\varphi\left(x\n\right)$, for any $x \in \g$.

\begin{proposition}\label{Proposition Central Surjections and Semisimple Centralisers}
    Suppose $y \in \g$ is semisimple.
\begin{itemize}
    \item[$\mathrm{(i)}$] $\md\varphi\left(\fc_{\g}y\right) = \check{\g} \cap \fc_{\h} \check{y}$.
    \item[$\mathrm{(ii)}$] $\md\varphi\left(\fd_{\g}y\right) = \check{\g} \cap \fd_{\h} \check{y}$.
    \item[$\mathrm{(iii)}$] $\md\varphi\left(\dreg{\g}y\right) = \check{\g} \cap \dreg{\h} \check{y}$.
\end{itemize}
\begin{proof}
    Fix a maximal torus $T \subseteq G$ such that $y \in \ft$.
    Since $\md\varphi \colon \g \rightarrow \h$ is a Lie algebra homomorphism, the inclusion $\md\varphi\left(\fc_{\g}y\right) \subseteq \check{\g} \cap \fc_{\h} \check{y}$ is immediate.
    For the converse, suppose $x \in \g$ is such that $\check{x} \in \fc_{\h}\check{y}$.
    Hence $[y,x] \in \ker \md\varphi \subseteq \z(\g) \subseteq \ft$, and since $\left[y,x_{\alpha}\right] = \md\alpha(y)x_{\alpha}$ -- for each $x_{\alpha} \in \g_{\alpha}$ and $\alpha \in \Phi$ -- it follows that $x \in \ft \oplus \bigoplus_{\alpha \in \Phi_y} \g_{\alpha}$.
    Therefore, \textsc{Lemma}~\ref{Lemma Semisimple Centraliser Description}(ii) implies (i).

    Suppose now that $x \in \fd_{\g}y \subseteq \ft$, and observe that $\Phi_y \subseteq \Phi_x$ by \textsc{Proposition}~\ref{Proposition Double Centraliser Properties}(iv).
    Since $\beta \mapsto \beta \circ \varphi$ is a bijection $\check{\Phi} \rightarrow \Phi$, it follows from the definitions that this restricts to a bijection $\check{\Phi}_{\check{z}} \rightarrow \Phi_z$, for any $z \in \ft$.
    Therefore, $\check{\Phi}_{\check{y}} \subseteq \check{\Phi}_{\check{x}}$, and thus $\md\varphi\left(\fd_{\g}y\right) \subseteq \check{\g} \cap \fd_{\h} \check{y}$.
    Conversely, suppose that $x \in \g$ is such that $\check{x} \in \fd_{\g}\check{y} = \left\{ z \in \h \midd \fc_{\h} \check{y} \subseteq \fc_{\h} z \right\}$.
    Then, for any $z \in \fc_{\g}y$, (i) implies that $\check{z} \in \check{\g} \cap \fc_{\h}\check{y} \subseteq \check{\g} \cap \fc_{\h}\check{x}$, and so the same argument used in (i) shows that $z \in \fc_{\g} x$.
    Therefore, $\fc_{\g}y \subseteq \fc_{\g}x$, from which (ii) follows.

    For (iii), suppose that $x \in \dreg{\g}y$, and observe that $\dreg{\g}y = \left\{ z \in \ft \midd \Phi_y = \Phi_x\right\}$, again by \textsc{Proposition}~\ref{Proposition Double Centraliser Properties}(iv).
    Thus $\check{\Phi}_{\check{y}} = \check{\Phi}_{\check{x}}$, and hence $\md\varphi\left(\dreg{\g}y\right) \subseteq \check{\g} \cap \dreg{\h} \check{y}$.
    Conversely, suppose $x \in \g$ is such that $\check{x} \in \dreg{\h}\check{y} \subseteq \fd_{\h}\check{y}$.
    Using (ii), we know that $x \in \fd_{\g}y$; however, $\check{y} \in \fd_{\h}\check{x}$, and so (ii) implies that $y \in \fd_{\g}x$.
    Therefore, $\fc_{\g}y = \fc_{\g}x$, from which the other direction of (iii) follows.
\end{proof}
\end{proposition}

\begin{theorem}\label{Theorem Central Surjections and Decomposition Classes}
    Suppose $x,y \in \g$.
\begin{itemize}
    \item[$\mathrm{(i)}$] $\md\varphi\left(\J{G}x\right) = \check{\g} \cap \J{H}\check{x}$.
    \item[$\mathrm{(ii)}$] $\md\varphi\left(\dv{G}{x}\right) = \widebar{\check{\g} \cap \J{H}\check{x}}$.
    \item[$\mathrm{(iii)}$] $\J{G}x \subseteq \dv{G}{y}$ if and only if $\md\varphi\left(\J{G}x\right) \subseteq \md\varphi\left(\dv{G}{y}\right)$.
\end{itemize}
\begin{proof}
    Using \textsc{Theorem}~\ref{Theorem Initial Decomposition Class Properties}(i) in conjunction with \textsc{Proposition}~\ref{Proposition Central Surjections and Semisimple Centralisers}(iii), we have that $\md\varphi\left(\J{G}x\right) = \varphi(G) \cdot \big(\md\varphi\left(\dreg{\g}x\s\right) + \md\varphi\left(x\n\right)\big) = H \cdot \left(\check{\g} \cap \dreg{\h}\check{x}\s + \check{x}\n\right)$.
    Since $\check{\g} = \md\varphi(\g)$ is $H$\mbox{-stable}, we have $H \cdot \left(\check{\g} \cap \dreg{\h}\check{x}\s + \check{x}\n\right) = \check{\g} \cap H \cdot \left(\dreg{\h}\check{x}\s + \check{x}\n\right)$, and thus (i) follows by using \textsc{Theorem}~\ref{Theorem Initial Decomposition Class Properties}(i) again.

    Since $\md\varphi \colon \g \rightarrow \h$ is linear, and $\ker\md\varphi \subseteq \z(\g)$, it follows from \textsc{Theorem}~\ref{Theorem Initial Decomposition Class Properties}(iv) and \textsc{Lemma}~\ref{Lemma Closure of Saturated Set} that $\md\varphi\left(\dv{G}{x}\right) = \overline{\md\varphi\left(\J{G}x\right)}$.
    Therefore, (ii) follows from (i).
    
    The forward direction of (iii) is immediate, so suppose that $\md\varphi\left(\J{G}x\right) \subseteq \md\varphi\left(\dv{G}{y}\right)$, and let $z \in \J{G}x$.
    Then $\md\varphi(z) \in \md\varphi\left(\dv{G}{y}\right)$, and so $z \in (\md\varphi)^{-1}\left(\md\varphi\left(\dv{G}{y}\right)\right) = \dv{G}{y} + \ker\md\varphi \subseteq \dv{G}{y} + \z(\g)$.
    It then follows from \textsc{Proposition}~\ref{Proposition Decomposition Varieties Properties}(i) that $z \in \dv{G}{y}$, which proves the other direction of (iii).
\end{proof}
\end{theorem}

\subsection{Preservation by Separable Central Surjections}\label{Subsection Preservation by Separable Central Surjections}

Suppose still that $\varphi \colon G \rightarrow H$ is a central surjection (of connected reductive algebraic groups), and retain the other notation from \S\ref{Subsection Preservation by Central Surjections}.
It then follows from \cite[Theorem 4.3.7(iii)]{S98} that $\varphi$ is separable if and only if $\md\varphi \colon \g \rightarrow \h$ is surjective (equivalently, $\check{\g} = \h$).
As indicated by \textsc{Proposition}~\ref{Proposition Central Surjections and Semisimple Centralisers} and \textsc{Theorem}~\ref{Theorem Central Surjections and Decomposition Classes}, separable central surjections preserve much more of the structure of decomposition classes.

\begin{theorem}\label{Theorem Separable Central Surjections and Decomposition Classes}
    Suppose that $\varphi \colon G \rightarrow H$ is a separable central surjection, and let $x,y \in \g$.
    Then $\J{G}x \mapsto \J{H}\check{x}$ is a bijection $\fD[G] \rightarrow \fD[H]$ with the following properties$:$
\begin{itemize}
    \item[$\mathrm{(i)}$] $\md\varphi \colon \g \rightarrow \h$ restricts to a surjection $\J{G}x \rightarrow \J{H}\check{x}$.
    \item[$\mathrm{(ii)}$] Preservation of closure$:$ $\md\varphi\left(\dv{G}{x}\right) = \dv{H}{\check{x}}$.
    \item[$\mathrm{(iii)}$] Preservation of the partial order$:$ $\J{G}x \preceq \J{G}y$ if and only if $\J{H}\check{x} \preceq \J{H}\check{y}$.
    \item[$\mathrm{(iv)}$] $\dim \J{G}x = \dim\ker \md\varphi +  \dim \J{H}\check{x}$.
\end{itemize}
\begin{proof}
    Using \textsc{Theorem}~\ref{Theorem Central Surjections and Decomposition Classes}(i), we know that $\md\varphi\left(\J{G}x\right) = \J{H}\check{x} \in \fD[H]$, from which we can conclude that $\J{G}x \mapsto \J{H}\check{x}$ is a well-defined map $\fD[G] \rightarrow \fD[H]$.
    Its surjectivity follows immediately from the surjectivity of $\md\varphi \colon \g \rightarrow \h$.
    For injectivity, suppose that $\J{H}\check{x} =\nobreak \J{H}\check{y}$, from which \textsc{Theorem}~\ref{Theorem Initial Decomposition Class Properties}(iv) implies that $\J{G}x = \J{G}x + \ker\md\varphi = (\md\varphi)^{-1}\left(\md\varphi\left(\J{G}x\right)\right) = (\md\varphi)^{-1}\left(\md\varphi\left(\J{G}y\right)\right) = \J{G}y + \ker\md\varphi = \J{G}y$.

    Since $\md\varphi \colon \g \rightarrow \h$ is a surjection, so is its restriction to $\J{G}x$, which proves (i).
    Using $\check{\g} = \h$, (ii) and (iii) immediately follow from \textsc{Theorem}~\ref{Theorem Central Surjections and Decomposition Classes}(ii) and (iii), respectively.
    
    Let $X = \dv{G}{x}$ and $Y = \dv{H}{\check{x}}$, and observe that $\eta \coloneqq \left(\md\varphi\right)\restr_X \colon X \rightarrow Y$ is a surjective morphism of irreducible varieties.
    Then (iv) follows from \cite[Theorem 5.1.6]{S98}, and the fact that $\dim \dvJ = \dim \fJ$, for any $\fJ \in \fD[G]$.
\end{proof}
\end{theorem}

This proves \textsc{Theorem}~\ref{Theorem Decomposition Class Structure and Separable Central Surjections Main Result} from the introduction.
It follows that the Hasse diagrams of $\fD[G]$ and $\fD[H]$ can be deduced from one another, whenever there is a separable central surjection $\varphi \colon G \rightarrow H$.
In particular, suppose we already have the Hasse diagram for $\fD[G]$.
To get the Hasse diagram for $\fD[H]$ we replace each decomposition class label with its image under $\md\varphi$, and subtract $\dim\ker\md\varphi$ from each dimension label.

Let $G_{\ad}$ denote the adjoint group corresponding to the semisimple algebraic group $\slant{G}{\mathcal{Z}^{\circ}_G}$, and let $\pi \colon G \rightarrow G_{\ad}$ denote the composition morphism of the projection $G \rightarrow \slant{G}{\mathcal{Z}^{\circ}_G}$ and the central isogeny $\slant{G}{\mathcal{Z}^{\circ}_G} \rightarrow G_{\ad}$.
Then \cite[Remark 2.3.6]{L05} shows that $\ker \pi = \mathcal{Z}_G$ and $\ker \md\pi = \z(\g)$, and thus $\pi \colon G \rightarrow G_{\ad}$ is a central surjection.
Using \cite[Corollary 2.3.7]{L05}, we have that $\pi \colon G \rightarrow G_{\ad}$ is separable if and only if $p$ does not divide $\lvert \left(\slant{\mathrm{X}(T)}{\mathbb{Z}\Phi}\right)_{\mathrm{tor}}\rvert$.
Therefore, if $p \geq 0$ is very good for $G$, then we can apply \textsc{Theorem}~\ref{Theorem Separable Central Surjections and Decomposition Classes} to the separable central surjection $\pi \colon G \rightarrow G_{\ad}$.

On the other hand, if $G = \GL_n$, then $G_{\ad} = \mathrm{PGL}_n$ and $\left(\slant{\mathrm{X}(T)}{\mathbb{Z}\Phi}\right)_{\mathrm{tor}}$ is trivial.
Therefore, (for any characteristic) the canonical projection $\pi \colon \GL_n \rightarrow \mathrm{PGL}_n$ is a separable central surjection, and thus we can apply \textsc{Theorem}~\ref{Theorem Separable Central Surjections and Decomposition Classes} to conclude that the Hasse diagram for $\fD\left[\mathrm{PGL}_n\right]$ is just the Hasse diagram for $\fD\left[\GL_n\right]$ with all of the dimension labels reduced by $1$.

\section{Sheets}\label{Section Sheets}

As explained in the introduction, decomposition classes were originally introduced in \cite{BK79} as a tool to study the sheets of $\g$.
In the existing literature, these are the maximal irreducible subsets of $\g$ consisting of equal-dimension orbits.
However, we will make a departure with the following definitions.

\begin{definition}\label{Definition Sheets}
~
    \begin{itemize}
        \item An irreducible component $S$ of a non-empty level set is called a \itbf{sheet} of $\g$.
        \item $S$ is a \itbf{stabiliser sheet} if it is an irreducible component of a stabiliser level set.
        \item $S$ is a \itbf{centraliser sheet} if it is an irreducible component of a centraliser level set.
    \end{itemize}
\end{definition}

Since each level set of $\g$ is (at least one of) a stabiliser level set or a centraliser level set, each sheet is (at least one of) a stabiliser sheet or a centraliser sheet.
With these definitions, it is stabiliser sheets that have been studied so far in the literature.
The change in nomenclature will allow us to uniformly state certain results about both types of sheet, whilst also highlighting differences (see \S\ref{Subsection Levi-Type Sheets}).

Each stabiliser sheet lies in a unique stabiliser level set, and similarly each centraliser sheet lies in a unique centraliser level set.
If $\Lie\left(\C_Gx\right) = \fc_{\g}x$ for all $x \in \g$, then $\g_{(m)} = \g_{[m]}$ for all $m \in \N$, and thus stabiliser sheets and centraliser sheets coincide; see \S\ref{Subsection Stabilisers and Centralisers} for a discussion on when this separability condition holds.

Given a (non-empty) level set $\ls$, we say that $S$ is a \emph{sheet of} $\ls$ if $S \subseteq \ls$ is an irreducible component of $\ls$.
We note that it is possible for a sheet to be a subset of a level set, without being a sheet of that level set.
For example, if $G = \mathrm{PGL}_2$ then $S = \g_{[2]}$ is a centraliser sheet, and $S \subseteq \g_{(1)}$, but $S$ is not an irreducible component of $\g_{(1)}$.

\begin{proposition}\label{Proposition Centre is a Sheet}
    If $d = \dim G$, then $\z(\g) = \g_{(d)} = \g_{[d]}$.
    Therefore, $\z(\g)$ is both a stabiliser sheet and a centraliser sheet.
\begin{proof}
    Firstly, $x \in \g_{[d]}$ if and only if $\dim \fc_{\g} x = \dim \g$, if and only if $\fc_{\g} x = \g$, if and only if $x \in \z(\g)$; and thus $\g_{[d]} = \z(\g)$.

    If $x \in \z(\g)$, then $x$ is semisimple, and so $\dim \C_G x = \dim \fc_{\g} x = \dim \g = d$; therefore, $\z(\g) \subseteq \g_{(d)}$.
    Conversely, suppose that $x \in \g_{(d)}$.
    Since $\dim \fc_{\g}x \geq \dim \C_G x = d$, we have that $\dim \fc_{\g}x = d$.
    Therefore, $\fc_{\g} x = \g$, and so $x \in \z(\g)$.

    Finally, $\z(\g) = \g_{(d)} = \g_{[d]}$ is its own irreducible component, from which it follows that $\z(\g)$ is both a stabiliser sheet and a centraliser sheet.
\end{proof}
\end{proposition}

\subsection{Properties of Sheets}

Many of the results here have already been established for stabiliser sheets, sometimes with additional assumptions on the characteristic (see \cite{BK79}, \cite{B81}, and \cite{S82}, for example).
However, we can now extend these to arbitrary sheets in all characteristics.

\begin{lemma}\label{Lemma Closure of Sheet in Level Set}
    Suppose $S$ is a sheet of $\ls$.
\begin{itemize}
    \item[$\mathrm{(i)}$] $S = \bbKx S = \widebar{S} \cap \ls = \overline{\bbKx S} \cap \ls = \overline{\bbK S} \cap \ls$.
    \item[$\mathrm{(ii)}$] $S$ is locally closed and $G$-stable.
    \item[$\mathrm{(iii)}$]
        \begin{itemize}[leftmargin=18pt]
            \item[$\mathrm{(a)}$] If $S$ is a stabiliser sheet, then $S = \widebar{S}^\reg = \overline{\bbKx S}^\reg = \overline{\bbK S}^\reg$.
            \item[$\mathrm{(b)}$] If $S$ is a centraliser sheet, then $S = \widebar{S}^\greg = \overline{\bbKx S}^\greg = \overline{\bbK S}^\greg$.
        \end{itemize}
\end{itemize}
\begin{proof}
    Consider the scalar multiplication map $\bbKx \times S \rightarrow \g$, which is a morphism of affine varieties.
    Both $\bbKx$ and $S$ are irreducible, and thus so is their image $\bbKx S$.
    Recall that $\bbKx \ls = \ls$, and thus $\bbKx S \subseteq \ls$; hence $S = \bbKx S$ by maximality.
    Since $S$ is necessarily closed in $\ls$, we have that $S = \widebar{S} \cap \ls = \overline{\bbKx S} \cap \ls$.
    
    For (i), it remains to prove the last equality.
    Observe that $\bbK S = \{0\} \cup \bbKx S$, and so $\overline{\bbK S} = \{0\} \cup \overline{\bbKx S}$.
    If $0 \notin \ls$ then $\overline{\bbK S} \cap \ls = \overline{\bbK^{\times} S} \cap \ls$.
    Otherwise, $0 \in \ls$, and so \textsc{Proposition}~\ref{Proposition Centre is a Sheet} implies that $\ls = S = \z(\g)$, and hence $\overline{\bbK S} \cap \ls = \z(\g) = \overline{\bbK^{\times} S} \cap \ls$.

    For (ii), since $S = \widebar{S} \cap \ls$ and $\ls$ is locally closed, it follows that $S$ is also locally closed.
    Since $\overline{G \cdot S} \cap \ls$ is also irreducible in $\ls$, maximality implies that $S = \overline{G \cdot S} \cap \ls$, from which $G \cdot S = S$ follows.

    If $S$ is a stabiliser sheet, then \textsc{Lemma}~\ref{Lemma Closure inside a Level Set}(a) implies that $\widebar{S} \cap \ls = \widebar{S}^\reg$, and so the first two equalities in (iii)(a) follow from (i).
    A similar argument using \textsc{Lemma}~\ref{Lemma Closure inside a Level Set}(b) shows the same for (iii)(b).
    The final equality in both cases follows from the fact that $\overline{\bbK S} = \{0\} \cup \overline{\bbKx S}$.
\end{proof}
\end{lemma}

Suppose $\ls$ is a level set, and $\fJ \subseteq \ls$ is a decomposition class.
Since $\fJ$ is irreducible, there must exist some (not necessarily unique) sheet $S$ of $\ls$ such that $\fJ \subseteq S$.
Therefore, each decomposition class lies in at least one stabiliser sheet, and at least one centraliser sheet.

\begin{proposition}\label{Proposition Each Sheet Contains Unique Dense Decomposition Class}
    Each sheet contains a unique dense decomposition class.
\begin{proof}
    Suppose that $S$ is a sheet of the level set $\ls$.
    Let $\fJ_1, \ldots, \fJ_r$ be the (finitely many) decomposition classes such that $\ls = \bigsqcup \fJ_i$.
    Since $S = \bigsqcup \left(S \cap \fJ_i\right)$ is closed in $\ls$, it follows that $S = \bigcup \left(\overline{S \cap \fJ_i} \cap \ls\right)$, which is a finite union of closed subsets of $\ls$.
    Since $S$ is irreducible, there exists some $1 \leq j \leq r$ such that $S = \overline{S \cap \fJ_j} \cap \ls \subseteq \widebar{S} \cap \overline{\fJ_j} \cap \ls = S \cap \overline{\fJ_j}$, where the final equality uses \textsc{Lemma}~\ref{Lemma Closure of Sheet in Level Set}(i).
    Then $S \subseteq \overline{\fJ_j}$, and so $\overline{\fJ_j} \cap \ls$ is an irreducible subset of $\ls$ containing $S$; then the maximality of $S$ forces $S = \overline{\fJ_j} \cap \ls$.
    It follows that $\fJ_j \subseteq \overline{\fJ_j} \cap \ls = S \subseteq \overline{\fJ_j}$, and hence $\fJ_j$ is dense in $S$.

    For uniqueness, suppose $\fJ_i$ is also dense in $S$, for some $1 \leq i \leq r$.
    By \textsc{Proposition}~\ref{Proposition Closure Order for Decomposition Classes}(i) there exist maximal subsets $U_i \subseteq \fJ_i$ and $U_j \subseteq \fJ_j$ which are open and dense in $\overline{\fJ_i} = \overline{S} = \overline{\fJ_j}$.
    Therefore, $U_i \cap U_j \neq \emptyset$ and thus $\fJ_i \cap \fJ_j \neq \emptyset$, which shows that $\fJ_i = \fJ_j$.
\end{proof}
\end{proposition}

Given a sheet $S$, let $\fD_S$ denote its (unique) dense decomposition class.
It is clear from the proof of \textsc{Proposition}~\ref{Proposition Each Sheet Contains Unique Dense Decomposition Class} that, if $S$ is a sheet of $\ls$, then $\fD_S$ is the unique element $\fJ \in \fD_{\am}[G]$ such that $S = \overline{\fJ} \cap \ls$.
Since $\widebar{S} = \overline{\fD_S}$, the following corollary is immediate from \textsc{Lemma}~\ref{Lemma Closure of Sheet in Level Set}.

\begin{corollary}\label{Corollary Closure of Dense Decomposition Class}
    Suppose $S$ is a sheet of $\ls$.
\begin{itemize}
    \item[$\mathrm{(i)}$] $S = \overline{\fD_S} \cap \ls$.
    \item[$\mathrm{(ii)}$]
        \begin{itemize}[leftmargin=18pt]
            \item[$\mathrm{(a)}$] If $S$ is a stabiliser sheet, then $S = \overline{\fD_S}^\reg$.
            \item[$\mathrm{(b)}$] If $S$ is a centraliser sheet, then $S = \overline{\fD_S}^\greg$.
        \end{itemize}
\end{itemize}
\end{corollary}

\subsection{The Closure Order in Level Sets}

Suppose that $\ls$ is a level set, and consider the restriction of the closure order $\preceq$ to $\fD_{\am}[G]$.
The \emph{Hasse diagram} of $\fD_{\am}[G]$ is defined to be the subgraph of the Hasse diagram of $\fD[G]$ induced by $\fD_{\am}[G]$.

\begin{definition}
    Suppose $\ls$ is a level set, and $\fJ,\fJ' \in \fD_{\am}[G]$.
\begin{itemize}
    \item $\fJ$ is \itbf{maximal in} $\ls$ if $\fJ \preceq \fJ'$ always implies that $\fJ = \fJ'$.
    \item $\fJ$ is \itbf{minimal in} $\ls$ if $\fJ' \preceq \fJ$ always implies that $\fJ = \fJ'$.
    \item $\fJ$ is \itbf{isolated in} $\ls$ if it is both maximal and minimal in $\ls$.
\end{itemize}
\end{definition}

By looking at the Hasse diagram of $\fD_{\am}[G]$, we can determine visually if a decomposition class $\fJ \subseteq \ls$ is maximal/minimal or isolated in $\ls$: $\fJ$ is maximal/minimal in $\ls$ if and only if there are no edges whose lower/upper end point is $\fJ$, and $\fJ$ is isolated in $\ls$ if and only if there are no edges for which $\fJ$ is an end point.

\begin{theorem}\label{Theorem Irreducible Components of Level Sets}
    Suppose $\ls$ is a level set, and $\fJ \in \fD_{\am}[G]$.
\begin{itemize}
    \item[$\mathrm{(i)}$] $\fJ$ is maximal in $\ls$ if and only if $\fJ = \fD_S$ $($for a sheet $S$ of $\ls)$.
    \item[$\mathrm{(ii)}$] The sheets of $\ls$ are in bijection with the maximal decomposition classes in $\ls$.
    \item[$\mathrm{(iii)}$] If $\fJ$ coincides with a sheet of $\ls$ then it is isolated in $\ls$.
\end{itemize}
\begin{proof}
    Suppose $\fJ = \fD_S$ for some sheet $S$ of $\ls$, and that $\fJ' \in \fD_{\am}[G]$ satisfies $\fD_S \subseteq \widebar{\fJ'}$.
    Let $S' \subseteq \ls$ be a sheet of $\ls$ such that $\fJ' \subseteq S'$.
    Then \textsc{Corollary}~\ref{Corollary Closure of Dense Decomposition Class}(i) and \textsc{Lemma}~\ref{Lemma Closure of Sheet in Level Set}(i) imply that $S = \overline{\fD_S} \cap \ls \subseteq \widebar{\fJ'} \cap \ls \subseteq \widebar{S'} \cap \ls = S'$.
    Since both are irreducible components of $\ls$, we have that $S = S'$, and so the above inclusions imply that $S = \widebar{\fJ'} \cap \ls$.
    Therefore, $\fJ' = \fD_S$, and so $\fD_S$ is maximal in $\ls$.

    Conversely, suppose $\fJ$ is maximal in $\ls$, and let $S$ be a sheet of $\ls$ such that $\fJ \subseteq S$.
    Then \textsc{Corollary}~\ref{Corollary Closure of Dense Decomposition Class}(i) implies that $\fJ \subseteq S = \overline{\fD_S} \cap \ls \subseteq \overline{\fD_S}$.
    Since $\fD_S \subseteq \ls$, the maximality of $\fJ$ in $\ls$ implies that $\fJ = \fD_S$, which proves (i).

    The bijection for (ii) is given by the map $S \mapsto \fD_S$, which sends sheets of $\ls$ to maximal decomposition classes in $\ls$; it is well-defined and surjective by (i), and injective by \textsc{Corollary}~\ref{Corollary Closure of Dense Decomposition Class}(i).
    
    Finally, for (iii), if $\fJ = S$ is itself a sheet of $\ls$ then $\fJ = \fD_S$, and so (i) implies that $\fJ$ is maximal in $\ls$.
    Moreover, if $\fJ' \in \fD_{\am}[G]$ satisfies $\fJ' \subseteq \overline{\fJ}$, then \textsc{Lemma}~\ref{Lemma Closure of Sheet in Level Set}(i) implies that $\fJ' \subseteq \overline{\fJ} \cap \ls = \fJ$, and thus $\fJ' = \fJ$.
    Since $\fJ = S$ is both maximal and minimal in $\ls$, it is isolated in $\ls$ by definition.
\end{proof}
\end{theorem}

This proves \textsc{Theorem}~\ref{Theorem Level Sheets and their Irreducible Components Main Result} from the introduction.
It is currently an open question as to whether the converse of \textsc{Theorem}~\ref{Theorem Irreducible Components of Level Sets}(iii) holds in all cases.
However, if we make an additional assumption about the sheets of $\ls$, then we can prove the converse to be true.

\begin{theorem}\label{Theorem Coincide with Irreducible Component iff Isolated}
    Suppose $\ls$ is a level set, and assume that every sheet of $\ls$ is a union of decomposition classes.
    Then a decomposition class $\fJ \in \fD_{\am}[G]$ coincides with a sheet of $\ls$ if and only if it is isolated in $\ls$.
\begin{proof}
    The forward direction is covered by \textsc{Theorem}~\ref{Theorem Irreducible Components of Level Sets}(iii), so suppose that $\fJ$ is isolated in $\ls$.
    Since $\fJ$ is maximal in $\ls$, \textsc{Theorem}~\ref{Theorem Irreducible Components of Level Sets}(i) implies there exists a sheet $S$ of $\ls$ such that $\fJ = \fD_S$; it is therefore sufficient to prove that $\fD_S = S$.

    It follows from our assumption that $S = \bigsqcup \fJ_i$, for some collection of decomposition classes $\fJ_i \in \fD_{\am}[G]$.
    Since $\fJ_i \subseteq S \subseteq \overline{\fD_S}$, we have that $\fJ_i \preceq \fD_S$, and so the minimality of $\fD_S$ in $\ls$ implies that $\fJ_i = \fD_S$.
    Therefore, $S = \fD_S$, as required.
\end{proof}
\end{theorem}

We come back to the assumption required for \textsc{Theorem}~\ref{Theorem Coincide with Irreducible Component iff Isolated} in \textsc{Corollary}~\ref{Corollary Coincide with Levi-Type Sheets iff Isolated}.
In particular, it always holds if the characteristic is good for $G$.

\section{Lusztig-Spaltenstein Induction}

The Lusztig-Spaltenstein induction of nilpotent orbits is already well-studied for connected reductive algebraic groups; see, for example, \cite[\S 7]{CM93} (over $\mathbb{C}$) and \cite[\S 2.1]{S82}.
However, as demonstrated in \cite[\S 2.2]{S82}, we do not have to limit ourselves to considering only nilpotent orbits.

We shall first cover the results regarding the induction of arbitrary adjoint orbits that Spaltenstein proved in \cite{S82}, and then establish some further useful properties, in line with the known properties of nilpotent Lusztig-Spaltenstein induction.
We note that some of these properties were established (for characteristic $0$) in \cite[\S 2]{B81}. 

It is important to note that \cite[\S 2.2]{S82} is carried out under the assumption that all centralisers of semisimple elements of $\g$ have only finitely many nilpotent orbits.
However, as already observed in \textsc{Lemma}~\ref{Lemma Semisimple Centraliser Description}(iii), this has since been shown to always be true for any connected reductive $G$; therefore, this assumption imposes no restriction on us.

\subsection{Construction and Initial Properties}

Fix a Levi subgroup $L \subseteq G$, and consider the $L$-orbit $\mathcal{O} \coloneqq L \cdot x \in \slant{\fl}{L}$, for some $x \in \fl$.
Let $P \in \mathfrak{P}(G,L)$ be any parabolic subgroup of $G$ for which $L$ is a Levi factor, with unipotent radical $U_P = \Ru(P)$.
Then \cite[\S 2.2]{S82} demonstrates that there exists a dense $G$-orbit in $G \cdot \left(\mathcal{O} + \fu_{\p}\right)$, which we denote $\Ind_{\fl}^{\g} \mathcal{O} =\nobreak \Ind_{\fl}^{\g} L \cdot x$.

By construction, this coincides with the usual nilpotent Lusztig-Spaltenstein induction, when it is restricted to nilpotent orbits.
The following lemma covers the properties of this induction map $\Ind_{\fl}^{\g} \colon \slant{\fl}{L} \rightarrow \slant{\g}{G}$ that can either be found explicitly in \cite[\S 2.2]{S82}, or follow readily as consequences.

\begin{lemma}\label{Lemma Initial LS Induction Properties}
    Suppose that $\tilde{x} \in \tilde{\mathcal{O}} = \Ind_{\fl}^{\g} \mathcal{O} = \Ind_{\fl}^{\g} L \cdot x$.
\begin{itemize}
    \item[$\mathrm{(i)}$] $G \cdot \tilde{x}$ is the unique dense $G$-orbit in $G \cdot \left(\mathcal{O} + \fu_{\p}\right)$.
    \item[$\mathrm{(ii)}$] $\tilde{\mathcal{O}} = \left(G \cdot \left(\mathcal{O} + \fu_{\p}\right)\right)^\reg$.
    \item[$\mathrm{(iii)}$] $\dim\C_G\tilde{x} = \dim\C_Lx$.
    \item[$\mathrm{(iv)}$] $\C^{\,\circ}_Lx\s$ is a Levi subgroup of $\C^{\,\circ}_Gx\s$, and thus $\mathcal{O}' \coloneqq \Ind_{\mathfrak{c}_{\fl}x\s}^{\mathfrak{c}_{\g}x\s} \left(\mathrm{C}^{\,\circ}_L x\s \cdot x\n\right)$ is a well-defined nilpotent $\C^{\,\circ}_Gx\s$-orbit.
    \item[$\mathrm{(v)}$] $\tilde{\mathcal{O}} = G \cdot \left( x\s + \mathcal{O}' \right);$ moreover, the Jordan decomposition of $\tilde{x}$ is $($up to $G$-conjugacy$)$ equal to $x\s + \tilde{n}$ for some $\tilde{n} \in \left(x\n + \fu_{\p}\right) \cap \mathcal{O}'$. 
    \item[$\mathrm{(vi)}$] If $L = \Co{G}x\s$, then $\tilde{\mathcal{O}} = G \cdot x$.
\end{itemize}
\begin{proof}
    If $y \in \g$ is such that $G \cdot y$ is dense in $G \cdot \left(\mathcal{O} + \fu_{\p}\right)$ then $G \cdot \tilde{x}$ and $G \cdot y$ are $G$-orbits with the same closure $\overline{G \cdot \left(\mathcal{O} + \fu_{\p}\right)}$.
    Since $\overline{\mathcal{O}_0}^\reg = \mathcal{O}_0$ for any $G$-orbit $\mathcal{O}_0$, both (i) and (ii) are immediate.
    We note that, although (ii) is not explicitly stated in \cite{S82}, it is implied by the notation towards the end of \cite[\S 2.2]{S82}.
    
    On the other hand, (iii) is stated explicitly in \cite[\S 2.2]{S82}, as is the first part of (iv); the second part of which is immediate from the Lusztig-Spaltenstein induction of nilpotent orbits (see \cite[\S 2.1]{S82}, for example).

    As a consequence of Spaltenstein's construction of $\tilde{\mathcal{O}}$ in \cite[\S 2.2]{S82}, we have that (up to $G$-conjugacy) $\tilde{x} = x\s + \tilde{n}$ for some $\tilde{n} \in \left(x\n + \fu_{\p}\right) \cap \mathcal{O}'$.
    Since $x\n \in \cN(\fl)$, it follows from \textsc{Lemma}~\ref{Lemma Nilpotents of Regular Connected Reductive Subgroup}(iv) that $x\n + \fu_{\p} \subseteq \Ng$, and thus $\tilde{n} \in \Ng \cap \mathcal{O}' \subseteq \Ng \cap \fc_{\g} x\s$.
    Therefore, $\tilde{x} = x\s + \tilde{n}$ is the Jordan decomposition of $\tilde{x} \in \g$, and so (v) follows from (iv).
    
    Finally, for (vi), suppose that $L = \Co{G}x\s$.
    Then $\Co{L}x\s = L$, and \textsc{Lemma}~\ref{Lemma Semisimple Centraliser Description}(ii) implies that $\fc_{\g}x\s = \fc_{\fl}x\s = \fl$.
    Therefore, it follows from (v) that $\tilde{\mathcal{O}} = G \cdot \left( x\s + \Ind_{\fl}^{\fl} L \cdot x\n \right) = G \cdot\nobreak \left(x\s + L \cdot x\n\right) = G \cdot \left(L \cdot x\right) = G \cdot x$.
\end{proof}
\end{lemma}

\textsc{Lemma}~\ref{Lemma Initial LS Induction Properties}(v) is very important because it describes the Lusztig-Spaltenstein induction of an arbitrary orbit in terms of the induction of a nilpotent orbit; this will allow us to generalise many of the well-known properties of nilpotent induction to hold for arbitrary orbits.

The following result is not stated anywhere in \cite{S82}, but is implicit from the notation.
We note that \cite[\S 2.1]{S82} does establish this result for nilpotent orbits.

\begin{corollary}\label{Corollary LS Independence of Parabolic}
    The induced orbit $\tilde{\mathcal{O}} = \Ind_{\fl}^{\g} \mathcal{O}$ is independent of the choice of parabolic used in its construction.
    Therefore, for any $P,Q \in \mathfrak{P}(G,L)$, we have $\left(G \cdot \left( \mathcal{O} + \fu_{\p} \right) \right)^\reg = \left(G \cdot \left( \mathcal{O} + \fu_{\mathfrak{q}} \right) \right)^\reg$.
\begin{proof}
    Recall from \textsc{Lemma}~\ref{Lemma Initial LS Induction Properties}(iv) that $\Co{L}x\s$ is a Levi subgroup $\Co{G}x\s$.
    It follows from \cite[\S 2.2]{S82} that $\Co{P}x\s$ and $\Co{Q}x\s$ are both elements of $\mathfrak{P}\left(\Co{G}x\s,\Co{L}x\s\right)$.
    As noted above, \cite[\S 2.1]{S82} shows that the nilpotent $\C^{\,\circ}_G x\s$-orbit $\mathcal{O}' = \Ind_{\fc_{\fl}x\s}^{\fc_{\g}x\s} \C^{\,\circ}_L x\s \cdot x\n$ is independent of the choice of element of $\mathfrak{P}\left(\Co{G}x\s,\Co{L}x\s\right)$.
    Therefore, \textsc{Lemma}~\ref{Lemma Initial LS Induction Properties}(v) implies that $\tilde{\mathcal{O}} = G \cdot \left( x\s + \mathcal{O}' \right)$ is independent of the choice of element of $\mathfrak{P}(G,L)$.
    The statement that $\left(G \cdot \left( \mathcal{O} + \fu_{\p} \right) \right)^\reg = \left(G \cdot \left( \mathcal{O} + \fu_{\mathfrak{q}} \right) \right)^\reg$ is then an immediate consequence of \textsc{Lemma}~\ref{Lemma Initial LS Induction Properties}(ii).
\end{proof}
\end{corollary}

This justifies the fact that our notation for the induced orbit $\Ind_{\fl}^{\g} L \cdot x$ makes no reference to the choice of parabolic $P \in \mathfrak{P}(G,L)$.
In subsequent results regarding Lusztig-Spaltenstein induction, we will implicitly use \textsc{Corollary}~\ref{Corollary LS Independence of Parabolic} without mention.
We also have the following consequence of the fact that Lusztig-Spaltenstein induction preserves stabiliser dimension, which allows us to calculate the dimension of the induced orbit directly from the dimension of the original orbit.

\begin{corollary}\label{Corollary Dimension of Induced Orbit}
    For any $P \in \mathfrak{P}(G,L)$, we have $\dim \Ind_{\fl}^{\g} \mathcal{O} = \dim \mathcal{O} +\nobreak \left( \dim G - \dim L \right) = \dim \mathcal{O} + 2 \dim \fu_{\p}$.
\begin{proof}
    Suppose that $x \in \mathcal{O}$ and $\tilde{x} \in \Ind_{\fl}^{\g} \mathcal{O}$.
    Then \textsc{Lemma}~\ref{Lemma Initial LS Induction Properties}(iii) implies that $\dim \Ind_{\fl}^{\g} \mathcal{O} = \dim G -\nobreak \dim\C_G\tilde{x} = \dim G - \dim\C_Lx = (\dim G - \dim L) + \left(\dim L - \dim\C_Lx\right)$ from which the first equality follows.
    The second equality subsequently follows from the observation that $\dim \g = \dim \fl + 2 \dim \fu_{\p}$, which can be seen by considering the root subspaces with respect to some maximal torus $T \subseteq L$.
\end{proof}
\end{corollary}

\subsection{Inducing Unions of Orbits}

The following result demonstrates that we can extend induction to unions of equal-dimension orbits.

\begin{corollary}\label{Corollary Induction of Equal Dimension Orbits}
    Suppose that $\mathfrak{O} \subseteq \slant{\fl}{L}$ is a collection of equal-dimension $L$-orbits, with union $X \coloneqq \bigcup_{\mathcal{O} \in \mathfrak{O}} \mathcal{O}$, and let $P \in \mathfrak{P}(G,L)$.
\begin{itemize}
    \item[$\mathrm{(i)}$] If $d = \mathrm{codim}_{\fl}\,\mathcal{O} = \dim L - \dim \mathcal{O}$, for any $\mathcal{O} \in \mathfrak{O}$, then $X \subseteq \fl_{(d)}$.
    \item[$\mathrm{(ii)}$] $\left(G \cdot \left(X + \fu_{\p}\right) \right)^\reg = \bigcup\limits_{\mathcal{O} \in \mathfrak{O}} \Ind_{\fl}^{\g} \mathcal{O} \subseteq \g_{(d)}$.
    \item[$\mathrm{(iii)}$] If $X$ is closed in $\fl_{(d)}$, then $\left(G \cdot \left(X + \fu_{\p}\right) \right)^\reg$ is closed in $\g_{(d)}$.
\end{itemize}
\begin{proof}
    Since the orbits in $X$ have the same dimension, they also have the same codimension $d = \mathrm{codim}_{\fl}\,\mathcal{O} = \dim L - \dim \mathcal{O}$.
    Then the (unique) $L$-stabiliser level set containing $X$ is precisely $\fl_{(d)} = \left\{ x \in \fl \,\middle|\, \dim\C_L x = d \right\}$, which proves (i).
    
    It follows from \textsc{Lemma}~\ref{Lemma Initial LS Induction Properties}(iii) that all of the corresponding $G$-orbits $\Ind_{\fl}^{\g} \mathcal{O}$ are also of the same codimension $d$.
    Therefore, $\bigcup_{\mathcal{O} \in \mathfrak{O}} \Ind_{\fl}^{\g} \mathcal{O} \subseteq \g_{(d)} = \left\{ y \in \g \,\middle|\, \dim\C_G y = d \right\}$.
    The arguments at the end of \cite[\S 2.2]{S82} then complete the proofs of parts (ii) and (iii).
\end{proof}
\end{corollary}

If $X$ is a union of equal-dimension $L$-orbits, then we define $\Ind_{\fl}^{\g} X \coloneqq \bigcup \Ind_{\fl}^{\g}\mathcal{O}$, where the union is taken over all $L$-orbits $\mathcal{O} \in \slant{\fl}{L}$ such that $\mathcal{O} \subseteq X$.
Using \textsc{Corollary}~\ref{Corollary Induction of Equal Dimension Orbits}(ii), this is equivalent to the definition given in \cite[\S 2.2]{S82}.

\subsection{Transitivity of Induction}

The following property (colloquially known as transitivity of induction) is already well-known for nilpotent orbits under certain assumptions (see \cite[\S 2.5]{PS18} for a proof assuming the Standard Hypotheses).
However, the proof we present here does not require the transitivity of nilpotent induction as a prerequisite, and so also serves as a proof of that result in arbitrary characteristic.

\begin{theorem}\label{Theorem Transitivity of Induction}
    If $L \subseteq M \subseteq G$ are nested Levi subgroups of $G$, then $\Ind_{\fl}^{\g} \mathcal{O} = \Ind_{\mathfrak{m}}^{\g} \Ind_{\fl}^{\mathfrak{m}} \mathcal{O}$ for any $L$-orbit $\mathcal{O} \in \slant{\fl}{L}$.
\begin{proof}
    Suppose that $Q_1 \in \mathfrak{P}(M,L)$ and $Q_2 \in \mathfrak{P}(G,M)$.
    Then $Q_1 = U_{Q_1} \rtimes L$ and $Q_2 = U_{Q_2} \rtimes M$.
    By considering root subgroups (with respect to a choice of maximal torus $T \subseteq L$) we have that $P \coloneqq U_{Q_2} \rtimes Q_1 \in \mathfrak{P}(G,L)$.
    Moreover, $U_P = U_{Q_2} \rtimes U_{Q_1}$, and thus $\fu_{\p} = \fu_{\mathfrak{q}_2} \oplus \fu_{\mathfrak{q}_1}$.

    It follows from \textsc{Lemma}~\ref{Lemma Initial LS Induction Properties}(i) that there exists $y_1 \in \Ind_{\fl}^{\mathfrak{m}} \mathcal{O} \cap \left(\mathcal{O} + \fu_{\mathfrak{q}_1} \right)$, and there exists $y_2 \in \Ind_{\mathfrak{m}}^{\g} \Ind_{\fl}^{\mathfrak{m}} \mathcal{O} \cap \left( \Ind_{\fl}^{\mathfrak{m}} \mathcal{O} + \fu_{\mathfrak{q}_2} \right)$.
    Since $\Ind_{\fl}^{\mathfrak{m}} \mathcal{O} = M \cdot y_1$, and $\fu_{\mathfrak{q}_2}$ is $M$-stable, there exists $h \in M$ such that $h \cdot y_2 \in y_1 + \fu_{\mathfrak{q}_2} \subseteq \mathcal{O} + \fu_{\mathfrak{q}_1} + \fu_{\mathfrak{q}_2} = \mathcal{O} + \fu_{\p}$.
    Therefore, $G \cdot y_2 = \Ind_{\mathfrak{m}}^{\g} \Ind_{\fl}^{\mathfrak{m}} \mathcal{O} \subseteq G \cdot \left( \mathcal{O} + \fu_{\p} \right)$.

    We have that $\dim \Ind_{\mathfrak{m}}^{\g} \Ind_{\fl}^{\mathfrak{m}} \mathcal{O} = \dim \mathcal{O} + 2 \dim \fu_{\mathfrak{q}_1} +\nobreak 2 \dim \fu_{\mathfrak{q}_2} = \dim \mathcal{O} + 2 \dim \fu_{\p} = \dim \Ind_{\fl}^{\g} \mathcal{O}$, where we have used \textsc{Corollary}~\ref{Corollary Dimension of Induced Orbit} thrice.
    Therefore, $\Ind_{\mathfrak{m}}^{\g} \Ind_{\fl}^{\mathfrak{m}} \mathcal{O}$ and $\Ind_{\fl}^{\g} \mathcal{O}$ are both equal-dimension $G$-orbits contained in $G \cdot \left( \mathcal{O} + \fu_{\p}\right)$, and so \textsc{Lemma}~\ref{Lemma Initial LS Induction Properties}(i) implies that $\Ind_{\mathfrak{m}}^{\g} \Ind_{\fl}^{\mathfrak{m}} \mathcal{O} = \Ind_{\fl}^{\g} \mathcal{O}$.
\end{proof}
\end{theorem}

Our final result of the section is the generalisation of a fact about nilpotent Lusztig-Spaltenstein induction regarding the intersection of $\Ind_{\fl}^{\g} \mathcal{O}$ and $\mathcal{O} + \fu_{\p}$.

\begin{theorem}\label{Theorem Intersection is a Single P-orbit}
    Suppose that $P \in \mathfrak{P}(G,L)$ and $\mathcal{O} \in \slant{\fl}{L}$.
\begin{itemize}
    \item[$\mathrm{(i)}$] $\mathcal{O} + \fu_{\p}$ is $P$-stable.
    \item[$\mathrm{(ii)}$] If $y \in \left(\Ind_{\fl}^{\g} \mathcal{O}\right) \cap \left(\mathcal{O} + \fu_{\p}\right)$, then $P \cdot y$ is open and dense in $\mathcal{O} + \fu_{\p}$.
    \item[$\mathrm{(iii)}$] The intersection $\left(\Ind_{\fl}^{\g} \mathcal{O}\right) \cap \left(\mathcal{O} + \fu_{\p}\right)$ is a single $P$-orbit.
\end{itemize}
\begin{proof}
    If $x \in \mathcal{O}$, then \cite[Lemma 2.6.6]{L05} implies that $U_P \cdot x \subseteq x + \fu_{\p}$, and therefore $P \cdot x \subseteq L \cdot \left(x + \fu_{\p}\right)$.
    Since $\fu_{\p}$ is $P$-stable (and thus also $L$-stable), it follows that $P \cdot x \subseteq \mathcal{O} + \fu_{\p}$, and thus $P \cdot \left(\mathcal{O} + \fu_{\p}\right) \subseteq \mathcal{O} + \fu_{\p}$, which proves (i).

    If $y \in \left(\Ind_{\fl}^{\g} \mathcal{O}\right) \cap \left(\mathcal{O} + \fu_{\p}\right)$, then \textsc{Lemma}~\ref{Lemma Initial LS Induction Properties}(iii) implies that $\dim \C_G y = \dim L - \dim \mathcal{O} = \dim P - \dim U_P - \dim \mathcal{O}$.
    Moreover, (i) implies that $P \cdot y \subseteq \mathcal{O} + \fu_{\p}$, and thus $\dim P \cdot\nobreak y \leq \dim \left(\mathcal{O} + \fu_{\p}\right) = \dim \mathcal{O} + \dim U_P$.
    Then we have that $\dim \C_P y \geq \dim P -\nobreak \dim U_P -\nobreak \dim \mathcal{O} = \dim \C_G y$, and so $\C_P y \subseteq \C_G y$ implies that $\dim \C_P y = \dim \C_G y$.
    Therefore, $\dim P \cdot y = \dim U_P + \dim \mathcal{O} = \dim\left(\mathcal{O} + \fu_{\p}\right)$.
    Since $P \cdot y$ has the same dimension as the irreducible variety $\mathcal{O} + \fu_{\p}$, we have that $P \cdot y$ is dense in $\mathcal{O} + \fu_{\p}$, and thus (using the fact it is locally closed) must also be open in $\mathcal{O} + \fu_{\p}$.

    Finally, if $y, z \in \left(\Ind_{\fl}^{\g} \mathcal{O}\right) \cap \left(\mathcal{O} + \fu_{\p}\right)$, then (ii) shows that $P \cdot y$ and $P \cdot z$ are both open and dense in $\mathcal{O} + \fu_{\p}$.
    Thus $P \cdot y \cap P \cdot z \neq \emptyset$, and so (iii) follows from the fact that $\left(\Ind_{\fl}^{\g} \mathcal{O}\right) \cap \left(\mathcal{O} + \fu_{\p}\right)$ is $P$-closed.
\end{proof}
\end{theorem}

Therefore, we have proved all of the properties of the Lusztig-Spaltenstein induction map $\Ind_{\fl}^{\g} \colon \slant{\fl}{L} \rightarrow \slant{\g}{G}$ that were claimed in \textsc{Theorem}~\ref{Theorem LS Induction Main Result}.

\section{Levi-Type Decomposition Classes}\label{Section Levi-Type Decomposition Classes}

As already mentioned, much of the existing literature on decomposition classes has been developed under the assumption that $p \geq 0$ is (at least) good for the connected reductive group $G$.
It is well-known (see \cite[\S1.2, Remark 1]{S82}, for example) that this is equivalent to the assertion that $\Co{G}y \subseteq G$ is a Levi subgroup, for each semisimple $y \in \g$ (or equivalently $\fc_{\g}y \subseteq \g$ is a Levi subalgebra, for each semisimple $y \in \g$).
Even outside of good characteristic, there exist semisimple $x \in \g$ such that $\Co{G}x \subseteq G$ is a Levi subgroup, and we shall see that the decomposition classes of such elements have certain nice properties.

\begin{lemma}\label{Lemma Levi-Type Elements}
    Suppose $y \in \g$ and let $L \subseteq G$ be a Levi subgroup.
\begin{itemize}
    \item[$\mathrm{(i)}$] If $L = \Co{G}y$, then $y$ is semisimple.
    \item[$\mathrm{(ii)}$] $L = \Co{G}y$ if and only if $\fl = \fc_{\g} y$, if and only if $y \in \z(\fl)_{[\dim \fl]}$.
\end{itemize}
\begin{proof}
    Suppose that $L = \Co{G}y$.
    If $T \subseteq L$ is a maximal torus, then there exist $y_0 \in \ft$ and $y_{\alpha} \in \g_{\alpha}$ (for each $\alpha \in \Phi$) such that $y = y_0 + \sum_{\alpha \in \Phi} y_{\alpha}$.
    Since $T \subseteq \C_Gy$, and $\ft \oplus \bigoplus_{\alpha \in \Phi} \g_{\alpha}$ is a $T$\mbox{-stable} direct sum decomposition, we have that $y_{\alpha} = t \cdot y_{\alpha}$ for each $\alpha \in \Phi$ and $t \in T$.
    For each fixed $\alpha \in \Phi$, pick $t \in T \setminus \ker \alpha$, from which $y_{\alpha} = \alpha(t)y_{\alpha}$ implies that $y_{\alpha} = 0$.
    Hence $y = y_0 \in \ft$, which proves (i).
    Moreover, \textsc{Lemma}~\ref{Lemma Semisimple Centraliser Description}(ii) implies that $\fl = \Lie\left(\Co{G}y\right) = \fc_{\g}y$.

    Conversely, suppose that $\fl = \fc_{\g}y$.
    This implies that $y \in \z(\fl) \subseteq \ft$, and so $y$ is semisimple.
    Using \textsc{Lemma}~\ref{Lemma Semisimple Centraliser Description}(ii), we know that $\fl = \ft \oplus \bigoplus_{\alpha \in \Phi_y} \g_{\alpha}$, and hence $L = \llangle T, \U_{\alpha} \midd \alpha \in \Phi_y\rrangle$.
    Then \text{Lemma}~\ref{Lemma Semisimple Centraliser Description}(i) implies that $L = \Co{G}y$, which proves the first equivalence in (ii).

    For each $z \in \z(\fl)$, we have that $\fl \subseteq \fc_{\g}z$, and thus $\dim \fl \leq \dim \fc_{\g}z$; moreover, $\fl = \fc_{\g}z$ if and only if $\dim \fl = \dim \fc_{\g}z$.
    The second equivalence in (ii) then follows from the definition of $\z(\fl)_{[\dim \fl]} = \left\{ z \in \z(\fl) \midd \dim \fc_{\g} z = \dim \fl \right\}$.
\end{proof}
\end{lemma}

Suppose $x \in \g$ is such that $\fc_{\g}x\s \subseteq \fl$ is a Levi subalgebra, and let $y \in \J{G}x$.
Then there exists $g \in G$ such that $\fc_{\g}y\s = g \cdot \fc_{\g}x\s$, and thus $\fc_{\g}y\s \subseteq \g$ is also a Levi subalgebra.
It follows from \textsc{Lemma}~\ref{Lemma Levi-Type Elements}(ii) that for any $x \sim y$, we have that $\Co{G}x\s$ is a Levi subgroup if and only if $\Co{G}y\s$ is a Levi subgroup.
This leads us to the following important definition.

\begin{definition}\label{Definition Levi-Type}
    An element $x \in \g$ is called \itbf{Levi-type} if $\Co{G}x\s \subseteq G$ is a Levi subgroup.
    A decomposition class is called \itbf{Levi-type} if any $($equivalently, all$)$ of its elements are Levi-type.
    A decomposition variety is called \itbf{Levi-type} if it is the closure of a Levi-type decomposition class.
\end{definition}

By \textsc{Lemma}~\ref{Lemma Levi-Type Elements}(ii), $x \in \g$ is Levi-type if and only if $\fc_{\g}x\s \subseteq \g$ is a Levi subalgebra.
Moreover, it follows from \textsc{Lemma}~\ref{Lemma Initial LS Induction Properties}(vi) that, if $x \in \g$ is Levi-type with $L = \Co{G}x\s$, then $\Ind_{\fl}^{\g} L \cdot x = G \cdot x$.
We observe that $p \geq 0$ is good for $G$ if and only if every element of $\g$ is Levi-type, if and only if every $G$-decomposition class is Levi-type, if and only if every $G$-decomposition variety is Levi-type.

\subsection{Stabiliser-Type Levi Subgroups}\label{Subsection Stabiliser-Type Levi Subgroups}

Clearly, a decomposition class is Levi-type if and only if any (equivalently, all) of its decomposition data are of the form $\Le$, where $L \subseteq G$ is a Levi subgroup and $e_0 \in \cN(\fl)$.
However, (in general) not every pair $\Le$ consisting of a Levi subgroup and a nilpotent element $e_0 \in \cN(\fl)$ is a decomposition datum for a decomposition class.

\begin{definition}
    A Levi subgroup $L \subseteq G$ is called \itbf{stabiliser-type} if there exists $y \in \g$ such that $L = \Co{G}y$.
\end{definition}

It follows from \textsc{Lemma}~\ref{Lemma Levi-Type Elements} that such $y \in \g$ are necessarily elements of $\z(\fl)_{[\dim \fl]}$; moreover, $L$ is stabiliser-type if and only if $\fl = \fc_{\g}y$ for some $y \in \z(\fl)_{[\dim \fl]}$.

\begin{lemma}\label{Lemma Stabiliser-Type Levis in Decomposition Data}
    Suppose $L \subseteq G$ is a Levi subgroup, and $e_0 \in \cN(\fl)$.
\begin{itemize}
    \item[$\mathrm{(i)}$] $L$ is stabiliser-type if and only if $\z(\fl)_{[\dim \fl]} \neq \emptyset$, if and only if $\z(\fl)_{[\dim \fl]} = \z(\fl)^{G\mhyphen\reg}$, if and only if $\z(\fl)_{[\dim \fl]} \subseteq \z(\fl)$ is open and dense.
    \item[$\mathrm{(ii)}$] $\Le$ is a decomposition datum for some decomposition class if and only if $L$ is stabiliser-type.
    \item[$\mathrm{(iii)}$] If $L$ is stabiliser-type, then $\z(\fl)^{G\mhyphen\reg} = \left\{ y \in \g \midd \fc_{\g}y = \fl \right\} = \left\{ y \in \g \midd \Co{G}y = L \right\}$.
\end{itemize}
\begin{proof}
    Observe that the first equivalence in (i) is immediate from \textsc{Lemma}~\ref{Lemma Levi-Type Elements}(ii).
    Since $\z(\fl)$ consists entirely of semisimple elements, we know that $\z(\fl)^{G\mhyphen\reg} = \z(\fl)^\greg$, which equals $\left\{ y \in \z(\fl) \midd \dim \fc_{\g} y \leq \dim \fc_{\g} z, \text{for~all~} z \in \z(\fl) \right\}$.
    Therefore, the second equivalence in (i) follows from the proof of \textsc{Lemma}~\ref{Lemma Levi-Type Elements}(ii).
    The final equivalence in (i) is then deduced from the fact that $\z(\fl)^{G\mhyphen\reg} \subseteq \z(\fl)$ is an open and dense subset.
    
    If $y \in \g$ is such that $L = \Co{G}y$, then $\Le$ is a decomposition datum of the decomposition class $\J{G}\left(y + e_0\right)$.
    Conversely, suppose that $\fJ$ is a decomposition class such that $\Le$ is a decomposition datum of $\fJ$.
    For an arbitrary $x \in \fJ$, consider the decomposition datum $\left(\Co{G}x\s;x\n\right)$ of $\fJ$.
    Then there exists $g \in G$ such that $g \cdot \Co{G}x\s = L$ and $g \cdot x\n = e_0$.
    Therefore, $L = \Co{G}\left(g \cdot x\s\right) \subseteq G$ is stabiliser-type, which proves (ii).
    Finally (iii) follows immediately from (i) alongside \textsc{Lemma}~\ref{Lemma Levi-Type Elements}(ii).
\end{proof}
\end{lemma}

We shall use $\z(\fl)^\reg$ to mean $\z(\fl)^{G\mhyphen\reg}$, whenever $L \subseteq G$ is a Levi subgroup.
A natural question to ask is what conditions on $G$ and the characteristic $p \geq 0$ guarantee that every Levi subgroup $L \subseteq G$ is stabiliser-type.

\begin{lemma}[{\cite[Lemma 2.6.13(i)]{L05}}]\label{Lemma Condition for Stabiliser-Type}
    Suppose $L \subseteq G$ is a Levi subgroup, and $T \subseteq L$ is a maximal torus with corresponding root system $\Phi$.
    Assume $p \geq 0$ is good for $G$ and that $p$ does not divide $\lvert \left(\slant{\mathrm{X}(T)}{\mathbb{Z}\Phi}\right)_{\mathrm{tor}}\rvert$.
    Then $L \subseteq G$ is stabiliser-type.
\begin{proof}
    Following \cite[Definition 2.6.10]{L05}, we see that Letellier refers to the elements $x \in \g$ with $L = \Co{G}x$ as ``$L$-regular elements in $\g$'' (we remark that we do not use this terminology in this paper, in order to not cause confusion with our definitions of $G$-regular and $\g$-regular elements from \S\ref{Subsection Stabilisers and Centralisers}).
    Then \textsc{Lemma}~\ref{Lemma Levi-Type Elements}(ii) implies that $\z(\fl)_{[\dim \fl]}$ is precisely the set of such elements, and so this result is a rephrasing of \cite[Lemma 2.6.13(i)]{L05}.
\end{proof}
\end{lemma}

We note that the conditions in \textsc{Lemma}~\ref{Lemma Condition for Stabiliser-Type} are not necessary for a given Levi subgroup to be stabiliser-type; for example, $G = \Co{G}0$ is always a stabiliser-type Levi subgroup of itself, regardless of the characteristic $p \geq 0$.
An example of a Levi subgroup which is not stabiliser-type is $L = \left\{ \begin{psmallmatrix} a & 0 \\ 0 & {a^{-1}} \end{psmallmatrix} \midd a \in \bbKx \right\} \subseteq \mathrm{SL}_2 = G$, with $p = 2$; here we have $\z(\fl) = \z(\fl)_{[3]}$, and so $\z(\fl)_{[\dim \fl]} = \z(\fl)_{[1]} = \emptyset$.
The complete classification of stabiliser-type Levi subgroups for simple type $\mathrm{A}$ algebraic groups will be included in the author's next paper. 

We refer to a pair $\Le$ consisting of a stabiliser-type Levi subgroup $L \subseteq G$, and a nilpotent element $e_0 \in \cN(\fl)$, as a \emph{Levi-type} $G$\emph{-decomposition datum}.
It is clear that a decomposition class $\fJ$ is Levi-type if and only if any (equivalently, all) of its decomposition data are Levi-type.
If $\Le$ is a Levi-type decomposition datum, then it follows from \textsc{Theorem}~\ref{Theorem Initial Decomposition Class Properties}(i) and \textsc{Lemma}~\ref{Lemma Stabiliser-Type Levis in Decomposition Data}(iii) that the corresponding Levi-type decomposition class can be written as $\J{G}\Le = G \cdot \left( \z(\fl)^\reg + e_0 \right)$.

\subsection{Nilpotent Decomposition Classes}

Suppose $e_0 \in \Ng$ is an arbitrary nilpotent element.
Recall from \textsc{Theorem}~\ref{Theorem Initial Decomposition Class Properties}(v) and \textsc{Proposition}~\ref{Proposition Decomposition Varieties Properties}(ii), respectively, that its decomposition class is given by $\J{G}e_0 = \z(\g) + G \cdot e_0$, and its decomposition variety is given by $\dv{G}{e_0} = \z(\g) + \overline{G \cdot e_0}$.

\begin{definition}
    A $G$-decomposition class is called \itbf{nilpotent} if it is of the form $\J{G}e_0$, for some nilpotent $e_0 \in \Ng$.
    Similarly, the closure of a nilpotent $G$-decomposition class is called a \itbf{nilpotent} $G$-decomposition variety.
\end{definition}

Let $\fD_{\cN}[G]$ denote the set of all nilpotent $G$-decomposition classes.
Observe that nilpotent decomposition classes coincide with nilpotent orbits if and only if $\z(\g) = 0$.
Since $G = \Co{G}0$, any decomposition datum of a nilpotent decomposition class is of the form $\left(G;e_0\right)$, which immediately proves that all nilpotent decomposition classes are Levi-type.

The set of nilpotent orbits $\slant{\Ng}{G}$ is often equipped with the closure order, defined such that $\mathcal{O}' \preceq \mathcal{O}$ if and only if $\mathcal{O}' \subseteq \overline{\mathcal{O}}$.
The following proposition demonstrates that this essentially coincides with the restriction of the closure order on $\fD[G]$ to $\fD_{\cN}[G]$.

\begin{proposition}\label{Proposition Closure Order on Nilpotent Decomposition Classes}
    Suppose $x, y \in \Ng$.
    Then $G \cdot x \subseteq \overline{G \cdot y}$ if and only if $\J{G}x \subseteq \dv{G}{y}$.
\begin{proof}
    If $G \cdot x \subseteq \overline{G \cdot y}$, then $\J{G}x = \z(\g) + G \cdot x \subseteq \z(\g) + \overline{G \cdot y} = \dv{G}{y}$.
    For the converse, suppose that $\J{G}x \subseteq \dv{G}{y}$.
    Since $x = x\n \in \dv{G}{y}$, \textsc{Proposition}~\ref{Proposition Decomposition Varieties Properties}(iii) implies that $x = x\n \in \overline{G \cdot y}$, and so the result follows from the $G$-stability of $\overline{G \cdot y}$.
\end{proof}
\end{proposition}

\begin{corollary}\label{Corollary Nilpotent Decomposition Varieties}
    Suppose $x \in \g$, and $y \in \Ng$.
    If $x \in \dv{G}{y}$, then $\J{G}x = \J{G}x\n$ is a nilpotent decomposition class, and $\J{G}x \subseteq \dv{G}{y}$.
\begin{proof}
    Using \textsc{Proposition}~\ref{Proposition Decomposition Varieties Properties}(iii), we know that $x\s \in \z(\g)$, and $x\n \in \overline{G \cdot y}$.
    It follows from \textsc{Theorem}~\ref{Theorem Initial Decomposition Class Properties} that $\J{G}x = \J{G}x\n$, and hence is a nilpotent decomposition class.
    Moreover, $x\n \in \overline{G \cdot y}$ implies that $G \cdot x\n \subseteq \overline{G \cdot y}$, and so the last part follows from \textsc{Proposition}~\ref{Proposition Closure Order on Nilpotent Decomposition Classes}.
\end{proof}
\end{corollary}

Since closures of nilpotent orbits are a finite union of nilpotent orbits, it follows from \textsc{\mbox{Proposition}}~\ref{Proposition Closure Order on Nilpotent Decomposition Classes} and \textsc{Corollary}~\ref{Corollary Nilpotent Decomposition Varieties} that nilpotent decomposition varieties are finite unions of nilpotent decomposition classes.
We shall see in \textsc{Theorem}~\ref{Theorem Levi-type Decomposition Variety as Union of Decomposition Classes} that (more generally) any Levi-type decomposition variety is a finite union of decomposition classes.

\subsection{Levi-Type Decomposition Varieties}

We shall now build towards a description of Levi-type decomposition varieties that was previously only proved under stricter assumptions: that $G$ is semisimple, and either $p = 0$ \cite{BK79}, or $G$ is adjoint and $p \geq 0$ is very good \cite{B98}.
We note that the proof has a similar structure to the one found in \mbox{\cite[Lemma 3.5.1]{B98}.}

It follows from \S\ref{Subsection Stabiliser-Type Levi Subgroups} that each Levi-type decomposition variety can be expressed in the form $\dv{G}{\Le} =\nobreak \overline{G \cdot \left( \z(\fl)^\reg + e_0 \right)}$, where $L \subseteq G$ is a stabiliser-type Levi subgroup and $e_0 \in \cN(\fl)$.
The following result, using \cite[Lemma 2.6.6]{L05}, allows us to obtain a generalisation of \cite[Lemma 3.5.1(i)]{B98} to arbitrary characteristic.

\begin{proposition}\label{Proposition Bijective morphism of UP varieties}
    Suppose $P \subseteq G$ is a parabolic subgroup, $L \subseteq P$ is a stabiliser-type Levi factor, and $e_0 \in \cN(\fl)$.
    Let $\mu \colon U_P \times \left(\z(\fl)^\reg + e_0\right) \rightarrow \g$ denote the restriction of the adjoint action $G \times \g \rightarrow \g$.
\begin{itemize}
    \item[$\mathrm{(i)}$] $\mu \colon U_P \times \left(\z(\fl)^\reg + e_0\right) \rightarrow \z(\fl)^\reg + e_0 + \fu_{\p}$ is a bijective morphism of varieties.
    \item[$\mathrm{(ii)}$] Suppose we let $U_P$ act on $U_P \times \left(\z(\fl)^\reg + e_0\right)$ via $h \cdot (g,z) \coloneqq (hg,z)$, and on $\z(\fl) + e_0 + \fu_{\p}$ via the adjoint action.
    Then $\mu \colon U_P \times \left(\z(\fl)^\reg + e_0\right) \rightarrow \z(\fl)^\reg + e_0 + \fu_{\p}$ is an isomorphism of $U_P$-varieties.
\end{itemize}
\begin{proof}
    If $z \in \z(\fl)^\reg + e_0 \subseteq \fl$, then $z\s \in \z(\fl)^\reg$ and $z\n = e_0$, hence \textsc{Lemma}~\ref{Lemma Stabiliser-Type Levis in Decomposition Data}(iii) implies that $\Co{G}z\s = L$.
    Therefore, \cite[Lemma 2.6.6]{L05} shows that $\mu\restr_{\left(U_P \times \{z\}\right)} \colon U_P \times \{z\} \rightarrow z + \fu_{\p}$ is an isomorphism of varieties, and so the image of $\mu$ is equal to $\bigcup_{z\s \in \z(\fl)^\reg} \left(z\s + e_0 + \fu_{\p}\right) = \z(\fl)^\reg +\nobreak e_0 +\nobreak \fu_{\p}$.
    
    Each $w \in \z(\fl)^\reg + e_0 + \fu_{\p} \subseteq \fl \oplus \fu_{\p}$ uniquely decomposes as $w = w_{\fl} + \left(w-w_{\fl}\right)$ with $w_{\fl} \in \z(\fl)^\reg + e_0$ and $w-w_{\fl} \in \fu_{\p}$.
    Thus, the injectivity of $\mu \colon U_P \times \left(\z(\fl)^\reg + e_0\right) \rightarrow\nobreak \z(\fl)^\reg + e_0 + \fu_{\p}$ follows from the injectivity of $\mu\restr_{\left(U_P \times \{w_{\fl}\}\right)} \colon U_P \times \{w_{\fl}\} \rightarrow w_{\fl} + \fu_{\p}$.
    Since it is the restriction of the adjoint action, we have that $\mu \colon U_P \times \left(\z(\fl)^\reg + e_0\right) \rightarrow \z(\fl)^\reg + e_0 + \fu_{\p}$ is a bijective morphism of varieties, completing the proof of (i).

    With the $U_P$-actions described in the statement of (ii), both $U_P \times \left(\z(\fl)^\reg + e_0\right)$ and $\z(\fl)^\reg +\nobreak e_0 +\nobreak \fu_{\p}$ are $U_P$-varieties.
    Moreover, $\mu \colon U_P \times \left(\z(\fl)^\reg + e_0\right) \rightarrow \z(\fl)^\reg + e_0 + \fu_{\p}$ is clearly $U_P$-equivariant, and thus it remains to prove that it is an isomorphism of varieties.
    Using (i) and \cite[Theorem 5.3.2(iii)]{S98}, it suffices to prove that, for some $(g,z) \in U_P \times \left(\z(\fl)^\reg + e_0\right)$, the corresponding differential $\md \mu_{(g,z)} \colon \mathcal{T}_{(g,z)}\left(U_P \times \left(\z(\fl)^\reg + e_0\right)\right) \rightarrow \mathcal{T}_{\mu(g,z)}\left(\z(\fl)^\reg + e_0 + \fu_{\p}\right)$ is a bijection between the relevant tangent spaces.

    Fix an arbitrary $z \in \z(\fl)^\reg + e_0$.
    By considering a suitable immersive representation of $G$, the differential $\md \mu_{(1,z)}$ can be interpreted as the map $\fu_{\p} \times \z(\fl) \rightarrow \z(\fl) + \fu_{\p}$ defined by $(x,y) \mapsto y + [x,z]$.
    Therefore, it remains to show that $x \mapsto [x,z]$ is a bijection $\fu_{\p} \rightarrow \fu_{\p}$.
    However, this is just the differential (at the identity) of the $U_P$-orbit map $U_P \rightarrow U_P \cdot z$.
    Since $z\s \in \z(\fl)^\reg$, \cite[Lemma 2.6.6]{L05} shows the $U_P$-orbit map is an isomorphism, and thus its differential is bijective.
\end{proof}
\end{proposition}

Suppose $\Le$ is a Levi-type decomposition datum, and $P \in \mathfrak{P}(G,L)$.
It follows from \textsc{Proposition}~\ref{Proposition Bijective morphism of UP varieties}(i) that $U_P \cdot \left(\z(\fl)^\reg + e_0\right) = \z(\fl)^\reg + e_0 + \fu_{\p}$.
Since $P = U_P \rtimes L$, we have that $P \cdot \left(\z(\fl)^\reg + e_0\right) = L \cdot \left(\z(\fl)^\reg + e_0 + \fu_{\p}\right)$.
Then, using the fact that both $\z(\fl)^\reg$ and $\fu_{\p}$ are $L$-stable, it follows that $P \cdot \left(\z(\fl)^\reg + e_0\right) = \z(\fl)^\reg + L \cdot e_0 + \fu_{\p}$.

\begin{lemma}\label{Lemma P-closed and stable set}
    Suppose $\Le$ is a Levi-type decomposition datum, and $P \in \mathfrak{P}(G,L)$.
    Then $\overline{P \cdot \left(\z(\fl)^\reg + e_0\right)} = \z(\fl) + \overline{L \cdot e_0} + \fu_{\p}$.
\begin{proof}
    Since $\p = \fl \oplus \fu_{\p}$ is a direct sum of vector spaces, and $\z(\fl)^\reg + L \cdot e_0 \subseteq \fl$, we know that $\overline{\z(\fl)^\reg + L \cdot e_0 + \fu_{\p}} = \overline{\z(\fl)^\reg + L \cdot e_0} + \overline{\fu_{\p}\vphantom{\z(\fl)^\reg}}$.
    Then \textsc{Lemma}~\ref{Lemma Closure of Semisimple + Nilpotent} implies that $\overline{\z(\fl)^\reg + L \cdot e_0} = \overline{\z(\fl)^\reg} + \overline{L \cdot e_0\vphantom{\z(\fl)^\reg}}$.
    Since $\fu_{\p} \subseteq \g$ is closed, and $\overline{\z(\fl)^\reg} = \z(\fl)$, then the result follows from $P \cdot \left(\z(\fl)^\reg + e_0\right) = \z(\fl)^\reg + L \cdot e_0 + \fu_{\p}$.
\end{proof}
\end{lemma}

Therefore, $\z(\fl) + \overline{L \cdot e_0} + \fu_{\p} \subseteq \g$ is closed and $P$-stable.
We can thus prove the following generalisation of \cite[Lemma 3.5.1(ii)]{B98}, which provides a description of Levi-type decomposition varieties in arbitrary characteristic.

\begin{theorem}\label{Theorem Levi-type Decomposition Variety description}
    Suppose $\Le$ is a Levi-type decomposition datum, and $P \in \mathfrak{P}(G,L)$.
    Then $\dv{G}{\Le} = G \cdot \left(\z(\fl) + \overline{L \cdot e_0} + \fu_{\p}\right)$.
\begin{proof}
    We have seen in \textsc{Lemma}~\ref{Lemma P-closed and stable set} that $\overline{P \cdot \left(\z(\fl)^\reg + e_0\right)} = \z(\fl) + \overline{L \cdot e_0} + \fu_{\p}$ is closed and $P$-stable.
    Therefore, \textsc{Lemma}~\ref{Lemma Parabolic-Stable Closed Subsets} implies that $G \cdot \left( \z(\fl) + \overline{L \cdot e_0} + \fu_{\p} \right) \subseteq \g$ is also closed.
    Since $\J{G}\Le = G \cdot \left( \z(\fl)^\reg + e_0 \right) \subseteq G \cdot \left( \z(\fl) + \overline{L \cdot e_0} + \fu_{\p} \right)$, it follows that $\dv{G}{\Le} \subseteq G \cdot\nobreak \left(\z(\fl) + \overline{L \cdot e_0} + \fu_{\p}\right)$.

    Conversely, we have that $P \cdot \left( \z(\fl)^\reg + e_0 \right) \subseteq \J{G}\Le$, and thus $G \cdot \left(\overline{P \cdot \left(\z(\fl)^\reg + e_0\right)}\right) \subseteq G \cdot \dv{G}{\Le} = \dv{G}{\Le}$.
    Therefore, $G \cdot \left( \z(\fl) + \overline{L \cdot e_0} + \fu_{\p} \right) \subseteq \dv{G}{\Le}$, which proves the required equality.
\end{proof}
\end{theorem}

This proves \textsc{Theorem}~\ref{Theorem Levi-Type Decomposition Varieties Main Result}(i) from the introduction.
The following result is just a rephrasing of \textsc{Theorem}~\ref{Theorem Levi-type Decomposition Variety description}, and thus follows immediately.

\begin{corollary}
    Suppose $x \in \g$ is Levi-type, and $P \in \mathfrak{P}\left(G,\Co{G}x\s\right)$.
    Then $\dv{G}x = G \cdot\nobreak \left(\fd_{\g}x\s + \widebar{\Co{G}x\s \cdot x\n} + \fu_{\p}\right)$.
\end{corollary}

\subsection{Decomposition Varieties as Unions of Decomposition Classes}\label{Subsection Decomposition Varieties as Unions of Decomposition Classes}

Now that we have the description of Levi-type decomposition varieties provided by \textsc{Theorem}~\ref{Theorem Levi-type Decomposition Variety description}, we shall prove that they are unions of decomposition classes.
Firstly, we need some results regarding conjugacy and the Jordan decomposition.

\begin{lemma}\label{Lemma Semisimple Part Conjugate Under Borel}
    Suppose $T \subseteq G$ is a maximal torus, and $B \in \mathfrak{P}(G,T)$.
    If $y \in \ft$ and $x \in y + \fu_{\mathfrak{b}}$, then $x\s \in B \cdot y$.
\begin{proof}
    Firstly, \cite[Lemma 2.6.6]{L05} implies that $U_B \cdot y \subseteq y + \fu_{\mathfrak{b}}$, and observe that $B = U_B \rtimes T$.
    Since $T$ acts trivially on $\ft$, and $\fu_{\mathfrak{b}}$ is $T$-stable, we have that $B \cdot y = U_B \cdot y \subseteq y + \fu_{\mathfrak{b}}$, and consequently $y + \fu_{\mathfrak{b}}$ is $B$-stable.

    Since $x\s \in \mathfrak{b}$ is semisimple, there exists $b \in B$ such that $b \cdot x\s \in \ft$.
    Then $b \cdot x \in b \cdot \left( y + \fu_{\mathfrak{b}} \right) = y + \fu_{\mathfrak{b}} \subseteq \ft \oplus \fu_{\mathfrak{b}}$.
    Finally, since $b \cdot x\n \in \cN(\mathfrak{b}) = \fu_{\mathfrak{b}}$, it follows from the uniqueness of the direct sum decomposition (and $b \cdot x\s + b \cdot x\n \in y + \fu_{\mathfrak{b}}$) that $b \cdot x\s = y$.
\end{proof}
\end{lemma}

\begin{corollary}\label{Corollary Semisimple Part Conjugate Under Parabolic}
    Suppose $P \subseteq G$ is a parabolic subgroup, with Levi factor $L \subseteq P$, and let $y \in \z(\fl)$.
    If $x \in y + \cN(\fl) + \fu_{\p}$, then $x\s \in P \cdot y$.
\begin{proof}
    Fix a maximal torus $T \subseteq L$, and a Borel subgroup $B' \in \mathfrak{P}(L,T)$.
    Then $\cN(\fl) = L \cdot \fu_{\mathfrak{b}'}$, and so there exists $h \in L$ such that $h \cdot x \in y + \fu_{\mathfrak{b}'} + \fu_{\p}$.
    Observe that $B = B'U_P \in \mathfrak{P}(G,T)$, and that $\fu_{\mathfrak{b}} = \fu_{\mathfrak{b}'} \oplus \fu_{\p}$.
    Since $y \in \z(\fl) \subseteq \ft$, and $h \cdot x \in y + \fu_{\mathfrak{b}}$, \textsc{Lemma}~\ref{Lemma Semisimple Part Conjugate Under Borel} implies that $h \cdot x\s \in B \cdot y$.
    The result then follows from the fact that $h \in L \subseteq P$ and $B \subseteq P$.
\end{proof}
\end{corollary}

\begin{theorem}\label{Theorem Levi-type Decomposition Variety as Union of Decomposition Classes}
    Suppose $\J{G}\Le \in \fD[G]$ is a Levi-type decomposition class, and let $x \in\nobreak \dv{G}{\Le}$.
    Then $\J{G}{x} \subseteq \dv{G}{\Le}$.
\begin{proof}
    Since decomposition classes (and decomposition varieties) are $G$-stable, we can (by \textsc{Theorem}~\ref{Theorem Levi-type Decomposition Variety description}) assume (without loss of generality) that $x \in \z(\fl) + \overline{L \cdot e_0} + \fu_{\p}$.
    Using \textsc{Corollary}~\ref{Corollary Semisimple Part Conjugate Under Parabolic}, there exists $h \in P$ such that $h \cdot x\s \in \z(\fl)$.
    Since $\z(\fl) + \overline{L \cdot e_0} + \fu_{\p}$ is $P$-stable (by \textsc{Lemma}~\ref{Lemma P-closed and stable set}), $h \cdot x \in \z(\fl) + \overline{L \cdot e_0} + \fu_{\p}$, and thus $h \cdot x\n = h \cdot x - h \cdot x\s \in \z(\fl) + \overline{L \cdot e_0} + \fu_{\p}$.

    Observe that $h \cdot x\s \in \z(\fl)$ implies that $\fl \subseteq \fc_{\g}\left(h \cdot x\s\right)$, and hence $\fd_{\g}\left(h \cdot x\s\right) \subseteq \fc_{\g} \fl = \z(\fl)$.
    Therefore, $\dreg{\g}\left(h \cdot x\s\right) + h \cdot x\n \subseteq \z(\fl) + \overline{L \cdot e_0} + \fu_{\p} \subseteq \dv{G}{\Le}$, and so \textsc{Theorem}~\ref{Theorem Initial Decomposition Class Properties}(i) implies that $\J{G}x = \J{G}\left(h \cdot x\right) \subseteq \dv{G}{\Le}$.
\end{proof}
\end{theorem}

Therefore, a Levi-type decomposition variety coincides with the union of the decomposition classes that it intersects, which proves \textsc{Theorem}~\ref{Theorem Levi-Type Decomposition Varieties Main Result}(ii).
It follows immediately that, for a Levi-type decomposition class $\J{G}\Le$, both $\dv{G}{\Le}^\reg$ and $\dv{G}{\Le}^\greg$ are also unions of decomposition classes -- where we have used \textsc{Lemma}~\ref{Lemma Closure inside a Level Set}(iv), alongside \textsc{Proposition}~\ref{Proposition Constant Dimension}.
This generalises statements found in \cite[\S 3.5]{B81} (for characteristic $0$) and \cite[\S 3.1]{A25} (for good characteristic).

\subsection{Strongly-Levi-Type Decomposition Classes}

If $\Le$ is a Levi-type decomposition datum, we define $\fD\left[G,L;e_0\right] \coloneqq \left\{ \fJ \in \fD[G] \midd \fJ \subseteq \dv{G}{\Le} \right\}$.
Then \textsc{Theorem}~\ref{Theorem Levi-type Decomposition Variety as Union of Decomposition Classes} implies that $\dv{G}{\Le}$ is the finite disjoint union of the decomposition classes in $\fD\left[G,L;e_0\right]$.
It follows from \textsc{Corollary}~\ref{Corollary Dimension and the Closure Order} that $\J{G}\Le$ is the unique decomposition class in $\fD\left[G,L;e_0\right]$ of maximal dimension.

Similarly, if $x \in \g$ is Levi-type, then we define $\fD[G,x] \coloneqq \left\{ \fJ \in \fD[G] \midd \fJ \subseteq \dv{G}{x} \right\}$, and observe that $\fD[G,x] = \fD\left[G,\Co{G}x\s;x\n\right]$.
We can now introduce the following strengthening of \textsc{Definition}~\ref{Definition Levi-Type}.

\begin{definition}\label{Definition Strongly-Levi-Type}
    An element $x \in \g$ is \itbf{strongly-Levi-type} if each $y \in\dv{G}{x}$ is Levi-type; we then also refer to $\J{G}x$ and $\dv{G}{x}$ as \itbf{strongly-Levi-type}.
\end{definition}

In other words, a Levi-type decomposition class $\J{G}\Le$ is strongly-Levi-type if $\dv{G}{\Le}$ is a (finite disjoint) union of Levi-type decomposition classes.
Once again, $p \geq 0$ is good for $G$ if and only if every element of $\g$ is strongly-Levi-type, if and only if every $G$-decomposition class is strongly-Levi-type, if and only if every $G$-decomposition variety is strongly-Levi-type.

\begin{proposition}\label{Proposition Strongly-Levi-type is Hereditary}
    Suppose $\J{G}\Le$ is a strongly-Levi-type decomposition class, and $\J{G}\left(M;e_1\right) \in \fD\left[G,L;e_0\right]$.
\begin{itemize}
    \item[$\mathrm{(i)}$] $\fD\left[G,M;e_1\right] \subseteq \fD\left[G,L;e_0\right]$.
    \item[$\mathrm{(ii)}$] $\J{G}\left(M;e_1\right)$ is also strongly-Levi-type.
    \item[$\mathrm{(iii)}$] Every nilpotent decomposition class is strongly-Levi-type.
\end{itemize}
\begin{proof}
    If $\fJ \in \fD\left[G,M;e_1\right]$, then $\fJ \subseteq \dv{G}{\left(M;e_1\right)}$.
    Since $\J{G}\left(M;e_1\right) \subseteq \dv{G}{\Le}$, we have that $\fJ \subseteq \dv{G}{\Le}$, from which (i) follows.
    Then (ii) is immediate from (i) and \textsc{Definition}~\ref{Definition Strongly-Levi-Type}.

    Now suppose that $x = x\n \in \Ng$.
    Then \textsc{Corollary}~\ref{Corollary Nilpotent Decomposition Varieties} implies that $\dv{G}{x}$ is a finite union of nilpotent decomposition classes, and thus (iii) follows from the fact that all nilpotent decomposition classes are Levi-type.
\end{proof}
\end{proposition}

We can also rephrase \textsc{Proposition}~\ref{Proposition Strongly-Levi-type is Hereditary}(ii) as follows: if $x \in \g$ is strongly-Levi-type and $y \in \dv{G}{x}$, then $y \in \g$ is also strongly-Levi-type.

\begin{theorem}\label{Theorem Strongly-Levi-Type Decomposition Classes are Locally Closed}
    If $\J{G}\Le$ is a strongly-Levi-type decomposition class, then $\J{G}\Le \subseteq \g$ is locally closed.
\begin{proof}
    Suppose $\fD\left[G,L;e_0\right] = \left\{ \J{G}\Le, \fJ_1, \ldots, \fJ_r \right\}$ is a labelling of the distinct decomposition classes contained in $\dv{G}{\Le}$, and observe that $\dv{G}{\Le} = \J{G}\Le \cup \bigcup_{1 \leq j \leq r} \overline{\fJ_j}$.
    
    Suppose, for a contradiction, that there exists $x \in \J{G}\Le \cap \overline{\fJ_j}$, for some $1 \leq j \leq r$.
    Since $\J{G}\Le = \J{G}x$ is strongly-Levi-type, we know that $\fJ_j$ is Levi-type, and thus \textsc{Theorem}~\ref{Theorem Levi-type Decomposition Variety as Union of Decomposition Classes} implies that $\J{G}\Le = \J{G}x \subseteq \overline{\fJ_j}$.
    However, $\fJ_j \subseteq \dv{G}{\Le}$, so \textsc{Proposition}~\ref{Proposition Closure Order for Decomposition Classes}(ii) implies that $\J{G}\Le = \fJ_j$, which is a contradiction.

    Therefore, if we let $Y = \bigcup_{1 \leq j \leq r} \overline{\fJ_j}$, we have $\dv{G}{\Le} = \J{G}\Le \sqcup Y$.
    It follows that $\J{G}\Le = \dv{G}{\Le} \cap \left( \g \setminus Y \right)$ is the intersection of a closed set and an open set, which means $\J{G}\Le \subseteq \g$ is locally closed.
\end{proof}
\end{theorem}

\textsc{Theorem}~\ref{Theorem Initial Decomposition Class Properties}(iii) established that every decomposition class is constructible, and hence a finite union of locally closed sets.
Therefore, \textsc{Theorem}~\ref{Theorem Strongly-Levi-Type Decomposition Classes are Locally Closed} strengthens this result, for the case of strongly-Levi-type decomposition classes.
By the above discussion, we have the following immediate corollary.

\begin{corollary}
    If $p \geq 0$ is good for $G$, then every $G$-decomposition class is locally closed.
\end{corollary}

This result is already known, but the only proofs we were able to find in the literature required characteristic $0$.
We note that it is currently an open problem as to whether non-strongly-Levi-type decomposition classes are locally closed.

\subsection{Decomposition Varieties and Lusztig-Spaltenstein Induction}\label{Subsection Decomposition Varieties and Lusztig-Spaltenstein Induction}

We next generalise the results found in \cite[\S 3.1]{A25} (under the assumption of good characteristic) which link decomposition varieties and Lusztig-Spaltenstein Induction.
We note that the characteristic $0$ case was first proved in \cite[\S 3]{B81} by Borho.

Fix a Levi subgroup $L \subseteq G$, and a nilpotent element $e_0 \in \cN(\fl)$.
Suppose $z \in \z(\fl)$, and consider the $L$-orbit $\mathcal{O} = L \cdot \left(z + e_0\right)$.
Since $L \subseteq \C_G z$, we have that $\mathcal{O} = z + L \cdot e_0$.
If $P \in \mathfrak{P}(G,L)$, then \textsc{Lemma}~\ref{Lemma Initial LS Induction Properties}(ii) implies that $\Ind_{\fl}^{\g} \mathcal{O} = \left( G \cdot \left( \mathcal{O} + \fu_{\p} \right) \right)^\reg$, and thus $\widebar{\Ind_{\fl}^{\g} \mathcal{O}} = \widebar{ G \cdot \left( \mathcal{O} + \fu_{\p} \right)\vphantom{\Ind_{\fl}^{\g} \mathcal{O}}} = \widebar{ G \cdot \left( z + L \cdot e_0 + \fu_{\p} \right)\vphantom{\Ind_{\fl}^{\g} \mathcal{O}}}$.

\begin{theorem}\label{Theorem Decomposition Variety as Union of Orbit Closures}
    If $\J{G}\Le \in \fD[G]$ is Levi-type, then $\dv{G}{\Le} = \bigcup\limits_{z \in \z(\fl)} \widebar{\Ind_{\fl}^{\g}L \cdot \left(z+e_0\right)}$.
\begin{proof}
    Fix $P \in \mathfrak{P}(G,L)$, and suppose that $z \in \z(\fl)$.
    Then $z + L \cdot e_0 + \fu_{\p} \subseteq \z(\fl) + \overline{L \cdot e_0} + \fu_{\p}$, and thus \textsc{Theorem}~\ref{Theorem Levi-type Decomposition Variety description} implies that $G \cdot \left(z + L \cdot e_0 + \fu_{\p}\right) \subseteq \dv{G}{\Le}$.
    It then follows from the above that $\widebar{\Ind_{\fl}^{\g}L \cdot \left(z+e_0\right)} = \widebar{ G \cdot \left( z + L \cdot e_0 + \fu_{\p} \right)\vphantom{\Ind_{\fl}^{\g}L \cdot \left(z+e_0\right)}} \subseteq \overline{\J{G}\Le\vphantom{\Ind_{\fl}^{\g}L \cdot \left(z+e_0\right)}}$, and therefore $\bigcup_{z \in \z(\fl)} \widebar{\Ind_{\fl}^{\g}L \cdot \left(z+e_0\right)} \subseteq \overline{\J{G}\Le\vphantom{\Ind_{\fl}^{\g}L \cdot \left(z+e_0\right)}}$.

    For the converse, observe that $\widebar{z + L \cdot e_0 + \fu_{\p}\vphantom{\Ind_{\fl}^{\g}L \cdot \left(z+e_0\right)}} \subseteq \overline{G \cdot \left( z + L \cdot e_0 + \fu_{\p} \right)\vphantom{\Ind_{\fl}^{\g}L \cdot \left(z+e_0\right)}} = \widebar{\Ind_{\fl}^{\g}L \cdot \left(z+e_0\right)}$.
    Since $\{z\} \subseteq \z(\fl)$ is closed, \textsc{Lemma}~\ref{Lemma Closure of Semisimple + Nilpotent} implies that $\widebar{z + L \cdot e_0\vphantom{\{z\}}} = \widebar{\{z\}} + \overline{L \cdot e_0\vphantom{\{z\}}} = z + \overline{L \cdot e_0\vphantom{\{z\}}}$.
    Thus $\widebar{z + L \cdot e_0 + \fu_{\p}} = z + \overline{L \cdot e_0} + \fu_{\p}$, and therefore $\z(\fl) + \overline{L \cdot e_0} + \fu_{\p} \subseteq \bigcup_{z \in \z(\fl)} \widebar{\Ind_{\fl}^{\g}L \cdot \left(z+e_0\right)}$.
    Since a union of orbit closures is $G$-stable, we have that $G \cdot \left( \z(\fl) + \overline{L \cdot e_0} + \fu_{\p} \right) \subseteq \widebar{\Ind_{\fl}^{\g}L \cdot \left(z+e_0\right)}$, and so the result follows from \textsc{Theorem}~\ref{Theorem Levi-type Decomposition Variety description}.
\end{proof}
\end{theorem}

This proves \textsc{Theorem}~\ref{Theorem Levi-Type Decomposition Varieties Main Result}(iii) from the introduction, which was first stated (for characteristic $0$) as \cite[Proposition 3.1(b)]{B81}.

\begin{theorem}\label{Theorem Regular Closure of Levi-type Decomposition Class}
    If $\J{G}\Le \in \fD[G]$ is Levi-type, then $\dv{G}{\Le}^\reg = \bigcup\limits_{z \in \z(\fl)} \Ind_{\fl}^{\g}L \cdot\nobreak \left(z+e_0\right)$.
\begin{proof}
    Fix a parabolic $P \in \mathfrak{P}(G,L)$, and let $z \in \z(\fl)$.
    It follows from \textsc{Corollary}~\ref{Corollary Dimension of Induced Orbit} that $\dim \Ind_{\fl}^{\g}L \cdot\nobreak \left(z+e_0\right) = \dim L \cdot \left(z+e_0\right) + 2 \dim \fu_{\p}$.
    Since $z \in \z(\fl)$, we have $L \subseteq \C_G z$, and thus $L \cdot \left(z+e_0\right) = z + L \cdot e_0$.
    Therefore, $\dim \Ind_{\fl}^{\g}L \cdot \left(z+e_0\right) = \dim \left(L \cdot e_0\right) +  2 \dim \fu_{\p}$ is constant across all $z \in \z(\fl)$.
    The result then follows from \textsc{Theorem}~\ref{Theorem Decomposition Variety as Union of Orbit Closures}, and the fact that $\overline{\mathcal{O}}^\reg = \mathcal{O}$ for any $G$-orbit.
\end{proof}
\end{theorem}

This proves the final part of \textsc{Theorem}~\ref{Theorem Levi-Type Decomposition Varieties Main Result} from the introduction, which was first stated (for characteristic $0$) as \cite[Proposition 3.1(a)]{B81}.
The first part of the following corollary is also stated (for good characteristic) in \cite[\S 3.1]{A25}, and (for characteristic $0$) in \cite[Corollary 3.2]{B81}.

\begin{corollary}\label{Corollary Unique Nilpotent Orbit in Regular Decomposition Variety}
    Suppose $x \in \g$ is Levi-type, with $L = \Co{G}x\s \subseteq G$, and let $y \in \dv{G}{x}^\reg$.
\begin{itemize}
    \item[$\mathrm{(i)}$] $\dv{G}{x}^\reg \cap \Ng = \Ind_{\fl}^{\g} L \cdot x\n$.
    \item[$\mathrm{(ii)}$] $\dv{G}{y} \cap \dv{G}{x}^\reg = \J{G}y$ if and only if $\dv{G}{y}^\reg = \J{G}y$.
    \item[$\mathrm{(iii)}$] If $\J{G}y = \z(\g) + \Ind_{\fl}^{\g} L \cdot x\n$, then $\dv{G}{y}^\reg = \J{G}y$.
    \item[$\mathrm{(iv)}$] If $\J{G}y$ is itself Levi-type and $\dv{G}{y}^\reg = \J{G}y$, then $\J{G}y = \z(\g) + \Ind_{\fl}^{\g} L \cdot x\n$.
\end{itemize}
\begin{proof}
    Suppose that $z \in \z(\fl)$, and let $y \in \Ind_{\fl}^{\g} L \cdot \left(z + x\n\right)$.
    Then \textsc{Lemma}~\ref{Lemma Initial LS Induction Properties}(v) implies that $y\s \in G \cdot z$, and thus $y \in \Ng$ if and only if $z = 0$.
    Therefore, (i) follows from \textsc{Theorem}~\ref{Theorem Regular Closure of Levi-type Decomposition Class}.
    
    Since $\dv{G}{x}^\reg$ is a union of decomposition classes, we have that $\J{G}y \subseteq \dv{G}{x}^\reg$, and so also $\dv{G}{y} \subseteq \dv{G}{x}$.
    If $m \in \N$ is such that $\J{G}x \subseteq \g_{(m)}$, then \textsc{Lemma}~\ref{Lemma Closure inside a Level Set}(a) implies that $\dv{G}{y} \cap \dv{G}{x}^\reg = \dv{G}{y} \cap \dv{G}{x} \cap \g_{(m)}  = \dv{G}{y} \cap \g_{(m)} = \dv{G}{y}^\reg$, from which (ii) is immediate.
    
    For (iii), suppose that $\J{G}y = \z(\g) + \Ind_{\fl}^{\g} L \cdot x\n$.
    Then $y\s \in \z(\g)$ and $y\n \in \Ind_{\fl}^{\g} L \cdot x\n$.
    As observed in \S\ref{Subsection Connected Reductive Algebraic Groups}, this implies that $\J{G}y = \J{G}y\n = \z(\g) + G \cdot y\n$.
    Then \textsc{Theorem}~\ref{Theorem Initial Decomposition Class Properties}(v) shows that $\dv{G}{y}^\reg = \left( \z(\g) + \overline{G \cdot y\n} \right)^\reg = \z(\g) + G \cdot y\n = \J{G}y$, as required.

    Finally, for (iv), suppose that $\J{G}y$ is itself Levi-type and $\dv{G}{y}^\reg = \J{G}y$.
    It follows from (i) applied to $y \in \g$ that $\J{G}y \cap \Ng = \Ind_{\mathfrak{m}}^{\g}M \cdot y\n$, where $M = \Co{G}y\s \subseteq G$.
    If $w \in \Ind_{\mathfrak{m}}^{\g}M \cdot y\n \subseteq \Ng$, then $w \sim y$ implies that $\fc_{\g}y\s = \fc_{\g}0 = \g$, and thus $y\s \in \z(\g)$.
    Once again, this implies that $\J{G}y = \J{G}y\n$, so $G \cdot y\n \subseteq \J{G}y \subseteq \dv{G}{x}^\reg$, and consequently $G \cdot y\n \subseteq \dv{G}{x}^\reg \cap \Ng = \Ind_{\fl}^{\g} L \cdot x\n$.
    Therefore, $G \cdot y\n = \Ind_{\fl}^{\g} L \cdot x$, and thus (iv) follows from \textsc{Theorem}~\ref{Theorem Initial Decomposition Class Properties}(v).    
\end{proof}
\end{corollary}

We observe that, if $x$ is strongly-Levi-type, then \textsc{Corollary}~\ref{Corollary Unique Nilpotent Orbit in Regular Decomposition Variety}(ii-iv) can all be summarised as follows.
Let $L = \Co{G}x\s \subseteq G$ and suppose that $y \in \dv{G}{x}^\reg$.
Then $\dv{G}{y} \cap \dv{G}{x}^\reg = \J{G}y$ if and only if $\dv{G}{y}^\reg = \J{G}y$, if and only if $\J{G}y = \z(\g) + \Ind_{\fl}^{\g} L \cdot x\n$.
This was stated (for good characteristic) in \cite[\S 3.1]{A25}.

\subsection{Levi-Type Sheets}\label{Subsection Levi-Type Sheets}

We can now link some of the results from this section to sheets, as introduced in \textsc{Definition}~\ref{Definition Sheets}.
Recall from \textsc{Proposition}~\ref{Proposition Each Sheet Contains Unique Dense Decomposition Class} that each sheet $S \subseteq \g$ contains a unique dense decomposition class $\fD_S$.

\begin{definition}\label{Definition Levi-type Sheets}
    A sheet $S$ is called \itbf{Levi-type} if $\fD_S$ is a Levi-type decomposition class.
\end{definition}

It follows from our earlier observations that, if we assume good characteristic, then every sheet is Levi-type.
Recall from \textsc{Lemma}~\ref{Lemma Closure of Sheet in Level Set}(ii) that every sheet is $G$-stable.
Therefore, for any sheet $S$, we have that $S \cap \Ng$ is a finite (possibly empty) union of nilpotent orbits.

\begin{corollary}\label{Corollary Levi-type Sheet Properties}
    Suppose $S \subseteq \g$ is a Levi-type sheet.
\begin{itemize}
    \item[$\mathrm{(i)}$] $S$ is a union of decomposition classes.
    \item[$\mathrm{(ii)}$] If $S$ is a stabiliser sheet, then $S$ contains a unique nilpotent orbit.
\end{itemize}
\begin{proof}
    As previously observed, \textsc{Theorem}~\ref{Theorem Levi-type Decomposition Variety as Union of Decomposition Classes} implies that (for any Levi-type decomposition class $\fJ$), both $\overline{\fJ}^\reg$ and $\overline{\fJ}^\greg$ are unions of decomposition classes.
    Applying this to $\fD_S$ shows that (i) follows from \textsc{Corollary}~\ref{Corollary Closure of Dense Decomposition Class}(ii).

    Now suppose that $S$ is a Levi-type stabiliser sheet, and let $x \in \fD_S$, with $L = \Co{G}x\s$.
    Then \textsc{Corollary}~\ref{Corollary Closure of Dense Decomposition Class}(ii)(a) implies that $S = \dv{G}{x}^\reg$.
    Therefore, \textsc{Corollary}~\ref{Corollary Unique Nilpotent Orbit in Regular Decomposition Variety}(i) shows that $S \cap \Ng = \Ind_{\fl}^{\g} L \cdot x\n$ is a single (nilpotent) $G$-orbit, as required.
\end{proof}
\end{corollary}

Recall that, in order to prove \textsc{Theorem}~\ref{Theorem Coincide with Irreducible Component iff Isolated}, we required an additional assumption on the sheets of a level set $\ls$.
In particular, we required that every sheet of $\ls$ was a union of decomposition classes.
Since \textsc{Corollary}~\ref{Corollary Levi-type Sheet Properties}(i) shows that this holds if all the sheets of $\ls$ are Levi-type sheets, we can restate a version of \textsc{Theorem}~\ref{Theorem Coincide with Irreducible Component iff Isolated} as follows.

\begin{corollary}\label{Corollary Coincide with Levi-Type Sheets iff Isolated}
    Suppose $\ls$ is a level set, and assume that every sheet of $\ls$ is Levi-type.
    Then a decomposition class $\fJ \in \fD_{\am}[G]$ coincides with a sheet of $\ls$ if and only if it is isolated in $\ls$.
\end{corollary}

Moreover, if the characteristic is good for $G$, then this assumption always holds (since in that case, every sheet is Levi-type).
We shall conclude this paper by drawing attention to connections between \textsc{Corollary}~\ref{Corollary Levi-type Sheet Properties}(ii) and the following conjecture of Spaltenstein, which we have reworded slightly in line with the new terminology introduced in \textsc{Definition}~\ref{Definition Sheets}.

\begin{conjecture}[{\cite[\S 1.2]{S82}}]\label{Conjecture Spaltenstein Sheets Contain Unique Nilpotent}
    For any connected reductive algebraic group $G$ $($over an algebraically closed field of arbitrary characteristic$)$, every stabiliser sheet of $\g$ contains exactly one nilpotent orbit.
\end{conjecture}

In \cite[\S 1.2(c)]{S82}, Spaltenstein establishes that every stabiliser sheet contains at least one nilpotent orbit, and observe that \cite{BK79} (although working in characteristic $0$) essentially prove \textsc{Conjecture}~\ref{Conjecture Spaltenstein Sheets Contain Unique Nilpotent} when the characteristic $p \geq 0$ is good for $G$.
Moreover, \mbox{Spaltenstein} proves in \cite[Theorem 2.8]{S82} that \textsc{Conjecture}~\ref{Conjecture Spaltenstein Sheets Contain Unique Nilpotent} holds when $G$ has no simple components of exceptional type.
In later work, they prove that \textsc{Conjecture}~\ref{Conjecture Spaltenstein Sheets Contain Unique Nilpotent} is also true when $G$ is a simple algebraic group of either type $\mathrm{E}_6$ \cite[\S 7, Corollary 3]{S83}, or type $\mathrm{F}_4$ when $p = 2$ \cite[\S 5, Theorem]{S84}.

However, it is noted in \cite[\S 3.1]{PS18} that \textsc{Conjecture}~\ref{Conjecture Spaltenstein Sheets Contain Unique Nilpotent} remains open for certain bad characteristics.
It follows from \textsc{Corollary}~\ref{Corollary Levi-type Sheet Properties}(ii) that \textsc{Conjecture}~\ref{Conjecture Spaltenstein Sheets Contain Unique Nilpotent} is at least true for Levi-type stabiliser sheets (regardless of characteristic). \\

We remark that \textsc{Conjecture}~\ref{Conjecture Spaltenstein Sheets Contain Unique Nilpotent} is false (in general) for centraliser sheets, and it suffices to show there exist non-empty centraliser level sets that contain no nilpotent orbits.

For example, consider $G = \mathrm{PGL}_2$ with $p = 2$, and let $\pi \colon \GL_2 \rightarrow \mathrm{PGL}_2$ be the canonical quotient homomorphism.
We claim that $\g_{[1]} \neq \emptyset$ contains no nilpotent orbits.
$\Ng$ only consists of two nilpotent $G$-orbits: the zero orbit, and $G \cdot x$, where $x = \md \pi \begin{psmallmatrix} 0 & 1 \\ 0 & 0 \end{psmallmatrix}$.
Simple computation reveals that $\fc_{\g} x = \left\{ \md \pi \begin{psmallmatrix} a & b \\ c & a \end{psmallmatrix} \midd a,b,c \in \bbK \right\}$, and thus $\dim \fc_{\g} x = 2$, whereas $\dim \fc_{\g} 0 = 3$.
Similar computation shows that $\md \pi \begin{psmallmatrix} 1 & 0 \\ 0 & 0 \end{psmallmatrix} \in \g_{[1]}$, thus proving our claim.

\printbibliography

\end{document}