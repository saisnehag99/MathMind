This paper aims to give an overview of the foundations and abstract properties of the $D$-graded Proj construction. The starting point is the following datum.

\begin{definition}\label{def:graded_ring}       
Let $D$ be a finitely generated abelian group. 
\begin{enumerate}[label=(\arabic*)]
    \item A \emph{$D$-graded} ring is a commutative ring $S$ with a decomposition $S = \oplus_{d \in D} S_d$ into a direct sum of subgroups $S_d \le S$ such that $S_d \cdot S_e \subseteq S_{d+e}$. 
The elements of $S_d$ are called \emph{$D$-homogeneous of degree $d$}. In particular, $0 \in S$ is homogeneous of every degree.
We will call elements \emph{homogeneous} instead of $D$-homogeneous when there is no ambiguity and denote the set of $D$-homogeneous elements by $h_D(S)$, respectively $h(S)$.

    \item A commutative ring $S$ with a grading by a finitely generated abelian group is called a \emph{multigraded ring}\footnote{If we fix a finitely generated abelian group $D$, we speak of a \emph{$D$-graded ring} to emphasize that the grading with respect to $D$ is part of the data. In other words, every $D$-graded ring is multigraded, but when we specify $D$, we regard the grading itself as explicit and fixed.}.
    
    \item Let $R$ be $D_R$-graded and $S$ be $D_S$-graded rings, where $D_R$ and $D_S$ are finitely generated abelian groups. A \emph{morphism of multigraded rings} is a pair $(\varphi, \alpha_\varphi)$, where $\varphi$ is a ring homomorphism $\varphi\colon R\to S$, and $\alpha_\varphi$ is a group homomorphism $\alpha_\varphi\colon D_R \to D_S$ satisfying $\varphi(R_d) \ \subseteq \ S_{\alpha_\varphi(d)}$ for all $d\in D_R$. 
    %If $D_R=D_S$, we will also call $\varphi$ a \emph{morphism of $D_R$-graded rings}. If $\varphi$ is an isomorphism, but $D_R \neq D_S$, we will call $\varphi$ (or rather the induced map $\alpha_\varphi\colon D_R \to D_S$) a \emph{regrading} of $S$.
\end{enumerate}
Formally, a multigraded ring is a pair $(D, S)$, where $D$ is a finitely generated abelian group and $S$ is a $D$-graded commutative unital ring.
For readability, we will usually suppress the grading group and denote a multigraded ring by its underlying ring $S$.
\end{definition}

Since $S = \oplus_{d\in D}S_d$, every element $f \in S$ has a unique decomposition $f = \sum_{d\in D} f_d$ into the \emph{$D$-homogeneous components} $f_d$ of $f$ of degree $d$, where $f_d = 0$ for almost all $d\in D$.
We say that an ideal $\Fa \unlhd S$ is \emph{$D$-graded}, if $f = \sum_{d \in D} f_d \in \Fa$ implies that $f_d \in \Fa$ for all $d\in D$. If the group $D$ is clear from the context, we might just say that $\Fa$ is graded.




\subsection{Multigraded Rings}\label{subsec:1.1}

In this section, we analyze some of the basic properties of multigraded rings. Note that those details have not been addressed in the literature yet. We closely follow \cite{Bosch}, 9.1/2 up to 9.1/6, which describes the $\BN$-graded case that we aim to generalize. 




\begin{lemma}\label{lem:bosch_9.1/2}
An ideal $\Fa \unlhd S$ is $D$-graded if and only if it is generated by $D$-homogeneous elements.
As a direct consequence, sums, products, and intersections of $D$-graded ideals remain $D$-graded ideals. 
\end{lemma}

\begin{proof}
The first statement holds by  the same arguments as found in \cite{Bosch}, Remark 9.1/2
Hence, also sums and products remain $D$-graded.
As the intersection of two $D$-graded ideals $\Fa$ and $\Fb$ also contains all homogeneous components of elements of $\Fa$ respectively $\Fb$, the claim follows.
\end{proof}



However, the radical of a $D$-graded ideal is not $D$-graded in general:

\begin{example}\label{ex:radical_not_D_hom}
Let $p$ be a prime number and let $S := \BF_p[x]/(x^p-1)$ be graded by $D := \BZ/ p\BZ$ such that $\deg(x) = \overline{1}$.
Then $(0)$ is a homogeneous ideal and
\begin{align*}
    (x-1)^p \equ x^p -1 \equ 0,
\end{align*}
that is, $x-1 \in \sqrt{(0)}$. But neither $x$ nor $1$ are nilpotent, i.e. $1, x \not\in\sqrt{(0)}$, therefore $\sqrt{(0)}$ cannot be $D$-graded.
\end{example}


\begin{remark}
 Let $\Fa \subset S$ be a $D$-homogeneous ideal. 
 If $D$ admits a total order compatible with the group structure, then $\sqrt{\Fa}$ remains $D$-homogeneous. In particular, this rule applies for $D=\BZ^r$ together with the lexicographic order.
\end{remark}



We want to consider ideals of the following type (cf.\  \cite{CRB}, Definition 1.5.3.1 (iv)).

\begin{definition}\label{def:D_prime}
Let $\Fp \unlhd S$ be a proper graded ideal and $h_D(S)$ denote the set of $D$-homogeneous elements of $S$. Then $\Fp$ is called a \emph{$D$-prime ideal} if, for $D$-homogeneous elements $f,g\in h_D(S)$ with $fg\in\Fp$, either $f\in\Fp$ or $g\in\Fp$.
\end{definition}

In the $\BN$-graded case we know that, if $\deg_D(S) \subseteq \BZ$ and $\Fp \unlhd S$ is a proper graded ideal, then $\Fp$ is $D$-prime if and only if $\Fp$ is prime (this is \cite{Bosch}, 9.1/4). 
In the $\BZ^r$-graded case, we can maintain this characterization of graded prime ideals, as shown in \cite{Rob}, Lemma 2.1.1.

\begin{proposition}\label{prop:D_prime_prime_Z^r}
Let $D = \BZ^r$, let $S$ be a factorially $\BZ^r$-graded integral domain, where $r \le \dim(S)$\footnote{If $r > \dim(S)$, $S$ has no relevant elements and $\Proj^D(S)$ is empty.}, and let $\Fp \unlhd S$ be a proper graded ideal. Then $\Fp$ is a prime ideal if and only if $\Fp$ is a $D$-prime ideal.
\end{proposition}




In general, only $\Ra$ holds, as the following example for finite $D$ shows.

 
\begin{example}\label{ex:D_prime_not_prime}
Let $S = \BR[x]$, let $D = \BZ/2\BZ$ and $\Fp = (x^2-1)$. Then $S = S_{\overline{0}} \oplus S_{\overline{1}}$, where $S_{\overline{i}} = \sum_{n \ge 0} \BR x^{2n+i}$. The ideal $\Fp$ is not prime, since $(x+1)(x-1) = 0$ in $S/\Fa$ but $(x\pm 1) \neq 0$ in $S/\Fa$. On the other hand, $x \pm 1$ is not $D$-homogeneous and $\Fp$ is graded since $x^2-1 \in S_{\overline{0}}$ (cf.\  \cite{Bosch}, Remark 9.1/2). In particular, $\Fp$ is $D$-prime: If $fg \in\Fp$ for homogeneous $f, g \in S$ then $(x^2-1)\mid fg$ and it follows $f \in \Fp$ or $g \in \Fp$ by degree arguments. On the other hand, $\Fq = (x^2+1)$ is a homogeneous prime ideal of $S$. 
\end{example}



As in the $\BN$-graded case, the $D$-graded Proj is defined in terms of degree-zero localizations. 

\begin{definition}
    For $f \in S_d$, the ring $S_f$ is defined to be the graded ring generated by the subgroups
\begin{align*}
    (S_f)_e \deq \left\{\frac{s}{f^k} \mid k \in \BN, s \in S_{e+kd} \right\}, \quad e \in D,
\end{align*}
in analogy with \cite{Bosch}, Proposition 9.1/5.
The subring $S_{(f)} := (S_{f})_0 \subset S_f$ is called the \emph{$D$-homogeneous localization} of $S$ by $f$. Likewise, we define the $D$-homogeneous localization of a $D$-prime ideal $\Fp \unlhd S$ to be
\begin{align*}
    S_{(\Fp)} \deq \left\{\frac{g}{f^k} \mid k \in \BN, f \not\in \Fp \text{ relevant}, g \in S_{k \cdot \deg(f)}\right\}.
\end{align*}
\end{definition}



In the classical setting, two homogeneous elements $f, g \in S$ are always `related', in the sense that there are integers $k, l$ such that $\deg(f^k) = \deg(g^l)$. It is immediate that the degree-zero localizations highly depend on the existence of such equations. In a multigraded ring, there can exist homogeneous elements that are not (integral) linearly dependent.

\begin{example}\label{ex:double_origin_P^1_first}
    Let $S = \BC[x, y, z]$ graded by $\BZ^2$, where $x \mapsto (1, 0)$, $y \mapsto (0, 1)$ and $z \mapsto (1, 1)$. Then the degrees of $x$ and $y$ are linearly independent in $D_\BR = \BR^2$. We claim that $S_+ = (xy, xz, yz)$ (we will prove this in general in Lemma~\ref{lem:mon_gen_S+}) and compute the corresponding degree-zero localizations.
    It holds that
    \begin{align*}
        S_{(xy)} \equ \BC[\frac{z}{xy}],\ S_{(xz)} \equ \BC[\frac{xy}{z}] \text{ and } S_{(yz)} \equ \BC[\frac{xy}{z}].
    \end{align*}
    In particular, we see that the generators of the degree-zero localizations correspond to the linear dependencies between the degrees of the variables in $D_\BR$, i.e.\ $\deg(xy) = \deg(z)$. It also shows why we do not want to take arbitrary homogeneous elements for degree-zero localizations, as $S_{(f)} = \BC$ for $f= x, y$.
\end{example}




%%%%%%%%%%%%%%%%%%%%%%%%%%%%%%%%%%%%%%%%%%%%%%%%%%%%%%%%%%%%%%%%%%%%%%%%%%%%%%%%%%%%%%%%%%%%


\subsection{Relevant Elements}\label{sec:relevant_elements}
In the classical $\BN$-graded Proj construction, the irrelevant ideal is defined as the positively graded part. As in general, $D$ does not need to be totally ordered, we may not have a notion of positivity. Thus, Brenner--Schröer defined the irrelevant ideal $S_+$ to be generated by all relevant elements, i.e.\ homogeneous elements $f$ such that the units in $S_f$ form a finite index subgroup in $D$. 
One may think of relevant elements as elements having sufficiently many homogeneous divisors, as those divisors exactly correspond to units in $S_f$ (cf.\  Lemma~\ref{lem:relevant}). We also present geometric criteria in terms of the \emph{weight cones} $\CC_D(f)$ in $D_\BR$, that are generated by the degrees of homogeneous divisors of (powers of) $f$.

 
Suppose $S = R[T_1,\ldots, T_n]$, $\rank(D) = r < n$, is a polynomial ring, where $R$ is a commutative unital ring. In that case, the minimal relevant generators of the irrelevant ideal (i.e.\ the ideal generated by all relevant elements) are monomials $f$, given by $f = \prod_{i=1}^r T_{j_i}^{\epsilon_i}$, $j_i \in \{1, \ldots, n\}$, such that the $\deg(T_{j_i})$ are pairwise linearly independent (cf.\ Lemma~\ref{lem:mon_gen_S+}), where the $\epsilon_i$ are either all equal to 1 or all equal to zero (if $1$ is relevant). 


Most importantly, relevant elements $f$ give affine toric varieties if $S$ is a noetherian polynomial ring, i.e.\ $\Spec(S_{(f)})$ is a simplicial affine toric variety (see \cite{BS}, Proposition 3.4). 
The following definition is due to \cite{BS}. If not explicitly stated otherwise, $D$ will always denote a finitely generated abelian group and $r = \rank(D)$.



\begin{definition} \label{def:relevant}
Let $S$ be a $D$-graded ring, where $D$ is a finitely generated abelian group. 
\begin{enumerate}[label=(\arabic*)]
    \item The ring $S$ is called \emph{periodic} if the degrees of the homogeneous units $f \in S^\times \cap h(S)$ form a subgroup $D'\le D$ of finite index, that is
    \begin{align*}
        D' \deq \{\deg_D(f) \mid f \in S^\times \text{ homogeneous }\}\le D
    \end{align*}
     is of finite index.
    \item An element $f \in S$ is called \emph{relevant} if
    \begin{enumerate}[label=(\roman*)]
        \item $f$ is homogeneous and
        \item the localization $S_f$ is \emph{periodic}.
    \end{enumerate}
    We denote the set of relevant elements of $S$ by $\Rel^D(S)$.
    \item Given a homogeneous element $f \in S_d$, we define the \emph{support of $f$} to be the subset of $D$ given by
\begin{align*}
    C^f \deq \{\deg_D(g) \mid g \text{ is a homogeneous divisor of $f^k$ for some $k\ge 0$}\}.
\end{align*}
Furthermore, we define the \emph{weight cone of $f$} to be the closed convex cone
\begin{align*}
    {\CC_D(f)} &\deq \overline{\Cone}(C^f) 
\end{align*}
in $D_\BR := D \otimes_\BZ \BR$. If $f$ is relevant, we also call $\CC_D(f)$ a \emph{relevant weight cone}.


\item For homogeneous $f \in h(S)$, we define the \emph{support group} of homogeneous units in $S_f$ to be
\begin{align*}
    D^f \deq \{\deg_D(g) \mid g \in (S_f)^\times \cap h(S_f)\}.
\end{align*}
\item Finally, we define the \emph{irrelevant ideal $S_+$} of $S$ to be the ideal generated by all relevant elements\footnote{The chosen language seems counterintuitive, but it is nevertheless the generalization of the $\BN$-graded case. 
The designation results from the interpretation of $V(S_+)$ as `irrelevant locus' and of $D(S_+)$ as `relevant locus', and from the fact that $\Proj^D(S)$ coincides with the quotient of $D(S_+)$ by $\Spec(S_0[D])$.} of $S$, i.e.\ $S_+ = (\Rel^D(S))$. 
\end{enumerate} 
\end{definition}


\begin{remark}
    Let $f \in S$ be homogeneous. Then by definition, $S_f$ has a unit in every degree $d \in D^f$. In particular, $f$ is relevant if and only if for all $d \in D$, $(D:D^f) d$ is the degree of a unit in $S_f$.
\end{remark}


Relevance depends on the context in which the grading is understood.

\begin{example}\label{ex:effective_needed}
    Let $S = \BC[x, y]$ and $D = \BZ$ with $\deg(x) = 1$ and $\deg(y) = 1$. Then $x$ and $y$ are relevant. However, we might want to view $D$ embedded in $\BZ^2$, such that $\deg(x) = \deg(y) = (1, 0)$. In this case, $x$ and $y$ are no longer relevant. But the actual support of the two gradings does not differ.
\end{example}

Therefore, we restrict to \emph{effective} gradings (cf.\  \cite{CRB}, Definition 1.1.1.4), so that the grading group is forced to be generated by its support.

\begin{definition}\label{def:grading_effective}
    Let $S$ be a $D$-graded ring, where $D$ is a finitely generated abelian group. 
\begin{enumerate}[label=(\arabic*)]
    \item The \emph{weight monoid} of $S$ is defined to be
    \begin{align*}
        \omega(S) \deq \{ d \in D \mid S_d \neq 0\} \subseteq D .
    \end{align*}
    \item The $D$-grading of $S$ is called \emph{effective}, if the weight monoid $\omega(S)$ generates $D$ as a group.
    \item The \emph{weight cone} of $S$ is defined to be
    \begin{align*}
        \sigma(S) \deq \overline{\Cone}(\omega(S)) \subseteq D \otimes_\BZ \BR
    \end{align*}
\end{enumerate}
\end{definition}


From now on, we will always assume that $D$ is effective. Otherwise, we may replace $D$ by the group generated by $\omega(S)$.  




\begin{lemma}
Let $f \in S$ be homogeneous. Then the support $C^f$ of $f$ is a finitely generated monoid, and the support group $D^f$ is indeed a (finitely generated) group. In particular, $D^f$ is exactly the group generated by the monoid $C^f$.
\end{lemma}

\begin{proof}
    Since $f$ can only have finitely many irreducible homogeneous divisors, $C^f$ is a finitely generated monoid. As $(S_f)^\times$ is a multiplicative group, $D^f$ is an additive group.
\end{proof}



The following characterizations are due to \cite{BS}, \cite{KU}, and \cite{May} (mostly without proof). Therefore, we give a proof whenever there is none in the literature, and cite otherwise.
  

\begin{lemma}[Criteria for relevance]\label{lem:relevant} 
Let $f \in S$, $f \neq 1$, be homogeneous. Then the following are equivalent:
\begin{enumerate}[label=(\arabic*)]
    \item $f$ is relevant.
    \item  The degrees of all homogeneous divisors $g \mid f^n$, $n\ge 0$, \emph{generate} a subgroup $D'\le D$ of finite index (in fact, $D'= D^f)$.
    \item For all $d \in D$ there exists a positive integer $N > 0$ such that $Nd \in D^f$.
    \item $\lint(\CC_D(f)) \subseteq D_\BR$ is non-empty.
    \item $\deg(f) \in \lint\CC_D(f)$. 
\end{enumerate}
Furthermore, relevance is closed under multiplication and also closed under multiplication with $S_0$.
\end{lemma}


\begin{proof}
The equivalence of $(1)$ and $(2)$ is straightforward, as every unit in $S_f$ corresponds to a divisor of (some power of) $f$ and vice versa.

$(3)$ is basically a reformulation of $(1)$, where $N$ is the index of $D^f$ in $D$.

Regarding $(4)$, one can see that the maximality of $\dim(\CC_D(f))$ is equivalent to $\rank(D) = \rank(D^f)$, which in turn is equivalent to $D^f$ having finite index in $D$.

For $(5)$, we use \cite{KU}, Lemma 2.10 for \Ra. 
Conversely, if $\deg(f) \in \lint(\CC(f))$, then $\lint(\CC(f)) \neq \oldemptyset$ and $f$ is relevant by (3).

Now, using $(4)$, one deduces that relevance is closed under multiplication, as for $f_1, f_2 \in S$ relevant, it holds $\CC_D(f_i) \subset \CC_D(f_1f_2)$. The same argument shows closedness with respect to multiplication by $S_0$.
\end{proof}



\begin{example}\label{ex:double_origin_P^1_2nd}
    Consider $S = \BC[x, y, z]$ graded by $\BZ^2$ with $x \mapsto (1, 0)$, $y \mapsto (0, 1)$ and $z \mapsto (1, 1)$ from Example~\ref{ex:double_origin_P^1_first}, where we claimed that $S_+ = (xy, xz, yz)$. Note that for $f = x, y, z$, $S_f$ can only contain the units $f$ and $f^{-1}$, so the units in $S_f$ do not give a finite index subgroup in $D$.
    It is also immediate that the cones $\CC_D(f)$ for those $f$ are one-dimensional.
    Therefore, those elements cannot be relevant. So the next step is to take all monomials that are given by the product of two variables, i.e.\ $f = xy, xz, yz$. It is not hard to see that every localization $S_f$ contains enough units. We will show this for $f = xy, xz$: \newline
    For $f = xy$, $S_f$  has units $x, y, \frac{1}{x}, \frac{1}{y}$ in degree $\pm e_i$ for $i = 1,2$. Thus $S_f$ contains a unit in every degree $d \in \BZ^2$, i.e.\ $D^f = D = \BZ^2$.
    For $f = xz$, there are units $x, z, \frac{1}{x}, \frac{1}{z}$ in degree $\pm e_1$ and $\pm (e_1+e_2)$ in $S_f$. In particular, $\frac{z}{x}$ is a unit of degree $e_2$, so that again $D^f = D$ and $S_f$ contains a unit in every degree $d \in D$.
    Now, as $x, y, z$ are not relevant, but $xy, xz, yz$ are, we deduce that $S_+ = (xy, xz, yz)$.
\end{example}




Lemma~\ref{lem:relevant} (3) has an important consequence:

\begin{corollary}\label{cor:rel_many_el_deg_zero}
    Let $h \in h(S)$ be homogeneous and $f \in S$ relevant. Then there exists an element $g_h \in h(S)$, an element $k \in \BZ$ and an element $N > 0$ such that
    \begin{align*}
        \deg\left(\frac{h^N g_h}{f^k}\right) \equ 0 .
    \end{align*}
\end{corollary}

\begin{proof}
    By Lemma~\ref{lem:relevant} (3), there exists an $N > 0$ and a homogeneous $f' =\frac{a}{f^k} \in (S_f)^\times$, such that $N\deg(h) = \deg(f')$, where $a \in S$ is homogeneous by definition and $\deg(f') = \deg(a) - k\deg(f)$. Thus for $g_h = a$ it holds $N\deg(h) + \deg(a) = k\deg(f)$ and the claim follows.
\end{proof}

We want to choose generators of $S_+$ when $S$ is a noetherian polynomial ring or a noetherian factorially graded domain, where factorially graded means having unique factorization on homogeneous elements (see also \cite{CRB}, Definition 1.5.3.1).


\begin{definition}\label{def:length_fact}
Let $0 \neq f \in S = R[T_1,\ldots, T_n]$ be homogeneous and $k \in \BN$ such that $k \le r$, where $R$ is a unital commutative ring and $r = \rank(D)$.
\begin{enumerate}
    \item We say that $f$ has a \emph{length $k$ factorization} if there is a subset $I \subseteq \{1, \ldots, n\}$ of size $k$, an element $u \in R$ and integers $\epsilon_i \in \BN_+$, $i \in I$, such that 
    \begin{align*}
        f \equ u \cdot \prod_{i\in I} T_i^{\epsilon_i}
    \end{align*}
    where the $d_i = \deg(T_i)$ generate a subgroup of $D$ of rank $k$.

        \item We say that $f$ has a \emph{linear length $k$ factorization} if there is a subset $I \subseteq \{1, \ldots, n\}$ of size $k$, and an element $u \in R$  such that 
    \begin{align*}
        f \equ u \cdot \prod_{i\in I} T_i,
    \end{align*}
    where the $d_i$, $i\in I$, generate a subgroup of rank $k$.

    \item Let $f$ be relevant. We say that $f$ is \emph{monomic} if $f$ has a linear length $r$ factorization and $u = 1_R$.
\end{enumerate}
\end{definition}

\begin{example}
    Let $S = \BC[x, y, z]$ be graded by $\BZ^2$. where $x \mapsto e_1$, $y, \mapsto e_2$ and $z \mapsto e_1 + e_2$. Then the elements $xy, xz, yz \in S_+$ are monomic and minimal, i.e. there is no relevant element $g$ of smaller degree being distinct to $xy, xz, yz$ that divides $xy, xz$ or $yz$.
    The element $f = x(xy+z)$, for example, is relevant, but not monomic. 
Note that $f = x^2y + xz$ is a sum of homogeneous elements of the same degree, where $f \in (xy, xz, yz)$. Also note that every summand of $f$, i.e.\ $x^2y$ and $xz$, has a factorization of length $\ge 2$.
\end{example}

We prove the following characterization of relevant elements for polynomial rings. It can be seen as the generalization of \cite{CRB}, Proposition 2.1.3.4.

\begin{lemma}\label{lem:mon_gen_S+}
Let $R$ be a ring and $S =R[T_1, \ldots, T_n]$ be a $D$-graded polynomial ring, such that $1$ is not relevant and $\rank(D) = r < n$.
    Then the irrelevant ideal $S_+$ is generated by all monomic relevant elements, and there are at most $\binom{n}{r}$ monomic relevant elements.
    Moreover, $f$ is relevant if and only if all homogeneous summands of $f$ have a factorization of length $\ge r$.
\end{lemma}


\begin{remark}\label{rem:monomic_dep}
Fix a homogeneous presentation \(S=R[T_1,\dots,T_n]\) with \(\deg T_i=d_i\in D\). 
The description of \(S_+\) in Lemma \ref{lem:mon_gen_S+} is relative to this presentation:
The ideal \(S_+\) (and hence \(\operatorname{Proj}^D(S)\)) is intrinsic to the graded ring \((D,S)\).
But the specific monomial generating set need not be preserved by arbitrary (non-graded) $R$-algebra automorphisms. It is canonical only up to graded automorphisms of \((D, S)\).
But the same holds for the Proj construction itself, so we do not have to worry about non-graded automorphisms at all.
\end{remark}


\begin{proof}
    First, note that $S$ is a UFD and therefore also factorially $D$-graded.
    Let $f \in S$ be relevant. Without loss of generality, we may assume that $f = h_1 \ldots h_r$ for irreducible homogeneous elements $h_1, \ldots, h_r$ (if $f$ has a factorization with more than $r$ irreducible factors, the proof goes through verbatim). If all $h_i$ are monomials, there is nothing to show. Thus assume  that some $h_j$, $j \in \{1, \ldots, r\}$, is not a monomial. Then $h_j$ has to be a sum of homogeneous elements of equal degree, so we may assume there are indices $i, k, l \in \BN$ such that $h_j$ is of type $h_j = (T_k T_l - T_j)$, where $\deg(T_kT_l) = \deg(T_j) = \deg(h_j)$ (any more complex equation like $(T_k T_l - T_jT_i)$ will result in the same argument).
    Note that the precise form of the relation is determined by the grading of $S$, but it must be a sum of monomials of same degree.
    Hence, the argument does not really depend on the form of the equation - we might just get more than $r$ factors - which is fine.
    Since $f = h_1 \ldots h_r$ and the degrees of the $h_i$ are pairwise linearly independent, every homogeneous summand of $f$ has a factorization of length $\ge r$ (after expanding). 
    More concretely, each summand arises by choosing one homogeneous summand from each of the $r$ linearly independent factors. Hence, every summand is a product of at least $r$ homogeneous (indeed irreducible) pieces and therefore an element of the ideal generated by the monomic relevant elements.
    The claim follows.
    Furthermore, $S_+$ is a radical ideal, as monomic relevant elements are square-free by definition. 
\end{proof}

As every relevant element is given by a finite linear combination of monomic relevant elements by Lemma~\ref{lem:mon_gen_S+}, monomic relevant elements are minimal generators for $S_+$.
We will denote the minimal generating set by monomic relevant elements for $S_+$ by $\Gen^D(S)$.

\begin{remark}
    Lemma~\ref{lem:mon_gen_S+} has a direct generalization to factorially graded noetherian rings. 
\end{remark}





\begin{example}\label{ex:standard_ex}
We want to use the previous Lemma to compute monomic relevant generators of $S_+$. 
\begin{enumerate}
    \item Let $S = \BC[x, y, z, w]$ and $D = \BZ^2$, where $\deg(x) = \deg(y) = (1, 0)$, $\deg(z) = (1, 1)$ and $\deg(w) = (0, 1)$. It holds that $\Gen^D(S) = \{xw, yw, zw, xz, yz\}$. 

    \item Let $S = \BC[x, y, z]$ be graded by $\BZ\times \BZ/2\BZ$, such that $\deg(x) = (1, 0)$, $\deg(y) = (0, 1)$ and $\deg(z) = (1, 1)$. Then $y$ is no longer relevant (compared to Example~\ref{ex:double_origin_P^1_first}), as $D^y = \BZ/2\BZ$ cannot have finite index in $D$. In particular, monomic relevant elements have length one. 
    Thus, $\Gen^D(S) = \{x, z\}$. 
\end{enumerate}    
\end{example}




%%%%%%%%%%%%%%%%%%%%%%%%%%%%%%%%%%%%%%%%%%%%%%%%%%%%%%%%%%%%%%%%%%%%%%%%%%%%%%%%%%%%%%%%%%%%


\subsection{$D$-graded Proj Construction}\label{sec:BS_Proj}


Having dealt with the notion of relevance, we can introduce the $D$-graded Proj construction.
A $D$-grading on $S$ gives rise to a coaction
\begin{align*}
    S \to S \otimes_{S_0} S_0[D], \ \sum_{d \in D} s_d \mapsto \sum_{d\in D} s_d \otimes d.
\end{align*}

If we view the group ring $S_0[D]$ as free $S_0$-module with basis $D$, and define the multiplication of $\alpha, \beta \in S_0[D]$ to be
\begin{align*}
    (\sum_i a_i d_i) \cdot (\sum_j b_j e_j) \deq \sum_{i,j} a_i b_j d_i e_j ,
\end{align*}
$S_0[D]$ becomes an $S_0$-algebra, called \emph{group algebra of $D$}. 


By defining $\Delta(\alpha) = \alpha \otimes \alpha$ (\emph{comultiplication}), $\epsilon(\alpha) = 1$ (\emph{counit}) and $S(\alpha) = \alpha^{-1}$ (\emph{coinverse}), $S_0[D]$ becomes a Hopf algebra (cf.\  \cite{Water}, §1.4).
In particular, $G := \Spec(S_0[D])$ is a diagonalizable group scheme by \cite{Water}, §2.2, and also the Cartier dual of the constant group scheme associated to $D$ by \cite{Water}, §2.2 and §2.3. Thus we have an induced action of $G$ on  $\Spec(S)$, corresponding to the coation of $S_0[D]$ on $S$ (also see \cite{SGA3}, I 4.7.3).


As $G$ is a diagonalizable group scheme, characters of $G$, i.e.\ homomorphisms $\chi\colon G \to \BG_m = \Spec(\CO_{S_0}[T, T^{-1}])$, correspond to invertible elements in $\CO(G) = S_0[D]$, i.e.\ elements $\alpha \in S_0[D]$, such that $\Delta(\alpha) = \alpha \otimes \alpha$ by \cite{Water}, §2.1. Elements of that type are called \emph{group-like}.



Using the quotient space $Q(S)$ from \cite{Liu}, Exercise 2.14 applied to the action of $\Spec(S_0[D])$ on $\Spec(S)$, we show that the affine schemes $\Spec(S_{(f)})$ for relevant $f \in S$ can be embedded in this quotient space. Then we can define the $D$-graded Proj to be the union of all $\Spec(S_{(f)})$ inside the quotient $Q(S)$.
Note that Liu's construction is a generalized version of \cite{SGA3}, Expose V, Theoreme 4.1.

In order to apply the following Proposition, we identify $G$ with its $S_0$-points $G(S_0) = \Hom(D, (S_0)^\times)$ (because $\Proj^D(S)$ is a scheme over $S_0$, cf.\  Remark~\ref{rem:proj_base}), as it is typically done in complex geometry.

\begin{proposition}\label{prop:Liu_quotient}
Let $G$ be a group acting on a ringed topological space $(X, \CO_X)$ and denote the corresponding group homomorphism by $\Phi\colon G \to \Aut(X)$, $\sigma \mapsto \Phi_\sigma$\footnote{Here $G$ is viewed as an abstract group; the homomorphism 
$\Phi\colon G \to \Aut(X)$ is purely algebraic, and no topology or continuity 
on $G$ is assumed.}. Then there exists a \emph{quotient space} of $X$ by $G$ consisting of a ringed topological space $(Y, \CO_Y)$, which we denote by $Q(S)$, and the \emph{quotient morphism} $p \colon X \to Y$ satisfying the following universal property: For every $\sigma \in G$ it holds $p = p \circ\sigma$ and any morphism of ringed topological spaces $f \colon X \to Z$ with the same property factors uniquely through $p$, i.e.\ $f = \Tilde{f}\circ p$ for a unique morphism $\Tilde{f} \colon Y \to Z$.
    \begin{align*}
        \xymatrix{
        X \ar[rr]^p \ar[rd]^f & &  \ar@{.>}[dl]^{\exists! \Tilde{f}} \\
        & Z &
        }
    \end{align*}
In particular, for an open subset $V \subseteq Y = Q(S)$ it holds
\begin{align*}
    \CO_Y(V) \equ \CO_X(p^{-1}(V))^G .
\end{align*}
\end{proposition}



We have to show that $\Spec(S_{(f)})$ for relevant $f \in S$ can be embedded in $Q(S)$. As $X = \Spec(S)$, $p$ is open and $\CO_X(D(g)) = S_g$ (where $D(g) = \Spec(S_g)$ is a basic open subset in $X$), it suffices to show that $(S_f)^G = S_{(f)}$.

\begin{proposition}\label{prop:S_f^G=S_(f)}
For all homogeneous $f \in S$, it holds that
\begin{align*}
    (S_f)^G \equ S_{(f)},
\end{align*}
where we view both $(S_f)^G$ and $S_{(f)}$ as subrings of $S_f$.
\end{proposition}

\begin{proof}
    Since the $G$-coaction sends a homogeneous $f$ to $f\otimes\chi^{\deg(f)}$, it extends to $S_f$,
so replacing $S$ by $S_f$ (where $f$ is a unit) reduces the claim to the case $f=1$. The action of $G$ on $X = \Spec(S)$ is given in terms of a morphism of schemes $\alpha\colon G \times_S X \to X$ and comes together with a projection $\pr\colon G \times_S X \to X$. The corresponding dual maps are given by $\alpha^\ast\colon S \to S_0[D] \otimes_S S$ and $\pr^\ast\colon S \to S_0[D] \otimes_S S$ and give rise to the exact sequence
    \begin{align*}
        S^G \to \left( S 
\ \overset{\alpha^\ast}{\underset{\mathrm{\pr^\ast}}{\rightrightarrows}}\ 
S \otimes_{S_0} S_0[D]\right).
    \end{align*}
    Thus, we can deduce that
    \begin{align*}
        S^G &\equ \{s \in S \mid \alpha^\ast(s) \equ \pr^\ast(s)\} \\
        &\equ \{s \in h(S) \mid s \otimes \chi^{\deg(s)} \equ  s \otimes \chi^0\},
    \end{align*}
    i.e.\ $s \in S^G$ if and only if $\deg(s) = 0$.
\end{proof}

 \begin{remark}
     Hence, the ringed space $\bigcup_{f \in \Rel^D(S)} \Spec(S_{(f)})$ is a well-defined sub-ringed space of the quotient space $\CQ(S)$. The structure sheaf is locally given by the restriction of the structure sheaf $\CO_{\CQ(S)}$ to $\Spec(S_{(f)})$. Since these sets are affine schemes, their union is a scheme by construction.
     This holds for any multigraded ring!
 \end{remark}

Thus, the following definition is justified.

\begin{definition}[Multigraded spectrum of a multigraded ring]\label{def:multihom_spec}\makebox{}{}\\
The \emph{multigraded spectrum of $S$} is defined as the scheme
\begin{align*}
    \Proj^D(S) \deq \bigcup_{f \in \Rel^D(S)} \Spec(S_{(f)}) \subseteq \CQ(S) .
\end{align*}
\end{definition}



%%%%%%%%%%%%%%%%%%%%%%%%%%%%%%%%%%%%%%%%%%%%%%%%%%%%%%

The next example can be seen as some sort of minimal example for the peculiarities arising from this construction. 

\begin{example}\label{standard_ex_origin}
Consider $S = \BC[x, y, z]$ graded by $\BZ^2$, where $x \mapsto (1, 0)$, $y \mapsto (0, 1)$, $z \mapsto (1, 1)$ and $S_+ = (xy, xz, yz)$. The action of $G = \Spec(\BC[\BZ^2]) = \BG_m^2$ on $\Spec(S)$ is given by $(\lambda , \mu)(a, b, c) \mapsto (\lambda a, \mu b, \lambda \mu c)$. It holds that $S^G = S_0 =\BC$, thus we cannot use affine GIT (for the action on $\Spec(S)$) to construct a good quotient. Instead, we look at the action of $G$ on $\Spec(S_f)$ for relevant $f$.
For $U = \Spec(S_{(xz)})$, $V = \Spec(S_{(yz)})$, and $X = \Proj^D(S)$, Proposition~\ref{prop:Liu_quotient} implies
    \begin{align*}
        \CO_X(\pi_+^{-1}(U)) &\equ (S_{xz})^G \equ S_{(xz)} \equ \BC[\frac{xy}{z}],\ \text{ and likewise} \\
        \CO_X(\pi_+^{-1}(V)) &\equ  (S_{yz})^G \equ S_{(yz)}  \equ \BC[\frac{xy}{z}].
    \end{align*}
    Therefore, distinct open affine subsets might have the same regular functions.
    Note that for $W = \Spec(S_{(xy)})$ we get
    \begin{align*}
         \CO_X(\pi_+^{-1}(W)) \equ  S_{(xy)}  \equ \BC[\frac{z}{xy}].
    \end{align*}
    In particular, the invariant rational functions of $\Quot(S)$ are generated by $\frac{xy}{z}$ and its inverse. Hence, one can already deduce that $\Proj^D(S)$ is a $\BP^1$ with doubled origin (because the rational function $xy/z$ has two zeroes). Hence, we get a $\BP^1$ if we throw away one of $U$ or $V$. 
    Also note that there are small modifications of the grading that do not change Proj. For example, let $S' = \BC[x, y, z]$ with $D = \BZ^2$, where $x \mapsto 2 e_1$, $y \mapsto e_2$ and $z \mapsto e_1 + e_2$. Then still $S'_+ = (xy, xz, yz)$, so $\Proj(S) = \Proj(S')$. The reason for that is quite simple: even though the equation changes from $\deg(xy) = \deg(z)$ to $\deg(xy^2) = \deg(z^2)$, the localizations cannot see this, i.e.\ $S_{(z^2)} = S_{(z)}$ and $S_{(xy^2)} = S_{(xy)}$. 
\end{example}



\begin{remark}[$\Proj^D(S)$ is an $S_0$-scheme]\label{rem:proj_base}
$X = \Proj^D(S)$ can be identified as a relative scheme over the base scheme $\Spec(S_0)$ in a natural way: Every $f \in \Rel^D(S)$ gives rise to a canonical ring homomorphism $S_0 \to S_{(f)}$ via $s \mapsto \frac{s}{1}$. This in turn induces a morphism $\Spec(S_{(f)}) \to \Spec(S_0)$ of affine spectra. Gluing all these morphisms together, we get a morphism $\Proj^D(S) \to \Spec(S_0)$.  
\end{remark}



Open subsets of $\Proj^D(S)$, defined by homogeneous elements, are covered by basic open subsets $\Spec(S_{(f)})$ for relevant $f \in S$.

\begin{lemma}[Distinguished open subsets]\label{lem:distinguish}
Let $f \in \Rel^D(S)$ and let $h \in S_d$, $d \in D$, be homogeneous such that $hf\neq 0$.
Then 
\begin{enumerate}[label=(\roman*)]
    \item $hf \in \Rel^D(S)$.

    \item the morphism $\pi_f\colon \Spec(S_f) \to \Spec(S_{(f)})$ is open.
    
    \item $h$ defines a principal open subset $H = D(hf)$ in $\Spec(S_f)$, whose image in $\Spec(S_{(f)})$ is open.
    
    \item the image of $H$ in $\Spec(S_{(f)})$ is equal to $\Spec(S_{(hf)})$.
    
    \item the sets 
    \begin{align*}
        D^+(h) \deq \bigcup_{\substack{f \in \Rel^D(S) \\ f \in (h)}} \Spec(S_{(f)}) \equ \bigcup_{\substack{f \in \Rel^D(S) \\ h \mid f}} \Spec(S_{(f)}) \subseteq \Proj^D(S)
    \end{align*}
    are open subschemes of $\Proj^D(S)$, which obey the usual formal properties, i.e.
        \begin{enumerate}[label=(\arabic*)]
            \item $D^+(hh') \equ D^+(h) \cap D^+(h')$.
            \item $\bigcup_{h \text{ homogeneous}} D^+(h) \equ \Proj^D(S)$.
        \end{enumerate}
\end{enumerate}
\end{lemma}


\begin{proof}
\begin{enumerate}[label=(\roman*)]
    \item Obviously, $hf$ is homogeneous. Every homogeneous divisor of $f$ is also a homogeneous divisor of $hf$, hence $D^f \subseteq D^{hf}$. As $D^f$ has finite index, $D^{hf}$ also has to have finite index. 
    
    \item Let $U \subseteq \Spec(S_f)$ be open. Then $\pi_f(U) = p\restrict_{D(f)} (U)$ is open by Proposition~\ref{prop:Liu_quotient}.
    
    \item We have $\Spec(S_f) = D(f)$, so $x \in \Spec(S_f)$ corresponds to a prime ideal $\Fp \lhd S$ not containing $f$. Then $h$ defines the set $\{P \lhd S_f \mid \frac{h}{1} \not\in P\}$ corresponding to 
    \begin{align*}
        \{Q \lhd S \mid f, h \not\in Q\} \equ \{Q \lhd S \mid f \not\in Q\} \cap \{Q \lhd S \mid h \not\in Q\} ,
    \end{align*}
    where the latter is just the principal open set $D(hf) \subset \Spec(S)$. Thus $h$ defines the principal open subset $\Spec(S_{hf}) \subseteq \Spec(S_f)$. Now apply (ii). 
    
    \item  This follows immediately from (iii).
    
    \item For (1), we observe that on the left-hand side, we take the union over all relevant elements that are divisible by $h$ and $h'$, which is exactly the right-hand side. Regarding (2), we see that for $h = 1$ the set $D^+(h)$ is exactly $\Proj^D(S)$.
\end{enumerate}
\end{proof}
